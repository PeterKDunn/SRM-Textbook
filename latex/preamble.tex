% DELETE to use Helvitica font:
%\renewcommand*\familydefault{\sfdefault}
%\usepackage[T1]{fontenc}

%%
\usepackage{booktabs}
\usepackage{longtable}
\usepackage[bf,singlelinecheck=off]{caption}
\usepackage{times,pifont} % Nice fonts

\usepackage{fancyhdr}

\usepackage{tabularx} % For column specifications, using the kable_styling function  column_spec()
\usepackage{tabu} % For column specifications, using the kable_styling function  column_spec()
\usepackage{float} % Needed for some kableExtra stuff
\usepackage{mdframed}
\usepackage{enumitem} % To change the itemize spacing in exercises.
\usepackage{subfig} % For sub-figure captions
\usepackage[font={small,it}]{caption}
\usepackage{wrapfig} % For text-wrapping figures

% FIGURE PLACEMENT: https://tex.stackexchange.com/questions/140568/how-to-set-default-positioning-of-figure-table-document-wide
\makeatletter
  \providecommand*\setfloatlocations[2]{\@namedef{fps@#1}{#2}}
\makeatother
\setfloatlocations{figure}{hbtp}
\setfloatlocations{table}{hbtp}


%\setmainfont[UprightFeatures={SmallCapsFont=AlegreyaSC-Regular}]{Alegreya}

\usepackage{framed,color}
\definecolor{shadecolor}{RGB}{245,245,245}
\definecolor{lightshadecolor}{RGB}{230, 230, 255}
\definecolor{textcolor}{RGB}{100,100,100}

\definecolor{exampleExtraColor}{RGB}{209, 223, 250}
\definecolor{examplecolor}{RGB}{245, 245, 245}

\renewcommand{\textfraction}{0.05}
\renewcommand{\topfraction}{0.8}
\renewcommand{\bottomfraction}{0.8}
\renewcommand{\floatpagefraction}{0.75}

%\renewenvironment{quote}{\begin{VF}}{\end{VF}}
\renewenvironment{quote}%
  {\begin{kframe}}
  {\end{kframe}}



% Redo the  href  command
\let\oldhref\href
\renewcommand{\href}[2]{#2\footnote{\textbf{\url{#1}}}}

\ifxetex
  \usepackage{letltxmacro}
  \setlength{\XeTeXLinkMargin}{1pt}
  \LetLtxMacro\SavedIncludeGraphics\includegraphics
  \def\includegraphics#1#{% #1 catches optional stuff (star/opt. arg.)
    \IncludeGraphicsAux{#1}%
  }%
  \newcommand*{\IncludeGraphicsAux}[2]{%
    \XeTeXLinkBox{%
      \SavedIncludeGraphics#1{#2}%
    }%
  }%
\fi

% Default figure width
\setkeys{Gin}{width=0.5\linewidth}


\makeatletter
\newenvironment{kframe}{%
\medskip{}
\setlength{\fboxsep}{.8em}
 \def\at@end@of@kframe{}%
 \ifinner\ifhmode%
  \def\at@end@of@kframe{\end{minipage}}%
  \begin{minipage}{\columnwidth}%
 \fi\fi%
 \def\FrameCommand##1{\hskip\@totalleftmargin \hskip-\fboxsep
 \colorbox{shadecolor}{##1}\hskip-\fboxsep
     % There is no \\@totalrightmargin, so:
     \hskip-\linewidth \hskip-\@totalleftmargin \hskip\columnwidth}%
 \MakeFramed {\advance\hsize-\width
   \@totalleftmargin\z@ \linewidth\hsize
   \@setminipage}}%
 {\par\unskip\endMakeFramed%
 \at@end@of@kframe}
 \makeatother
 




\newenvironment{rmdthinkHTML}{}{} % DO nothing

\newenvironment{rmdblock}[1]
  {\setlength{\itemindent}{10em}\begin{quote}\vspace{-1em}
  \begin{itemize}\setlength{\parskip}{2mm}
  \renewcommand{\labelitemi}{
    \raisebox{-.5\height}[0pt][0pt]{
      {\setkeys{Gin}{width=1.5em,keepaspectratio}\includegraphics{icons/#1}}
    }
  }
  \setlength{\fboxsep}{1em}
  \begin{kframe}
  \item
  }
  {
  \end{kframe}
  \end{itemize}\vspace{-0.75em}\end{quote}
  }
\newenvironment{rmdblockNoIcon}
  {\begin{quote}\vspace{-0.75em}
  \setlength{\parskip}{2mm}
  %\begin{kframe}
  }
  {%\end{kframe}
  \vspace{-0.05em}\end{quote}
  }
  


\newenvironment{rmdobjectives}
  {\definecolor{shadecolor}{RGB}{230, 249, 255}\begin{rmdblock}{iconmonstr-target-4-240}}%230, 249, 255
  {\end{rmdblock}}
\newenvironment{rmdcontext}
  {\begin{rmdblock}{iconmonstr-puzzle-18-240}}
  {\end{rmdblock}}
\newenvironment{answer}
  {\null\vspace{-2mm}\small}
  {\null\vspace{-2mm}} %{\smallskip}
\newenvironment{rmdnote}
  {\begin{rmdblock}{iconmonstr-light-bulb-2-240}}
  {\end{rmdblock}}
\newenvironment{rmdspss}
  {\begin{rmdblock}{iconmonstr-laptop-4-240}}
  {\end{rmdblock}}
\newenvironment{rmdcaution}
  {\begin{rmdblock}{iconmonstr-warning-6-240}}
  {\end{rmdblock}}
\newenvironment{rmdimportant}
  {\definecolor{shadecolor}{RGB}{248, 237, 237}\begin{rmdblock}{iconmonstr-warning-8-240}}
  {\end{rmdblock}}
\newenvironment{rmdtip}
  {\begin{rmdblock}{iconmonstr-info-6-240}}
  {\end{rmdblock}}
\newenvironment{rmdwarning}
  {\begin{rmdblock}{iconmonstr-danger-13-240}}
  {\end{rmdblock}}
\newenvironment{rmdthink}
  {\definecolor{shadecolor}{RGB}{238, 235, 249}\begin{rmdblock}{iconmonstr-school-17-240}}
  {\end{rmdblock}}
\newenvironment{rmdpronunciation}
  {\begin{rmdblock}{iconmonstr-microphone-7-240}}
  {\end{rmdblock}}
  %\usepackage{xcolor}
\newenvironment{exampleExtra}
%  {\begin{rmdblock}{iconmonstr-idea-12-240}\small\textbf{Example.}}
  {\begin{rmdblockNoIcon}\small\textbf{Extra example:}}
  {\end{rmdblockNoIcon}}
\newenvironment{extraInfo}
%  {\begin{rmdblock}{iconmonstr-idea-12-240}\small\textbf{Example.}}
  {\begin{rmdblockNoIcon}\small\textbf{Extra information:}}
  {\end{rmdblockNoIcon}}



\newenvironment{darkgraytext}{\color{textcolor}}{\ignorespacesafterend}
%\AtBeginEnvironment{rmdblockNoIcon}{\begin{comment}}
%\AtEndEnvironment{rmdblockNoIcon}{\end{comment}}

%\usepackage{amsthm}
%\newtheorem*{ExampleFold}{Example}


%%% TRY HIDING EXAMPLE FOLDS, EXAMPLE FOLDS: To save paper in printing...???



% Redfine some existing environments
% -QUOTE
\makeatletter
\renewenvironment{quote}
               {\list{}{\vspace{-0.5em}\begin{kframe}\small%
                        \itemindent    \listparindent
                        \rightmargin   \leftmargin
                        \parsep        \z@ \@plus\p@}%
                \item\relax}
               {\end{kframe}\endlist}
\makeatother


% Change the lemma environment to be shaded
\usepackage{etoolbox}
\usepackage{framed}
\AtBeginEnvironment{lemma}{\begin{shaded}\begin{quote}\vspace{-2em}\setlength{\parskip}{6pt}} % Order must be shaded, then quote
\AtEndEnvironment{lemma}{\end{quote}\end{shaded}}


% ORIGINAL:
%\newenvironment{fold}
%  {\begin{rmdblock}{iconmonstr-idea-12-240}\vspace{-1em}\begin{darkgraytext}\footnotesize \textbf{Answer:}}
%  {\end{darkgraytext}\end{rmdblock}}
\newenvironment{fold}
  {\vspace{-6pt}\begin{rmdblockNoIcon}\scriptsize \textbf{Answer:}}
  {\end{rmdblockNoIcon}}
  
% Change the  fold  environment to be disappeared
% Note: In the LaTeX code, it appears like this: \BeginKnitrBlock{fold}
%\usepackage{comment}
%\AtBeginEnvironment{fold}{\begin{comment}}
%\AtEndEnvironment{fold}{\end{comment}}

%\begin{shaded}}
%\AtEndEnvironment{fold}{\end{shaded}}

%%% IF ONLY THIS WORKED:
% \usepackage{comment}
% \includecomment{fold}


%%% Two column chunks: from https://github.com/grantmcdermott/two-col-test/blob/master/preamble.css
\usepackage{multicol}
%% Note: Pandoc (which is doing all of the output conversion behind the scenes) does not parse the content of LaTeX environments. This creates problems when you try to include LaTeX commands with curly brackets directly in your Rmd file. E.g. You can't just use `\begin{multicols}{2}` directly in your Rmd file. Luckily, a straightforward workaround is to simply define some new shortcut commands yourself as per the below.
%% See: https://stackoverflow.com/questions/25849814/rstudio-rmarkdown-both-portrait-and-landscape-layout-in-a-single-pdf/27334272#27334272
\newcommand{\btwocol}{\begin{multicols}{2}}
\newcommand{\etwocol}{\end{multicols}}
            

\usepackage{makeidx}
\makeindex

\urlstyle{tt}


% NOTE:  clashes with  ntheorem  package... if I want to use that. I tried to use if to put lemma in shaded backgrounds... without luck. So I just put this code back.
\usepackage{amsthm}
\makeatletter
\def\thm@space@setup{%
  \thm@preskip=8pt plus 2pt minus 4pt
  \thm@postskip=\thm@preskip
}

\makeatother



\frontmatter

%%% Turn off maketitle to get pdf image as title page (https://stackoverflow.com/questions/45963505/coverpage-and-copyright-notice-before-title-in-r-bookdown):
\author{Peter K. Dunn}
\date{Last update: \today}
\let\oldmaketitle\maketitle
\AtBeginDocument{\let\maketitle\relax}





