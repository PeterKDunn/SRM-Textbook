% Options for packages loaded elsewhere
\PassOptionsToPackage{unicode}{hyperref}
\PassOptionsToPackage{hyphens}{url}
\PassOptionsToPackage{dvipsnames,svgnames,x11names}{xcolor}
\documentclass[
  krantz2]{krantz}
\usepackage{xcolor}
\usepackage{amsmath,amssymb}
\setcounter{secnumdepth}{5}
\usepackage{iftex}
\ifPDFTeX
  \usepackage[T1]{fontenc}
  \usepackage[utf8]{inputenc}
  \usepackage{textcomp} % provide euro and other symbols
\else % if luatex or xetex
  \usepackage{unicode-math} % this also loads fontspec
  \defaultfontfeatures{Scale=MatchLowercase}
  \defaultfontfeatures[\rmfamily]{Ligatures=TeX,Scale=1}
\fi
\usepackage{lmodern}
\ifPDFTeX\else
  % xetex/luatex font selection
\fi
% Use upquote if available, for straight quotes in verbatim environments
\IfFileExists{upquote.sty}{\usepackage{upquote}}{}
\IfFileExists{microtype.sty}{% use microtype if available
  \usepackage[]{microtype}
  \UseMicrotypeSet[protrusion]{basicmath} % disable protrusion for tt fonts
}{}
\makeatletter
\@ifundefined{KOMAClassName}{% if non-KOMA class
  \IfFileExists{parskip.sty}{%
    \usepackage{parskip}
  }{% else
    \setlength{\parindent}{0pt}
    \setlength{\parskip}{6pt plus 2pt minus 1pt}}
}{% if KOMA class
  \KOMAoptions{parskip=half}}
\makeatother
\usepackage{longtable,booktabs,array}
\usepackage{calc} % for calculating minipage widths
% Correct order of tables after \paragraph or \subparagraph
\usepackage{etoolbox}
\makeatletter
\patchcmd\longtable{\par}{\if@noskipsec\mbox{}\fi\par}{}{}
\makeatother
% Allow footnotes in longtable head/foot
\IfFileExists{footnotehyper.sty}{\usepackage{footnotehyper}}{\usepackage{footnote}}
\makesavenoteenv{longtable}
\usepackage{graphicx}
\makeatletter
\newsavebox\pandoc@box
\newcommand*\pandocbounded[1]{% scales image to fit in text height/width
  \sbox\pandoc@box{#1}%
  \Gscale@div\@tempa{\textheight}{\dimexpr\ht\pandoc@box+\dp\pandoc@box\relax}%
  \Gscale@div\@tempb{\linewidth}{\wd\pandoc@box}%
  \ifdim\@tempb\p@<\@tempa\p@\let\@tempa\@tempb\fi% select the smaller of both
  \ifdim\@tempa\p@<\p@\scalebox{\@tempa}{\usebox\pandoc@box}%
  \else\usebox{\pandoc@box}%
  \fi%
}
% Set default figure placement to htbp
\def\fps@figure{htbp}
\makeatother
\setlength{\emergencystretch}{3em} % prevent overfull lines
\providecommand{\tightlist}{%
  \setlength{\itemsep}{0pt}\setlength{\parskip}{0pt}}
\usepackage[]{natbib}
\bibliographystyle{plainnat}
%%%%% START PREAMBLE %%%%%
%%%%%

% Explicitly need for CRC, according to `Run_This_Example.tex`:
\usepackage{fixltx2e,fix-cm}
%\usepackage{amssymb} % Already loaded
%\usepackage{amsmath} % Already loaded
\usepackage{amsthm}
\usepackage{graphicx}
\usepackage{makeidx} % For making index
\usepackage{multicol} % For occasional two-column parts
\usepackage{multirow} % For occasional tables with rows combined in cols
\usepackage{imakeidx} % Add some pre-index text

% Place 'see also' on separate lines, as per https://www.ams.org/arc/tex/howto/index/0index-notes.pdf (p. 4)
% and https://tex.stackexchange.com/questions/86646/indexing-subentries-and-see-also
% and updated at https://tex.stackexchange.com/questions/231840/what-is-the-proper-use-of-several-seealso-in-the-same-index-entry-with-makeinde
%\newcommand{\gobblecomma}[1]{\ignorespaces}
%\providecommand{\indexalso}[2]{%
%    \index{#1!zzzzz@\gobblecomma|seealso{#2}}}
\def\igobble#1 {}
% Then use like this:
% \index{c!zzzzz@\igobble |seealso {a, b}}

    
    
\usepackage{customdice} % Help with die unicode!
\usepackage{pifont} % Checkmark \ding{51}
% Added by me:
\usepackage{booktabs} % Nicer tables
\usepackage{microtype} % Better spacing and easier reading
\usepackage{tabularx} % For column specifications, using the kable_styling function  column_spec()
\usepackage{tabu} % For column specifications, using the kable_styling function  column_spec()
\usepackage{float} % Needed for some kableExtra stuff
\usepackage{mdframed}  % Needed for some environment definitions (callouts)
\usepackage{enumitem} % To change the itemize spacing in exercises.
\usepackage{wrapfig} % For text-wrapping figures
\usepackage[x11names]{xcolor}


% Set captions apart from text
\usepackage{caption} % Format captions
\captionsetup[figure]{font=small}
\captionsetup[table]{font=small}

% Circled letters (e.g., Heads and Tails):
\usepackage{tikz}
\newcommand*\circled[1]{\tikz[baseline=(char.base)]{
            \node[shape=circle,draw,inner sep=0.8pt] (char) {{\textsc{#1}}};}}
\newcommand{\Heads}{\circled{h}}
\newcommand{\Tails}{\circled{t}}

% For sometimes stacking things on top of each other (used in table, in the selecting a test chapter)
\usepackage{stackengine} 
\renewcommand\stacktype{L}
\def\stackalignment{l}


% For some fancy table things on occasion:
\usepackage{array}
\usepackage{ragged2e}
\newcolumntype{P}[1]{>{\RaggedLeft\hspace{0pt}}p{#1}} % This allows right-aligned, fixed with cols


% FIGURE PLACEMENT: https://tex.stackexchange.com/questions/140568/how-to-set-default-positioning-of-figure-table-document-wide
\makeatletter
  \providecommand*\setfloatlocations[2]{\@namedef{fps@#1}{#2}}
\makeatother
\setfloatlocations{figure}{hbtp}
\setfloatlocations{table}{hbtp}


\usepackage{framed,color}
\definecolor{shadecolor}{RGB}{245,245,245}
\definecolor{lightshadecolor}{RGB}{230, 230, 255}
\definecolor{textcolor}{RGB}{100,100,100}

%\definecolor{exampleExtraColor}{RGB}{209, 223, 250}
%\definecolor{examplecolor}{RGB}{245, 245, 245}

% \renewcommand{\textfraction}{0.05}
% \renewcommand{\topfraction}{0.8}
% \renewcommand{\bottomfraction}{0.8}
% \renewcommand{\floatpagefraction}{0.75}

%\renewenvironment{quote}{\begin{VF}}{\end{VF}}
\renewenvironment{quote}%
  {\begin{kframe}}
  {\end{kframe}}
  

% Default figure width
\setkeys{Gin}{width=0.5\linewidth}

% Define kframe
\makeatletter
\newenvironment{kframe}{%
  \smallskip{}
  \setlength{\fboxsep}{0.8em} % Space between text and shaded box
  \def\at@end@of@kframe{}%
  \ifinner\ifhmode%
    \def\at@end@of@kframe{\end{minipage}}%
    \begin{minipage}{\textwidth-\fboxsep} % Ensure the minipage width matches the text width
  \fi\fi%
  \def\FrameCommand##1{%
    \hspace{3mm}% Indentation from the left edge of the shaded box
    \colorbox{shadecolor}{%
      \hspace{0mm}% Shift the text inside the shaded box 3mm to the left
      \begin{minipage}{\dimexpr\linewidth-8mm}% Adjust the width of the text inside the shaded box
        ##1
      \end{minipage}%
    }%
    \hspace{3mm}% Indentation from the right edge of the shaded box
  }%
  \MakeFramed {\setlength{\hsize}{\textwidth-\fboxsep} % Ensure the frame width matches the text width
    \@setminipage}}%
  {\par\unskip\endMakeFramed%
  \at@end@of@kframe}
\makeatother



\makeatletter
\@ifundefined{Shaded}{
}{\renewenvironment{Shaded}{\begin{kframe}}{\end{kframe}}}
\makeatother


%%%%%%%%%%%%%%%%%%%%%%%%%%%

% Generic rmd block
\newenvironment{rmdblock}[1]
  {\setlength{\itemindent}{10em}\begin{quote}\vspace{-1em}
  \begin{itemize}\setlength{\parskip}{2mm}
  \renewcommand{\labelitemi}{
    \raisebox{-.5\height}[0pt][0pt]{
      {\setkeys{Gin}{width=1.5em,keepaspectratio}\includegraphics{icons/#1}}
    }
  }
  \setlength{\fboxsep}{1em}
  \begin{kframe}
  \item
  }
  {
  \end{kframe}
  \end{itemize}\vspace{-0.75em}\end{quote}
  }
\newenvironment{rmdblockNoIcon}
  {\begin{quote}\vspace{-0.75em}
  \setlength{\parskip}{2mm}
  %\begin{kframe}
  }
  {%\end{kframe}
  \vspace{-0.05em}\end{quote}
  }

\usepackage{tcolorbox} % Provides  \newtcolorbox

%%%%%%%%%%%%%%%%%%%%%%%%%%%
% Objectives boxes
\definecolor{ObjColour}{RGB}{237, 237, 237}
\newtcolorbox{objectivescolourbox}{
  colback=ObjColour,
  colframe=ObjColour,
  coltext=black,
  boxsep=5pt,
  arc=4pt}

\newenvironment{objectivesBox}[1]
  {\footnotesize
  %\begin{itemize}[leftmargin=.5in]
  %\renewcommand{\labelitemi}{
  %  \raisebox{-.4\height}[0pt][0pt]{
  %    {\setkeys{Gin}{width=1.5em,keepaspectratio}
  %      \includegraphics{icons/#1}\qquad}
  %  }
  %}
  \setlength{\fboxsep}{1em}
  \begin{objectivescolourbox}\setlength{\parskip}{1mm}
  %\item
  }
  {
  \end{objectivescolourbox}
  %\end{itemize}
%  \vspace{mm}
  }  
  
  

%%%%%%%%%%%%%%%%%%%%%%%%%
% Tip boxes
\definecolor{TipColour}{RGB}{220, 220, 220}
\newtcolorbox{tipcolourbox}{
  colback=TipColour,
  colframe=TipColour,
  coltext=black,
  boxsep=5pt,
  arc=4pt}

\newenvironment{tipBox}[1]
  {\small
  \begin{itemize}[leftmargin=.5in]
  \renewcommand{\labelitemi}{
    \raisebox{-.4\height}[0pt][0pt]{
      {\setkeys{Gin}{width=1.5em,keepaspectratio}
        \includegraphics{icons/#1}\qquad}
    }
  }
  \setlength{\fboxsep}{1em}
  \begin{tipcolourbox}\setlength{\parskip}{2mm}
  \item
  }
  {
  \end{tipcolourbox}
  \end{itemize}\vspace{-1mm}
}
  
%%%%%%%%%%%%%%%%%%%%%%%%%
% Important Box
\definecolor{ImportantColour}{RGB}{220, 220, 220}
\newtcolorbox{importantcolourbox}{
  colback=ImportantColour,
  colframe=ImportantColour,
  coltext=black,
  boxsep=5pt,
  arc=4pt}
\newenvironment{importantBox}[1]
  {\small
  \begin{itemize}[leftmargin=.5in]
  \renewcommand{\labelitemi}{
    \raisebox{-.4\height}[0pt][0pt]{
      {\setkeys{Gin}{width=1.5em,keepaspectratio}
        \includegraphics{icons/#1}\qquad}
    }
  }
  \setlength{\fboxsep}{1em}
  \begin{importantcolourbox}\setlength{\parskip}{2mm}
  \item
  }
  {
  \end{importantcolourbox}
  \end{itemize}\vspace{-1mm}
  } 

%%%%%%%%%%%%%%%%%%%%%%%%%
% Software Box
\definecolor{SoftwareColour}{RGB}{220, 220, 220}
\newtcolorbox{softwarecolourbox}{
  colback=SoftwareColour,
  colframe=SoftwareColour,
  coltext=black,
  boxsep=5pt,
  arc=4pt}
\newenvironment{softwareBox}[1]
  {\small
  \begin{itemize}[leftmargin=.5in]
  \renewcommand{\labelitemi}{
    \raisebox{-.4\height}[0pt][0pt]{
      {\setkeys{Gin}{width=1.5em,keepaspectratio}
        \includegraphics{icons/#1}\qquad}
    }
  }
  \setlength{\fboxsep}{1em}
  \begin{softwarecolourbox}\setlength{\parskip}{2mm}
  \item
  }
  {
  \end{softwarecolourbox}
  \end{itemize}\vspace{-1mm}
  } 


%%%%%%%%%%%%%%%%%%%%%%%%%
% Think Box
\definecolor{ThinkColour}{RGB}{245, 245, 245}
\newtcolorbox{thinkcolourbox}{
  colback=ThinkColour,
  colframe=ThinkColour,
  coltext=black,
  boxsep=5pt,
  arc=4pt}
\newenvironment{thinkBox}[1]
  {\small
  \begin{itemize}[leftmargin=.5in]
  \renewcommand{\labelitemi}{
    \raisebox{-.4\height}[0pt][0pt]{
      {\setkeys{Gin}{width=1.5em,keepaspectratio}
        \includegraphics{icons/#1}\qquad}
    }
  }
  \setlength{\fboxsep}{1em}
  \begin{thinkcolourbox}\setlength{\parskip}{2mm}
  \item
  }
  {
  \end{thinkcolourbox}
  \end{itemize}\vspace{-1mm}
  } 



%%%%%%%%%%%%%%%%%%%%%%%%%
%%% End of chapter answer boxes
\definecolor{EOCAnswerColour}{RGB}{240, 240, 240}
\newtcolorbox{EOCanswercolourbox}{
  colback=EOCAnswerColour,
  colframe=EOCAnswerColour,
  coltext=black,
  boxsep=5pt,
  arc=4pt}
  
  \newenvironment{EOCanswerBox}[1]
  {\footnotesize
  \begin{itemize}[leftmargin=.5in]
  \renewcommand{\labelitemi}{
    \raisebox{-.4\height}[0pt][0pt]{
      {\setkeys{Gin}{width=1.5em,keepaspectratio}
        \includegraphics{icons/#1}\qquad}
    }
  }
  \setlength{\fboxsep}{1em}
  \begin{EOCanswercolourbox}\setlength{\parskip}{1mm}
  \item
  }
  {
  \end{EOCanswercolourbox}
  \end{itemize}\vspace{-2mm}
  }  



%%%%%%%%%%%%%%%%%%%%%%%%%
% Pronounce Box
\definecolor{PronounceColour}{RGB}{220, 220, 220}
\newtcolorbox{pronouncecolourbox}{
  colback = PronounceColour,
  colframe = PronounceColour,
  coltext = black,
  boxsep = 5pt,
  arc = 4pt}
\newenvironment{pronounceBox}[1]
  {\small
  \begin{itemize}[leftmargin=.5in]
  \renewcommand{\labelitemi}{
    \raisebox{-.4\height}[0pt][0pt]{
      {\setkeys{Gin}{width=1.5em,keepaspectratio}
        \includegraphics{icons/#1}\qquad}
    }
  }
  \setlength{\fboxsep}{1em}
  \begin{pronouncecolourbox}\setlength{\parskip}{2mm}
  \item
  }
  {
  \end{pronouncecolourbox}
  \end{itemize}\vspace{-1mm}
  } 



%%%%%%%%%%%%%%%%%%%%%%%%%
% Answers chapter

% Answer Box
\newenvironment{answer}
  {\footnotesize}
  {\null\vspace{-5mm}} %{\smallskip}

% ChapAnswers
\newenvironment{ChapAnswers}
  {\null\vspace{-10mm}\null\begin{multicols}{2}\footnotesize\raggedright}
  {\null\vspace{-16mm}\end{multicols}\null\vspace{-16mm}\null} %{\smallskip}



%%%%%%%%%%%%%%%%%%%%%%%%%
% Note Box
\newenvironment{rmdnote}
  {\begin{rmdblock}{iconmonstr-light-bulb-2-240}}
  {\end{rmdblock}}
  

%%%%%%%%%%%%%%%%%%%%%%%%%
% Example extra Box
% exampleExtra: Omit this is latex
\usepackage{comment}
\excludecomment{exampleExtra}

\newenvironment{darkgraytext}{\color{textcolor}}{\ignorespacesafterend}


%%% REDEFINE some environments

%%% BASED ONL
%%%   https://stackoverflow.com/questions/58195679/how-can-i-redefine-a-standard-bookdown-theorem-environment-in-latex-pdf (I asked)
%% and adapted using:
%%%   https://stackoverflow.com/questions/1565988/making-a-small-modification-to-a-latex-environment

% REDEFINE some standard environments

%%%%%%%%%%%%%%%%%%%%%%%%%
% Redefine DEFINITION
\AtBeginDocument{%
\let\origenddefinition=\enddefinition%
\let\origdefinition=\definition%
\renewenvironment{definition}%
  {\begingroup\definecolor{shadecolor}{RGB}{230, 230, 230}\vspace{2mm}% Gap between above text, and top of box/shading  
   \begin{quote}\setlength{\parskip}{1mm}%
   \vspace{-4mm}% Gap between top of box/shading and the start of the text that is inside the box
   \origdefinition
   %\vspace{5mm}% Gap between top of box/shading and the start of the text that is inside the box
   }%
  {%\vspace{5mm}% Space between text and bottom of shaded box
  \origenddefinition\end{quote}
  \vspace{2mm}% Gap between bottom of box/shading and the start of the text below
  \endgroup}}


% %%%%%%%%%%%%%%%%%%%%%%%%%
% % Redefine EXAMPLE
\AtBeginDocument{%
\let\origendexample=\endexample%
\let\origexample=\example%
\renewenvironment{example}%
  {\begingroup\definecolor{shadecolor}{RGB}{245, 245, 245}\vspace{2mm}% Gap between above text, and top of box/shading
  \begin{quote}\setlength{\parskip}{1mm}% Gap between top of shading/box and start of in-box text
  \vspace{-4mm}% Gap between top of box/shading and the start of the text that is inside the box
  \origexample
  % Last vspace adjusts space between start of gray box, and start of text
  }%
  {%\vspace{5mm}% Space between text and bottom of shaded box
  \origendexample\end{quote}
  \vspace{2mm}% Gap between bottom of box/shading and the start of the text below
  \endgroup}}% Gap between bottom of box/shading and the start of the text below


%%%%%%%%%%%%%%%%%%%%%%%%%
% Redefine EXERCISE
\AtBeginDocument{%
\let\origendexercise=\endexercise%
\let\origexercise=\exercise%
\renewenvironment{exercise}%
  {\small\begingroup\setlist{nosep,topsep=-0.8\parskip}\vspace{5mm}% That final 5mm influences the gap between exercises
  \setlength{\parskip}{1.5mm}%
  \vspace*{-5mm}\origexercise}% First: influence the gap between exercises; second makes a non-breaking space between "Ex 3.1" and text that follows, even if an enumerate-like enviroment
  {\origendexercise\vspace{0mm}\endgroup}}
    


% Redfine some existing environments
% -QUOTE
\makeatletter
\renewenvironment{quote}
               {\list{}{%
               \vspace{-0.5em}% Top spacing, before the quote box itself
               \begin{kframe}\small%
                        \itemindent%
                        \listparindent
                        \rightmargin\leftmargin
                        \parsep        \z@ \@plus\p@}%
                \item\relax}
               {\end{kframe}\endlist\vspace{-0.25em}}
\makeatother

\let\origquote\quote
\let\origendquote\endquote
\renewenvironment{quote}{%
  \vspace{0\parskip} % Gap between text above, and start of quote shading
  \origquote
}%
{\origendquote\vspace{-0.5\parskip}}% Gap between text below, and bottom of quote shading



% Fold environment for LaTeX
\newenvironment{fold}
  {\vspace{-6pt}\begin{rmdblockNoIcon}\scriptsize \textbf{Answer:}}
  {\end{rmdblockNoIcon}}
  

% For CRC, according to `Run_This_Example.tex`:
\usepackage{hyperref}

% Some corrections to spacing
\usepackage{xspace}
\newcommand{\spacex}{\@ } % Using  \xspace  does not work for some reason

% Indexing
\makeindex
\indexsetup{othercode=\footnotesize} % Change font size
%\usepackage[nottoc,notbib]{tocbibind} # bib  already added

%% DELETE to use Helvitica font:
%\renewcommand*\familydefault{\sfdefault}
%\usepackage[T1]{fontenc}

%%
\usepackage{booktabs}
\usepackage{longtable}
\usepackage[bf,singlelinecheck=off]{caption}
\usepackage{times,pifont} % Nice fonts

\usepackage{fancyhdr}

\usepackage{tabularx} % For column specifications, using the kable_styling function  column_spec()
\usepackage{tabu} % For column specifications, using the kable_styling function  column_spec()
\usepackage{float} % Needed for some kableExtra stuff
\usepackage{mdframed}
\usepackage{enumitem} % To change the itemize spacing in exercises.
\usepackage{subfig} % For sub-figure captions
\usepackage[font={small,it}]{caption}
\usepackage{wrapfig} % For text-wrapping figures

% FIGURE PLACEMENT: https://tex.stackexchange.com/questions/140568/how-to-set-default-positioning-of-figure-table-document-wide
\makeatletter
  \providecommand*\setfloatlocations[2]{\@namedef{fps@#1}{#2}}
\makeatother
\setfloatlocations{figure}{hbtp}
\setfloatlocations{table}{hbtp}


%\setmainfont[UprightFeatures={SmallCapsFont=AlegreyaSC-Regular}]{Alegreya}

\usepackage{framed,color}
\definecolor{shadecolor}{RGB}{245,245,245}
\definecolor{lightshadecolor}{RGB}{230, 230, 255}
\definecolor{textcolor}{RGB}{100,100,100}

\definecolor{exampleExtraColor}{RGB}{209, 223, 250}
\definecolor{examplecolor}{RGB}{245, 245, 245}

\renewcommand{\textfraction}{0.05}
\renewcommand{\topfraction}{0.8}
\renewcommand{\bottomfraction}{0.8}
\renewcommand{\floatpagefraction}{0.75}

%\renewenvironment{quote}{\begin{VF}}{\end{VF}}
\renewenvironment{quote}%
  {\begin{kframe}}
  {\end{kframe}}



% Redo the  href  command
\let\oldhref\href
\renewcommand{\href}[2]{#2\footnote{\textbf{\url{#1}}}}

\ifxetex
  \usepackage{letltxmacro}
  \setlength{\XeTeXLinkMargin}{1pt}
  \LetLtxMacro\SavedIncludeGraphics\includegraphics
  \def\includegraphics#1#{% #1 catches optional stuff (star/opt. arg.)
    \IncludeGraphicsAux{#1}%
  }%
  \newcommand*{\IncludeGraphicsAux}[2]{%
    \XeTeXLinkBox{%
      \SavedIncludeGraphics#1{#2}%
    }%
  }%
\fi

% Default figure width
\setkeys{Gin}{width=0.5\linewidth}


\makeatletter
\newenvironment{kframe}{%
\medskip{}
\setlength{\fboxsep}{.8em}
 \def\at@end@of@kframe{}%
 \ifinner\ifhmode%
  \def\at@end@of@kframe{\end{minipage}}%
  \begin{minipage}{\columnwidth}%
 \fi\fi%
 \def\FrameCommand##1{\hskip\@totalleftmargin \hskip-\fboxsep
 \colorbox{shadecolor}{##1}\hskip-\fboxsep
     % There is no \\@totalrightmargin, so:
     \hskip-\linewidth \hskip-\@totalleftmargin \hskip\columnwidth}%
 \MakeFramed {\advance\hsize-\width
   \@totalleftmargin\z@ \linewidth\hsize
   \@setminipage}}%
 {\par\unskip\endMakeFramed%
 \at@end@of@kframe}
 \makeatother
 




\newenvironment{rmdthinkHTML}{}{} % DO nothing

\newenvironment{rmdblock}[1]
  {\setlength{\itemindent}{10em}\begin{quote}\vspace{-1em}
  \begin{itemize}\setlength{\parskip}{2mm}
  \renewcommand{\labelitemi}{
    \raisebox{-.5\height}[0pt][0pt]{
      {\setkeys{Gin}{width=1.5em,keepaspectratio}\includegraphics{icons/#1}}
    }
  }
  \setlength{\fboxsep}{1em}
  \begin{kframe}
  \item
  }
  {
  \end{kframe}
  \end{itemize}\vspace{-0.75em}\end{quote}
  }
\newenvironment{rmdblockNoIcon}
  {\begin{quote}\vspace{-0.75em}
  \setlength{\parskip}{2mm}
  %\begin{kframe}
  }
  {%\end{kframe}
  \vspace{-0.05em}\end{quote}
  }
  


\newenvironment{rmdobjectives}
  {\definecolor{shadecolor}{RGB}{230, 249, 255}\begin{rmdblock}{iconmonstr-target-4-240}}%230, 249, 255
  {\end{rmdblock}}
\newenvironment{rmdcontext}
  {\begin{rmdblock}{iconmonstr-puzzle-18-240}}
  {\end{rmdblock}}
\newenvironment{answer}
  {\null\vspace{-2mm}\small}
  {\null\vspace{-2mm}} %{\smallskip}
\newenvironment{rmdnote}
  {\begin{rmdblock}{iconmonstr-light-bulb-2-240}}
  {\end{rmdblock}}
\newenvironment{rmdspss}
  {\begin{rmdblock}{iconmonstr-laptop-4-240}}
  {\end{rmdblock}}
\newenvironment{rmdcaution}
  {\begin{rmdblock}{iconmonstr-warning-6-240}}
  {\end{rmdblock}}
\newenvironment{rmdimportant}
  {\definecolor{shadecolor}{RGB}{248, 237, 237}\begin{rmdblock}{iconmonstr-warning-8-240}}
  {\end{rmdblock}}
\newenvironment{rmdtip}
  {\begin{rmdblock}{iconmonstr-info-6-240}}
  {\end{rmdblock}}
\newenvironment{rmdwarning}
  {\begin{rmdblock}{iconmonstr-danger-13-240}}
  {\end{rmdblock}}
\newenvironment{rmdthink}
  {\definecolor{shadecolor}{RGB}{238, 235, 249}\begin{rmdblock}{iconmonstr-school-17-240}}
  {\end{rmdblock}}
\newenvironment{rmdpronunciation}
  {\begin{rmdblock}{iconmonstr-microphone-7-240}}
  {\end{rmdblock}}
  %\usepackage{xcolor}
\newenvironment{exampleExtra}
%  {\begin{rmdblock}{iconmonstr-idea-12-240}\small\textbf{Example.}}
  {\begin{rmdblockNoIcon}\small\textbf{Extra example:}}
  {\end{rmdblockNoIcon}}
\newenvironment{extraInfo}
%  {\begin{rmdblock}{iconmonstr-idea-12-240}\small\textbf{Example.}}
  {\begin{rmdblockNoIcon}\small\textbf{Extra information:}}
  {\end{rmdblockNoIcon}}



\newenvironment{darkgraytext}{\color{textcolor}}{\ignorespacesafterend}
%\AtBeginEnvironment{rmdblockNoIcon}{\begin{comment}}
%\AtEndEnvironment{rmdblockNoIcon}{\end{comment}}

%\usepackage{amsthm}
%\newtheorem*{ExampleFold}{Example}


%%% TRY HIDING EXAMPLE FOLDS, EXAMPLE FOLDS: To save paper in printing...???



% Redfine some existing environments
% -QUOTE
\makeatletter
\renewenvironment{quote}
               {\list{}{\vspace{-0.5em}\begin{kframe}\small%
                        \itemindent    \listparindent
                        \rightmargin   \leftmargin
                        \parsep        \z@ \@plus\p@}%
                \item\relax}
               {\end{kframe}\endlist}
\makeatother


% Change the lemma environment to be shaded
\usepackage{etoolbox}
\usepackage{framed}
\AtBeginEnvironment{lemma}{\begin{shaded}\begin{quote}\vspace{-2em}\setlength{\parskip}{6pt}} % Order must be shaded, then quote
\AtEndEnvironment{lemma}{\end{quote}\end{shaded}}


% ORIGINAL:
%\newenvironment{fold}
%  {\begin{rmdblock}{iconmonstr-idea-12-240}\vspace{-1em}\begin{darkgraytext}\footnotesize \textbf{Answer:}}
%  {\end{darkgraytext}\end{rmdblock}}
\newenvironment{fold}
  {\vspace{-6pt}\begin{rmdblockNoIcon}\scriptsize \textbf{Answer:}}
  {\end{rmdblockNoIcon}}
  
% Change the  fold  environment to be disappeared
% Note: In the LaTeX code, it appears like this: \BeginKnitrBlock{fold}
%\usepackage{comment}
%\AtBeginEnvironment{fold}{\begin{comment}}
%\AtEndEnvironment{fold}{\end{comment}}

%\begin{shaded}}
%\AtEndEnvironment{fold}{\end{shaded}}

%%% IF ONLY THIS WORKED:
% \usepackage{comment}
% \includecomment{fold}


%%% Two column chunks: from https://github.com/grantmcdermott/two-col-test/blob/master/preamble.css
\usepackage{multicol}
%% Note: Pandoc (which is doing all of the output conversion behind the scenes) does not parse the content of LaTeX environments. This creates problems when you try to include LaTeX commands with curly brackets directly in your Rmd file. E.g. You can't just use `\begin{multicols}{2}` directly in your Rmd file. Luckily, a straightforward workaround is to simply define some new shortcut commands yourself as per the below.
%% See: https://stackoverflow.com/questions/25849814/rstudio-rmarkdown-both-portrait-and-landscape-layout-in-a-single-pdf/27334272#27334272
\newcommand{\btwocol}{\begin{multicols}{2}}
\newcommand{\etwocol}{\end{multicols}}
            

\usepackage{makeidx}
\makeindex

\urlstyle{tt}


% NOTE:  clashes with  ntheorem  package... if I want to use that. I tried to use if to put lemma in shaded backgrounds... without luck. So I just put this code back.
\usepackage{amsthm}
\makeatletter
\def\thm@space@setup{%
  \thm@preskip=8pt plus 2pt minus 4pt
  \thm@postskip=\thm@preskip
}

\makeatother



\frontmatter

%%% Turn off maketitle to get pdf image as title page (https://stackoverflow.com/questions/45963505/coverpage-and-copyright-notice-before-title-in-r-bookdown):
\author{Peter K. Dunn}
\date{Last update: \today}
\let\oldmaketitle\maketitle
\AtBeginDocument{\let\maketitle\relax}





 %place custom commands and macros here

% Seem to need this expicitly to get similar vertical spacing on Windows and Mac...!
\usepackage{parskip}
\setlength{\parskip}{0.5\baselineskip}

%%%%% END PREAMBLE %%%%%
%%%%%

\newenvironment{cols}[1][]{}{}

\newenvironment{col}[1]{\begin{minipage}[t]{#1}\vspace{-6pt}\ignorespaces}{% % The \vspace seems needed, or image is not top aligned: https://tex.stackexchange.com/questions/580709/two-minipage-top-align-one-with-figure?rq=1
\end{minipage}
\ifhmode\unskip\fi
\aftergroup\useignorespacesandallpars}

\def\useignorespacesandallpars#1\ignorespaces\fi{%
#1\fi\ignorespacesandallpars}

\makeatletter
\def\ignorespacesandallpars{%
  \@ifnextchar\par
    {\expandafter\ignorespacesandallpars\@gobble}%
    {}%
}
\makeatother
\newenvironment{colsCentre}[1][]{}{}

\newenvironment{colCentre}[1]{\begin{minipage}[c]{#1}\vspace{0pt}\ignorespaces}{% % The \vspace seems needed, or image is not top aligned: https://tex.stackexchange.com/questions/580709/two-minipage-top-align-one-with-figure?rq=1
\end{minipage}
\ifhmode\unskip\fi
\aftergroup\useignorespacesandallpars}

\def\useignorespacesandallpars#1\ignorespaces\fi{%
#1\fi\ignorespacesandallpars}

\makeatletter
\def\ignorespacesandallpars{%
  \@ifnextchar\par
    {\expandafter\ignorespacesandallpars\@gobble}%
    {}%
}
\makeatother
\usepackage{bookmark}
\IfFileExists{xurl.sty}{\usepackage{xurl}}{} % add URL line breaks if available
\urlstyle{same}
% Make links footnotes instead of hotlinks:
\DeclareRobustCommand{\href}[2]{#2\footnote{\url{#1}}}
\hypersetup{
  pdftitle={Scientific Research and Methodology},
  pdfauthor={Peter K. Dunn},
  colorlinks=true,
  linkcolor={Maroon},
  filecolor={Maroon},
  citecolor={Blue},
  urlcolor={Blue},
  pdfcreator={LaTeX via pandoc}}

\title{Scientific Research and Methodology}
\usepackage{etoolbox}
\makeatletter
\providecommand{\subtitle}[1]{% add subtitle to \maketitle
  \apptocmd{\@title}{\par {\large #1 \par}}{}{}
}
\makeatother
\subtitle{An introduction to quantitative research and statistics}
\author{Peter K. Dunn}
\date{2025-07-18 09:36:19.631277}

\usepackage{amsthm}
\newtheorem{theorem}{Theorem}[chapter]
\newtheorem{lemma}{Lemma}[chapter]
\newtheorem{corollary}{Corollary}[chapter]
\newtheorem{proposition}{Proposition}[chapter]
\newtheorem{conjecture}{Conjecture}[chapter]
\theoremstyle{definition}
\newtheorem{definition}{Definition}[chapter]
\theoremstyle{definition}
\newtheorem{example}{Example}[chapter]
\theoremstyle{definition}
\newtheorem{exercise}{Exercise}[chapter]
\theoremstyle{definition}
\newtheorem{hypothesis}{Hypothesis}[chapter]
\theoremstyle{remark}
\newtheorem*{remark}{Remark}
\newtheorem*{solution}{Solution}
\begin{document}
\maketitle

% For CRC, according to `Run_This_Example.tex`:


\frontmatter

\title{Scientific Research and Methodology} %This is a placeholder titlepage, it will not be final.
\author{Peter K. Dunn}
%%%\maketitle

%%%Placeholder for front matter

\halftitle

\booktitle

\locpage

%%% START FORMER: DEDICATION.TEX
\cleardoublepage
\thispagestyle{empty}
\vspace*{\stretch{1}}
\begin{center}
\Large\itshape
To a random sample of seven people,\index{Sampling!random}\\ 
drawn from the scores of people who have contributed to the production of this book\\
and whose contributions I greatly appreciate.
\end{center}
\vspace{\stretch{2}}
%%% END FORMER: DEDICATION.TEX
%\cleardoublepage
\thispagestyle{empty}
\vspace*{\stretch{1}}
\begin{center}
\Large\itshape
To a random sample of seven people,\\ 
drawn from the thousands of people who have contributed to the production of this book.
\end{center}
\vspace{\stretch{2}}

\cleardoublepage
\setcounter{page}{7} %previous pages will be reserved for frontmatter to be added in later.
%\setcounter{tocdepth}{2}
%\tableofcontents
%\include{frontmatter/foreword}
%\include{frontmatter/preface}
%%\listoffigures
%%\listoftables
%\include{frontmatter/contributor}
%\include{frontmatter/symbollist}



\index{Cohort study|see{Study types; Study types, directionality, forward}}
\index{Cross-sectional studies|see{Study types; Study types, directionality, non-directional}}
\index{Cross-over studies|see{Study types}}
\index{Case-control studies|see{Study types; Study types, directionality, backward}}

\index{Quantitative research|see{Research}}
\index{Qualitative research|see{Research}}
\index{Mixed-methods research|see{Research}}

\index{Descriptive RQs|see{Research question}}
\index{Relational RQs|see{Research question}}
\index{Repeated-measures RQs|see{Research question}}
\index{Correlational RQs|see{Research question}}

\index{Individuals!zzzzz@\igobble |seealso{Units of analysis}}
\index{Cases!zzzzz@\igobble |seealso{Units of analysis}}
\index{Subjects!zzzzz@\igobble |seealso{Units of analysis}}
\index{Random allocation|see{Confounding, random allocation}}
\index{Treatments!zzzzz@\igobble |seealso{Conditions; Interventions}}
\index{Conditions!zzzzz@\igobble |seealso{Treatments}}

\index{Variables!response|see{Response variable}}
\index{Variables!explanatory|see{Explanatory variable}}


\index{Random sampling|see{Sampling, random}}
\index{Non-random sampling|see{Sampling, non-random}}
\index{Representative sampling|see{Sampling, representative}}

\index{Sampling distribution!zzzzz@\igobble |seealso{Sampling mean; Standard error; Sampling variation}}
\index{Sampling variation!zzzzz@\igobble |seealso{Sampling distribution; Standard error; Sampling variation}}
\index{Sampling mean!zzzzz@\igobble |seealso{Sampling distribution; Standard error; Sampling variation}}
\index{Standard error!zzzzz@\igobble |seealso{Sampling distribution; Sampling mean; Sampling variation}}


\index{Tables!two-way|see{Two-way tables}}
\index{Contingency tables|see{Two-way tables}}
\index{Frequency table!zzzzz@\igobble |seealso{Two-way tables}}


\index{Variables!control!zzzzz@\igobble |seealso{Confounding, control variables}}
\index{Control group|see{Control}}

\index{Bimodal|see{Shape}}
\index{Skewness!zzzzz@\igobble |seealso{Shape}}

\index{Bell-shaped distribution|see{Normal distribution}}
\index{Exclusion criteria!zzzzz@\igobble |seealso{Inclusion criteria}}
\index{Inclusion criteria!zzzzz@\igobble |seealso{Exclusion criteria}}

\index{Paired data|see{Data, paired}}
\index{Levels|see{Qualitative data, levels}}

\index{Independent variable!zzzzz@\igobble |seealso{Explanatory variable}}
\index{Dependent variable!zzzzz@\igobble |seealso{Response variable}}

\index{Experimental units|see{Unit of analysis}}
\index{Experimenter effect|see{Observer effect}}

\index{PICO|see {POCI}}

\index{Practical importance!zzzzz@\igobble |seealso{Statistical significance}}
\index{Statistical significance!zzzzz@\igobble |seealso{Practical importance}}

\index{Sampling!zzzzz@\igobble |seealso{Accuracy; External validity; Precision}}
\index{Sampling!non-random!self-selecting|see{Sampling, non-random, voluntary}}

\index{Confounder|see{Confounding; Variables, confounding}}
\index{Lurking variable|see{Variables, lurking}}
\index{Confounding variable|see{Variables, confounding}}

\index{Research question!zzzzz@\igobble |seealso{POCI}}
\index{RQ|see{Research question}}
\index{Research process!zzzzz@\igobble |seealso{Research, six steps}}

\index{Making decisions|see{Decision making}}

\index{Categorical data|see{Qualitative data}}


\index{CI|see{Confidence interval}}
\index{Inference|see{Confidence intervals; Hypothesis testing}}

\index{t@$t$-score|see{Test statistic}}
\index{z@$z$-score!zzzzz@\igobble |seealso{Test statistic}}
\index{$\chi^2$-score|see{Test statistic}}
\index{Chi@$\chi^2$-score|see{Test statistic}}
\index{Chi-square score|see{Test statistic}}

\index{r@$r$|see{Correlation coefficient}}
\index{Q@$Q_1$|see{Quartiles}}
\index{Q@$Q_3$|see{Quartiles}}
\index{Q@$Q_2$|see{Median; Quartiles}}
\index{R@$R^2$!zzzzz@\igobble |seealso{Correlation coefficient}}

\index{Comparison!within individuals!zzzzz@\igobble |seealso{Paired data}}

\index{Observer effect!zzzzz@\igobble |seealso{Blinding, researchers}}
\index{Hawthorne effect!zzzzz@\igobble |seealso{Blinding, individuals}}
\index{Carryover effect!zzzzz@\igobble |seealso{Washouts}}
\index{Double blinding|see{Blinding, double}}
\index{Triple blinding|see{Blinding, triple}}
\index{Single blinding|see{Blinding, single}}

%\index{Natural variation!zzzzz@\igobble |seealso{Variation}}

\index{Generalisability|see{Limitations}}
\index{Effectiveness|see{Limitations}}
\index{Practicality|see{Limitations}}

\index{Study types!zzzzz@\igobble |seealso{Research design}}
\index{Study design|see{Research design}}

\index{Data!zzzzz@\igobble |seealso{Qualitative data; Quantitative data}}
\index{Questionnaire!zzzzz@\igobble |seealso{Survey}}
\index{Survey!zzzzz@\igobble |seealso{Questionnaire}}
\index{Data collection!zzzzz@\igobble |seealso{Protocol}}

\index{Averages!zzzzz@\igobble |seealso{Mean; Median}}
\index{Mean!zzzzz@\igobble |seealso{Average; Median}}
\index{Median!zzzzz@\igobble |seealso{Average; Mean}}
\index{Quantitative data!averages!zzzzz@\igobble |seealso{Averages; Mean; Median}}

\index{Quantitative data!variation!zzzzz@\igobble |seealso{Interquartile range (IQR); Range; Percentiles; Standard deviation; Variation, compared}}
\index{Quantitative data!outliers!zzzzz@\igobble |seealso{Outliers}}

\index{Percentages!zzzzz@\igobble |seealso{Proportions}}

\index{Estimate!zzzzz@\igobble |seealso{Confidence interval}}
\index{Parameter!zzzzz@\igobble |seealso{Statistics}}
\index{Statistic!zzzzz@\igobble |seealso{Parameter}}
\index{Protocol!zzzzz@\igobble |seealso{Data collection}}

\index{Model!zzzzz@\igobble |seealso{Normal distribution}}
\index{Normal model|see{Normal distribution}}
\index{Distribution!zzzzz@\igobble |seealso{Normal distribution}}
\index{Normal distribution!zzzzz@\igobble |seealso{$68$--$95$--$99.7$ rule; $z$-scores}}

\index{Empirical rule|see{$68$--$95$--$99.7$ rule}}

\index{Correlation coefficient (Pearson)!zzzzz@\igobble |seealso{$R^2$}}
\index{Correlation coefficient!Pearson|see{Correlation coeffficient (Pearson)}}
\index{Regression!equation!zzzzz@\igobble |seealso{Linear equations}}
\index{Regression equation!zzzzz@\igobble |seealso{Linear equation; Regression}}
\index{Linear equations!zzzzz@\igobble |seealso{Regression}}

\index{False positive|see{Type\ I error}}
\index{False negative|see{Type\ II error}}

\index{True experiment|see{Study types}}
\index{Quasi-experiment|see{Study types}}
\index{Observational study|see{Study types}}
\index{Experiment|see{Study types}}
\index{Experimenter effect|see{Observer effect}}

\index{Jittering|see{Overplotting}}
\index{Stacking|see{Overplotting}}

\index{ANOVA|see{Analysis of variance}}

\index{Conditional probability|see{Probability, conditional}}

\index{Software|see{Computers and software}}
\index{Computing|see{Computers and software}}

\index{Accuracy!zzzzz@\igobble |seealso{Precision}}
\index{Precision!zzzzz@\igobble |seealso{Accuracy}}

\index{Back-to-back stemplot|see{Graphs}}
\index{Bar chart|see{Graphs}}
\index{Boxplot|see{Graphs}}
\index{Box-and-whisker plot|see{Graphs, boxplot}}
\index{Parallel boxplot|see{Graphs, boxplot}}
\index{Side-by-side boxplot|see{Graphs, boxplot}}
\index{Case-profile plot|see{Graphs}}
\index{Dot chart|see{Graphs}}
\index{Error bar chart|see{Graphs}}
\index{Histogram|see{Graphs}}
\index{Histogram of differences|see{Graphs}}
\index{Pie chart|see{Graphs}}
\index{Scatterplot|see{Graphs}}
\index{Side-by-side bar chart|see{Graphs}}
\index{Stacked bar chart|see{Graphs}}
\index{Stemplot|see{Graphs}}
\index{Stem-and-leaf plot|see{Graphs, stemplot}}
\index{Bins (in histograms)|see{Graphs, histograms}}
\index{Bins (in frequency tables)|see{Frequency table}}

\index{Graphs!overplotting|see{Overplotting}}


\index{Difference between proportions!zzzzz@\igobble |seealso{Odds ratio}}
\index{Odds ratio!zzzzz@\igobble |seealso{Difference between proportions}}
\index{OR|see{Odds ratio}}

\index{Subjective data|see{Data, subjective}}
\index{Objective data|see{Data, objective}}

\index{AI|see{Artificial intelligence}}

\index{Sensitivity!zzzzz@\igobble |seealso{Specificity}}
\index{Specificity!zzzzz@\igobble |seealso{Sensitivity}}
\index{Statistical significance!seealso{$P$-value}}

{
\hypersetup{linkcolor=}
\setcounter{tocdepth}{1}
\tableofcontents
}
\newcommand{\cms}{\,\text{cm}}
\newcommand{\dLs}{\,\text{dL}}
\newcommand{\xdLs}{\text{dL}}

\newcommand{\fmols}{\,\text{fmol}}
\newcommand{\gs}{\,\text{g}}
\newcommand{\hs}{\,\text{h}}
\newcommand{\xhs}{\text{h}}

\newcommand{\has}{\,\text{ha}}
\newcommand{\xhas}{\text{ha}}

\newcommand{\kgs}{\,\text{kg}}
\newcommand{\kms}{\,\text{km}}
\newcommand{\kWhs}{\,\text{kWh}}
\newcommand{\lbs}{\,\text{lb}}
\newcommand{\Ls}{\,\text{L}}
\newcommand{\xLs}{\text{L}}

\newcommand{\mgs}{\,\text{mg}}
\newcommand{\microgs}{\,\ensuremath{\mu}\text{g}}
\newcommand{\millis}{\,\text{ms}}
\newcommand{\mins}{\,\text{mins}}
\newcommand{\mJs}{\,\text{mJ}}
\newcommand{\mmols}{\,\text{mmol}}
\newcommand{\mLs}{\,\text{mL}}
\newcommand{\mms}{\,\text{mm}}
\newcommand{\ms}{\,\text{m}}
\newcommand{\xms}{\text{m}}
\newcommand{\ozs}{\,\text{oz}}
\newcommand{\secs}{\,\text{s}}
\newcommand{\xsecs}{\text{s}}
\newcommand{\ppms}{\,\text{ppm}}
\newcommand{\ys}{\,\text{y}}
\newcommand{\vs}{\,\text{V}}

\frontmatter

\chapter*{Preface}\label{preface}

This book introduces quantitative research in the scientific and health disciplines, with an emphasis on introductory statistics. Unlike many introductory statistics textbooks, this textbook gives context to the statistics by first covering the basics of the research design process; it connects the research question with the means to answer that question. I believe this is crucial to understanding the need and purpose of using statistics. The research process is broken into six steps, which provide the framework for the content.

The book is designed for teaching at first-year undergraduate level, with examples mostly drawn from science, health and engineering. Many real journal articles are used throughout the text in examples, to demonstrate the use of the techniques. Almost every dataset used in this book is real and available in the \textbf{R}~package \textbf{SRMData} (see App.~\ref{AppendixDataSets}).

The main focus of the book is the analysis of data, with an emphasis on understanding the underlying concepts rather than a focus on using mathematics. Software output\index{Software output} is often used to help when calculations become onerous. The output is from jamovi \citep{Software:jamovi},\index{Computers and software!jamovi} but is sufficiently generic that no knowledge of jamovi is necessary to use this book, and this book can be read without relying on any specific statistical software. (jamovi, however, is \emph{free} to download and use.)

The following call-outs are used in this book:

\begin{objectivesBox}{iconmonstr-target-4-240.png}
These chunks introduce the objectives for the chapters of the book.

\end{objectivesBox}

\smallskip

\begin{importantBox}{iconmonstr-warning-8-240.png}
These chunks highlight common mistakes or warnings, about a particular concept or about using a formula.

\end{importantBox}

\begin{tipBox}{iconmonstr-info-6-240.png}
These chunks offer helpful information.

\end{tipBox}

\begin{softwareBox}{iconmonstr-laptop-4-240.png}
These chunks refer to information about using software or a calculator.

\end{softwareBox}

\begin{pronounceBox}{iconmonstr-microphone-7-240.png}
These chunks indicate how certain symbols and terms are pronounced.

\end{pronounceBox}

\begin{EOCanswerBox}{iconmonstr-check-mark-14-240.png}
These end-of-chapter chunks provide answers to the end-of-chapter \emph{Quick review questions}.

\end{EOCanswerBox}

This book was made using \textbf{R} \citep{Software:Rsoftware} with the \textbf{bookdown} package \citep{Software:Rbookdown}, using \textbf{Markdown} syntax and \textbf{knitr} \citep{package:knitr} and numerous other \textbf{R} packages. All of this software is \emph{free} and open source. Other resources used include:

\begin{itemize}
\tightlist
\item
  various icons from \textbf{iconmonstr} (freely available).
\item
  the images of the cards (e.g., in Sect.~\ref{NeedForDecisionMaking}), which are in the public domain and available from \url{https://code.google.com/archive/p/vector-playing-cards/}.
\end{itemize}

Earlier drafts of this textbook have been used to teach thousands of students, and the book has been used by many fantastic teaching assistants. I thank all of them for their feedback. Special thanks to Dr Amanda Shaker (La~Trobe University), who reported numerous issues in earlier editions (and often provided corrections).

\section*{Learning Outcomes}\label{learning-outcomes}

\begin{objectivesBox}{iconmonstr-target-4-240.png}

In this book, you will learn to:

\begin{itemize}
\tightlist
\item
  develop quantitative research questions and testable hypotheses.
\item
  design quantitative studies to answer simple quantitative research questions.
\item
  select and produce appropriate graphical, numerical and statistical analyses.
\item
  select, apply and interpret the results of the correct statistical technique to analyse data.
\item
  comprehend, apply and communicate in the language of research and statistics.
\item
  demonstrate professional integrity in planning, interpreting and reporting the results of quantitative studies.
\end{itemize}

\end{objectivesBox}

\mainmatter

\chapter{Research: an introduction}\label{Intro}

\begin{objectivesBox}{iconmonstr-target-4-240.png}

In this chapter, you will learn to:

\begin{itemize}
\tightlist
\item
  identify quantitative and qualitative research.
\item
  identify the steps in the quantitative research process.
\end{itemize}

\end{objectivesBox}

\section{Introduction: how we know what we know}\label{HowDoWeKnow}

Scientists once believed that all life regularly and commonly arose spontaneously from non-living matter. \emph{Recipes} even existed; for example, \citet{data:VanHelmont:Transformations} gave this recipe for making a mouse \citep{pasteur1922generations}:

\begin{quote}
If a soiled shirt is placed in the opening of a vessel containing grains of wheat, the reaction of the leaven in the shirt with fumes from the wheat will, after approximately twenty-one days, transform the wheat into mice.
\end{quote}

This was called `spontaneous generation' (or `abiogenesis'). This theory is clearly incorrect, so how did the idea emerge? How was it disproven? Through \emph{observation} and \emph{research}.

Spontaneous generation was consistent with \emph{observations}: following the above recipe \emph{did} produce mice. However, the hypothesis\index{Hypotheses} (`possible explanation') of spontaneous generation was rejected when later evidence, in better-designed research studies, contradicted the hypothesis. A new hypothesis\index{Hypotheses} was proposed to explain the appearance of the mice, which was tested against the evidence, and so on. Briefly, this is the \emph{evidence-based, scientific process}.\index{Scientific process}

As a more recent example, the dangers of smoking were still being debated into the 1990s:

\begin{quote}
\ldots{} a causal role for smoking {[}has{]} not been proved beyond reasonable doubt.

\VA{--- \citet{eysenck1991were}, p.~429}{}
\end{quote}

All scientific knowledge emerges in a similar way: observations lead to questions and hypotheses, which are tested against \emph{evidence}. If the evidence \emph{contradicts} the hypothesis, the hypothesis is rejected. If the evidence is \emph{consistent} with the hypothesis, the hypothesis is \emph{temporarily} accepted (until any contradictory evidence emerges, if ever).

Hypotheses not contradicted by large amounts of evidence, over a long time, are sometimes called \emph{laws} or \emph{theories} (such as the `Law of conservation of energy'). Theories and laws can be disproven if contradictory evidence emerges. Knowledge in all scientific disciplines is accumulated using a similar evidence-based process.

\section{Evidence-based research}\label{EvidenceBasedResearch}

\index{Evidence-based research}

Every discipline changes, develops, improves, and adapts---usually through \emph{research}. Your discipline is not the same as it was ten years ago; it will change in the next ten years. Scientists, engineers and health practitioners need to know how to understand and adapt to this change.

Remaining current in your discipline requires understanding research, even if you will not be researching yourself. You still need to know the language, tools, concepts and ideas of research, and you need to be able to critique research. Research is the foundation of science.

Science seeks \emph{evidence-based answers}:\index{Evidence-based research} reaching conclusions based on \emph{evidence}, rather than hunches, feelings, intuition, hopes, or tradition. The \emph{evidence} comes from analysing \emph{data}.\index{Data}

\begin{definition}[Data]
\protect\hypertarget{def:Data}{}\label{def:Data}\index{Data}\index{Dataset} \emph{Data} refers to information (observations or measurements), such as numbers, labels, recordings, videos, text, etc.

A \emph{dataset} refers to an organised and structured collection of data.
\end{definition}

Research involves asking research questions, designing studies to collect data, analysing data, and accurately reporting the results. This book covers all these components.

\section{Quantitative, qualitative, and mixed-methods research}\label{TypesOfResearch}

Research can be broadly classified as \emph{qualitative} or \emph{quantitative}. These are different yet complementary approaches to answering research questions (Table~\ref{tab:TypesOfResearch}).\index{Research!qualitative}\index{Research!quantitative} Both methods have advantages and disadvantages, and can be used together (called \emph{mixed-methods} research).\index{Research!mixed-methods} The decision to use qualitative, quantitative or mixed-methods approaches depends on the purpose of the research.

\begin{table}
\centering
\caption{\label{tab:TypesOfResearch}Concisely comparing qualitative and quantitative research.}
\centering
\fontsize{8}{10}\selectfont
\begin{tabular}[t]{r>{}cl}
\toprule
\textbf{Qualitative} & \textbf{Aspect} & \textbf{Quantitative}\\
\midrule
Feelings, opinions & \textbf{What} & Measured or observed data\\
Suggest hypotheses, explore, depth & \textbf{Why} & Make objective conclusions\\
Non-random samples & \textbf{Studied} & Representative samples; random samples\\
Very detailed; for specific groups & \textbf{Conclusions} & General\\
\addlinespace
Words, audio, video, pictures, ... & \textbf{Data} & Numbers, measurements, counts, ...\\
Usually small samples are studied & \textbf{Size} & Often large samples are studied\\
Often time-consuming & \textbf{Time} & Usually more efficient\\
Rarely generalisable & \textbf{Applicability} & Often generalisable\\
\addlinespace
Thematic analysis, content analysis, etc. & \textbf{Analysis} & Numerical summaries, test hypotheses, etc.\\
Interviews, focus groups & \textbf{Examples} & Experiments, closed surveys, lab. studies\\
\bottomrule
\end{tabular}
\end{table}

Briefly, \emph{qualitative research} leads to a deeper understanding, usually for a very specific group. Meanings, motivations, opinions or themes often emerge from qualitative research. In contrast, \emph{quantitative research} summarises and analyses data usually from large groups, using \emph{numerical} methods, such as averages and percentages. In quantitative research, typically information about a large group of interest (a \emph{population})\index{Population} is found from a subset of the population (a \emph{sample}).\index{Sample}

\begin{definition}[Quantitative research]
\protect\hypertarget{def:QualitativeResearch}{}\label{def:QualitativeResearch}\emph{Quantitative research} summarises and analyses data using numerical methods, such as averages, proportions and percentages.
\end{definition}

\begin{importantBox}{iconmonstr-warning-8-240.png}
This book is about \emph{quantitative} research.

\end{importantBox}

\begin{example}[Types of research]
\protect\hypertarget{exm:BroadTypesOfResearch}{}\label{exm:BroadTypesOfResearch}\citet{oliveira2020wireless} used mixed-methods research to study the adoption of electric taxis in Nottingham (UK).

In the \emph{quantitative} component of the study, taxis were tracked for over \(32\,000\,\text{km}\) and \(9\,764\,\text{h}\). The researchers determined where taxis often stopped (as potential charging locations). The Trent Street taxi rank (near the main train station) was the most-used stop with an average wait time of \(20\,\text{mins}\); Milton St (\(17\,\text{mins}\)) and Wheeler gate (\(10\,\text{mins}\)) also recorded high average stop-times. Numerical information about speeds was also obtained.

In the \emph{qualitative} component of the study, nine taxis drivers participated in interviews and focus groups. This allowed the researchers to (p.~6)

\begin{quote}
\ldots explore themes and motivations in a way that was not possible with the initial quantitative analysis.
\end{quote}

Participants' responses were classified as \emph{barriers} (safety; costs; speeds) or \emph{facilitators} (opportunities to charge regularly; convenient locations) to the proposed charging locations.
\end{example}

\begin{example}[Quantitative research]
\protect\hypertarget{exm:QuantResearchLegionella}{}\label{exm:QuantResearchLegionella}During 1988/1989, an unusually high number of \emph{Legionella longbeachae} infections were observed in South Australians. The researchers \citep{data:oconnor:pottingmix} wanted to identify the source to prevent further infections.

The researchers noticed that many of those infected were gardeners who had recently handled potting mix, so they hypothesised that the infection was associated with using potting mix. They designed a study to test this hypothesis, then collected data from \(100\)~people (\(25\)~\emph{with} the infection, and \(75\)~people of similar age and sex \emph{without} the infection).\index{Study types!case-control studies}

The researchers classified and summarised the data, then analysed the data to reach an evidence-based conclusion: potting mix was partially, but not solely, responsible for the infections. The researchers communicated their recommendations to reduce the risks of people contracting the infection in the future.
\end{example}

\section{The steps in research}\label{SixStepsOfResearch}

\index{Research!six steps}

The research process (for both qualitative and quantitative research) ideally follows the process summarised in Fig.~\ref{fig:SixSteps}, but this is not always possible or practical. The process is not always linear: researchers may jump from step to step as necessary, and research often leads to new research questions (RQs) so that the process restarts. Each step is important.\index{Research!six steps}

\begin{figure}[hbtp]

{\centering \includegraphics[width=0.45\linewidth]{01-Introduction_files/figure-latex/SixSteps-1} 

}

\caption{The six basic steps in research.}\label{fig:SixSteps}
\end{figure}

\begin{itemize}
\tightlist
\item
  \emph{Asking} the RQ (Chap.~\ref{RQs}). Research begins with a research question to answer.
\item
  \emph{Designing} the study (Chaps.~\ref{ResearchDesignOverview} to~\ref{Interpretation}). Evidence-based research uses data to answer the RQ. A study is designed to obtain that data: determining who or what to study; finding those to study; deciding what information to obtain; and ensuring data are obtained ethically.
\item
  \emph{Collecting} the data (Chap.~\ref{CollectingDataProcedures}). The data collection process must be ethical, reproducible and clearly documented.
\item
  \emph{Classifying} and \emph{summarising} the data (Chaps.~\ref{DescribingVars} to~\ref{SummariseComments}). Before analysis, the data must be classified and summarised to inform the analysis. (A computer is useful.)
\item
  \emph{Analysing} the data (Chaps.~\ref{Probability} to~\ref{SelectTest}). Analysis refers to using the data to find an answer to the research question. (A computer is useful.)
\item
  \emph{Reporting} the results (Chaps.~\ref{WritingResearch} and~\ref{Reading}). Communicating the results appropriately, accurately and ethically is important, including identifying any limitations of the research.
\end{itemize}

\section{Using computers in research}\label{Software-In-Research}

\index{Computers and software!in research}

\emph{Statistical} software (such as Python, jamovi, R, SAS, SPSS, Stata, etc.) is useful for summarising and analysing data.\index{Computers and software!statistical} Statistical software:

\begin{itemize}
\tightlist
\item
  is designed for working with large datasets.
\item
  encourages reproducible research (Sect.~\ref{ReproducibleResearch}).\index{Research!reproducibility}
\item
  allows high-precision formatting and graphics.
\item
  is powerful; with some programming skills, almost anything is possible.
\item
  is specifically designed for analysing and working with data.
\end{itemize}

\begin{softwareBox}{iconmonstr-laptop-4-240.png}
This book sometimes shows output from jamovi \citep{Software:jamovi}, but using jamovi is \emph{not} essential for understanding this book.\index{Computers and software!jamovi}

\end{softwareBox}

Using spreadsheets\index{Computers and software!spreadsheets} for storing and analysing data requires care. Expensive and dangerous errors have been made due to using spreadsheets \citep{altarawneh2017pilot}. Some challenges with using spreadsheets include that:

\begin{itemize}
\tightlist
\item
  spreadsheets may \emph{change data} (e.g., reformatting entries as dates) when not appropriate \citep{ziemann2016gene}.
\item
  spreadsheets may include \emph{formulas with errors} that are difficult to locate and hence fix \citep{panko2016we, Retraction:London:Excel}.
\item
  spreadsheets \emph{do not leave a record} of how the data were analysed or prepared. Keeping a record of the analysis, preparation of variables, and other operations with the data is good scientific practice (\emph{reproducible research}; see Sect.~\ref{ReproducibleResearch}) \citep{simons2019reproducible}.
\item
  spreadsheets often produce poor graphs \citep{su2008s}.
\end{itemize}

Problems with spreadsheets, as with any software, are often due to human error, but \emph{spreadsheets make errors hard to find and hence hard to fix}. Spreadsheets are useful for data collection and basic data manipulation, but are not designed for scientific analysis. Be careful using spreadsheets for research and analysis.

\section{Exercises}\label{IntroExercises}

\hyperref[Answers]{Answers to odd-numbered exercises} are given at the end of the book.

\captionsetup{font=small}

\begin{exercise}
\protect\hypertarget{exr:RQsTypeTourniquet}{}\label{exr:RQsTypeTourniquet}

Consider this RQ: `For three different junctional tourniquets, which is the quickest, on average, to apply?'

\begin{enumerate}
\def\labelenumi{\arabic{enumi}.}
\tightlist
\item
  What data would be likely be needed to answer this research question?
\item
  Is this RQ likely to be answered using a \emph{quantitative} or \emph{qualitative} research study? Explain.
\end{enumerate}

\end{exercise}

\begin{exercise}
\protect\hypertarget{exr:RQsTypeMangroves}{}\label{exr:RQsTypeMangroves}

Consider this RQ: `Why do people dump rubbish in mangroves?'

\begin{enumerate}
\def\labelenumi{\arabic{enumi}.}
\tightlist
\item
  What data would be likely be needed to answer this research question?
\item
  Is this RQ likely to be answered using a \emph{quantitative} or \emph{qualitative} research study? Explain.
\end{enumerate}

\end{exercise}

\begin{exercise}
\protect\hypertarget{exr:RQsTypeSideEffects}{}\label{exr:RQsTypeSideEffects}

Consider this RQ: `What percentage of Egyptians experience side effects from a specific medication?'

\begin{enumerate}
\def\labelenumi{\arabic{enumi}.}
\tightlist
\item
  What data would be likely be needed to answer this research question?
\item
  Is this RQ likely to be answered using a \emph{quantitative} or \emph{qualitative} research study? Explain.
\end{enumerate}

\end{exercise}

\begin{exercise}
\protect\hypertarget{exr:RQsTypeSolarPanels}{}\label{exr:RQsTypeSolarPanels}

Consider this RQ: `What is the average number of rooftop solar panels on domestic homes in a certain city?'

\begin{enumerate}
\def\labelenumi{\arabic{enumi}.}
\tightlist
\item
  What data would be likely be needed to answer this research question?
\item
  Is this RQ likely to be answered using a \emph{quantitative} or \emph{qualitative} research study? Explain.
\end{enumerate}

\end{exercise}

\begin{exercise}
\protect\hypertarget{exr:RQsTypeGreenery}{}\label{exr:RQsTypeGreenery}\citet{frost2023encouraging} conducted a study to better understand the views of adults regarding planting greenery in front gardens. To do so, they conducted (p.~80):

\begin{quote}
\ldots{} five online focus groups with \(20\)~participants aged~\(20\)--\(64\) in England\ldots{[}then{]} audio recorded each focus group, transcribed it verbatim and analysed transcripts using thematic analysis.
\end{quote}

What type of research study is this: qualitative, quantitative or mixed-methods?
\end{exercise}

\begin{exercise}
\protect\hypertarget{exr:RQsTypeGreenHydrogen}{}\label{exr:RQsTypeGreenHydrogen}\citet{haussermann2023social} studied Germany's transition to green energy, and (p.~1):

\begin{quote}
\ldots{} investigated social acceptance of green hydrogen at an early stage in its implementation, before wider rollout.
\end{quote}

To do so, they used (p.~1):

\begin{quote}
\ldots{} semi-structured interviews (\(n = 24\)) and two participatory workshops (\(n = 51\)) in a selected region in central Germany serve alongside a representative survey (\(n = 2\,054\))\ldots{}
\end{quote}

What type of research study is this: qualitative, quantitative or mixed-methods?
\end{exercise}

\captionsetup{font=normalsize}

\part{Asking research questions}\label{part-asking-research-questions}

\chapter{Research questions}\label{RQs}

\begin{cols}
\begin{col}{0.52\textwidth}

\begin{objectivesBox}{iconmonstr-target-4-240.png}
\textbf{In this chapter}, you will learn to:

\begin{itemize}\tightlist
  \item
  identify and write quantitative research questions.
  \item
  identify the variables implied by a quantitative research question.
  \item
  identify and distinguish observational and experimental studies.
  \item
  identify and distinguish the units of analysis and units of observations in a study.
  \item
  write operational and conceptual definitions.
\end{itemize} 
\end{objectivesBox}

\end{col}

\begin{col}{0.03\textwidth}
~
\end{col}

\begin{col}{0.45\textwidth}

\includegraphics[width=0.95\linewidth]{02-RQs_files/figure-latex/unnamed-chunk-6-1} 
\end{col}
\end{cols}

\section{Introduction}\label{Chap2-Intro}

The research question (RQ) directs all other components of the research. Since quantitative research summarises and analyses data using numerical methods (like averages or percentages), the RQ must be \emph{written} carefully so it can be \emph{answered} effectively. Four different types of RQs are studied:

\begin{itemize}
\tightlist
\item
  descriptive RQs (Sect.~\ref{RQsDescriptive}).
\item
  relational RQs (Sect.~\ref{RQsRelational}).
\item
  repeated-measures RQs (Sect.~\ref{RQsRepeatedMeasures}).
\item
  correlational RQs (Sect.~\ref{RQsCorrelational}).
\end{itemize}

Since the RQ directs all other components of the research, writing RQs should be the first step of any research study. Specifically, RQs should be asked before data are collected.

\begin{importantBox}{iconmonstr-warning-8-240.png}
RQs should be written \emph{before data are collected}.

\end{importantBox}

\section{Descriptive RQs}\label{RQsDescriptive}

\index{Research question!descriptive|(}

All RQs identify a large group of interest to be studied (called a \emph{population}),\index{Population} and study something \emph{about} that population (called the \emph{outcome}).\index{Outcome}

The population is any broad group of interest; for example:

\begin{itemize}
\tightlist
\item
  all German males between~\(18\) and~\(35\) years of age.
\item
  all bamboo flooring materials manufactured in China.
\item
  all elderly females with glaucoma in Canada.
\item
  all \emph{Pinguicula grandiflora} growing in Europe.
\end{itemize}

\begin{definition}[Population]
\protect\hypertarget{def:Population}{}\label{def:Population}\index{Population}\index{Individuals} A \emph{population} is a group of \emph{individuals} from which the total set of observations of interest \emph{could} be made, and to which the results will generalise.
\end{definition}

Populations comprise many \emph{individuals} (or \emph{cases}).\index{Individuals}\index{Cases} If the individuals are people, individuals may also be called \emph{subjects}.\index{Subjects}

\begin{importantBox}{iconmonstr-warning-8-240.png}
The words \emph{population}, \emph{individuals} and \emph{cases} do \emph{not} just refer to people, though they may be commonly used that way in general conversation.

\end{importantBox}

Data are rarely taken from all the individuals in the population: \emph{all} individuals are rarely accessible in practice. For example, testing a new drug cannot possibly study \emph{all} people who might use the drug (some may not even be born yet). In contrast, a \emph{sample} is a \emph{subset} of the population from which data are obtained (Chap.~\ref{Sampling}). Countless samples are possible from any given population, but only one is studied.

\begin{definition}[Sample]
\protect\hypertarget{def:Sample}{}\label{def:Sample}\index{Sample} A \emph{sample} is a subset of individuals from the population. The data are collected from the sample.
\end{definition}

\begin{importantBox}{iconmonstr-warning-8-240.png}
The \emph{population} in an RQ is \emph{not} just those studied; it is the whole group to which results could generalise.

\end{importantBox}

\begin{example}[Samples]
\protect\hypertarget{exm:Samples}{}\label{exm:Samples}A study of American college women \citep{data:woolf:ironstatus} compared iron status in highly-active and sedentary women.

The study compared \(28\) active and~\(28\) sedentary American college women, from which data were collected. The \emph{population} was \emph{all} active and sedentary American college women. The group of \(56\)~subjects was the \emph{sample}.
\end{example}

Descriptive RQs study something \emph{about} the identified population, called the \emph{outcome}. Because the RQ concerns a large group (the population), the outcome numerically describes a \emph{group} of individuals (not single individuals). The outcome is, for example, an \emph{average}\index{Averages} or \emph{proportion}\index{Proportions} summarising a group of individuals.

\begin{definition}[Outcome]
\protect\hypertarget{def:Outcome}{}\label{def:Outcome}\index{Outcome} The \emph{outcome} in an RQ is the result, output, consequence or effect of interest in a study, numerically summarised for a group of individuals.
\end{definition}

The outcome of interest in a population may be (for example) the

\begin{itemize}
\tightlist
\item
  \emph{average} amount of wear after~\(1\,000\,\text{h}\) of use.
\item
  \emph{proportion} of people whose pupils dilate.
\item
  \emph{average} weight loss after three weeks on a diet.
\item
  \emph{percentage} of seedlings that die.
\end{itemize}

\begin{importantBox}{iconmonstr-warning-8-240.png}
\index{Outcome}\index{Population}\index{Individuals} The \emph{outcome} in an RQ summarises a \emph{population}; it does not describe the \emph{individuals} in the population.

\end{importantBox}

Descriptive RQs can now be introduced.

\begin{definition}[Descriptive RQ]
\protect\hypertarget{def:DescriptiveRQ}{}\label{def:DescriptiveRQ}\emph{Descriptive RQs} have a population and an outcome.
\end{definition}

Some RQs ask about the \emph{value} of some population quantity (such as: what is the average internal body temperature?); these are called \emph{estimation} RQs. Some RQs require \emph{making a decision} about the population (such as: is the average internal body temperature the same for females and males?); these are called \emph{decision-making} RQs. Descriptive RQs have one of these forms, depending on what information is sought (Sect.~\ref{TwoPurposesOfRQs}):

\begin{itemize}
\tightlist
\item
  \emph{estimation} RQs: Among \{\emph{the population}\}, what is \{\emph{the outcome}\}?\index{Research question!estimation}
\item
  \emph{decision-making} RQs: Among \{\emph{the population}\}, is \{\emph{the outcome}\} equal to \{\emph{a given value}\}?\index{Research question!decision-making}
\end{itemize}

\begin{importantBox}{iconmonstr-warning-8-240.png}
These templates are \emph{not} `recipes', but guidelines.

\end{importantBox}

Answering \emph{estimation} descriptive RQs is studied in Chaps.~\ref{CIOneProportion} and~\ref{OneMeanConfInterval}. Answering \emph{decision-making} descriptive RQs is studied in Chaps.~\ref{TestOneProportion} and~\ref{TestOneMean}.

\begin{example}[Descriptive RQ]
\protect\hypertarget{exm:DescriptiveRQBodyTemp}{}\label{exm:DescriptiveRQBodyTemp}

\citet{data:mackowiak:bodytemp} studied men and women aged~\(18\) to~\(40\); this is the \emph{population}. The \emph{outcome} of interest in this population is the \emph{average body temperature}. The sample comprised~\(148\) `healthy men and women' aged~\(18\) to~\(40\). One descriptive RQ was:

\begin{quote}
What is the average body temperature?
\end{quote}

This is an \emph{estimation} RQ.\spacex They also studied a \emph{decision-making} descriptive RQ (where~\(98.6\)\textsuperscript{o}F (or~\(37.0\)\textsuperscript{o}C) is a commonly-accepted value for the internal body temperature):

\begin{quote}
Is the average body temperature really~\(98.6\)\textsuperscript{o}F (\(37.0\)\textsuperscript{o}C)?
\end{quote}

\end{example}

\index{Research question!descriptive|)}

\section{Relational RQs}\label{RQsRelational}

\index{Research question!relational|(}

Studying relationships usually is more interesting than simply describing a population. \emph{Relational RQs} compare the outcome for groups of different individuals in the population, or compare two different sub-populations. These comparisons are called \emph{between-individuals} comparisons,\index{Comparison!between individuals} as they compare the outcome \emph{between} (or among) groups of \emph{different} individuals. Examples include:

\begin{itemize}
\tightlist
\item
  comparing the average amount of wear in floorboards \emph{between} two different groups: standard wooden floorboards, and bamboo floorboards.
\item
  comparing the average heart rate \emph{across} three groups of people: those not receiving the drug, those receiving a weekly dose, and those receiving a daily dose of the drug.
\end{itemize}

\begin{definition}[Comparison (between individuals)]
\protect\hypertarget{def:ComparisonBetween}{}\label{def:ComparisonBetween}The \emph{between-individuals comparison} in an RQ identifies the small number of groups of different individuals for which the outcome is compared.
\end{definition}

\begin{example}[Between-individuals comparison]
\protect\hypertarget{exm:BetweenPossums}{}\label{exm:BetweenPossums}\citet{data:Williams2022:Possums} compared the average weight of female and male Leadbeater's possums. `Sex of the possum' is the \emph{between-individuals} comparison; average weight is the outcome.
\end{example}

Relational RQs can now be introduced.

\begin{definition}[Relational RQ]
\protect\hypertarget{def:RelationalRQ}{}\label{def:RelationalRQ}\emph{Relational RQs} have a population, outcome, and a \emph{between}-individuals comparison.
\end{definition}

Relational RQs have one of these forms, depending on what information is sought:

\begin{itemize}
\tightlist
\item
  \emph{estimation} RQ: Among \{\emph{the population}\}, what is the difference in \{\emph{the outcome}\} for \{\emph{the groups being compared}\}?
\item
  \emph{decision-making} RQ: Among \{\emph{the population}\}, is \{\emph{the outcome}\} the same for \{\emph{the groups being compared}\}?
\end{itemize}

\begin{example}[Relational RQ]
\protect\hypertarget{exm:RelationalRQ}{}\label{exm:RelationalRQ}Consider this RQ (based on \citet{estevez2019influence}):

\begin{quote}
Among Cubans between~\(13\) and~\(20\) years of age, is the average heart rate the same for females and males?
\end{quote}

The \emph{population} is `Cubans~\(13\) and~\(20\) years of age', the \emph{outcome} is `\emph{average} heart rate', and the \emph{between-individuals comparison} is between two separate groups: `between females and males'. This is a \emph{relational RQ}.

This is a \emph{decision-making RQ},\index{Research question!decision-making} since it asks if the average heart rate is the same for females and males. An \emph{estimation}-type relational RQ would ask about the \emph{size} of difference in the average heart rate between females and males.
\end{example}

\index{Research question!relational|)}

\section{Repeated-measures RQs}\label{RQsRepeatedMeasures}

\index{Research question!repeated-measures|(}

Rather than comparing the outcome for groups of different individuals, \emph{repeated-measures RQs} compare the outcome multiple times within the \emph{same} individuals.

These comparisons are called \emph{within-individuals} comparisons,\index{Comparison!within individuals} as they compare the outcome \emph{within the same individuals}, not across groups of \emph{different} individuals. The multiple measurements may be different points in time (e.g., the height of the same trees at one, two and five years after planting), but do not have to be time points.

\clearpage

Examples include:

\begin{itemize}
\tightlist
\item
  comparing the average strength of hind legs of horses to the forelegs of the same horses.
\item
  comparing the average thickness of the cornea in left eyes and right eyes of the same individuals.
\item
  comparing the average amount of wear in many individual floorboards after one, five and ten years of use.
\end{itemize}

\begin{definition}[Within-individuals comparison]
\protect\hypertarget{def:ComparisonWithin}{}\label{def:ComparisonWithin}The \emph{within-individuals comparison} in the RQ identifies the small number of different, distinct situations for which the outcome is compared for each individual.
\end{definition}

\begin{example}[Between- and within-individual comparisons]
\protect\hypertarget{exm:WithinBetweenComparison}{}\label{exm:WithinBetweenComparison}Consider comparing the strength of the dominant and non-dominant legs of professional football players.

A \emph{between}-individuals comparison would compare the average strengths of the dominant and non-dominant legs \emph{between different} groups of footballers: one group would have their dominant-leg strength measured, and the other would have their non-dominant-leg strength measured. This is a \emph{between}-individuals comparison.

In contrast, the strengths of the dominant and non-dominant legs could be recorded on the \emph{same} individuals. This study examines \emph{within}-individuals changes: the average differences between the strengths of the dominant and non-dominant legs \emph{within} the same individuals. In this study, \emph{no between-individuals comparison} exists: different groups are not being compared.
\end{example}

Studies may use \emph{both} within- and between-individuals comparisons (see Sect.~\ref{ChamomileTea-TwoMeans}). For instance, a study may examine the \emph{change} in individuals' heart rate (the \emph{within}-individuals comparison), for two drugs given to different groups (the \emph{between}-groups comparison).

Repeated-measures RQs can now be introduced.

\begin{definition}[Repeated-measures RQ]
\protect\hypertarget{def:RepeatedMeasuresRQ}{}\label{def:RepeatedMeasuresRQ}\emph{Repeated-measures RQs} have a population, outcome and a \emph{within}-individuals comparison.
\end{definition}

Repeated-measures RQs have one of these forms, depending on what information is sought:

\begin{itemize}
\tightlist
\item
  \emph{estimation} RQ: Among \{\emph{the population}\}, what is the change in \{\emph{the outcome}\} for \{\emph{the alternatives being compared within individuals}\}?
\item
  \emph{decision-making} RQ: Among \{\emph{the population}\}, is \{\emph{there a change in the outcome}\} for \{\emph{the alternatives being compared within individuals}\}?
\end{itemize}

\begin{example}[Repeated-measure RQ]
\protect\hypertarget{exm:WithinRelationalRQ}{}\label{exm:WithinRelationalRQ}\citet{rowland2017comparing} compared the temperature in the \emph{same} tree hollows in summer and winter:

\begin{quote}
For tree hollows in the Strathbogie Ranges, Australia, what is the average temperature difference between summer and winter?
\end{quote}

The comparison is \emph{within individuals}, as the temperature is measured for the \emph{same} tree hollows at the two times. This is a repeated-measures, estimation-type RQ.
\end{example}

Repeated-measures RQs with only two within-individual comparisons are often called \emph{paired}.\index{Data!paired}\index{Study types!paired}

\begin{example}[Paired repeated-measures study]
\protect\hypertarget{exm:RepeatedMeasuresPaired}{}\label{exm:RepeatedMeasuresPaired}\citet{levitsky2004freshman} compared the weights of the same university students at the beginning of university, and then after \(12\)~weeks. The comparison is \emph{within} individuals, and the study is a \emph{repeated-measures} study. Since each student has a \emph{pair} of weight measurements, this is a \emph{paired} study.
\end{example}

\index{Research question!repeated-measures|)}

\section{Variables}\label{Variables}

\index{Variables}

RQs are about \emph{populations}. However, the data to answer an RQ come from \emph{individuals} in that population. The aspects or characteristics that can \emph{vary} called \emph{variables}.

\begin{definition}[Variable]
\protect\hypertarget{def:Variable}{}\label{def:Variable}A \emph{variable} is a single aspect or characteristic, associated with the individuals, whose values can vary.
\end{definition}

\begin{example}[Variables]
\protect\hypertarget{exm:Variables2}{}\label{exm:Variables2}Examples of variables include: the duration of cold symptoms; sex; tree girth; response to a survey question (Yes, Maybe, No); city of birth; hair colour.
\end{example}

Some variables change from one individual to another individual, such as sex and height. These are called \emph{between}-individuals variables. In repeated-measures studies, some variables of interest change over repeated measurements from the same individuals; these are called \emph{within}-individuals variables.

\begin{definition}[Between- and within-individuals variables]
\protect\hypertarget{def:BetweenWithinVariable}{}\label{def:BetweenWithinVariable}\emph{Between}-individuals variables vary from one individual to another individual.\index{Variables!between-individuals} \emph{Within}-individuals variables vary from one recording or measurement to another \emph{within} the same individuals.\index{Variables!within-individuals}
\end{definition}

\begin{importantBox}{iconmonstr-warning-8-240.png}
A between-individuals variable is a single aspect that can vary from \emph{individual to individual}. While \emph{your} city of birth does not change, `city of birth' is a variable because it varies from \emph{individual} to \emph{individual}.

\end{importantBox}

\begin{example}[Within-individuals variables]
\protect\hypertarget{exm:WithinIndividualsVariables}{}\label{exm:WithinIndividualsVariables}\citet{rowland2017comparing} compared the temperature in the \emph{same} tree hollows in summer and winter (Example~\ref{exm:WithinRelationalRQ}). The comparison is \emph{within individuals}: the temperature is measured for the \emph{same} tree hollows (the \emph{individuals}) at two different times.

`Season' is a within-individuals variable, as each tree hollow is studied for two different seasons. `Temperature' is also a within-individuals variable, as it is measured twice for each tree hollow.
\end{example}

\begin{example}[Between-individuals comparison]
\protect\hypertarget{exm:BetweenPossums2}{}\label{exm:BetweenPossums2}\citet{data:Williams2022:Possums} compared the average weight of female and male Leadbeater's possums (Example~\ref{exm:BetweenPossums}).

`Sex of the possum' is a \emph{between-individuals} variable; it can vary from possum to possum. `Weight' is also a \emph{between-individuals} variable; it can vary from possum to possum.
\end{example}

\begin{example}[Variables]
\protect\hypertarget{exm:Variables}{}\label{exm:Variables}`Duration of cold symptoms' is a between-individuals \emph{variable}: its value can vary from individual to individual. The `\emph{average} duration of cold symptoms' is the \emph{outcome}, a numerical summary of many individuals' cold durations.
\end{example}

While many variables can be recorded, two essential variables are (Table~\ref{tab:RQsPopulationIndividuals}):

\begin{itemize}
\tightlist
\item
  the \emph{response variable}, which records information to determine the outcome.\index{Response variable}
\item
  the \emph{explanatory variable}, which records information to determine the comparison.\index{Explanatory variable}
\end{itemize}

Usually, one variable can be considered as perhaps influencing the value of the other variable. This variable is called the \emph{explanatory variable} (which may \emph{explain} changes in the other variable). The other is the \emph{response variable} (whose values \emph{respond} to changes in the explanatory variable). To be able to influence the response variable, the explanatory variable must occur before (or at the same time) as the response variable.

\begin{table}
\centering\centering
\caption{\label{tab:RQsPopulationIndividuals}The relationship between the population and the individuals.}
\centering
\fontsize{8}{10}\selectfont
\begin{tabular}[t]{rcl}
\toprule
\textbf{Population} & \textbf{} & \textbf{Individuals}\\
\midrule
Outcome & $\rightarrow$ & Response variable\\
Comparison & $\rightarrow$ & Explanatory variable\\
\bottomrule
\end{tabular}
\end{table}

The value of the \emph{response} variable may change in \emph{response} to the value of the explanatory variable. The value of the \emph{explanatory} variable may \emph{explain} changes in the value of the response variable.

\begin{definition}[Explanatory variable]
\protect\hypertarget{def:ExplanatoryVariable}{}\label{def:ExplanatoryVariable}An \emph{explanatory variable} may (partially) explain or be associated with changes in another variable of interest (the response variable).
\end{definition}

\begin{definition}[Response variable]
\protect\hypertarget{def:ResponseVariable}{}\label{def:ResponseVariable}A \emph{response variable} records the result, output, consequence or effect of interest from changes in another variable (the explanatory variable).
\end{definition}

\begin{importantBox}{iconmonstr-warning-8-240.png}
The \emph{response variable} is sometimes called the \emph{dependent variable},\index{Dependent variable} and the \emph{explanatory variable} is sometimes called the \emph{independent variable}.\index{Independent variable} We avoid these terms, since the words `dependent' and `independent' have many meanings in research.

\end{importantBox}

The RQ cannot be answered without data for the response and explanatory variables. The \emph{outcome} is a numerical summary of the values of the response variable (Table~\ref{tab:RQsPopulationIndividualsExamplesOutcome}) recorded from many individuals. The values of the explanatory variable distinguish between the values of the \emph{comparison} for the individuals (Tables~\ref{tab:RQsPopulationIndividualsExamplesComparison} and~\ref{tab:RQsPopulationIndividualsExamplesComparisonWithin}) being made.\index{POCI}

\begin{table}
\centering
\caption{\label{tab:RQsPopulationIndividualsExamplesOutcome}Outcomes and corresponding response variable.}
\centering
\fontsize{8}{10}\selectfont
\begin{tabular}[t]{rcl}
\toprule
\textbf{Outcome describing the population} & \textbf{} & \textbf{Response variable in individuals}\\
\midrule
\emph{Average} diastolic blood pressure & $\rightarrow$ & Diastolic blood pressure of \emph{individuals}\\
\emph{Percentage} of seedlings that sprout & $\rightarrow$ & Whether an \emph{individual} seedling sprouts\\
\emph{Proportion} owning iPad & $\rightarrow$ & Whether an \emph{individual} owns an iPad\\
\emph{Average} cold duration & $\rightarrow$ & Cold duration for \emph{individuals}\\
\bottomrule
\end{tabular}
\end{table}



\begin{table}
\centering
\caption{\label{tab:RQsPopulationIndividualsExamplesComparison}\emph{Between-individuals} comparisons and the corresponding \emph{between-individuals} explanatory variable.}
\centering
\fontsize{8}{10}\selectfont
\begin{tabular}[t]{rcl}
\toprule
\textbf{Comparison being made} & \textbf{} & \textbf{Explanatory variable in individuals}\\
\midrule
Between jarrah, beech, bamboo boards & $\rightarrow$ & Type of floorboard in \emph{different} individual homes\\
Between $3\,\text{kg}$/ha, $4\,\text{kg}$/ha fertiliser rates & $\rightarrow$ & Application rate in \emph{different} individual paddocks\\
Between people in $20$s, $30$s and $40$s & $\rightarrow$ & Age group for each \emph{different} individual person\\
\bottomrule
\end{tabular}
\end{table}



\begin{table}
\centering
\caption{\label{tab:RQsPopulationIndividualsExamplesComparisonWithin}\emph{Within-individuals} comparison and corresponding \emph{within-individuals} explanatory variable.}
\centering
\fontsize{8}{10}\selectfont
\begin{tabular}[t]{rcl}
\toprule
\textbf{Comparison being made} & \textbf{} & \textbf{Explanatory variable in individuals}\\
\midrule
Before and after receiving a drug & $\rightarrow$ & When measured on \emph{each} individual person\\
Between left and right arms & $\rightarrow$ & Which arm in \emph{each} individual person is used\\
Between forelegs and hind legs & $\rightarrow$ & Which legs are measured on \emph{each} individual horse\\
\bottomrule
\end{tabular}
\end{table}

\begin{example}[Variables]
\protect\hypertarget{exm:POCIplaygrounds}{}\label{exm:POCIplaygrounds}Consider a study of the ground surface temperature of public playgrounds in Boston in summer.

The \emph{population} comprises all public playgrounds in Boston; each public playground is an \emph{individual}. The \emph{outcome} is the \emph{average} ground surface temperature in summer over many playgrounds; the \emph{response variable} is the ground surface temperature for \emph{individual} ground surfaces in summer.

The between-individuals \emph{comparison} is between the four types of ground surfaces (rubber, soil, sand, mulch). The \emph{explanatory variable} is the type of surface for individual playgrounds.
\end{example}

\section{Correlational RQs}\label{RQsCorrelational}

\index{Research question!correlational|(}

\emph{Correlational RQs} are not concerned with summarising outcomes in comparison \emph{groups}. Instead, correlational RQs explore relationships between two variables measured or observed on or about the individuals.

\begin{definition}[Correlational RQ]
\protect\hypertarget{def:CorrelationalRQ}{}\label{def:CorrelationalRQ}\emph{Correlational RQs} explore the relationship between two variables.
\end{definition}

Correlational RQs have one of these forms, depending on what information is sought:

\begin{itemize}
\tightlist
\item
  \emph{estimation} RQ: Among \{\emph{the population}\}, how strong is the relationship between \{\emph{the response variable}\} and \{\emph{the explanatory variable}\}?
\item
  \emph{decision-making} RQ: Among \{\emph{the population}\}, is \{\emph{the response variable}\} related to \{\emph{the explanatory variable}\}?
\end{itemize}

Examples include studying the relationship between:

\begin{itemize}
\tightlist
\item
  the height of plants (response variable) and the number of hours of sunlight per day (explanatory variable).
\item
  heart rate (response variable) and the number of grams of caffeine consumed that day (explanatory variable).
\end{itemize}

Usually, one variable can be considered as the explanatory variable, and the other as the response variable (Sect.~\ref{Variables}). To be able to influence the response variable, the explanatory variable must occur before (or at the same time) as the response variable. Explanatory and response variables may be either within- or between-individuals variables.

\begin{example}[Correlational RQ]
\protect\hypertarget{exm:CorrelationalRunners}{}\label{exm:CorrelationalRunners}Consider studying marathon runners. An RQ exploring the relationship between the individuals' water intake on the day before the race and the individuals' race times would be a correlational RQ.\spacex The water intake on the day before the race \emph{may} influence the race time.

The water intake on the day before the race is the explanatory variable, and the race time is the response variable.
\end{example}

\begin{example}[Correlational RQ]
\protect\hypertarget{exm:CorrelationalPine}{}\label{exm:CorrelationalPine}The Wollemi pine was discovered by science in~1994. \citet{offord2023home} studied the growth of these rare plants.

One correlational RQ concerned the relationship between the diameter of trees at breast height (DBH; response variable), and the pH of the soil (explanatory variable). The two variables are the DBH and pH, both recorded for many trees.

Also studied was the relationship between the DBH for each tree at various times after the planting date (a repeated-measure RQ). Each tree has the DBH measured over time, for many time points. Time is the \emph{within}-individuals comparison.
\end{example}

In some situations, the variables are neither response nor explanatory variable; the interest is just in the association between the two variables.

\begin{example}[Correlation RQ]
\protect\hypertarget{exm:ResearchDesignFishSize}{}\label{exm:ResearchDesignFishSize}

\citet{gonzalez2024length} recorded the length and weight of \(14\,040\)~fish for~\(39\) demersal fish species. The study has two variables (fish length; fish weight), but identifying a response variable and explanatory variable is meaningless. The estimation-type correlational RQ is:

\begin{quote}
Among demersal fish, how strong is the relationship between length and weight?
\end{quote}

\end{example}

\index{Research question!correlational|)}

\section{Interventions}\label{Intervention}

\index{Intervention}

Sometimes, the explanatory variable naturally occurs without manipulation by the researchers (e.g., the height of people; the sex of oxen; the pH of forest soil). Sometimes, however, the explanatory variable is manipulated by researchers (e.g., the dose of fertiliser applied; the dose of drug given); this is called an \emph{intervention}.

\begin{definition}[Intervention]
\protect\hypertarget{def:Intervention}{}\label{def:Intervention}An \emph{intervention} is present when \emph{researchers} can manipulate (or impose) the values of the \emph{explanatory variable} on the individuals to determine the impact on the response variable.
\end{definition}

When an intervention is present, the values of the explanatory variable are \emph{manipulated} by the researchers, and are called \emph{treatments}. When an intervention is \emph{not} present, the values of the explanatory variable are \emph{not} manipulated by the researchers, and are called \emph{conditions}. The \emph{analysis} is the same whether an intervention is used or not, but the \emph{interpretation} of the results depend on whether an intervention is used (Sect.~\ref{CompareStudyTypes}).

\begin{definition}[Treatments]
\protect\hypertarget{def:Treatments}{}\label{def:Treatments}\index{Treatments} The \emph{treatments} are the values of the explanatory variable that the researchers can manipulate and impose upon the individuals.
\end{definition}

\begin{definition}[Condition]
\protect\hypertarget{def:Conditions}{}\label{def:Conditions}\index{Conditions} The \emph{conditions} are the values of the explanatory variable that those in the study have or experience, but are not manipulated or imposed by the researchers.
\end{definition}

An intervention is present when the researchers:

\begin{itemize}
\tightlist
\item
  explicitly give a dose of a new drug to patients.
\item
  explicitly apply wear-testing loads to two different flooring materials.
\item
  explicitly expose people to different stimuli.
\item
  explicitly apply different doses of fertiliser.
\end{itemize}

\begin{example}[Intervention]
\protect\hypertarget{exm:InterventionHimalaya}{}\label{exm:InterventionHimalaya}\citet{data:Bird2008:wholegrain} \emph{supplied} one group of participants with a diet using refined flour, and \emph{supplied} another group of participants with a diet using a new flour variety. `Type of diet' is the (between-individuals) explanatory variable. Since the researchers manipulate which subjects ate which flour, this study has an intervention. `Type of diet' is the treatment.
\end{example}

\begin{example}[No intervention]
\protect\hypertarget{exm:Interventions}{}\label{exm:Interventions}To compare the average blood pressure in female and male Scots, blood pressure was measured using a blood pressure machine (a sphygmomanometer). The researchers interact with the participants to measure blood pressure, but there is \emph{no} intervention. Using the sphygmomanometer is just a way to measure blood pressure, to \emph{obtain} the data.

The \emph{comparison} is between females and males (the conditions), which cannot be manipulated or imposed on the individuals by the researchers; \emph{there is no intervention}.
\end{example}

Often, one of the comparison groups is the \emph{control group}. The \emph{control group} is a comparison group \emph{not} receiving the treatment being studied, or \emph{not} having the condition being studied, but \emph{as similar as possible} to the other individuals in all other ways. The control group is like a benchmark for detecting changes in the outcome due to the treatment or condition of interest (Sect.~\ref{PlaceboEffect}). Sometimes the control group is given a \emph{placebo}: a non-effective treatment that appears to be the real treatment.

\begin{definition}[Control]
\protect\hypertarget{def:Control}{}\label{def:Control}A \emph{control} is an individual without the treatment or condition of interest, but as similar as possible in \emph{every other way} to other individuals. A \emph{control group} is a group of controls.
\end{definition}

\begin{definition}[Placebo]
\protect\hypertarget{def:Placebo}{}\label{def:Placebo}A \emph{placebo} is a treatment with no intended effect or active ingredient, but appears to be the real treatment.
\end{definition}

\begin{example}[Control group]
\protect\hypertarget{exm:ControlGroup}{}\label{exm:ControlGroup}To test the effectiveness of a new medication, patients report to a doctor to receive injections of the new drug. Patients assigned to the \emph{control group} do not receive the drug. The controls should also report to a doctor and receive an injection (like those receiving the drug); the injection, however, would contain no active ingredients (a placebo).
\end{example}

Together, the \textbf{P}opulation, \textbf{O}utcome, \textbf{C}omparison and \textbf{I}ntervention form the POCI acronym\index{POCI} (sometimes written as PICO) to aid remembering the elements of RQs.\spacex The POCI acronym is not helpful for correlational RQs.

\begin{example}[POCI]
\protect\hypertarget{exm:POCIWomen}{}\label{exm:POCIWomen}\citet{data:woolf:ironstatus} measured iron status in highly-active and sedentary American college women.

The \emph{outcome} is the `average iron status'. The between-individuals \emph{comparison} is between highly-active and sedentary women. For this comparison to be an intervention, the \emph{researchers} would need to tell each individual woman to be highly active or sedentary. This seems unlikely, so the study does not have an intervention.
\end{example}

\section{Estimation and decision-making RQs}\label{TwoPurposesOfRQs}

\index{Research question!estimation}\index{Research question!decision-making}

As noted earlier, RQs can be written with one of two purposes. \emph{Estimation RQs} ask how precisely an unknown \emph{value} in the \emph{population} is estimated by the \emph{sample}. Estimation RQs are answered using \emph{confidence intervals}, which are discussed in Chaps.~\ref{CIOneProportion} to~\ref{OneMeanConfInterval}, Chaps.~\ref{AnalysisPaired} to~\ref{AnalysisOddsRatio}, plus Sects.~\ref{CorrelationTesting} and~\ref{RegressionHT}.

\emph{Decision-making RQs} require a decision to be made about the unknown values in the population. They are answered using \emph{hypothesis tests}, and discussed in Chaps.~\ref{TestOneProportion} to~\ref{TestOneMean}, Chaps.~\ref{AnalysisPaired} to~\ref{AnalysisOddsRatio}, plus Sects.~\ref{CorrelationTesting} and~\ref{RegressionHT}.

\begin{example}[Decision-making RQs]
\protect\hypertarget{exm:TypesOfRQS}{}\label{exm:TypesOfRQS}

\citet{data:Thane2004:ZincVitA} studied `British young people aged~\(4\)--\(18\)' and asked numerous RQs. One \emph{decision-making} relational RQ was:

\begin{quote}
In British young people aged~\(4\)--\(18\), is the average daily zinc intake the same for boys and girls?
\end{quote}

\end{example}

Decision-making RQ have two possible answers.\index{Decision making} For the example above, the average zinc intake either \emph{is} the same for boys and girls, or \emph{is not} the same for boys and girls, in the \emph{population} (Fig.~\ref{fig:ZincRQ}). These two options are \emph{hypotheses}: potential answers to the RQ.\spacex\index{Hypotheses} However, answers are rarely clear in practice, since only one of the countless possible samples from the population is studied. Instead, researchers decide \emph{how strongly} the sample evidence supports a particular hypothesis about the \emph{population}.\index{Hypotheses}

Evidence may \emph{support} or \emph{contradict} a hypothesis; evidence rarely \emph{proves} a hypothesis (at least, without any other support, such as theoretical support). Ultimately, after collecting data from a \emph{sample}, a decision must be made about which explanation about the \emph{population} is more consistent with the data collected.

\begin{figure}[hbtp]

{\centering \includegraphics[width=1\linewidth]{02-RQs_files/figure-latex/ZincRQ-1} 

}

\caption{Two possible answers to the RQ (two hypotheses) about zinc intake in children.}\label{fig:ZincRQ}
\end{figure}

Decision-making RQs can be asked in different ways.\index{Research question!one- and two-tailed} For the zinc-intake study above (Fig.~\ref{fig:ZincRQ}), the RQ could ask (about the population):

\begin{itemize}
\tightlist
\item
  is the average zinc intake \emph{the same} for boys and girls?
\item
  is the average zinc intake \emph{different} for boys and girls?
\item
  is the average zinc intake \emph{lower} for boys, compared to girls?
\item
  is the average zinc intake \emph{higher} for boys, compared to girls?
\end{itemize}

The first two are \emph{two-tailed RQs} (and are essentially asking the same question but in different ways): the average zinc intake could be higher for girls or higher for boys. We are just interested in whether any difference is present; that is, two options are being considered. The last two are \emph{one-tailed RQ}, since they ask specifically about a difference in just one direction: boys lower than girls, or boys higher than girls.

Most RQs are two-tailed, unless a good reason exists to ask a one-tailed RQ \emph{before} the data are collected (e.g., a drug has been developed specifically to \emph{reduce} blood pressure). RQs should be formed before the data are collected.

\begin{importantBox}{iconmonstr-warning-8-240.png}
In general, RQs should be two-tailed RQs, unless a justifiable reason exists for asking a one-tailed question \emph{before data are collected}.

\end{importantBox}

\section{Units of observation and analysis}\label{UnitsObsAnalysis}

\index{Units of observation}\index{Units of analysis}

\emph{Units of observation} and \emph{units of analysis} are different yet similar concepts that must be distinguished to properly identify a population.

Consider this descriptive RQ:

\begin{quote}
In English \(20\)-something men, what is the average thickness of head-hair strands?
\end{quote}

To answer this question, the thickness of individual hair strands needs to be measured. The `things' from or about which measurements are taken are called \emph{units of observation}.

\begin{definition}[Unit of observation]
\protect\hypertarget{def:UnitOfObservation}{}\label{def:UnitOfObservation}The \emph{unit of observation} is the entity that is observed, from or about which measurements are taken and data collected.
\end{definition}

For this RQ, the unit of observation is the hair strand: the thickness measurements are taken from the hair strands. Suppose the thickness of \(100\) hair strands is recorded. These \(100\) hair strands could be obtained in many different ways. Two options are to:

\begin{itemize}
\tightlist
\item
  take \(100\) hair strands, all from the same man.
\item
  take one hair strand from each of \(100\) different men.
\end{itemize}

While each approach gives \(100\) measurements, these two approaches are \emph{very} different. Only one man is represented in the first scenario, so every hair strand is likely to be similar. However, \(100\) different men are represented in the second. The difference is related to the concept of \emph{unit of analysis}.

The purpose of the study is to make conclusions about `men': the RQ is asking about `men'. Each different man provides a separate, independent measurement of hair strand thickness. The `man' is the unit of analysis; each man provides a unique example of a hair strand.

The first scenario above has one unit of analysis (which provided all~\(100\) units of observation). The second scenario has~\(100\) units of analysis (each providing one unit of observation).

Identifying units of analysis takes care. The units of analysis:

\begin{itemize}
\tightlist
\item
  can be single units of observation, or \emph{collections} of units of observations (as in the hair-strand example).
\item
  are usually determined by the RQ: what is being compared or studied?
\item
  must be distinct, and separate to, each other (or nearly so).
\end{itemize}

\begin{definition}[Unit of analysis]
\protect\hypertarget{def:UnitOfAnalysis}{}\label{def:UnitOfAnalysis}The \emph{unit of analysis} is the smallest collection of units of observations (and perhaps the units of observations themselves) about which conclusions are made; the smallest distinct elements of the population for which information is analysed.
\end{definition}

\begin{tipBox}{iconmonstr-info-6-240.png}
Sometimes the \emph{units of analysis} and \emph{units of observation} are the same.

\end{tipBox}

In the hair-strand study, all the hair strands from the same man have essentially `lived their life together': they are all washed together with the same shampoo, exposed to the same amount of sunlight and exercise, share the same genetics, etc. However, different men potentially use different shampoo, exercise differently, have different genetics, and so on. The hair of different men tends to exhibit distinct characteristics. Each man is a collection of units of observations (hair strands). This study has a sample size of just two: \(n = 2\).

\begin{definition}[Sample size]
\protect\hypertarget{def:SampleSize}{}\label{def:SampleSize}The sample size~\(n\) is the number of units of analysis.\index{Sample size}
\end{definition}

\begin{example}[Units of analysis, observation]
\protect\hypertarget{exm:UnitsDescriptiveBread}{}\label{exm:UnitsDescriptiveBread}To compare the average amount of fibre in wholemeal and white bread, researchers take ten slices from one loaf of wholemeal bread, and ten slices from one loaf of white bread. The amount of fibre (in grams) in each slice is determined. The units of \emph{observation} are the `slices': the type of bread (explanatory variable) and the amount of fibre (response variable) are observed on individual slices.

The unit of \emph{analysis} is the `loaf' (a collection of slices), because the RQ is comparing types of \emph{bread}, and the slices for each type of bread are all from the same loaf. Slices from the same loaf share the same baker and bakery; they were made with the same ingredients, in the same oven, baked at the same temperature, etc.
\end{example}

\begin{example}[Units of analysis, observation]
\protect\hypertarget{exm:UnitsDescriptive}{}\label{exm:UnitsDescriptive}The \emph{Spectrum} website reported a study where researchers examined `\(10\)~neurons from each of the \(16\)~mice' (November 2022). The researchers treated each neuron as an independent observation, so \(n = 16\times 10 = 160\).

However, neurons in the brain of the same animal are \emph{not} independent observations. The unit of analysis is the mouse; the unit of observation is the neuron. The actual sample size was \(n = 16\); each unit of analysis has \(10\)~units of observation. A total of \(160\) neurons from~\(16\) mice is very different to a study of \(160\) neurons from \(160\) genetically-different mice.
\end{example}

The units of observation and units of analysis \emph{may} be the same, and often are the same. However, they are sometimes different, and identifying these situations is \emph{crucial}. Importantly, studies compare units of analysis, not units of observation.

\begin{example}[Units of analysis, observation]
\protect\hypertarget{exm:UnitsDescriptiveBP}{}\label{exm:UnitsDescriptiveBP}Suppose researchers record the diastolic blood pressure (DBP) from \(15\) patients aged under \(40\)~years of age, and \(15\) different patients aged \(40\)~or older. The DBP is measured on every patients' right arm, so there are \(15\) observations for the `Under~\(40\)' group, and \(15\) observations for the `\(40\)~and over' group.

Provided the patients are not closely related, the patients are independent of each other. (If all \(15\)~observations were all from the same family, for example, this would not be true.) The `patient' is the unit of analysis \emph{and} the unit of observation.

Later, the researchers decide to take measurements from the left \emph{and} right arms of every patient. Thus, there are now \(30\) observations for the `Under~\(40\)' group, and \(30\) observations for the `\(40\)~and over' group. However, the left and right arm measurements for each person are likely to be very similar. The `patient' is the unit of analysis, and each patient provides two observations (one from each arm).

In both cases, the sample size is \(n = 30\): both have \(30\) units of analysis.
\end{example}

\begin{example}[Units of analysis]
\protect\hypertarget{exm:UnitsAnalysis3}{}\label{exm:UnitsAnalysis3}A study compared two physical activity (PA) programs. Each of \(44\)~children in the study, chosen from schools across the region, was allocated to one of two PA programs (with parental agreement). The children's fitness was measured for every student at the end of the six-month study.

The \emph{units of observation} are the students: fitness measurements are taken from each student. The \emph{units of analysis} are also the students: students using the different programs are being compared. In addition, the PA program was \emph{allocated} to each student individually, and each student has their own family routines and activities, etc. and lives separate, distinct lives. Each unit of analysis (student) has one unit of observation.

The study has \(44\)~units of analysis, each with one unit of observation.
\end{example}

\begin{example}[Units of analysis]
\protect\hypertarget{exm:UnitsAnalysisGroups}{}\label{exm:UnitsAnalysisGroups}Consider comparing the percentage of females and males wearing hats at a specific beach.

People in \emph{groups} at the beach will probably not be independent: people in groups tend to behave similarly. For example, a couple will often (but not always) \emph{both} be wearing or not wearing hats; friends often behave in similar ways.

Hence, the researchers may decide to use data from individual people, and not groups (`person' is the unit of analysis \emph{and} unit of observation). Alternatively, the researchers may decide to use people \emph{groups} as the \emph{unit of analysis} (some will be groups of one), and record data from just \emph{one} person in any group (e.g., the person closest to the researchers when the group is noticed).
\end{example}

\section{Definitions}\label{OperationDefinitions}

\index{Definitions}

Research studies usually include terms that must be carefully and precisely defined, so that others know \emph{exactly} what words and terms mean, without ambiguity. Two types of definitions can be given when necessary.

\begin{definition}[Conceptual definition]
\protect\hypertarget{def:ConceptualDefinition}{}\label{def:ConceptualDefinition}\index{Definitions!conceptual} A \emph{conceptual definition} articulates precisely \emph{what} words or phrases mean in a study.
\end{definition}

\begin{definition}[Operational definition]
\protect\hypertarget{def:OperationalDefinition}{}\label{def:OperationalDefinition}\index{Definitions!operational} An \emph{operational definition} articulates exactly \emph{how} something will be identified, measured, observed or assessed.
\end{definition}

In many cases, a clear \emph{operational definition} is needed to describe how data will be collected to ensure repeatability and consistent data collection, by removing any ambiguity about how data are obtained.

\begin{example}[Operational and conceptual definitions]
\protect\hypertarget{exm:DefinitionsStress}{}\label{exm:DefinitionsStress}Consider a study examining stress in students. A \emph{conceptual definition} would describe \emph{what is meant} by `stress' (in contrast to, say, `anxiety').

An \emph{operational definition} would describe \emph{how} `stress' is \emph{measured}, since stress cannot be measured directly (like height, for example). `Stress' could be \emph{measured} using a questionnaire or measuring physical characteristics, for instance. Other ways of measuring stress are also possible, and all have advantages and disadvantages.
\end{example}

Sometimes the definitions themselves are not important; a clear definition is simply needed. To avoid confusion, commonly-accepted definitions should be used unless good reasons exist for using a different definition. When a commonly-accepted definition does not exist, the definition being used should be very clearly articulated, and the reason given if necessary.

\begin{example}[Operational and conceptual definitions]
\protect\hypertarget{exm:DefinitionsFlexibility}{}\label{exm:DefinitionsFlexibility}A research article \citep{gillet2018shoulder} entitled `Shoulder range of motion and strength in young competitive tennis players with and without history of shoulder problems' provided these necessary conceptual definitions (among others):

\begin{itemize}
\tightlist
\item
  `young': \(8\)--\(15\) years of age.
\item
  `competitive tennis players': the best players in their age category in France, and members of a French tennis centre of excellence.
\end{itemize}

An operational definition was provided for `Shoulder strength': as measured using a hand-held dynamometer.
\end{example}

\begin{exampleExtra}

Players, administrators and fans are wary of concussions and head injuries in sport. A conference on concussion in sport developed this \emph{conceptual definition} \citep{McCrory250}:

\begin{quote}
\ldots{} a complex pathophysiological process affecting the brain, induced by biomechanical forces\ldots{}
\end{quote}

However, an \emph{operational definition} is needed to explain \emph{how} to identify a player with concussion during a game. Rugby decided on this \emph{operational definition} \citep{Raftery642}:

\begin{quote}
\ldots{} a concussion applies with any of the following:

\begin{enumerate}
\def\labelenumi{\arabic{enumi}.}
\item
  The presence, pitch side, of any Criteria Set~1 signs or symptoms (table~1)\ldots{} {[}this table includes symptoms such as `convulsion', `clearly dazed', etc.{]};
\item
  An abnormal post game, same day assessment\ldots;
\item
  An abnormal~\(36\)--\(48\,\text{h}\) assessment\ldots;
\item
  The presence of clinical suspicion by the treating doctor at any time\ldots{}
\end{enumerate}
\end{quote}

\end{exampleExtra}

\begin{example}[Operational and conceptual definitions]
\protect\hypertarget{exm:DefinitionsWaterTemp}{}\label{exm:DefinitionsWaterTemp}Consider a study requiring water temperature to be measured.

An \emph{operational definition} would explain \emph{how} the temperature is measured: the thermometer type, how the thermometer was positioned, how long was it left in the water; and so on.

A \emph{conceptual} definition would describe the scientific definition of temperature, and would not be needed (as `temperature' is a well-understood term).
\end{example}

\begin{exampleExtra}
A study of snacking in Australia \citep{data:Fayet2017:Snacks} used this conceptual definition of an `eating occasion':

\begin{quote}
\ldots one or more food or beverage items consumed at the same time of day\ldots{}
\end{quote}

and a `snacking occasion' as

\begin{quote}
\ldots one or more food or beverage items consumed at the same time of day within a snacking time period\ldots{}
\end{quote}

Finally then, `snacking' was defined as:

\begin{quote}
Eating occasions that occurred during breakfast, midday and evening meals were meals and all eating occasions that occurred between these meals were classified as snacking.
\end{quote}

These are all \emph{conceptual} definitions, explaining what the terms \emph{mean}.

An \emph{operational} definition would explain \emph{how} the data were obtained from the participants (e.g., using a food diary).

\end{exampleExtra}

\begin{exampleExtra}
\citet{data:Meline2006:InclusionExclusion} discusses five studies about stuttering, each using a different \emph{operational} definition:

\begin{itemize}
\tightlist
\item
  Study~1: as diagnosed by speech-language pathologist.
\item
  Study~2: within-word disfluences greater than~\(5\) per~\(150\) words.
\item
  Study~3: unnatural hesitation, interjections, restarted or incomplete phrases, etc.
\item
  Study~4: more than three stuttered words per minute.
\item
  Study~5: state guidelines for fluency disorders.
\end{itemize}

People may be classified as stutterers by some definitions but not others, so it is important to know which definition is used.

\end{exampleExtra}

\begin{exampleExtra}

A study examined the possible relationship between the `pace of life' and the incidence of heart disease \citep{data:levine1990:paceoflife} in \(36\)~US cities.

The researchers used four different \emph{operational} definitions for `pace of life' (remember the article was published in 1990!):

\begin{enumerate}
\def\labelenumi{\arabic{enumi}.}
\tightlist
\item
  The walking speed of randomly chosen pedestrians.
\item
  The speed with which bank clerks gave `change for two \$20~bills or {[}gave{]} two \$20~bills for change'.
\item
  The talking speed of postal clerks.
\item
  The proportion of men and women wearing a wristwatch.
\end{enumerate}

None of these \emph{perfectly} measure `pace of life', of course. Nonetheless, the researchers found that, compared to people on the West Coast,

\begin{quote}
\ldots{} people in the Northeast walk faster, make change faster, talk faster and are more likely to wear a watch\ldots{}

\VA{--- \citet{data:levine1990:paceoflife} (p.~455)}{}
\end{quote}

\end{exampleExtra}

\section{Example: writing a RQ}\label{Writing-RQs}

\index{Research question!writing}

Suppose you notice some people taking echinacea (a herb) after they get a common cold. You may wonder: does taking echinacea help in any way with a cold? You may ask:

\begin{quote}
Is it better to take echinacea when you have a cold?
\end{quote}

\emph{This RQ is clearly poor, but is a starting point.} This RQ can be refined by clarifying the POCI elements.\index{POCI} For example, what \emph{population} is of interest? Many options exist: all residents of your country, or just adults in a specific part of your country. Some of these may not be practical (i.e., when a sample cannot easily be obtained that represents the population).

What \emph{outcome} could be used to determine echinacea's effectiveness? Options include the \emph{average} cold duration, or the \emph{percentage} of people who take days off work due to the cold.

The initial RQ is also vague: better than \emph{what}? The outcome could be \emph{compared} between groups (between those taking echinacea and the controls (those who do not)). A within-individuals comparison seems unsuitable for this RQ.\spacex

The study could also have \emph{intervention} or not, which has implications for how the study is conducted and how the results are interpreted. If the study \emph{did not have an intervention}, the subjects would decide for themselves how to treat their cold. If the study \emph{did have an intervention}, the use of echinacea would be imposed by the researchers.

Many terms need defining, too. What is meant by `echinacea' (fresh? tablet form? as a tea?); `cold' (self-diagnosed? diagnosed by a doctor?), and so on.

Based on the above, this RQ could be considered (based on \citet{data:barrett:echinacea}):

\begin{quote}
Among Australian teenagers with a common cold, is the average duration of cold symptoms shorter for teens given a daily dose of echinacea, compared to teens taking no echinacea?
\end{quote}

\section{Preparing software}\label{DataEntry}

\index{Computers and software!data entry}\index{Computers and software!statistical}\index{Computers and software!jamovi}

Statistical software packages are used to store data for subsequent analyses. Datasets that \emph{do not} contain any within-individuals variables are organised so that:

\begin{itemize}
\tightlist
\item
  each \emph{row} represents one unit of analysis.\index{Units of analysis}
\item
  each \emph{column} represents one between-individuals variable.\index{Variables!between-individuals}
\end{itemize}

An additional column of identifying information may also appear, such as the person's name, or concrete batch number.

\begin{softwareBox}{iconmonstr-laptop-4-240.png}
In statistical software, the variable \emph{names} are not placed in a row (say, in Row~1, above the data itself), which might happen when using a spreadsheet. The \emph{names} of the variables are the names of the columns.

\end{softwareBox}

\begin{example}[Preparing statistical software]
\protect\hypertarget{exm:SoftwarePrep}{}\label{exm:SoftwarePrep}In Sect.~\ref{Writing-RQs}, an RQ was asked about whether using echinacea reduced the duration of the common cold.

For this RQ, the two between-individuals \emph{variables} are `Duration of cold symptoms' (response variable), and `Type of treatment' (explanatory variable). The person is the unit of analysis, so the number of \emph{rows} in the data worksheet is the sample size. The data worksheet needs at least two columns (Fig.~\ref{fig:DataPrepNormaljamovi}):

\begin{itemize}
\tightlist
\item
  one for duration of each individual's cold symptoms.
\item
  one for whether the individual received a dose of echinacea or received no medication.
\end{itemize}

An additional column may record the name or ID of each individual, and more columns may record other within-individuals variables (such as age and height of the individuals).
\end{example}

\begin{figure}[hbtp]

{\centering \includegraphics[width=0.55\linewidth]{DataPrep/Echinacea/DataPrep} 

}

\caption{Software prepared for data with no within-individuals variable. Each row represents an individual; each column represents a between-individuals variable.}\label{fig:DataPrepNormaljamovi}
\end{figure}

Datasets that \emph{do} contain within-individuals variables can be organised in \emph{wide} or \emph{long} format. Some analyses are easier using wide format, and some using long format.

In \emph{wide} format:\index{Computers and software!wide format}

\begin{itemize}
\tightlist
\item
  each \emph{row} represents one unit of analysis.\index{Units of analysis}
\item
  each between-individuals variable is represented in a column.
\item
  each within-individuals variable is represented in \emph{multiple} columns, one for each measurement of that variable on the individuals.
\end{itemize}

In \emph{long} format:\index{Computers and software!long format}

\begin{itemize}
\tightlist
\item
  each unit of analysis is represented by \emph{multiple} rows.\index{Units of analysis}
\item
  each between-individuals variable is represented in a column, and the data repeated in each row corresponding to that unit of analysis.
\item
  each within-individuals variables is represented by one column.
\end{itemize}

\begin{example}[Long and wide data formats]
\protect\hypertarget{exm:SoftwarePrepWithin}{}\label{exm:SoftwarePrepWithin}Example~\ref{exm:RepeatedMeasuresPaired} discussed a study where the weights of university students were recorded in both Weeks~\(1\) and~\(12\).

In \emph{wide} format, each \emph{row} represents one individual (Fig.~\ref{fig:DataPrepWithinjamovi}, left panel). In \emph{long} format, each individual is represented by multiple rows (Fig.~\ref{fig:DataPrepWithinjamovi}, right panel).
\end{example}



\begin{figure}[hbtp]

{\centering \includegraphics[width=0.51\linewidth]{DataPrep/Students/WideFormat} \includegraphics[width=0.01\linewidth]{OtherImages/SPACER} \includegraphics[width=0.45\linewidth]{DataPrep/Students/LongFormat} 

}

\caption{Software prepared for data with a within-individuals variable; the same data is shown in both panels. Left: in \emph{wide} format, with one individual per row. Right: in \emph{long} format, with multiple rows per individual. Both include a column of identifying information.}\label{fig:DataPrepWithinjamovi}
\end{figure}

\section{Chapter summary}\label{Chap2-Summary}

In this chapter, you have learnt to write \emph{research questions} for quantitative analysis. All research questions (RQs) study a \emph{population} (P). Descriptive RQs study some \emph{outcome} (O) in the population. Relational RQs \emph{compare} the outcome between different groups of individuals (a between-individuals comparison). Repeated-measures RQs compare the \emph{same} outcome when measured on the same individuals multiple times (a within-individuals comparison). Some RQs also have an \emph{intervention} (I): when the values of the comparison can be manipulated by the researchers. Correlational RQs ask about the relationship between variables. RQs may be \emph{decision-making} RQs (one- or two-tailed) or \emph{estimation} RQs.

Data comes from a sample of \emph{individuals} in the population. The \emph{outcome} is a numerical summary of the values of the response variable from many individuals. Similarly, the data concerning the comparison comes from measuring or observing the values of the \emph{explanatory} variables from individuals.

The \emph{who} or \emph{what} that observations are made from are called the \emph{units of observation}. The smallest independent collections of units of observations (that is, independent examples of the population) are called the \emph{units of analysis}.

\section{Quick review questions}\label{Chap2-QuickReview}

Consider this RQ:

\begin{quote}
In elite female netball players, do players in defence positions have the same average number of knee injuries (per player, per season) compared to players in attacking positions?
\end{quote}

Are the following statements \emph{true} or \emph{false}?

\begin{enumerate}
\def\labelenumi{\arabic{enumi}.}
\item
  The \emph{comparison} is `between knee injuries and other types of injuries'.\tightlist 
\item
  The \emph{comparison} is this RQ is a \emph{between}-individuals comparison.
\item
  The \emph{outcome} is `the average number of knee injuries per player, per season'.
\item
  The \emph{response variable} is `the average number of knee injuries per season'.
\item
  The \emph{unit of analysis} is `the number of knee injuries'.
\item
  The \emph{unit of observation} is `the elite netball player'.
\item
  This RQ is a descriptive RQ.
\item
  This RQ is an estimation-type RQ.
\end{enumerate}

\section{Exercises}\label{RQsExercises}

\hyperref[Answers]{Answers to odd-numbered exercises} are given at the end of the book.

\captionsetup{font=small}

\begin{exercise}
\protect\hypertarget{exr:RQsOutcomeResponse1}{}\label{exr:RQsOutcomeResponse1}

For the following \emph{response} variables, what are the corresponding \emph{outcomes}?

\begin{enumerate}
\def\labelenumi{\arabic{enumi}.}
\tightlist
\item
  Whether a vehicle crashes or not.
\item
  The height people can jump.
\item
  The number of tomatoes per plant.
\end{enumerate}

\end{exercise}

\begin{exercise}
\protect\hypertarget{exr:RQsOutcomeResponse2}{}\label{exr:RQsOutcomeResponse2}

For the following \emph{response} variables, what are the corresponding \emph{outcomes}?

\begin{enumerate}
\def\labelenumi{\arabic{enumi}.}
\tightlist
\item
  Whether a person owns a car.
\item
  The time it takes for seedlings to sprout.
\item
  The amount of caffeine in cola drinks.
\end{enumerate}

\end{exercise}

\begin{exercise}
\protect\hypertarget{exr:RQsComparisonExplanatory1}{}\label{exr:RQsComparisonExplanatory1}

For the following \emph{comparisons}, what are the corresponding \emph{explanatory} variables?

\begin{enumerate}
\def\labelenumi{\arabic{enumi}.}
\tightlist
\item
  Between vegans and non-vegans.
\item
  Between caffeinated and decaffeinated coffee.
\item
  Between taking zero, one or two \(7\,\text{mg}\) iron tablets per day.
\end{enumerate}

\end{exercise}

\begin{exercise}
\protect\hypertarget{exr:RQsComparisonExplanatory2}{}\label{exr:RQsComparisonExplanatory2}

For the following \emph{comparisons}, what are the corresponding \emph{explanatory} variables?

\begin{enumerate}
\def\labelenumi{\arabic{enumi}.}
\tightlist
\item
  Between frozen vegetables and fresh vegetables.
\item
  Between \(91\)-octane, \(95\)-octane, and ethanol-blended car fuel.
\item
  Between large cities and small cities.
\end{enumerate}

\end{exercise}

\begin{exercise}
\protect\hypertarget{exr:RQsComparisonVsPaired1}{}\label{exr:RQsComparisonVsPaired1}

For the following studies, determine whether the study is likely to use a \emph{between}-individuals comparison or a \emph{within}-individuals comparison. In each case, identify the outcome.

\begin{enumerate}
\def\labelenumi{\arabic{enumi}.}
\tightlist
\item
  A study to determine if a higher percentage of people at a particular city park wear hats in summer compared to winter.
\item
  A study to determine if the average yield of a specific variety of tomato plants is the same when three different fertilisers are applied.
\end{enumerate}

\end{exercise}

\begin{exercise}
\protect\hypertarget{exr:RQsComparisonVsPaired2}{}\label{exr:RQsComparisonVsPaired2}

For the following studies, determine whether the study is likely to use a \emph{between}-individuals comparison or a \emph{within}-individuals comparison. In each case, identify the outcome.

\begin{enumerate}
\def\labelenumi{\arabic{enumi}.}
\tightlist
\item
  A study to determine if the average balance time on right legs is the same as on left legs.
\item
  A study to determine if average cholesterol levels are the same when measured on the same people before and after a diet change.
\end{enumerate}

\end{exercise}

\begin{exercise}
\protect\hypertarget{exr:RQsDogs}{}\label{exr:RQsDogs}

A study of Phu Quoc Ridgeback dogs (\emph{Canis familiaris}) explored the relationship between body length and body height \citep{quan2017relation}.

\begin{enumerate}
\def\labelenumi{\arabic{enumi}.}
\tightlist
\item
  What type of RQ would be asked about the dogs?
\item
  What are the response and explanatory variables?
\end{enumerate}

\end{exercise}

\begin{exercise}
\protect\hypertarget{exr:RQsTypingCor}{}\label{exr:RQsTypingCor}

\citet{pinet2022typing} recorded typing speed and age for \(1\,301\)~students.

\begin{enumerate}
\def\labelenumi{\arabic{enumi}.}
\tightlist
\item
  What type of RQ could be asked in this study?
\item
  What are the response and explanatory variables?
\end{enumerate}

\end{exercise}

\begin{exercise}
\protect\hypertarget{exr:RQsBloodPressure}{}\label{exr:RQsBloodPressure}

Consider this RQ:

\begin{quote}
Among Danish university students, is the average resting diastolic blood pressure the same for students who regularly drive to university and those who regularly ride bicycles to university?
\end{quote}

\begin{enumerate}
\def\labelenumi{\arabic{enumi}.}
\tightlist
\item
  For this RQ, identify the population, outcome, and comparison (if any).
\item
  For this RQ, is there an intervention? Explain.
\item
  What \emph{type} of question is used (descriptive; relational; repeated measures; correlational)?
\item
  What is the \emph{purpose} of the RQ: estimation or decision-making?
\item
  What \emph{operational} and \emph{conceptual definitions} would be needed?
\item
  What information \emph{must} be collected from each individual to answer the RQ (i.e., the variables)?
\item
  Identify the units of analysis and the units of observation.
\end{enumerate}

\end{exercise}

\begin{exercise}
\protect\hypertarget{exr:RQsNutrition}{}\label{exr:RQsNutrition}

\citet{data:checkley:diarrhea} (p.~210) conducted:

\begin{quote}
a \(4\)-year (1995--1998) field study in a Peruvian peri-urban community\ldots{} to examine the relation between diarrhea and nutritional status in \(230\)~children \(< 3\)~years of age
\end{quote}

For this study:

\begin{enumerate}
\def\labelenumi{\arabic{enumi}.}
\tightlist
\item
  identify P, O, C and I (where relevant).
\item
  infer the primary research question.
\item
  what \emph{type} of question is used (descriptive; relational; repeated measures; correlational)?
\item
  what is the \emph{purpose} of the RQ: estimation or decision-making?
\item
  what \emph{operational definitions} would be needed?
\item
  what are the \emph{response} and \emph{explanatory} variables?
\item
  what are the units of observation and units of analysis?
\end{enumerate}

\end{exercise}

\begin{exercise}
\protect\hypertarget{exr:RQsWalkingSpeed}{}\label{exr:RQsWalkingSpeed}

Consider this RQ: `Is the average walking speed the same when texting and talking on a mobile phone?'

\begin{enumerate}
\def\labelenumi{\arabic{enumi}.}
\tightlist
\item
  What \emph{type} of question is used (descriptive; relational; repeated measures; correlational)?
\item
  Is this RQ one- or two-tailed?
\item
  Is there an intervention? Explain.
\item
  What is the \emph{explanatory} variable?
\item
  What is the \emph{response} variable?
\item
  What is the \emph{outcome}?
\item
  What are the units of observation and units of analysis?
\end{enumerate}

\end{exercise}

\begin{exercise}
\protect\hypertarget{exr:RQsCommonCold}{}\label{exr:RQsCommonCold}

Consider this RQ, with an intervention:

\begin{quote}
For Japanese adults with a common cold, do people who take vitamin~C tablets daily have, on average, a shorter cold duration than people who do not take any vitamin~C tablets?
\end{quote}

\begin{enumerate}
\def\labelenumi{\arabic{enumi}.}
\tightlist
\item
  Identify the population, comparison and outcome.
\item
  What is the response variable?
\item
  What is the explanatory variable?
\item
  What type of RQ is this: estimation or decision-making?
\item
  Is the RQ one-tailed or two-tailed?
\end{enumerate}

\end{exercise}

\begin{exercise}
\protect\hypertarget{exr:RQsAnimals}{}\label{exr:RQsAnimals}

Animals in an experiment are divided into pens (three animals per pen), and feed is allocated to each pen \citep{sterndale2017increasing}. Animals in different pens receive different feed; animals in the same pen receive the same feed. The weight gain of each animal is recorded.

\begin{enumerate}
\def\labelenumi{\arabic{enumi}.}
\tightlist
\item
  What is the \emph{unit of observation}? Why?
\item
  What is the \emph{unit of analysis}? Why?
\item
  Identify the between-individuals comparison.
\end{enumerate}

\end{exercise}

\begin{exercise}
\protect\hypertarget{exr:RQsBlueGum}{}\label{exr:RQsBlueGum}A research study was comparing the average length of Blue Gum eucalypt leaves in two areas of Queensland. A student takes \(40\)~leaves from each of ten trees in Area~A, and~\(40\)~leaves from each of ten trees in Area~B.

Are the following statements \emph{true} or \emph{false}?

\begin{enumerate}
\def\labelenumi{\arabic{enumi}.}
\item
  The unit of analysis is the individual leaf. \tightlist
\item
  The unit of observation is the individual leaf.
\item
  The unit of analysis is the tree.
\end{enumerate}

What is the size of the sample in the study?
\end{exercise}

\begin{exercise}
\protect\hypertarget{exr:ProjectRQ2}{}\label{exr:ProjectRQ2}Consider this actual student RQ from the university where I work.

\begin{quote}
Among \(10\) Australian adults, does the time taken to read a passage of text change when different fonts are used?
\end{quote}

Critique the RQ, and write a better RQ (if necessary).
\end{exercise}

\begin{exercise}
\protect\hypertarget{exr:ProjectRQ3}{}\label{exr:ProjectRQ3}Consider this actual student RQ from the university where I work.

\begin{quote}
Of students that study at (a University), do males have a larger lung capacity than females?
\end{quote}

Critique the RQ, and write a better RQ (if necessary).
\end{exercise}

\begin{exercise}
\protect\hypertarget{exr:RQsNoseHair}{}\label{exr:RQsNoseHair}

\citet{gs2023pullout} examined the strength needed to pull out nose-hairs. Fifty nose-hairs were pulled from one author's nose, and \(50\)~nose hairs pulled from the other author's nose, and the average pull-out strengths for each man compared.

\begin{enumerate}
\def\labelenumi{\arabic{enumi}.}
\tightlist
\item
  What are the units of analysis and units of observation?
\item
  What is the sample size in this study?
\end{enumerate}

\end{exercise}

\begin{exercise}
\protect\hypertarget{exr:RQsenvironments}{}\label{exr:RQsenvironments}

\citet{huang2020trees} placed different people into one of three different virtual-reality (VR) environments: trees, grass or concrete. Stress levels were measured using `skin conductance level' (SCL) for each individual, before and after exposure to the VR environment.

\begin{enumerate}
\def\labelenumi{\arabic{enumi}.}
\tightlist
\item
  Identify the \emph{between}-individuals comparisons.
\item
  Identify the \emph{within}-individuals comparisons.
\item
  Is their definition for SCL (p.~2) \emph{conceptual} or \emph{operational}?
\end{enumerate}

\begin{quote}
SCLs are an unbiased measure of sympathetic activity via the electric impulses on the skin's surface and sweat glands, which are innervated only by the sympathetic nervous system\ldots{}
\end{quote}

\end{exercise}

\begin{exercise}
\protect\hypertarget{exr:POCIaccelerometer}{}\label{exr:POCIaccelerometer}

Consider this two-tailed RQ (based on \citet{tudor2015comparison}):

\begin{quote}
For American adults, is the average number of recorded steps per day the same when recorded using both a waist accelerometer, and a wrist accelerometer?
\end{quote}

\begin{enumerate}
\def\labelenumi{\arabic{enumi}.}
\tightlist
\item
  Identify the population and the individuals.
\item
  Identify the outcome.
\item
  Identify the response and explanatory variables.
\item
  Determine if the comparison is \emph{between}- or \emph{within}-individuals.
\end{enumerate}

\end{exercise}

\begin{exercise}
\protect\hypertarget{exr:RQsTypesZinc}{}\label{exr:RQsTypesZinc}Studies can incorporate many types of RQs. For example, \citet{data:Thane2004:ZincVitA} studied `British young people aged~\(4\)--\(18\)' and answered numerous RQs, including:

\begin{itemize}
\tightlist
\item
  what is the average zinc intake of the children?
\item
  does the average zinc intake meet recommended dietary guidelines?
\item
  what is the strength of the association between plasma zinc and retinol concentrations?
\item
  is the average zinc intake the same for boys and girls?
\end{itemize}

For each RQ, classify these RQs as descriptive, relational, repeated-measures, or correlational RQs. Then, classify them as estimation or decision-making RQs. Does the study have an invention?
\end{exercise}

\begin{exercise}
\protect\hypertarget{exr:RQsComparisonConnectionCaloric}{}\label{exr:RQsComparisonConnectionCaloric}\citet{CaloricIntake} studied the relationship between daily sodium excretion and whether people had been diagnosed with diabetes or not, in Israeli adults. The study also explored the strength of the relationship between the daily sodium excretion and the systolic blood pressure.

Classify the two RQs as descriptive, relational, repeated-measures, or correlational RQs. Then, classify them as estimation or decision-making RQs. Does the study have an invention?
\end{exercise}

\begin{exercise}
\protect\hypertarget{exr:InterventionalRQWithin}{}\label{exr:InterventionalRQWithin}\citet{ghasemi2019effectiveness} studied the incidence of musculoskeletal disorders in Iranian bus drivers. They introduced a program that aimed to provide relief for the drivers. Each bus driver was evaluated both before and after the intervention.

Classify the RQ as descriptive, relational, repeated-measures, or correlational RQs. Then, classify the RQ as estimation or decision-making RQs. Does the study have an invention?
\end{exercise}

\begin{exercise}
\protect\hypertarget{exr:WhatAreUnitsAnalysisA}{}\label{exr:WhatAreUnitsAnalysisA}To determine the average length of the legs of emus, \(27\)~emus from various zoos were studied. For each emu, the length of the left and right leg were recorded, resulting in~\(54\) measurements.

What is the sample size for this study? Explain.
\end{exercise}

\begin{exercise}
\protect\hypertarget{exr:WhatAreUnitsAnalysisB}{}\label{exr:WhatAreUnitsAnalysisB}A study compared the percentage of females and males that wear closed-in shoes to the supermarket. For each person they observed, the type of shoe on each person's left and right foot (as either closed-in; not closed-in) was recorded. This approach resulted in \(310\)~observations.

What is the sample size for this study? Explain.
\end{exercise}

\begin{exercise}
\protect\hypertarget{exr:UnitsAnalysis}{}\label{exr:UnitsAnalysis}A study compares the wear on two brands of car tyres. Four tyres of Brand~A are allocated to each of Cars~1--5, and four tyres of Brand~B are allocated to each of Cars~6--10. After \(12\)~months, the amount of wear is recorded on each tyre, and the two brands compared.

What are the units of analysis, the units of observation and the sample size?
\end{exercise}

\begin{exercise}
\protect\hypertarget{exr:UnitsAnalysis2}{}\label{exr:UnitsAnalysis2}\citet{parsons2018unit} discuss a scenario where six subjects with colorectal cancer underwent therapy. Another six similar subjects did not receive the therapy. The size of all the subjects' removed lymph nodes were then measured. Each subject's specimen (p.~6):

\begin{quote}
was divided into two sub-samples after collection {[}\ldots{]} processed and analysed at two occasions, by different members of the laboratory team {[}\ldots{]} Three slices per sub-sample were collected for each subject.
\end{quote}

How many units of analysis and the units of observation are present?
\end{exercise}

\begin{exercise}
\protect\hypertarget{exr:UnitsAnalysisBamboo}{}\label{exr:UnitsAnalysisBamboo}

Bamboo is a fast-growing, strong grass often used for green building practices. A small research study explored the hardness of bamboo when used as flooring material.

The \emph{Janka hardness}\footnote{The force required to embed an \(11.28\,\text{mm}\) steel ball into wood to half the diameter of the ball.} of bamboo flooring provided by Bamboo Flooring Australia Pty Ltd was measured by the Queensland Department of Primary Industries \citep{data:Gerber:BambooFlooring}. Five floorboards were taken, and two hardness measurements were taken on \emph{each} board (units not given, but probably kilonewtons; Table \ref{tab:JankaBoards}).

\begin{enumerate}
\def\labelenumi{\arabic{enumi}.}
\tightlist
\item
  What is the unit of analysis: the test, the board, each measurement, kilonewtons, or something else? Explain your answer.
\item
  How many units of analysis are there?
\item
  How many units of observation are there?
\item
  Comment on the amount of variation \emph{between} the boards compared to the amount of variation \emph{within} boards.
\item
  Suppose the measurements were taken from \(10\) \emph{different} places on the \emph{same} board (rather than from five different boards). How many units of analysis are there now? Explain your answer.
\end{enumerate}

\end{exercise}

\begin{table}
\centering
\caption{\label{tab:JankaBoards}Two Janka hardness measurements from five different bamboo boards.}
\centering
\fontsize{8}{10}\selectfont
\begin{tabular}[t]{ccccc}
\toprule
\textbf{Board 1} & \textbf{Board 2} & \textbf{Board 3} & \textbf{Board 4} & \textbf{Board 5}\\
\midrule
$10.5$ & $\phantom{0}8.0$ & $11.5$ & $10.3$ & $10.2$\\
$\phantom{0}7.5$ & $\phantom{0}8.0$ & $11.2$ & $\phantom{0}9.9$ & $\phantom{0}9.3$\\
\bottomrule
\end{tabular}
\end{table}

\begin{exercise}
\protect\hypertarget{exr:PoorRQs}{}\label{exr:PoorRQs}Critique the following research questions, outlining how and why they can be improved (if at all).

\begin{enumerate}
\def\labelenumi{\arabic{enumi}.}
\tightlist
\item
  Among domestic water tanks used in south-east Queensland, are lead concentrations in water in concrete tanks higher than in poly tanks?
\item
  Are lower-limb amputees more likely to die?
\item
  Is the amount of salt the same for home brand as for non-home brand beans?
\item
  Among zoo animals, is the weight of adult elephants greater than juvenile kangaroos (joeys)?
\item
  Is the average reaction time related to gender?
\end{enumerate}

What terms might need defining for each RQ?
\end{exercise}

\captionsetup{font=normalsize}

\begin{EOCanswerBox}{iconmonstr-check-mark-14-240.png}
\textbf{Answers to \emph{Quick review} questions:} \textbf{1.} False. \textbf{2.} True. \textbf{3.} True. \textbf{4.} False. \textbf{5.} False. \textbf{6.} True. \textbf{7.} False. \textbf{8.} False.

\end{EOCanswerBox}

\part{Research design}\label{part-research-design}

\chapter{Overview of research design}\label{ResearchDesignOverview}

\index{Internal validity} \index{Research design|(}

\begin{cols}
\begin{col}{0.52\textwidth}

\begin{objectivesBox}{iconmonstr-target-4-240.png}
So far, you have learnt to ask an RQ.
\textbf{In this chapter}, you will learn why research design is important, by learning to:

\begin{itemize}\tightlist
  \item
  identify reasons why the value of the response variable varies.
  \item
  identify and distinguish extraneous, confounding and lurking variables.
  \item
  understand how chance impacts the values of the response variable.
  \item
  explain external and internal validity.
\end{itemize}
\end{objectivesBox}

\end{col}

\begin{col}{0.03\textwidth}
~
\end{col}

\begin{col}{0.45\textwidth}

\includegraphics[width=0.95\linewidth]{03-ResearchDesign-Overview_files/figure-latex/unnamed-chunk-4-1} 
\end{col}
\end{cols}

\section{Introduction: internal and external validity}\label{IntroInternalValidity}

\index{Internal validity}

An RQ asks about a \emph{population}. However, studying every member of a population is generally impossible due to cost, time, ethics, logistics and/or practicality. A subset of the population (a \emph{sample}) is studied, comprising some \emph{individuals} from the population. \emph{Countless} different samples are possible.\index{Sample}

\begin{importantBox}{iconmonstr-warning-8-240.png}
One challenge of research is learning about a population from studying just one of the countless possible samples.

\end{importantBox}

Being able to generalise about the population of interest from studying a sample is called \emph{external validity}.\index{External validity} Chapter~\ref{Sampling} discusses how to select a suitable sample to study to enhance external validity.

\begin{definition}[External validity]
\protect\hypertarget{def:ExternalValidity}{}\label{def:ExternalValidity}\emph{External validity} refers to the ability to generalise the results to the rest of the population, beyond just those in the sample studied.
\end{definition}

Apart from being externally valid, well-designed research studies should be \emph{internally valid}. An internally valid study allows the researchers to focus on the relationship of interest between the response and explanatory variables, by eliminating, or accounting for, other sources of variation in the values of the response variable. These other sources are discussed in the rest of this chapter.

\begin{definition}[Internal validity]
\protect\hypertarget{def:InternalValidity}{}\label{def:InternalValidity}\emph{Internal validity} refers to the extent to which a cause-and-effect relationship can be established in a study.

A study with \emph{high} internal validity shows that the changes in the response variable can be (at least partially) attributed to changes in the explanatory variables; other explanations have been ruled out.
\end{definition}

\begin{importantBox}{iconmonstr-warning-8-240.png}
One challenge of research is learning about the relationship between the response and explanatory variables, when the value of the response variable can also be influenced by other factors.

\end{importantBox}

Studies with \emph{low} internal validity leave open other possibilities, apart from changes in the value of the explanatory variable, to explain changes in the value of the response variable. Ideally, all studies should be designed to be \emph{internally valid} as far as possible. Internal validity is studied in more detail in Chap.~\ref{DesignInternal}. Different research studies (Chap.~\ref{ResearchDesign}) differ in the extent to which they can achieve internal validity.

\section{Variation in the values of the response variable}\label{VariationInY}

\index{Internal validity!sources of variation}

In any study, the values of the response variable vary from individual to individual. Many reasons explain \emph{why} these values vary.

\begin{example}[Study design]
\protect\hypertarget{exm:Typing}{}\label{exm:Typing}

Consider this RQ:

\begin{quote}
For students in a large university course, is the average typing speed (in words per minute) the same for those aged under~\(25\) (`younger') and \(25\)~or over (`older')?
\end{quote}

\end{example}

The typing speed (the \emph{response variable}) of the many individuals will vary: every student in the study recording exactly the same typing speed is highly unlikely. The variation in the values of students' typing speeds (Fig.~\ref{fig:Influences}) may be due to:

\begin{itemize}
\tightlist
\item
  \emph{the explanatory variable} (Sect.~\ref{FactorYExplanatory}). The values of the explanatory variable may influence the values of the response variable. Of course, they may not either; the purpose of the study is to find if, or to what extent, this is true. In this example, the \emph{explanatory variable} is the age group of the student, which may impact typing speed.
\item
  \emph{other variables}, called \emph{extraneous variables} (Sect.~\ref{ExtraneousVariables}). Other variables (apart from the explanatory variable) may influence the response variable (perhaps more than the explanatory variable), such as `sex of the person', or `whether the person wears glasses'. The impact of these variables can be accommodated if the study is designed appropriately.
\item
  \emph{chance} (or \emph{randomness}, or \emph{natural variation})\index{Chance}\index{Natural variation} (Sect.~\ref{Chance}). The same person doing the same thing repeatedly under the same conditions will not record exactly the same typing speed every attempt. This is unavoidable, but good research design can minimise the size of this variation.
\end{itemize}

Designing studies to maximise internal validity requires identifying important extraneous variables, and eliminating (as far as possible) anything that obscures the relationship between the response and explanatory variables.

\begin{example}[Design]
\protect\hypertarget{exm:TypingPoor}{}\label{exm:TypingPoor}In the typing-speed study, suppose younger students were \emph{always} asked to use their dominant hand, and older students \emph{always} asked to use their non-dominant hand. Younger students would probably have a faster average typing speed, simply because they use their dominant hands (not due to age). This research design would produce a study with poor internal validity.
\end{example}

\begin{figure}[hbtp]

{\centering \includegraphics[width=0.9\linewidth]{03-ResearchDesign-Overview_files/figure-latex/Influences-1} 

}

\caption{Other factors can influence the values of the response variable, besides the explanatory variable.}\label{fig:Influences}
\end{figure}

\begin{definition}[Research design]
\protect\hypertarget{def:StudyDesign}{}\label{def:StudyDesign}\emph{Research design} refers to the decisions made by the researchers to maximise \emph{external validity} and \emph{internal validity}.
\end{definition}

Internal validity is one of the most important properties of scientific studies, and is relevant for reasoning about evidence more generally. Designing studies to maximise internal validity is the focus of Chap.~\ref{DesignInternal}.

\begin{tipBox}{iconmonstr-info-6-240.png}
Data collection is often tedious, time-consuming and expensive: you usually get one chance to collect data. In contrast, data (once collected) can be analysed as many times as necessary. Design the study properly the first time!

\end{tipBox}

\section{Variation due to changes in the explanatory variable}\label{FactorYExplanatory}

\index{Explanatory variable}

Changes in the values of the explanatory variable may, or may not, be associated with changes in the values of the response variable. If nothing else influenced the values of the response variable, life would be easy: any change of a given size in the value of the explanatory variable would \emph{always} result in a change of the same size in the value of the response variable.

\begin{example}[Explanatory variable]
\protect\hypertarget{exm:TypingExplanatory}{}\label{exm:TypingExplanatory}In the typing-speed study (Example~\ref{exm:Typing}), the explanatory variable is the age group. If nothing else influenced typing speed, every younger student would record the same typing speed every time, and every older student would record the same typing speed every time. This is clearly unreasonable.
\end{example}

\clearpage

\section{Variation due to changes in the extraneous variables}\label{ExtraneousVariables}

\index{Variables!extraneous}

Other variables (besides the explanatory variable) almost certainly exist which are associated with changes in the value of the response variable. These are called \emph{extraneous variables}.

\begin{definition}[Extraneous variable]
\protect\hypertarget{def:ExtraneousVariable}{}\label{def:ExtraneousVariable}An \emph{extraneous variable} is any variable associated with the response variable, but is not the explanatory variable.
\end{definition}

\begin{example}[Extraneous variables]
\protect\hypertarget{exm:Typing2Extraneous}{}\label{exm:Typing2Extraneous}In the typing-speed study (Example~\ref{exm:Typing}), potential extraneous variables may include age, the presence or absence of certain medical conditions, the level of familiarity with computers, whether the person wears glasses, etc.
\end{example}

The impact of some extraneous variables on the response variable can be reduced by fixing the values of the variable. These variables are called \emph{control variables}.\index{Variables!control}

\begin{definition}[Control variables]
\protect\hypertarget{def:ControlVariables}{}\label{def:ControlVariables}\emph{Control (or controlled) variables} are extraneous variables whose values are fixed for the study.
\end{definition}

A \emph{control variable} is different from a \emph{control group} (Def.~\ref{def:Control}).

\begin{example}[Control variables]
\protect\hypertarget{exm:TypingControl}{}\label{exm:TypingControl}In the typing-speed study (Example~\ref{exm:Typing}), typing speeds would vary greatly if students used different types of keyboards; for example, if some students used mechanical keyboards, and some used on-screen keyboards (e.g., on a tablet). The impact of age is easier to detect if all students use the \emph{same} keyboards, as this would reduce the variation in the typing speeds.

`Type of keyboard' is a \emph{control variable}.
\end{example}

If \emph{many} other variables are fixed in value (i.e., are control variables), the relationship between the explanatory and response variables is far easier to detect and measure. However, using too many control variables may limit the population, and hence the generalisability of the results. In the typing-speed study, for example, restricting the study to left-handed males who do not wear glasses would restrict the results to a very narrow group of people.

All extraneous variables are, by definition, related to the response variable. They may or may not also be associated with the explanatory variable. Extraneous variable \emph{also} related to the explanatory variable are called \emph{confounding variables} (or \emph{confounders}); see Fig.~\ref{fig:InfluencesConfounding} (left panel). A confounding variable can obscure the true relationship between the response and explanatory variables.

\begin{definition}[Confounding variable]
\protect\hypertarget{def:ConfoundingVariable}{}\label{def:ConfoundingVariable}\index{Variables!confounding}\index{Confounding} A \emph{confounding variable} (or a \emph{confounder}) is an extraneous variable associated with the response \emph{and} explanatory variables.
\end{definition}

\begin{definition}[Confounding]
\protect\hypertarget{def:Confounding}{}\label{def:Confounding}\emph{Confounding} is when a third variable influences the observed relationship between the response and explanatory variable.
\end{definition}

\begin{importantBox}{iconmonstr-warning-8-240.png}
Confounding variables are \emph{associated} with both the response and explanatory variables. This does not necessarily mean the value of the confounding variable \emph{causes} changes in the values of the response or explanatory variables.

\end{importantBox}

\begin{example}[Confounding variables and associations]
\protect\hypertarget{exm:ConfoundingAssociations}{}\label{exm:ConfoundingAssociations}Consider a study comparing the proportion of females and males wearing sunglasses while walking in a local park. To determine if the variable `whether it is raining' is an \emph{extraneous} variable, we ask:

\begin{enumerate}
\def\labelenumi{\arabic{enumi}.}
\tightlist
\item
  Is the wearing of sunglasses (the response variable) more or less likely if it is raining?
\end{enumerate}

The absence of rain may influence people to be more likely to wear sunglasses. Hence, `whether it is raining' is very likely an extraneous variable.

To determine if it is a \emph{confounding} variable, we also ask:

\begin{enumerate}
\def\labelenumi{\arabic{enumi}.}
\setcounter{enumi}{1}
\tightlist
\item
  Is one sex (the explanatory variable) more likely to be walking in the park depending on whether it is raining?
\end{enumerate}

We do \emph{not} ask if the presence of rain \emph{changes} the sex of the person; we ask if the presence of rain is \emph{associated} with different proportions of males and females walking in the presence of rain. It \emph{may} be the case (for example) that males are more likely to walk in the rain than females, so `whether it is raining' \emph{may} be an extraneous variable (but it is not obvious).
\end{example}

A relationship between the response and explanatory variables may be apparent, but only because \emph{both} of these variables are associated with the confounding variable (Fig.~\ref{fig:InfluencesConfounding}). No relationship actually exists between the response and explanatory variables.

\begin{example}[Confounding variables]
\protect\hypertarget{exm:ConfoundingCigaretteLighters}{}\label{exm:ConfoundingCigaretteLighters}People who carry cigarette lighters are more likely to get lung cancer. The reason this relationship exists, however, is because of a \emph{confounding variable}. `Whether the person is a smoker' is probably associated with \emph{both} the response and explanatory variables:

\begin{itemize}
\tightlist
\item
  smokers are more likely to carry a cigarette lighter (the explanatory variable) than non-smokers.
\item
  smokers are more likely to develop lung cancer (the response variable) than non-smokers.
\end{itemize}

No relationship actually exists between carrying a cigarette lighter are getting lung cancer.
\end{example}

\begin{exampleExtra}

Consider this RQ:

\begin{quote}
Among university students, is the percentage of females who know their own blood pressure the same as the percentage of males who know their own blood pressure?
\end{quote}

For this RQ, the \emph{explanatory variable} is the sex of person, and the \emph{response variable} is whether a student knows their own blood pressure. A potential confounding variable is `The program of study', since this is (potentially) related to \emph{both} the response and explanatory variables:

\begin{itemize}
\tightlist
\item
  `Program of study' is related to sex (the explanatory variable): a higher percentage of females study nursing, while a greater percentage of males study engineering (at least, in Australia).
\item
  `Program of study' is related to knowing your blood pressure (the response variable): nursing students probably practice taking each other's blood pressures so probably know their own, whereas engineering students do not.
\end{itemize}

\end{exampleExtra}

Managing confounding is \emph{very} important, as ignoring confounding can completely change the observed relationship between the response and explanatory variables (see Sect.~\ref{KidneyExample}) and hence can compromise internal validity. Managing confounding is discussed in Sect.~\ref{ManagingConfounding}.

\begin{figure}[hbtp]

{\centering \includegraphics[width=1\linewidth]{03-ResearchDesign-Overview_files/figure-latex/InfluencesConfounding-1} 

}

\caption{Confounding variables are extraneous variables associated with the response and explanatory variables. Left: If the confounding variable is not measured (and so a lurking variable is present), an apparent association does exist between the response and explanatory variables. Usually, confounding is not as extreme as shown in this diagram, and the confounding variable may slightly change the relationship between response and explanatory variables. Right: In extreme confounding situations, as shown here, no real association between exists between the response and explanatory variables; the association is explained by a confounding variable. }\label{fig:InfluencesConfounding}
\end{figure}

If the values of potential confounding variables are recorded, their impact can be managed. However, sometimes the values of the confounding variables are not recorded (perhaps due to poor design); then, they are called \emph{lurking variables} (Fig.~\ref{fig:InfluencesConfounding}, left panel). Lurking variables can lead to wrong conclusions.

\begin{definition}[Lurking variable]
\protect\hypertarget{def:LurkingVariable}{}\label{def:LurkingVariable}\index{Variables!lurking} A \emph{lurking variable} is an extraneous variable associated with the response \emph{and} explanatory variables (that is, is a \emph{confounding} variable), but whose values \emph{are not} recorded in the study data.
\end{definition}

\begin{example}[Lurking variables]
\protect\hypertarget{exm:LurkingMoulds}{}\label{exm:LurkingMoulds}\citeauthor{joiner1981lurking} \citetext{\citeyear{joiner1981lurking}; \citealp{wilson1952eb}} wanted to determine if the time in the production mould influenced the strength of plastic parts (p.~55--56):

\begin{quote}
Hot plastic was introduced in the mold, pressed for \(10\,\text{s}\), and removed. Another batch was then introduced into the same mold, pressed for \(20\,\text{s}\), and so on, the time increasing with each batch.
\end{quote}

Greater time in the mould (explanatory variable) was found to be associated with greater plastic strength (response variable). However, mould temperature was later found to be a \emph{lurking variable}, since it was associated with \emph{both} the response and explanatory variables:

\begin{itemize}
\tightlist
\item
  higher mould temperatures (the lurking variable) were associated with greater strength (the response variable).
\item
  higher mould temperatures (the lurking variable) were experienced by later batches with longer mould times (the explanatory variable), since the mould was hotter for the later batches.
\end{itemize}

The cause of the greater strength was \emph{not} the time in the mould; it was the higher temperature experienced by the later moulds (Fig.~\ref{fig:LurkingExampleMould}).
\end{example}

\begin{figure}[hbtp]

{\centering \includegraphics[width=1\linewidth]{03-ResearchDesign-Overview_files/figure-latex/LurkingExampleMould-1} 

}

\caption{An example of a lurking variable. Left: the relationship as originally understood. Right: the relationship after the lurking variable was eventually exposed.}\label{fig:LurkingExampleMould}
\end{figure}

To clarify the language (Table~\ref{tab:ClassifyExtraneous}):\index{Variables!lurking}\index{Variables!confounding}\index{Variables!extraneous}

\begin{itemize}
\tightlist
\item
  extraneous variables are, by definition, always associated with the response variable. If they are not recorded, and so the researchers are unaware of them, they become part of unexplained chance.
\item
  extraneous variables are called \emph{confounding variables} if they are also related to the explanatory variable.
\item
  confounding variables are called \emph{lurking variables}, if they are not recorded.
\end{itemize}

These terms are not always used consistently by all researchers, but the \emph{ideas} are important nonetheless.

\begin{table}
\centering
\caption{\label{tab:ClassifyExtraneous}The relationships between extraneous, confounding and lurking variables. Entries in \emph{italics} indicate different types of extraneous variables.}
\centering
\fontsize{8}{10}\selectfont
\begin{tabular}[t]{>{}lcc}
\toprule
\multicolumn{1}{c}{\textbf{}} & \multicolumn{1}{c}{\textbf{Related to}} & \multicolumn{1}{c}{\textbf{Related to}} \\
\textbf{ } & \textbf{response only} & \textbf{response and explanatory}\\
\midrule
\textbf{Measured/observed} & \em{Extraneous} & \em{Extraneous (confounding)}\\
\textbf{Not measured/observed} & Chance & \em{Extraneous (lurking)}\\
\bottomrule
\end{tabular}
\end{table}

To avoid lurking variables, researchers generally collect lots of information that may be relevant about the \emph{individuals in the study} (such as age and sex if the study involves people) and \emph{circumstances of the individuals in the study} (such as the temperature at the time of data collection), in case they are confounding variables.

\begin{example}[Low internal validity]
\protect\hypertarget{exm:LowInternal}{}\label{exm:LowInternal}\citet{larson2021can} reviewed numerous studies that used double-fortified salt to manage iodine and iron deficiencies. They concluded that the internal validity of studies was `generally weak' (p.~265) due, in part, to `unaccounted confounders' (i.e., lurking variables).
\end{example}

\section{Variation due to natural variation (chance)}\label{Chance}

\index{Chance}\index{Natural variation}

\emph{Chance variation} (or natural variation) refers to variation that cannot otherwise be explained: even repeating a study exactly the same way every time on the same individuals will not always produce the same values of the response variable.

The influence of the explanatory variable is hard to detect if the amount of chance variation contributing to the response variable overwhelms the amount of variation produced by changes in the value of the explanatory variable. \emph{Minimising the amount of the chance variation} requires using good design principles, and measuring as many other extraneous variables that may explain variation in the response variable as reasonable.

Chance can impact the values of the response variable in different ways:

\begin{itemize}
\tightlist
\item
  each \emph{individual} can produce different values of the response variable each time the response variable is measured (\emph{within}-individuals \emph{variation}).\index{Within-individual variation}
\item
  each individual in the study can produce different values of the response variable compared to \emph{other} individuals (\emph{between}-individuals \emph{variation}).\index{Between-individual variation}
\end{itemize}

Different strategies are needed to understand each of these sources of variation:

\begin{itemize}
\tightlist
\item
  to estimate the amount of variation \emph{within} individuals, multiple observations are needed from each unit of analysis (each individual).\index{Within-individual variation}
\item
  to estimate the amount of variation \emph{between} individuals, multiple units of analysis (individuals) are needed.\index{Between-individual variation}
\end{itemize}

\begin{example}[Three ways to sample]
\protect\hypertarget{exm:TypingSpeedThree}{}\label{exm:TypingSpeedThree}

Consider the typing-speed study (Example~\ref{exm:Typing}) again, and these three sampling approaches:

\begin{itemize}
\tightlist
\item
  taking \(30\)~observations from one younger student would tell us about variation in that student's typing speed, but very little about variation in younger students' typing speeds more generally.
\item
  taking one observation from \(30\)~different younger students would tell us about variation in younger students' typing speeds in general. We only have one measurement from each student, but since we might expect that the same person to produce similar (not identical) typing speeds, this should not be a big problem.
\item
  taking three observations from each of~\(10\) different younger students would tell us about variation in younger students' typing speeds in general, and a little about the variation in each students' typing speeds too.
\end{itemize}

\end{example}

\section{Chapter summary}\label{Chap6-Summary}

Research questions are about populations, but samples are studied in practice. Studies that use a sample that represents the population of interest are called \emph{externally valid}.

In a research study, the main interest is usually the relationship between a \emph{response variable} and \emph{explanatory variable}. Well-designed studies that allow the researchers to focus on this relationship have good \emph{internal validity}. Such studies eliminate, or account for, other explanations for the variation in the values of the response variable.

The values of the response variable can be influenced by more than just the explanatory variable, such as \emph{extraneous variables} (other variables not of primary interest), and \emph{chance}.

Some extraneous variables are also related to the explanatory variable, and are called \emph{confounding variables} (and are \emph{lurking variables} if not recorded). If the research design makes it difficult to separate the relationship between the response and explanatory variable from other possible causes, the study has poor \emph{internal validity}.

\section{Quick review questions}\label{Chap6-QuickReview}

\citet{martnes2023physical} compared the average time to complete a journey when (p.~1)

\begin{quote}
\ldots{} riding an electric-assisted bicycle with cargo (\(30\,\text{kg}\)) and without cargo\ldots{}
\end{quote}

They recorded the age, height, weight, and resting metabolic rate of all subjects who completed the \(4.5\,\text{km}\) ride. Each subject was allocated to ride both with \emph{and} without cargo.

Are the following statements \emph{true} or \emph{false}?

\begin{enumerate}
\def\labelenumi{\arabic{enumi}.}
\item
  The explanatory variable is `the age of the subjects'. \tightlist
\item
  `The height of the subjects' is a lurking variable.
\item
  The explanatory variable is `whether the bicycle is ridden with or without cargo'.
\item
  `Weight' is an extraneous variable.
\item
  The response variable is `the time to complete the journey'.
\item
  `Age' is a possible confounding variable.
\item
  `Resting metabolic rate' is a possible confounding variable.
\end{enumerate}

\section{Exercises}\label{ResearchDesignOverviewExercises}

\hyperref[Answers]{Answers to odd-numbered exercises} are given at the end of the book.

\captionsetup{font=small}

\begin{exercise}
\protect\hypertarget{exr:MineArsenic}{}\label{exr:MineArsenic}

The \emph{Giant Mine} in Yellowknife, Canada, ceased operation in 1999 after \(50\)~years, during which \(237\,000\) tonnes of arsenic trioxide was released. \citet{houben2016factors} examined the arsenic concentration in \(25\)~lakes within a \(25\,\text{km}\) radius of the mine \(11\)~years after the mine closed, to determine if the arsenic concentration was related to the distance of the lake from the mine. They also recorded:

\begin{cols}

\begin{col}{0.46\textwidth}

\begin{itemize}
\tightlist
\item
  the type of bedrock (volcanic; sedimentary; grandiorite). \tightlist
\item
  the ecology type (lowland; upland).
\end{itemize}

\end{col}

\begin{col}{0.05\textwidth}
~

\end{col}

\begin{col}{0.46\textwidth}

\begin{itemize}
\tightlist
\item
  the elevation of the lake (in metres). \tightlist
\item
  the lake area (in hectares).
\item
  the catchment area (in hectares).
\end{itemize}

\end{col}

\end{cols}

Use this information to answer the following.

\begin{enumerate}
\def\labelenumi{\arabic{enumi}.}
\item
  What is the \emph{response} variable?\tightlist  
\item
  What is the \emph{explanatory} variable?
\item
  Is the variable `Catchment area' likely to be a \emph{lurking} variable?
\item
  Is the variable `Type of bedrock' likely to be a \emph{confounding} variable?
\item
  What is the \emph{best} description of the variable `Ecology type': response, explanatory, confounding, or lurking variable?
\end{enumerate}

\end{exercise}

\begin{exercise}
\protect\hypertarget{exr:ResearchDesignOverviewStudy1}{}\label{exr:ResearchDesignOverviewStudy1}

A study examined the relationship between diet quality and depression in Australian adolescents \citep{jacka2010associations}. The researchers used a sample of \(7\,114\) adolescents aged~\(10\)--\(14\) years old, and also measured information about (p.~435):

\begin{quote}
\ldots{} age, gender, socioeconomic status, parental education, parental work status, family conflict, poor family management, dieting behaviours, body mass index, physical activity, and smoking\ldots{}
\end{quote}

\begin{enumerate}
\def\labelenumi{\arabic{enumi}.}
\tightlist
\item
  Identify the response and explanatory variables.
\item
  Which of the other listed variable reasonably could be considered \emph{extraneous variables}, \emph{confounding variables} and \emph{lurking variables}?
\end{enumerate}

\end{exercise}

\begin{exercise}
\protect\hypertarget{exr:ResearchDesignOverviewStudy2}{}\label{exr:ResearchDesignOverviewStudy2}A newspaper article \citep{data:GreenTeaCutsRiskOfCancer} reported that `Women who drank green tea at least three times a week were \(14\)~per cent less likely to develop a cancer of the digestive system'. However, the final paragraph of the article notes that:

\begin{quote}
Nobody can say whether green tea itself is the reason, since green tea lovers are often more health-conscious in general.
\end{quote}

Identify the explanatory and response variables, and explain the quotation using language introduced in this chapter.
\end{exercise}

\begin{exercise}
\protect\hypertarget{exr:ResearchDesignOverviewStudy3}{}\label{exr:ResearchDesignOverviewStudy3}A study recorded the lung capacity (using Forced Expiratory Volume, or FEV, in litres) of children aged~\(3\) to~\(19\) \citep{data:Tager:FEV, BIB:data:FEV}, and also recorded whether not the children were smokers. One finding was that children who smoke have a \emph{larger} average FEV (i.e., larger average lung capacity) than children who do \emph{not} smoke, in general.

Name a confounding variable that may explain this surprising finding. Would it be likely that this variable is a \emph{lurking} variable?
\end{exercise}

\begin{exercise}
\protect\hypertarget{exr:ResearchDesignOverviewRTEC}{}\label{exr:ResearchDesignOverviewRTEC}

Consider a study to determine if the percentage of children who consume Ready-To-Eat-Cereals (\textsc{rtec}) for breakfast is the same for children aged between~\(5\) and~\(10\), as for children aged between~\(11\) and~\(15\). The researchers also measured the age of the child, the number of siblings living with the child, and the sex of the child.

\begin{enumerate}
\def\labelenumi{\arabic{enumi}.}
\item
  Which of these variables are extraneous variables? \tightlist

  \begin{itemize}
  \item
    The sex of the child.
  \item
    Whether the child consumes \textsc{rtec}.
  \item
    The age group of the child.
  \item
    The age of the child.
  \item
    The number of siblings living with the child.
  \end{itemize}
\item
  Is the variable `the sex of the child' a lurking variable?
\item
  Is it reasonable to consider the weight of the child as a lurking variable?
\end{enumerate}

\end{exercise}

\begin{exercise}
\protect\hypertarget{exr:DesignExpExtraneous}{}\label{exr:DesignExpExtraneous}

Which of the following types of variables are special types of extraneous variables?\\
(a)~Lurking variables; (b)~explanatory variables; (c)~confounding variables.

\end{exercise}

\begin{exercise}
\protect\hypertarget{exr:DesignFindConfounderA}{}\label{exr:DesignFindConfounderA}A study of New Zealanders found that people wearing hearing aids were more likely to have grey hair than people \emph{not} wearing hearing aids. What confounding variable is likely to be present?
\end{exercise}

\begin{exercise}
\protect\hypertarget{exr:DesignFindConfounderB}{}\label{exr:DesignFindConfounderB}Researchers are studying a new (but expensive) insecticide that is claimed to be more effective for use in apple orchards than other (cheaper) insecticides. A study found that apple orchards where the farmers chose to use the new insecticide had a similar number of insects per tree than orchards where the farmers chose \emph{not} to use the new insecticide, What confounding variable is likely to be present?
\end{exercise}

\begin{exercise}
\protect\hypertarget{exr:DesignConfoundingVarA}{}\label{exr:DesignConfoundingVarA}

An agricultural study recorded the wheat yield for \(18\)~organic farms and \(29\)~conventional farms. Farms across North Dakota and Kansas (USA) were used for the study, and the yield (in tonnes per hectare) was recorded from each farm. The organic farms were generally smaller than the non-organic farms, and located in areas with better soil quality.

Which of these are likely to be confounding variables (if any)? Which may be useful control variables (if any)? Explain your reasoning.

\begin{cols}

\begin{col}{0.30\textwidth}

\begin{enumerate}
\def\labelenumi{\arabic{enumi}.}
\tightlist
\item
  Crop yield.
\item
  Soil quality.
\item
  Climate.
\end{enumerate}

\end{col}

\begin{col}{0.025\textwidth}
~

\end{col}

\begin{col}{0.65\textwidth}

\begin{enumerate}
\def\labelenumi{\arabic{enumi}.}
\setcounter{enumi}{3}
\tightlist
\item
  The colour of the farmer's main tractor.
\item
  The size of the farm.
\item
  The hours of sunlight per day over the growing season.
\end{enumerate}

\end{col}

\end{cols}

\end{exercise}

\begin{exercise}
\protect\hypertarget{exr:DesignConfoundingVarB}{}\label{exr:DesignConfoundingVarB}

A study of school teachers found a relationship between the average number of children plus grandchildren for the teacher, and having high blood pressure.

Which of these is likely to be a confounding variable? Which may be useful control variables? Explain.

\begin{cols}

\begin{col}{0.37\textwidth}

\begin{enumerate}
\def\labelenumi{\arabic{enumi}.}
\tightlist
\item
  Age of the teacher.
\item
  Sex of the teacher.
\item
  The colour of the teacher's car.
\item
  Whether the teacher is a smoker.
\end{enumerate}

\end{col}

\begin{col}{0.025\textwidth}
~

\end{col}

\begin{col}{0.58\textwidth}

\begin{enumerate}
\def\labelenumi{\arabic{enumi}.}
\setcounter{enumi}{4}
\tightlist
\item
  Whether the teacher is very health conscious.
\item
  Whether the teacher has high blood pressure.
\item
  Whether the teacher teaches a health subject.
\end{enumerate}

\end{col}

\end{cols}

\end{exercise}

\captionsetup{font=normalsize}

\begin{EOCanswerBox}{iconmonstr-check-mark-14-240.png}
\textbf{Answers to \emph{Quick review} questions:} \textbf{1.} False. \textbf{2.} False. \textbf{3.} True. \textbf{4.} True. \textbf{5.} True. \textbf{6.} False. \textbf{7.} False (confounding variables are potentially related to the response \emph{and} explanatory variables).

\end{EOCanswerBox}

\chapter{Types of research studies}\label{ResearchDesign}

\begin{cols}
\begin{col}{0.52\textwidth}

\begin{objectivesBox}{iconmonstr-target-4-240.png}
You have learnt how to ask an RQ and understand the main principles of research design.
\textbf{In this chapter}, you will learn to:
\begin{itemize}\tightlist
  \item
  identify and describe the types of quantitative research studies.
  \item
  compare and distinguish experimental and observational studies.
  \item
  describe and identify the directionality in observational studies.
  \item
  describe and identify true experimental and quasi-experimental studies.
\end{itemize}
\end{objectivesBox}

\end{col}

\begin{col}{0.03\textwidth}
~
\end{col}

\begin{col}{0.45\textwidth}

\includegraphics[width=0.95\linewidth]{04-ResearchDesign-TypesOfDesigns_files/figure-latex/unnamed-chunk-4-1} 
\end{col}
\end{cols}

\section{Introduction}\label{introduction}

Chapter~\ref{RQs} introduced four types of research questions: descriptive, relational, repeated-measures and correlational. This chapter discusses the types of research studies needed to answer these RQs, while Chaps.~\ref{Ethics} to~\ref{CollectingDataProcedures} discuss the details of designing these studies and collecting the data.

Different types of studies can be used to collect the data needed to answer RQs:

\begin{itemize}
\tightlist
\item
  \emph{descriptive} studies (Sect.~\ref{DescriptiveStudies}) answer descriptive RQs.
\item
  \emph{observational} studies (Sect.~\ref{ObservationalStudies}) answer RQs with an explanatory variable, but \emph{no intervention}.\index{Intervention}
\item
  \emph{experimental} studies (Sect.~\ref{ExperimentalStudies}) answer RQs with an explanatory variable and \emph{an intervention}.
\end{itemize}

Observational and experimental studies are sometimes collectively called \emph{analytical studies}.\index{Study types!analytical}

\section{Descriptive studies}\label{DescriptiveStudies}

\index{Research question!descriptive}\index{Study types!descriptive}

\begin{definition}[Descriptive study]
\protect\hypertarget{def:DescriptiveStudy}{}\label{def:DescriptiveStudy}\emph{Descriptive studies} answer descriptive RQs.
\end{definition}

Descriptive studies are not explicitly studied further, as the relevant ideas are present in the discussion of observational and experimental studies.

\begin{example}[Descriptive study]
\protect\hypertarget{exm:DescriptiveStudy}{}\label{exm:DescriptiveStudy}\citet{lee2020practice} studied the percentage of people in Hong Kong wearing face masks in various situations. A descriptive RQ is being asked: the population is `residents of Hong Kong', and the outcome is (for example) `the percentage who wear face masks when taking care of family members with fever'. Answering this RQ requires a \emph{descriptive study}.
\end{example}

\section{Observational studies}\label{ObservationalStudies}

\index{Research question!relational}\index{Research question!repeated-measures}\index{Research question!correlational}\index{Study types!observational}

\emph{Observational studies} are used for RQs with no intervention. They are commonly-used, and sometimes are the only type of research design possible. Observational studies do not have an intervention, and hence have \emph{conditions}\index{Conditions} (Def.~\ref{def:Conditions}) rather than \emph{treatments}.\index{Treatments}

\begin{definition}[Observational study]
\protect\hypertarget{def:ObservationalStudy}{}\label{def:ObservationalStudy}\emph{Observational studies} study relationships \emph{without} an intervention.
\end{definition}

\begin{example}[Between-individuals observational study]
\protect\hypertarget{exm:ObservationalRelationalEchinacea}{}\label{exm:ObservationalRelationalEchinacea}Consider again this one-tailed, decision-making RQ (based on the ideas in Sect.~\ref{Writing-RQs}):

\begin{quote}
Among Australian teenagers with a common cold, is the average duration of cold symptoms \emph{shorter} for teens taking a daily dose of echinacea compared to teens taking no medication?
\end{quote}

This RQ has a \emph{between-individuals} comparison\index{Comparison!within individuals}, so is a relational RQ.\spacex If the researchers \emph{do not} impose the taking of echinacea (that is, the individuals make this decision themselves), the study is observational. The two \emph{conditions}\index{Conditions} are `taking echinacea', and `not taking echinacea' (Fig.~\ref{fig:ObsStudiesImageBetween}).
\end{example}

\begin{figure}[hbtp]

{\centering \includegraphics[width=0.65\linewidth]{04-ResearchDesign-TypesOfDesigns_files/figure-latex/ObsStudiesImageBetween-1} 

}

\caption{Observational studies with a relational RQ.\spacex The dashed lines indicate steps not under the control of the researchers.}\label{fig:ObsStudiesImageBetween}
\end{figure}

\begin{example}[Within-individuals observational study]
\protect\hypertarget{exm:ResearchDesignWeightLoss}{}\label{exm:ResearchDesignWeightLoss}\citet{levitsky2004freshman} recorded the weights of university students at the beginning of university, and then after \(12\) weeks from the same students. The comparison is \emph{within} individuals; this is a \emph{repeated-measures} (paired) RQ.\index{Data!paired}\index{Study types!paired} Since the researchers do not impose anything on the students, there is \emph{no intervention} (Fig.~\ref{fig:ObsStudiesImageWithin}).

The \emph{outcome} is the average weight. The \emph{response variable} is the weight of individuals. The \emph{within-individuals comparison}\index{Comparison!within individuals} is the week of the university semester (\(1\) and~\(12\)).
\end{example}

\begin{figure}[hbtp]

{\centering \includegraphics[width=0.75\linewidth]{04-ResearchDesign-TypesOfDesigns_files/figure-latex/ObsStudiesImageWithin-1} 

}

\caption{Observational studies with a repeated-measures RQ.\spacex The dashed lines indicate steps not under the control of the researchers.}\label{fig:ObsStudiesImageWithin}
\end{figure}

\begin{example}[Correlational observational study]
\protect\hypertarget{exm:ResearchDesignKneeFluid}{}\label{exm:ResearchDesignKneeFluid}\citet{poovaragavan2023estimation} explored the relationship between time since death, and the concentration of sodium in synovial (knee) fluid. This is a correlational RQ as groups are not being compared. The time since death is the explanatory variable, and the concentration of sodium in synovial fluid is the response variable. The researchers do not impose the time since death, so there is \emph{no intervention} (Fig.~\ref{fig:ObsStudiesImageCorrelational}).
\end{example}

\begin{figure}[hbtp]

{\centering \includegraphics[width=0.65\linewidth]{04-ResearchDesign-TypesOfDesigns_files/figure-latex/ObsStudiesImageCorrelational-1} 

}

\caption{Observational studies with a correlational RQ.\spacex The dashed lines indicate steps not under the control of the researchers.}\label{fig:ObsStudiesImageCorrelational}
\end{figure}

\section{Experimental studies}\label{ExperimentalStudies}

\index{Research question!relational}\index{Research question!repeated-measures}\index{Research question!correlational}\index{Study types!experimental}

\emph{Experimental studies}, or \emph{experiments}, are used for RQs with an intervention, and are commonly-used. Well-designed experimental studies can establish a \emph{cause-and-effect relationship} between the response and explanatory variables. However, using experimental studies is not always possible. In general, well-designed experimental studies are more likely to be internally valid than observational studies. Experimental studies have an intervention, and hence \emph{treatments} (Def.~\ref{def:Treatments}).

\begin{definition}[Experiment]
\protect\hypertarget{def:Experiment}{}\label{def:Experiment}\emph{Experimental studies} (or \emph{experiments}) study relationships \emph{with} an intervention.
\end{definition}

\begin{importantBox}{iconmonstr-warning-8-240.png}
In an \emph{experimental study}, the unit of analysis (Def.~\ref{def:UnitOfAnalysis}) is the smallest collection of units of observations that can be randomly allocated to separate treatments.

\end{importantBox}

\begin{example}[Within-individuals experimental study]
\protect\hypertarget{exm:ResearchDesignWalk400m}{}\label{exm:ResearchDesignWalk400m}Consider this RQ:

\begin{quote}
For obese men over~\(60\) years-of-age, what is the average increase in heart rate after walking \(400\,\text{m}\)?
\end{quote}

This RQ uses a \emph{within-individuals comparison} (before and after walking \(400\,\text{m}\)) so is a repeated-measures (and paired) RQ.\spacex The study has an intervention if researchers impose the \(400\,\text{m}\) walk on the subjects (Fig.~\ref{fig:ExpStudiesImageWithin}). The \emph{outcome} is the average heart rate. The \emph{response variable} is the heart rate for each individual man.
\end{example}

\begin{figure}[hbtp]

{\centering \includegraphics[width=0.85\linewidth]{04-ResearchDesign-TypesOfDesigns_files/figure-latex/ExpStudiesImageWithin-1} 

}

\caption{Experimental studies with a repeated-measures RQ.\spacex The dashed lines indicate steps not under the control of the researchers.}\label{fig:ExpStudiesImageWithin}
\end{figure}

\begin{example}[Correlational experimental study]
\protect\hypertarget{exm:ResearchDesignLeafDrip}{}\label{exm:ResearchDesignLeafDrip}\citet{xu2023design} studied leaf-drip irrigation, exploring the relationship between the water pressure and flow rate. This is a correlational RQ, where the hydraulic pressure time is the explanatory variable, and the flow rate is the response variable. The researchers imposed nine different values for water pressure, so there is an intervention (Fig.~\ref{fig:ObsStudiesImageCorrelationalintervention}).
\end{example}

\begin{figure}[hbtp]

{\centering \includegraphics[width=0.65\linewidth]{04-ResearchDesign-TypesOfDesigns_files/figure-latex/ObsStudiesImageCorrelationalintervention-1} 

}

\caption{Experimental studies with a correlational RQ.\spacex The dashed lines indicate steps not under the control of the researchers.}\label{fig:ObsStudiesImageCorrelationalintervention}
\end{figure}

\emph{Between}-individuals experimental studies can be either \emph{true experiments} (Sect.~\ref{TrueExperiments}) or \emph{quasi-experiments} (Sect.~\ref{QuasiExperiments}); see Table~\ref{tab:ExperimentalStudyDesigns}.

\begin{table}
\centering
\caption{\label{tab:ExperimentalStudyDesigns}Comparing analytical designs with a between-individuals comparison.}
\centering
\fontsize{8}{10}\selectfont
\begin{tabular}[t]{>{\raggedleft\arraybackslash}p{24mm}>{\centering\arraybackslash}p{40mm}>{\centering\arraybackslash}p{40mm}>{\raggedright\arraybackslash}p{16mm}}
\toprule
\textbf{Study type} & \textbf{Individuals allocated to groups?} & \textbf{Treatments allocated to groups?} & \textbf{Reference}\\
\midrule
Observational & No & No & Sect. \ref{ObservationalStudies}\\
True experiment & Yes & Yes & Sect. \ref{TrueExperiments}\\
Quasi-experiment & No & Yes & Sect. \ref{QuasiExperiments}\\
\bottomrule
\end{tabular}
\end{table}

\subsection{True experimental studies}\label{TrueExperiments}

\index{Study types!experimental!true}

\emph{True experiments} are commonly used to answer relational RQs. An example of a true experiment is a \emph{randomised controlled trial}, often used in drug trials.

\begin{definition}[True experiment]
\protect\hypertarget{def:TrueExperiment}{}\label{def:TrueExperiment}In a \emph{true experiment}, the researchers:

\begin{enumerate}
\def\labelenumi{\arabic{enumi}.}
\tightlist
\item
  allocate treatments to groups of individuals (i.e., allocate the values of the explanatory variable to the individuals), \emph{and}
\item
  determine who or what individuals are in those groups.
\end{enumerate}

While the steps may not happen \emph{explicit}, they happen \emph{conceptually}.
\end{definition}

\begin{example}[True experiment]
The echinacea study (Sect.~\ref{Writing-RQs}) could be designed as a \emph{true experiment}. The researchers would allocate individuals to one of two groups, and then decide which group took echinacea and which group did not (Fig.~\ref{fig:TrueExpStudiesImage}).

These steps may happen implicitly: researchers may allocate each person at random to one of the two groups (echinacea; no echinacea). This is still a true experiment, since the researchers could decide to switch which group receives echinacea; ultimately, the decision is still made by the researchers.
\end{example}

\begin{figure}[hbtp]

{\centering \includegraphics[width=0.65\linewidth]{04-ResearchDesign-TypesOfDesigns_files/figure-latex/TrueExpStudiesImage-1} 

}

\caption{True experimental studies: researchers allocate individuals to groups, and treatments to groups.}\label{fig:TrueExpStudiesImage}
\end{figure}

\subsection{Quasi-experimental studies}\label{QuasiExperiments}

\index{Study types!experimental!quasi}

Quasi-experiments are similar to true experiments (i.e., answer relational RQs) but treatments are \emph{allocated} to groups that \emph{already exist} (e.g., may be naturally occurring).

\begin{definition}[Quasi-experiment]
\protect\hypertarget{def:QuasiExperiment}{}\label{def:QuasiExperiment}

In a \emph{quasi-experiment}, the researchers:

\begin{enumerate}
\def\labelenumi{\arabic{enumi}.}
\tightlist
\item
  allocate treatments to groups of individuals (i.e., allocate the values of the explanatory variable to the individuals), but
\item
  do \emph{not} determine who or what individuals are in those groups.
\end{enumerate}

\end{definition}

\begin{example}[Quasi-experiments]
\protect\hypertarget{exm:QuasiEchinacea}{}\label{exm:QuasiEchinacea}The echinacea study (Sect.~\ref{Writing-RQs}) could be designed as a quasi-experiment. The researchers could \emph{find} two existing groups of people (say, from Suburbs~A and~B), then decide to allocate people in Suburb~A to take echinacea, and people in Suburb~B to \emph{not} take echinacea (Fig.~\ref{fig:QuasiExpStudiesImage}).
\end{example}

\begin{figure}[hbtp]

{\centering \includegraphics[width=0.65\linewidth]{04-ResearchDesign-TypesOfDesigns_files/figure-latex/QuasiExpStudiesImage-1} 

}

\caption{Quasi-experimental studies: researchers do not allocate individuals to groups, but do allocate treatments to groups. The dashed lines indicate steps not under the control of the researchers.}\label{fig:QuasiExpStudiesImage}
\end{figure}

\begin{example}[Quasi-experiments]
\protect\hypertarget{exm:QuasiAlcoholAwareness}{}\label{exm:QuasiAlcoholAwareness}A researcher wants to examine the effect of an alcohol awareness program (based on \citet{macdonald2008enough}) on the average amount of alcohol consumed per student in a university Orientation Week. She runs the program at University A only, then compares the average amount of alcohol consumed per person at two universities (A and~B).

This study is a \emph{quasi-experiment} since the researcher did not (and can not) determine the groups: the students (not the researcher) would have chosen University~A or University~B for many reasons. However, the researcher \emph{did} decide whether to allocate the program to University~A or University~B.
\end{example}

\section{Comparing study types}\label{CompareStudyTypes}

\index{Study types!compared}

In \emph{experimental} studies, researchers \emph{create} differences in the values of the explanatory variable through allocation, and then note the effect this has on the values of the response variable. In \emph{observational} studies, researchers \emph{observe} differences in the values of the explanatory variable, and observe the values of the response variable.

Importantly, \emph{only well-designed true experiments can show cause-and-effect}. Nonetheless, well-designed observational and quasi-experimental studies can provide evidence to \emph{support} cause-and-effect conclusions, especially when supported by other evidence. Although only true experimental studies can show cause-and-effect, true experimental studies are often not possible for ethical, financial, practical and/or logistical reasons.

The advantages and disadvantages of each study type are discussed later (Sect.~\ref{InterpretStudyDesign}), after these study types are discussed in greater detail in the following chapters.

\begin{example}[Cause and effect]
\protect\hypertarget{exm:Autism}{}\label{exm:Autism}Many studies report that the bacteria in the gut of people on the autism spectrum is different from the bacteria in the gut of people \emph{not} on the autism spectrum \citep{kang2019long, ho2020gut}, and suggest the bacteria may contribute whether a person is autistic. These studies were observational, so the\index{Cause and effect} suggestion of a cause-and-effect relationship may be inaccurate.

Other studies \citep{yap2021autism} suggest that people on the autism spectrum are more likely to be `picky eaters', which contributes to the differences in gut bacteria.
\end{example}

\section{Directionality in research studies}\label{Directionality}

\index{Study types!directionality}

Analytical research studies (observational; experimental) can be classified by their \emph{directionality} (Table~\ref{tab:TypesOfObsStudies}).\index{Study types!analytical}

\begin{itemize}
\tightlist
\item
  \emph{Forward direction} (Sect.~\ref{Forward}): the values of the explanatory variable are obtained, and the study determines what values of the response variable occur in the future. \emph{All experimental studies have a forward direction.}
\item
  \emph{Backward direction} (Sect.~\ref{Backward}): the values of the response variable are obtained, then the study determines what values of the explanatory variable occurred in the past.
\item
  \emph{No direction} (Sect.~\ref{NonDirectional}): the values of the response and explanatory variables are obtained at the same time.
\end{itemize}

Directionality is important for understanding cause-and-effect relationships. If the explanatory variable occurs \emph{before} the outcome is observed, a cause-and-effect relationship \emph{may} be possible. That is, studies with a forward direction are more likely to provide evidence of causality.

\begin{table}
\centering
\caption{\label{tab:TypesOfObsStudies}Classifying observational studies. (Experimental studies have a forward direction.)}
\centering
\fontsize{8}{10}\selectfont
\begin{tabular}[t]{rcc}
\toprule
\textbf{Type} & \textbf{Explanatory variable} & \textbf{Response variable}\\
\midrule
Forward direction & When study begins & Determined in the future\\
Backward direction & Determined from the past & When study begins\\
No direction & When study begins & When study begins\\
\bottomrule
\end{tabular}
\end{table}

\begin{example}[Directionality]
\protect\hypertarget{exm:BackwardStudy}{}\label{exm:BackwardStudy}In South Australia in 1988--1989, \(25\)~cases of legionella infections (an unusually high number) were investigated. All \(25\)~cases were gardeners.

\citet{data:oconnor:pottingmix} compared \(25\)~people with legionella infections with \(75\) similar people without the infection, and found that recent (past) use of potting mix was associated with an increase in the risk of contracting illness.

This study has a backward \emph{direction}: people were identified with an infection, and then the researchers looked \emph{back} at past activities.
\end{example}

Research studies are sometimes described as `prospective'\index{Study types!prospective} or `retrospective',\index{Study types!prospective} but these terms can be misleading \citep{ranganathan2018study} and their use not recommended \citep{VANDENBROUCKE20141500}.

\emph{Experimental studies always have a forward direction.} Observational studies may have any directionality, and may be given different names accordingly.

\subsection{Forward-directional studies}\label{Forward}

\index{Study types!directionality!forward}

All experimental studies have a forward direction, and include \emph{randomised controlled trials} (RCTs) and \emph{clinical trials}.\index{Clinical trials}\index{Randomised controlled trials}

Observational studies with a \emph{forward} direction are often called \emph{cohort studies}.\index{Study types!directionality!forward}\index{Study types!cohort studies} Both experimental studies and cohort studies can be expensive and tricky: tracking individuals (a \emph{cohort}) into the future is not always easy, and the ability to track some individuals into the future may be lost (\emph{drop-outs}):\index{Drop outs} plants or animals may die, people may move or decide to no longer participate, etc. Forward-directional observational studies:

\begin{itemize}
\tightlist
\item
  may add support to cause-and-effect conclusions, since the comparison occurs \emph{before} the outcome (only well-designed experimental studies can establish cause-and-effect).
\item
  can examine many outcomes in one study, since the outcome(s) occur in the future.
\item
  can be problematic for rare outcomes, as the outcome of interest may never (or rarely) appear in the future.
\end{itemize}

\begin{example}[Forward study]
\citet{chih2018incidence} studied dogs and cats who had been recommended to receive intermittent nasogastric tube (NGT) aspiration for up to~\(36\,\text{h}\). Some pet owners did not give permission for NGT, while some did; thus, whether the animal received NGT was \emph{not} determined by the researchers (the study is observational). The researchers then observed whether the animals developed hypochloremic metabolic alkalosis (HCMA) in the next \(36\,\text{h}\).

Since the explanatory variable (whether NGT was used) was recorded at the start of the study, and the response variable (whether HCMA was observed) was determined within the following \(36\,\text{h}\), this study has a \emph{forward direction}.
\end{example}

\subsection{Backward-directional studies}\label{Backward}

\index{Study types!directionality!backward}

Observational studies with a \emph{backward} direction are often called \emph{case-control} studies.\index{Study types!directionality!backward}\index{Study types!case-control studies} The `cases' are often individuals with a certain disease, and then the controls are those without the disease (see Example~\ref{exm:BackwardStudy}). Researchers find individuals with specific values of the response variable (cases and controls), and determine values of the explanatory variable from the past. Case-control studies:

\begin{itemize}
\tightlist
\item
  only allow one outcome to be studied, since individuals are chosen to be in the study based on the value of the response variable of interest.
\item
  are useful for rare outcomes, as the researchers can purposely select large numbers with the rare outcome of interest.
\item
  do not effectively eliminate other explanations for the relationship between the response and explanatory variables (\emph{confounding}; Def.~\ref{def:Confounding}).
\item
  may suffer from \emph{selection bias} (Sect.~\ref{SelectionBias}), as researchers purposively try to locate individuals with a rare outcome.\index{Bias!selection}
\item
  may suffer from \emph{recall bias} (Sect.~\ref{Biases}) when the individuals are people: accurately recalling the past can be unreliable.\index{Bias!recall}
\end{itemize}

\begin{example}[Backwards study]
\citet{data:Pamphlett:toxins} examined patients with and without sporadic motor neurone disease (\textsc{smnd}), and asked about \emph{past} exposure to metals.

The response variable (whether the respondent had \textsc{smnd}) is assessed when the study begins, and whether subjects had exposure to metals (explanatory variable) is determined from the \emph{past}. This observational study has a \emph{backward} direction.
\end{example}

\subsection{Non-directional studies}\label{NonDirectional}

\index{Study types!directionality!non-directional}

\emph{Non-directional} observational studies are called \emph{cross-sectional} studies.\index{Study types!directionality!non-directional}\index{Study types!cross-sectional studies} Cross-sectional studies:

\begin{itemize}
\tightlist
\item
  are good for findings associations between variables (and these associations may or may not be causation).
\item
  are generally quicker and cheaper to conduct than other types of studies.
\item
  are not useful for studying rare outcomes.
\item
  do not effectively eliminate other explanations for the relationship between the response and explanatory variables (\emph{confounding}; Def.~\ref{def:Confounding}).
\end{itemize}

\begin{example}[Non-directional study]
\citet{data:Russell2014:FoodInsecurity} asked older Australian their opinions of their own food security, and recorded their living arrangements. Individuals' responses to both the response variable and explanatory variable were gathered at the same time. This observational study is \emph{non-directional}.
\end{example}

\section{The role of research design}\label{DesignImportance}

Choosing the \emph{type} of study is only one part of research design; many other decisions must be made also. The purpose of these decisions is to ensure researchers can confidently study the relationship between the response and explanatory variables (\emph{internal validity}) in the population of interest (\emph{external validity}) from studying one the many possible samples. This is related to the idea of \emph{bias}.\index{Bias}

\begin{definition}[Bias]
\protect\hypertarget{def:Bias}{}\label{def:Bias}\emph{Bias} refers to any systematic misrepresentation of the target population or a parameter caused by the sampling or the study design.
\end{definition}

Various types of bias are possible, some of which are studied later. Maximising internal and external validity reduces bias. Bias may occur during research design, sample selection (Sect.~\ref{SelectionBias}), data collection (selection bias; Sect.~\ref{def:SelectionBias}), analysis, or interpretation of results (Chap.~\ref{Interpretation}). This book only discusses some possible biases.

Designing a study to maximise \emph{internal validity} means:

\begin{itemize}
\tightlist
\item
  identifying \emph{what else} might influence the values of the response variable, apart from the explanatory variable (Chap.~\ref{ResearchDesignOverview}).
\item
  designing the study to be \emph{effective} (Chap.~\ref{DesignInternal}).
\end{itemize}

In general, experimental studies have better internal validity than observational studies.

Designing a study to maximise \emph{external validity} means:

\begin{itemize}
\tightlist
\item
  identifying who or what to study, since the whole population cannot be studied (Chap.~\ref{Sampling}).
\item
  determining \emph{how many} individuals to study. (We need to learn more before we can answer this critical question in Chap.~\ref{EstimatingSampleSize}.)
\end{itemize}

Details of the data \emph{collection} (Chap.~\ref{CollectingDataProcedures}) and \emph{ethical} issues (Chap.~\ref{Ethics}) also form part of the study design.

\section{Chapter summary}\label{Chap3-Summary}

Three types of research studies are: \emph{descriptive studies} (for studying descriptive RQs), \emph{observational studies} (for studying relationships \emph{without} an intervention), and \emph{experimental} (for studying relationships \emph{with} an intervention).

Observational studies can be classified as having a \emph{forward direction} (cohort studies), \emph{backward direction} (case-control studies), or \emph{no direction} (cross-sectional studies). Experimental studies always have a forward direction. Relational RQs with an intervention can be classified as \emph{true experiments} or \emph{quasi-experiments}. Cause-and-effect conclusions can only be made from well-designed \emph{true experiments}.

Ideally studies should be designed to be \emph{internally} and \emph{externally} valid. In general, experimental studies have better internal validity than observational studies.

\section{Quick review questions}\label{Chap3-QuickReview}

Are the following statements \emph{true} or \emph{false}?

\begin{enumerate}
\def\labelenumi{\arabic{enumi}.}
\item
  \citet{fraboni2018red} studied the `red-light running behaviour of cyclists in Italy'. \tightlist This study is most likely to be observational.
\item
  In a true experiment, the researchers apply treatments to groups that they have determined; in a quasi-experiment, the researchers apply treatments to groups that they have not determined.
\item
  In a quasi-experiment, the researchers allocate treatments to groups that they cannot manipulate.
\item
  True experiments generally have a higher internal validity than observational studies.
\item
  Observational studies generally have a higher external validity than quasi-experimental studies.
\end{enumerate}

\section{Exercises}\label{ResearchDesignsExercises}

\hyperref[Answers]{Answers to odd-numbered exercises} are given at the end of the book.

\captionsetup{font=small}

\begin{exercise}
\protect\hypertarget{exr:TypesOfDesignsAcuteOtitis}{}\label{exr:TypesOfDesignsAcuteOtitis}

Consider this RQ \citep{data:McLinn:otitis}:

\begin{quote}
In children with acute otitis media, what is the difference in the average duration of symptoms when treated with cefuroxime compared to amoxicillin?
\end{quote}

\begin{enumerate}
\def\labelenumi{\arabic{enumi}.}
\tightlist
\item
  Is the comparison a within- or between-individuals comparison?
\item
  Is this RQ descriptive, relational, repeated-measures or correlational?
\item
  Is there likely an intervention?
\item
  Is the RQ an estimation or decision-making RQ?
\item
  Is the study observational or experimental? If observational, what is the \emph{direction}? If experimental, is this a quasi-experiment or true experiment?
\end{enumerate}

\end{exercise}

\begin{exercise}
\protect\hypertarget{exr:TypesOfDesignsWorms}{}\label{exr:TypesOfDesignsWorms}

\citet{data:Khair2015:Earthworms} studied the time needed for organic waste to turn into compost. For some batches of compost, earthworms were added. In other batches, earthworms were \emph{not} added to the waste. One RQ asked whether the composting times for waste with and without earthworms was the same or not.

\begin{enumerate}
\def\labelenumi{\arabic{enumi}.}
\tightlist
\item
  Is the comparison a within- or between-individuals comparison?
\item
  Is this RQ descriptive, relational, repeated-measures or correlational?
\item
  Is there an intervention?
\item
  Is the RQ an estimation or decision-making RQ?
\item
  Is the study observational or experimental? If observational, what is the \emph{direction}? If experimental, is this a quasi-experiment or true experiment?
\end{enumerate}

\end{exercise}

\begin{exercise}
\protect\hypertarget{exr:ResearchDesignConcreteBeams}{}\label{exr:ResearchDesignConcreteBeams}\citet{gonzalez2007shear} studied recycled concrete beams. Beams were divided into three groups, different loads were then applied to each group, then the shear strength needed to fracture the beams was measured. Is this a \emph{quasi-experiment} or a \emph{true experiment}? Explain.
\end{exercise}

\begin{exercise}
\protect\hypertarget{exr:ResearchDesignVAP}{}\label{exr:ResearchDesignVAP}

A research study compared the use of two different education programs to reduce the percentage of patients experiencing ventilator-associated pneumonia (VAP). Paramedics from two cities were chosen to participate. Paramedics in City~A were allocated to receive Program~1, and paramedics in the other city to receive Program~2.

\begin{enumerate}
\def\labelenumi{\arabic{enumi}.}
\tightlist
\item
  Is this RQ descriptive, relational, repeated-measures or correlational?
\item
  Is the comparison a within- or between-individuals comparison?
\item
  Is there likely an intervention?
\item
  Is the study observational or experimental? If observational, what is the \emph{direction}? If experimental, is this a quasi-experiment or true experiment?
\end{enumerate}

\end{exercise}

\begin{exercise}
\protect\hypertarget{exr:ResearchDesignMatresses}{}\label{exr:ResearchDesignMatresses}\citet{data:Manzano2013:Matresses} compared `the effectiveness of alternating pressure air mattresses vs.~overlays, to prevent pressure ulcers' (p.~\(2\,099\)). Patients were \emph{provided} with alternating pressure air overlays (in 2001) or alternating pressure air mattresses (in 2006). The number of pressure ulcers were recorded.

This study is experimental, because the researchers \emph{provided} the mattresses. Is this a \emph{true} experiment or \emph{quasi}-experiment? Explain.
\end{exercise}

\begin{exercise}
\protect\hypertarget{exr:ResearchDesignDietsForWeightLoss}{}\label{exr:ResearchDesignDietsForWeightLoss}

\citet{data:sacks:weightloss} compared four weight-loss diets, using \(811\) overweight adults each randomly assigned to one diet. The diets used comparable foods. The authors state (p.~859):

\begin{quote}
The primary outcome was the change in body weight after \(2\)~years in {[}\ldots{]} comparisons of low fat versus high fat and average protein versus high protein and in the comparison of highest and lowest carbohydrate content.
\end{quote}

\begin{enumerate}
\def\labelenumi{\arabic{enumi}.}
\tightlist
\item
  What is the \emph{between}-individuals comparison?
\item
  What is the \emph{within}-individuals comparison?
\item
  Is this study observational or experimental? Why?
\item
  Is this study a quasi-experiment or a true experiment? Why?
\item
  What are the units of analysis?
\item
  What are the units of observation?
\item
  What is the response variable?
\item
  What is the explanatory variable?
\end{enumerate}

\end{exercise}

\begin{exercise}
\protect\hypertarget{exr:ResearchDesignPetsAndHealth}{}\label{exr:ResearchDesignPetsAndHealth}

Consider this initial RQ (based on \citet{friedmann1985health}), that clearly needs refining: `Are people with pets healthier?'

\begin{enumerate}
\def\labelenumi{\arabic{enumi}.}
\tightlist
\item
  Briefly describe useful and practical definitions for P,~O and~C.
\item
  Briefly describe an \emph{experimental} study to answer the RQ.
\item
  Briefly describe an \emph{observational} study to answer the RQ.
\end{enumerate}

\end{exercise}

\begin{exercise}
\protect\hypertarget{exr:ResearchDesignSeedsSprout}{}\label{exr:ResearchDesignSeedsSprout}

Consider this initial RQ, that clearly needs refining: `Are seeds more likely to sprout when a seed-raising mix is used?'

\begin{enumerate}
\def\labelenumi{\arabic{enumi}.}
\tightlist
\item
  Briefly describe useful and practical definitions for P,~O and~C.
\item
  Briefly describe an \emph{experimental} study to answer the RQ.
\item
  Briefly describe an \emph{observational} study to answer the RQ.
\end{enumerate}

\end{exercise}

\captionsetup{font=normalsize}

\begin{EOCanswerBox}{iconmonstr-check-mark-14-240.png}
\textbf{Answers to \emph{Quick review} questions:} \textbf{1.} True. \textbf{2.} True. \textbf{3.} True. \textbf{4.} True. \textbf{5.} False (irrelevant).

\end{EOCanswerBox}

\chapter{Ethics in research}\label{Ethics}

\begin{cols}
\begin{col}{0.52\textwidth}

\begin{objectivesBox}{iconmonstr-target-4-240.png}
You have learnt how to ask an RQ, and identify different types of studies to obtain data.
\textbf{In this chapter}, you will learn to:

\begin{itemize}\tightlist
  \item
  list common ethical issues to be considered in research design.
  \item
  understand the purpose of reproducible research.
\end{itemize}
\end{objectivesBox}

\end{col}

\begin{col}{0.03\textwidth}
~
\end{col}

\begin{col}{0.45\textwidth}

\includegraphics[width=0.95\linewidth]{05-ResearchDesign-Ethics_files/figure-latex/unnamed-chunk-4-1} 
\end{col}
\end{cols}

\section{Introduction: obtaining ethical clearance}\label{EthicalGuidelines}

\index{Ethics}

All research \emph{must} be ethical, and \emph{must} meet ethical guidelines, to minimise risk of harm to the environment, property and to participants, and to preserve the well-being, dignity, rights and safety of participants (including animals). Practically every university and research organisation in the world promotes and enforces ethical research practices.

Most research studies require an ethics committee to formally grant ethics approval \emph{before} research begins. Only brief comments about research ethics are given here.

\begin{example}[Ethics]

Ethics are important for \emph{all} studies, not just those involving people or animals. For example:

\begin{itemize}
\tightlist
\item
  in engineering, \(238\) articles published between~1945 and~2015 were retracted, mostly for unethical research practice \citep{rubbo2019retractions}.
\item
  in the chemical sciences, \(331\) retractions were reported in~2017 and~2018 due to ethical issues, such as falsification of data\index{Data falsification} and plagiarism \citep{coudert2019correcting}.\index{Plagiarism}
\end{itemize}

\end{example}

\section{Ethical issues in research design}\label{Common-Ethical-Issues}

Ethical issues embrace many areas when designing research studies.

\begin{itemize}
\tightlist
\item
  \emph{Acknowledgements}: all those who contributed to the research should be acknowledged, including those who prepare figures, take photographs, or have helped collect data.
\item
  \emph{Analysis}: the analysis must use appropriate methods.
\item
  \emph{Confidentiality}: data should be kept confidential and secure.
\item
  \emph{Consent}: when appropriate, people should consent to being in the study, and hence should be told what the study involves. People should also be able to withdraw from the study without penalty.
\item
  \emph{Economic risks}: financial loss to participants should be avoided. Reimbursements of reasonable costs for participating may be appropriate.
\item
  \emph{Environmental risks}: environmental impacts and damage should be avoided or minimised.
\item
  \emph{Funding}: sources of funding should be disclosed. Any studies funded by, or sanctioned by, companies or organisations with vested interests need to be carefully scrutinised. These may lead to, or may give the impression of, conflicts of interest.
\item
  \emph{Incentives to participate}: if participants are offered incentives to participate (above reimbursement of costs), these should be acknowledged as it may cause (perhaps unconsciously), or may give the impression of causing, participants to influence the results.
\item
  \emph{Legal risks}: participants should not be put in the position of breaking laws, and the research itself should not break any laws.
\item
  \emph{Plagiarism}:\index{Plagiarism} the work of others should be appropriately acknowledged and not claimed to be original (see Sect.~\ref{Referencing}).
\item
  \emph{Physical risks}: physical harm or discomfort (to researchers, participants or bystanders) should be avoided or minimised.
\item
  \emph{Psychological risks}: psychological harm or discomfort (to researchers, participants or bystanders) should be avoided or minimised.
\item
  \emph{Resourcing}: the study should not waste resources, time or money (e.g., if the answer to the RQ is already known, the study is not necessary).
\item
  \emph{Sample size}: the study should not use more individuals than necessary.\index{Sample size}
\item
  \emph{Social risks}: social harm or discomfort (to researchers, participants or bystanders) should be avoided or minimised.
\item
  \emph{Storage of data}: data should be stored securely, kept for the required amount of time, then (if appropriate) securely disposed.
\end{itemize}

\begin{example}[Poor ethics]
\protect\hypertarget{exm:EthicsTuskegee}{}\label{exm:EthicsTuskegee}In the Tuskegee syphilis experiment, conducted between~1932 and~1972, treatments were actively withheld from men with syphilis \citep{corbie1999continuing}. The men's wives and children were often affected, and the men were lied to about their treatments. This study was highly unethical, and could not be conducted now.
\end{example}

\begin{example}[Poor ethics in analysis]
\protect\hypertarget{exm:EthicsChallenger}{}\label{exm:EthicsChallenger}In~1986, the American space shuttle \emph{Challenger} exploded just after launch, killing all seven astronauts on board. A review \citep{data:dalal:shuttle} found the cause was partly that engineers failed to use some data that should have been used. This was unethical.
\end{example}

\clearpage

\section{Reproducible research}\label{ReproducibleResearch}

\index{Computers and software}\index{Research!reproducibility}

One way to ensure that research results are reliable and trustworthy is through \emph{reproducible} research: enabling someone else to repeat the study and analysis, to confirm the findings. For research to be reproducible, the methods, data, analysis methods and relevant computer code must be available \citep{laine2007reproducible} when possible. (Sometimes releasing data is unethical, such as when individuals may be identified, so should not be released.)

Methods for ensuring reproducible research are often discipline dependent, and beyond the scope of this book. Different journals also have different expectations regarding reproducibility. Nonetheless, the basic ideas are important.

The importance of reproducibility in the analysis phase is crucial; for example:

\begin{quote}
There are serious medical consequences to errors attributable to the effects of spreadsheet programs and software operated through a graphical user interface {[}\ldots{]} that could have been avoided through a reproducible research paradigm\ldots{}

\VA{--- \citet{simons2019reproducible}, p.~471}{}
\end{quote}

Using purely point-and-click interfaces for statistical analysis (e.g., spreadsheets)\index{Computers and software!spreadsheets} is not recommended, as results are not reproducible.

Rather than using spreadsheets (see Sect.~\ref{DataEntry}), using tools which encourage reproducible research are recommended. Statistical software packages, such as jamovi, Python, R, SAS, SPSS and Stata, are recommended as the analysis commands can be recorded (even when using the point-and-click interfaces), and hence the analysis is reproducible.\index{Computers and software!statistical}

\section{Chapter summary}\label{Chap4-Summary}

Studies must be ethical, and any formal study must obtain ethical approval \emph{before} beginning. Ethics covers issues including, but not restricted to:

\begin{cols}

\begin{col}{0.3\textwidth}

\begin{itemize}
\tightlist
\item
  acknowledgements.
\item
  analysis methods.
\item
  confidentiality.
\item
  consent.
\item
  economic risks.
\item
  environmental risks.
\end{itemize}

\end{col}

\begin{col}{0.025\textwidth}
~

\end{col}

\begin{col}{0.35\textwidth}

\begin{itemize}
\tightlist
\item
  funding.
\item
  incentives for participants.
\item
  legal risks.
\item
  plagiarism.
\item
  physical risks.
\end{itemize}

\end{col}

\begin{col}{0.025\textwidth}
~

\end{col}

\begin{col}{0.3\textwidth}

\begin{itemize}
\tightlist
\item
  psychological risks.
\item
  resourcing.
\item
  sample size.
\item
  social risks.
\item
  storage of data.
\end{itemize}

\end{col}

\end{cols}

\clearpage

\section{Quick review questions}\label{Chap4-QuickReview}

Are the following statements \emph{true} or \emph{false}?

\begin{enumerate}
\def\labelenumi{\arabic{enumi}.}
\item
  Ethics apply for \emph{any} type of study. \tightlist 
\item
  Ethics only refer to the interactions of the researchers with participants in the study.
\item
  Ethics only apply when \emph{people} are the individuals.
\item
  Ethics only apply when \emph{people} or \emph{animals} are the individuals.
\item
  Ethics can extend to storage of data.
\item
  Ethics only apply to the design of the study.
\item
  Ethics apply even to the analysis of the data.
\end{enumerate}

\section{Exercises}\label{EthicsExercises}

\hyperref[Answers]{Answers to odd-numbered exercises} are given at the end of the book.

\captionsetup{font=small}

\begin{exercise}
\protect\hypertarget{exr:EthicsCougars}{}\label{exr:EthicsCougars}Consider this conundrum \citep{crozier2015towards}:

\begin{quote}
A research team has an extraordinarily successful long-term study of a population of bighorn sheep (\emph{Ovis canadensis}) on Ram Mountain\ldots{} \smallskip

The population contains marked individuals for which the research team has incredibly detailed data {[}\ldots{]} this research has lead to numerous important publications. \smallskip

Recently, however, a cougar (\emph{Puma concolor}) that has learned to specialize on these sheep is slowly but surely eating all of them. This is a study of a natural population, which includes predation, but this cougar is drastically reducing the sample size of the study. \smallskip

Since it is legal to hunt cougars in the region where this study is taking place, one option is to try to kill the predator; however, even if a cougar were successfully hunted, this would not ensure that it was the correct one.
\end{quote}

What action would you recommend, from an ethical point-of-view?
\end{exercise}

\begin{exercise}
\protect\hypertarget{exr:EthicsPlacebo}{}\label{exr:EthicsPlacebo}Suppose a research group is testing a new drug, with the potential to cure a debilitating illness. The researchers could (a)~\emph{use} a control group that does not receive the new drug, and so obtain stronger evidence for using the drug if it works; or (b)~\emph{not use} a control group, so that everyone in the study potentially benefits from the using the new drug.

What would you decide? Explain.
\end{exercise}

\begin{exercise}
\protect\hypertarget{exr:EthicsSideEffects}{}\label{exr:EthicsSideEffects}Suppose a very deadly and highly contagious disease breaks out. Is it ethical to use a new drug to treat those affected, even though the drug is experimental and the potentially harmful side effects are unknown? Discuss your point-of-view.
\end{exercise}

\begin{exercise}
\protect\hypertarget{exr:EthicsLying}{}\label{exr:EthicsLying}Is it ethical to lie to subjects? \emph{Deception}\index{Deception} is used in some disciplines, and may be approved by ethics committees under some circumstances (such as potential benefits of the study, and whether the deception may cause physical or psychological discomfort to the participants).

Is it ethical to tell participants that they are taking an active medication, when it is actually ineffective (a `placebo')? Discuss the advantages and disadvantages.
\end{exercise}

\captionsetup{font=normalsize}

\begin{EOCanswerBox}{iconmonstr-check-mark-14-240.png}
\textbf{Answers to \emph{Quick review} questions:} \textbf{1.} True. \textbf{2.} False. \textbf{3.} False. \textbf{4.} False. \textbf{5.} True \textbf{6.} False. \textbf{7.} True.

\end{EOCanswerBox}

\chapter{External validity: sampling}\label{Sampling}

\begin{cols}
\begin{col}{0.52\textwidth}

\begin{objectivesBox}{iconmonstr-target-4-240.png}
You have learnt to ask an RQ, and identify a study design.
\textbf{In this chapter}, you will learn to:

\begin{itemize}\tightlist
  \item
  distinguish and explain precision and accuracy.
  \item
  distinguish and explain random and non-random sampling.
  \item
  explain why random samples are preferred over non-random samples.
  \item
  identify, describe and use different sampling methods.
  \item
  identify ways to obtain samples likely to be representative.
\end{itemize}
\end{objectivesBox}

\end{col}

\begin{col}{0.03\textwidth}
~
\end{col}

\begin{col}{0.45\textwidth}

\includegraphics[width=0.95\linewidth]{06-ResearchDesign-Sampling_files/figure-latex/unnamed-chunk-12-1} 
\end{col}
\end{cols}

\section{Introduction}\label{IntroExternalValidity}

\index{External validity}\index{Research design!external validity|(}

In a research study, the researchers learn about the \emph{population} by studying just one of the countless possible \emph{samples}. Ideally the sample that is studied is representative of the population, so the results from the sample generalise to the population. This is called \emph{external validity}.\index{External validity} \emph{External validity} does \emph{not} mean that the results apply more widely than the intended population.

\begin{example}[External validity]
\protect\hypertarget{exm:ExternalValidPop}{}\label{exm:ExternalValidPop}Suppose the \emph{population} in a study is \emph{Californian home-owners}. The sample comprises the Californian home-owners studied by the researchers. If the study is externally valid, the sample is representative of all Californian home-owners.

The results will not necessarily apply to home-owners outside of Californian, or all Californian residents. However, this \emph{is irrelevant for external validity}. External validity concerns how the \emph{sample} represents the intended population in the RQ, which is \emph{Californian home-owners}.
\end{example}

\section{The idea of sampling}\label{IdeaOfSampling}

\index{Sampling}

Studying every member of a population is very rare due to cost, time, ethics, logistics and/or practicality. Instead, a subset of the population (a \emph{sample}) is studied, and \emph{many} different samples are possible.\index{Sample}

\begin{importantBox}{iconmonstr-warning-8-240.png}
The challenge of research is learning about a population from studying just one of the countless possible samples.

\end{importantBox}

\begin{example}[Samples]
\protect\hypertarget{exm:SamplesAspirin}{}\label{exm:SamplesAspirin}A study of the effectiveness of aspirin in treating headaches cannot possibly study every single human who may one day take aspirin. Not only would this be prohibitively expensive, time consuming, and impractical, but the study would not even study those not yet born who might use aspirin.

Using the whole target population is \emph{impossible}, and a sample must be used.
\end{example}

Only studying one sample out of countless possible samples raises questions:

\begin{itemize}
\tightlist
\item
  \emph{which} individuals should be included in the sample to be studied?
\item
  \emph{how many} individuals should be included in the sample to be studied?
\end{itemize}

The first issue is studied in this chapter. The second issue is studied later (Chap.~\ref{EstimatingSampleSize}), after learning about the implications of studying samples rather than populations.

Many samples are possible, and \emph{every sample is likely to be different}. Hence, the results of studying a sample are likely to vary, depending on which individuals are in the studied sample. The differences between the samples, and differences in the results from each sample, are called \emph{sampling variation}.\index{Sampling variation} That is, each sample has different individuals, produces different data, and may even suggest different answers to the RQ.

\begin{example}[Number of samples]
In a `population' of just~\(100\), the number of possible samples of size~\(25\) is more than twice the number of people currently living on earth.
\end{example}

This is the challenge of research: \emph{making decisions about populations, using just one of the many possible samples}. A lot can be learnt about the population if selecting a sample is approached correctly.\index{Decision making}

\begin{importantBox}{iconmonstr-warning-8-240.png}
Almost always, researchers study \emph{samples},\index{Sample} not \emph{populations}.\index{Population} Many samples are possible, and \emph{every sample is likely to be different}, and the \emph{results from every sample are likely to be different}. This is called \emph{sampling variation}.\index{Sampling variation}

As a result, \emph{conclusions from a sample are never certainties}, though special techniques allow us to still learn about the \emph{population} from a \emph{sample}.

\end{importantBox}

\begin{example}[Sampling variation]
\protect\hypertarget{exm:SamplingVarInCards}{}\label{exm:SamplingVarInCards}Consider a fair pack of cards (a \emph{population}), where \(50\)\% of cards are red. The percentage of red cards is not the same in every hand (every \emph{sample}) of ten cards. This is a simple example of \emph{sampling variation}.\index{Sampling variation}
\end{example}

\clearpage

\section{Precision and accuracy}\label{PrecisionAccuracy}

Two questions concerning sampling in Sect.~\ref{IdeaOfSampling} were: \emph{which} individuals should be in the sample, and \emph{how many} individuals should be in the sample. The first question addresses the \emph{accuracy}\index{Accuracy} of using a sample value to estimate a population value. The second addresses the \emph{precision}\index{Precision} with which a population value is estimated using a sample. An estimate that is not accurate is called \emph{biased} (Def.~\ref{def:Bias}).

\begin{definition}[Accuracy]
\index{Accuracy} \emph{Accuracy} refers to how close a \emph{sample} estimate is likely to be to the \emph{population} value, on average.
\end{definition}

\begin{definition}[Precision]
\index{Precision} \emph{Precision} refers to how similar the sample estimates from different samples are likely to be to each other (that is, how much variation is likely in the sample estimates).
\end{definition}

Using this language:

\begin{itemize}
\tightlist
\item
  the sampling \emph{method} (i.e., \emph{how} the sample is selected) impacts the \emph{accuracy} of the sample estimate (i.e., \emph{external validity}).
\item
  the \emph{size} of the sample impacts the \emph{precision} of the sample estimate (i.e., \emph{internal} validity).
\end{itemize}

Large samples are more likely to produce \emph{precise} estimates, but they may or may not be accurate estimates. Similarly, random samples are likely to produce \emph{accurate} estimates, but they may or may not be \emph{precise}. As an analogy, consider an archer aiming at a target. The shots can be accurate, or precise, or (ideally) both (Fig.~\ref{fig:PrecisionAccuracy}).

\begin{figure}[hbtp]

{\centering \includegraphics[width=0.55\linewidth]{06-ResearchDesign-Sampling_files/figure-latex/PrecisionAccuracy-1} 

}

\caption{Precision and accuracy: each dot indicates where a shot at the target lands, and is like a sample estimate of the population value (shown by the central $\times$).}\label{fig:PrecisionAccuracy}
\end{figure}

\begin{example}[Precision and accuracy]
\protect\hypertarget{exm:PrecisionAccuracyQld}{}\label{exm:PrecisionAccuracyQld}To estimate the average age of \emph{all Canadians}, \(9\,000\) Canadian school children could be sampled.

The answer obtained from the sample will be \emph{inaccurate} because the sample is not representative of \emph{all} Canadians. Since the sample is large, the answer will give a \emph{precise} answer but to a \emph{different} question: `What is the average age of Canadian school children?'
\end{example}

\section{Types of sampling}\label{types-of-sampling}

\index{Sampling}

One key to obtaining accurate estimates about the population from the sample is to ensure that the sample faithfully represents the population. So, \emph{how} is such a sample selected from the population?

The individuals selected for the sample can be chosen using either \emph{random sampling} or \emph{non-random sampling}. The word \emph{random} here has a specific meaning that is different from how it is often used in everyday use. It does \emph{not} mean `haphazard', `erratically' or `picking individuals as aimlessly as I can'.

\begin{definition}[Random]
\protect\hypertarget{def:Random}{}\label{def:Random}\emph{Random} means determined completely by impersonal chance.
\end{definition}

\subsection{Random sampling}\label{RandomSamples}

\index{Sampling!random}

In a \emph{random sample}, both of these statements are true:

\begin{enumerate}
\def\labelenumi{\arabic{enumi}.}
\tightlist
\item
  each individual in the population can be selected.
\item
  each individual is chosen on the basis of \emph{impersonal} chance (such as using a random number generator, or a table of random numbers).
\end{enumerate}

Some examples of random sampling methods appear in Table~\ref{tab:TypesOfRandomSampling}, and are explained further in Sect.~\ref{RandomSamplingMethods}.

\begin{definition}[Random sample]
\protect\hypertarget{def:RandomSampling}{}\label{def:RandomSampling}In a \emph{random} sample, each individual in the population can be selected; and each individual is chosen on the basis of \emph{impersonal} chance.
\end{definition}

\begin{importantBox}{iconmonstr-warning-8-240.png}
The results obtained from a random sample are likely to generalise to the population from which the sample is drawn; that is, \emph{random samples} are likely to produce \emph{externally valid} and \emph{accurate} studies.

\end{importantBox}

\begin{table}
\centering
\caption{\label{tab:TypesOfRandomSampling}Comparing five types of random sampling.}
\centering
\fontsize{8}{10}\selectfont
\begin{tabular}[t]{>{\raggedleft\arraybackslash}p{20mm}>{\raggedright\arraybackslash}p{45mm}>{\raggedright\arraybackslash}p{45mm}>{\raggedleft\arraybackslash}p{10mm}}
\toprule
\textbf{Type} & \textbf{Stage 1} & \textbf{Stage 2} & \textbf{Ref.}\\
\midrule
Simple random & Individuals chosen at \emph{random} &  & \S \ref{SRS}\\
\addlinespace
Systematic & Start at a \emph{random} location & Take every $n$th element thereafter & \S \ref{SystematicSampling}\\
\addlinespace
Stratified & Split into a few large groups (`strata') of similar individuals & Select a \emph{simple random sample} from \emph{every} stratum & \S \ref{StratifiedSampling}\\
\addlinespace
Cluster & Split into many small groups (`clusters'); select a \emph{simple random sample} of clusters & Select \emph{all} individuals in the chosen clusters & \S \ref{ClusterSampling}\\
\addlinespace
Multi-stage & Select a \emph{simple random sample} from the larger collection of units & Select a \emph{simple random sample} from those chosen in Stage 1; etc. & \S \ref{MultistageSampling}\\
\bottomrule
\end{tabular}
\end{table}

A pot of soup can be tested randomly or non-randomly. If the soup is stirred (randomised), the small spoonful of soup can be tasted to obtain an overall impression. However, an \emph{overall} impression is not obtained from a non-random sample (i.e., a non-stirred pot of soup).

\subsection{Non-random sampling}\label{NonRandomSamples}

\index{Sampling!non-random}

A \emph{non-random} sample is selected using personal input from the researchers. Examples include:

\begin{itemize}
\tightlist
\item
  \emph{judgement samples}.\index{Sampling!non-random!judgement} Individuals are selected based on the researchers' judgement (possibly unconsciously), perhaps because the individuals are (or may appear) agreeable, supportive, easily accessible, or helpful. For example, researchers may select rats that are less aggressive, or plants that are accessible, or people that look approachable.
\item
  \emph{convenience samples}.\index{Sampling!non-random!convenience} Individuals are selected because they are convenient for the researcher. For example, researchers may study beaches that are nearby, or use their friends for a study.
\item
  \emph{voluntary response (self-selecting) samples}.\index{Sampling!non-random!voluntary} Individuals participate if they wish to. For example, researchers may ask people to volunteer to take a survey.
\item
  \emph{cherry-picking}.\index{Sampling!non-random!cherry-picking} Individuals are specifically chosen to reach the conclusion that the researchers want.
\end{itemize}

In non-random sampling, the individuals \emph{in} the study are probably different from those \emph{not in} the study. That is, \emph{non-random samples are not likely to be externally valid}.\index{External validity}

Researchers may use a non-random sample intentionally (e.g., to deceive) which is unethical, or unintentionally (e.g., accidentally, or due to practicality (such as meeting budgets)). Ethically, a random (or somewhat representative sample; Sect.~\ref{Representative-samples}) should be used when possible.

\begin{importantBox}{iconmonstr-warning-8-240.png}
Using a non-random sample means that the results probably do not generalise to the intended population: they probably do not produce externally valid or accurate studies.

\end{importantBox}

\section{Methods of random sampling}\label{RandomSamplingMethods}

\index{Sampling!random}

\subsection{Simple random sampling}\label{SRS}

\index{Sampling!random!simple random}

The most straightforward idea for obtaining a random sample is a \emph{simple random sample}.

\begin{definition}[Simple random sample]
\protect\hypertarget{def:SamplingSRS}{}\label{def:SamplingSRS}In a \emph{simple random sample}, \emph{every} possible sample of a given size has the \emph{same} chance of being selected.
\end{definition}

Selecting a simple random sample requires a list of all members of the population, called the \emph{sampling frame}, from which to select a sample. Obtaining the sampling frame is often difficult or impossible, and so finding a simple random sample is also difficult. For example, finding a simple random sample of wombats would require having a list and location of all wombats. This is absurd; other random sampling methods, like special ecological sampling methods (e.g., \citet{manly2014introduction}), would be used instead.

\begin{definition}[Sampling frame]
\protect\hypertarget{def:SamplingFrame}{}\label{def:SamplingFrame}The \emph{sampling frame} is a list of \emph{all} the individuals in the population.\index{Sampling frame}
\end{definition}

Selecting a simple random sample from the \emph{sampling frame} can be performed using \emph{random numbers} (e.g., using random number tables, or websites like \url{https://www.random.org}). Other random sampling methods avoid the need for a sampling frame, but still use randomness rather than human choice.

\begin{importantBox}{iconmonstr-warning-8-240.png}
This book assumes simple random samples, unless otherwise noted.

\end{importantBox}

\begin{example}[Simple random sampling]
\protect\hypertarget{exm:TypingSRS}{}\label{exm:TypingSRS}Consider the letter-typing RQ again (Example~\ref{exm:Typing}, p.~\pageref{exm:Typing}):

\begin{quote}
For students in a large university course, is the average typing speed (in words per minute) the same for those aged under~\(25\) (`younger') and \(25\)~or over (`older')?
\end{quote}

Suppose budget and time constraints mean approximately~\(40\) students (out of~\(441\)) can be selected for the study. The \emph{sampling frame} is the list of all students enrolled in the course. Obtaining the sampling frame is feasible here; instructors have access to this information for grading.

A simple random sample could be found using the course enrolment list, by first placing all \(441\)~student names into rows of a spreadsheet\index{Computers and software!spreadsheets} (ordered by name, student~ID, or any way). Then, using random numbers, \(40\)~rows are selected at random (without repeating numbers) between~\(1\) and~\(441\) inclusive. For instance, when I used \texttt{https://random.org/integers}, the first few random numbers were: \texttt{410}, \texttt{215}, \texttt{384}, \texttt{158}, \texttt{296}.

Every student chosen using this method becomes part of the study. If a student could not be contacted or did not respond, more students could be chosen at random to ensure \(40\) students participated (Fig.~\ref{fig:SamplesA}, left panel). By chance, the sample comprises~\(15\) younger students and \(25\)~older students.
\end{example}

\begin{figure}[hbtp]

{\centering \includegraphics[width=1\linewidth]{06-ResearchDesign-Sampling_files/figure-latex/SamplesA-1} 

}

\caption{A simple random sample (left) and a systematic random sample (right) for obtaining a random sample of size\ $40$ from a class of\ $441$. Triangles $\bigtriangledown$ represent younger students (there are\ $294$), circles $\bigcirc$ represent older students (there are\ $147$), and filled shapes represent those individuals selected in the sample. In the right panel, the boxed individual in the bottom row is the initial, randomly-chosen person (person number nine).}\label{fig:SamplesA}
\end{figure}

\subsection{Systematic sampling}\label{SystematicSampling}

\index{Sampling!random!systematic}

In \emph{systematic sampling}, the first case is \emph{randomly} selected; then, more individuals are selected at regular intervals thereafter. In general, we say that every~\(n\)th individual is selected after the initial random selection.

\begin{example}[Systematic sampling]
\protect\hypertarget{exm:SystematicCourse}{}\label{exm:SystematicCourse}For the study in Example~\ref{exm:Typing}, a sample of \(40\)~students in a course of \(441\) is needed. To find a systematic random sample, select a random number between~\(1\) and~\(441/40\) (approximately~\(11\)) as a starting point; suppose the random number selected is~\(9\) (as in Fig.~\ref{fig:SamplesA}, right panel).

The first student selected is the \(9\)th~person in the student list (which may be ordered alphabetically, by student~ID, or other means). Thereafter, every~\(441/40\)th person, or~\(11\)th person, in the list is selected: people in rows \(9\),~\(20\),~\(31\), \(42\),~and so on (Fig.~\ref{fig:SamplesA}, right panel). By chance, the sample comprises~\(17\) younger students and \(23\)~older students.
\end{example}

\begin{importantBox}{iconmonstr-warning-8-240.png}
Care needs to be taken when using systematic samples to ensure a pattern is not hidden. Consider taking a systematic sample of every \(10\)th residence on a long street. In many countries, odd numbers are usually on one side of the street, and even numbers usually on the other side. Selecting every~\(10\)th house (for example) would include houses all on the same side of the street, and hence with similar exposure to the sun, traffic, etc.

\end{importantBox}

\begin{example}[Systematic sampling]
\protect\hypertarget{exm:SystematicQuebec}{}\label{exm:SystematicQuebec}

\citet{alary1991risk} studied households in Quebec to determine if their hot water systems kept their water sufficiently hot to avoid Legionellae bacteria. They used a systematic random sample to select households to study (p.~\(2\,361\)):

\begin{quote}
The first house was selected by using a random-number table. Thereafter, each fifth house that satisfied the {[}\ldots{]} criteria was eligible for the study.
\end{quote}

\end{example}

\subsection{Stratified sampling}\label{StratifiedSampling}

\index{Sampling!random!stratified}

In \emph{stratified sampling}, the population is split into a \emph{small} number of \emph{large} (usually similar) groups called \emph{strata}, then cases are selected using a \emph{simple random sample} from \emph{each} stratum. Every individual in the population must be in one, and only one, stratum.

\begin{example}[Stratified sampling]
\protect\hypertarget{exm:StratifiedUni}{}\label{exm:StratifiedUni}For the typing study in Example~\ref{exm:Typing}, \(20\)~younger and \(20\)~older students could be selected to obtain a sample of size~\(40\). The sample is stratified by \emph{age group} of the person (Fig.~\ref{fig:SamplesStrat}, left panel).

Since \(66.7\)\% of the students are younger in the population, the sample could be selected so that two-thirds of the sample of size~\(40\) (i.e., \(27\)~students) were younger students (Fig.~\ref{fig:SamplesStrat}, right panel). This is a \emph{proportional} stratified sample.\index{Sampling!proportional}
\end{example}

\begin{figure}[hbtp]

{\centering \includegraphics[width=1\linewidth]{06-ResearchDesign-Sampling_files/figure-latex/SamplesStrat-1} 

}

\caption{Two stratified sampling methods for taking a random sample of size\ $40$ from a class of\ $441$. Left: equal numbers of younger and older students. Right: proportional numbers of younger and older students. Triangles $\bigtriangledown$ represent younger students, circles $\bigcirc$ represent older students, and filled shapes represent those individuals selected in the sample.}\label{fig:SamplesStrat}
\end{figure}

\subsection{Cluster sampling}\label{ClusterSampling}

\index{Sampling!random!cluster}

In \emph{cluster sampling}, the population is split into a \emph{large} number of \emph{small} groups called \emph{clusters}. Then, a \emph{simple random sample} of clusters is selected, and \emph{every} member of the chosen clusters become part of the sample. Every individual in the population must be in one, and only one, cluster.

\begin{example}[Cluster sampling]
For the study in Example~\ref{exm:Typing}, a simple random sample of (say) three of the many small-group classes for the course could be selected, and \emph{every} student enrolled in those selected small groups constitute the sample (Fig.~\ref{fig:SamplesB}, left panel). By chance, the chosen classes produce a sample size of \(n = 47\) (\(31\)~younger; \(16\)~older).
\end{example}

\begin{figure}[hbtp]

{\centering \includegraphics[width=1\linewidth]{06-ResearchDesign-Sampling_files/figure-latex/SamplesB-1} 

}

\caption{Cluster sampling (left) and multi-stage sampling (right) for taking a random sample of size approximately\ $40$. Classes shown bold and shaded represent classes randomly selected to be in the sample in the first stage. Triangles $\bigtriangledown$ represent younger students, circles $\bigcirc$ represent older students, and filled shapes represent those individuals selected in the sample.}\label{fig:SamplesB}
\end{figure}

\subsection{Multi-stage sampling}\label{MultistageSampling}

\index{Sampling!random!multi-stage}

In \emph{multi-stage sampling}, larger collections of individuals are selected using a \emph{simple random sample}, then smaller collections of individuals \emph{within} those large collections are selected using a \emph{simple random sample}. The simple random sampling continues for as many levels as necessary, until individuals are being selected (at random in each step).

\begin{example}[Multi-stage sampling]
For the study in Example~\ref{exm:Typing}, a \emph{simple random sample} of ten of the many small-group classes could be selected (Stage~1), and then four students are \emph{randomly} selected from each of these \(10\) selected small groups (Stage~2) (Fig.~\ref{fig:SamplesB}, right panel). The sample size is \(10\times 4 = 40\), comprising (by chance) \(24\)~younger students and \(16\)~older students.
\end{example}

\begin{example}[Multi-stage sampling]
Multi-stage sampling is often used by national statistical agencies. For example, to obtain a multi-stage random sample from a country:

\begin{itemize}
\tightlist
\item
  \emph{Stage~1}: randomly select some cities in the nation.
\item
  \emph{Stage~2}: randomly select some suburbs in these chosen cities.
\item
  \emph{Stage~3}: randomly select some streets in these chosen suburbs.
\item
  \emph{Stage~4}: randomly select some houses in these chosen streets.
\end{itemize}

This is cheaper than simple random sampling, as data collectors can be deployed in a small number of cities (only those chosen in Stage~1).
\end{example}

\subsection{Comparing the samples}\label{comparing-the-samples}

The different random sampling methods produce different samples, with different proportions of younger and older students by chance (Table~\ref{tab:samplesSummaryTable}). Of course, repeating the random sampling processes would produce different samples each time. In all cases, only \emph{one} of the countless possible samples is studied.

\begin{table}
\centering
\caption{\label{tab:samplesSummaryTable}A summary of the various random samples selected using different random sampling methods. In the population, $66.7$\% of students are younger students. Repeating any random sampling method is likely to produce a different sample each time.}
\centering
\fontsize{8}{10}\selectfont
\begin{tabular}[t]{>{}lcccc}
\toprule
\multicolumn{1}{c}{\textbf{ }} & \multicolumn{3}{c}{\textbf{Number of students selected}} & \multicolumn{1}{c}{\textbf{ }} \\
\cmidrule(l{3pt}r{3pt}){2-4}
\textbf{ } & \textbf{Younger} & \textbf{Older} & \textbf{Total} & \textbf{Percentage younger}\\
\midrule
\textbf{Simple random sample} & $26$ & $14$ & $40$ & $65.0$\\
\textbf{Systematic sample} & $31$ & $\phantom{0}9$ & $40$ & $77.5$\\
\addlinespace
\textbf{Stratified sample: equal} & $20$ & $20$ & $40$ & $50.0$\\
\textbf{Stratified sample: proportional} & $27$ & $13$ & $40$ & $67.5$\\
\addlinespace
\textbf{Cluster sample} & $31$ & $16$ & $47$ & $66.0$\\
\textbf{Multi-stage sample} & $24$ & $16$ & $40$ & $60.0$\\
\bottomrule
\end{tabular}
\end{table}

\section{Representative samples}\label{Representative-samples}

\index{Sampling!representative}

Obtaining a truly random sample is usually hard or impossible in practice. Sometimes the best compromise is to select a sample sufficiently diverse so that it is likely to be \emph{somewhat representative} of the diversity in the population. Specifically, those \emph{in} the sample are not likely to be different from those \emph{not in} the sample, at least for the variables of interest. This is often the only practical way to sample.

\begin{definition}[Representative sample]
\protect\hypertarget{def:RepresentativeSample}{}\label{def:RepresentativeSample}In a \emph{representative} sample, those \emph{in} the sample are not likely to be different from those \emph{not in} the sample, at least for the variables of interest. A representative sample is \emph{not} a random sample.
\end{definition}

As always, the results from any non-random sample \emph{may not generalise} to the intended population (but will generalise to the population which the sample \emph{does} represent).

\begin{example}[Representative sample]
Suppose we wish to evaluate the functionality of two types of hand prosthetics.

If a randomly-chosen group of Alaska and Texas residents is asked for their feedback, probably (but not certainly) their views would be similar to those of all Americans. No obvious reason exists for why residents of Alaska and Texas would be very different from residents in the rest of the United States, regarding their view of hand prosthetic functionality.

Even though the sample is not a random sample of all Americans, the results \emph{may} generalise to all Americans (though we cannot be sure). This sample \emph{may} be representative of the population.
\end{example}

\begin{example}[Non-representative samples]
\protect\hypertarget{exm:AirConUse}{}\label{exm:AirConUse}Suppose we wish to determine the average time per day that Americans households use their air-conditioners for \emph{cooling} in summer.

A sample of Texas residents would not be expected to represent all Americans: it would \emph{over}-represent the average number of hours air-conditioners are used for \emph{cooling} in summer. In this case, those \emph{in} the sample are very different to those \emph{not in} the sample, regarding their air-conditioners usage for cooling in summer.

In contrast, suppose a sample of Alaskans was asked the same question. This sample would not represent all Americans either (it would \emph{under}-represent air-conditioner use). Again, those \emph{in} the sample are likely to be very different to those \emph{not in} the sample, regarding their air-conditioners usage for \emph{cooling} in summer. This sample would not be representative of the population.
\end{example}

Sometimes, a \emph{combination} of sampling methods is used.\index{Sampling!combination of methods} If the combination includes a non-random sampling method, the sampling method does \emph{not} produce a random sample, but is probably more likely to produce an externally valid sample than a completely non-random sample.

\begin{example}[A combination of sampling methods]
In a study of pathogens present on magazines in doctors' surgeries in Dublin, some suburbs can be selected at \emph{random}, and then (within each suburb) all surgeries are contacted, and some surgeries \emph{volunteer} to be part of the study. This study does not use a random sample.
\end{example}

\begin{exampleExtra}

In a study of diets of children at child-care centres, researchers used samples in~2010 and~2016, described as follows \citep[p.~336]{larson2019staff}:

\begin{quote}
In~2010, a stratified random sampling procedure was used to select representative cross-sections of providers working in licensed center-based programs and licensed providers of family home-based care from publicly available lists. {[}\ldots{]} Additional participants were also recruited in~2016 using a combination of stratified random and open, convenience-based sampling.
\end{quote}

\end{exampleExtra}

Sometimes, practicalities dictate how the sample is obtained, which may result in a non-random sample. Even so, the impact of using a non-random sample on the conclusions should be discussed (Chap.~\ref{Interpretation}). Sometimes, simple steps can be taken to obtain a sample that is \emph{more likely} to be representative.

\begin{importantBox}{iconmonstr-warning-8-240.png}
Random samples are often difficult to obtain, and sometimes \emph{representative} samples are the best that can be achieved, In a representative sample, those \emph{in} the sample are not obviously different from those \emph{not in} the sample. Try to ensure that a broad cross-section of the target population appears in the sample.

\end{importantBox}

Even if a random or representative sample cannot be obtained, the study can still be useful. The results still apply to the population represented by the sample. If individuals in the sample are unlikely to be different from individuals \emph{not} in the sample, for the variables important to the study, the results are likely to approximately apply to the population.

\begin{example}[Representative sample]
\protect\hypertarget{exm:RepresentativeUni}{}\label{exm:RepresentativeUni}For the typing study in Example~\ref{exm:Typing}, only selecting students who attend the gym, or only students who are at a certain café, is unlikely to be somewhat representative of the student population. Instead, the researchers could approach students at different days, times and locations:

\begin{itemize}
\tightlist
\item
  at the café on Monday at~\(8\)am.
\item
  at the gym on Tuesday at~\(11\):\(30\)am.
\item
  at the library on Thursdays at~\(2\)pm.
\end{itemize}

\emph{This is not a random sample}, but should contain a variety of students. Ideally, \emph{students would not be included more than once in the sample}, though this is often difficult to ensure. The students \emph{in} the sample are probably somewhat similar to those \emph{not} in the sample in terms of average typing speeds (there is no obvious reason why they would not be), but we cannot be sure.
\end{example}

To determine if the sample is somewhat representative of the population, sometimes information about the sample and population can be compared. The researchers may then be able to make some comment about whether the sample seems reasonably representative. For example, the sex and age of a sample of university students may be recorded; if the proportion of females in the sample, and the average age of students in the sample, are similar to those of the whole university population, then the sample may be considered somewhat representative of the population (though we cannot be sure).

\begin{example}[Comparing samples and populations]
\protect\hypertarget{exm:CFSamplePop}{}\label{exm:CFSamplePop}\citet{egbue2017mass} studied the adoption of electric vehicles (EVs) by Americans, using a sample of \(121\)~people found through social media (such as Facebook) and professional engineering channels. This is \emph{not} a random sample of Americans.

The authors compared some characteristics of the sample with the American population from the \(2010\)~census. Compared to the US population, the sample contained a higher percentage of males, a higher percentage of people aged~\(18\)--\(44\), and a higher percentage of wealthy individuals.
\end{example}

\section{Sampling biases}\label{SelectionBias}

\index{Sampling!bias}

The sample may not be representative of the population for many reasons, all of which compromise how well the sample represents the population (i.e., compromises \emph{external} validity and accuracy). This is called \emph{selection bias}.

\begin{definition}[Selection, non-response and response bias]
\protect\hypertarget{def:SelectionBias}{}\label{def:SelectionBias}\index{Bias!selection}\index{Bias!non-response}\index{Bias!response} \emph{Selection bias} is the tendency of a sample to over- or under-estimate a population quantity.

\emph{Non-response bias} occurs when chosen participants do not respond: those responding may be different from those not responding.

\emph{Response bias} occurs when participants provide \emph{incorrect or false information}.
\end{definition}

Selection bias is less common in studies with forward directionality, compared to studies that are non-directional or have backward directionality (Sect.~\ref{Directionality}). \emph{Selection bias} may occur if the wrong sampling frame is used, or non-random sampling is used. The sample is biased because those \emph{in} the sample may be different from those \emph{not in} the sample (which may not always be obvious). Biased samples are less likely to produce externally valid studies.

\begin{example}[Selection bias]
Consider Example~\ref{exm:AirConUse}, about estimating the average time per day that air conditioners are used for cooling in summer. Even a \emph{random} sample of Alaskans produces a biased sample of Americans: the sampling frame (Alaskans) does not represent the target population (`Americans'). This is \emph{selection bias}.
\end{example}

\emph{Non-response bias} occurs when chosen participants do not respond.\index{Bias!non-response} Bias occurs because those who \emph{do not} respond may be different from those who \emph{do} respond. Non-response bias can occur because of a poorly-designed survey, using voluntary-response sampling, chosen participants refusing to participate, participants forgetting to return completed surveys, etc.

\begin{example}[Non-response bias]
\protect\hypertarget{exm:BiasOvertime}{}\label{exm:BiasOvertime}Consider a study to determine the average number of hours of overtime worked by various professions. People who work a large amount of overtime may be too busy to answer the survey. Those who answer the survey may be likely to work less overtime than those who do not answer the survey. This is an (extreme) example of \emph{non-response bias}.
\end{example}

\emph{Response bias} occurs when participants provide \emph{incorrect or false information}.\index{Bias!response} This may be intentional (for example, respondents lie) or non-intentional (for example, the question is poorly written (see Sect.~\ref{WritingQuestions}), personal, or misunderstood).

\begin{example}[Poor sampling]
\protect\hypertarget{exm:PoorSampling}{}\label{exm:PoorSampling}Obtaining data using a telephone survey only includes people who own a telephone, who answer the phone, who do not hang up, who volunteer to complete the survey, and who then finish the whole survey. The people who participate in the survey must meet these criteria, and probably do not represent the population.

Obtaining data using a TV station call-in at \(6\):\(15\)pm only includes people watching that channel, at that time, and who are sufficiently motivated to call. These people must meet very specific criteria, and probably do not represent the population.

Randomly sampling students at your university, because it is easier than finding a random sample of all university students in your country, will only generalise to students at that university and not to students at \emph{all} universities in your country.
\end{example}

\index{Research design!external validity|)}

\section{Chapter summary}\label{Chap5-Summary}

Almost always, the entire population of interest cannot be studied, so a \emph{sample} (a subset of the population) must be studied. \emph{Many} samples are possible; only one sample is studied. Samples can be obtained using \emph{random} or \emph{non-random} methods. Conclusions made from random samples can usually be generalised to the population (that is, they are externally valid and accurate).

Random sampling methods include \emph{simple random samples}, \emph{systematic samples}, \emph{stratified samples}, \emph{cluster samples}, and \emph{multi-stage samples}. Random samples are likely to be \emph{externally valid} and \emph{accurate}.

Non-random sampling methods include \emph{convenience samples}, \emph{judgement samples}, \emph{voluntary (self-selecting) samples}, and \emph{cherry-picking}. Random samples are often very difficult to obtain, so \emph{reasonably representative} samples are sometimes used, where those \emph{in} the sample are unlikely to be very different from those \emph{not in} the sample. Non-random samples \emph{may not be externally valid} or \emph{accurate}.

\section{Quick review questions}\label{Chap5-QuickReview}

Are the following statements \emph{true} or \emph{false}?

\begin{enumerate}
\def\labelenumi{\arabic{enumi}.}
\item
  Suppose students are randomly selected and sent postal surveys from their university, but some students have moved and so never receive the survey. \tightlist This is \emph{response} bias.
\item
  A \emph{large} sample is \emph{always} better than a \emph{random} sample.
\item
  \emph{Convenience} sampling and \emph{judgement} sampling are examples of non-random sampling.
\end{enumerate}

\section{Exercises}\label{SamplingExercises}

\hyperref[Answers]{Answers to odd-numbered exercises} are given at the end of the book.

\captionsetup{font=small}

\begin{exercise}
\protect\hypertarget{exr:SamplingAdvantageRandom}{}\label{exr:SamplingAdvantageRandom}

What is the main advantage of using a \emph{random} sample?

\begin{enumerate}
\def\labelenumi{\alph{enumi}.}
\tightlist
\item
  It is easier.
\item
  It is more likely to produce an experimental study.
\item
  It is more likely to produce an externally-valid study.
\item
  It is more likely to produce precise estimates.
\end{enumerate}

\end{exercise}

\begin{exercise}
\protect\hypertarget{exr:SamplingAdvantageLarge}{}\label{exr:SamplingAdvantageLarge}

What is the main advantage of using a \emph{large} sample?

\begin{enumerate}
\def\labelenumi{\alph{enumi}.}
\tightlist
\item
  It is easier.
\item
  It is more likely to produce an experimental study.
\item
  It is more likely to produce an externally-valid study.
\item
  It is more likely to produce precise estimates.
\end{enumerate}

\end{exercise}

\begin{exercise}
\protect\hypertarget{exr:SamplingOverUnderA}{}\label{exr:SamplingOverUnderA}

For the following scenarios, is the selected sample likely to \emph{over}- or \emph{under}-estimate the unknown population value, or estimate the value accurately? Explain \emph{why} the over- or -under-estimation occurs, if relevant, and whether this is likely to be intentional or unintentional.

\begin{enumerate}
\def\labelenumi{\arabic{enumi}.}
\tightlist
\item
  In a study by biologists to estimate biodiversity, researchers decide to focus only on easily accessible areas of a forest due to budget and time constraints.
\item
  A city council wishes to report the crime rate of various neighbourhoods, so employs interviewers to go door-to-door interviewing residents, between \(8\)am and \(5\)pm.
\item
  In a campaign speech, a politician reports on some large successes during her term.
\end{enumerate}

\end{exercise}

\begin{exercise}
\protect\hypertarget{exr:SamplingOverUnderB}{}\label{exr:SamplingOverUnderB}

For the following scenarios, is the selected sample likely to \emph{over}- or \emph{under}-estimate the unknown population value, or estimate the value accurately? Explain \emph{why} the over- or -under-estimation occurs, if relevant, and whether this is likely to be intentional or unintentional.

\begin{enumerate}
\def\labelenumi{\arabic{enumi}.}
\tightlist
\item
  A county wants to report the number of homeless, so researchers record data from homeless shelters.
\item
  In a study of soil fertility, a junior researcher takes soil from the surface for testing.
\item
  A university has introduced a complex and time-consuming system for professors to report students suspected of cheating. When the university produces its \emph{Annual Report}, the reported incidence of cheating is used to claim that `reports of cheating have dropped'.
\end{enumerate}

\end{exercise}

\begin{exercise}
\protect\hypertarget{exr:SamplingSystematicProblem}{}\label{exr:SamplingSystematicProblem}

A researcher has three months in which to collect the data for a study on car park usage at a shopping centre. Suppose the researcher wants to take a systematic sample of days, and on each of the selected days records the number of cars in the car park.

To select the days in which to collect data, she decides (by using random numbers) to start data collection on a Tuesday, and then every seventh day thereafter.

\begin{enumerate}
\def\labelenumi{\arabic{enumi}.}
\tightlist
\item
  What problem is evident in this sampling scheme?
\item
  What suggestions would you give to improve the sampling?
\end{enumerate}

\end{exercise}

\begin{exercise}
\protect\hypertarget{exr:SamplingBooks}{}\label{exr:SamplingBooks}Suppose you need to estimate the average number of pages in physical books in a university library (with a library in each of five campuses). Describe how to select a sample of \(200\)~books using:

\begin{cols}

\begin{col}{0.42\textwidth}

\begin{enumerate}
\def\labelenumi{\arabic{enumi}.}
\tightlist
\item
  a \emph{simple random sample} of books.
\item
  a \emph{stratified sample} of books.
\item
  a \emph{cluster sample} of books.
\end{enumerate}

\end{col}

\begin{col}{0.05\textwidth}
~

\end{col}

\begin{col}{0.52\textwidth}

\begin{enumerate}
\def\labelenumi{\arabic{enumi}.}
\setcounter{enumi}{3}
\tightlist
\item
  a \emph{convenience sample} of books.
\item
  a \emph{multi-stage sample} of books.
\end{enumerate}

\end{col}

\end{cols}

Which sampling scheme would be most \emph{practical}?
\end{exercise}

\begin{exercise}
\protect\hypertarget{exr:SamplingApartments}{}\label{exr:SamplingApartments}

Suppose you need a sample of residents from apartments in a large residential complex, comprising \(30\)~floors with \(15\)~apartments on each floor. You plan to survey the residents of these apartments. For each of the possible sampling schemes given below, first describe the sampling scheme, and then determine which methods are likely to give random (or representative) sample (explaining your answers).

\begin{enumerate}
\def\labelenumi{\arabic{enumi}.}
\tightlist
\item
  \emph{Randomly} select five floors, then \emph{randomly} select four apartments from each of those five floors, and interview a randomly-chosen adult living at the apartment.
\item
  \emph{Randomly} select one floor, and select the \(15\) apartments on that floor, then interview the oldest resident of that apartment.
\item
  Wait at the ground-floor elevator, and ask people who emerge to complete the survey.
\item
  \emph{Randomly} select five floors, then wait by the elevator on those floors and survey residents as they arrive at the elevator.
\end{enumerate}

\end{exercise}

\begin{exercise}
\protect\hypertarget{exr:SamplingShoppingCentre}{}\label{exr:SamplingShoppingCentre}

Suppose a researcher needs a sample of customers from a large, local shopping centre to complete a questionnaire. Four sampling schemes are listed below. For each, describe the type of sampling. Then, determine which would be the best method (explain why), and determine which (if any) produce a random sample.

\begin{enumerate}
\def\labelenumi{\arabic{enumi}.}
\tightlist
\item
  The researcher locates themselves outside the supermarket at the shopping centre one morning, and approaches every tenth person who walks past.
\item
  The researcher waits at the main entrance for \(30\,\text{mins}\) at \(8\)am every morning for a week, and approaches every fifth person.
\item
  The researcher leaves a pile of survey forms at an unattended booth in the shopping centre, and a locked barrel in which to place completed surveys.
\item
  The researcher goes to the shopping centre every day for two weeks, at a different time and location each day, and approaches someone every~\(15\,\text{mins}\).
\end{enumerate}

\end{exercise}

\begin{exercise}
\protect\hypertarget{exr:SamplingSchools}{}\label{exr:SamplingSchools}\citet{ridgewell2009school} investigated how children in Brisbane travel to state schools. Researchers randomly sampled four schools from a list of all Brisbane state schools, and invited every family at each of those four schools to complete a survey.

What \emph{type} of sampling method is this? How could the researchers determine if the resulting sample was approximately representative?
\end{exercise}

\begin{exercise}
\protect\hypertarget{exr:SamplingMalaria}{}\label{exr:SamplingMalaria}

A study comparing two new malaria vaccines recruited \(200\) Kenyans who had contracted malaria. These recruits were obtained by approaching all patients with a confirmed malaria diagnosis who were admitted to hospitals. Patients could volunteer for the study or not. The study was then conducted to a high standard. Which of the following statements are \emph{true}?

\begin{enumerate}
\def\labelenumi{\arabic{enumi}.}
\tightlist
\item
  This is a voluntary response sample.
\item
  The study is likely to have high \emph{external} validity.
\item
  The sample size is too small for the study results to provide useful information.
\end{enumerate}

\end{exercise}

\begin{exercise}
\protect\hypertarget{exr:SamplingForest}{}\label{exr:SamplingForest}Suppose a natural forest region is classified into two quite different zones. Zone~A is mostly dunes and lightly vegetated, and on the coastal side of a ridge; Zone~B is more densely vegetated and on the inland side of the ridge.

A sample of sugar ants (\emph{Camponotus app}) is taken from Zone~A, and another sample of sugar ants from Zone~B, to study the average size of the ants. What is the best description of the \emph{type} of sampling method being used?
\end{exercise}

\begin{exercise}
\protect\hypertarget{exr:BiasOnline}{}\label{exr:BiasOnline}A survey in~2001 concluded (\citet{data:Heiger:Homebuyer}, cited in \citet{bock2010stats}, p.~283):

\begin{quote}
All but~\(2\)\% of the home buyers have at least one computer at home, and \(62\)\% have two or more. Of those with a computer, \(99\)\% are connected to the internet.
\end{quote}

The article later reveals the survey was conducted \emph{online} (recall the survey was conducted in~2001). The target population is home buyers; however, home buyers \emph{with} internet access were far more likely to complete the survey than home buyers \emph{without} internet access.

What type of bias is this?
\end{exercise}

\begin{exercise}
\protect\hypertarget{exr:NotMultistageSampling}{}\label{exr:NotMultistageSampling}Researchers are studying the percentage of farms that use a specific management technique. The researchers \emph{randomly} select \(20\)~regions around the country, then \emph{randomly} select farms within each region, then ask farmers to volunteer to be in the study.

Explain why this is \emph{not} a multi-stage sample, and what changes are necessary for the researchers to have a multi-stage sample.
\end{exercise}

\begin{exercise}
\protect\hypertarget{exr:NotClusterSampling}{}\label{exr:NotClusterSampling}Researchers are comparing the average time that experienced school teachers and first-year school teachers spend in the sun. The researchers select schools by asking school principals to volunteer their schools, then record information for \emph{every} teacher in those schools.

Explain why this is \emph{not} a cluster sample, and what changes are necessary for the researchers to have a cluster sample.
\end{exercise}

\begin{exercise}
\protect\hypertarget{exr:SolarSampling}{}\label{exr:SolarSampling}

\citet{walters2018factors} asked this RQ:

\begin{quote}
What factors are preventing the adoption of household solar technologies in Santiago?
\end{quote}

\begin{enumerate}
\def\labelenumi{\arabic{enumi}.}
\tightlist
\item
  For this RQ, what is the \emph{Population}? \tightlist
\item
  The study will be \emph{externally valid} if which of these statements is true?

  \begin{enumerate}
  \def\labelenumii{\alph{enumii}.}
  \tightlist
  \item
    The sample is representative of all households in the world.
  \item
    The sample is representative of all solar technologies.
  \item
    The sample is representative of all households in Santiago.
  \item
    The sample is representative of all households in Chile.
  \end{enumerate}
\item
  Suppose the researchers mail surveys to all households in Santiago, and people return the survey if they wished to. What is the \emph{best} description of this sampling method?
\item
  Suppose the researchers randomly select five suburbs in Santiago; then ten streets within each suburb; then ten households on each street. What is the \emph{best} description of this sampling method?
\end{enumerate}

\end{exercise}

\captionsetup{font=normalsize}

\begin{EOCanswerBox}{iconmonstr-check-mark-14-240.png}
\textbf{Answers to \emph{Quick review} questions:} \textbf{1.} False. \textbf{2.} False (larger: more precise; random: more accurate). \textbf{3.} True.

\end{EOCanswerBox}

\chapter{Internal validity}\label{DesignInternal}

\index{Internal validity}\index{Research design!internal validity|(}

\begin{cols}
\begin{col}{0.52\textwidth}

\begin{objectivesBox}{iconmonstr-target-4-240.png}
So far, you have learnt to ask an RQ, select a study type, and select a sample.
\textbf{In this chapter}, you will learn to:

\begin{itemize}\tightlist
  \item
  maximise the internal validity of studies.
  \item
  manage confounding in studies.
  \item
  explain, identify and manage the Hawthorne, observer, placebo and carryover effect in studies.
  \item
  explain different types of blinding.
\end{itemize}
\end{objectivesBox}

\end{col}

\begin{col}{0.03\textwidth}
~
\end{col}

\begin{col}{0.45\textwidth}

\includegraphics[width=0.95\linewidth]{07-ResearchDesign-Internal_files/figure-latex/unnamed-chunk-7-1} 
\end{col}
\end{cols}

\section{Introduction}\label{Chap7-Intro}

A well-designed study is needed to draw solid conclusions: a study with high \emph{external validity} (Sect.~\ref{def:ExternalValidity}) and high \emph{internal validity} (Sect.~\ref{def:InternalValidity}). This chapter discusses some research design decisions to maximise internal validity.

\begin{example}[Importance of internal validity]
\protect\hypertarget{exm:InternalValidity}{}\label{exm:InternalValidity}\citet{beaman2013profitability} describe an experiment where free fertiliser was provided to a sample of female farmers in Mali (at the recommended rate, or at half the recommended rate).

All farmers knew they were part of a study, so changed their farm management: they employed more hired labour and used more herbicide than usual. Consequently, the yields for \emph{all} farmers improved. Knowing if changes in yield were the result of applying the fertiliser is difficult, as the study had poor \emph{internal validity}.
\end{example}

Specific design strategies for maximising internal validity include:

\begin{itemize}
\tightlist
\item
  managing confounding (Sect.~\ref{ManagingConfounding}).
\item
  managing the Hawthorne effect by blinding individuals (Sect.~\ref{HawthorneEffect}).
\item
  managing the observer effect by blinding the researchers (Sect.~\ref{ObserverEffect}).
\item
  managing the placebo effect by using controls, objective measures and blinding (Sect.~\ref{PlaceboEffect}).
\item
  managing the carryover effect by using washouts (Sect.~\ref{CarryOverEffect}).
\end{itemize}

Not all of these strategies will be relevant to every study.

\section{Managing confounding}\label{ManagingConfounding}

\index{Internal validity!managing confounding}

For this chapter, the following RQ will be used to demonstrate ideas.

\begin{example}[Himalaya study]
\protect\hypertarget{exm:HimalayaStudy}{}\label{exm:HimalayaStudy}

Consider this relational RQ (based on \citet{data:Bird2008:wholegrain}):

\begin{quote}
Among Australians, is the average faecal weight the same for people eating provided food made from wholegrain \emph{Himalaya~292} compared to eating provided food made from refined cereal?
\end{quote}

\end{example}

Suppose that the researchers created two groups of individuals for this experimental study:

\begin{itemize}
\tightlist
\item
  \emph{Group~A}: women recruited from a female-only gym.
\item
  \emph{Group~B}: men recruited from a local nursing home.
\end{itemize}

The researchers gave \emph{Himalaya~292} to Group~A, and the refined cereal to Group~B.\spacex  If a difference in faecal weight was detected between the two groups, many reasons may explain the difference:

\begin{itemize}
\tightlist
\item
  the different \emph{diets} (the explanatory variable in the RQ) for each group.
\item
  the different \emph{sexes} in each group (Group~A was all women; Group~B was all men).
\item
  the different \emph{ages} in each group (Group~A is likely to be younger on average than those in Group~B).
\item
  the different \emph{overall health} in each group (Group~A would generally be healthier than those in Group~B).
\end{itemize}

Any difference in faecal weight detected between the two groups may not be due to the diets (Table~\ref{tab:ConfoundingGroups}): the study has very poor internal validity, due to poor research design.

Sex, age and overall health are \emph{confounding variables} (Def.~\ref{def:ConfoundingVariable}): they are associated with the type of diet (explanatory variable) \emph{and} faecal weight (response variable).\index{Variables!confounding} For example, the age of the subject may be associated with faecal weight (older people tend to eat less, and eat differently, than younger people), and the research design means older people are more likely to be consuming the refined cereal. This is an extreme case of \emph{confounding} (Fig.~\ref{fig:ConfoundingDiagram}); usually, confounding is more subtle (and more difficult to detect) than in this example.

\begin{figure}
\begin{minipage}{0.37\textwidth}
\captionof{table}{Comparing Groups\ A and\ B: extreme confounding\label{tab:ConfoundingGroups}.}
\fontsize{8}{12}\selectfont
\begin{@empty}

\begin{tabular}{rcl}
\toprule
\textbf{Group A} & \textbf{Variable} & \textbf{Group B}\\
\midrule
Women & \textbf{Sex} & Men\\
Younger & \textbf{Age} & Older\\
\textit{Himalaya 292} & \textbf{Cereal} & Refined\\
Fitter & \textbf{Fitness} & Less fit\\
\bottomrule
\end{tabular}
\end{@empty}
\end{minipage}
%% Add a gap between elements
\hspace{0.08\textwidth}
%%%
\begin{minipage}{0.55\textwidth}%

\includegraphics[width=0.95\linewidth]{07-ResearchDesign-Internal_files/figure-latex/unnamed-chunk-5-1} 
\captionof{figure}{An extreme example of confounding\label{fig:ConfoundingDiagram}.}
\end{minipage}
\end{figure}

\begin{importantBox}{iconmonstr-warning-8-240.png}
\emph{The groups being compared should be as similar as possible}, apart from the difference being studied.

\end{importantBox}

Since the groups being compared should be as similar as possible, apart from what is being studied, researchers often compare the comparison groups on potential confounding variables (e.g., the average age of people in each comparison group).

In \emph{experimental} studies, an excellent way to manage confounding is:

\begin{enumerate}
\def\labelenumi{\arabic{enumi}.}
\tightlist
\item
  \emph{Randomly allocating} individuals to the comparison groups.\index{Confounding!analysis}
\end{enumerate}

Random allocation should ensure that the values of potential confounding variables are approximately evenly distributed between the comparison groups.\index{Confounding!random allocation} This is true for identified potential confounders (such as age), but also for unidentified potential confounders, or variables that are hard to measure or observe (e.g., genetic conditions). One of the comparison groups is often a control group (Def.~\ref{def:Control}).

\begin{example}
\protect\hypertarget{exm:RandomAllocationThirst}{}\label{exm:RandomAllocationThirst}\citet{lian2024effect} studied alleviating post-operative thirst experienced by patients admitted to the intensive care unit. They compared standard procedures with the use of ice-water spray. To use random allocation of patients to the two groups, the researchers:

\begin{quote}
\ldots{} assigned unique numbers from~\(1\) to~\(56\) according to {[}students'{]} admission order {[}\ldots{]} two-digit numbers were read from the random number table's rows and columns, generating random values that were matched with the respective admission numbers {[}\ldots{]}
\end{quote}

Any student assigned a number between~\(1\) to~\(28\) (inclusive) was allocated to the control group, while students assigned numbers~\(29\) to~\(56\) were assigned to the experimental group.
\end{example}

\begin{example}
\protect\hypertarget{exm:RandomAllocationControlgroup}{}\label{exm:RandomAllocationControlgroup}\citet{witmer2020preliminary} studied using bear faeces to prevent bears damaging trees in an Idaho forest. The researchers painted the bear faeces on sample of trees. As a control, researchers could take observations from trees that they had not approached, and hence had no bear faeces applied. However, if a difference was found between the trees with bear faeces and trees they had not approached, the difference may have been due to the presence of humans near the trees rather than the treatment (i.e., poor internal validity).

For this reason, the control group\index{Control group} comprised trees on which the researchers applied water.\index{Control} This is a better control, since trees in both groups (faeces; water) had been approached by humans. Now, if a difference was found between the faeces and water-sprayed trees, the presence-of-humans explanation has been eliminated.
\end{example}

\emph{Randomly allocating} individuals to comparison groups is \emph{not possible} in observational or quasi-experimental studies. For this reason, confounding is often a major threat to internal validity in these studies, as individuals who are in one comparison group may be different, in general, to those who are in another group.

Fortunately, other (though less effective) means for managing confounding also exist.

\begin{enumerate}
\def\labelenumi{\arabic{enumi}.}
\setcounter{enumi}{1}
\tightlist
\item
  \emph{Restricting} the study to a certain subgroup of the population.\index{Confounding!restricting}
\end{enumerate}

Sometimes, specifically excluding or including members of the population is helpful for reducing confounding. For the \emph{Himalaya~292} study, for example, age is a potential confounder: older people have different dietary needs, general health and gut health when compared to younger people. Hence, the researchers may decide to use an \emph{inclusion criterion},\index{Inclusion criteria} restricting the study to people aged from~\(30\) to~\(50\).

In addition, some people may have specific conditions or diseases that mean participating in the study will be problematic. For instance, coeliacs have an autoimmune disorder which results in a severe intolerance to gluten (found in wheat, barley and rye). Hence, the researchers may decide to use \emph{exclusion criteria}\index{Exclusion criteria}, excluding coeliacs from participating in the study. Those individuals that are excluded from the population are not less important than those individuals that are included.

Inclusion and exclusion criteria may be applied for other reasons too; for example, to clarify a population\index{Population!refining} of interest, to address ethical concerns\index{Ethics} (i.e., by excluding children) or to exclude rare and unusual individuals.

\begin{definition}[Inclusion and exclusion criteria]
\protect\hypertarget{def:InclusionExclusionCriteria}{}\label{def:InclusionExclusionCriteria}\index{Inclusion criteria}\index{Exclusion criteria} \emph{Inclusion criteria} are characteristics that individuals must meet explicitly to be included in the study.

\emph{Exclusion criteria} are characteristics that explicitly disqualify potential individuals from being included in the study.
\end{definition}

Exclusion and inclusion criteria clarify which individuals are explicitly included or excluded from the population for the purposes of the study, and their use should be explained when their purpose is not obvious. Exclusion and inclusion criteria are not both necessary; none, one or both may be used. These are a type of \emph{control variable} (Def.~\ref{def:ControlVariables}).\index{Confounding!control variables}

\begin{example}[Inclusion and exclusion criteria]
\protect\hypertarget{exm:ExclusionCriteriaEG}{}\label{exm:ExclusionCriteriaEG}In a strength study where the population is `concrete test cylinders', cylinders with severe cracks may be \emph{excluded}.

In a study of exercise regimes for people over~\(60\), severe asthmatics may be \emph{excluded} from the study for health reasons.
\end{example}

\begin{example}[Inclusion, exclusion criteria]
\protect\hypertarget{exm:InclusionBodyTemp}{}\label{exm:InclusionBodyTemp}\citet{data:mackowiak:bodytemp} studied men and women aged~\(18\) to~\(40\); this is the \emph{population}. The \emph{exclusion} criteria include people under~\(18\) years of age and over~\(40\) years of age; alternatively, the \emph{inclusion} criteria are people aged between~\(18\) and~\(40\) years of age. Either of these can be stated; both are not needed.
\end{example}

\begin{exampleExtra}

In a study on the influenza vaccine, \citet{kheok2008efficacy} listed the \textbf{P}opulation as `health-care workers' \citep[p.~466]{kheok2008efficacy}, and the sample comprised healthcare workers at two specific hospitals. The population was refined using exclusion criteria: those (p.~466)

\begin{quote}
\ldots declining to give consent, a history of egg protein allergy, and neurological or immunological conditions that are contraindications to the influenza vaccine.
\end{quote}

\end{exampleExtra}

\begin{example}[Inclusion and exclusion criteria]
\protect\hypertarget{exm:ExclusionAmoutees}{}\label{exm:ExclusionAmoutees}

\citet{data:Guirao2017:amputees} studied the walking abilities of amputees. Inclusion criteria included (p.~27):

\begin{quote}
\ldots{} length of the femur of the amputated limb of at least~\(15\,\text{cm}\) measured from the greater trochanter; use of the prosthesis for at least~\(12\) months prior to enrollment and more than~\(6\,\text{h}\)/day\ldots{}
\end{quote}

Exclusion criteria included (p.~27) people with:

\begin{quote}
\ldots{} cognitive impairment hindering the ability to follow instructions and/or perform the tests; body weight over~\(100\,\text{kg}\)\ldots{}
\end{quote}

\end{example}

\begin{enumerate}
\def\labelenumi{\arabic{enumi}.}
\setcounter{enumi}{2}
\tightlist
\item
  \emph{Blocking},\index{Confounding!blocking} when units of analysis are arranged into different groups containing individuals that are similar to each another (see Sect.~\ref{PairedInvasivePlants} for an example).
\end{enumerate}

For the \emph{Himalaya~292} study, for example, subjects may be \emph{paired} (i.e., groups of two).\index{Study types!paired} That is, each person is paired with another person of the same sex and of a similar age and weight; one of each pair is given the \emph{Himalaya~292} diet, and the other is given the refined cereal diet. Each pair is called a \emph{block}.

\begin{definition}[Blocking]
\protect\hypertarget{def:Blocking}{}\label{def:Blocking}\emph{Blocking} occurs when units of analysis are analysed as separate groups of similar units (called \emph{blocks}).
\end{definition}

\begin{enumerate}
\def\labelenumi{\arabic{enumi}.}
\setcounter{enumi}{3}
\tightlist
\item
  \emph{Analysing} using special methods (beyond this book), after recording the values of potential confounding variables.\index{Confounding!analysis}
\end{enumerate}

To use this approach, \emph{recording all potential extraneous variables} is important. Most studies involving people record the participants' age and sex if possible, as these two variables are common confounders. Once a sample is obtained, recording this extra information usually requires little extra effort. Then, these extraneous variables can be included in the analysis.

\emph{Restricting} and \emph{blocking} are useful if one or two confounding variables are suspected. Multiple approaches can be used, such as randomly allocating individuals to groups, \emph{and} recording other variables that can be managed through analysis.

\emph{Randomly allocating} is superior when possible, because confounding is reduced for variables not even suspected as being confounders. Hence, \emph{experimental} studies should use random allocation whenever possible.

For any study (but especially for observational and quasi-experimental studies), recording the values of any potential confounding variables is useful, so that special analysis methods can be used to manage confounding.

\begin{importantBox}{iconmonstr-warning-8-240.png}
\emph{Record} all the extraneous variables likely to be important (Sect.~\ref{RecordExtraneous}). This may include information about the \emph{individuals} in the study, and the \emph{circumstances} of the individuals in the study (that is, the circumstances the individuals find themselves in; these may not be measured on the individuals themselves).

\end{importantBox}

\begin{example}[Managing confounding: experimental study]
\protect\hypertarget{exm:HimalayaConfounding}{}\label{exm:HimalayaConfounding}For the \emph{Himalaya} study, different methods can be used to manage confounding due to age.

The study could be \emph{restricted} to people under~\(30\). Age would be a \emph{control variable}.

\emph{Blocking} could be used by finding similar pairs of subjects (e.g., pairs of subjects of the same sex, with similar age and weight). One of each pair is given the refined cereal diet, and one given the \emph{Himalaya~292} diet. The \emph{differences} in faecal weight for each pair can be analysed using special methods (see Chap.~\ref{AnalysisPaired} for example).

Information \emph{about the individuals} could be recorded, such as age and pre-study weight. Information \emph{about the circumstances} of the individuals could also be recorded, such as where they live. Then, special methods of \emph{analysis} could be used to analyse the data.

Since the study is experimental, participants could be \emph{randomly allocated} into one of two groups, so both groups would have a similar distribution of ages (and other potential confounders). Then groups could be randomly allocated to receive one of the diets (Fig.~\ref{fig:RandomAllocationHimalaya}).

In the \emph{Himalaya~292} study, individuals were randomly allocated to the diets (p.~\(1\,033\)), which manages confounding due to age and other potential confounding variables also.
\end{example}



\begin{figure}[hbtp]

{\centering \includegraphics[width=0.65\linewidth]{07-ResearchDesign-Internal_files/figure-latex/RandomAllocationHimalaya-1} 

}

\caption{Random allocation can occur in two places for the \emph{Himalaya} study.}\label{fig:RandomAllocationHimalaya}
\end{figure}

\begin{exampleExtra}

An experiment to study the effect of using ginkgo to enhance memory \citep{data:Solomon2002:ginkgo} compared two groups: one using ginkgo (\(n = 111\)), and one using a fake, non-active supplement (\(n = 108\)). The authors randomly allocated participants to each group, then compared the two groups to ensure that no obvious differences initially existed between the groups that might explain differences in the response variable (Table~\ref{tab:ginkgoDemographics}).

Two groups are similar in terms of age, education and gender distribution. Any difference in outcome between the groups is probably due to the treatment.

\begin{table}
\centering
\caption{\label{tab:ginkgoDemographics}Comparing the two groups in the ginkgo-memory study.}
\centering
\fontsize{8}{10}\selectfont
\begin{tabular}[t]{rcc}
\toprule
\textbf{Characteristic} & \textbf{Group A (ginkgo)} & \textbf{Group B (Fake)}\\
\midrule
Average age (in years) & 68.7 & 69.9\\
Men (number; percentage) & 46 (41\%) & 45 (42\%)\\
Average years of education & 14.4 & 14.0\\
\bottomrule
\end{tabular}
\end{table}

\end{exampleExtra}

\begin{exampleExtra}
Researchers explored the use of dominant and non-dominant hands for chest compression in student paramedics using an experimental study \citep{cross2019impact}. Students were randomly divided into two groups: DHOS (dominant hand on chest) and NDHOC (non-dominant hand on chest). The two groups were then compared:

\begin{longtable}[]{@{}
  >{\raggedleft\arraybackslash}p{(\linewidth - 6\tabcolsep) * \real{0.2990}}
  >{\centering\arraybackslash}p{(\linewidth - 6\tabcolsep) * \real{0.3196}}
  >{\centering\arraybackslash}p{(\linewidth - 6\tabcolsep) * \real{0.1959}}
  >{\centering\arraybackslash}p{(\linewidth - 6\tabcolsep) * \real{0.1856}}@{}}
\toprule\noalign{}
\begin{minipage}[b]{\linewidth}\raggedleft
Demographic
\end{minipage} & \begin{minipage}[b]{\linewidth}\centering
All participants (\(n = 75\))
\end{minipage} & \begin{minipage}[b]{\linewidth}\centering
DHOC (\(n = 37\))
\end{minipage} & \begin{minipage}[b]{\linewidth}\centering
NDHOC (\(n = 38\))
\end{minipage} \\
\midrule\noalign{}
\endhead
\bottomrule\noalign{}
\endlastfoot
Average age (years) & \(23.4\) & \(22.5\) & \(24.3\) \\
Gender: percentage Female & \(51\)\% & \(53\)\% & \(47\)\% \\
\end{longtable}

The two groups appear to be very similar in terms of average age of participants, and the percentage of female participants. If differences are observed in the study between the DHOC and NDHOC groups, it is probably due to the treatment. The study should have reasonable internal validity.

\end{exampleExtra}

\begin{example}[Managing confounding: observational study]
\protect\hypertarget{exm:ConfoundingKiwi}{}\label{exm:ConfoundingKiwi}\citet{data:froud2018:kiwifruit} studied \(2\,599\)~kiwifruit orchards using an observational study, exploring the relationship between the time since a bacterial canker was first detected (in weeks) as the explanatory variable, and the orchard productivity (in tray-equivalents per hectare) as the response variable.

The researchers also recorded potential extraneous variables such as `whether the farm was organic', `elevation of the orchard' and `whether general fungicides were used'. These variables were used in their analysis to manage the potential effects of confounding.
\end{example}

\begin{example}[Comparing study groups: observational study]
\protect\hypertarget{exm:ActiveSedentaryWomen}{}\label{exm:ActiveSedentaryWomen}An observational study compared the iron levels of active and sedentary women aged~\(18\) to~\(35\) \citep{data:woolf:ironstatus}. The active women (\(n = 28\)) and sedentary women (\(n = 28\)) were compared on a variety of characteristics (Table~\ref{tab:SedentaryDemographics}). The active women were similar to the sedentary women on these characteristics, but were (in general) slightly younger, slightly heavier, and slightly more likely to use hormonal contraceptives.
\end{example}

\begin{table}
\centering
\caption{\label{tab:SedentaryDemographics}The demographic information for those in the study of iron levels in women.}
\centering
\fontsize{8}{10}\selectfont
\begin{tabular}[t]{rcc}
\toprule
\textbf{Characteristic} & \textbf{Active women} & \textbf{Sedentary women}\\
\midrule
Average age (in years) & $20$ & $24$\\
Average weight (in kg) & $68$ & $62$\\
Percentage using hormonal contraceptives & $13$ & $11$\\
\bottomrule
\end{tabular}
\end{table}

\begin{exampleExtra}
A study \citep{data:Gunnarsson2017:helicopter} examined the difference between two types of helicopter transfer (physician-staffed; non-physician-staffed) of patients with a specific type of myocardial infarction (\textsc{stemi}). The purpose of the study was:

\begin{quote}
\ldots to evaluate the characteristics and outcomes of physician-staffed \textsc{hems} (Physician-HEMS) versus non-physician-staffed (Standard-\textsc{hems}) in patients with STEMI.

\VA{--- \citet{data:Gunnarsson2017:helicopter}, p.~1}{}
\end{quote}

The researchers

\begin{quote}
\ldots studied \(398\) \textsc{stemi} patients transferred by either Physician-\textsc{hems} (\(n = 327\)) or Standard-\textsc{hems} (\(n = 71\)) for {[}\ldots{]} intervention at \(2\)~hospitals between~2006 and~2014.

\VA{--- \citet{data:Gunnarsson2017:helicopter}, p.~1}{}
\end{quote}

Since the study is an observational study (patients were not allocated by the researchers to the type of helicopter transport), the researchers recorded information about the patients being transported. They compared the patients in both groups, and found (for example) that both groups had similar average ages, and similar percentages of females and smokers, and so on. They also compared information about the transportation, and found (for example) that both groups had similar average flight times and flight distances.

One conclusion from the study was that `Patients with \textsc{stemi} transported by Standard-\textsc{hems} had longer transport times' (p.~1), but one limitation of the study was that:

\begin{quote}
The patient cohorts received treatment by~\(2\) different care teams at two hospitals, which is a potential confounder despite similar baseline characteristics

\VA{--- \citet{data:Gunnarsson2017:helicopter}, p.~5}{}
\end{quote}

In other words, the difference between hospitals and the staff may have been a confounding variable.

\end{exampleExtra}

\begin{importantBox}{iconmonstr-warning-8-240.png}
Observational studies \emph{can} (and often do) have control groups. Indeed, one specific type of observational study is called a \emph{case-control study}\index{Study types!case-control studies} (Sect.~\ref{Backward}). However, individuals are \emph{not allocated to the control group} by the researchers in observational studies, so initially the control and study groups may be very different, which may explain any differences in the outcome.

\end{importantBox}

Random \emph{sampling} and random \emph{allocation}\index{Confounding!random allocation}\index{Sampling} are different concepts (Fig.~\ref{fig:RandomAllocationSampling}) with different purposes, but are often confused.

\begin{itemize}
\tightlist
\item
  \emph{Random sampling} impacts \emph{external} validity. Its purpose is \emph{finding individuals} to study, and is possible in both observational and experimental studies.
\item
  \emph{Random allocation} helps eliminate confounding, by distributing possible confounders across treatment groups, and is only possible in \emph{experimental} studies. \emph{Random allocation} impacts \emph{internal} validity. Its purpose is \emph{allocating treatments to individuals}, which does not occur in observational studies.
\end{itemize}

\begin{figure}[hbtp]

{\centering \includegraphics[width=0.75\linewidth]{07-ResearchDesign-Internal_files/figure-latex/RandomAllocationSampling-1} 

}

\caption{Comparing random allocation and random sampling.}\label{fig:RandomAllocationSampling}
\end{figure}

\section{Hawthorne effect and blinding individuals}\label{HawthorneEffect}

\index{Hawthorne effect}\index{Blinding!individuals}

People, and perhaps animals, may behave differently if they know (or think) they are being watched, which could compromise the internal validity of the study. This is called the \emph{Hawthorne effect}.

\begin{definition}[Hawthorne effect]
\protect\hypertarget{def:HawthorneEffect}{}\label{def:HawthorneEffect}The \emph{Hawthorne effect} is the tendency of individuals to change their behaviour if they know (or think) they are being observed.
\end{definition}

\begin{example}[Hawthorne effect: observational study]
\protect\hypertarget{exm:HawthorneHH}{}\label{exm:HawthorneHH}\citet{wu2018identifying} examined hand hygiene (HH) of staff in a tertiary teaching hospital, using \emph{covert} (secret) observers and \emph{overt} (obvious) observers. HH compliance was higher with overt observation (\(78\)\%) than with covert observation (\(55\)\%).
\end{example}

The impact of the Hawthorne effect can be minimised by blinding the individuals, so that:

\begin{itemize}
\tightlist
\item
  the individuals do not know that they are \emph{participating} in a study.
\item
  the individuals do not know the \emph{aims of the study}.
\item
  the individuals do not know which \emph{comparison group they are in}.
\end{itemize}

Any or all of these may be true, depending on the study design. Blinding individuals in all three of these ways is not always possible.

In \emph{experimental} studies, \emph{people} are often informed that they are in a study, due to ethics requirements (Sect.~\ref{Common-Ethical-Issues}); they may not, however, know \emph{which} treatment they have received. In \emph{observational} studies, individuals \emph{may} or \emph{may not} know they are being observed. For instance, in a study where subjects' blood pressure is measured, subjects clearly know they are being observed, which has the potential to alter the subjects' behaviour (e.g., people become tense, called `white-coat hypertension'). As far as possible, efforts should be made to ensure that individuals do not know that they are being observed (the participants are \emph{blinded}).

\begin{example}[Hawthorne effect: experimental study]
\protect\hypertarget{exm:HawthorneHimalaya}{}\label{exm:HawthorneHimalaya}For the \emph{Himalaya} study (Example~\ref{exm:HimalayaStudy}), the article reports that (p.~\(1\,033\)):

\begin{quote}
The study was explained fully to the subjects, both verbally and in writing, and each gave their written, informed consent\ldots{}
\end{quote}

That is, the subjects knew they were in a study, and knew the aims of the study, so the Hawthorne effect may influence the results in this study. However, the subjects did not know \emph{which} diet they were given.
\end{example}

\begin{example}[Hawthorne effect: experimental study]
\protect\hypertarget{exm:HawthorneFruitVege}{}\label{exm:HawthorneFruitVege}People are more health-conscious if they know they will be examined regularly. For example, a study aiming to increase fruit and vegetable intake in young adults \citep{clark2019educational} noted that the observed increases in intake `could be explained by the Hawthorne effect' as adults `know they are being observed\ldots{}' (p.~96).
\end{example}

\begin{example}[Hawthorne effect: observational study]
\protect\hypertarget{exm:HawthorneHealth}{}\label{exm:HawthorneHealth}During the \textsc{covid}-19 lockdowns in Denmark, \citet{olesen2021we} covertly observed adults entering a large mall in Copenhagen. They noticed that (p.~1)

\begin{quote}
Almost all subjects {[}\(340/345\) (\(99\)\%){]} wore a personal protective face mask, but only~\(141\) (\(41\)\%) made use of the hand sanitizer.
\end{quote}

Both masks and hand sanitiser were recommended by the Danish Health Authority, but the adherence to the safety measures were very different. The authors surmised (p.~1):

\begin{quote}
\ldots{} wearing a face mask corresponded to being observed continuously {[}\ldots{]} hand hygiene takes moments to perform, and no one can see whether or not it has been done.
\end{quote}

In other words, wearing a face mask is obvious (that is, others could \emph{observe} whether the subjects was adhering to this guideline) but hand hygiene is not (so other people \emph{could not} observe whether the subject was adhering to this guideline). The authors conclude that `the Hawthorne effect may explain why almost all subjects wore a face mask'.
\end{example}

\section{Observer effect and blinding researchers}\label{ObserverEffect}

\index{Observer effect}\index{Blinding!researchers}

Perhaps surprisingly, researchers' expectations or hopes may unconsciously influence how the researchers interact with the individuals and record observations. In addition, this may (unconsciously) influence the behaviour of the individuals in the study. This is called \emph{observer effect}. (In experiments, the observer effect is sometimes called the \emph{experimenter effect}.) This could compromise the internal validity of the study.

\begin{definition}[Observer effect]
\protect\hypertarget{def:ObserverEffect}{}\label{def:ObserverEffect}The \emph{observer effect} occurs when the researchers unconsciously change their behaviour to conform to expectations because they know what values of the explanatory variable apply to the individuals. This may then cause the \emph{individuals} to change their behaviour or reporting also.
\end{definition}

The impact of the observer effect can be minimised by blinding the \emph{researchers}, so that they do not know which treatments the individuals are receiving. The researchers \emph{giving} the treatment and the researchers \emph{evaluating} the treatment can both be blinded, by using a third party. For example, the researchers may give an assistant two drugs, labelled~A and~B.\spacex The assistant administers the drug and evaluates the participants' response to the treatments. Later, the assistant tells the researchers whether Drug~A or Drug~B performed better, but only the researchers know which drugs the labels~A and~B refer to (Fig.~\ref{fig:BlindingThirdParty}).

\begin{figure}[hbtp]

{\centering \includegraphics[width=0.85\linewidth]{07-ResearchDesign-Internal_files/figure-latex/BlindingThirdParty-1} 

}

\caption{Using a third party to avoid the observer effect.}\label{fig:BlindingThirdParty}
\end{figure}

\begin{example}[Observer effect: experimental study]
\protect\hypertarget{exm:ObsEffectyPain}{}\label{exm:ObsEffectyPain}

\citet{seo2020role} examined the impact of an injection to alleviate post-operative umbilical pain, and stated (p.~392):

\begin{quote}
\ldots the postoperative pain scores were gathered by a nurse practitioner who was blinded to the usage of bupivacaine to avoid observer-expectancy bias {[}i.e., the observer effect{]}.
\end{quote}

\end{example}

The observer effect does not just apply to situations with \emph{people} as individuals.

\begin{example}[Observer effect]
\protect\hypertarget{exm:CleverHans}{}\label{exm:CleverHans}Clever Hans was a horse that seemed to perform simple mental arithmetic. By using an experiment where the people interacting with the horse were blinded, Carl Stumpf realised that the horse was responding to involuntary (and unconscious) cues from the trainer.

The same effect has been observed in narcotic sniffer dogs \citep{bambauer2012defending}, who may respond to their handlers' unconscious cues.
\end{example}

\begin{importantBox}{iconmonstr-warning-8-240.png}
The \emph{observer effect} is when the researcher \emph{unconsciously} influence the individuals, and are not aware it is occurring. \emph{Intentionally} influencing the individuals is fraud.

\end{importantBox}

The observer effect can impact both observational and experimental studies. For example, consider a study measuring the blood pressure of smokers and non-smokers \citep{verdecchia1995cigarette}. This study is observational (individuals cannot be allocated to be a smoker or non-smoker), but if the researchers \emph{know} an individual is a smoker when they measure blood pressure, then the observer effect could impact the results (recalling that the observer effect is an \emph{unconscious} effect). For example, the researchers may \emph{expect} smokers to have a high blood pressure.

The observer effect could be managed by \emph{first} measuring the blood pressure, and \emph{then} asking if the individual was a smoker or not. That is, the researchers may be \emph{blinded} to whether the subject is a smoker when they measure blood pressure. This may only be partially successful; the researcher may see the subject carrying cigarettes, or can smell smoke on their breath, for example. Nonetheless, since it may prove at least partially successful and is easy to implement, this strategy should form part of the research design.

\begin{example}[Observer effect: observational study]
\protect\hypertarget{exm:ObsEffectObs}{}\label{exm:ObsEffectObs}

\citet{zimova2020using} took photos of snowshoe hares, at various stages of moulting and in various environmental conditions. Eighteen independent observers rated the moult stage from the photographs (p.~4):

\begin{quote}
\ldots{} images were randomly named and sorted, with the dates {[}\ldots{]} removed to minimize observer expectancy bias {[}i.e., the observer effect{]}.
\end{quote}

\end{example}

Blinding the observer is not always possible, but should be used when possible to improve the internal validity of the study.

\begin{exampleExtra}
A study of the scats of gray wolves was used to study their diet \citep{SpauldingScats}. A scat analysis is where humans examine the scat of carnivores to determine the prey. However, the accuracy of the results was questioned, due to `perpetuation of the assumption that wolf scats contain only~\(1\)~prey item/scat' (p.~949).

The observers might be seeing what they expect to see: that ``wolf scats contain only \(1\)~prey item/scat''.

\end{exampleExtra}

\section{Placebo effect, controls, objective data, and blinding}\label{PlaceboEffect}

\index{Placebo effect}\index{Blinding!individuals}\index{Control}

Perhaps surprisingly, individuals in a study may report effects of a treatment, even if they have not received an active treatment. This could compromise the internal validity of the study. This is called the \emph{placebo effect}, which generally only impacts people as individuals.

\begin{definition}[Placebo effect]
\protect\hypertarget{def:PlaceboEffect}{}\label{def:PlaceboEffect}The \emph{placebo effect} occurs when individuals report perceived or actual effects, despite not receiving an active treatment.
\end{definition}

For example, people who attend therapy expect a positive outcome; this expectation may result in temporary or perceived (or sometimes even real) improvements in their condition. This is the placebo effect.

To manage the placebo effect, researchers should record \emph{objective}\index{Response variable}\index{Data!objective} data (Sect.~\ref{RecordObjectiveData}) rather than patient-reported (subjective) outcomes when possible \citep{enck2013placebo}. (The operational definitions (Sect.~\ref{OperationDefinitions})\index{Definitions!operational} for the variables should make clear whether subjective or objective data are recorded.) Using a \emph{control} group (Def.~\ref{def:Control}), if possible, is also useful: it acts as a benchmark for detecting changes in the outcome due to the treatment of interest. In addition, \emph{blinding} the individuals and the researchers may help manage the placebo effect, as then the individuals cannot know which group they are in.

\begin{example}[Placebo effect]
\protect\hypertarget{exm:PlaceboColours}{}\label{exm:PlaceboColours}Three active pain relievers were compared to different-coloured placebo \citep{data:Huskisson1974:placebo} in \(22\)~patients. The most pain relief was experienced by those taking \emph{red} placebos (Fig.~\ref{fig:Placebos}), who experienced even more pain relief than those given true pain relievers. Note that the outcome is subjective:\index{Data!subjective} a \emph{patient}-reported outcome.\index{Response variable}\index{Data!subjective}
\end{example}

\begin{figure}[hbtp]

{\centering \includegraphics[width=0.65\linewidth]{07-ResearchDesign-Internal_files/figure-latex/Placebos-1} 

}

\caption{Pain relief, for various pain relief medicine and different-coloured placebos.}\label{fig:Placebos}
\end{figure}

Since the placebo effect is concerned with individual responses to \emph{allocated treatments}, it is not directly relevant to observational studies.

\begin{example}[Placebo effect]
\protect\hypertarget{exm:HimalayaPlacebo}{}\label{exm:HimalayaPlacebo}In the \emph{Himalaya} study, the individuals `were not told the identity of the test cereal in the foods provided' (\citet{data:Bird2008:wholegrain}, p.~\(1\,033\)). The subjects were blinded to the diet they were exposed to. However, some may \emph{think} they are on the refined cereal or \emph{Himalaya} diet, and respond accordingly (perhaps unconsciously). The use of the refined cereal was acting as a control (Def.~\ref{def:Control}). Researchers measured faecal weight, an \emph{objective} outcome\index{Data!objective}, to minimise the placebo effect.
\end{example}

\begin{exampleExtra}
A study of placebos \citep{data:Waber2008:Placebo} gave half the subjects a placebo, but told them the pill was an expensive (implying `effective') painkiller. The other half were also given a placebo, but were told the pill was a discount (implying `less effective') painkiller. About~\(85\)\% of participants in the first group reported a pain reduction, yet only~\(61\)\% in the second group reported a pain reduction. Remember: \emph{both} groups actually received a placebo! Again, `pain relief' is subjective.

\end{exampleExtra}

\section{Carryover effect and washouts}\label{CarryOverEffect}

\index{Carryover effect}\index{Washout}

In the \emph{Himalaya} study (Example~\ref{exm:HimalayaStudy}), the diet is a \emph{between-individuals} comparison:\index{Comparison!between individuals} one group of patients was given the refined cereal diet (the control), and a different group of people was given \emph{Himalaya~292}. The study \emph{also} used a \emph{within-individuals} comparison:\index{Comparison!within individuals} each person in the study was actually placed on both diets at different times.

Suppose all patients spent four weeks on the \emph{Himalaya~292} diet, then the next four weeks on the refined cereal diet. Potentially, the first diet could still be impacting the subjects' faecal weight for a little while after stopping the first diet. This could compromise the internal validity of the study. This is an example of the \emph{carryover effect}: when the influence of one treatment or condition on the response variable carries over to influence the value of the response variable for next treatment or condition. The carryover effect is only a concern for \emph{within-individuals} comparisons.

\begin{definition}[Carryover effect]
\protect\hypertarget{def:CarryoverEffect}{}\label{def:CarryoverEffect}The carryover effect occurs when the influence of one treatment or condition on the response variable influences the response variable for subsequent treatments or conditions (in a repeated-measures study).
\end{definition}

The impact of the carryover effect may be minimised by using a \emph{washout} or similar between treatments or conditions. For example, after tasting a food sample, participants may rinse their mouth with water before tasting another food sample. For the \emph{Himalaya} study, the participants could spend two weeks on their usual (before-study) diet, before starting each of the diets in the study. This is called a \emph{washout period}.\index{Washout period}

\begin{example}[Carryover effect: experimental study]
\protect\hypertarget{exm:HimalayaCarryOver}{}\label{exm:HimalayaCarryOver}In the \emph{Himalaya} study, `there was no washout period' (\citet{data:Bird2008:wholegrain}, p.~\(1\,033\)) since the response variable was only recorded after individuals spent four weeks on each diet. Since faecal weight was not measured until the \emph{end} of the four-week periods, the carryover effect is essentially irrelevant.
\end{example}

\begin{exampleExtra}
In \citet{jaskiewicz2020chest}, student paramedics performed chest compression in real-life (RL), and also using virtual reality (VR). Researchers were assessing the relaxation percentage of the students while undertaking the compression (a relaxation percentage of about~\(50\)\% is ideal).

When used by itself, the VR method produced an average relaxation percentage of~\(45.5\)\%. However, when the RL method was used first, and then followed by the VR method, the average VR relaxation method percentage was~\(74.7\)\%.

The response of the individuals was different depending on whether the RL method was used first. This is an example of the carryover effect.

\end{exampleExtra}

Sometimes, in experimental studies, researchers can \emph{randomly allocate} the \emph{order} in which the treatments are used (a \emph{crossover study}).\index{Study types!crossover studies} That is, some participants start by spending four weeks on the \emph{Himalaya~292} diet, then four weeks on the refined cereal diet; meanwhile, other participants start by spending four weeks on the refined cereal diet, then four weeks on the \emph{Himalaya~292} diet.

\begin{example}[Carryover effect]
\protect\hypertarget{exm:HimalayaCarryOver2}{}\label{exm:HimalayaCarryOver2}In the \emph{Himalaya} study (Example~\ref{exm:HimalayaStudy}), subjects were allocated randomly to begin the study on the \emph{Himalaya~292} diet \emph{or} the refined cereal diet.
\end{example}

\begin{example}[Washout periods: experimental study]
\protect\hypertarget{exm:ParamedOrder}{}\label{exm:ParamedOrder}\citet{data:MacDonald:Resuscitation} required paramedics to conduct eight different tasks (such as electrical defibrillation and intravenous cannulation). Each of the paramedics began the series of tasks at a random task, to mitigate the carryover effect. A washout period between tasks (i.e., a rest time) was also used.
\end{example}

The carryover effect also is a potential concern to internal validity in observational studies involving a within-individuals comparison. However, since treatments are \emph{not allocated} in observational studies, carryover effects may be difficult to prevent, as washouts cannot be imposed, and the order of the conditions cannot be imposed. However, \emph{observing} individuals exposed to Condition~A\index{Conditions} then Condition~B, and other individuals exposed to Condition~B then Condition~A, may be possible.

\begin{example}[Carryover effects: observational study]
\protect\hypertarget{exm:CarryoverObs}{}\label{exm:CarryoverObs}

\citet{norris2005carry} studied the carryover effect in ecological observational studies of animals (p.~181):

\begin{quote}
\ldots individuals occupying poor quality winter habitat may experience reduced reproductive success the following breeding season when compared to individuals occupying high quality winter habitat.
\end{quote}

\end{example}

\section{Describing blinding}\label{DescribingBlinding}

\index{Blinding!descriptions}\index{Blinding}

\emph{Blinding} occurs when those involved in the study do not know information about the study. The \emph{individuals} in the study may be blinded to

\begin{itemize}
\tightlist
\item
  whether they are \emph{involved in a study}.
\item
  the \emph{aims of the study} in which they are participants.
\item
  \emph{which comparison group} they are in.
\end{itemize}

Any or all of these may be true, depending on the study design.

The \emph{researchers} and the \emph{analysts} can be blinded to which comparison groups apply to the individuals (to help manage the observer effect).

When blinding is used in as many ways as possible, the internal validity of the study is increased and bias reduced. However, when people are the individuals, ethics requirements may mean that the individuals need to know they are in a study (especially if experimental), and the purpose of the study.

If \emph{only} the individuals are blinded to the comparison groups, the study is called \emph{single blind}.\index{Blinding!single} If \emph{both} the researchers and participants are blinded to the comparison groups, the study is called \emph{double-blind}.\index{Blinding!double} If the researchers, participants \emph{and} the analyst are blinded to the comparison groups, the study is sometimes called \emph{triple-blind}.\index{Blinding!triple} Rather than using these terms, explicitly stating who or what is blinded to which parts of the study is clearer.

Blinding should be considered in all studies when possible (it is \emph{not} always possible). \emph{Blinding participants does not just apply to people}; it also may apply to animals (Example~\ref{exm:CleverHans}).

\begin{example}[Double-blinding]
\protect\hypertarget{exm:BlindingCowpea}{}\label{exm:BlindingCowpea}\citet{bulte2014behavioral} compared yields from modern and traditional cowpea crops in Tanzania. The two seed types (`traditional' and `modern') were made similar in appearance, so the farmers were blinded to which group they were in (control or treatment). The seed type would eventually become obvious as the crop grew, but `key inputs were already provided' by then (p.~817). In addition, the researchers interacting with the farmers were not informed about the type of seed distributed.
\end{example}

In observational studies, blinding individuals \emph{may} be easier than in experimental studies (Sect.~\ref{HawthorneEffect}). Blinding the researchers may be difficult, since the researchers need to record the value of the explanatory variable.

\begin{example}[Blinding: observational studies]
\protect\hypertarget{exm:BlindingGymnasts}{}\label{exm:BlindingGymnasts}\citet{emerson2010ultrasonographically} studied Achilles tendinopathy in gymnasts, by comparing \(40\)~elite gymnasts with~\(41\) similar controls who were non-gymnasts. The authors state (p.~38) that

\begin{quote}
Although the primary investigator was blind to the clinical status of the subjects, there was no blinding to whether each subject was in the gymnast or control group during image collection {[}\ldots{]}
\end{quote}

When the images were reviewed, however, the article explains that the examiner was unaware of the clinical state and group of the subjects.

The paper explains who was blinded and to what parts of the study they were blinded.
\end{example}

\section{Recording extraneous variables}\label{RecordExtraneous}

\index{Variables!extraneous}

One way to design a high-quality study is to record information about many (potential) extraneous variables. Various reasons for doing this have been given.

\begin{itemize}
\item
  To evaluate \emph{external validity} to determine if the sample is representative of the population (Sect.~\ref{Representative-samples}), by comparing the sample and population. \tightlist
\item
  To improve \emph{internal validity}, by helping to manage confounding:

  \begin{itemize}
  \tightlist
  \item
    by avoiding lurking variables (Sect.~\ref{ExtraneousVariables}).\tightlist
  \item
    by determining if the comparison groups are similar (Sect.~\ref{ManagingConfounding}).
  \item
    by using the information in analysis (Sect.~\ref{ManagingConfounding}).
  \end{itemize}
\end{itemize}

\begin{importantBox}{iconmonstr-warning-8-240.png}
\emph{Record the values of all extraneous variables that may be important in the study}!

\end{importantBox}

\begin{example}[Poor internal validity]
\protect\hypertarget{exm:ConfoundingChildbirth}{}\label{exm:ConfoundingChildbirth}In the~1800s, Semmelweis recorded mortality rates of women after childbirth over many years \citep{dunn2005ignac} at two clinics:

\begin{itemize}
\tightlist
\item
  in Clinic~1, with male doctors delivering babies:~\(9.9\)\%.
\item
  in Clinic~2, with female midwives delivering babies:~\(3.4\)\%.
\end{itemize}

Was the difference in mortality rate (the outcome) due to the sex of the person delivering the babies (the comparison)?

One possible confounder was the clinic; however, the clinic was eliminated as an explanation. For example, Clinic~2 was actually \emph{more} overcrowded than Clinic~1, and the climate was similar for both clinics.

However, an important \emph{lurking variable} was present. At the time, the benefits of hand-washing were not understood, nor commonplace. Many (male) doctors performed autopsies immediately before delivering babies, without washing their hands between procedures. In contrast, autopsies were not performed by the (female) nurses.

The lurking variable was `whether the baby was delivered by someone with clean hands', which was related to the mortality rate \emph{and} to the sex of the person delivering the baby. The female midwives had clean hands, and hence the mortality rate was (relatively) low. The male doctors did \emph{not} have clean hands, and hence the mortality rate was high.

After instituting hand-washing for doctors, the mortality rate in Clinic~1 reduced to a rate similar to that in Clinic~2.
\end{example}

\section{Recording objective data}\label{RecordObjectiveData}

\index{Data!objective}\index{Data!subjective}

Recording \emph{objective} data is often more reliable than recording \emph{subjective} data, as subjective data can be influenced by the Hawthorne, observer or placebo effects. Perceptions are often unreliable also. However, sometimes recording objective data is not possible, and sometimes the researchers are explicitly interested in the subjective responses of people to certain treatments or conditions.

\begin{definition}[Subjective and objective data]
\protect\hypertarget{def:SubjectiveObjective}{}\label{def:SubjectiveObjective}\emph{Subjective} data refers to opinions, feelings, and interpretations (by the subjects or the researchers). \emph{Objective} data refers to facts and measurable evidence.
\end{definition}

\begin{importantBox}{iconmonstr-warning-8-240.png}
If possible, objective data should be recorded whenever possible.\index{Data!objective}

\end{importantBox}

\begin{example}[Subjective and objective data]

\citet{ueberham2019cyclists} studied cyclists using everyday routes over one week in Leipzig (Germany). Sixty-six cyclists wore sensors that \emph{objectively} recorded particle number counts (i.e., pollution), noise, humidity and temperature. The cyclists also \emph{subjectively} recorded similar information.

The researchers concluded that (p.~1)

\begin{quote}
Except for heat, no significant associations between the objective and subjective data were found.
\end{quote}

That is, the subjective and objective data generally did not agree, except for heat. The perceptions of heat may have been influenced by the Hawthorne effect (p.~7):

\begin{quote}
\ldots most people are pre-informed about the daily temperature by the weather forecast and expect a certain degree of heat, which ultimately also affects their perception of it to a great extent.
\end{quote}

\end{example}

\begin{example}[Subjective and objective data]
\citet{johnson2021association} asked \(70\)~people in south-west Ireland to \emph{subjectively} self-report their Body-Mass index (BMI) category, as `Normal' or `Overweight'. \emph{Thirty-six} subjects self-reported their BMI category as `Normal'.

The researchers also \emph{objectively} recorded the BMI of the same subjects. \emph{Twenty-nine} subjects were objectively categorised as `Normal'.
\end{example}

\index{Research design!internal validity|)}

\section{Chapter summary}\label{Chap7-Summary}

Designing effective studies (Fig.~\ref{fig:DesignConsiderations}) requires researchers to \emph{manage or minimise confounding} where possible, by: \emph{restricting} the study to certain groups; \emph{blocking} individuals into similar groups; through special \emph{analysis} methods; and/or through \emph{random allocation} of the units of analysis. Random allocation is only possible for experimental studies.

Well-designed studies manage the \emph{Hawthorne effect} (e.g., by blinding \emph{participants}); the \emph{observer effect} (e.g., by blinding the \emph{researchers}); the \emph{placebo effect} (experimental studies only; e.g., by using controls, objective outcomes and blinding subjects); and the \emph{carryover effect} (e.g., by using a washout, or randomly allocating the treatment order). Recording objective data is usually better than recording subjective data.

Often, however, not all of these strategies can be used. For instance, people usually know they are involved in an experimental study, so the Hawthorne effect may impact conclusions. In these cases, the possible impacts should be minimised as far as possible, and then the likely impact on the conclusions discussed. The impact of these issues are often reported as \emph{limitations}\index{Limitations} (Chap.~\ref{Interpretation}).



\begin{figure}[hbtp]

{\centering \includegraphics[width=0.9\linewidth]{07-ResearchDesign-Internal_files/figure-latex/DesignConsiderations-1} 

}

\caption{Design considerations for designing studies. Note: lurking variables become confounding variables when recorded in the study, and so can be managed as confounding variables. The arrows indicate the main design strategies to (perhaps partially) manage the indicated potential bias. Not all strategies are possible for every study.}\label{fig:DesignConsiderations}
\end{figure}

\begin{example}[Research design]
\protect\hypertarget{exm:DesignExample}{}\label{exm:DesignExample}\citet{cross2019impact} (p.~3) compared chest compressions by student paramedics using dominant and non-dominant hands, and stated:

\begin{quote}
\ldots participants were allocated randomly to one of two groups: `dominant hand on chest' or `non-dominant hand on chest'. Group allocation was determined by a computer-generated randomisation schedule\ldots{}
\end{quote}

The participants were blinded to the \emph{purpose} of the study, but not to which group they were allocated. The analyst was also blinded to the group allocations. This study used many good design features.
\end{example}

\section{Quick review questions}\label{Chap7-QuickReview}

\citet{doosti2016development} wanted to determine the relationship between the depth of bruising on apples and the impact force. The researchers purposefully hit apples with three different \emph{forces} (\(200\),~\(700\) and~\(1200\,\text{mJ}\)) to inflict bruises on the apples. The researchers then recorded the \emph{depth} of the bruising. The study was conducted separately for three different \emph{regions} of the apple (lower; middle; upper), and each apple was only used once.

Are the following statements \emph{true} or \emph{false}?

\begin{enumerate}
\def\labelenumi{\arabic{enumi}.}
\item
  The \emph{response variable} is `the depth of bruising'.\tightlist
\item
  The \emph{explanatory variable} is `the force used on the apples'.
\item
  The variable `location of the bruising' would be classified as a confounding variable'.
\item
  The researchers could minimise the effects of confounding by incorporating potential confounding variables in the analysis. \tightlist
\item
  The researchers could use random allocation of the treatments to the apples to minimise confounding.
\item
  The \emph{carryover} effect is likely to be a big problem in this study.
\item
  The \emph{Hawthorne} effect is likely to be a big problem in this study.
\item
  The \emph{placebo} effect is likely to be a big problem in this study.
\item
  The \emph{observer} effect is likely to be a big problem in this study.
\end{enumerate}

\section{Exercises}\label{DesigningExperimentsExercises}

\hyperref[Answers]{Answers to odd-numbered exercises} are given at the end of the book.

\captionsetup{font=small}

\begin{exercise}
\protect\hypertarget{exr:DesignTrueFalse}{}\label{exr:DesignTrueFalse}

Are the following statements \emph{true} or \emph{false}?

\begin{enumerate}
\def\labelenumi{\arabic{enumi}.}
\item
  Experimental studies \emph{must} use random samples. \tightlist  
\item
  An experimental study \emph{must} blind the researchers.
\item
  Only observational studies can manage the observer effect.
\item
  Experimental studies \emph{must} use a control group.
\item
  Using random samples is important in observational studies as a way to manage confounding.
\end{enumerate}

\end{exercise}

\begin{exercise}
\protect\hypertarget{exr:DesignTrueFalse2}{}\label{exr:DesignTrueFalse2}

Which of the following statements are true?

\begin{enumerate}
\def\labelenumi{\arabic{enumi}.}
\item
  Observational studies cannot have a control group.\tightlist
\item
  Only experimental studies can use random allocation to avoid confounding.
\item
  An experimental study \emph{must} blind the participants.
\item
  Only experimental studies can use random sampling.
\item
  In experimental studies, the treatments \emph{must} be allocated by the researchers.
\end{enumerate}

\end{exercise}

\begin{exercise}
\protect\hypertarget{exr:DesignExpImproveIV}{}\label{exr:DesignExpImproveIV}

Which of the following can improve internal validity in \emph{experimental} studies?

\begin{cols}

\begin{col}{0.425\textwidth}

\begin{itemize}
\tightlist
\item
  Blinding the individuals
\item
  Using a control group.
\item
  Using special methods of analysis.
\end{itemize}

\end{col}

\begin{col}{0.025\textwidth}
~

\end{col}

\begin{col}{0.475\textwidth}

\begin{itemize}
\tightlist
\item
  Randomly allocating treatments to groups.
\item
  Blinding the researchers.
\item
  Using random samples.
\end{itemize}

\end{col}

\end{cols}

\end{exercise}

\begin{exercise}
\protect\hypertarget{exr:DesignObsImproveIV}{}\label{exr:DesignObsImproveIV}

Which of the following can improve internal validity in \emph{observational} studies?

\begin{cols}

\begin{col}{0.425\textwidth}

\begin{itemize}
\tightlist
\item
  Blinding the individuals
\item
  Using a control group.
\item
  Using special methods of analysis.
\end{itemize}

\end{col}

\begin{col}{0.025\textwidth}
~

\end{col}

\begin{col}{0.475\textwidth}

\begin{itemize}
\tightlist
\item
  Randomly allocating treatments to groups.
\item
  Blinding the researchers.
\item
  Using random samples.
\end{itemize}

\end{col}

\end{cols}

\end{exercise}

\begin{exercise}
\protect\hypertarget{exr:ResearchDesignHawthorneEffect}{}\label{exr:ResearchDesignHawthorneEffect}Is the Hawthorne effect only a (potential) issue for experiments. Explain.
\end{exercise}

\begin{exercise}
\protect\hypertarget{exr:HawthorneTeeth}{}\label{exr:HawthorneTeeth}\citet{lorenz2019effect} compared the efficacy of a new type of toothpaste. Participants were given either a new or an existing toothpaste, and evaluations of plaque remaining on the teeth were taken after brushing. All participants knew they were being assessed after brushing.

Would the Hawthorne effect likely impact the internal validity of this study? Explain.
\end{exercise}

\begin{exercise}
\protect\hypertarget{exr:ResearchDesignObsPollen}{}\label{exr:ResearchDesignObsPollen}A study compared the average amount of pollen returned to the hive per bee, for two types of native Australian bees: yellow and black carpenter bees, and green carpenter bees. The researchers also recorded the size of the hive, among other variables. \emph{Why} did they do this?
\end{exercise}

\begin{exercise}
\protect\hypertarget{exr:ResearchDesignSeptic}{}\label{exr:ResearchDesignSeptic}In a study to treat septic shock, \citet{hwang2020combination} used two study groups of size \(n = 58\) each: one group received the treatment of interest (intravenous infusion of vitamin~C and thiamine) and the other group received intravenous saline.

Explain why the researchers gave saline to \(58\) subjects, when it has no chance of successfully treating septic shock. Is this ethical?
\end{exercise}

\begin{exercise}
\protect\hypertarget{exr:DesignExpWeightLoss}{}\label{exr:DesignExpWeightLoss}

Consider a study comparing the average weight loss for patients who are \emph{instructed} to do about \(30\,\text{mins}\) of exercise a day (Group~A), to patients who are instructed to do about~\(60\,\text{mins}\) of exercise a day (Group~B). Which of the following statements are true?

\begin{enumerate}
\def\labelenumi{\arabic{enumi}.}
\item
  This is an experimental study.
\item
  The extraneous variable is the amount of exercise per day (in h).\tightlist
\item
  The response variable is the weight loss for each person.
\item
  The explanatory variable is whether the patient performs about \(30\) or \(60\,\text{mins}\) of exercise per day.
\item
  The response variable is the \emph{average} weight loss.
\item
  The explanatory variable is the number of minutes of exercise the patient does per day.
\item
  Age is likely to be a lurking variable.
\item
  Age is likely to be an extraneous variable.
\item
  Age is likely to be a confounding variable.
\item
  Which (if any) of the following are possible \emph{confounding} variables?

  \begin{itemize}
  \item
    The sex of the patients. \tightlist
  \item
    The initial weight of the patients.
  \item
    The names of the patients.
  \end{itemize}
\end{enumerate}

\end{exercise}

\begin{exercise}
\protect\hypertarget{exr:ResearchSmokingAlfresco}{}\label{exr:ResearchSmokingAlfresco}

\citet{data:Stafford2010:Alfresco} studied smoking in alfresco restaurants in two cities in Western Australia. The concentration of particulate matter with a diameter smaller than or equal to~\(2.5\) (per cubic metre of air) was recorded (PM\textsubscript{\(2.5\)}) from~\(12\) cafés and \(16\)~pubs. The researchers were interested in the relationship between PM\textsubscript{\(2.5\)} and the number of smokers. They also recorded the wind strength (calm; light breeze; windy) and the amount of cover (fully open; overhead cover only; overhead cover and enclosed sides).

\begin{enumerate}
\def\labelenumi{\arabic{enumi}.}
\tightlist
\item
  Is this an experimental or observational study?
\item
  What are the response and explanatory variables?
\item
  What are the extraneous variables, if any?
\item
  Is blinding the individuals possible?
\item
  Is random allocation possible?
\end{enumerate}

\end{exercise}

\begin{exercise}
\protect\hypertarget{exr:ResearchDesignSunscreen}{}\label{exr:ResearchDesignSunscreen}

In a study of time spent applying sunscreen \citep{data:Heerfordt2018:sunscreen}, the aim was to `determine whether time spent on sunscreen application is related to the amount of sunscreen used' (p.~117). The study is described as follows (p.~118):

\begin{quote}
The volunteers were asked to apply the provided sunscreen {[}\ldots{]} the way they would normally do on a sunny day at the beach in Denmark {[}\ldots{]} The volunteers wore swimwear during the whole session. No other information was given. Participants applied sunscreen behind a curtain and were not observed during application. Measurements of time and sunscreen weight were made without the subjects' being aware of this.
\end{quote}

\begin{enumerate}
\def\labelenumi{\arabic{enumi}.}
\tightlist
\item
  Is this an experimental or observational study?
\item
  What are the response and explanatory variables?
\item
  The researchers also recorded age, height, weight and body surface area of each participant. Why would they have done this?
\item
  The researchers also compared the average values of the response variable for males and females, and the average values of the explanatory variable for males and females. Why would they have done this?
\item
  What design features are evident in the quote?
\end{enumerate}

\end{exercise}

\begin{exercise}
\protect\hypertarget{exr:DesignExpParamedicsPills}{}\label{exr:DesignExpParamedicsPills}

Paramedics were involved in a study to compare two treatments (A and~B) for Post Traumatic Stress Disorder (PTSD), as randomly allocated to two groups of patients.

\begin{enumerate}
\def\labelenumi{\arabic{enumi}.}
\item
  Is this an experimental or observational study?\tightlist
\item
  What would be the control group?

  \begin{enumerate}
  \def\labelenumii{\alph{enumii}.}
  \tightlist
  \item
    The group receiving Treatment~A.
  \item
    A group of paramedics who do not have PTSD.
  \item
    The group receiving Treatment~B.
  \item
    A group of paramedics not involved in the study.
  \item
    A group of people with PTSD who are not paramedics.
  \item
    A group receiving a pill that looks just like Treatment~A and~B, but has no effective ingredient.
  \end{enumerate}
\item
  The patients did not know which treatment they received. What is this called?
\item
  What is the purpose of blinding the participants?
\end{enumerate}

\end{exercise}

\begin{exercise}
\protect\hypertarget{exr:ResearchIcelandicGood}{}\label{exr:ResearchIcelandicGood}\citet{blondal2023homefood} studied older Icelandic adults, to (p.~632)

\begin{quote}
\ldots investigate effects of six-month nutrition therapy on hospital readmissions, LOS {[}length of hospital stay{]}, mortality and need for long-term care residence\ldots{}
\end{quote}

Patients in the study were all aged over \(64\)-years-of-age, and were randomly allocated into either the intervention (\(n = 53\)) or control (\(n = 53\)) groups. The intervention group received `nutrition therapy', which included free delivery of energy- and protein-rich foods for six months after discharge from hospital. Patients with cognitive impairment, dietary allergies and undergoing active cancer treatment were excluded from the study.

Statistical software (SPSS, version~26.0) was used to generate the random numbers for randomly allocating patients into groups, and the allocations were hidden from the researchers. Table~1 of the published article compared the two groups on (among other variables) age, percentage female, height, weight, and percentage with a higher education. Ethics approval was given by the Ethics Committee for Health Research of the National University Hospital of Iceland.

Identify good design principles used in this study, and their purpose. Can you identify any improvements that could be made to the study design?
\end{exercise}

\begin{exercise}
\protect\hypertarget{exr:ResearchBodyCameras}{}\label{exr:ResearchBodyCameras}\citet{braga2021body} evaluated the use of body-worn cameras (BWC) by police on residents' perceptions of the police. The \(40\)~precincts in New York City with the highest numbers complaints against New York Police Department officers were part of the study. Twenty precincts were matched with \(20\)~other precincts (using demographics, socioeconomic characteristics, crime, and police activity). Using a computer, one of each pair of precincts was allocated to have officers wear BWC, and the other to not wear BWCs. The researchers compared the officers in each group, and found the officers in the two groups were similar `in terms of demographics, length of service, rank, work activities, and number of citizen complaints' (p.~285).

Identify the good design principles used in this study (using the language of this chapter where possible), and their purpose. Can you identify any improvements that could be made to the study design?
\end{exercise}

\begin{exercise}
\protect\hypertarget{exr:ResearchTanzaniaFarm}{}\label{exr:ResearchTanzaniaFarm}

\citet{bulte2014behavioral} compared yields from two types of cowpea in rural Tanzania. Two different studies were used for the comparison; for both studies, traditional and new cowpea seeds were randomly allocated to a sample of farmers.

In Study~A, farmers knew which seeds were the newer variety, and the researchers interacting with the farmers also knew the types of seed allocated. In Study~B, the farmers were blinded to which seeds were new and which were traditional (both seeds were the same size and colour), and the researchers interacting with the farmers also did \emph{not} know the types of seed allocated.

Farmers who knew they received the modern seed (Study~A) planted their seed on larger plots than farmers in Study~B; that is, they behaved differently.

When the data from Study~A were analysed, the new seed yield was \(27\)\% greater, compared to the traditional seed. However, in Study~B, the yield of new seed was very similar to the yield for the traditional seed.

\begin{enumerate}
\def\labelenumi{\arabic{enumi}.}
\tightlist
\item
  Discuss the blinding used in each study.
\item
  Explain the discrepancies from the two methods, using the language of this chapter.
\item
  Which method would be more likely to be recording accurate information? Why?
\end{enumerate}

\end{exercise}

\begin{exercise}
\protect\hypertarget{exr:ResearchAustriaPA}{}\label{exr:ResearchAustriaPA}

\citet{greier2021objective} studied \(36\)~Austrian eighth-grade students, recording the time spent performing moderate-to-vigorous physical activity (\textsc{mvpa}). For each student, the time spent in \textsc{mvpa} was recorded in two ways: (a)~using a wrist-worn accelerometer, worn for seven days; and (b)~self-reported using a questionnaire at the end of the week.

\textsc{Mvpa} reported in the questionnaire was more than double the time measured using the accelerometer; thus, \(88.9\)\% of students met PA recommendations based on self-report, but only \(22.2\)\% did so based on the accelerometer data.

\begin{enumerate}
\def\labelenumi{\arabic{enumi}.}
\tightlist
\item
  Which method recorded \emph{subjective} data? Which method recorded \emph{objective} data?
\item
  Explain the discrepancies in the results from the methods, using the language of this chapter.
\item
  Which method would be more likely to be recording accurate information? Why?
\end{enumerate}

\end{exercise}

\begin{exercise}
\protect\hypertarget{exr:ResearchTasteOfWater2}{}\label{exr:ResearchTasteOfWater2}A scientist is testing whether tap water tastes the same as bottled water (based on \citet{teillet2010consumer}). Subjects taste either bottled or tap water from a plastic cup, and give a rating of the taste on a scale of~\(1\) (terrible) to~\(5\) (fantastic). You decide to address this question using an \emph{experiment}. Describe what these might look like for this study, and which are feasible: random allocation; blinding; double-blinding; control; carryover effect; finding a random sample.

What potential problems can you identify with the research design?
\end{exercise}

\begin{exercise}
\protect\hypertarget{exr:ResearchDesignTasteOfWater}{}\label{exr:ResearchDesignTasteOfWater}A scientist is testing whether tap water tastes the same as bottled water (based on \citet{teillet2010consumer}). Subjects taste either bottled or tap water from a plastic cup, and give a rating of the taste on a scale of~\(1\) (terrible) to~\(5\) (fantastic).

You want to answer this question using an \emph{observational study}. Describe what these might look like for this study, and which are feasible: random allocation; blinding; double-blinding; control; carryover effect; finding a random sample.
\end{exercise}

\begin{exercise}
\protect\hypertarget{exr:ResearchDesignFertilizer}{}\label{exr:ResearchDesignFertilizer}A scientist compares the effects of two types of fertiliser on the yield of tomatoes (based on \citet{klanian2018integrated}). He plants tomato seedlings, and fertilises with Fertiliser~I, and later records the yield of tomatoes. He then immediately plants more tomato seedlings in the same field, fertilises with Fertiliser~II, and measures the yield of tomatoes.

What potential problems can you identify with the research design?
\end{exercise}

\begin{exercise}
\protect\hypertarget{exr:ResearchDesignDust}{}\label{exr:ResearchDesignDust}

\citet{data:skulberg:dust} compared two office-cleaning methods (p.~72):

\begin{quote}
The participants were randomly allocated to an intervention group or a control group using group level matching by sex, level of irritation symptom index, and allergy status\ldots{} The participants and the field researchers were blinded to the group status of the participants\ldots{}
\end{quote}

All offices were cleaned after the employees had left the building for the evening.

The researchers compared the change in nasal congestion for the two groups (intervention: `a comprehensive cleaning'; control: `a superficial cleaning'), finding only small differences between the two groups. In the analysis, the researchers incorporated age and sex of the office workers.

\begin{enumerate}
\def\labelenumi{\arabic{enumi}.}
\tightlist
\item
  How did the researchers manage confounding?
\item
  What other design features are evident from the quote?
\item
  What is the response variable?
\item
  What is the explanatory variable?
\item
  What extraneous variables are apparent?
\end{enumerate}

\end{exercise}

\begin{exercise}
\protect\hypertarget{exr:ResearchDesignFormwork}{}\label{exr:ResearchDesignFormwork}

\emph{Formwork} is used in construction with reinforced concrete, and can be labour intensive. \citet{mine2015observational} examined the relationship between the floor area of the building (in m\textsuperscript{\(2\)} per storey) and the number of hours of labour needed for constructing the formwork (in person-mins per storey). The researchers also recorded the average age of the workers (in years); the average years of experience of the workers (in years); and the storey height (in meters) for each of \(n = 15\) multi-storey buildings in the study.

Two observers recorded the labour time by observing workers from the start to the end of the work.

\begin{enumerate}
\def\labelenumi{\arabic{enumi}.}
\item
  What is the \emph{explanatory variable}? \tightlist
\item
  What is the \emph{response variable}?
\item
  What \emph{type} of description is appropriate for the variable `age of the workers'?

  \begin{enumerate}
  \def\labelenumii{\alph{enumii}.}
  \tightlist
  \item
    A confounding variable, since it is likely to be related to the explanatory variable only.
  \item
    A confounding variable, since it is likely to be related to the response variable only.
  \item
    An extraneous variable, because it is likely to be related to the response variable only.
  \item
    A lurking variable, since we don't know how it might be related to the response and explanatory variables.
  \end{enumerate}
\item
  What is the most likely way to manage confounding in this study restricting, blocking, analysis, random allocation?
\item
  True or false: the \emph{carryover} effect is likely to be a big problem in this study.
\item
  True or false: the \emph{Hawthorne} effect is likely to be a big problem in this study.
\item
  True or false: the \emph{placebo} effect is likely to be a big problem in this study.
\item
  True or false: \emph{observer} effect is likely to be a big problem in this study.
\end{enumerate}

\end{exercise}

\captionsetup{font=normalsize}

\begin{EOCanswerBox}{iconmonstr-check-mark-14-240.png}
\textbf{Answers to \emph{Quick review} questions:} \textbf{1.} True. \textbf{2.} True. \textbf{3.} False; extraneous. \textbf{4.} True. \textbf{5.} True. \textbf{6.} False. \textbf{7.} False. \textbf{8.} False. \textbf{9.} False.

\end{EOCanswerBox}

\chapter{Research design limitations}\label{Interpretation}

\begin{cols}
\begin{col}{0.52\textwidth}

\begin{objectivesBox}{iconmonstr-target-4-240.png}
So far, you have learnt to ask an RQ and design research studies.
\textbf{In this chapter}, you will learn to identify limitations to:

\begin{itemize}\tightlist
  \item
  internal validity.
  \item
  external validity.
  \item
  ecological validity.
\end{itemize}
\end{objectivesBox}
\end{col}

\begin{col}{0.03\textwidth}
~
\end{col}

\begin{col}{0.45\textwidth}

\includegraphics[width=0.95\linewidth]{08-ResearchDesign-Limitations_files/figure-latex/unnamed-chunk-4-1} 
\end{col}
\end{cols}

\section{Introduction}\label{Chap9-Intro}

The type of study (Chap.~\ref{ResearchDesign}) and the research design determine how the results of the study should be interpreted. Ideally, all studies would be perfectly externally and internally valid; in practice this is very difficult to achieve. Practically \emph{every} study has limitations. The results of a study should be interpreted in light of these limitations. Limitations are not necessarily \emph{problems}.

Limitations generally can be discussed through three components.

\begin{itemize}
\tightlist
\item
  \emph{Internal validity} (Sect.~\ref{IntroInternalValidity}): discuss any limitations to internal validity due to the research design (such as identifying possible confounding variables). This is related to the \emph{effectiveness} of the study within the sample (Sect.~\ref{InterpretStudyDesign}).
\item
  \emph{External validity} (Sect.~\ref{IntroExternalValidity}): discuss how well the sample represents the intended population. This is related to the \emph{generalisability} of the study to the intended population (Sect.~\ref{InterpretGeneralisability}).
\item
  \emph{Ecological validity}: discuss how well the study methods, materials and context approximate the real situation of interest. This is related to the \emph{practicality} of the results to real life (Sect.~\ref{InterpretApplicability}).
\end{itemize}

All these issues should be addressed when considering the study limitations.

\begin{importantBox}{iconmonstr-warning-8-240.png}
Almost every study has limitations. \emph{Identifying} potential limitations, and \emph{discussing the likely impact} they have on the interpretation of the study results, is important and ethical.\index{Ethics}

\end{importantBox}

Different types of research studies have limitations. Experimental studies, in general, have higher internal validity than observational studies, since more of the research design in under the control of the researchers; for example, random allocation of treatments is possible to minimise confounding.

\begin{importantBox}{iconmonstr-warning-8-240.png}
Only well-designed and well-conducted experimental studies can show cause-and-effect relationships.

\end{importantBox}

However, experimental studies may suffer from poor \emph{ecological} validity; for instance, laboratory experiments are often conducted under controlled temperature and humidity. Many experiments also require that people be told about being in a study (due to ethics), and so internal validity may be comprised (the Hawthorne effect).

\begin{example}[Limitations due to the study type]
\protect\hypertarget{exm:LimitationsStudyType}{}\label{exm:LimitationsStudyType}\citet{giandomenico2022systematic} studied retro-fitting houses with energy-saving devices, and found large discrepancies in energy savings for observational studies (\(12.2\)\%) and experimental studies (\(6.2\)\%). The authors say that `this finding reinforces the importance of using study designs with high internal validity to evaluate program savings' (p.~692).
\end{example}

\section{Limitations related to internal validity}\label{InterpretStudyDesign}

\index{Limitations!research design (effectiveness)}

Internal validity refers to the extent to which a cause-and-effect relationship can be established in a study, eliminating other possible explanations (Sect.~\ref{IntroInternalValidity}); that is, the \emph{effectiveness} of the study in the sample. A discussion of the limitations of internal validity should cover, as appropriate: possible confounding and lurking variables; the impact of the Hawthorne, observer, placebo and carryover effects; the impact of any other design decisions.

If anything is likely to compromise internal validity, the implications on the interpretation of the results should be discussed. For example, if the participants were not blinded, this should be clearly stated, and the conclusion should indicate that the individuals in the study may have behaved differently than usual.

\begin{example}[Study limitations]
\protect\hypertarget{exm:LimitationsSeeds}{}\label{exm:LimitationsSeeds}\citet{axmann2020access} randomly allocated Ugandan farmers to receive, or not receive, hybrid maize seeds. One potential threat to internal validity was that farmers receiving the hybrid seeds could share seeds with their neighbours.

Hence, the researchers contacted the \(75\)~farmers allocated to receive the hybrid seeds; none of the contacted farmers reported selling or giving seeds to other farmers. This extra step increased the internal validity of the study.
\end{example}

Maximising internal validity in \emph{observational studies} is more difficult than in experimental studies (since random allocation is not possible). However, the internal validity of \emph{experimental studies} involving people is often compromised because people must be informed that they are participating in a study.

\begin{example}[Internal validity]
\protect\hypertarget{exm:InternalHandHygiene}{}\label{exm:InternalHandHygiene}In a study of the hand-hygiene practices of paramedics \citep{barr2017self}, \emph{self}-reported hand-hygiene practices were very different from what was reported by \emph{peers}. That is, how people self-report their behaviours may not align with how they actually behave, which influenced the internal validity of the study.
\end{example}

\begin{exampleExtra}

A study evaluated using a new therapy on elderly men, and listed some limitations of their study:

\begin{quote}
\ldots{} the researcher was not blinded and had prior knowledge of the research aims, disease status, and intervention. As such, these could all have influenced data recording {[}\ldots{]} The potential of reporting bias and observer bias could be reduced by implementing blinding in future studies.

\VA{--- \citet{kabata2021effect}, p.~10}{}
\end{quote}

\end{exampleExtra}

\section{Limitations related to external validity}\label{InterpretGeneralisability}

\index{Limitations!external validity (generalisability)}

External validity refers to the ability to \emph{generalise} the findings made from the sample to the entire \emph{intended} population (Sect.~\ref{IntroExternalValidity}). For a study to be externally valid, it must first be internally valid: that is, if the study is not effective in the sample studied (i.e., internally valid), the results may not apply to the intended population either.

\begin{importantBox}{iconmonstr-warning-8-240.png}
External validity refers to how well the sample is likely to represent the \emph{intended population} in the RQ.

If the population is Californians, then the study is externally valid if the sample is representative of Californians. The results \emph{do not} have to apply to people in the rest of the United States to be externally valid (though this can be commented on too). The intended population is \emph{Californians}.

\end{importantBox}

External validity depends on \emph{how} the sample was obtained. Results from random samples\index{Sampling!random} (Sect.~\ref{RandomSamplingMethods}) are likely to generalise to the population and be externally valid. (The analyses in this book assume all samples are \emph{simple random samples}.) Furthermore, results from approximately representative samples\index{Sampling!representative} (Sect.~\ref{Representative-samples}) \emph{may} generalise to the population and be externally valid if those \emph{in} the study are not obviously different from those \emph{not in} the study.

Any inclusion criteria,\index{Inclusion criteria} exclusion criteria\index{Inclusion criteria} or control variables\index{Variables!control} may also limit the external validity of the study.

\begin{example}[External validity]
\protect\hypertarget{exm:ExternalNZ}{}\label{exm:ExternalNZ}\citet{data:Gammon2012:B12} identified (for well-documented reasons) a \emph{population} of interest: `women of South Asian origin living in New Zealand' (p.~21). The women in the \emph{sample} were `women of South Asian origin {[}\ldots{]} recruited using a convenience sample method throughout Auckland' (p.~21).

The results may not generalise to the \emph{intended} population (\emph{all} women of South Asian origin living in New Zealand) because all the women in the sample came from Auckland, and the sample was not a \emph{random} sample from this population anyway. The study was still useful however, since we have still learnt information about the population that is represented by the sample, which may be \emph{similar} to the intended population.
\end{example}

\begin{example}[Using biochar]
\protect\hypertarget{exm:ExternalBiochar}{}\label{exm:ExternalBiochar}\citet{farrar2018short} studied growing ginger using biochar on one farm at Mt~Mellum, Australia. The results may only generalise to growing ginger at Mt~Mellum, but since ginger is usually grown in similar types of climates and soils, the results \emph{may} apply to other ginger farms also.
\end{example}

\clearpage

\section{Limitations related to ecological validity}\label{InterpretApplicability}

\index{Limitations!ecological (practicality)}

The likely \emph{practicality} of the study results in the real world should also be discussed. This is called \emph{ecological validity}.

\begin{definition}[Ecological validity]
\protect\hypertarget{def:EcologicalValidity}{}\label{def:EcologicalValidity}A study is \emph{ecologically valid} if the study methods, materials and context closely approximate the real situation of interest.
\end{definition}

Studies don't \emph{need} to be ecologically valid to be useful; much can be learnt under special conditions, as long as the potential limitations are understood when applying the results to the real world. The ecological validity of experimental studies may be compromised because the experimental conditions are sometimes highly controlled (for good reason).

\begin{example}[Ecological validity]
\protect\hypertarget{exm:EcologicalCups}{}\label{exm:EcologicalCups}Consider a study to determine the proportion of people that buy coffee in a reusable cup. People could be \emph{asked} about their behaviour. This study may not be ecologically valid, as what people \emph{do} may not align with what they \emph{say} they will do (i.e., subjective data).\index{Data!subjective}

An alternative study could \emph{watch} people buy coffee at various coffee shops, and record what people \emph{do} in practice. (i.e., objective data).\index{Data!objective} This second study is more likely to be \emph{ecologically valid}, as real-world behaviour is observed.
\end{example}

\begin{exampleExtra}
A study observed the effect of using high-mounted rear brake lights \citep{data:Kahgane1998:RearBrakeLights}, which are now commonplace. The American study showed that such lights reduced rear-end collisions by about~\(50\)\%. However, after making these lights mandatory, rear-end collisions reduced by only~\(5\)\%. Why?

\end{exampleExtra}

\section{Chapter summary}\label{Chap9-Summary}

The limitations in a study need to be identified, and may be related to:

\begin{itemize}
\tightlist
\item
  \emph{internal validity} (effectiveness); how well the study is conducted within the sample, isolating the relationship of interest.
\item
  \emph{external validity} (generalisability); how well the sample results are likely to apply to the intended population.
\item
  \emph{ecological validity} (practicality); how well the results may apply to the real-world situation of interest.
\end{itemize}

\section{Quick review questions}\label{Chap9-QuickReview}

Are the following statements \emph{true} or \emph{false}?

\begin{enumerate}
\def\labelenumi{\arabic{enumi}.}
\item
  When interpreting the results of a study, the steps taken to maximise internal validity should be evaluated. \tightlist
\item
  If studies are not externally valid, then they are not useful.
\item
  When interpreting the results of a study, the steps taken to maximise external validity do not need to be evaluated.
\item
  When interpreting the results of a study, ecological validity is about the impact of the study on the environment.
\end{enumerate}

\section{Exercises}\label{InterpretationExercises}

\hyperref[Answers]{Answers to odd-numbered exercises} are given at the end of the book.

\captionsetup{font=small}

\begin{exercise}
\protect\hypertarget{exr:ValidityLighting}{}\label{exr:ValidityLighting}\citet{gentile2022improving} examined how people can save energy through lighting choices. The study states (p.~9) that the results `are limited to the {[}sample{]} and cannot be easily projected to other similar settings'.

What type of validity is being discussed here?
\end{exercise}

\begin{exercise}
\protect\hypertarget{exr:InterpretationExerciseValidities}{}\label{exr:InterpretationExerciseValidities}Fill the blanks with the correct word: \emph{internal}, \emph{external} or \emph{ecological}.

When interpreting the results of studies, we consider the practicality (\_\_\_\_\_\_\_ validity), the generalisability (\_\_\_\_\_\_\_ validity) and the effectiveness (\_\_\_\_\_\_\_ validity).
\end{exercise}

\begin{exercise}
\protect\hypertarget{exr:InterpretationExerciseExternalValidity}{}\label{exr:InterpretationExerciseExternalValidity}A student project asked if `the average word retention is higher in male students than female students at UniX'. When discussing \emph{external validity}, the students stated:

\begin{quote}
We cannot say whether or not the general public have better or worse word retention compared to the students that we will be studying.
\end{quote}

Why is the statement \emph{not relevant} in a discussion of external validity?
\end{exercise}

\begin{exercise}
\protect\hypertarget{exr:InterpretationExerciseParachutes}{}\label{exr:InterpretationExerciseParachutes}\citet{data:Yeh2018:Parachutes} conducted an experimental study to `determine if using a parachute prevents death or major traumatic injury when jumping from an aircraft'.

The researchers randomised \(23\)~volunteers into one of two groups: wearing a parachute, or wearing an empty backpack. The response variable was a measurement of death or major traumatic injury upon landing. From the study, death or major injury was the same in both groups (\(0\)\% for each group). However, the study used `small stationary aircraft on the ground, suggesting cautious extrapolation to high altitude jumps' (p.~1).

Discuss the internal, external and ecological validity based on this information.
\end{exercise}

\begin{exercise}
\protect\hypertarget{exr:InterpretationSleep}{}\label{exr:InterpretationSleep}\citet{delaney2018they} examined how well hospital patients sleep at night. The researchers state that `convenience sampling was used to recruit patients' (p.~2). Later, the researchers state that (p.~7):

\begin{quote}
\ldots{} patients requiring hospitalization will likely require some daytime nap periods. This study looks at sleep only in the night-time period \(22\):\(00\)--\(07\):\(00\)h, without the context of daytime sleep considered.
\end{quote}

Discuss the internal, external and ecological validity based on this information.
\end{exercise}

\begin{exercise}
\protect\hypertarget{exr:LimitationsShopping}{}\label{exr:LimitationsShopping}\citet{botelho2019effect} examined the food choices made when subjects were asked to shop for ingredients to make a last-minute meal. Half were told to prepare a `healthy meal', and the other half told just to prepare a `meal'. The authors emphasise that the purchases were `simulated' (p.~436):

\begin{quote}
As participants did not have to pay for their selection, actual choices could be different. Participants may also have not behaved in their usual manner since they were taking part in a research study\ldots{}
\end{quote}

Discuss the internal, external and ecological validity based on this information.
\end{exercise}

\begin{exercise}
\protect\hypertarget{exr:InterpretationCoughDrops}{}\label{exr:InterpretationCoughDrops}\citet{johnson2018menthol} studied the use of over-the-counter menthol cough-drops in people with a cough. One conclusion from the \emph{observational} study of \(548\) people was that, taking `too many cough drops {[}\ldots{]} may actually make coughs more severe', as one author explained (University of Wisconsin, March 2018). Critique this statement.
\end{exercise}

\begin{exercise}
\protect\hypertarget{exr:StatementsWithErrors}{}\label{exr:StatementsWithErrors}

Suppose a group of students was studying this RQ:

\begin{quote}
Among Australians, is the average serum cholesterol concentration different for smokers and non-smokers?
\end{quote}

The students gave the following information about their study. Explain \emph{why} each of these statements is incorrect.

\begin{enumerate}
\def\labelenumi{\arabic{enumi}.}
\tightlist
\item
  The design is observational, as we cannot manipulate each person's serum cholesterol.
\item
  The outcome is `the average serum cholesterol concentration for smokers and non-smokers'.
\item
  The study is not externally valid, as the results may not apply to all people in the world.
\item
  The response variable is serum cholesterol.
\item
  In this experiment, the population is `Australians'.
\item
  The data file will have two columns: one for smokers, and one for non-smokers.
\item
  `Whether the person owns a cat' is likely to be a confounding variable.
\item
  The observer effect is not relevant, as the participants will know they are involved in a study.
\end{enumerate}

\end{exercise}

\begin{exercise}
\protect\hypertarget{exr:ValidityBeer}{}\label{exr:ValidityBeer}\citet{delarue2019taking} discuss studies where subjects rate the taste of new food products. They note that taste-testing studies should be externally and internally valid (p.~78): However, even with good internal and external validity, these studies often result in a `high rate of failures of new launched products'.

Discuss the internal, external and ecological validity based on this information.
\end{exercise}

\captionsetup{font=normalsize}

\begin{EOCanswerBox}{iconmonstr-check-mark-14-240.png}
\textbf{Answers to \emph{Quick review} questions:} \textbf{1.} True. \textbf{2.} False. \textbf{3.} False. \textbf{4.} False.

\end{EOCanswerBox}

\part{Collecting data}\label{part-collecting-data}

\chapter{Collecting data}\label{CollectingDataProcedures}

\begin{cols}
\begin{col}{0.52\textwidth}

\begin{objectivesBox}{iconmonstr-target-4-240.png}
So far, you have learnt to ask an RQ and design the study.
\textbf{In this chapter}, you will learn to:

\begin{itemize}\tightlist
  \item
  record the important steps in data collection.
  \item
  describe study protocols.
  \item
  ask questionnaire questions.
\end{itemize}
\end{objectivesBox}

\end{col}

\begin{col}{0.03\textwidth}
~
\end{col}

\begin{col}{0.45\textwidth}

\includegraphics[width=0.95\linewidth]{09-Collect_files/figure-latex/unnamed-chunk-4-1} 
\end{col}
\end{cols}

\section{Introduction}\label{ProtocolsIntro}

If the RQ is well-constructed, terms are clearly defined, and the study is well-designed and explained, then the process for collecting the data should be easy to describe. Data collection is often time-consuming, tedious and expensive, so collecting the data correctly first time is important, hence an accurate description of the data collection process is essential.

\begin{tipBox}{iconmonstr-info-6-240.png}
Data collection is often tedious, time-consuming and expensive: you usually get one chance to collect data. In contrast, data (once collected) can be analysed as many times as necessary. Design the study properly the first time!

\end{tipBox}

\section{Protocols}\label{Protocols}

\index{Protocol}\index{Data collection}

\emph{Before} collecting the data, a plan should be established and documented that explains exactly \emph{how} the data will be obtained, which will include \emph{operational definitions} (Sect.~\ref{OperationDefinitions}).\index{Definitions!operational} This plan is called a \emph{protocol}, and allows results to be confirmed and compared.

\begin{definition}[Protocol]
\protect\hypertarget{def:Protocol}{}\label{def:Protocol}A \emph{protocol} is a procedure documenting the details of the design and implementation of studies, and for data collection.
\end{definition}

A protocol usually has at least three components that describe:

\begin{enumerate}
\def\labelenumi{\arabic{enumi}.}
\tightlist
\item
  how individuals are chosen from the population (i.e., external validity).
\item
  how data are collected from the individuals (i.e., internal validity).
\item
  the types of analyses and software (including version) used.\index{Computers and software}
\end{enumerate}

Data collection often encounters problems or challenges, which should be documented also.

\begin{example}[Protocol]
\protect\hypertarget{exm:ProtocolExample}{}\label{exm:ProtocolExample}\citet{data:Romanchik2018:cookies} made cookies using puréed green peas in place of margarine (to increase the nutritional value of cookies). They assessed the acceptance of these cookies to college students.

The protocol discussed \emph{how the individuals were chosen} (p.~4):

\begin{quote}
\ldots through advertisement across campus from students attending a university in the southeastern United States.
\end{quote}

This voluntary sample comprised \(80.6\)\%~women, a higher percentage than in the general population, and in the college population. (Other extraneous variables were also recorded.)

Exclusion criteria were also applied, excluding people `with an allergy or sensitivity to an ingredient used in the preparation of the cookies' (p.~5). The researchers also described \emph{how the data were obtained}, including these steps (p.~5):

\begin{itemize}
\tightlist
\item
  tasters sat at individual tables, so they could not be influenced by other tasters.
\item
  cookies were presented individually, on individual plates.
\item
  the cookies were presented in a randomised order (fat substituted with green pea puree at \(25\)\%,~\(0\)\%,~\(50\)\%,~\(100\)\% and~\(75\)\%).
\item
  between tastes, tasters `cleansed their palates' by drinking distilled water at \(25\)\textsuperscript{o}C.
\item
  the tasters recorded (p.~5):
\end{itemize}

\begin{quote}
characteristics of color, smell, moistness, flavor, aftertaste, and overall acceptability, for each sample of cookies\ldots{}
\end{quote}

Thus, internal validity was addressed using random allocation (to manage confounding), blinding individuals (to partially manage the Hawthorne effect), and washouts (to manage the carryover effect). Details are also given of how the cookies were made, and how objective measurements (such as moisture content) were determined. Subjects were not blinded to being in a study, but were blinded to which substitution percentage was in each cookie.

The \emph{type of analyses and software used} were also given.
\end{example}

\begin{example}[Operational definitions]
\protect\hypertarget{exm:DefinitionsSex}{}\label{exm:DefinitionsSex}In a study where the sex of a person is the explanatory variable, an \emph{operational definition} for the sex of the person is needed. That is, a description is needed for \emph{how} the researchers determine the sex of the person.

One option is to \emph{ask} each individual their sex. Another option is to make an \emph{informed assessment}, based on clothing and body shape, for instance. A third option is to have subjects indicate their sex in a \emph{multiple-choice question}.

Different types of studies may require different means for determining the sex of a person. The study protocol should explain \emph{how} the sex of the person will be established.
\end{example}

Unforeseen complications are not unusual, so often a \emph{pilot study} (or a \emph{practice run}) is conducted before the actual data collection, to:\index{Pilot study}

\begin{itemize}
\tightlist
\item
  determine the feasibility of the data collection protocol.
\item
  identify unforeseen challenges.
\item
  obtain data to determine appropriate sample sizes (Sect.~\ref{EstimatingSampleSize}).
\item
  identify ways to potentially save time and money.
\end{itemize}

The pilot study may suggest changes to the protocol.

\begin{definition}[Pilot study]
\protect\hypertarget{def:PilotStudy}{}\label{def:PilotStudy}A \emph{pilot study} is a small test run of the study used to check that the protocol is appropriate and practical, and to identify (and hence fix) possible problems with the research design or protocol.
\end{definition}

The data can be collected once the protocol\index{Protocol} has been finalised. Protocols ensure studies are reproducible (Sect.~\ref{ReproducibleResearch}),\index{Research!reproducibility} so others can confirm or compare results, and others can understand exactly what was done, and how. Protocols should indicate how design aspects (such as blinding the individuals, random allocation of treatments, etc.) will happen.\index{Research design} The final \emph{protocol}, without pedantic detail, should be reported. Diagrams can be useful to support explanations. All studies should have a well-established protocol for describing how the study was done.

\begin{exampleExtra}

Consider this partial protocol, which shows honesty in describing a protocol:

\begin{quote}
Fresh cow dung was obtained from free-ranging, grass fed, and antibiotic-free Milking Shorthorn cows (\emph{Bos taurus}) in the Tilden Regional Park in Berkeley, CA.\spacex Resting cows were approached with caution and startled by loud shouting, whereupon the cows rapidly stood up, defecated, and moved away from the source of the annoyance. Dung was collected in ZipLoc bags (\(1\)~gallon), snap-frozen and stored at \(-80\)~C.

\VA{--- \citet{hare2008sepsid}, p.~10}{}
\end{quote}

\end{exampleExtra}

\section{Collecting data using questionnaires}\label{AskSurveyQuestions}

\index{Questionnaire!questions}\index{Questionnaire}\index{Survey}

\subsection{Writing questions}\label{WritingQuestions}

Collecting data using \emph{questionnaires} is common for both observational and experimental studies. Questionnaires are very difficult to do well: question wording is crucial, and surprisingly difficult \citep{fink1995survey}. Pilot testing questionnaires is crucial.

\begin{definition}[Questionnaire]
\protect\hypertarget{def:Questionnaire}{}\label{def:Questionnaire}A questionnaire is a set of questions for respondents to answer.\index{Questionnaire}
\end{definition}

\begin{tipBox}{iconmonstr-info-6-240.png}
A \emph{questionnaire} is a set of questions to obtain information from individuals.\index{Questionnaire}\index{Survey} A \emph{survey} is an entire methodology, that includes gathering data using a questionnaire, finding a sample, and other components.

\end{tipBox}

Questions in a questionnaire may be \emph{open-ended} (respondents can write their own answers) or \emph{closed} (respondents select from a small number of possible answers, as in multiple-choice questions).\index{Questionnaire!questions!open-ended}\index{Questionnaire!questions!closed} Open-ended and closed questions both have advantages and disadvantages. Answers to open-ended questions more easily lend themselves to qualitative analysis\index{Research!qualitative} and closed questions more to quantitative research.\index{Research!quantitative} This section briefly discusses writing questions.

\begin{example}[Open and closed questions]
\protect\hypertarget{exm:OpenClosedQuestions}{}\label{exm:OpenClosedQuestions}German students were asked a series of questions about microplastics \citep{raab2021conceptions}, including:

\begin{enumerate}
\def\labelenumi{\arabic{enumi}.}
\tightlist
\item
  Name sources of microplastics in the household.
\item
  In which ecosystems are microplastics in Germany? Tick the answer (multiple ticks are possible). \emph{Options}: (a)~sea; (b)~rivers; (c)~lakes; (d)~groundwater.
\item
  Assess the potential danger posed by microplastics. \emph{Options}: (a)~very dangerous; (b)~dangerous; (c)~hardly dangerous; (d)~not dangerous.
\end{enumerate}

The first question is \emph{open-ended}: respondents provide their own answers. The second question is \emph{closed}, where \emph{multiple} options can be selected. The third question is \emph{closed}, where only \emph{one} option can be selected.
\end{example}

Writing good questionnaire questions is difficult. Good practice requires:

\begin{itemize}
\tightlist
\item
  \emph{avoiding leading questions} that may lead respondents to answer a certain way.\index{Leading questions}
\item
  \emph{avoiding ambiguity} by avoiding unfamiliar or technical terms.
\item
  \emph{avoiding asking the uninformed}; avoid asking respondents about issues they don't know about. Many people will give a response even if they do not understand (and such responses are worthless). For example, people may give directions to places that do not even exist \citep{collett1976pointing}.
\item
  \emph{avoiding complex and double-barrelled questions}, which are hard to understand and the answers hard to interpret.
\item
  \emph{avoiding problems with ethics},\index{Ethics} such as questions about people breaking laws, or revealing confidential or private information. In special cases and with justification, ethics committees may allow such questions.
\item
  \emph{ensuring clarity} in question wording.
\item
  (for closed questions) ensuring options are \emph{mutually exclusive} (responses fit into only one category)\index{Mutually exclusive} and \emph{exhaustive} (categories cover \emph{all} options).\index{Exhaustive}
\end{itemize}

\begin{example}[Poor question wording]
\protect\hypertarget{exm:LeadingQns}{}\label{exm:LeadingQns}Consider a questionnaire asking these questions:

\begin{enumerate}
\def\labelenumi{\arabic{enumi}.}
\tightlist
\item
  Because bottles from bottled water create enormous amounts of non-biodegradable landfill and hence threaten native wildlife, do you support banning bottled water?
\item
  Do you drink more water now?
\item
  Are you more concerned about Coagulase-negative \emph{Staphylococcus} or \emph{Neisseria pharyngis} in bottled water?
\item
  Do you drink water in plastic and/or glass bottles?
\item
  Do you have a water tank installed illegally, without permission?
\item
  Do you avoid purchasing water in plastic bottles unless it is carbonated, unless the bottles are plastic but not necessarily if the lid is recyclable?
\end{enumerate}

Question~1 is \emph{leading} because the expected response is obvious. Better would be: `Do you support or not support banning bottled water?'

Question~2 is \emph{ambiguous}: it is unclear what `more water now' is being compared to.

Question~3 is unlikely to give sensible answers, as most people will be \emph{uninformed}. Many people will still give an opinion, but the data will be effectively useless (though the researcher may not realise).

Question~4 is \emph{double-barrelled}, and would be better asked as two separate questions (one asking about plastic bottles, and one about glass bottles).

Question~5 is unlikely to be given \emph{ethical approval} or to obtain truthful answers, as respondents are unlikely to admit to breaking rules.

Question~6 is \emph{unclear}, since knowing what a \emph{yes} or \emph{no} answer means is confusing.
\end{example}

\begin{example}[Question wording]
\protect\hypertarget{exm:QuestionWording}{}\label{exm:QuestionWording}Question \emph{wording} can be important. In the~2014 \emph{General Social Survey} (\url{https://gss.norc.org}), when white Americans were asked for their opinion of the amount America spends on \emph{welfare}, \(58\)\% of respondents answered `Too much'. However, when white Americans were asked for their opinion of the amount America spends on \emph{assistance to the poor}, only \(16\)\% of respondents answered `Too much' \citep{PovertyWording}.
\end{example}

\begin{example}[Mutually exclusive options]
\protect\hypertarget{exm:MutuallyExclusiveQns}{}\label{exm:MutuallyExclusiveQns}In a study to determine the time doctors spent with patients (from \citet{chan2008exploration}), doctors were given the options:\index{Questionnaire!questions!closed}

\begin{itemize}
\tightlist
\item
  \(0\)--\(5\,\text{mins}\).
\item
  \(5\)--\(10\,\text{mins}\).
\item
  more than \(10\,\text{mins}\).
\end{itemize}

This is a poor question, because a respondent does not know which option to select for an answer of `\(5\,\text{mins}\)'. The options are not \emph{mutually exclusive}.
\end{example}

\subsection{Challenges using questionnaires}\label{Biases}

\index{Questionnaire!biases}

Using questionnaires presents myriad challenges.

\begin{itemize}
\tightlist
\item
  \emph{Non-response bias} (Sect.~\ref{SelectionBias}):\index{Bias!selection} non-response bias is common with questionnaires, as they are often used with voluntary-response samples (Sect.~\ref{NonRandomSamples}).\index{Sampling!non-random!voluntary} The people who \emph{do not} respond to the survey may be different from those who \emph{do} respond.
\item
  \emph{Response bias} (Sect.~\ref{SelectionBias}):\index{Bias!response} people do not always answer truthfully; for example, what people \emph{say} may not correspond with what people \emph{do} (Example \ref{exm:EcologicalCups}). Sometimes this is unintentional (e.g., poor questions wording), due to embarrassment, or because questions are controversial. Sometimes, respondents repeatedly provide the same answer (without reading the question) to a series of multiple-choice questions (i.e., always select `Always').
\item
  \emph{Recall bias}:\index{Bias!recall} people may not be able to accurately recall past events clearly, or recall when they happened.
\item
  \emph{Question order}: the order of the questions can influence the responses.\index{Bias!question order}
\item
  \emph{Interpretation}: phrases and words such as `Sometimes' and `Somewhat disagree' may have different meanings to different people.\index{Bias!interpretation}
\end{itemize}

Many of these can be managed with careful questionnaire design, but discussing the methods are beyond the scope of this book.

\begin{example}
\citet{alharthy2023knowledge} studied the knowledge, attitude, and level of confidence of paramedics when managing patients with visual or hearing problems. They used a questionnaire, which was sent to (p.~5)

\begin{quote}
\ldots{} \(372\) potential participants with the expectation that at least~\(310\) questionnaires {[}would{]} be returned (\(83\)\% assumed return rate). However, only~\(97\) out of~\(372\) participants completed the questionnaire, resulting in actual return rate of \(26\)\%.
\end{quote}

Response rates from questionnaires are often very low (and unrepresentative).\index{Questionnaire!response rate}
\end{example}

\subsection{Preparing software for questionnaire data}\label{SoftwareQuestionnaires}

\index{Questionnaire!software}\index{Computers and software!data entry}

Care is needed when preparing software for data collected using a questionnaire. Sometimes, of course, the data are collected by computer (e.g., online) and are supplied to the researchers already formatted and in electronic format.

Data from open questions are usually text-based (such as words, sentences or paragraphs of text). These can generally be included in the data worksheet (though there may be a limit to the length of such data), but cannot be analysed using the quantitative methods described in this book.\index{Research!qualitative}

Closed questions are easily included in a data worksheet. In closed questions where respondents can select \emph{one} option only, one column is needed for the question that records which option was selected. In closed questions where respondents can select \emph{all} options that apply, each option requires its own column that records each respondents' answer for that option.

\begin{example}[Open and closed questions: software]
\protect\hypertarget{exm:OpenClosedQuestionsSoftware}{}\label{exm:OpenClosedQuestionsSoftware}In Example~\ref{exm:OpenClosedQuestions}, three questionnaire questions are given that were asked of German students about microplastics \citep{raab2021conceptions}. Some (artificial) data are shown entered in Fig.~\ref{fig:Microplasticsjamovi}.

The first question requires open-ended, text-based answers (\texttt{Sources}).\index{Questionnaire!questions!open-ended} For the second (closed) question, students could select \emph{multiple} options, so each option needs one column in the data worksheet (\texttt{WhereSeas} to \texttt{WhereGroundwater}).\index{Questionnaire!questions!open-ended}\index{Questionnaire!questions!closed} The third (closed) question required students to select \emph{one} option from a given list, so one column (\texttt{Danger}) is needed to record responses. As usual (Sect.~\ref{DataEntry}), each row represents one unit of analysis (student).
\end{example}

\begin{figure}[hbtp]

{\centering \includegraphics[width=1\linewidth]{DataPrep/Microplastics/MicroplasticsSetup} 

}

\caption{The data worksheet for some example data, for the microplastics study.}\label{fig:Microplasticsjamovi}
\end{figure}

\index{Research design|)}

\section{Chapter summary}\label{Chap10-Summary}

Having a detailed procedure for collecting the data (the \emph{protocol}) is important. Using a \emph{pilot study} to trial the protocol can reveal unexpected changes necessary for a good protocol. Creating good questionnaires questions is difficult, but important.

\section{Quick review questions}\label{Chap10-QuickReview}

\begin{enumerate}
\def\labelenumi{\arabic{enumi}.}
\item
  What is the biggest problem with this question: \tightlist `Do you have bromodosis?'
\item
  What is the biggest problem with this question: `Do you spend too much time connected to the internet?'
\item
  What is the biggest problem with this question: `Do you eat fruits and vegetables?'
\item
  \emph{True} or \emph{false}: A well-defined protocol allows the researchers to make the study externally valid.
\item
  \emph{True} or \emph{false}: This question is likely to be a \emph{leading} question. `Do you support a ban on drinks sold in unrecyclable plastic bottles?'
\end{enumerate}

\section{Exercises}\label{CollectionExercises}

\hyperref[Answers]{Answers to odd-numbered exercises} are given at the end of the book.

\captionsetup{font=small}

\begin{exercise}
\protect\hypertarget{exr:CollectSurveyQuestions1}{}\label{exr:CollectSurveyQuestions1}

What is the problem with this question?

\begin{quote}
What is your age? (Select one option)

\begin{itemize}
\tightlist
\item
  Under~\(18\).
\item
  Over~\(18\).
\end{itemize}
\end{quote}

\end{exercise}

\begin{exercise}
\protect\hypertarget{exr:CollectSurveyQuestions1B}{}\label{exr:CollectSurveyQuestions1B}

What is the problem with this question?

\begin{quote}
How many children do you have? (Select one option)

\begin{itemize}
\tightlist
\item
  None.
\item
  \(1\)~or~\(2\).
\item
  \(2\)~or~\(3\).
\item
  More than~\(4\).
\end{itemize}
\end{quote}

\end{exercise}

\begin{exercise}
\protect\hypertarget{exr:CollectSurveyQuestions2}{}\label{exr:CollectSurveyQuestions2}

Which of these questionnaire questions is better? Why?

\begin{enumerate}
\def\labelenumi{\arabic{enumi}.}
\tightlist
\item
  Should concerned cat owners vaccinate their pets?
\item
  Should domestic cats be required to be vaccinated or not?
\item
  Do you agree that pet-owners should have their cats vaccinated?
\end{enumerate}

\end{exercise}

\begin{exercise}
\protect\hypertarget{exr:CollectSurveyQuestions2B}{}\label{exr:CollectSurveyQuestions2B}

Which of these questionnaire questions is better? Why?

\begin{enumerate}
\def\labelenumi{\arabic{enumi}.}
\tightlist
\item
  Do you own an environmentally-friendly electric vehicle?
\item
  Do you own an electric vehicle?
\item
  Do you own or do you not own an electric vehicle?
\end{enumerate}

\end{exercise}

\begin{exercise}
\protect\hypertarget{exr:SunscreenQuestions}{}\label{exr:SunscreenQuestions}\citet{data:Falk2013:SunProtection} studied sunscreen use, and asked participants questions, including these:

\begin{itemize}
\tightlist
\item
  how often do you sun bathe with the intention to tan during the summer in Sweden? (Possible answers: never, seldom, sometimes, often, always).
\item
  how long do you usually stay in the sun between \(11\)am and \(3\)pm, during a typical day-off in the summer (June--August)? (Possible answers: \(< 30\,\text{mins}\), \(30\,\text{mins}\)--\(1\,\text{h}\), \(1\)--\(2\,\text{h}\), \(2\)--\(3\,\text{h}\), \(>3\,\text{h}\)).
\end{itemize}

Critique these questions. What biases may be present?
\end{exercise}

\begin{exercise}
\protect\hypertarget{exr:KidsEnvironmentQuestions}{}\label{exr:KidsEnvironmentQuestions}\citet{moron2021children} studied primary-school children's knowledge of their natural environment. They were asked three questions:

\begin{enumerate}
\def\labelenumi{\arabic{enumi}.}
\tightlist
\item
  Do you usually visit Guadaira Park?

  \begin{itemize}
  \tightlist
  \item
    No, I don't like parks.
  \item
    No, I don't usually visit it.
  \item
    Yes, once per week.
  \item
    Yes, more than once a week
  \end{itemize}
\item
  How many times have you visited nature (the beach, countryside, mountains, etc.) in the last month?

  \begin{itemize}
  \tightlist
  \item
    Never.
  \item
    Once.
  \item
    Two to three times.
  \item
    More than three times.
  \end{itemize}
\item
  Which is your favourite natural place?

  \begin{itemize}
  \tightlist
  \item
    Write a story.
  \item
    Draw a picture.
  \end{itemize}
\end{enumerate}

Which questions are \emph{open} and which are \emph{closed}? Which questions will produce \emph{qualitative} data? Critique the questions.
\end{exercise}

\captionsetup{font=normalsize}

\begin{EOCanswerBox}{iconmonstr-check-mark-14-240.png}
\textbf{Answers to \emph{Quick review} questions:} \textbf{1.} \emph{Language}: most people do not know what `bromodosis' is. \textbf{2.} \emph{Ambiguous}: `too much', compared to what? \textbf{3.} \emph{Double-barrelled}: some people may eat fruits but not vegetables, for example.\\
Over what time frame? \textbf{4.} False. \textbf{5.} True.

\end{EOCanswerBox}

\part{Classifying and summarising data}\label{part-classifying-and-summarising-data}

\chapter{Classifying data and variables}\label{DescribingVars}

\begin{cols}
\begin{col}{0.52\textwidth}

\begin{objectivesBox}{iconmonstr-target-4-240.png}
So far, you have learnt to ask an RQ, design a study, and collect the data.
\textbf{In this chapter}, you will learn to:

\begin{itemize}\tightlist
  \item
  identify and distinguish qualitative and quantitative variables.
  \item
  identify and distinguish nominal and ordinal qualitative variables.
  \item
  identify and distinguish continuous and discrete quantitative variables.
\end{itemize}
\end{objectivesBox}

\end{col}

\begin{col}{0.03\textwidth}
~
\end{col}

\begin{col}{0.45\textwidth}

\includegraphics[width=0.95\linewidth]{10-DescribingVariables_files/figure-latex/unnamed-chunk-4-1} 
\end{col}
\end{cols}

\section{Introduction}\label{Describing-Intro}

Understanding the type of data collected is essential before summarising or analysing,\index{Data} because the \emph{type} of data determines how to proceed. Broadly, data may be classified as either \emph{quantitative} data (Sect.~\ref{QuantData}) or \emph{qualitative} data (Sect.~\ref{QualData}). The \emph{data} are the recorded \emph{values} of the variables, so we also talk about quantitative and qualitative \emph{variables}. Quantitative variables record quantitative data; qualitative variables record qualitative data.

\begin{example}[Variables and data]
\protect\hypertarget{exm:VariablesData}{}\label{exm:VariablesData}`Age' is a \emph{variable} because age varies from individual to individual (Def.~\ref{def:Variable}). The \emph{data} may include values like \(13\)~months, \(21\)~years and \(76\)~years.
\end{example}

\begin{importantBox}{iconmonstr-warning-8-240.png}
\emph{Quantitative research} summarises and analyses data using numerical methods (Sect.~\ref{TypesOfResearch}). \emph{Quantitative research} can involve both \emph{quantitative} and \emph{qualitative} data, because both can be summarised numerically (Chaps.~\ref{SummariseQuantData} and~\ref{SummariseQualData} respectively) and analysed numerically.

\end{importantBox}

\section{Quantitative data: discrete and continuous data}\label{QuantData}

\index{Quantitative data|(}\index{Variables!quantitative|(}

\emph{Quantitative} data are mathematically numerical. Most data arising from counting or measuring are quantitative. Quantitative data often (but not always) have measurement units (such as~\emph{kg} or~\emph{cm}).\index{Units of measurement} Be careful: numerical data are not necessarily quantitative; only \emph{mathematically} numerical data are quantitative (numbers with numerical \emph{meanings}).

\begin{definition}[Quantitative data]
\protect\hypertarget{def:QuantitativeData}{}\label{def:QuantitativeData}\emph{Quantitative data} are \emph{mathematically} numerical: the numbers have numerical meaning, and represent quantities or amounts. Quantitative data generally arise from counting or measuring.
\end{definition}

\begin{example}[Quantitative data]
\protect\hypertarget{exm:QuantitativePostcodes}{}\label{exm:QuantitativePostcodes}The weight of numbats, the thickness of sheet metal, and blood pressure are all \emph{measured}, and are quantitative variables.

The number of power failures per year, the number of solar panels per home, and the number of tangelos per tree are all \emph{counts}, and are quantitative variables.

Australian postcodes are four-digit numbers, but are \emph{not} quantitative; the numbers are labels. A postcode of~4556 isn't one `better' or `more' than a postcode of~4555. The values do not have numerical \emph{meanings}. Indeed, alphabetic postcodes could have been chosen. For example, the postcode of Caboolture (Queensland) is~4510, but could have been~QCAB.
\end{example}

Quantitative data may be further classified as \emph{discrete} or \emph{continuous}.\index{Quantitative data!discrete} \emph{Discrete} quantitative data have possible values that can be \emph{counted}, at least in theory. Sometimes, the possible values may have no theoretical upper limit, yet are still considered `countable'. \emph{Continuous} quantitative data have values that cannot, at least in theory, be recorded exactly: another value can always be found between any two given values of the variable, if we \emph{measure} to a greater number of decimal places. In practice, though, values must be rounded to a reasonable number of decimal places.

\begin{definition}[Discrete data]
\protect\hypertarget{def:DiscreteData}{}\label{def:DiscreteData}\index{Quantitative data!discrete}\index{Variables!discrete quantitative} \emph{Discrete} quantitative data has a countable number of possible values between any two given values of the variable.
\end{definition}

\begin{example}[Discrete quantitative data]
\protect\hypertarget{exm:QuantDiscrete}{}\label{exm:QuantDiscrete}

These quantitative variables are \emph{discrete}:

\begin{itemize}
\tightlist
\item
  the \emph{number} of people in passenger vehicles being driven on a certain road. Possible values: \(1\),~\(2\),~\(\dots\), with an upper limit of perhaps~\(8\).
\item
  the \emph{number} of cracked eggs in a carton of~\(12\). Possible values: \(0\),~\(1\), \(2\), \(\dots\)~\(12\).
\item
  the \emph{number} of orthotic devices a person has used. Possible values: \(0\),~\(1\), \(2\),~\(\dots\)
\item
  the \emph{number} of turbine cracks after \(750\,\text{h}\) use. Possible values: \(0\),~\(1\), \(2\),~\(\dots\)
\end{itemize}

\end{example}

\begin{definition}[Continuous data]
\protect\hypertarget{def:ContinuousData}{}\label{def:ContinuousData}\index{Quantitative data!continuous}\index{Variables!continuous quantitative} \emph{Continuous} quantitative data have (at least in theory) an infinite number of possible values between any two given values.
\end{definition}

Height is continuous: between the heights of~\(179\,\text{cm}\) and~\(180\,\text{cm}\), many heights exist, depending on how many decimal places are used to record height. In practice, however, heights are usually rounded to the nearest centimetre for convenience. All continuous data are rounded.

\begin{example}[Continuous quantitative data]
\protect\hypertarget{exm:QuantContinuous}{}\label{exm:QuantContinuous}

These quantitative variables are \emph{continuous}:

\begin{itemize}
\tightlist
\item
  the \emph{weight} of \(6\)-year-old Fijian children. Values exist between any two given values of weight, by measuring to more decimal places of a kilogram. However, weights are usually reported to the nearest kilogram.
\item
  the \emph{energy consumption} of houses in London. Values exist between any two given values of energy consumption, by measuring to more and more decimal places of a kiloWatt-hour (kWh). Consumption would usually be given to the nearest~kWh.
\item
  the \emph{time} spent in front of a computer each day for employees in a given industry. Values exist between any two given times, by measuring to more decimal places of a second. The values may be reported to the nearest minute, or the nearest~\(15\,\text{mins}\).
\end{itemize}

\end{example}

Sometimes, discrete quantitative data with a very large number of possible values may be treated as continuous.

\begin{example}[Treating discrete data as continuous]
\protect\hypertarget{exm:DiscreteAsContinuous}{}\label{exm:DiscreteAsContinuous}Annual income is discrete, since no income is between \$\(80\,000.00\) and \$\(80\,000.01\). However, annual incomes are much larger than cents, and vary at scales much greater than cents, and so are often treated as continuous.
\end{example}

\index{Quantitative data|)}\index{Variables!quantitative|)}

\section{Qualitative data: nominal and ordinal data}\label{QualData}

\index{Qualitative data|(}\index{Variables!qualitative|(}

\emph{Qualitative} data has distinct labels or categories, and are not mathematically numerical. Be careful: \emph{numerical} data may be qualitative if those numbers don't have numerical \emph{meanings}. The categories of a qualitative variable are called the \emph{levels} or the \emph{values} of the variable.

\begin{definition}[Qualitative data]
\protect\hypertarget{def:QualitativeData}{}\label{def:QualitativeData}\emph{Qualitative data} are not \emph{mathematically} numerical data: they comprise mutually exclusive (and usually exhaustive) categories or labels.\index{Mutually exclusive}\index{Exhaustive}
\end{definition}

\begin{definition}[Levels]
\protect\hypertarget{def:Levels}{}\label{def:Levels}\index{Qualitative data!levels} The \emph{levels} (or the \emph{values}) of a qualitative variable refer to the names of the distinct categories.
\end{definition}

\begin{example}[Qualitative data]
\protect\hypertarget{exm:QualData}{}\label{exm:QualData}\index{Qualitative data!levels} `Brand of mobile phone' is a variable (as `brand' varies from phone to phone) that is qualitative. Many levels (i.e., brands) are possible, but could be simplified by using the levels as `Apple', `Samsung', `Google' and `Other'.
\end{example}

\begin{example}[Qualitative data]
\protect\hypertarget{exm:QualData2}{}\label{exm:QualData2}Social Security Numbers (\textsc{ssn}) in the US are nine-digit numbers unique to each individual. The first three digits represent geographic regions; the next two digits are assigned to groups in that region. The last four digits are unique to individuals.

Although the \textsc{ssn} is a nine-digit number, \textsc{ssn} is a qualitative variable.
\end{example}

\begin{example}[Clarity in variables]
\protect\hypertarget{exm:DefinitionsClarity}{}\label{exm:DefinitionsClarity}`Age' is a \emph{continuous quantitative} variable, since age could be measured to many decimal places of a second. Age is usually rounded down to the number of completed years, for convenience. However, the age of young children may be given as `\(3\)~days' or `\(10\)~months'.

Sometimes \emph{Age group} is used (such as Under~\(20\); \(20\) to under~\(50\); \(50\)~or over) instead of Age. `Age group' is \emph{qualitative}. Ensure you are clear about which is used.
\end{example}

\begin{example}[Levels]
\protect\hypertarget{exm:DefinitionsLevels}{}\label{exm:DefinitionsLevels}The levels of a variable depend on how the variable is defined. For example, the variable `How does the person commute to work' may have two levels: `Using public transport' and `Not using public transport'.

Alternatively, the variable could be written as `Does the person use public transport to commute to work?' For this variable, the levels are `Yes' and `No'.
\end{example}

Qualitative data can be further classified as \emph{nominal} or \emph{ordinal}.

\begin{definition}[Nominal qualitative variables]
\protect\hypertarget{def:Nominal}{}\label{def:Nominal}\index{Qualitative data!nominal}\index{Variables!nominal qualitative} A \emph{nominal} qualitative variable is a qualitative variable where the levels\index{Qualitative data!levels} \emph{do not} have a natural order.
\end{definition}

\begin{definition}[Ordinal qualitative variables]
\protect\hypertarget{def:Ordinal}{}\label{def:Ordinal}\index{Qualitative data!ordinal}\index{Variables!ordinal qualitative} An \emph{ordinal} qualitative variable is a qualitative variable where the levels\index{Qualitative data!levels} \emph{do} have a natural order.
\end{definition}

\begin{example}[Nominal and ordinal data]
\protect\hypertarget{exm:NominalOrdinal}{}\label{exm:NominalOrdinal}\emph{Blood type} is qualitative with four levels: Type~A; Type~B; Type~AB; Type~O.\spacex These levels have no natural order; they can be ordered alphabetically, or by prevalence. \emph{Blood type} is nominal.

\emph{Age group} could be listed with levels Under~\(20\); \(20\)~to under \(50\); \(50\)~or over. These levels have a natural order: youngest to oldest. \emph{Age group} is ordinal.
\end{example}

\begin{example}[Ordinal data]
\protect\hypertarget{exm:OrdinalData}{}\label{exm:OrdinalData}Consider this questionnaire question:

\begin{quote}
Please indicate the extent to which you agree or disagree with this statement: `Vaping should be banned'.\smallskip  

Strongly disagree;~~Disagree;~~Neither agree nor disagree;~~Agree;~~Strongly agree.
\end{quote}

The responses will be \emph{ordinal} with five levels. Giving the levels in the given order (or the reverse order) makes sense; giving the levels in alphabetical order, for example, would be very confusing. The levels have a natural order.
\end{example}

\begin{example}[Types of variables]
\protect\hypertarget{exm:TypesVariables}{}\label{exm:TypesVariables}Consider a study to determine if the weight of \(500\,\text{g}\) bags of pasta actually weigh~\(500\,\text{g}\) (or more) on average. One approach is to record the weight of pasta in each bag (a \emph{quantitative} variable), and compare the \emph{average} weight to the target weight of \(500\,\text{g}\).

Another approach is to record whether each bag of pasta was underweight using a balance scale. This variable would be \emph{qualitative}, with two \emph{levels} (underweight; not underweight). The \emph{percentage} of underweight bags could be reported.
\end{example}

\begin{softwareBox}{iconmonstr-laptop-4-240.png}
Most \emph{statistical} software requires variables to be classified as quantitative or qualitative (and perhaps discrete or continuous; ordinal or nominal).\index{Computers and software!statistical} This enables the software to produce appropriate output and suggest appropriate analyses.

\end{softwareBox}

\index{Qualitative data|)}\index{Variables!qualitative|)}

\clearpage

\section{Example: water access}\label{WaterAccessVariables}

\citet{lopez2022farmers} studied three rural communities in Cameroon, and recorded information about their access to water. The study could be used to determine contributors to the incidence of diarrhoea in young children (\(85\) households had children under~\(5\) years of age). The variables in the \texttt{WaterAccess} dataset are classified in Tables~\ref{tab:WaterAccessQualVariables} and~\ref{tab:WaterAccessQuantVariables}.

\begin{table}
\centering
\caption{\label{tab:WaterAccessQualVariables}The qualitative variables in the water-access dataset.}
\centering
\fontsize{8}{10}\selectfont
\begin{tabular}[t]{>{\raggedleft\arraybackslash}p{47mm}>{\centering\arraybackslash}p{11mm}>{\raggedright\arraybackslash}p{64mm}}
\toprule
\textbf{Qualitative variable} & \textbf{Type} & \textbf{Levels}\\
\midrule
Region & Nominal & Mbeng; Mbih; Ntsingbeu\\
Education & Ordinal & Primary or less; Secondary or higher\\
Distance to water source & Ordinal & Under $100\,\text{m}$; $100\,\text{m}$\ to\ $1000\,\text{m}$; over $1000\,\text{m}$\\
Queuing time at water source & Ordinal & Under $5\,\text{mins}$; $5$ to\ $15\,\text{mins}$; Over $15\,\text{mins}$\\
Household has a garden & Nominal & Yes; No\\
\addlinespace
Household keeps livestock & Nominal & Yes; No\\
Water source & Nominal & Well; Bore; Tap; River\\
How often water container washed & Ordinal & Before each fill; Once per week; Once per month\\
Diarrhoea in children under\ $5$ & Nominal & Yes; No\\
\bottomrule
\end{tabular}
\end{table}

\begin{table}
\centering
\caption{\label{tab:WaterAccessQuantVariables}The quantitative variables in the water-access dataset.}
\centering
\fontsize{8}{10}\selectfont
\begin{tabular}[t]{rcl}
\toprule
\textbf{Quantitative variable} & \textbf{Type} & \textbf{Extra information}\\
\midrule
Household coordinator's (woman's) age & Continuous & Rounded to nearest year\\
Number of people in household & Discrete & \\
Number of children under\ $5$ in household & Discrete & \\
\bottomrule
\end{tabular}
\end{table}

\section{Chapter summary}\label{Describing-Summary}

The \emph{type} of data collected determines the types of summaries and analyses that are needed. Data and variables can be classified as either:

\begin{itemize}
\tightlist
\item
  \emph{quantitative} (\emph{discrete} or \emph{continuous}) if mathematically numerical.
\item
  \emph{qualitative} (\emph{nominal} or \emph{ordinal}) if not mathematically numerical.
\end{itemize}

\section{Quick review questions}\label{Chap11-QuickReview}

\citet{benetou2020diet} studied school-aged adolescents in Greece. Among other variables, for each child they recorded the body-mass index (weight, divided by height-squared), diet quality (poor; moderate; good), the region where they lived (Attica; Thessaloniki; Other), the number of days they performed physical exercise in the last week, and school grade.

Are the following statements \emph{true} or \emph{false}?

\begin{enumerate}
\def\labelenumi{\arabic{enumi}.}
\item
  `Body-mass index' is a quantitative discrete variable.\tightlist
\item
  `Diet quality' is a qualitative ordinal variable.
\item
  `Region of residence' is a qualitative nominal variable.
\item
  `Number of days the child performed physical exercise in the last week' is a quantitative discrete variable.
\item
  `School grade' is a quantitative continuous variable.
\end{enumerate}

\section{Exercises}\label{DescribeExercises}

\hyperref[Answers]{Answers to odd-numbered exercises} are given at the end of the book.

\captionsetup{font=small}

\begin{exercise}
\protect\hypertarget{exr:DescribeClassifying1}{}\label{exr:DescribeClassifying1}

Classify these variables as quantitative (discrete or continuous) or qualitative (nominal or ordinal).

\begin{enumerate}
\def\labelenumi{\arabic{enumi}.}
\tightlist
\item
  The knee-flex angle after treatment. \tightlist
\item
  Whether laser drilling of small holes in concrete is successful.
\item
  Length of time between arrival at an emergency department, and admission.
\item
  Telephone numbers.
\end{enumerate}

\end{exercise}

\begin{exercise}
\protect\hypertarget{exr:DescribeClassifying1B}{}\label{exr:DescribeClassifying1B}

Classify these variables as quantitative (discrete or continuous) or qualitative (nominal or ordinal).

\begin{enumerate}
\def\labelenumi{\arabic{enumi}.}
\tightlist
\item
  Number of eggs laid by female brush turkeys. \tightlist
\item
  Whether a child eats the recommended serving of fruit each day.
\item
  Bar code numbers on supermarket products.
\item
  The breed of dog used for koala detection.
\end{enumerate}

\end{exercise}

\begin{exercise}
\protect\hypertarget{exr:DescribeClassifying2}{}\label{exr:DescribeClassifying2}

True or false: these variables are \emph{qualitative} and \emph{nominal}.

\begin{enumerate}
\def\labelenumi{\arabic{enumi}.}
\item
  The age group of respondents to a survey. \tightlist
\item
  Whether a cyclist is wearing a helmet or not.
\item
  The dosage of a medication applied: \(40\),~\(60\) or~\(80\,\text{mg}\) per day.
\end{enumerate}

\end{exercise}

\begin{exercise}
\protect\hypertarget{exr:DescribeClassifying2B}{}\label{exr:DescribeClassifying2B}

True or false: these variables are \emph{qualitative} and \emph{ordinal}.

\begin{enumerate}
\def\labelenumi{\arabic{enumi}.}
\item
  The brand of fertiliser being applied.
\item
  The age of trees.
\item
  Highest level of education (never finished school; primary school; secondary school; beyond secondary school).
\end{enumerate}

\end{exercise}

\begin{exercise}
\protect\hypertarget{exr:DescribeClassifying3}{}\label{exr:DescribeClassifying3}

A study recorded whether people (who were not swimming) were wearing head-protection at the beach. The results were recorded as None; Cap; or~Hat. Which of the following words could be used to classify this variable:

Nominal;\enskip qualitative;\enskip continuous;\enskip quantitative;\enskip ordinal.

\end{exercise}

\begin{exercise}
\protect\hypertarget{exr:DescribeClassifyingGraphsLimeTrees}{}\label{exr:DescribeClassifyingGraphsLimeTrees}\citet{data:ForestBiomass2017} studied lime trees (\emph{Tilia cordata}), and recorded these variables for \(385\)~trees in Russia: the foliage biomass (in~kg); the tree diameter (in~cm); the age of the tree (in~years); and the origin of the tree (one of Coppice, Natural, or Planted).

Classify the variables in the study using the language of this chapter.
\end{exercise}

\begin{exercise}
\protect\hypertarget{exr:VariablesLevelsA}{}\label{exr:VariablesLevelsA}

A study is comparing the proportion of females and males who wear hats between \(10\)am and \(2\)pm. Which one of these could be the \emph{explanatory} variable?

\begin{itemize}
\tightlist
\item
  The sex of the person.
\item
  `Female' and `male'.
\item
  The percentage of people who are female.
\end{itemize}

\end{exercise}

\begin{exercise}
\protect\hypertarget{exr:VariablesLevelsB}{}\label{exr:VariablesLevelsB}

A study is comparing the proportion of older women (aged~\(40+\)) and younger women (under~\(40\)) who work full-time. Which one of these could be the explanatory variable?

\begin{itemize}
\tightlist
\item
  `Full-time' and `part-time'.
\item
  The percentage of women who are aged under~\(40\).
\item
  Whether a woman is aged under \(40\).
\item
  `Yes' and `No'.
\end{itemize}

\end{exercise}

\begin{exercise}
\protect\hypertarget{exr:DescribeClassifyingVariables1}{}\label{exr:DescribeClassifyingVariables1}

Are these variables quantitative (discrete or continuous; what units of measurement), or qualitative (nominal or ordinal, and with what levels?)?

\begin{enumerate}
\def\labelenumi{\arabic{enumi}.}
\tightlist
\item
  Systolic blood pressure.
\item
  Diet (vegan; vegetarian; neither vegan nor vegetarian).
\item
  Socioeconomic status (low income; middle income; high income).
\item
  Number of times a person visited the doctor last year.
\end{enumerate}

\end{exercise}

\begin{exercise}
\protect\hypertarget{exr:DescribeClassifyingVariables2}{}\label{exr:DescribeClassifyingVariables2}\citet{data:Alley2017:SocialMedia} studied body-mass index and its relationship with use of social media, and recorded these variables (among others) from a group of \(1\,140\) participants:

\begin{enumerate}
\def\labelenumi{\arabic{enumi}.}
\tightlist
\item
  age (under \(45\); \(45\)~to \(64\); \(65\)~or over).
\item
  gender (male; female).
\item
  location (urban; rural).
\item
  social media use (none; low; high).
\item
  total sitting time, in minutes~per day.
\end{enumerate}

For each variable, classify the \emph{type} of variable: quantitative (discrete or continuous; what units of measurement?), or qualitative (nominal or ordinal; what levels?).
\end{exercise}

\begin{exercise}
\protect\hypertarget{exr:BRFSS}{}\label{exr:BRFSS}The \emph{Behavioral Risk Factor Surveillance System} (\textsc{brfss}; \citet{data:BRFSS}) survey collects data annually in all~\(50\)~US states, the District of Columbia and three US territories, from more than \(400\,000\) adults each year. The following questions, among many others, appear in the 2023~\textsc{brfss} survey.

\begin{enumerate}
\def\labelenumi{\arabic{enumi}.}
\tightlist
\item
  Do you own or rent your home? (Options: Own, Rent; Other.)
\item
  How many children less than~\(18\) years of age live in your household?
\item
  How many cell (mobile) phones do you have for personal use? (Options: \(1\);~\(2\);~\(3\);~\(4\);~\(5\); \(6\)~or more.)
\item
  Have you ever served on active duty in the United States Armed Forces? (Options: Yes; No.)
\item
  About how much do you weigh without shoes?
\end{enumerate}

Classify the type of data collected from each question.
\end{exercise}

\begin{exercise}
\protect\hypertarget{exr:NHANES}{}\label{exr:NHANES}The \emph{National Health and Nutrition Examination Survey} (\textsc{nhanes}; \citet{data:NHANES3:Data}):

\begin{quote}
\ldots{} examines a nationally representative sample of about \(5\,000\) persons each year\ldots{}
\end{quote}

The following questions, among many others, appear in the 2021--2023 \textsc{nhanes} survey, and are asked about the person selected in the household (SP) to complete the questionnaire.

\begin{enumerate}
\def\labelenumi{\arabic{enumi}.}
\tightlist
\item
  Do you consider SP now to be overweight, underweight, or about the right weight?
\item
  How many rooms are in SP's home? (Count the kitchen and do not count any bathrooms, or an unfinished basement, or a laundry room.)
\item
  How many people who live in SP's home smoke cigarettes, cigars, little cigars, pipes, water pipes, hookah, or any other tobacco product?
\item
  Has SP ever been told by a doctor or other health professional that SP had asthma? (Options: Yes; No; Don't know.)
\item
  Overall, how would SP rate the health of SP's teeth and gums? (Options: Excellent; Very good; Good; Fair; Poor.)
\end{enumerate}

Classify the type of data collected from each question.
\end{exercise}

\begin{exercise}
\protect\hypertarget{exr:DescribeClassifyingOrthoses}{}\label{exr:DescribeClassifyingOrthoses}\citet{swinnen2018influence} studied the use of ankle-foot orthoses in children with cerebral palsy. Table~\ref{tab:DescribeAnkleFoot} give the data for the \(15\) subjects. (\textsc{gmfcs} is the Gross Motor Function Classification System describing the impact of cerebral palsy on motor function; \emph{lower} levels mean \emph{better} functionality.) Classify the variables in the study using the language of this chapter.
\end{exercise}

\begin{table}
\centering
\caption{\label{tab:DescribeAnkleFoot}The orthoses dataset.}
\centering
\fontsize{8}{10}\selectfont
\begin{tabular}[t]{ccccc}
\toprule
\textbf{Gender} & \textbf{Age (years)} & \textbf{Height (cm)} & \textbf{Weight (kg)} & \textbf{GMFCS}\\
\midrule
M & $\phantom{0}9$ & $136$ & $34.5$ & $1$\\
M & $\phantom{0}7$ & $106$ & $16.2$ & $2$\\
M & $\phantom{0}7$ & $129$ & $21.1$ & $1$\\
M & $12$ & $152$ & $40.4$ & $1$\\
M & $11$ & $146$ & $39.3$ & $2$\\
\addlinespace
M & $\phantom{0}5$ & $113$ & $18.1$ & $1$\\
M & $\phantom{0}6$ & $112$ & $16.7$ & $2$\\
M & $\phantom{0}8$ & $112$ & $19.1$ & $1$\\
M & $\phantom{0}8$ & $138$ & $28.6$ & $1$\\
M & $\phantom{0}6$ & $116$ & $19.3$ & $1$\\
\addlinespace
F & $\phantom{0}7$ & $113$ & $17.6$ & $1$\\
M & $11$ & $141$ & $34.9$ & $1$\\
M & $\phantom{0}7$ & $136$ & $34.5$ & $1$\\
F & $\phantom{0}9$ & $128$ & $21.9$ & $1$\\
F & $\phantom{0}8$ & $133$ & $23.0$ & $1$\\
\bottomrule
\end{tabular}
\end{table}

\begin{exercise}
\protect\hypertarget{exr:DescribeClassifyingNitrogenInSoil}{}\label{exr:DescribeClassifyingNitrogenInSoil}\citet{data:Lane2002:GLMsoilscience} studied fertiliser use, and recorded the soil nitrogen after applying different fertiliser doses. These variables were recorded for each field:

\begin{enumerate}
\def\labelenumi{\arabic{enumi}.}
\tightlist
\item
  the \emph{fertiliser dose}, in kilograms of nitrogen per hectare;
\item
  the \emph{soil nitrogen}, in kilograms of nitrogen per hectare; and
\item
  the \emph{fertiliser source}; one of `inorganic' or `organic'.
\end{enumerate}

Classify the variables in the study.
\end{exercise}

\begin{exercise}
\protect\hypertarget{exr:DescribeClassifyingKangaroos}{}\label{exr:DescribeClassifyingKangaroos}\citet{brunton2019fright} recorded the response of kangaroos to overhead drones (one of `No vigilance', `Vigilance', `Flee~\(<10\)\,\text{m}', or `Flee~\(>10\,\text{m}\)') and the altitude of the drone (\(30\,\text{m}\), \(60\,\text{m}\), \(100\,\text{m}\) or~\(120\,\text{m}\)). The mob size and sex of the kangaroo was also recorded. Classify the variables in the study.
\end{exercise}

\begin{exercise}
\protect\hypertarget{exr:DescribeSelfieDeaths}{}\label{exr:DescribeSelfieDeaths}

\citet{data:Dokur2018:SelfieDeaths} studied people who died while taking selfies, and recorded the data in Table~\ref{tab:TableSelfieDeaths}. Which of the following are the \emph{variables} in the table? For each that is a variable, classify the variable.

\begin{enumerate}
\def\labelenumi{\arabic{enumi}.}
\tightlist
\item
  The location.
\item
  The number of people who died at each location.
\item
  The percentage of people who died at each location.
\end{enumerate}

\end{exercise}

\begin{table}
\centering
\caption{\label{tab:TableSelfieDeaths}Locations of people dying while taking selfies.}
\centering
\fontsize{8}{10}\selectfont
\begin{tabular}[t]{lcc}
\toprule
\textbf{ } & \textbf{Number} & \textbf{Percentage}\\
\midrule
Nature, associated environments & $48$ & $43.2$\\
Train, railway, associated structures & $22$ & $19.9$\\
\addlinespace
Buildings, associated structures & $17$ & $15.3$\\
Road, bridge, associated structures & $12$ & $10.8$\\
Dams, associated structures & $\phantom{0}7$ & $\phantom{0}6.3$\\
\addlinespace
Fields, farms, associated structures & $\phantom{0}4$ & $\phantom{0}3.6$\\
Others & $\phantom{0}1$ & $\phantom{0}0.9$\\
\bottomrule
\end{tabular}
\end{table}

\captionsetup{font=normalsize}

\begin{EOCanswerBox}{iconmonstr-check-mark-14-240.png}
\textbf{Answers to \emph{Quick review} questions:} \textbf{1.} False (quantitative continuous). \textbf{2.} True. \textbf{3.} True. \textbf{4.} True \textbf{5.} False (qualitative ordinal).

\end{EOCanswerBox}

\chapter{Summarising quantitative data}\label{SummariseQuantData}

\index{Quantitative data!summarising|(}

\begin{cols}
\begin{col}{0.52\textwidth}

\begin{objectivesBox}{iconmonstr-target-4-240.png}
So far, you have learnt to ask an RQ, design a study, collect the data, and classify the data.
\textbf{In this chapter}, you will learn to:

\begin{itemize}\tightlist
  \item
  summarise quantitative data using the appropriate graphs.
  \item
  summarise quantitative data using shape, average, variation and unusual features.
\end{itemize}
\end{objectivesBox}

\end{col}

\begin{col}{0.03\textwidth}
~
\end{col}

\begin{col}{0.45\textwidth}

\includegraphics[width=0.95\linewidth]{11-SummaryQuant_files/figure-latex/unnamed-chunk-23-1} 
\end{col}
\end{cols}

\section{Introduction}\label{Summarise-Quant}

Many quantitative research studies involve quantitative variables. Except for very small amounts of data, understanding the data is difficult without a summary. Quantitative data can be summarised by knowing how often various values of the variable appear. This is called the \emph{distribution} of the data.

\begin{definition}[Distribution]
\protect\hypertarget{def:Distribution}{}\label{def:Distribution}The \emph{distribution}\index{Distribution} of a variable describes what values are present in the data, and how often those values appear.
\end{definition}

The distribution can be displayed using a frequency table (Sect.~\ref{QuantFreqTable}) or a graph (Sect.~\ref{QuantitativeGraphs}). The distribution of quantitative data can be described by the shape (Sect.~\ref{SummaryShape}), and summarised numerically by computing the average value (Sect.~\ref{ComputeAverage}), computing the amount of variation (Sect.~\ref{Variation}), and identifying outliers (Sect.~\ref{SummaryOutliers}).

\section{Frequency tables for quantitative data}\label{QuantFreqTable}

\index{Quantitative data!frequency tables}\index{Distribution!quantitative data}

Quantitative data can be collated in a \emph{frequency table}\index{Frequency table!quantitative data} by grouping the variables into appropriate intervals (`bins').\index{Frequency table!bins} The intervals should be \emph{exhaustive} (cover all values) and \emph{mutually exclusive} (observations belong to one and only one category). While not essential, usually the categories have equal width.

A frequency table for \emph{discrete} quantitative data uses bins defined to contain single discrete values, or a small number of discrete values.

\begin{example}[Frequency table: discrete data]
\protect\hypertarget{exm:CyclonesTable}{}\label{exm:CyclonesTable}The data in Table~\ref{tab:CycloneDataLATEX} show the number of severe cyclones in the Australian region, for each year from~1969 to~2005.

A frequency table can be constructed by binning each discrete value individually (Table~\ref{tab:CycloneFrequencyTable}, left table) or grouped in pairs (Table~\ref{tab:CycloneFrequencyTable}, right table). The table gives the number of years, and the corresponding percentages, that recorded the given number of cyclones.
\end{example}

\begin{table} \centering \centering\caption{\label{tab:CycloneDataLATEX}The first five and last five observations (of $37$) of the number of severe cyclones recorded in the Australian region for each year.}

\fontsize{8}{10}\selectfont
\begin{tabular}{cc}
\toprule
\textbf{Year} & \textbf{Cyclones recorded}\\
\midrule
$1969$ & $3$\\
$1970$ & $3$\\
$1971$ & $9$\\
$1972$ & $6$\\
$1973$ & $4$\\
$\vdots$ & $\vdots$\\
\bottomrule
\end{tabular} \qquad\qquad 
\begin{tabular}{cc}
\toprule
\textbf{Year} & \textbf{Cyclones recorded}\\
\midrule
$\vdots$ & $\vdots$\\
$2001$ & $3$\\
$2002$ & $3$\\
$2003$ & $5$\\
$2004$ & $5$\\
$2005$ & $8$\\
\bottomrule
\end{tabular}
\end{table}

\begin{table} \centering \centering\caption{\label{tab:CycloneFrequencyTable}Two frequency tables for the severe cyclone data.}

\fontsize{8}{10}\selectfont
\begin{tabular}[t]{rcc}
\toprule
\multicolumn{1}{c}{\textbf{Cyclones}} & \multicolumn{1}{c}{\textbf{Num. of}} & \multicolumn{1}{c}{\textbf{Percentage of}} \\
\textbf{recorded} & \textbf{years} & \textbf{years}\\
\midrule
$3$ cyclones & $\phantom{0}8$ & $22$\\
$4$ cyclones & $10$ & $27$\\
$5$ cyclones & $\phantom{0}3$ & $\phantom{0}8$\\
\addlinespace
$6$ cyclones & $\phantom{0}5$ & $14$\\
$7$ cyclones & $\phantom{0}2$ & $\phantom{0}5$\\
$8$ cyclones & $\phantom{0}4$ & $11$\\
\addlinespace
$9$ cyclones & $\phantom{0}4$ & $11$\\
$10$ cyclones & $\phantom{0}0$ & $\phantom{0}0$\\
$11$ cyclones & $\phantom{0}1$ & $\phantom{0}3$\\
\bottomrule
\end{tabular} \qquad 
\begin{tabular}[t]{rcc}
\toprule
\multicolumn{1}{c}{\textbf{Cyclones}} & \multicolumn{1}{c}{\textbf{Number of}} & \multicolumn{1}{c}{\textbf{Percentage of}} \\
\textbf{recorded} & \textbf{years} & \textbf{years}\\
\midrule
$3$ or $4$ cyclones & $18$ & $49$\\
$5$ or $6$ cyclones & $\phantom{0}8$ & $22$\\
$7$ or $8$ cyclones & $\phantom{0}6$ & $16$\\
\addlinespace
$9$ or $10$ cyclones & $\phantom{0}4$ & $11$\\
$11$ cyclones & $\phantom{0}1$ & $\phantom{0}3$\\
\bottomrule
\end{tabular}
\end{table}

For \emph{continuous} data, care is needed when creating frequency tables. Bins must be carefully constructed, since all continuous data are rounded. The bins should be defined to ensure no values lie on the border between bins, and hence creating ambiguity.

\begin{example}[Frequency table: continuous data]
\protect\hypertarget{exm:BabyBoomTable}{}\label{exm:BabyBoomTable}Table~\ref{tab:BabyBoomDataLATEX} give the weights of babies born in a hospital on one day \citep{mypapers:Dunn:dataset:1999, data:Steele:BabyBoom}, plus the gender of each baby, and the number of minutes after midnight of the birth (shown in birth order).

To display the distribution of birth weights, the weights can be grouped into clearly-defined weight intervals (Table~\ref{tab:BabyBoomTable}, left column). Alternatively, the breaks between the bins can be given to one more decimal place than the data to avoid observations landing exactly on the bin divisions (final column).

The table also gives percentage of births in each bin; for example, the percentage of babies over~\(4.0\,\text{kg}\) is \(1/44 \times 100 = 2.27\)\%, or about~\(2\)\%. Most babies in the sample are between~\(3\) and~\(4\,\text{kg}\) at birth.
\end{example}

\begin{table} \centering \centering\caption{\label{tab:BabyBoomDataLATEX}The first nine observations (of $n = 44$) of the baby-births data: the number of babies born in a Brisbane (Australia) hospital on one specific day. The `birth time' is the number of minutes after midnight.}

\fontsize{8}{10}\selectfont
\begin{tabular}{ccc}
\toprule
\textbf{Gender} & \textbf{Weight (kg)} & \textbf{Birth time}\\
\midrule
Female & $3.8$ & $\phantom{0}\phantom{0}\phantom{0}5$\\
Female & $3.3$ & $\phantom{0}\phantom{0}64$\\
Male & $3.6$ & $\phantom{0}\phantom{0}78$\\
Male & $3.8$ & $\phantom{0}115$\\
Male & $3.6$ & $\phantom{0}177$\\
\bottomrule
\end{tabular} \quad 
\begin{tabular}{ccc}
\toprule
\textbf{Gender} & \textbf{Weight (kg)} & \textbf{Birth time}\\
\midrule
Female & $2.2$ & $\phantom{0}245$\\
Female & $1.7$ & $\phantom{0}247$\\
Male & $2.8$ & $\phantom{0}262$\\
Male & $3.2$ & $\phantom{0}271$\\
$\vdots$ & $\vdots$ & $\vdots$\\
\bottomrule
\end{tabular}
\end{table}

\begin{table}
\centering
\caption{\label{tab:BabyBoomTable}The baby-weights data, displayed in a frequency table. The first and last columns show two different (but equivalent) ways to group the data.}
\centering
\fontsize{8}{10}\selectfont
\begin{tabular}[t]{cccc}
\toprule
\textbf{Weight group} & \textbf{Number of babies} & \textbf{Percentage of babies} & \textbf{Alterative weight group}\\
\midrule
$1.5$\,\text{kg}\ to under $2.0$\,\text{kg}& $\phantom{0}1$ & $\phantom{0}2$ & $1.45$\,\text{kg}\ to $1.95$\,\text{kg}\\
$2.0$\,\text{kg}\ to under $2.5$\,\text{kg}& $\phantom{0}4$ & $\phantom{0}9$ & $1.95$\,\text{kg}\ to $2.45$\,\text{kg}\\
$2.5$\,\text{kg}\ to under $3.0$\,\text{kg}& $\phantom{0}4$ & $\phantom{0}9$ & $2.45$\,\text{kg}\ to $2.95$\,\text{kg}\\
\addlinespace
$3.0$\,\text{kg}\ to under $3.5$\,\text{kg}& $17$ & $39$ & $2.95$\,\text{kg}\ to $3.45$\,\text{kg}\\
$3.5$\,\text{kg}\ to under $4.0$\,\text{kg}& $17$ & $39$ & $3.45$\,\text{kg}\ to $3.95$\,\text{kg}\\
$4.0$\,\text{kg}\ to under $4.5$\,\text{kg}& $\phantom{0}1$ & $\phantom{0}2$ & $3.95$\,\text{kg}\ to $4.45$\,\text{kg}\\
\bottomrule
\end{tabular}
\end{table}

Sometimes trial and error is needed to find useful intervals for continuous data. Usually, but not universally, the intervals \emph{include} values at the lower end of the interval, but \emph{exclude} values at the upper end (as in Table~\ref{tab:BabyBoomTable}).

\section{Graphs for quantitative data}\label{QuantitativeGraphs}

\index{Quantitative data!graphs|(}\index{Software output!graphs}

The graphs in this section are appropriate for \emph{continuous} quantitative data, and sometimes for \emph{discrete} quantitative data if many values are possible. Sometimes, \emph{discrete} data with very few recorded values are better displayed using graphs designed for qualitative data (Sect.~\ref{QualitativeGraphs}).

Graphs used to display the distribution of one quantitative variable include:

\begin{itemize}
\tightlist
\item
  \emph{histograms} (Sect.~\ref{Histograms}), which are best for moderate to large amounts of data.
\item
  \emph{stemplots} (Sect.~\ref{StemAndLeafPlots}), which are best for small amounts of data, and are only sometimes useful.
\item
  \emph{dot charts} (Sect.~\ref{DotChartsOneVar}), which are used for small to moderate amounts of data.
\end{itemize}

\begin{importantBox}{iconmonstr-warning-8-240.png}
The purpose of a graph is to display the information in the clearest, simplest possible way, to facilitate understanding the message(s) in the data.

\end{importantBox}

\subsection{Histograms}\label{Histograms}

\index{Graphs!histogram}

Histograms are a series of boxes, where the width of the box represents an interval of \emph{values} of the variable being graphed, and the height of the box represents the \emph{number} (or \emph{percentage}) of observations within that range of values.\footnote{Actually, the \emph{area} of the box is proportional to the number of observations. We only consider histograms where the boxes have the same width, so the statements are equivalent.} The height of the histogram bars indicate the number (or percentage) in each category (often called `bins').\index{Graphs!histogram!bins} A histogram is essentially a picture of a frequency table. The vertical axis can be counts (labelled as `Counts of trees', `Number of frogs', `Frequency of people', or similar) or percentages.

When the quantitative variable is \emph{discrete}, the labels usually are placed on the axis aligned with the centre of the bar (see Example~\ref{exm:HistogramsDiscrete}).

\begin{example}[Histograms: discrete data]
\protect\hypertarget{exm:HistogramsDiscrete}{}\label{exm:HistogramsDiscrete}Consider again the number of severe cyclones in the Australian region (Table \ref{tab:CycloneDataLATEX}). A histogram can be constructed from either frequency table in Table~\ref{tab:CycloneFrequencyTable}; see Fig.~\ref{fig:HistCyclones}. For example, the left histogram shows there were eight years in which three severe cyclones were recorded.

Notice that different bin locations and widths change the appearance of the distribution.
\end{example}

\begin{figure}[hbtp]

{\centering \includegraphics[width=1\linewidth]{11-SummaryQuant_files/figure-latex/HistCyclones-1} 

}

\caption{Two histograms of the severe-cyclone data.}\label{fig:HistCyclones}
\end{figure}

\begin{importantBox}{iconmonstr-warning-8-240.png}
The axis displaying the counts (or percentages) should \emph{start from zero}, since the height of the bars visually implies the frequency of those observations (see Example~\ref{exm:VerticalTruncation}).

\end{importantBox}

When the quantitative variable is \emph{continuous}, care is needed when constructing the histogram. Since the data are continuous, the data must be rounded. (For instance, the birthweights in Table~\ref{tab:BabyBoomDataLATEX} are rounded to one decimal place of a kilogram.) This means care is needed when defining boundaries between bins, and ensuring clarity about which bin contains observations when they lie on (or near) a boundary. One way to do this is to define the boundaries between bins to one more decimal place than the given data (as in the final column of Table~\ref{tab:BabyBoomTable}).

The choice of bin size and bin boundaries can substantially change how a histogram displays the data (Examples~\ref{exm:Histograms} and \ref{exm:BinWidthFaithful}). For large datasets, these choices tend to matter less.

\begin{softwareBox}{iconmonstr-laptop-4-240.png}
When observations lie on the boundary of the boxes, some software includes these observations in the lower box (which is common) and some in the higher box.

\end{softwareBox}

\begin{example}[Histograms: continuous data]
\protect\hypertarget{exm:Histograms}{}\label{exm:Histograms}Consider again the weights (in kg) of babies born in a Brisbane hospital in one day (Table~\ref{tab:BabyBoomDataLATEX}). A histogram can be constructed for these data; Fig.~\ref{fig:BBHist1} shows the histogram in the process of being constructed.

An observation on a boundary between the bins may be placed in the \emph{higher} box (i.e., \(2.5\,\text{kg}\) is in the `\(2.5\) to~\(3.0\,\text{kg}\)' box, not the `\(2.0\) to~\(2.5\,\text{kg}\)' box): see the left panel. Alternatively, a boundary observation may be placed in the \emph{lower} box; see the centre panel. This histogram is a picture of the frequency table in Table~\ref{tab:BabyBoomTable}.

To avoid confusion, the boundaries can be defined to one more decimal place than the data (right panel), which is equivalent to counting the observations in the lower box (as in the left panel). Notice that the choice impacts the appearance of the histogram.
\end{example}

\begin{figure}[hbtp]

{\centering \includegraphics[width=1\linewidth]{11-SummaryQuant_files/figure-latex/BBHist1-1} 

}

\caption{Making the histogram for the baby-birth data: the first six observations added.}\label{fig:BBHist1}
\end{figure}

\begin{figure}[hbtp]

{\centering \includegraphics[width=1\linewidth]{11-SummaryQuant_files/figure-latex/BBHistTwo-1} 

}

\caption{Histograms can be constructed in different ways to manage observations on the boundary of bins. Left: boundary values counted in the higher box. Centre: boundary values counted in the lower box. Right: defining boundaries with one more decimal place than the data may be clearer.}\label{fig:BBHistTwo}
\end{figure}

\begin{example}[Histograms]
\protect\hypertarget{exm:Histograms2}{}\label{exm:Histograms2}\citet{data:Mages2017:BrainFreeze} recorded the length of `brain freezes' after consuming cold food or drink. A histogram of the data (Fig.~\ref{fig:HistBrainFreeze}), shows \(11\)~people experience symptoms less than~\(5\,\text{s}\) in length; nine people experienced symptoms for at least~\(5\) but less than~\(10\,\text{s}\); and \(1\)~person experienced symptoms for at least~\(35\,\text{s}\) but under~\(40\,\text{s}\).
\end{example}

\begin{figure}[hbtp]

{\centering \includegraphics[width=0.45\linewidth]{11-SummaryQuant_files/figure-latex/HistBrainFreeze-1} 

}

\caption{Histogram of the duration of brain-freeze symptoms after drinking ice water. Boundary observations are counted in the lower box.}\label{fig:HistBrainFreeze}
\end{figure}

Software tries to use sensible default choices for the number of bins, and width of the bins.\index{Graphs!histogram!bins} However, the bin size can substantially change the appearance of the histogram. Software makes it easy to try different bin sizes to find one that displays the overall distribution well.

\begin{example}[Histograms: bin width]
\protect\hypertarget{exm:BinWidthFaithful}{}\label{exm:BinWidthFaithful}A histogram for the time between eruptions \citep{hardle1991smoothing} of the \emph{Old Faithful} geyser in Yellowstone National Park (USA)\index{Graphs!histogram!bins} (Fig.~\ref{fig:BimodalFaithfulHistoChangeBins}) shows the shape of the distribution depends on the bin width.
\end{example}

\begin{figure}[hbtp]

{\centering \includegraphics[width=1\linewidth]{11-SummaryQuant_files/figure-latex/BimodalFaithfulHistoChangeBins-1} 

}

\caption{Histograms of the waiting times between eruptions for the \textit{Old Faithful} geyser; changing the bins can change the impression of the distribution. Left: sensible number of bins. Centre: too many bins. Right: too few bins. For all histograms, boundary values are counted in the lower box.}\label{fig:BimodalFaithfulHistoChangeBins}
\end{figure}

\subsection{Stemplots}\label{StemAndLeafPlots}

\index{Graphs!stemplot}

\emph{Stemplots} (or \emph{stem-and-leaf plots}) are best described and explained using an example. Consider again the data in Table~\ref{tab:BabyBoomDataLATEX} and Fig.~\ref{fig:BBHist1}: the weights of babies born in a Brisbane hospital on one day.

In a stemplot, part of each number is placed to the left of a vertical line (the \emph{stems}), and the rest of each number to the right of the line (the \emph{leaves}). The weights in Table~\ref{tab:BabyBoomDataLATEX} are given to one decimal place of a kilogram, so the whole number of kilograms is placed to the left of the line (as the \emph{stem}), and the first decimal place is placed on the right of the line (as a \emph{leaf}). Figure~\ref{fig:BBStem1} shows the stemplot in the process of being built, and Fig.~\ref{fig:BBStem44} shows the final stemplot. The first weight, of~\(1.7\,\text{kg}\), is entered with the~\(1\) to the left of the line, and the~\(7\) to the right: \texttt{1\ \textbar{}\ 7}. Similarly, \(2.1\,\text{kg}\) is entered as \texttt{2\ \textbar{}\ 1} and~\(2.2\,\text{kg}\) is entered as \texttt{2\ \textbar{}\ 2}, sharing the same stem as for~\(2.1\,\text{kg}\). The plot shows that most birthweights are \(3\)-point-something kilograms.

\begin{figure}[hbtp]

{\centering \includegraphics[width=1\linewidth]{11-SummaryQuant_files/figure-latex/BBStem1-1} 

}

\caption{Starting to make the stemplot for the baby-weight data: the first four observations added. The data are on the left; the stemplot during construction on the right.}\label{fig:BBStem1}
\end{figure}

\begin{figure}[hbtp]

{\centering \includegraphics[width=0.65\linewidth]{11-SummaryQuant_files/figure-latex/BBStem44-1} 

}

\caption{The completed stemplot for the baby-weight data.}\label{fig:BBStem44}
\end{figure}

For stemplots:

\begin{itemize}
\tightlist
\item
  the original data remain visible.
\item
  place the left-most digit(s) (e.g., kilograms) on the left (stems).
\item
  place the right-most digit (e.g., first decimal of a kilogram) on the right (leaves).
\item
  some data do not work well for a stemplot.
\item
  data may sometimes need suitable rounding before creating the stemplot (the baby weights were originally given to three decimal places).
\item
  the numbers in each row should be evenly spaced, with the numbers in the columns under each other, so the length of each stem is proportional to the number of observations.
\item
  the observations are \emph{ordered} within each stem, so patterns in the data can be seen.
\item
  add an explanation for reading the stemplot. For example, the stemplot for the baby-birth data says `\(2\)~\textbar~\(6\) means \(2.6\,\text{kg}\)' (rather than, say,~\(0.26\,\text{kg}\), or~\(2\,\text{lb}\) \(6\,\text{oz}\)).
\end{itemize}

\begin{example}[Stemplots]
\protect\hypertarget{exm:StemLeafPlots}{}\label{exm:StemLeafPlots}\citet{wright2021chest} recorded the chest-beating rate of gorillas. The stemplot (Fig.~\ref{fig:GorillasStem}) for gorillas aged under~\(20\)~years of age shows a lot of variation in the chest-beating rate.
\end{example}

\begin{figure}[hbtp]

{\centering \includegraphics[width=0.85\linewidth]{11-SummaryQuant_files/figure-latex/GorillasStem-1} 

}

\caption{The stemplot (right) for the gorilla chest-beating data (shown on the left).}\label{fig:GorillasStem}
\end{figure}

\subsection{Dot charts (quantitative data)}\label{DotChartsOneVar}

\index{Graphs!dot chart!one quantitative variable}

Dot charts show the data on a single (usually horizontal) axis, with each observation represented by a dot (or other symbol). Sometimes, observations are identical, or nearly so; to avoid points being plotted on top of other points (called \emph{overplotting}),\index{Overplotting} the points are \emph{jittered}\index{Overplotting!jittering} (placed with some added randomness in the vertical direction)\index{Overplotting!stacking} or \emph{stacked} (placed above each other).

\begin{example}[Dot charts]
\protect\hypertarget{exm:DotsChartsQuant2}{}\label{exm:DotsChartsQuant2}Consider the weights (in~kg) of babies born in a Brisbane hospital (Table~\ref{tab:BabyBoomDataLATEX}). A dot chart (Fig~\ref{fig:FriesBabiesPDF}, left panel) shows that most babies were born between~\(3\) and~\(4\,\text{kg}\). The points have been \emph{jittered}.\index{Overplotting!jittering}
\end{example}

\begin{example}[Dot charts]
\protect\hypertarget{exm:DotChartsQuant}{}\label{exm:DotChartsQuant}The chest-beating rate of young gorillas (Example~\ref{exm:StemLeafPlots}) can be displayed using a dot chart (Fig.~\ref{fig:FriesBabiesPDF}, right panel). The points have been \emph{stacked}.\index{Overplotting!stacking}
\end{example}

\begin{figure}[hbtp]

{\centering \includegraphics[width=0.95\linewidth]{11-SummaryQuant_files/figure-latex/FriesBabiesPDF-1} 

}

\caption{Left: a dot chart of the baby-weight data (with similar observations jittered). Right: a dot chart of the gorilla chest-beating rates (with similar observations stacked).}\label{fig:FriesBabiesPDF}
\end{figure}

\subsection{Describing the distribution}\label{SummariseData}

\index{Quantitative data!distribution}

Graphs are constructed to help readers understand the data. Hence, after producing a graph, the \emph{distribution} of the data should be described, focusing on four features:

\begin{enumerate}
\def\labelenumi{\arabic{enumi}.}
\tightlist
\item
  The \emph{shape} of the distribution.\index{Quantitative data!shape} That is, are most of the values smaller or larger, or about evenly distributed between smaller and larger values?
\item
  The \emph{average} of the data.\index{Quantitative data!averages} What is an average, central or typical value?
\item
  The \emph{variation} in the bulk of the data. \index{Quantitative data!variation}
\item
  Any \emph{outliers} (unusually large or small observations) or unusual features.\index{Quantitative data!outliers}\index{Outliers}
\end{enumerate}

These can be described in rough terms. The average, variation and outliers are usually described numerically, too (Sect.~\ref{ComputeAverage} to~Sect.~\ref{SummaryOutliers}).

\begin{example}[Describing quantitative data]
\protect\hypertarget{exm:DescribeQuantData}{}\label{exm:DescribeQuantData}The weights of babies (Example~\ref{exm:Histograms}) are typically between about~\(2.5\,\text{kg}\) and~\(3\,\text{kg}\) (the \emph{average}), with most between~\(1.5\,\text{kg}\) and~\(4.5\,\text{kg}\) (\emph{variation}). A few babies have very low weights (\emph{shape}), probably premature births. No unusual values are present.
\end{example}

\index{Quantitative data!graphs|)}

\section{Parameters and statistics}\label{ParametersAndStatistics}

The purpose of describing \emph{sample} data is to understand the \emph{population} that the sample comes from, and which the RQ asks about. Any computed numerical quantities (such as averages) are computed from the \emph{sample}, even though the \emph{population} is of interest. As a result, distinguishing \emph{parameters} and \emph{statistics} is important.\index{Parameter}\index{Statistic}

\begin{definition}[Parameter]
\protect\hypertarget{def:Parameter}{}\label{def:Parameter}A \emph{\textbf{p}arameter} is a number, usually unknown, describing some feature of a \textbf{p}opulation.\index{Population}
\end{definition}

\begin{definition}[Statistic]
\protect\hypertarget{def:Statistic}{}\label{def:Statistic}A \emph{\textbf{s}tatistic} is a number describing some feature of a \textbf{s}ample (to estimate the unknown value of the population \emph{parameter}).\index{Estimate}\index{Sample}
\end{definition}

A statistic is a numerical value estimating an unknown population value.\index{Estimate} However, countless samples are possible (Sect.~\ref{IdeaOfSampling}), and so countless possible values for the statistic---all of which are estimates of the value of the parameter---are possible. The observed value of the statistic depends on which one of the countless possible samples is selected.

\begin{importantBox}{iconmonstr-warning-8-240.png}
The RQ identifies the population, but in practice only one of the many possible samples is studied. \emph{Statistics} are estimates of \emph{parameters}, and the value of the \emph{statistic} is not the same for every possible \emph{sample}. We only observe one value of the statistic from our single observed sample.

\end{importantBox}

\clearpage

\section{Describing shape}\label{SummaryShape}

\index{Shape}

The \emph{shape} of a distribution may be able to be described using some common terminology.

\begin{itemize}
\tightlist
\item
  In \emph{right} (or \emph{positively}) skewed distributions, most data are smaller, with some larger values.\index{Shape!right skewed}\index{Shape!positively skewed}
\item
  In \emph{left} (or \emph{negatively}) skewed distributions, most data are larger, with some smaller values.\index{Shape!left skewed}\index{Shape!negatively skewed}
\item
  In symmetric distributions,\index{Shape!symmetric} the left and right sides of the graph are roughly similar.
\item
  In bimodal distributions, the distribution has two peaks.\index{Shape!bimodal}
\end{itemize}

Figure~\ref{fig:ShapeDescriptionExamples} shows typical shapes. Sometimes, no short descriptions (as above) are suitable.

\begin{figure}[hbtp]

{\centering \includegraphics[width=1\linewidth]{11-SummaryQuant_files/figure-latex/ShapeDescriptionExamples-1} 

}

\caption{Some common shapes of the distribution of qualitative data.}\label{fig:ShapeDescriptionExamples}
\end{figure}

\begin{example}[Bimodal data]
\protect\hypertarget{exm:BimodalFaithful}{}\label{exm:BimodalFaithful}The \emph{Old Faithful} geyser in Yellowstone National Park (USA) erupts regularly \citep{hardle1991smoothing}. The histogram for the time between eruptions (Fig.~\ref{fig:BimodalFaithfulHistoChangeBins}, left panel) is bimodal, with peaks near~\(55\,\text{mins}\) and~\(80\,\text{mins}\).

The weight of babies born in Brisbane (Fig.~\ref{fig:BBHistTwo}) are slightly skewed left.
\end{example}

\section{Numerical summary: averages}\label{ComputeAverage}

\index{Quantitative data!averages}\index{Averages}

The average (or \emph{location}, or \emph{central value}) for \emph{quantitative sample data} can be described numerically in many ways. The two most common are:

\begin{itemize}
\tightlist
\item
  the \emph{sample mean} (or \emph{sample arithmetic mean}), which estimates the unknown population mean (Sect.~\ref{Mean}).
\item
  the \emph{sample median}, which estimates the unknown population median (Sect.~\ref{Median}).
\end{itemize}

In both cases, the parameter is \emph{estimated} by a statistic. Understanding whether to use the mean or median is important.

\begin{tipBox}{iconmonstr-info-6-240.png}
`Average' can refer to means, medians or other measures of centre. Use the precise term `mean' or `median', rather than `average', when possible.

\end{tipBox}

\begin{example}[Averages]
\protect\hypertarget{exm:Averages}{}\label{exm:Averages}\index{Averages!compared} Consider the \emph{daily} river flow volume (`streamflow') at the Mary River (Queensland, Australia) from 01~October 1959 to 17~January 2019 \citep{mypapers:Marshman:Experiential}. The `average' daily streamflow in February could be described using either:

\begin{itemize}
\tightlist
\item
  the sample \emph{mean} daily streamflow, which is \(1\,123.2\)~ML.\spacex
\item
  the sample \emph{median} daily streamflow, which is \(146.1\)~ML.\spacex
\end{itemize}

Both are an `average value', and of the same data, yet give \emph{very} different values. This implies the mean and median measure the `average' differently, and have different meanings. Which is the best `average' to use? To decide, both averages need to be studied.
\end{example}

\subsection{Average: the mean}\label{Mean}

\index{Mean!of a sample}\index{Mean|(}

The mean of the population is denoted by \(\mu\),\index{Mean!of a population} and its value is almost always unknown. The mean of the population is \emph{estimated} by the mean of the sample, denoted \(\bar{x}\). In this context, the value of the unknown \emph{parameter} is \(\mu\), and the value of the \emph{statistic} is \(\bar{x}\).

\begin{importantBox}{iconmonstr-warning-8-240.png}
The sample mean \emph{estimates} the population mean, and every sample is likely to give a different value for the sample mean. We usually only have one sample.

\end{importantBox}

\begin{pronounceBox}{iconmonstr-microphone-7-240.png}

The Greek letter \(\mu\) is pronounced `mew' (rhymes with `chew'). \(\bar{x}\) is pronounced `ex-bar'.

\end{pronounceBox}

\begin{example}[Estimating a population mean]
\protect\hypertarget{exm:SmallDataSet}{}\label{exm:SmallDataSet}Consider a small dataset for answering this descriptive RQ: `For gorillas aged under~\(20\), what is the average chest-beating rate?' The population mean rate (denoted \(\mu\)) is to be estimated.

Every gorilla cannot be studied, so a \emph{sample} is studied. The unknown population mean \(\mu\) is estimated using the sample mean (\(\bar{x}\)) of \(n = 14\)~young gorillas (Fig~\ref{fig:GorillasStem}, left panel). Of course, a different sample would likely give a different value for~\(\bar{x}\).
\end{example}

The sample mean is the `balance point' of the observations, as shown in Fig.~\ref{fig:MeansFigLATEX} (left panel) for the gorilla data. Also, the positive and negative distances (the `deviations') of the observations from the mean add to zero (Fig.~\ref{fig:MeansFigLATEX}, right panel). Both of these explanations seem reasonable for identifying an `average' value for the data.

\begin{figure}[hbtp]

{\centering \includegraphics[width=1\linewidth]{11-SummaryQuant_files/figure-latex/MeansFigLATEX-1} 

}

\caption{Two ways to understand the (arithmetic) mean. Left: the mean is the balance point of the data. Right: the mean is the value such that the positive and negative distances sum to zero.}\label{fig:MeansFigLATEX}
\end{figure}

\begin{definition}[Mean]
\protect\hypertarget{def:Mean}{}\label{def:Mean}The \emph{mean} is one way to measure the `average' value of quantitative data. The \emph{arithmetic mean} is the `balance point' of the data. The positive and negative distances from the mean add to zero.
\end{definition}

To find the \emph{value} of the sample mean, \emph{add} (denoted by \(\sum\)) all the observations (denoted by \(x\)) then \emph{divide} by the number of observations (denoted by \(n\)). In symbols:\index{Mean!of a sample} \[
\bar{x} = \frac{\sum x}{n}.
\]

\begin{example}[Computing a sample mean]
\protect\hypertarget{exm:ComputeMean}{}\label{exm:ComputeMean}For the chest-beating data (Fig~\ref{fig:GorillasStem}, left panel), an \emph{estimate} of the population mean (i.e., the sample mean) chest-beating rate is found by summing all \(n = 14\) observations then dividing by \(n = 14\): \[
\overline{x} 
= \frac{\sum x}{n} 
= \frac{0.7 + 0.9 + \cdots + 4.4}{14}
= \frac{31.1}{14}  
=  2.221429.
\] The sample mean, the best estimate of the population mean, is~\(2.22\) beats per~\(10\,\text{h}\).
\end{example}

\begin{tipBox}{iconmonstr-info-6-240.png}
The sample mean is usually calculated using statistical software for large amounts of data, or a calculator for small amounts of data. However, knowing \emph{how} the mean is computed is helpful.

Software and calculators often produce numerical answers to many decimal places, not all of which may be meaningful or useful. A simple, but often useful, rule-of-thumb is to round to one or two more significant figures than the original data. Software usually does not add measurement units to the answer either.

The chest-beating data are given to one decimal place, so the \emph{sample mean} rate is given as \(\bar{x} = 2.22\) beats per \(10\,\text{h}\).

\end{tipBox}

\begin{example}[Computing a sample mean]
\protect\hypertarget{exm:BatsMean}{}\label{exm:BatsMean}\citet{griffin1960echolocation} recorded the distance at which flies (\emph{Drosophila}) were detected by bats for \(n = 11\) detections (Table~\ref{tab:BatData}). The population mean distance is estimated by the sample mean as \(\bar{x} = 532/11 = 48.4\,\text{cm}\).
\end{example}

\index{Mean|)}

\begin{table}
\centering
\caption{\label{tab:BatData}The distance at which small fruit flies were detected by bats.}
\centering
\fontsize{8}{10}\selectfont
\begin{tabular}[t]{lllllllllll}
\toprule
\multicolumn{11}{c}{\textbf{Detection distance (in cm)}} \\
\cmidrule(l{3pt}r{3pt}){1-11}
$62$ & $52$ & $68$ & $23$ & $34$ & $45$ & $27$ & $42$ & $83$ & $56$ & $40$\\
\bottomrule
\end{tabular}
\end{table}

\subsection{Average: the median}\label{Median}

\index{Median!of a sample}\index{Median|(}

A median is a value separating the largest \(50\)\% of the data from the smallest~\(50\)\% of the data. In a dataset with~\(n\) values, the median is \emph{ordered observation number} \((n + 1)\div 2\). (The value of the median is \emph{not} \((n + 1)\div 2\), and the median \emph{not} necessarily halfway between the minimum and maximum values in the data.)

\begin{tipBox}{iconmonstr-info-6-240.png}
Many calculators cannot find the median. The median has no commonly-used symbol, though \(\tilde{\mu}\) and \(\tilde{x}\) are sometimes used for the population and sample medians respectively.

\end{tipBox}

\begin{definition}[Median]
\protect\hypertarget{def:Median}{}\label{def:Median}The \emph{median} is one way to measure the `average' value of data. A \emph{median} is a value such that half the values are larger than the median, and half the values are smaller than the median.
\end{definition}

\begin{example}[Computing a sample median]
\protect\hypertarget{exm:SampleMedian}{}\label{exm:SampleMedian}To find a sample median for the chest-beating data (Fig~\ref{fig:GorillasStem}, left panel), first arrange the data \emph{in numerical order} (Table~\ref{tab:GYoungSorted}). The median separates the larger seven numbers from the smaller seven numbers. With \(n = 14\) ordered observations, the median is at position \((14 + 1)/2 = 7.5\) (the \emph{median itself is not \(7.5\)}). This means that the median is located between the seventh and eighth ordered observations.

Thus, the sample median, an estimate of the \emph{population} median\index{Median!of a population}, is between~\(1.7\) (ordered observation~\(7\)) and~\(1.7\) (ordered observation~\(8\)). Since these values are the same, the sample median is~\(1.7\) beats per~\(10\,\text{h}\).
\end{example}

\begin{table}
\centering
\caption{\label{tab:GYoungSorted}The chest-beating rate of young gorillas, in increasing order.}
\centering
\fontsize{8}{10}\selectfont
\begin{tabular}[t]{llllllllllllll}
\toprule
\multicolumn{14}{c}{\textbf{Chest-beating rate, per 10\,\text{h}}} \\
\cmidrule(l{3pt}r{3pt}){1-14}
$0.7$ & $0.9$ & $1.3$ & $1.5$ & $1.5$ & $1.5$ & $1.7$ & $1.7$ & $1.8$ & $2.6$ & $3.0$ & $4.1$ & $4.4$ & $4.4$\\
\bottomrule
\end{tabular}
\end{table}

To clarify:

\begin{itemize}
\tightlist
\item
  if the sample size~\(n\) is \emph{odd} (see Example~\ref{exm:BatsMedian}), the median is the middle number when the observations are ordered.
\item
  if the sample size~\(n\) is \emph{even} (such as the chest-beating data; Example~\ref{exm:SampleMedian}), the median is halfway between the two middle numbers, when the observations are ordered.
\end{itemize}

\begin{softwareBox}{iconmonstr-laptop-4-240.png}
Software may use slightly different rules when \(n\) is even, producing slightly different values for the median.

\end{softwareBox}

\begin{importantBox}{iconmonstr-warning-8-240.png}
The sample median \emph{estimates} the population median, and every sample is likely to have a different value for the sample median. We usually only have one sample.

\end{importantBox}

\begin{example}[Computing a sample median]
\protect\hypertarget{exm:BatsMedian}{}\label{exm:BatsMedian}For the bat data (Table~\ref{tab:BatData}), the estimate of the population \emph{median} distance at which bats detect the flies is the \emph{sample} median. With \(n = 11\), the median is the \((11 + 1)/2 = 6\)th ordered value, which is~\(45\,\text{cm}\).
\end{example}

\index{Median|)}

\subsection{Which average to use?}\label{CompareMeanMedian}

\index{Averages!compared}

Consider the daily streamflow at the Mary River again (Example~\ref{exm:Averages}): the sample \emph{mean} daily streamflow is \(1\,123\)~ML, and the sample \emph{median} daily streamflow is \(146.1\)~ML.\spacex Which is `best' for measuring the average streamflow?

For these data, \(86\)\% of observations are \emph{smaller} than the mean, but \(50\)\%~of the observations are smaller than the median (by definition). The mean is hardly a \emph{central} value.

A dot chart of the daily streamflow (Fig.~\ref{fig:DailyStreamflow}; jittered)\index{Overplotting!jittering} shows that the data are \emph{very} highly right-skewed, with many \emph{very} large outliers\index{Outliers} (presumably flood events).

\begin{figure}[hbtp]

{\centering \includegraphics[width=1\linewidth]{11-SummaryQuant_files/figure-latex/DailyStreamflow-1} 

}

\caption{A dot plot of the daily streamflow at Mary River from\ 1960 to\ 2017, for February ($n = 1\,650$). Many very large outliers exist. Note: values have been jittered in the vertical direction, but overplotting is still present near\ $0$.\index{Overplotting}}\label{fig:DailyStreamflow}
\end{figure}

The streamflow data are \emph{very} right skewed, which is important for explaining the difference between the values of the sample mean and the sample median:

\begin{itemize}
\tightlist
\item
  \emph{means} are best used for approximately symmetric data, because the mean is influenced by outliers and skewness.
\item
  \emph{medians} are best used for data that are highly skewed or contain outliers, because the median is \emph{not} influenced by outliers or skewness.
\end{itemize}

Means tend to be too large if the data contain large outliers or severe right skewness, and too small if the data contain small outliers or severe left skewness. The Mary River data contains extremely large outliers---and many of them---so the mean is much larger than the median. \emph{The median is the better measure of average for these data}. However, understanding the variation is probably more important than understanding the average value (Sect.~\ref{Variation}), and the data may even be better described using percentiles (Sect.~\ref{VariationPercentiles}).

The mean is generally used rather than the median if possible (for practical and mathematical reasons), and is the most commonly-used measure of location. However, the mean is not always appropriate (as the mean is influenced by outliers\index{Outliers} and skewness).\index{Skewness} The mean and median are similar in approximately symmetric distributions. Sometimes, quoting \emph{both} the mean and the median may be appropriate.

\section{Numerical summary: variation}\label{Variation}

\index{Quantitative data!variation}

For quantitative data, the amount of \emph{variation} in the bulk of the data should be described. Many ways exist to measure the variation in a dataset, including:

\begin{itemize}
\tightlist
\item
  the \emph{range}, which is very simple and simplistic so is not often used (Sect.~\ref{VariationRange}).
\item
  the \emph{standard deviation}, which is commonly used (Sect.~\ref{VariationStdDev}).
\item
  the \emph{interquartile range (or IQR)}, which is commonly used (Sect.~\ref{VariationIQR}).
\item
  \emph{percentiles}, which are useful in specific situations (Sect.~\ref{VariationPercentiles}), especially for very skewed data.
\end{itemize}

As always, a value computed from a \emph{sample} (the statistic) estimates the unknown value in the \emph{population} (the parameter), and every sample can produce a different estimate.

\subsection{Variation: the range}\label{VariationRange}

\index{Range}

The range is the simplest and easiest-to-compute measure of variation.

\begin{definition}[Range]
\protect\hypertarget{def:Range}{}\label{def:Range}The range is the maximum value \emph{minus} the minimum value.
\end{definition}

The range is not often used as it only uses two values in a data set, both of which are extreme observations. As a result, the range is highly influenced by outliers.\index{Outliers} Sometimes, the \emph{range} is given by stating both the maximum and the minimum value in the data instead of the \emph{difference} between these values. The range is measured in the same measurement units as the data, and is usually quoted with the median.

\begin{example}[The range]
\protect\hypertarget{exm:RangeEG}{}\label{exm:RangeEG}For the chest-beating data (Table~\ref{tab:GYoungSorted}), the largest value is~\(4.4\), and the smallest value is~\(0.7\); hence \[
\text{Range} = 4.4 - 0.7 = 3.7.
\] The range of the chest-beating rate is~\(3.7\)~beats per~\(10\,\text{h}\).
\end{example}

\subsection{Variation: the standard deviation}\label{VariationStdDev}

\index{Standard deviation!of a sample}

The population standard deviation (a parameter) is denoted by~\(\sigma\) and is estimated by the sample standard deviation~\(s\) (a statistic). The standard deviation is the most commonly-used measure of variation. It is tedious to compute manually, so is usually calculated using statistical software\index{Computers and software!statistical} for large amounts of data, or a calculator for small amounts of data.

The \emph{standard deviation} is (approximately) the mean distance that observations are from the mean. This seems like a reasonable way to measure the amount of variation in data.

\begin{pronounceBox}{iconmonstr-microphone-7-240.png}

The Greek letter \(\sigma\) is pronounced~`sigma'.

\end{pronounceBox}

\begin{importantBox}{iconmonstr-warning-8-240.png}
\index{Standard deviation!of a population} The sample standard deviation \emph{estimates} the population standard deviation, and every sample is likely to have a different value for the sample standard deviation. We usually only have one sample.

\end{importantBox}

\begin{definition}[Standard deviation]
\protect\hypertarget{def:StandardDeviation}{}\label{def:StandardDeviation}The \emph{standard deviation} is, approximately, the mean distance of the observations from the mean.
\end{definition}

Even though \emph{you do not have to use the formula} to calculate~\(s\) (use software), we will demonstrate to show exactly what~\(s\) calculates.\index{Standard deviation!of a sample} The formula for computing the value of \(s\) is \[
s = \sqrt{ \frac{\sum(x - \bar{x})^2}{n - 1} },
\] where~\(\bar{x}\) is the sample mean, \(x\)~represents the individual data values, \(n\)~is the sample size, and the symbol~`\(\sum\)' means to \emph{add} (Sect.~\ref{Mean}). Using the formula requires these steps.

\begin{enumerate}
\def\labelenumi{\arabic{enumi}.}
\tightlist
\item
  Calculate the sample mean:~\(\overline{x}\).
\item
  Calculate the \emph{deviation} of each observation~\(x\) from the sample mean: \(x - \bar{x}\).
\item
  Square these deviations (to make them all \emph{positive} values): \((x - \bar{x})^2\).
\item
  Add these squared deviations: \(\sum(x - \bar{x})^2\).
\item
  Divide the answer by \(n - 1\).
\item
  Take the (positive) square root of the answer (to `undo' the squaring of the data).
\end{enumerate}

\begin{importantBox}{iconmonstr-warning-8-240.png}
\emph{You do not need to use the formula!} You should know how to use software or a calculator to find the value of the standard deviation, what the standard deviation measures, and how and when to use it.

\end{importantBox}

\begin{example}[Computing a sample standard deviation]
\protect\hypertarget{exm:StdDev}{}\label{exm:StdDev}For the chest-beating data (Table~\ref{tab:GYoungSorted}), the squared \emph{deviations} of each observation from the mean of \(2.2214\) (using four decimal places in calculations) are shown in Fig.~\ref{fig:ShowVar}. The sum of the squared distances is~\(20.9636\). Then, the sample standard deviation is: \[ 
  s = \sqrt{\frac{20.9636}{14 - 1}}
    = \sqrt{ 1.612585} 
    = 1.269876.
\] The sample standard deviation of the chest-beating rate is \(1.27\) per~\(10\,\text{h}\).
\end{example}



\begin{figure}[hbtp]

{\centering \includegraphics[width=0.9\linewidth]{11-SummaryQuant_files/figure-latex/ShowVar-1} 

}

\caption{The standard deviation is related to the sum of the squared-distances from the mean. The chest-beating data are used. The sum of the deviations is \emph{always} zero.}\label{fig:ShowVar}
\end{figure}

The sample standard deviation~\(s\) is:

\begin{itemize}
\tightlist
\item
  positive (unless all observations are the same, when \(s = 0\); that is, \emph{zero} variation).
\item
  best used for (approximately) symmetric data.
\item
  usually quoted with the mean.
\item
  the most commonly-used measure of variation.
\item
  measured in the same units as the data.
\item
  influenced by skewness\index{Skewness} and outliers,\index{Outliers} like the mean.
\end{itemize}

\subsection{Variation: the interquartile range (IQR)}\label{VariationIQR}

\index{Interquartile range (IQR)}

The standard deviation uses the value of \(\bar{x}\), so is impacted by skewness and outliers just like the sample mean. A measure of variation \emph{not} affected by skewness\index{Skewness} and outliers is the interquartile range, or~IQR.\spacex To understand the IQR, understanding \emph{quartiles} is necessary first.

\begin{definition}[Quartiles]
\protect\hypertarget{def:Quartiles}{}\label{def:Quartiles}

\index{Quartiles} \emph{Quartiles} describe the shape of the data.

\begin{itemize}
\tightlist
\item
  The first quartile \(Q_1\) is a value separating the smallest~\(25\)\% of observations from the largest~\(75\)\%. The \(Q_1\) is like the median of the \emph{smaller} half of the data, halfway between the minimum value and the median.
\item
  The second quartile \(Q_2\) is a value separating the smallest~\(50\)\% of observations from the largest~\(50\)\%. (This is also the \emph{median}.)\index{Median}
\item
  The third quartile \(Q_3\) is a value separating the smallest~\(75\)\% of observations from the largest~\(25\)\%. The \(Q_3\) is like the median of the \emph{larger} half of the data, halfway between the median and the maximum value.
\end{itemize}

\end{definition}

Quartiles divide the data into four parts of approximately equal numbers of observations. The \emph{interquartile range} (or \emph{IQR}) is the difference between~\(Q_3\) and~\(Q_1\).

\begin{definition}[IQR]
\protect\hypertarget{def:IQR}{}\label{def:IQR}The \emph{IQR} is the range in which the middle~\(50\)\% of the data lie: the difference between the third and the first quartiles.
\end{definition}

\begin{importantBox}{iconmonstr-warning-8-240.png}
The sample IQR \emph{estimates} the population IQR, and every sample is likely to have a different value for the sample IQR.\spacex We usually only have one sample.

\end{importantBox}

For the chest-beating data (Table~\ref{tab:GYoungSorted}), where \(n = 14\) (an \emph{even} number of observations), the median is \(1.7\) (Example~\ref{exm:SampleMedian}). The data then can be split into \emph{smaller} and \emph{larger} halves, each with seven values:

\begin{itemize}
\tightlist
\item
  \makebox[23mm][l]{Smaller half:} \(0.7\)~~~ \(0.9\)~~~ \(1.3\)~~~ \(1.5\)~~~ \(1.5\)~~~ \(1.5\)~~~ \(1.7\)
\item
  \makebox[23mm][l]{Larger half:} \(1.7\)~~~ \(1.8\)~~~ \(2.6\)~~~ \(3.0\)~~~ \(4.1\)~~~ \(4.4\)~~~ \(4.4\)
\end{itemize}

Since each half has seven observations (an \emph{odd} number), the median of each half is the \((7 + 1)/2 = 4\)th ordered value. Hence:\index{Quartiles}

\begin{itemize}
\tightlist
\item
  \(Q_1\), the \emph{first quartile}, is the median of the smaller half; \(Q_1 = 1.5\). About~\(25\)\% of observations are smaller than~\(1.5\).
\item
  \(Q_2\), the \emph{second quartile} or \emph{median}, is~\(1.7\), so \(50\)\% of observations are smaller than~\(1.7\).
\item
  \(Q_3\), the \emph{third quartile}, is the median of the larger half; \(Q_3 = 3.0\). About~\(75\)\% of observations are smaller than~\(3.0\).
\end{itemize}

To divide the data into four parts of equal numbers of observations, each part would need \(14/4 = 3.5\) observations, which is not possible. Hence, we say the values of~\(Q_1\) and~\(Q_3\) are `about' the values given. (Software sometimes uses a different method for computing the quartiles.) Using these values, the IQR is \(Q_3 - Q_1\) = \(3.0 - 1.5 = 1.5\), as shown in Fig.~\ref{fig:QuartilesYgorillas}.

Since the IQR measures the range of the central~\(50\)\% of the data, the IQR is not influenced by outliers. The IQR is measured in the same measurements units as the data.

\begin{softwareBox}{iconmonstr-laptop-4-240.png}
Software often uses different rules to compute quartiles (and medians) that may produce slightly different answers. In large datasets, the differences are usually minimal.

\end{softwareBox}

\begin{figure}[hbtp]

{\centering \includegraphics[width=1\linewidth]{11-SummaryQuant_files/figure-latex/QuartilesYgorillas-1} 

}

\caption{A dot chart (with jittering) for the chest-beating data for young gorillas, showing the IQR.}\label{fig:QuartilesYgorillas}
\end{figure}

When \(n\)~is odd, the median \emph{may} or \emph{may not} be included in each of these halves when computing \(Q_1\) and~\(Q_3\); we decide \emph{not} to include the median in each half.

\begin{example}[Computing the IQR for $n$ odd]
\protect\hypertarget{exm:BatsIQR}{}\label{exm:BatsIQR}The bat data (Table~\ref{tab:BatData}) has \(n = 11\) observations. The smaller and larger halves of the data, \emph{without} the median of~\(45\), are:

\begin{itemize}
\tightlist
\item
  \makebox[23mm][l]{Smaller half:}~~\(23\)~~\(27\)~~\(34\)~~\(40\)~~\(42\): the median is \(Q_1 = 34\).
\item
  \makebox[23mm][l]{Larger half:}~~\(52\)~~\(56\)~~\(62\)~~\(68\)~~\(83\): the median is \(Q_3 = 62\).
\end{itemize}

Hence, the IQR is \(62 - 34 = 28\,\text{cm}\). (If the median \emph{is} included in each half, the IQR is \(59 - 37 = 22\,\text{cm}\).)
\end{example}

\subsection{Variation: percentiles}\label{VariationPercentiles}

\index{Percentiles}

\emph{Percentiles} are like quartiles; in fact, quartiles are a special case of percentiles.

\begin{definition}[Percentiles]
\protect\hypertarget{def:Percentiles}{}\label{def:Percentiles}The \(p\)th percentile of the data is a value separating the smallest~\(p\)\% of the data from the rest.
\end{definition}

For example:

\begin{itemize}
\tightlist
\item
  the \(12\)th percentile separates the smallest~\(12\)\% of the data from the rest.
\item
  the \(67\)th percentile separates the smallest~\(67\)\% of the data from the rest.
\item
  the \(94\)th percentile separates the smallest~\(94\)\% of the data from the rest.
\end{itemize}

This means that the first quartile~\(Q_1\) is the \(25\)th~percentile, the second quartile~\(Q_2\) is the \(50\)th~percentile (and median),\index{Median} and the third quartile~\(Q_3\) is the \(75\)th~percentile.\index{Quartiles}

\begin{softwareBox}{iconmonstr-laptop-4-240.png}
Software uses various rules to compute percentiles. In large datasets, the differences are usually minimal.

\end{softwareBox}

Percentiles are especially useful for very skewed data in certain applications. For instance, scientists who monitor rainfall and stream heights, and engineers who use this information, are more interested in extreme weather events rather than the `average' event. Structures need to be designed to withstand \(1\)-in-\(100\) year events (the \(99\)th percentile) or similar, rather than `average' events. Percentiles are measured in the same measurements units as the data.

\begin{example}[Percentiles]
\protect\hypertarget{exm:PercentilesEG}{}\label{exm:PercentilesEG}For the streamflow data at the Mary River (Example~\ref{exm:Averages}), the February data are highly right-skewed (Fig.~\ref{fig:DailyStreamflow}). The median (\(50\)th percentile) is~\(146.1\)~ML.\spacex However, the \(95\)th percentile is~\(3\,480\)~ML and the \(99\)th~percentile is~\(19\,043\)~ML.

Constructing infrastructure for the \emph{median} streamflow would be highly inadequate.
\end{example}

\subsection{Which measure of variation to use?}\label{CompareVariations}

\index{Variation measures compared}

Which is the `best' measure of variation for quantitative data? As with measures of location, the answer depends on the data.

Since the standard deviation formula uses the mean, it is impacted in the same way as the mean by outliers\index{Outliers} and skewness.\index{Skewness} Hence, the standard deviation is best used with approximately symmetric data. The IQR is best used when data are skewed or outlier are present. Sometimes, both the standard deviation and the IQR can be quoted.

\section{Numerical summary: identifying outliers}\label{SummaryOutliers}

\index{Quantitative data!outliers}

Outliers are `unusual' observations: those quite different from the bulk of the data (larger or smaller). Outliers are `unusual', but not necessarily `incorrect' or `bad' observations. Rules for deciding if an observation is an outlier are always arbitrary.

\begin{definition}[Outliers]
\protect\hypertarget{def:Outliers}{}\label{def:Outliers}An \emph{outlier} is an observation that is `unusual' (either larger or smaller) compared to the bulk of the data. Rules for identifying outliers are arbitrary.
\end{definition}

Two rules for identifying outliers are:

\begin{itemize}
\tightlist
\item
  the \emph{standard deviation rule}, which is only useful when the data have an approximately symmetric distribution (Sect.~\ref{OutliersStdDevRule}).
\item
  the \emph{IQR rule}, which is useful in most situations (Sect.~\ref{OutliersIQRrule}).
\end{itemize}

\subsection{The standard deviation rule}\label{OutliersStdDevRule}

\index{Outliers!standard deviation rule}

The standard deviation rule uses the mean and the standard deviation, so is suitable for approximately symmetric distributions (when means and standard deviations are sensible numerical summaries). The rationale behind this rule is explained in Sect.~\ref{NormalDistribution}.

\begin{definition}[Standard deviation rule for identifying outliers]
\protect\hypertarget{def:StandardDeviationRuleForIdentifyingOutliers}{}\label{def:StandardDeviationRuleForIdentifyingOutliers}For approximately symmetric distributions, an observation more than three standard deviations from the mean may be considered an outlier.
\end{definition}

All rules for identifying outliers are arbitrary, and sometimes the standard deviation rule is given slightly differently. For example, outliers may be identified as observations more than~\(2.5\) standard deviations from the mean. Both rules are acceptable, since the definition is arbitrary.

\begin{example}[Standard deviation rule for identifying outliers]
\protect\hypertarget{exm:SDOutliersEG}{}\label{exm:SDOutliersEG}An engineering project \citep{data:hald:statistical} studied a new building material, to estimate the average permeability. Permeability time (the time for water to permeate the sheets) was measured from \(81\)~pieces of material (in seconds).

For these data, the mean is \(\bar{x} = 43.162\) and the standard deviation is \(s = 27.358\). Using the standard deviation rule, outliers are observations \emph{smaller} than \(43.162 - (3\times 27.358)\) or \emph{larger} than \(43.162 + (3\times 27.358)\); that is, \emph{smaller} than~\(-38.9\,\text{s}\) (which is clearly not appropriate here, as the data must be positive values), or \emph{larger} than~\(125.2\,\text{s}\). This rule is shown in Fig.~\ref{fig:SDOutliers}; two observations are identified as outliers using the standard deviation rule.
\end{example}

\begin{figure}[hbtp]

{\centering \includegraphics[width=0.99\linewidth]{11-SummaryQuant_files/figure-latex/SDOutliers-1} 

}

\caption{Outliers identified using the standard deviation rule for the permeability data.}\label{fig:SDOutliers}
\end{figure}

\subsection{The IQR rule}\label{OutliersIQRrule}

\index{Outliers!IQR rule}

Since the standard deviation rule for identifying outliers relies on the mean and standard deviation, it is not appropriate for non-symmetric distributions. Another rule is needed for identifying outliers in these situations: the IQR rule.

\begin{definition}[IQR rule for identifying outliers]
\protect\hypertarget{def:IQRRuleForIdentifyingOutliers}{}\label{def:IQRRuleForIdentifyingOutliers}

The IQR rule identifies mild and extreme outliers.

\begin{itemize}
\tightlist
\item
  \emph{Extreme outliers}: observations \(3\times \text{IQR}\) more unusual than \(Q_1\) or \(Q_3\).\index{Outliers!IQR rule!extreme outliers}
\item
  \emph{Mild outliers}: observations \(1.5\times \text{IQR}\) more unusual than \(Q_1\) or \(Q_3\) (that are not extreme outliers).\index{Outliers!IQR rule!mild outliers}
\end{itemize}

\end{definition}

This definition is easier to understand using an example.

\begin{example}[IQR rule for identifying outliers]
\protect\hypertarget{exm:IQROutliersEG}{}\label{exm:IQROutliersEG}Using the permeability data seen in Example~\ref{exm:SDOutliersEG}, a computer shows that \(Q_1 = 24.7\) and \(Q_3 = 50.6\), so \(\text{IQR} = {50.6 - 24.7 = 25.9}\). Then, \emph{extreme} outliers are observations \(3\times 25.9 = 77.7\) more unusual than~\(Q_1\) or~\(Q_3\). That is, \emph{extreme} outliers are observations are:

\begin{itemize}
\tightlist
\item
  more unusual than \(24.7 - 77.7 = -53.0\) (that is, \emph{less} than \(-53.0\)); or
\item
  more unusual than \(50.6 + 77.7 = 128.3\) (that is, \emph{greater} than \(128.3\)).
\end{itemize}

\emph{Mild} outliers are observations \(1.5\times 25.9 = 38.9\) more unusual than~\(Q_1\) or~\(Q_3\) (that are not extreme outliers). That is, \emph{mild} outliers are

\begin{itemize}
\tightlist
\item
  more unusual than \(24.7 - 38.9 = -14.2\) (that is, \emph{less} than~\(-14.2\)); or
\item
  more unusual than \(50.6 + 38.9 = 89.5\) (that is, \emph{greater} than~\(89.5\)).
\end{itemize}

Three observations are identified as outliers using the IQR rule (Fig.~\ref{fig:IQROutliers}): two extreme outliers, and one mild outlier.
\end{example}

\begin{figure}[hbtp]

{\centering \includegraphics[width=1\linewidth]{11-SummaryQuant_files/figure-latex/IQROutliers-1} 

}

\caption{Mild and extreme outliers, using the IQR rule, for the permeability data.}\label{fig:IQROutliers}
\end{figure}

\subsection{Which outlier rule to use?}\label{CompareOutlierRules}

\index{Outliers!rules compared}

The standard deviation rule is most appropriate for \emph{approximately symmetric distributions}; the IQR rule can be used for \emph{any distribution}, but primarily for those skewed or with outliers.

\begin{importantBox}{iconmonstr-warning-8-240.png}
Remember: all rules for identifying outliers are arbitrary.

\end{importantBox}

\subsection{What to do with outliers?}\label{OutliersWhatToDo}

\index{Outliers!managing}

What should be done if outliers are identified in data? Deleting or removing outliers simply because they are identified as outliers is \emph{very poor practice}. After all, the outliers were obtained from the study like all other observations; they belong in the data as much as any other observation. In addition, the rules for identifying outliers are \emph{arbitrary}: some observations may be identified as outliers using one rule, but not by another.

\begin{importantBox}{iconmonstr-warning-8-240.png}
Outliers are \emph{unusual} observations; they are not necessarily \emph{mistakes}.

\end{importantBox}

Managing outliers depends on \emph{why} they occur (\citet{mypapers:dunnsmyth:glms}, p.~138):

\begin{itemize}
\tightlist
\item
  \emph{the outlier is clearly a mistake} (e.g., an age of~\(222\)). If the mistake cannot be fixed (e.g., the completed questionnaire form is lost), the observation can be \emph{deleted}. Similarly, if the outlier comes from an error or mistake in the data collection (e.g., too much fertiliser was accidentally applied), the observation can be deleted.
\item
  \emph{the outlier represents a different population}. Suppose an outlier is identified in a study of students, corresponding to a student aged 65. If the next oldest student in the data is aged~\(39\), the outlier can be removed, since it belongs to a different population (`students aged over~\(40\)') than the other observations (`students aged~\(40\) and under'). The remaining observations can be analysed, but the results only apply to students aged under~\(40\) (which should be clearly communicated).
\item
  \emph{the reason for the outlier is unknown}. In these cases, \emph{discarding outliers routinely is not recommended}; the outliers are probably real observations that are just as valid as the others. Perhaps a different analysis is necessary (e.g., using medians rather than means). Furthermore, very large datasets are expected to have a small number of observations identified as outliers using the above arbitrary rules.
\end{itemize}

In all cases, whenever observations are removed from a dataset, this should be clearly explained and documented.

\begin{example}[Outliers]
\protect\hypertarget{exm:OutliersWhatToDo}{}\label{exm:OutliersWhatToDo}The Mary River dataset (Sect.~\ref{ComputeAverage}) has many \emph{extremely} large outliers identified by software, but each is reasonable. They probably correspond to flood events (which could be confirmed). Removing these from the analysis would be inappropriate.
\end{example}

\begin{example}[Outliers]
\protect\hypertarget{exm:OutliersWhatToDo2}{}\label{exm:OutliersWhatToDo2}The permeability data (Example~\ref{exm:SDOutliersEG}) has large outliers, but all seem reasonable. Removing these from the analysis would be inappropriate.
\end{example}

\section{Numerical summary tables}\label{QuantSummaryTable}

\index{Quantitative data!summary tables}

In studies with quantitative variables, the quantitative variables should be summarised in a table. The table should include, as a minimum, measures of average, variation and the sample sizes. An example is given in the next section (Table~\ref{tab:WaterAccessQuant}).

\clearpage

\section{Example: water access}\label{WaterAccessQuant}

\citet{lopez2022farmers} recorded data about access to water for three rural communities in Cameroon. Three quantitative variables are recorded. Part of understanding the data requires summarising the quantitative variables; histograms are shown in Fig.~\ref{fig:WaterAcessQuant}, and a summary table in Table~\ref{tab:WaterAccessQuant}.

Many households are coordinated by women in their late~\(50\)s. The number of people and number of children under~\(5\) years of age are both right-skewed. One household has over \(30\)~people, and has \(10\)~children in that household. (These are identified as outliers, but are unlikely to be mistakes.) Some observations are missing for some variables, explaining the differences in sample sizes in Table~\ref{tab:WaterAccessQuant}.

\begin{table}
\centering
\caption{\label{tab:WaterAccessQuant}Summarising the quantitative data in the water-access study.}
\centering
\fontsize{8}{10}\selectfont
\begin{tabular}[t]{lccccccc}
\toprule
\textbf{ } & \textbf{$n$} & \textbf{Mean} & \textbf{Median} & \textbf{Std dev.} & \textbf{IQR} & \textbf{Min.} & \textbf{Max.}\\
\midrule
Woman's age (years) & $120$ & $41.6$ & $40.5$ & $14.56$ & $30.25$ & $19$ & $61$\\
Household size & $121$ & $\phantom{0}7.0$ & $\phantom{0}6.0$ & $\phantom{0}4.80$ & $\phantom{0}4.00$ & $\phantom{0}0$ & $32$\\
Children aged under $5$ & $120$ & $\phantom{0}1.6$ & $\phantom{0}1.0$ & $\phantom{0}1.65$ & $\phantom{0}2.00$ & $\phantom{0}0$ & $10$\\
\bottomrule
\end{tabular}
\end{table}

\begin{figure}[hbtp]

{\centering \includegraphics[width=1\linewidth]{11-SummaryQuant_files/figure-latex/WaterAcessQuant-1} 

}

\caption{The age of the female household coordinator, the number of people in the household, and the number of children in the household aged under $5$ years, for the water-access study.}\label{fig:WaterAcessQuant}
\end{figure}

\index{Quantitative data!summarising|)}

\section{Chapter summary}\label{Summarise-Quant-Summary}

Quantitative data can be graphed using a histogram, stemplot (in special circumstances), or dot charts. Quantitative data can be summarised numerically; the most common techniques are indicated in Table~\ref{tab:SummaryQuantStats}. The \emph{mean} and \emph{standard deviation} are usually used whenever possible, for practical and mathematical reasons. Sometimes quoting both the mean and median (and the standard deviation and IQR) may be appropriate.

\begin{table}
\centering
\caption{\label{tab:SummaryQuantStats}Summarising quantitative data.}
\centering
\fontsize{8}{10}\selectfont
\begin{tabular}[t]{rcc}
\toprule
\multicolumn{1}{c}{\textbf{ }} & \multicolumn{2}{c}{\textbf{For distributions with a shape that is:}} \\
\cmidrule(l{3pt}r{3pt}){2-3}
\textbf{Feature} & \textbf{Approximately symmetric} & \textbf{Not symmetric, or has outliers}\\
\midrule
Average & Mean & Median\\
Variation & Standard deviation & IQR\\
Outliers & Standard deviation rule & IQR rule\\
\bottomrule
\end{tabular}
\end{table}

\section{Quick review questions}\label{Summarise-Quant-QuickReview}

Are the following statements \emph{true} or \emph{false}?

\begin{enumerate}
\def\labelenumi{\arabic{enumi}.}
\item
  The IQR measures the amount of variability in data. \tightlist
\item
  The mean and the median can both be called an `average'.
\item
  The mean and the median are not always the same value.
\item
  The range is a simple measure of variation in a set of data.
\item
  The standard deviation measures the amount of variability in data.
\item
  Another name for the median is \(Q_2\).
\item
  \(Q_3\) is the median of the largest half of the data.
\item
  The IQR is a useful measure of variation in data that are skewed.
\item
  The IQR is the difference between the first and second quartiles.
\item
  Another name for the \(75\)th percentile is \(Q_3\).
\item
  The units of the standard deviation and the IQR are the same as for the original data.
\end{enumerate}

\section{Exercises}\label{Summarise-Quant-Exercises}

\hyperref[Answers]{Answers to odd-numbered exercises} are given at the end of the book.

\captionsetup{font=small}

\begin{exercise}
\protect\hypertarget{exr:GraphABSdeaths}{}\label{exr:GraphABSdeaths}The \emph{Australian Bureau of Statistics} (ABS) records the age at death of Australians. The histogram of the age of death for females in 2012 is shown in Fig.~\ref{fig:DeathAgeLime} (left panel). Describe the distribution.
\end{exercise}

\begin{figure}[hbtp]

{\centering \includegraphics[width=1\linewidth]{11-SummaryQuant_files/figure-latex/DeathAgeLime-1} 

}

\caption{Left: histograms of age at death for female Australians in 2012. Right: the oven-dried foliage biomass for naturally-grown lime trees.}\label{fig:DeathAgeLime}
\end{figure}

\begin{exercise}
\protect\hypertarget{exr:GraphLimeTrees}{}\label{exr:GraphLimeTrees}{[}\emph{Dataset}: \texttt{Lime}{]} \citet{schepaschenko2017bpdb} measured the oven-dried foliage biomass of lime trees grown in natural environments. The histogram of the foliage biomass is shown in Fig.~\ref{fig:DeathAgeLime} (right panel). Describe the distribution.
\end{exercise}

\begin{exercise}
\protect\hypertarget{exr:NumericalQuantNHANES}{}\label{exr:NumericalQuantNHANES}

{[}\emph{Dataset}: \texttt{NHANES}{]} The histogram of the direct HDL cholesterol concentration from the American National Health and Nutrition Examination Survey (\textsc{nhanes}) \citep{data:NHANES:Rpackage} from \(1999\)--\(2004\) is shown in Fig.~\ref{fig:NHANESCherryRipeHist} (left panel).

\begin{enumerate}
\def\labelenumi{\arabic{enumi}.}
\tightlist
\item
  Should the mean or median be used to measure the `average' HDL cholesterol concentration? Explain.
\item
  Describe the distribution.
\end{enumerate}

\end{exercise}



\begin{figure}[hbtp]

{\centering \includegraphics[width=1\linewidth]{11-SummaryQuant_files/figure-latex/NHANESCherryRipeHist-1} 

}

\caption{Left: the histogram of direct HDL cholesterol concentration from the \textsc{nhanes} study (large outliers exist but are hard to see, as the sample size is very large). Right: the weights of `Fun Size' \emph{Cherry Ripe} chocolate bars.}\label{fig:NHANESCherryRipeHist}
\end{figure}

\begin{exercise}
\protect\hypertarget{exr:NumericalQuantCherryRipes}{}\label{exr:NumericalQuantCherryRipes}

{[}\emph{Dataset}: \texttt{CherryRipe}{]} The histogram of the weights of `Fun Size' \emph{Cherry Ripe} chocolate bars between~2016 and~2019 is shown in Fig.~\ref{fig:NHANESCherryRipeHist} (right panel).

\begin{enumerate}
\def\labelenumi{\arabic{enumi}.}
\tightlist
\item
  Should the mean or median be used to measure the `average' weight of a `Fun Size' \emph{Cherry Ripe} bar? Explain.
\item
  Describe the distribution.
\end{enumerate}

\end{exercise}

\begin{exercise}
\protect\hypertarget{exr:NumericalQuantRides}{}\label{exr:NumericalQuantRides}

\citet{levenson2005amusement} recorded the number of fatalities at amusement rides in the US from~\(1994\) to~\(2003\) (Table~\ref{tab:Fatalities}). Using software or a calculator, compute:

\begin{enumerate}
\def\labelenumi{\arabic{enumi}.}
\tightlist
\item
  the sample mean number of fatalities per year over this period.
\item
  the sample median number of fatalities per year over this period.
\item
  the sample standard deviation of the number of fatalities per year over this period.
\item
  the sample IQR of the number of fatalities per year over this period.
\end{enumerate}

\end{exercise}

\begin{table}
\centering
\caption{\label{tab:Fatalities}Fatalities at amusement park rides in the US.}
\centering
\fontsize{8}{10}\selectfont
\begin{tabular}[t]{lcccccccccc}
\toprule
\textbf{ } & \textbf{1994} & \textbf{1995} & \textbf{1996} & \textbf{1997} & \textbf{1998} & \textbf{1999} & \textbf{2000} & \textbf{2001} & \textbf{2002} & \textbf{2003}\\
\midrule
Fatalities: & $2$ & $4$ & $3$ & $4$ & $7$ & $6$ & $1$ & $3$ & $2$ & $5$\\
\bottomrule
\end{tabular}
\end{table}

\begin{exercise}
\protect\hypertarget{exr:NumericalQuantFulmars}{}\label{exr:NumericalQuantFulmars}

\citet{data:Furness1996:Fulmars} studied fulmars (a seabird). The mass of the female birds were (in grams): \(635\);\enskip \(635\);\enskip \(668\);\enskip \(640\);\enskip \(645\);\enskip \(635\).

\begin{enumerate}
\def\labelenumi{\arabic{enumi}.}
\tightlist
\item
  Construct a stemplot (using the first two digits as the stems).
\item
  Using your calculator, find the value of the \emph{sample} mean.
\item
  Using your calculator, find the value of the \emph{sample} standard deviation.
\item
  Find the value of the \emph{sample} median.
\item
  Find the value of the \emph{population} standard deviation.
\end{enumerate}

\end{exercise}

\begin{exercise}
\protect\hypertarget{exr:NumericalQuantSOI}{}\label{exr:NumericalQuantSOI}

Draw a stemplot of the average monthly SOI (from the Australian \emph{Bureau of Meteorology}) in August from~\(1995\) to~\(2003\) (Table~\ref{tab:SOIvalues}). Then, use your calculator (where possible) to calculate the:

\begin{cols}

\begin{col}{0.4\textwidth}

\begin{enumerate}
\def\labelenumi{\arabic{enumi}.}
\tightlist
\item
  sample mean
\item
  sample median.
\end{enumerate}

\end{col}

\begin{col}{0.05\textwidth}
~

\end{col}

\begin{col}{0.5\textwidth}

\begin{enumerate}
\def\labelenumi{\arabic{enumi}.}
\setcounter{enumi}{2}
\tightlist
\item
  range.
\item
  sample standard deviation.
\item
  sample IQR.
\end{enumerate}

\end{col}

\end{cols}

\end{exercise}

\begin{table}
\centering
\caption{\label{tab:SOIvalues}The average monthly SOI values in August from 1995 to 2003.}
\centering
\fontsize{8}{10}\selectfont
\begin{tabular}[t]{lccccccccc}
\toprule
\textbf{ } & \textbf{1995} & \textbf{1996} & \textbf{1997} & \textbf{1998} & \textbf{1999} & \textbf{2000} & \textbf{2001} & \textbf{2002} & \textbf{2003}\\
\midrule
Monthly average SOI: & $0.8$ & $4.6$ & $\phantom{0}\llap{$-{}$}19.8$ & $9.8$ & $2.1$ & $5.3$ & $\phantom{0}\llap{$-{}$}8.2$ & $\phantom{0}\llap{$-{}$}14.6$ & $\phantom{0}\llap{$-{}$}1.8$\\
\bottomrule
\end{tabular}
\end{table}

\begin{exercise}
\protect\hypertarget{exr:WeightFries}{}\label{exr:WeightFries}{[}\emph{Dataset}: \texttt{FriesWt}{]} \citet{data:Wetzel2005:McDonalds} weighed orders of french fries to determine how they matched the target weight of \(171\,\text{g}\) (Table~\ref{tab:FriesWtTable}).

\begin{enumerate}
\def\labelenumi{\arabic{enumi}.}
\tightlist
\item
  Produce graphs to summarise the data.
\item
  Use software to produce numerical summary information.
\end{enumerate}

Do you think the weights meet the target weight, on average?
\end{exercise}

\begin{table}
\centering
\caption{\label{tab:FriesWtTable}The weight of servings of french fries.}
\centering
\fontsize{8}{10}\selectfont
\begin{tabular}[t]{ccccccccccc}
\toprule
\multicolumn{11}{c}{\textbf{Weight of large orders of fries (in g)}} \\
\cmidrule(l{3pt}r{3pt}){1-11}
$117.0$ & $132.0$ & $134.0$ & $139.0$ & $141.0$ & $143.0$ & $146.0$ & $152.0$ & $154.0$ & $155.0$ & $157.0$\\
$126.0$ & $133.0$ & $137.0$ & $139.0$ & $142.0$ & $143.5$ & $146.0$ & $152.0$ & $154.5$ & $156.0$ & $176.0$\\
$128.0$ & $133.0$ & $138.0$ & $140.0$ & $142.5$ & $145.0$ & $151.0$ & $154.0$ & $154.5$ & $156.5$ & $117.0$\\
\bottomrule
\end{tabular}
\end{table}

\begin{exercise}
\protect\hypertarget{exr:SummaryQuantOrthoses}{}\label{exr:SummaryQuantOrthoses}

{[}\emph{Dataset}: \texttt{Orthoses}{]} \citet{swinnen2018influence} studied the influence of using ankle-foot orthoses in children with cerebral palsy. The data for the \(15\) subjects is shown in Table~\ref{tab:DescribeAnkleFoot}.

\begin{enumerate}
\def\labelenumi{\arabic{enumi}.}
\tightlist
\item
  Compute the values of the sample mean, sample median, sample standard deviation and sample IQR for the heights.
\item
  What are the values of the population mean, population median, population standard deviation and population IQR for the heights?
\item
  Produce a stemplot of the children's heights.
\item
  Produce a dot chart of the children's heights.
\item
  Produce a histogram of the children's heights.
\item
  Describe the distribution of the children's heights.
\end{enumerate}

\end{exercise}

\begin{exercise}
\protect\hypertarget{exr:NumericalQuantMatchingMicroPlastics}{}\label{exr:NumericalQuantMatchingMicroPlastics}

An article studied patients who had been admitted to Castle Hill Hospital \citep{data:detection:jenner2022}. The total number of microplastics found in the lungs of each patient are shown in Table~\ref{tab:Microplastics}. For these patients:

\begin{enumerate}
\def\labelenumi{\arabic{enumi}.}
\tightlist
\item
  Draw a stemplot, using the numbers as (say)~\(8.0\), with the decimals as the leaves.
\item
  What are the values of the sample mean and sample median number of microplastics?
\item
  What are the values of the population mean and population median number of microplastics?
\item
  What is the value of the sample standard deviation of the number of microplastics?
\item
  What is the value of the sample IQR of the number of microplastics?
\end{enumerate}

\end{exercise}

\begin{table}
\centering
\caption{\label{tab:Microplastics}The number of microplastics found in $11$ patients.}
\centering
\fontsize{8}{10}\selectfont
\begin{tabular}[t]{ccccccccccc}
\toprule
\multicolumn{11}{c}{\textbf{Number of microplastics}} \\
\cmidrule(l{3pt}r{3pt}){1-11}
$8$ & $3$ & $5$ & $2$ & $0$ & $2$ & $1$ & $7$ & $5$ & $1$ & $0$\\
\bottomrule
\end{tabular}
\end{table}

\begin{exercise}
\protect\hypertarget{exr:NumericalQuantDescribeBrainFreezeHistogram}{}\label{exr:NumericalQuantDescribeBrainFreezeHistogram}Describe the histogram in Fig.~\ref{fig:HistBrainFreeze} for the brain-freeze data.
\end{exercise}

\begin{exercise}
\protect\hypertarget{exr:NumericalQuantCompareSD}{}\label{exr:NumericalQuantCompareSD}The standard deviation for Dataset~A in Fig.~\ref{fig:TwoDatasets} is \(s = 2\). Will the standard deviation of Dataset~B be \emph{smaller} than or \emph{greater} than~\(2\)? Why?
\end{exercise}

\begin{figure}[hbtp]

{\centering \includegraphics[width=0.9\linewidth]{11-SummaryQuant_files/figure-latex/TwoDatasets-1} 

}

\caption{Dotplots of two datasets (with jittering).}\label{fig:TwoDatasets}
\end{figure}

\begin{exercise}
\protect\hypertarget{exr:JeansIQR1}{}\label{exr:JeansIQR1}

{[}\emph{Dataset}: \texttt{Jeans}{]} \citet{PuddingJeans} recorded the size of pockets in men's and women's jeans, including the minimum heights of the front pockets (Fig.~\ref{fig:JeansBoxplot}, left panel).

\begin{enumerate}
\def\labelenumi{\arabic{enumi}.}
\tightlist
\item
  What proportion of jeans in the sample have a minimum height less than~\(17\,\text{cm}\), for men's and women's jeans?
\item
  What proportion of jeans in the sample have a minimum height less than~\(13.25\,\text{cm}\), for men's and women's jeans?
\end{enumerate}

\end{exercise}

\begin{figure}[hbtp]

{\centering \includegraphics[width=1\linewidth]{11-SummaryQuant_files/figure-latex/JeansBoxplot-1} 

}

\caption{Left: the minimum height of the height of front pockets in jeans. Right: the price of different styles of women's jeans.}\label{fig:JeansBoxplot}
\end{figure}

\begin{exercise}
\protect\hypertarget{exr:JeansIQR2}{}\label{exr:JeansIQR2}

\citet{PuddingJeans} recorded data on the price of different styles of women's jeans (Fig.~\ref{fig:JeansBoxplot}, right panel).

\begin{enumerate}
\def\labelenumi{\arabic{enumi}.}
\tightlist
\item
  What proportion of boot-cut jeans in the sample cost less than~\$\(80\)?
\item
  What proportion of skinny jeans in the sample cost less than~\$\(80\)?
\item
  What proportion of straight jeans in the sample cost less than~\$\(80\)?
\item
  In general, which type of jeans are the cheapest?
\end{enumerate}

\end{exercise}

\begin{exercise}
\protect\hypertarget{exr:StatsHistograms}{}\label{exr:StatsHistograms}

A professor has recorded the marks (as a percentage) for all students in her four classes for an assignment. All classes have the same number of students. The corresponding histograms are shown in Fig.~\ref{fig:VariationHistograms}.

\begin{enumerate}
\def\labelenumi{\arabic{enumi}.}
\tightlist
\item
  In which class would the median be the largest?
\item
  In which class would the median be the smallest?
\item
  In which class would the standard deviation be the largest?
\item
  In which class would the standard deviation be the smallest?
\end{enumerate}

\end{exercise}

\begin{figure}[hbtp]

{\centering \includegraphics[width=0.75\linewidth]{11-SummaryQuant_files/figure-latex/VariationHistograms-1} 

}

\caption{Histogram of marks for four classes.}\label{fig:VariationHistograms}
\end{figure}

\begin{exercise}
\protect\hypertarget{exr:StatsHistograms2}{}\label{exr:StatsHistograms2}Consider the four histograms in Fig.~\ref{fig:FourHistograms}. Which histogram is most likely to describe the \emph{shape} of the following data? Why?

\begin{enumerate}
\def\labelenumi{\arabic{enumi}.}
\tightlist
\item
  The time that students remain in an examination room for a \emph{short}, \emph{easy} two-hour examination.
\item
  The heights of females at a local adults' dance club.
\item
  The \emph{starting} salaries of new science graduates employed full-time.
\item
  The volume of drink in~\(375\,\text{mL}\) cans of soft drink.
\item
  The time that students remain in the examination room for a \emph{hard}, \emph{long} two-hour examination.
\end{enumerate}

(The first bar of the histogram is not necessarily at zero; it is the \emph{shape} of the histogram that is of interest here: right skewed, left skewed, symmetric, etc.)
\end{exercise}

\begin{figure}[hbtp]

{\centering \includegraphics[width=0.75\linewidth]{11-SummaryQuant_files/figure-latex/FourHistograms-1} 

}

\caption{Four histograms: where would they be useful?}\label{fig:FourHistograms}
\end{figure}

\captionsetup{font=normalsize}

\begin{EOCanswerBox}{iconmonstr-check-mark-14-240.png}
\textbf{Answers to \emph{Quick review} questions:} Only Statement~9 is false.

\end{EOCanswerBox}

\chapter{Summarising qualitative data}\label{SummariseQualData}

\index{Qualitative data!summarising|(}

\begin{cols}
\begin{col}{0.52\textwidth}

\begin{objectivesBox}{iconmonstr-target-4-240.png}
So far, you have learnt to ask an RQ, design a study, collect the data, classify the data, and summarise quantitative data.
\textbf{In this chapter}, you will learn to:

\begin{itemize}\tightlist
  \item
  summarise qualitative data using the appropriate graphs.
  \item
  summarise qualitative data using, for example, medians, proportions and odds.
\end{itemize}
\end{objectivesBox}

\end{col}

\begin{col}{0.03\textwidth}
~
\end{col}

\begin{col}{0.45\textwidth}

\includegraphics[width=0.95\linewidth]{12-SummaryQual_files/figure-latex/unnamed-chunk-8-1} 
\end{col}
\end{cols}

\section{Introduction}\label{SummaryQual-Intro}

Many quantitative research studies involve qualitative variables. Except for very small amounts of data, understanding the data is difficult without a summary. As with quantitative data, qualitative data can be understood by knowing how often values of the variables appear. This is called the \emph{distribution} of the data (Def.~\ref{def:Distribution}).\index{Distribution!qualitative data}

The distribution can be displayed using a frequency table (Sect.~\ref{QualitativeTables}) or a graph (Sect.~\ref{QualitativeGraphs}). Qualitative data can be summarised by finding modes or, for ordinal qualitative data, using medians (Sect.~\ref{SummariseDataQualitative}). The distribution of qualitative data can be summarised numerically by computing proportions, percentages (Sect.~\ref{QualitativeProportionsPercentages}) or odds (Sect.~\ref{QualOdds}).

\section{Frequency tables for qualitative data}\label{QualitativeTables}

\index{Qualitative data!frequency table}

Qualitative data are typically collated in a \emph{frequency table}.\index{Frequency table!qualitative data} The rows (or the columns) should list the \emph{levels} of the variable, and these should be \emph{exhaustive} (cover all levels) and \emph{mutually exclusive} (observations belong to only one level).\index{Qualitative data!levels} The number of observations or the percentage of observations (or both) are then given for each level.

For \emph{nominal} data, the levels of the variables can be displayed in alphabetical order, in order of size, in order of personal preference, or in any other order: use the order most likely to be useful to readers. For \emph{ordinal} data, the natural order of the levels should almost always be used.

\begin{example}[Opinions of AV vehicles]
\protect\hypertarget{exm:AVstudy}{}\label{exm:AVstudy}\citet{pyrialakou2020perceptions} surveyed \(400\)~residents of Phoenix (Arizona) about their opinions of autonomous vehicles (AVs). Demographic information (Table~\ref{tab:AVtable1}) and respondents' opinions of sharing roads with AVs (Table~\ref{tab:AVtable2}) were recorded.

The gender of the respondent is \emph{nominal} (two levels), while the age group is \emph{ordinal} (six levels). The levels are shown in the rows. The three questions about safety (Table~\ref{tab:AVtable2}) all yield \emph{ordinal} responses (five levels, in columns).
\end{example}

\begin{table}
\centering
\caption{\label{tab:AVtable1}Demographic information for the AV data for $400$ respondents.}
\centering
\fontsize{8}{10}\selectfont
\begin{tabular}[t]{lcc}
\toprule
\textbf{ } & \textbf{Number} & \textbf{Percentage}\\
\midrule
\addlinespace[0.3em]
\multicolumn{3}{l}{\textbf{Gender ($n = 400$)}}\\
\hspace{1em}Female & $204$ & $51$\\
\hspace{1em}Male & $196$ & $49$\\
\addlinespace[0.3em]
\multicolumn{3}{l}{\textbf{Age group ($n = 400$)}}\\
\hspace{1em}$18$ to $24$ & $\phantom{0}52$ & $13$\\
\hspace{1em}$25$ to $34$ & $\phantom{0}76$ & $19$\\
\hspace{1em}$35$ to $44$ & $\phantom{0}76$ & $19$\\
\hspace{1em}$45$ to $54$ & $\phantom{0}72$ & $18$\\
\hspace{1em}$55$ to $64$ & $\phantom{0}56$ & $14$\\
\hspace{1em}$65+$ & $\phantom{0}68$ & $17$\\
\bottomrule
\end{tabular}
\end{table}

\begin{table}
\centering
\caption{\label{tab:AVtable2}Responses to three scenarios for the AV data for $400$ respondents (rows sum to $n = 400$).}
\centering
\fontsize{8}{10}\selectfont
\begin{tabular}[t]{>{}lcccccccccc}
\toprule
\multicolumn{1}{c}{\textbf{ }} & \multicolumn{2}{c}{\textbf{ }} & \multicolumn{2}{c}{\textbf{Somewhat}} & \multicolumn{2}{c}{\textbf{ }} & \multicolumn{2}{c}{\textbf{Somewhat}} & \multicolumn{2}{c}{\textbf{ }} \\
\multicolumn{1}{c}{\textbf{ }} & \multicolumn{2}{c}{\textbf{Unsafe}} & \multicolumn{2}{c}{\textbf{unsafe}} & \multicolumn{2}{c}{\textbf{Neutral}} & \multicolumn{2}{c}{\textbf{safe}} & \multicolumn{2}{c}{\textbf{Safe}} \\
\cmidrule(l{3pt}r{3pt}){2-3} \cmidrule(l{3pt}r{3pt}){4-5} \cmidrule(l{3pt}r{3pt}){6-7} \cmidrule(l{3pt}r{3pt}){8-9} \cmidrule(l{3pt}r{3pt}){10-11}
\textbf{ } & \textbf{$n$} & \textbf{\%} & \textbf{$n$} & \textbf{\%} & \textbf{$n$} & \textbf{\%} & \textbf{$n$} & \textbf{\%} & \textbf{$n$} & \textbf{\%}\\
\midrule
\textbf{Driving near an AV} & $58$ & $14$ & $\phantom{0}79$ & $20$ & $\phantom{0}96$ & $24$ & $97$ & $24$ & $70$ & $18$\\
\textbf{Cycling near an AV} & $77$ & $19$ & $104$ & $26$ & $\phantom{0}87$ & $22$ & $76$ & $19$ & $56$ & $14$\\
\textbf{Walking near an AV} & $63$ & $16$ & $\phantom{0}86$ & $22$ & $103$ & $26$ & $82$ & $20$ & $66$ & $16$\\
\bottomrule
\end{tabular}
\end{table}

\section{Graphs for qualitative data}\label{QualitativeGraphs}

\index{Qualitative data!graphs|(}\index{Graphs!qualitative data}\index{Software output!graphs}

Three options for graphing qualitative data include:

\begin{itemize}
\tightlist
\item
  \emph{dot charts} (Sect.~\ref{DotChartsOneQual}), which are usually a good choice.
\item
  \emph{bar charts} (Sect,~\ref{BarCharts}), which are usually a good choice.
\item
  \emph{pie charts} (Sect.~\ref{PieCharts}), which are only useful in special circumstances, and can be hard to interpret.
\end{itemize}

Sometimes these graphs are used for \emph{discrete} quantitative data with a small number of possible options.

\begin{importantBox}{iconmonstr-warning-8-240.png}
The purpose of a graph is to display the information in the clearest, simplest possible way, to facilitate understanding the message(s) in the data.

\end{importantBox}

\subsection{Dot charts (qualitative data)}\label{DotChartsOneQual}

\index{Graphs!dot chart!one qualitative variable}

Dot charts indicate the counts (or corresponding percentages) in each level using dots (or some other symbol). The levels can be on the horizontal or vertical axis, and the counts or percentages on the other. Placing the levels on the vertical axis often makes for easier reading, and space for long labels.

\begin{example}[Dot plots]
\protect\hypertarget{exm:DotPlotsQual}{}\label{exm:DotPlotsQual}For the AV study in Example~\ref{exm:AVstudy}, a dot chart of the age group of respondents is shown in Fig.~\ref{fig:AVDotBarPie} (top left panel).
\end{example}

\begin{figure}[hbtp]

{\centering \includegraphics[width=1\linewidth]{12-SummaryQual_files/figure-latex/AVDotBarPie-1} 

}

\caption{The age group of respondents in the AV study. All graphs present the same data.}\label{fig:AVDotBarPie}
\end{figure}

For dot charts:

\begin{itemize}
\tightlist
\item
  place the qualitative variable on the horizontal or vertical axis (and label with the levels of the variable).
\item
  use counts or percentages on the other axis.
\item
  for nominal data, \emph{think about the most helpful order} for the levels.
\end{itemize}

\begin{importantBox}{iconmonstr-warning-8-240.png}
The axis displaying the counts (or percentages) should \emph{start from zero}, since the distance of the dots from the axis visually implies the frequency of those observations (see Example~\ref{exm:VerticalTruncation}).

\end{importantBox}

\subsection{Bar charts}\label{BarCharts}

\index{Graphs!bar chart}

Bar charts use bars to represent the number (or percentage) of observations in each level. As with dot charts, the levels can be on the horizontal or vertical axis, but placing the level names on the vertical axis often makes for easier reading, and room for long labels.

\begin{example}[Bar plots]
\protect\hypertarget{exm:BarchartQual}{}\label{exm:BarchartQual}For the AV study in Example~\ref{exm:AVstudy}, a bar chart of the age group of respondents is shown in Fig.~\ref{fig:AVDotBarPie} (top right panel).
\end{example}

For bar charts:

\begin{itemize}
\tightlist
\item
  place the qualitative variable on the horizontal or vertical axis (and label with the levels of the variable).
\item
  use counts or percentages on the other axis.
\item
  for nominal data, levels can be ordered any way: \emph{think about the most helpful order}.
\item
  bars have gaps between bars, as the bars represent distinct categories.
\end{itemize}

In contrast to bar charts, bars in histograms are butted together (except when an interval has a zero count), as the variable-axis usually represents a continuous numerical scale.

\begin{importantBox}{iconmonstr-warning-8-240.png}
The axis displaying the counts (or percentages) should \emph{start from zero}, since the height of the bars visually implies the frequency of those observations (see Example~\ref{exm:VerticalTruncation}).

\end{importantBox}

\subsection{Pie charts}\label{PieCharts}

\index{Graphs!pie chart}

In pie charts, a circle is divided into segments proportional to the number in each level of the qualitative variable.

\begin{example}[Pie charts]
\protect\hypertarget{exm:PieChartsQual}{}\label{exm:PieChartsQual}For the AV study in Example~\ref{exm:AVstudy}, a pie chart of the age group of respondents is shown in Fig.~\ref{fig:AVDotBarPie} (bottom left panel).
\end{example}

Using pie charts may present challenges (see Sect.~\ref{PieChartProblems}):

\begin{itemize}
\tightlist
\item
  pie charts only work when graphing parts of a whole.
\item
  pie charts only work when \emph{all} options are present (`exhaustive').
\item
  pie charts are difficult to use with levels having zero or small counts (see Example~\ref{fig:PieSmallCounts}).
\item
  pie charts are difficult to interpret when many categories are present.
\item
  pie charts are hard to read, as humans compare \emph{lengths} (bar and dot charts) better than \emph{angles} (pie charts) \citep{data:Friel:Graphs}.
\end{itemize}

\begin{example}[Pie chart unsuitable]
\protect\hypertarget{exm:PieUnsuitable}{}\label{exm:PieUnsuitable}Consider studying the percentage of people who use Firefox, Chrome, and Safari as web browsers. A pie chart is \emph{not suitable} for displaying the data, as people can use more than one of these browsers (i.e., the options are not \emph{mutually exclusive}) nor \emph{exhaustive} (i.e., other options exist).
\end{example}

\subsection{Comparing dot, bar and pie charts}\label{CompareBarPie}

\index{Graphs!bar chart!compared to other graphs}\index{Graphs!pie chart!compared to other graphs}\index{Graphs!dot chart!compared to other graphs}

In the pie chart (Fig.~\ref{fig:AVDotBarPie}, bottom left panel), determining \emph{which} age groups have the fewest and most respondents is hard. The equivalent bar chart or dot chart makes the comparison easy. The \emph{tilted} pie chart makes this comparison even harder (Fig.~\ref{fig:AVDotBarPie}, bottom right panel).

Recall that the \emph{purpose of a graph is to display the information in the clearest, simplest possible way, to facilitate understanding the message(s) in the data}. A pie chart often makes the message hard to see \citep{siegrist1996use}.\index{Graphs!pie chart!warnings} \index{Qualitative data!graphs|)}

\clearpage

\section{Numerical summary: proportions and percentages}\label{QualitativeProportionsPercentages}

\index{Proportions}\index{Percentages}

Qualitative data can be summarised numerically by using the \emph{proportion} or \emph{percentage} of individuals in each level. These can be given instead of, or with, the counts (Tables~\ref{tab:AVtable1} and~\ref{tab:AVtable2}).

\begin{definition}[Proportion]
\protect\hypertarget{def:Proportion}{}\label{def:Proportion}A \emph{proportion} is a fraction out of a total, and is a number between~\(0\) and~\(1\).
\end{definition}

\begin{definition}[Percentages]
\protect\hypertarget{def:Percentage}{}\label{def:Percentage}A \emph{percentage} is a proportion, multiplied by~\(100\). In this context, percentages are numbers between~\(0\)\% and~\(100\)\%.
\end{definition}

\emph{Population} proportions are almost always unknown. Instead, the \emph{population} proportion (the parameter), denoted~\(p\), is estimated by a \emph{sample} proportion (a statistic), denoted by~\(\hat{p}\).\index{Estimate}

\begin{pronounceBox}{iconmonstr-microphone-7-240.png}
The symbol~\(\hat{p}\) is pronounced `pee-hat', and refers to a \emph{sample} proportion. The caret above the~\(p\) is called a `hat'.

\end{pronounceBox}

\begin{importantBox}{iconmonstr-warning-8-240.png}
As always, only one possible sample is studied. \emph{Statistics} are estimates of \emph{parameters}, and the value of the \emph{statistic} is not the same for every possible \emph{sample}.

\end{importantBox}

\begin{example}[Proportions and percentages]
\protect\hypertarget{exm:AVProportionsPercentages}{}\label{exm:AVProportionsPercentages}Consider the AV data in Table~\ref{tab:AVtable1}, summarising results from a sample of \(n = 400\) respondents. The \emph{sample proportion} of respondents aged~\(25\) to~\(34\) is \(76\div 400\), or~\(0.19\). The \emph{sample percentage} of respondents aged~\(25\) to~\(34\) is \(0.19 \times 100\), or~\(19\)\%, as in the table.
\end{example}

\section{Numerical summary: odds}\label{QualOdds}

\index{Odds}

For the AV data in Table~\ref{tab:AVtable1}, the number of females is slightly larger than the number of males. Specifically, the \emph{ratio} of females to males is \(204\div 196 = 1.04\); that is, there are~\(1.04\) \emph{times} as many females as males. This value of~\(1.04\) is the \emph{odds} that a respondent is female in the sample. An alternative interpretation is that there are \(1.04\times 100 = 104\) females for every~\(100\) males in the sample.

While proportions and percentages are computed as the number of results of interest divided by the \emph{total number}, the \emph{odds} are computed as the number of results of interest divided by \emph{the remaining number} (Fig.~\ref{fig:PropOdds}).

\begin{definition}[Odds]
\protect\hypertarget{def:Odds}{}\label{def:Odds}The \emph{odds} are the number (or proportion, or percentage) of results of interest, divided by the remaining number (or proportion, or percentage) of results: \[
         \text{Odds} = \frac{\text{Number of results of interest}}{\text{Remaining number of results}}
\] or (equivalently) \[
         \text{Odds} 
          = 
            \frac{\text{Proportion of results of interest}}
                 {\text{Remaining proportion of results}}
          = 
            \frac{\text{Percentage of results of interest}}
                 {\text{Remaining percentage of results}}.
\] The \emph{odds} are how many \emph{times} the result of interest \emph{occurs} compared to the number of times the results of interest does \emph{not occur}.
\end{definition}

\begin{figure}[hbtp]

{\centering \includegraphics[width=0.9\linewidth]{12-SummaryQual_files/figure-latex/PropOdds-1} 

}

\caption{Proportions (left) are the number of interest divided by the total number. Odds (right) are the number of interest divided by the rest.}\label{fig:PropOdds}
\end{figure}

\begin{example}[Interpreting odds]
\protect\hypertarget{exm:AVOddsMale}{}\label{exm:AVOddsMale}The AV data (Table~\ref{tab:AVtable1}) includes~\(204\) females and \(196\)~males. The \emph{odds} that a respondent is female is~\(1.04\). The odds are greater than one, as there are more females than males. Alternatively, there are~\(104\) females for every~\(100\) males.

The \emph{odds} that a respondent is male is \(196/204 = 0.96\); there are \(0.96\)~\emph{times} the number of males as females. The odds are less than one, as there are fewer males than females. Alternatively, there are \(96\)~males for every~\(100\) females.
\end{example}

When interpreting odds:

\begin{itemize}
\tightlist
\item
  odds \emph{greater} than~\(1\) mean the result of interest is \emph{more} likely to happen than not.
\item
  odds \emph{equal to}~\(1\) mean the result of interest is \emph{equally likely} to happen as not.
\item
  odds \emph{less} than~\(1\) mean the result of interest is \emph{less} likely to happen than not.
\end{itemize}

\begin{example}[Odds and percentages]
\protect\hypertarget{exm:AVOdds}{}\label{exm:AVOdds}Consider the AV data in Table~\ref{tab:AVtable1}, summarising results from a sample of \(n = 400\) respondents.

The percentage of respondents aged~\(18\) to~\(24\) is \(52/400\times 400 = 13\)\%. The \emph{odds} that a respondent is aged~\(18\) to~\(24\) is \(52/(400 - 52) = 0.15\). This means the number of respondents aged~\(18\) to~\(24\) is \(0.15\)~times (i.e., less then) the number of respondents aged over~\(24\).

The \emph{odds} that a respondent is aged~\(18\) to~\(54\) is \((52 + 76 + 76 + 72)/(56 + 68) = 2.23\). This means the number of respondents aged~\(18\) to~\(54\) is~\(2.23\) times (i.e., greater than) the number of respondents aged~\(55\) or over.
\end{example}

The \emph{population} odds (the parameter) are almost always unknown, and are estimated by the \emph{sample} odds (the statistic). No symbol is commonly used to denote odds.

Take care: proportions and odds are similar, but are different ways of numerically summarising quantitative data (Fig.~\ref{fig:PropOdds}).

\section{Describing the distribution: modes and medians}\label{SummariseDataQualitative}

\index{Qualitative data!distribution}

Graphs are constructed to help readers understand the data, so any important features in the graph should be described. One simple way is to identify the level (or levels) with the \emph{most} observations. This is called the \emph{mode}.\index{Mode}

\begin{definition}[Mode]
\protect\hypertarget{def:Mode}{}\label{def:Mode}A \emph{mode} is the level (or levels) of a qualitative variable with the most observations.
\end{definition}

\begin{example}[Modes]
\protect\hypertarget{exm:OrdinalModes}{}\label{exm:OrdinalModes}

Consider the data in Tables~\ref{tab:AVtable1} and~\ref{tab:AVtable2}:

\begin{itemize}
\tightlist
\item
  the \emph{mode} for gender is `Female' (with~\(204\) respondents, or~\(51\)\%).
\item
  the \emph{mode} age groups are \(25\) to~\(34\) and \(35\) to~\(44\) (each with \(19\)~respondents, or~\(4.8\)\%).
\item
  the \emph{modal} response to the question about \emph{driving} near AVs is `Somewhat safe'.
\item
  the \emph{modal} response to the question about \emph{cycling} near AVs is `Somewhat unsafe'.
\item
  the \emph{modal} response to the question about \emph{walking} near AVs is `Neutral'.
\end{itemize}

\end{example}

\emph{Medians}\index{Median!qualitative ordinal data} can be found for \emph{ordinal} data (but \emph{not} nominal data), since ordinal data have levels with a natural order. The \emph{median} is the level in which the middle response is located, when the levels from all individuals are placed in order. The sample median estimates the unknown \emph{population} median.

\begin{importantBox}{iconmonstr-warning-8-240.png}
Medians can be used to summarise \emph{quantitative data} and \emph{ordinal} data, but \emph{never} nominal data.

\end{importantBox}

\begin{example}[Medians]
\protect\hypertarget{exm:OrdinalMedians}{}\label{exm:OrdinalMedians}Consider the data in Tables~\ref{tab:AVtable1} and~\ref{tab:AVtable2}. `Gender' is \emph{nominal} qualitative, so medians are not appropriate. However, the other variables are \emph{ordinal}, so medians could be used to describe each variable. Since \(n = 400\), the median response will be halfway between the location of the~\(200\)th and~\(201\)st response when ordered:

\begin{itemize}
\tightlist
\item
  the \emph{median} age group is \(35\) to~\(44\).
\item
  the \emph{median} response to the driving-near-AVs question is `Neutral'.
\item
  the \emph{median} response to the cycling-near-AVs question is `Neutral'.
\item
  the \emph{median} response to the walking-near-AVs question is `Neutral'.
\end{itemize}

For each variable, ordered observations~\(200\) and~\(201\) both fall into the indicated level.
\end{example}

Importantly, all these numerical quantities are computed from a sample (i.e., are statistics; Def.~\ref{def:Statistic}), even though the whole population is of interest (i.e., the parameter; Def.~\ref{def:Parameter}).

Means (Sect.~\ref{Mean}) are generally not suitable for numerically summarising qualitative data. However, \emph{ordinal} data \emph{may be} numerically summarised like quantitative data in \emph{rare and very special circumstances}. Means may be appropriate if both of these are true:

\begin{itemize}
\tightlist
\item
  the levels are considered equally spaced.
\item
  assigning a number to each level is appropriate (for example, using a mid-point for numerical age groups).
\end{itemize}

We will not consider means for ordinal data further.

\section{Numerical summary tables}\label{QualSummaryTable}

\index{Qualitative data!summary tables}

Qualitative variables should be summarised in a table. The table should include, as a minimum, numbers and/or percentages for each level. While useful in other contexts (see Chap.~\ref{CompareQualData}), odds are usually not given in the summary table. Examples are shown in Tables~\ref{tab:AVtable1} and~\ref{tab:AVtable2}, and in the next section.

\section{Example: water access}\label{WaterAccessQual}

\citet{lopez2022farmers} recorded data about access to water for three rural communities in Cameroon (see Sect.~\ref{WaterAccessQuant}). Numerous qualitative variables are recorded; some are displayed in Fig.~\ref{fig:WaterAcessQual}, and summarised in Table~\ref{tab:WaterAccessQual}. Notice that the levels\index{Levels} of the two ordinal variables are displayed in their natural order.

The distance to the nearest water source is usually less than~\(1\,\text{km}\), and the wait is often over~\(15\,\text{mins}\). The most common water source (i,e., the mode) is a bore (\(68.6\)\%). The median (and mode) distance to the water source was \(100\,\text{m}\) to \(1000\,\text{m}\); the median wait time was~\(5\) to~\(15\,\text{mins}\) (the mode wait time was under \(5\,\text{mins}\)).

\begin{table} \centering \centering\caption{\label{tab:WaterAccessQual}Summarising some qualitative data in the water-access study. Left: the ordinal variables. Right: the nominal variable.}

\fontsize{8}{10}\selectfont
\begin{tabular}[t]{lccc}
\toprule
\textbf{ } & \textbf{Number} & \textbf{\%} & \textbf{Odds}\\
\midrule
\addlinespace[0.3em]
\multicolumn{4}{l}{\textbf{Distance to water source ($n = 121$)}}\\
\hspace{1em}Under $100\,\text{m}$ & $55$ & $45.5$ & $0.83$\\
\hspace{1em}$100\,\text{m}$\ to $1000\,\text{m}$ & $57$ & $47.1$ & $0.89$\\
\hspace{1em}Over $1000\,\text{m}$ & $\phantom{0}9$ & $\phantom{0}7.4$ & $0.08$\\
\addlinespace[0.3em]
\multicolumn{4}{l}{\textbf{Wait time at water source ($n = 120$)}}\\
\hspace{1em}Under $5\,\text{mins}$ & $50$ & $41.7$ & $0.71$\\
\hspace{1em}$5$ to $15\,\text{mins}$ & $28$ & $23.3$ & $0.30$\\
\hspace{1em}Over $15\,\text{mins}$ & $42$ & $35.0$ & $0.54$\\
\bottomrule
\end{tabular} \qquad\qquad 
\begin{tabular}[t]{lccc}
\toprule
\textbf{ } & \textbf{Number} & \textbf{\%} & \textbf{Odds}\\
\midrule
\addlinespace[0.3em]
\multicolumn{4}{l}{\textbf{Water source ($n = 121$)}}\\
\hspace{1em}Tap & $\phantom{0}7$ & $\phantom{0}5.8$ & $0.06$\\
\hspace{1em}Bore & $83$ & $68.6$ & $2.18$\\
\hspace{1em}Well & $16$ & $13.2$ & $0.15$\\
\hspace{1em}River & $15$ & $12.4$ & $0.14$\\
\bottomrule
\end{tabular}
\end{table}

\begin{figure}[hbtp]

{\centering \includegraphics[width=1\linewidth]{12-SummaryQual_files/figure-latex/WaterAcessQual-1} 

}

\caption{The distance to the water source (left), the wait time at the water source (centre), and the water sources (right) for the water-access study.}\label{fig:WaterAcessQual}
\end{figure}

\index{Qualitative data!summarising|)}

\clearpage

\section{Chapter summary}\label{SummaryQual-Summary}

Qualitative data can be graphed with a dot chart, bar chart or (in special circumstances) pie chart. Qualitative data can be described using the mode or (for \emph{ordinal} data only) a median. Qualitative data can be numerically summarised using \emph{proportions}, \emph{percentages} or \emph{odds}.

\section{Quick review questions}\label{SummaryQual-QuickReview}

Are the following statements \emph{true} or \emph{false}?

\begin{enumerate}
\def\labelenumi{\arabic{enumi}.}
\item
  Nominal data can be summarised using a median. \tightlist
\item
  Ordinal data can be summarised using a mode.
\item
  Odds are the ratio of how often a result of interest occurs, to how often it does \emph{not} occur.
\item
  Proportions and percentages are the same.
\end{enumerate}

\section{Exercises}\label{SummariseQualDataExercises}

\hyperref[Answers]{Answers to odd-numbered exercises} are given at the end of the book.

\captionsetup{font=small}

\begin{exercise}
\protect\hypertarget{exr:SpiderMonkeys}{}\label{exr:SpiderMonkeys}

A study of spider monkeys \citep{data:Chapman1990:SpiderMonkeys} examined the types of social groups present (Table~\ref{tab:SpiderMonkeysLATEX}).

\begin{enumerate}
\def\labelenumi{\arabic{enumi}.}
\tightlist
\item
  Construct a suitable plot, and explain what the data reveal.
\item
  Determine, if appropriate, the median and mode social group.
\end{enumerate}

\end{exercise}

\begin{table} \centering \centering\caption{\label{tab:SpiderMonkeysLATEX}Social groups for spider monkeys.}

\fontsize{8}{10}\selectfont
\begin{tabular}{rc}
\toprule
\textbf{Social group} & \textbf{Percentage}\\
\midrule
Solitary & $\phantom{0}8$\\
All males & $\phantom{0}3$\\
Female + no young & $\phantom{0}2$\\
Mixed young & $15$\\
\bottomrule
\end{tabular} \qquad 
\begin{tabular}{rc}
\toprule
\textbf{Social group} & \textbf{Percentage}\\
\midrule
Mixed + no young & $\phantom{0}1$\\
One female + offspring & $23$\\
Many females + offspring & $48$\\
 & \\
\bottomrule
\end{tabular}
\end{table}

\begin{exercise}
\protect\hypertarget{exr:QualSummary}{}\label{exr:QualSummary}

\citet{czarniecka2021consumer} studied how Poles prepared and consumed coffee using a sample of \(1\,500\)~Poles. Some data are shown in Table~\ref{tab:CoffeePoles}.

\begin{enumerate}
\def\labelenumi{\arabic{enumi}.}
\tightlist
\item
  Classify the variables as quantitative, nominal or ordinal.
\item
  Sketch appropriate graphs for the three variables.
\item
  Summarise the three variables.
\item
  Where appropriate, compute the median and mode for each variable.
\end{enumerate}

\end{exercise}

\begin{table} \centering \centering\caption{\label{tab:CoffeePoles}Location of coffee consumption, brewing temperature and brewing time, from $1\,500$ Poles.}

\fontsize{8}{10}\selectfont
\begin{tabular}[t]{rc}
\toprule
\textbf{Where consumed} & \textbf{$n$}\\
\midrule
Home & $1432$\\
Canteen & $\phantom{0}687$\\
Cafe & $\phantom{0}922$\\
Others' homes & $\phantom{0}994$\\
Work & $1196$\\
\bottomrule
\end{tabular} \quad\quad 
\begin{tabular}[t]{rc}
\toprule
\textbf{Brew Temp.} & \textbf{$n$}\\
\midrule
\text{$100^\circ$C} & $748$\\
\text{$98^\circ$C} & $269$\\
\text{$93^\circ$C} & $453$\\
\addlinespace
Unknown & $\phantom{0}30$\\
\bottomrule
\end{tabular} \quad\quad 
\begin{tabular}[t]{rc}
\toprule
\textbf{Brew time} & \textbf{$n$}\\
\midrule
Under $3$ mins & $226$\\
About $3$ mins & $267$\\
About $4$ mins & $114$\\
About $5$ mins & $\phantom{0}82$\\
About $6$ mins & $\phantom{0}30$\\
\addlinespace
Unknown & $781$\\
\bottomrule
\end{tabular}
\end{table}

\begin{exercise}
\protect\hypertarget{exr:GraphsCars}{}\label{exr:GraphsCars}\citet{henderson1981building} recorded the number of cylinders in many models of cars: eleven cars had four cylinders, seven cars had six cylinders, and fourteen cars had eight cylinders. The \emph{number} of cylinders is quantitative discrete, but with so few different values, the data could be plotted with a graph used for qualitative data. For these data:

\begin{cols}

\begin{col}{0.45\textwidth}

\begin{enumerate}
\def\labelenumi{\arabic{enumi}.}
\tightlist
\item
  Produce a dot chart.
\item
  Produce a histogram.
\end{enumerate}

\end{col}

\begin{col}{0.05\textwidth}
~

\end{col}

\begin{col}{0.45\textwidth}

\begin{enumerate}
\def\labelenumi{\arabic{enumi}.}
\setcounter{enumi}{2}
\tightlist
\item
  Produce a bar chart.
\item
  Produce a pie chart.
\end{enumerate}

\end{col}

\end{cols}

\null\smallskip

What graph do you think is best? Why?
\end{exercise}

\begin{exercise}
\protect\hypertarget{exr:GraphSurveyData}{}\label{exr:GraphSurveyData}A survey of voice assistants (such as Amazon Echo; Google Home; etc.) conducted by Nielsen asked respondents to indicate how they used their voice assistant. The options were:

\begin{cols}

\begin{col}{0.33\textwidth}

\begin{itemize}
\tightlist
\item
  listening to music;
\item
  listen to news;
\item
  use alarms, timer;
\end{itemize}

\end{col}

\begin{col}{0.03\textwidth}
~

\end{col}

\begin{col}{0.60\textwidth}

\begin{itemize}
\tightlist
\item
  search for real-time information (e.g., traffic; weather);
\item
  search for factual information (e.g., trivia; history);
\item
  chat with voice assistant for fun.
\end{itemize}

\end{col}

\end{cols}

Respondents could select all options that applied. What would be the best graph for displaying respondents answers? Would a pie chart be suitable? Explain your answer.
\end{exercise}

\begin{exercise}
\protect\hypertarget{exr:OrdinalMedians}{}\label{exr:OrdinalMedians}

\citet{gkebski2019impact} studied the taste of bread with varying salt and fibre content. Information was recorded from \(300\) subjects, including the subjects' responses to the statement `Rolls with lower salt content taste worse than regular ones', on a five-point ordinal scale from `Strongly Agree' to `Strongly Disagree'; see Table~\ref{tab:Bread}.

\begin{enumerate}
\def\labelenumi{\arabic{enumi}.}
\tightlist
\item
  Identify the variables, then classify them as nominal or ordinal.
\item
  For which variables is a mode an appropriate summary (if any)?
\item
  For which variables is a median an appropriate summary (if any)?
\item
  Compute the above statistics where appropriate.
\item
  Compute and interpret the odds of a respondent coming from a city background.
\item
  Compute and interpret the odds of a respondent agreeing \emph{or} strongly agreeing with the statement.
\item
  Compute and interpret the odds of a respondent being male.
\end{enumerate}

\end{exercise}

\begin{table}
\centering
\caption{\label{tab:Bread}The bread-tasting data ($n = 300)$.}
\centering
\fontsize{8}{10}\selectfont
\begin{tabular}[t]{rcc}
\toprule
\textbf{} & \textbf{Number} & \textbf{Percentage}\\
\midrule
\addlinespace[0.3em]
\multicolumn{3}{l}{\textbf{Gender}}\\
\hspace{1em}Female & $150$ & $50$\\
\hspace{1em}Male & $150$ & $50$\\
\addlinespace[0.3em]
\multicolumn{3}{l}{\textbf{Place of residence}}\\
\hspace{1em}Rural & $49$ & $16$\\
\hspace{1em}City up to $20\, 000$ residents & $38$ & $13$\\
\hspace{1em}City $20\, 000$ to $100\, 000$ residents & $83$ & $28$\\
\hspace{1em}City $> 100\, 000$ residents & $130$ & $43$\\
\addlinespace[0.3em]
\multicolumn{3}{l}{\textbf{Response to statement}}\\
\hspace{1em}Strongly agree & $30$ & $10$\\
\hspace{1em}Agree & $84$ & $28$\\
\hspace{1em}Neutral & $78$ & $26$\\
\hspace{1em}Disagree & $66$ & $22$\\
\hspace{1em}Strongly disagree & $42$ & $14$\\
\bottomrule
\end{tabular}
\end{table}

\begin{exercise}
\protect\hypertarget{exr:ReclaimedWater}{}\label{exr:ReclaimedWater}

\citet{lopez2022farmers} asked \(231\)~farmers what they considered to be the advantages and disadvantages of using reclaimed water on the farm. The responses are shown in Table~\ref{tab:ReclaimedWater} (not all farmers responded).

\begin{enumerate}
\def\labelenumi{\arabic{enumi}.}
\tightlist
\item
  Produce two bar charts to display the data.
\item
  Produce two dot charts to display the data.
\item
  Produce two pie charts to display the data.
\item
  Determine the mode for both the advantages and disadvantages.
\item
  Compute the percentages for both the advantages and disadvantages.
\item
  Compute the odds of a farmer stating `high price' as a disadvantage, among \emph{all} farmers.
\item
  Compute the odds of a farmer stating `high price' as a disadvantage, among farmers who listed a disadvantage.
\item
  What is the difference in the meaning of the last two statements?
\end{enumerate}

\end{exercise}

\begin{table} \centering \centering\caption{\label{tab:ReclaimedWater}The advantages and disadvantages of using reclaimed water, reported by $231$ farmers. (Not all farmers responded.)}

\fontsize{8}{10}\selectfont
\begin{tabular}[t]{rc}
\toprule
\textbf{Advantage} & \textbf{No. farmers}\\
\midrule
Water reutilization & $15$\\
Availability & $27$\\
Sustainability & $16$\\
\bottomrule
\end{tabular} \quad\quad 
\begin{tabular}[t]{rc}
\toprule
\textbf{Disadvantage} & \textbf{No. farmers}\\
\midrule
High price & $40$\\
Growing conductivity & $12$\\
Lack of proper filtering & $21$\\
\bottomrule
\end{tabular}
\end{table}

\begin{exercise}
\protect\hypertarget{exr:StudentTransport}{}\label{exr:StudentTransport}

\citet{henning2020modelling} studied \(284\) university students in Joinville, Brazil, tabulating how students got to campus (Table~\ref{tab:TransportTable}; each student could select one option only).

\begin{enumerate}
\def\labelenumi{\arabic{enumi}.}
\tightlist
\item
  What is the mode type of active transport? What about motorised transport?
\item
  What is the mode type of transport overall?
\item
  Are medians appropriate? If so, compute the median for active transport types, and motorised transport types.
\item
  Compute the proportions for each option, out of the total sample.
\item
  Compute the odds that a randomly-chosen student uses motorised transport to get to campus. Explain what this means.
\item
  Compute the odds that a student walks to campus. Explain what this means.
\item
  Construct appropriate plots to display the data.
\end{enumerate}

\end{exercise}

\begin{table} \centering \centering\caption{\label{tab:TransportTable}Modes of transport for students getting to campus.}

\fontsize{8}{10}\selectfont
\begin{tabular}[t]{lc}
\toprule
\textbf{ } & \textbf{Number: active methods}\\
\midrule
Bicycle & $29$\\
Walking & $35$\\
\bottomrule
\end{tabular} \quad\quad 
\begin{tabular}[t]{lc}
\toprule
\textbf{ } & \textbf{Number: motorised methods}\\
\midrule
Car & $\phantom{0}70$\\
Bus & $117$\\
Other & $\phantom{0}33$\\
\bottomrule
\end{tabular}
\end{table}

\begin{exercise}
\protect\hypertarget{exr:QualSumBabyBoom}{}\label{exr:QualSumBabyBoom}

{[}\emph{Dataset}: \texttt{BabyBoom}{]} Table~\ref{tab:BabyBoomDataLATEX} shows the gender of \(44\)~babies born in a hospital on one day \citep{mypapers:Dunn:dataset:1999, data:Steele:BabyBoom}. The data are given in the order in which the births occurred.

\begin{enumerate}
\def\labelenumi{\arabic{enumi}.}
\tightlist
\item
  What is the mode sex?
\item
  If appropriate, compute the median sex.
\item
  Compute the percentages for each sex.
\item
  Compute the odds that a randomly-chosen baby from the sample is female. Explain what this means.
\item
  Construct appropriate plots to display sex of the baby.
\end{enumerate}

\end{exercise}

\begin{exercise}
\protect\hypertarget{exr:FEVplots}{}\label{exr:FEVplots}

{[}\emph{Dataset}: \texttt{LungCap}{]} \citet{data:Tager:FEV} studied the lung volume of \(654\)~children in East Boston in the~1970s (Table~\ref{tab:LungCapTab}).

\begin{enumerate}
\def\labelenumi{\arabic{enumi}.}
\tightlist
\item
  Construct suitable plots for all variables.
\item
  For each qualitative variable, determine the mode.
\item
  For each qualitative variable, compute the percentage and odds of one of the levels occurring in the data.
\item
  Compute appropriate statistics for each quantitative variable.
\end{enumerate}

\end{exercise}

\begin{table}
\centering
\caption{\label{tab:LungCapTab}The lung volume (FEV) for youth in East Boston in the 1970s; the first six observations in the dataset ($n = 654$).}
\centering
\fontsize{8}{10}\selectfont
\begin{tabular}[t]{ccccc}
\toprule
\textbf{Age} & \textbf{FEV} & \textbf{Height} & \textbf{Gender} & \textbf{Smoking}\\
\midrule
$3$ & $1.072$ & $46$ & F & No\\
$4$ & $0.839$ & $48$ & F & No\\
$4$ & $1.102$ & $48$ & F & No\\
\addlinespace
$4$ & $1.389$ & $48$ & F & No\\
$4$ & $1.577$ & $49$ & F & No\\
$4$ & $1.418$ & $49$ & F & No\\
\addlinespace
$\vdots$ & $\vdots$ & $\vdots$ & $\vdots$ & $\vdots$\\
\bottomrule
\end{tabular}
\end{table}

\begin{exercise}
\protect\hypertarget{exr:SummariseUniOrthoses}{}\label{exr:SummariseUniOrthoses}

\citet{swinnen2018influence} studied the influence of using ankle-foot orthoses in children with cerebral palsy. The data in Table~\ref{tab:DescribeAnkleFoot} give the data for the \(15\)~subjects. (\textsc{Gmfcs} is the Gross Motor Function Classification System) used to describe the impact of cerebral palsy on their motor function; where \emph{lower} levels mean \emph{better} functionality.)

\begin{enumerate}
\def\labelenumi{\arabic{enumi}.}
\tightlist
\item
  Construct suitable plots for all variables.
\item
  For each qualitative variable, determine the mode.
\item
  For each qualitative variable, compute the percentage and odds of one of the levels occurring in the data.
\item
  Compute appropriate statistics for each quantitative variable.
\end{enumerate}

\end{exercise}

\captionsetup{font=normalsize}

\begin{EOCanswerBox}{iconmonstr-check-mark-14-240.png}
\textbf{Answers to \emph{Quick review} questions:} \textbf{1.} False. \textbf{2.} True. \textbf{3.} True. \textbf{4.} Percentages are proportions multiplied by~\(100\), so similar (but different).

\end{EOCanswerBox}

\chapter{Comparing quantitative data within individuals}\label{SummariseWithin}

\index{Quantitative data!changes \textit{within} individuals}

\begin{cols}
\begin{col}{0.52\textwidth}

\begin{objectivesBox}{iconmonstr-target-4-240.png}
So far, you have learnt to ask an RQ, design a study, collect the data, describe the data and summarise the data.
\textbf{In this chapter}, you will learn to:

\begin{itemize}\tightlist
  \item
   summarise within-individual changes in quantitative data using appropriate graphs.
  \item
  summarise within-individual changes using summary tables.
\end{itemize}
\end{objectivesBox}

\end{col}

\begin{col}{0.03\textwidth}
~
\end{col}

\begin{col}{0.45\textwidth}

\includegraphics[width=0.95\linewidth]{13-CompareWithin_files/figure-latex/unnamed-chunk-5-1} 
\end{col}
\end{cols}

\section{Introduction}\label{introduction-1}

Sometimes the same quantitative variable is measured on each individual more than once (i.e., \emph{within}-individual changes for each unit of analysis) but only a small number of times. Examples of this type of data include:

\begin{itemize}
\tightlist
\item
  measurements of weekly household water consumption for many households, \emph{before} and \emph{after} installing water-saving devices.
\item
  blood pressure recorded for people at \(8\)am, \(1\)pm and \(8\)pm each day.
\end{itemize}

In both cases, the same variable is measured multiple times for each individual. This chapter studies summarising within-individuals changes in \emph{quantitative} variables.

\section{Numerical summary: mean differences}\label{WithinSummaryTables}

\index{Quantitative data!changes \textit{within} individuals!summary tables}\index{Summary table!mean difference}\index{Mean difference}

When each individual has \emph{two} observations, the difference between the observations can be computed for each individual. Then, the appropriate numerical summary is to summarise these \emph{differences} over all individuals; for example, using the \emph{mean} of these \emph{differences}.

When \emph{more than two} observations are made for each individual, the changes from the first observation (the \emph{reference level})\index{Reference level} may be computed (e.g., if the first observation is pre-intervention, or a benchmark, observation); for example, using the \emph{mean change} (see Sect.~\ref{PainRelievingTape}).

\begin{example}[Within-individual comparisons]
\protect\hypertarget{exm:WithinIndividualComparison}{}\label{exm:WithinIndividualComparison}\citet{data:Lothian2006:Whey} studied children with atopic asthma, and measured the immunogoblin~E concentrations (IgE) before and after an intervention for each child (Table~\ref{tab:IgEChanges}), plus the \emph{reduction} in IgE for each child. The child is the \emph{individual}.\index{Individuals}

For the IgE data, the numerical summary table is shown in Table~\ref{tab:IgETable}. The direction of the difference is implied by the word `\emph{reduction}'.
\end{example}

\begin{importantBox}{iconmonstr-warning-8-240.png}
In the numerical summary table, the information for the differences is \emph{not} found by subtracting the information in one row from the other. In Table~\ref{tab:IgETable}, for example, the number of differences is not \(11 - 11 = 0\); the standard deviation of the differences is \emph{not} \(1615.53 - 1354.4 = 261.13\,\ensuremath{\mu}\text{g}\).\(\,\text{L}^{-1}\). These statistics are computed from the differences (i.e., the \textbf{Reductions} in Table~\ref{tab:IgEChanges}).

\end{importantBox}

\begin{table} \centering \centering\caption{\label{tab:IgEChanges}The IgE before and after an intervention, and the change in IgE (in $\mu$g.$\,\text{L}^{-1}$).}

\fontsize{8}{10}\selectfont
\begin{tabular}{ccc}
\toprule
\multicolumn{1}{c}{\textbf{Before}} & \multicolumn{1}{c}{\textbf{After}} & \multicolumn{1}{c}{\textbf{Reduction}} \\
\textbf{(in $\mu$g.$\,\text{L}^{-1}$)} & \textbf{(in $\mu$g.$\,\text{L}^{-1}$)} & \textbf{(in $\mu$g.$\,\text{L}^{-1}$)}\\
\midrule
$\phantom{0}\phantom{0}83$ & $\phantom{0}\phantom{0}83$ & $\phantom{0}\phantom{0}\phantom{0}0$\\
$\phantom{0}292$ & $\phantom{0}292$ & $\phantom{0}\phantom{0}\phantom{0}0$\\
$\phantom{0}293$ & $\phantom{0}292$ & $\phantom{0}\phantom{0}\phantom{0}1$\\
\addlinespace
$\phantom{0}623$ & $\phantom{0}542$ & $\phantom{0}\phantom{0}81$\\
$\phantom{0}792$ & $\phantom{0}709$ & $\phantom{0}\phantom{0}83$\\
$1543$ & $1000$ & $\phantom{0}543$\\
\bottomrule
\end{tabular} \qquad 
\begin{tabular}{ccc}
\toprule
\multicolumn{1}{c}{\textbf{Before}} & \multicolumn{1}{c}{\textbf{After}} & \multicolumn{1}{c}{\textbf{Reduction}} \\
\textbf{(in $\mu$g.$\,\text{L}^{-1}$)} & \textbf{(in $\mu$g.$\,\text{L}^{-1}$)} & \textbf{(in $\mu$g.$\,\text{L}^{-1}$)}\\
\midrule
$1668$ & $1000$ & $\phantom{0}668$\\
$1960$ & $1626$ & $\phantom{0}334$\\
$2877$ & $2502$ & $\phantom{0}375$\\
\addlinespace
$2961$ & $2711$ & $\phantom{0}250$\\
$5504$ & $4504$ & $1000$\\
 &  & \\
\bottomrule
\end{tabular}
\end{table}

\begin{table}
\centering\centering
\caption{\label{tab:IgETable}A numerical summary of the IgE data in $\mu$g.$\,\text{L}^{-1}$).}
\centering
\fontsize{8}{10}\selectfont
\begin{tabular}[t]{lccc}
\toprule
\textbf{ } & \textbf{Mean} & \textbf{Std dev.} & \textbf{Sample size}\\
\midrule
Before & $1690.5$ & $1615.53$ & $11$\\
After & $1387.4$ & $1354.28$ & $11$\\
\midrule
\em{Reduction} & \em{$\phantom{0}303.2$} & \em{$\phantom{0}325.28$} & \em{$11$}\\
\bottomrule
\end{tabular}
\end{table}

\section{Graphs for the differences}\label{graphs-for-the-differences}

\index{Quantitative data!changes \textit{within} individuals!graphs}\index{Graphs!changes \textit{within} individuals}\index{Software output!graphs}

For within-individual changes for a \emph{quantitative} variable, options for plotting include:

\begin{itemize}
\tightlist
\item
  \emph{histograms of differences} (Sect.~\ref{HistoDiffPlot}), which are useful for changes in \emph{pairs} of measurements or observations for each individual.
\item
  \emph{case-profile plots} (Sect.~\ref{CaseProfilePlot}), which are useful when the same individuals are measured or observed a small number of times.
\end{itemize}

\subsection{Histogram of differences}\label{HistoDiffPlot}

\index{Graphs!histogram of  differences}

Sometimes the same variable is measured on each unit of analysis twice, when the \emph{changes} (or \emph{differences}) for each individual can be produced, and a histogram of the changes or differences can be constructed. The direction of the differences should be clear (e.g., first measurement minus second, or second measurement minus first).

Issues relevant for constructing histograms (Sect.~\ref{Histograms}), such as bin widths and boundary values, also apply here.

\begin{importantBox}{iconmonstr-warning-8-240.png}
The axis displaying the counts (or percentages) should \emph{start from zero}, since the height of the bars visually implies the frequency of those differences (see Example~\ref{exm:VerticalTruncation}).

\end{importantBox}

\begin{example}[Within-individual comparisons]
\protect\hypertarget{exm:CaseProfileHistDiffPlots}{}\label{exm:CaseProfileHistDiffPlots}For the IgE data (Table~\ref{tab:IgEChanges}), the \emph{reduction} in IgE for each child can be shown using a histogram (Fig.~\ref{fig:PairedGraphCaseProfileLATEX}, left panel).
\end{example}

\begin{figure}[hbtp]

{\centering \includegraphics[width=1\linewidth]{13-CompareWithin_files/figure-latex/PairedGraphCaseProfileLATEX-1} 

}

\caption{The IgE data. Left: a histogram of the differences (reductions); boundary observations are counted in the lower box. Right: a case-profile plot, where each line joins each person's pre-intervention to their post-intervention measurement.}\label{fig:PairedGraphCaseProfileLATEX}
\end{figure}

\subsection{Case-profile plots}\label{CaseProfilePlot}

\index{Graphs!case-profile plot}

Sometimes the variable is measured or recorded more than twice, and so a single set of differences cannot be produced. In these cases, the values for each individual can be plotted using a case-profile plot: the measurements are shown on one axis (usually the vertical), and the various points at which measurements are taken are shown on the other axis. A case-profile plot is still useful for paired data, of course.

\begin{importantBox}{iconmonstr-warning-8-240.png}
The axis displaying the counts (or percentages) \emph{need not start from zero}, since the distance from the axis to the lines \emph{do not} visually imply any quantity of interest. Rather, the \emph{relative changes} represented by the lines display the quantity of interest.

\end{importantBox}

\begin{example}[Case-profile plot]
\protect\hypertarget{exm:CaseProfilePlots}{}\label{exm:CaseProfilePlots}For the IgE data (Table~\ref{tab:IgEChanges}), the measurements of IgE for each child at both times can be shown in a case-profile plot (Fig.~\ref{fig:PairedGraphCaseProfileLATEX}, right panel). Each line corresponds to a unit of analysis (i.e., a child).
\end{example}

\begin{example}[Case-profile plot]
\protect\hypertarget{exm:CaseProfileRunning}{}\label{exm:CaseProfileRunning}Runners use wearable devices to measure many performance indicators, including vertical oscillation (VO). VO contributes to running economy and injury risk, so reliable VO measurements are crucial. \citet{smith2022validity} compared four devices, and obtained data from video analysis for \(n = 150\) athletes; that is, each participant had the same runs measured using five methods. The case-profile plot (Fig.~\ref{fig:RunningRM}) shows the means for each method using a solid point; each line represents one runner. NOVA and Footpod give smaller VO measurements in general.
\end{example}

\begin{figure}[hbtp]

{\centering \includegraphics[width=1\linewidth]{13-CompareWithin_files/figure-latex/RunningRM-1} 

}

\caption{Vertical oscillation (VO) measured using five methods for $15$ runners. The solid black points represent the means for each method. Left: a line is plotted for each individual. Right: only the means are shown, with vertical lines from the minimum value to the maximum value for each method.}\label{fig:RunningRM}
\end{figure}

As in Example~\ref{exm:CaseProfileRunning}, the case-profile plot is hard to read with large numbers of individuals, and so sometimes the mean (or median, as appropriate) is shown, with some measure of the variation of the observations (Fig.~\ref{fig:RunningRM} shows the minimum and maximum values for each method, for instance).\index{Range}

\section{Example: invasive plants}\label{CompareWithinInvasivePlants}

Skypilot (\emph{Polemonium viscosum}) is a native alpine wildflower growing in the Colorado Rocky Mountains (USA). In recent years, a willow shrub (\emph{Salix}) has been encroaching on skypilot territory and, because willow often flowers early, researchers \citep{kettenbach2017shrub} are concerned that the willow may `negatively affect pollination regimes of resident alpine wildflower species' (p.~\(6\,965\)). One RQ was:

\begin{quote}
In the Colorado Rocky Mountains, what is the mean difference between first-flowering day for the native skypilot and the encroaching willow?
\end{quote}

Data for both species was collected at~\(25\) different sites (Table~\ref{tab:FloweringData}). The site is the \emph{individual}; the data are \emph{paired} (Sect.~\ref{PairedIntro}), a form of blocking\index{Blocking} (Sect.~\ref{ManagingConfounding}). The `first-flowering day' is the number of days since the start of the year (e.g., January~12 is `day~12') when flowers were first observed.

\begin{table} \centering \centering\caption{\label{tab:FloweringData}The day of the year of first flowering by encroaching willow and native skypilot.}

\fontsize{8}{10}\selectfont
\begin{tabular}[t]{ccc}
\toprule
\multicolumn{1}{c}{\textbf{ }} & \multicolumn{2}{c}{\textbf{First-flowering day}} \\
\cmidrule(l{3pt}r{3pt}){2-3}
\textbf{Site} & \textbf{Willow} & \textbf{Skypilot}\\
\midrule
$\phantom{0}1$ & $201$ & $201$\\
$\phantom{0}2$ & $178$ & $179$\\
$\phantom{0}3$ & $189$ & $189$\\
$\phantom{0}4$ & $189$ & $189$\\
$\phantom{0}5$ & $196$ & $203$\\
$\phantom{0}6$ & $207$ & $203$\\
$\phantom{0}7$ & $199$ & $199$\\
$\phantom{0}8$ & $178$ & $182$\\
$\phantom{0}9$ & $178$ & $178$\\
$10$ & $191$ & $191$\\
$11$ & $187$ & $192$\\
$12$ & $190$ & $197$\\
$13$ & $190$ & $190$\\
\bottomrule
\end{tabular} \quad\quad 
\begin{tabular}[t]{ccc}
\toprule
\multicolumn{1}{c}{\textbf{ }} & \multicolumn{2}{c}{\textbf{First-flowering day}} \\
\cmidrule(l{3pt}r{3pt}){2-3}
\textbf{Site} & \textbf{Willow} & \textbf{Skypilot}\\
\midrule
$14$ & $209$ & $209$\\
$15$ & $221$ & $221$\\
$16$ & $179$ & $188$\\
$17$ & $174$ & $179$\\
$18$ & $172$ & $166$\\
$19$ & $196$ & $196$\\
$20$ & $173$ & $173$\\
$21$ & $180$ & $173$\\
$22$ & $181$ & $179$\\
$23$ & $186$ & $186$\\
$24$ & $194$ & $209$\\
$25$ & $197$ & $197$\\
 &  & \\
\bottomrule
\end{tabular}
\end{table}

\clearpage

Since the data are available, the data should be summarised graphically (Fig.~\ref{fig:FloweringPlots}) and numerically (Table~\ref{tab:FloweringSummary}), using software output (Fig.~\ref{fig:Floweringjamovi}).\index{Software output!mean differences}

\begin{figure}[hbtp]

{\centering \includegraphics[width=1\linewidth]{jamovi/Flowering/Flowering-Descriptives} 

}

\caption{Software output for the flowering-day data.}\label{fig:Floweringjamovi}
\end{figure}

\begin{table}
\centering
\caption{\label{tab:FloweringSummary}The day of first flowering for encroaching willow and native skypilot.}
\centering
\fontsize{8}{10}\selectfont
\begin{tabular}[t]{lccc}
\toprule
\textbf{ } & \textbf{Mean} & \textbf{Std dev.} & \textbf{Sample size}\\
\midrule
Willow (encroaching) & $189.4$ & $12.20$ & $25$\\
Skypilot (native) & $190.8$ & $13.06$ & $25$\\
\midrule
\em{Differences} & \em{$\phantom{0}\phantom{0}1.4$} & \em{$\phantom{0}4.70$} & \em{$25$}\\
\bottomrule
\end{tabular}
\end{table}

\begin{figure}[hbtp]

{\centering \includegraphics[width=1\linewidth]{13-CompareWithin_files/figure-latex/FloweringPlots-1} 

}

\caption{The flowering-day data. Left: a histogram of the difference between the first-flowering days (skypilot minus willow). Right: a case-profile plot of days of first flowering (unfilled points and dashed lines indicate earlier or same dates (smaller or equal values) for willow).}\label{fig:FloweringPlots}
\end{figure}

\section{Example: pain-relieving tape}\label{PainRelievingTape}

\citet{naugle2021effect} studied the effect of using Kinesio Tape to alleviate pain in athletes. Pain was measured by applying a slow constant rate of pressure on the left arm, and subjects pressed a button when the sensation moved from pressure to pain. The pressure at which this occurred was recorded. This was repeated \(5\,\text{mins}\) before applying the tape, \(5\,\text{mins}\) after applying the tape, and again \(15\)--\(20\,\text{mins}\) after applying the tape.

Figure~\ref{fig:TapeRepeated} shows the reported pain for~\(16\) subjects. A summary table is shown in Table~\ref{tab:TapeTable}. The pain thresholds are increasing slightly as time progresses.

\begin{figure}[hbtp]

{\centering \includegraphics[width=0.8\linewidth]{13-CompareWithin_files/figure-latex/TapeRepeated-1} 

}

\caption{Pain threshold (left arm) at three time points when using Kinesio Tape, without applying tension, for $n = 16$ subjects. The filled, black points represent the means for each time point.}\label{fig:TapeRepeated}
\end{figure}

\begin{table}
\centering\centering
\caption{\label{tab:TapeTable}A numerical summary of the tape data: pain thresholds in kPa.}
\centering
\fontsize{8}{10}\selectfont
\begin{tabular}[t]{lccccc}
\toprule
\multicolumn{1}{c}{\textbf{ }} & \multicolumn{1}{c}{\textbf{ }} & \multicolumn{1}{c}{\textbf{ }} & \multicolumn{1}{c}{\textbf{Sample}} & \multicolumn{1}{c}{\textbf{Mean of change}} & \multicolumn{1}{c}{\textbf{Std dev. of change}} \\
\textbf{ } & \textbf{Mean} & \textbf{Std dev.} & \textbf{size} & \textbf{from Pre} & \textbf{from Pre}\\
\midrule
Pre: $5\,\text{mins}$ & $446.5$ & $175.18$ & $16$ &  & \\
Post: $5\,\text{mins}$ & $479.6$ & $199.61$ & $16$ & $33.1$ & $\phantom{0}73.93$\\
Post: $15-20\,\text{mins}$ & $506.9$ & $214.36$ & $16$ & $60.4$ & $102.72$\\
\bottomrule
\end{tabular}
\end{table}

\section{Chapter summary}\label{Chap13-Summary}

Quantitative data measured within individuals can be summarised using a histogram of differences when the variable is measured or observed twice, or a case-profile plot (with two or more measurement or observations per individual). A summary table should show the numerical summaries for the quantitative variable at each measurement or observation and for appropriate changes.

\section{Quick review questions}\label{Chap13-QuickReview}

Are the following statements \emph{true} or \emph{false}?

\begin{enumerate}
\def\labelenumi{\arabic{enumi}.}
\item
  A histogram of the differences is only appropriate for with two within-individuals measurements or observations. \tightlist
\item
  A case-profile plot is only appropriate for showing changes for two within-individuals measurements or observations.
\item
  The median and IQR are \emph{not} appropriate for summarising differences.
\item
  Explaining \emph{how} the differences are computed is important.
\end{enumerate}

\section{Exercises}\label{SummariseWithin-Exercises}

\hyperref[Answers]{Answers to odd-numbered exercises} are given at the end of the book.

\captionsetup{font=small}

\begin{exercise}
\protect\hypertarget{exr:CompareWithinInsulation}{}\label{exr:CompareWithinInsulation}

{[}\emph{Dataset}: \texttt{Insulation}{]} The \emph{Electricity Council} in Bristol wanted to determine if a certain type of wall-cavity insulation reduced average energy consumption in winter \citep{data:OpenUni:insulationBA, data:hand:handbook}:

\begin{quote}
In Bristol homes, what is the \emph{mean reduction} in energy consumption after adding home insulation?
\end{quote}

\begin{enumerate}
\def\labelenumi{\arabic{enumi}.}
\tightlist
\item
  What are the individuals (units of analysis)?
\item
  Explain why this study uses a within-individuals comparison.
\item
  Use the collected data (Table \ref{tab:DataInsulation}) to sketch a case-profile plot.
\item
  Use the data to sketch a histogram of the differences.
\item
  Use software or a calculator to prepare a summary table.
\end{enumerate}

\end{exercise}

\begin{table} \centering \centering\caption{\label{tab:DataInsulation}The house insulation data: energy consumption before and after adding insulation, and the energy saving (all in MWh).}

\fontsize{8}{10}\selectfont
\begin{tabular}[t]{cccc}
\toprule
\textbf{Home} & \textbf{Before} & \textbf{After} & \textbf{Saving}\\
\midrule
A & $12.1$ & $12.0$ & $0.1$\\
B & $11.0$ & $10.6$ & $0.4$\\
C & $14.1$ & $13.4$ & $0.7$\\
D & $13.8$ & $11.2$ & $2.6$\\
E & $15.5$ & $15.3$ & $0.2$\\
\bottomrule
\end{tabular} \quad\quad 
\begin{tabular}[t]{cccc}
\toprule
\textbf{Home} & \textbf{Before} & \textbf{After} & \textbf{Saving}\\
\midrule
F & $12.2$ & $13.6$ & $\phantom{0}\llap{$-{}$}1.4$\\
G & $12.8$ & $12.6$ & $\phantom{0}0.2$\\
H & $\phantom{0}9.9$ & $\phantom{0}8.8$ & $\phantom{0}1.1$\\
I & $10.8$ & $\phantom{0}9.6$ & $\phantom{0}1.2$\\
J & $12.7$ & $12.4$ & $\phantom{0}0.3$\\
\bottomrule
\end{tabular}
\end{table}

\begin{exercise}
\protect\hypertarget{exr:CompareWithinExercisesCaptopril}{}\label{exr:CompareWithinExercisesCaptopril}

{[}\emph{Dataset}: \texttt{Captopril}{]} In a study of hypertension \citep{data:hand:handbook, data:macgregor:essential}, \(15\) patients were given a drug (Captopril) and their systolic blood pressure measured (in mm~Hg) immediately before and two hours after being given the drug (Table~\ref{tab:CICaptoprilData}).

\begin{enumerate}
\def\labelenumi{\arabic{enumi}.}
\tightlist
\item
  Explain why this study uses a within-individuals comparison.
\item
  Construct a histogram of the differences.
\item
  Construct a case-profile plot for the data.
\end{enumerate}

\end{exercise}

\begin{table} \centering \centering\caption{\label{tab:CICaptoprilData}The Captopril data: before after systolic blood pressures (in mm Hg).}

\fontsize{8}{10}\selectfont
\begin{tabular}{ccc}
\toprule
\textbf{Before} & \textbf{After} & \textbf{Differences}\\
\midrule
$210$ & $201$ & $\phantom{0}9$\\
$169$ & $165$ & $\phantom{0}4$\\
$187$ & $166$ & $21$\\
$160$ & $157$ & $\phantom{0}3$\\
$167$ & $147$ & $20$\\
\addlinespace
$176$ & $145$ & $31$\\
$185$ & $168$ & $17$\\
$206$ & $180$ & $26$\\
\bottomrule
\end{tabular} \quad 
\begin{tabular}{ccc}
\toprule
\textbf{Before} & \textbf{After} & \textbf{Differences}\\
\midrule
$173$ & $147$ & $26$\\
$146$ & $136$ & $10$\\
$174$ & $151$ & $23$\\
$201$ & $168$ & $33$\\
$198$ & $179$ & $19$\\
\addlinespace
$148$ & $129$ & $19$\\
$154$ & $131$ & $23$\\
 &  & \\
\bottomrule
\end{tabular}
\end{table}

\begin{exercise}
\protect\hypertarget{exr:CompareWithinPainRelief}{}\label{exr:CompareWithinPainRelief}{[}\emph{Dataset}: \texttt{PainRelief}{]} \citet{augustino2023dataset} measured the reported pain of new mothers in Dodoma (Tanzania) at four times: near giving birth, then \(20\),~\(40\) and~\(60\,\text{mins}\) after giving birth. Mothers were administered either paracetamol or a cold pack as pain relief. Pain was recorded using a `numeric rating scale represented by the horizontal line marked from zero to ten', where higher scores mean greater pain.

Since the number of individuals is large (\(n = 912\)), use the summary data in Table~\ref{tab:PainReliefTable} to sketch a plot of the means and the range, like that in Figure~\ref{fig:TapeRepeated}.
\end{exercise}

\begin{table}
\centering
\caption{\label{tab:PainReliefTable}A summary table of reported pain for mothers after giving birth.}
\centering
\fontsize{8}{10}\selectfont
\begin{tabular}[t]{>{}lrcccc}
\toprule
\multicolumn{1}{c}{\textbf{ }} & \multicolumn{1}{c}{\textbf{ }} & \multicolumn{1}{c}{\textbf{ }} & \multicolumn{1}{c}{\textbf{After}} & \multicolumn{1}{c}{\textbf{After}} & \multicolumn{1}{c}{\textbf{After}} \\
\textbf{ } & \textbf{} & \textbf{At birth} & \textbf{$20\,\text{mins}$} & \textbf{$40\,\text{mins}$} & \textbf{$60\,\text{mins}$}\\
\midrule
\textbf{Paracetamol} & Mean & $\phantom{0}7.44$ & $\phantom{0}6.89$ & $4.69$ & $2.84$\\
\textbf{($n = 456$)} & Standard deviation & $\phantom{0}2.01$ & $\phantom{0}1.83$ & $1.49$ & $1.19$\\
\textbf{} & Minimum & $\phantom{0}2.00$ & $\phantom{0}2.00$ & $2.00$ & $0.00$\\
\textbf{} & Maximum & $10.00$ & $10.00$ & $9.00$ & $7.00$\\
\addlinespace
\textbf{Cold pack} & Mean & $\phantom{0}8.63$ & $\phantom{0}5.67$ & $3.19$ & $0.99$\\
\textbf{($n = 455$)} & Standard deviation & $\phantom{0}1.40$ & $\phantom{0}2.03$ & $1.63$ & $0.99$\\
\textbf{} & Minimum & $\phantom{0}4.00$ & $\phantom{0}0.00$ & $0.00$ & $0.00$\\
\textbf{} & Maximum & $10.00$ & $\phantom{0}9.00$ & $6.00$ & $4.00$\\
\bottomrule
\end{tabular}
\end{table}

\begin{exercise}
\protect\hypertarget{exr:CompareWithinStressSurgeryCI}{}\label{exr:CompareWithinStressSurgeryCI}

{[}\emph{Dataset}: \texttt{Stress}{]} The concentration of beta-endorphins in the blood is a sign of stress. One study (\citet{data:hand:handbook}, Dataset~232; \citet{hoaglin2011exploring}) measured the beta-endorphin concentration for \(19\)~patients about to undergo surgery.

Each patient had their beta-endorphin concentrations measured \(12\)--\(14\,\text{h}\) before surgery, and also \(10\,\text{mins}\) before surgery. A numerical summary (from software output) is in Table~\ref{tab:StressTable}.\index{Software output!mean differences}

\begin{table}
\centering
\caption{\label{tab:StressTable}The numerical summary for the presurgical stress data.}
\centering
\fontsize{8}{10}\selectfont
\begin{tabular}[t]{lccc}
\toprule
\textbf{ } & \textbf{Mean} & \textbf{Std deviation} & \textbf{Sample size}\\
\midrule
12--14 h before surgery & $\phantom{0}8.35$ & $\phantom{0}4.397$ & $19$\\
10 min before surgery & $16.05$ & $12.509$ & $19$\\
\midrule
\em{Increase} & \em{$\phantom{0}7.70$} & \em{$13.519$} & \em{$19$}\\
\bottomrule
\end{tabular}
\end{table}

\begin{enumerate}
\def\labelenumi{\arabic{enumi}.}
\tightlist
\item
  Explain why this study uses a within-individuals comparison.
\item
  Explain why the standard deviation for the \emph{increase} is not the difference between the two individuals time-point standard deviations.
\item
  Using the data file and software, construct a histogram of the differences.
\item
  Using the data file and software, construct a case-profile plot for the data.
\end{enumerate}

\end{exercise}

\begin{exercise}
\protect\hypertarget{exr:CompareWithinRunning}{}\label{exr:CompareWithinRunning}{[}\emph{Dataset}: \texttt{Running}{]} Create a summary table for the data in Example~\ref{exm:CaseProfileRunning}.
\end{exercise}

\begin{exercise}
\protect\hypertarget{exr:CompareWithinWCTennis}{}\label{exr:CompareWithinWCTennis}

{[}\emph{Dataset}: \texttt{WCTennis}{]} \citet{alberca2022sprint} recorded the push-time for French wheelchair tennis players, while holding and not holding a racquet (Table~\ref{tab:WCTennis}; \citet{alberca2022sprintDATA}).

\begin{enumerate}
\def\labelenumi{\arabic{enumi}.}
\tightlist
\item
  What do the differences mean (as given in the table)?
\item
  Create a plot of the data.
\item
  Create a numerical summary table for the data.
\end{enumerate}

\end{exercise}

\begin{table}
\centering
\caption{\label{tab:WCTennis}The wheelchair-tennis data. One observation is missing.}
\centering
\fontsize{8}{10}\selectfont
\begin{tabular}[t]{cccc}
\toprule
\multicolumn{1}{c}{\textbf{ }} & \multicolumn{2}{c}{\textbf{Push-time (in s)}} & \multicolumn{1}{c}{\textbf{ }} \\
\cmidrule(l{3pt}r{3pt}){2-3}
\textbf{Person} & \textbf{With racquet} & \textbf{Without racquet} & \textbf{Difference (in s)}\\
\midrule
$\phantom{0}1$ & $0.2625$ & $0.1833$ & $\phantom{0}0.0792$\\
$\phantom{0}2$ & $0.2375$ & $0.2250$ & $\phantom{0}0.0125$\\
$\phantom{0}3$ & $0.2583$ & $0.2042$ & $\phantom{0}0.0542$\\
$\phantom{0}4$ & $0.1917$ & $0.1875$ & $\phantom{0}0.0042$\\
$\phantom{0}5$ & $0.1875$ & $0.1708$ & $\phantom{0}0.0167$\\
\addlinespace
$\phantom{0}6$ & $0.2542$ & $0.1750$ & $\phantom{0}0.0792$\\
$\phantom{0}7$ & $0.2333$ & $0.1917$ & $\phantom{0}0.0417$\\
$\phantom{0}8$ & $0.1917$ & $0.1708$ & $\phantom{0}0.0208$\\
$\phantom{0}9$ & $0.2208$ & $0.2208$ & $\phantom{0}0.0000$\\
$10$ & $0.2583$ & $0.2750$ & $\phantom{0}\llap{$-{}$}0.0167$\\
\addlinespace
$11$ & $0.2083$ & $0.1750$ & $\phantom{0}0.0333$\\
$12$ & --- & $0.2042$ & ---\\
$13$ & $0.2208$ & $0.2292$ & $\phantom{0}\llap{$-{}$}0.0083$\\
\bottomrule
\end{tabular}
\end{table}

\begin{exercise}
\protect\hypertarget{exr:CompareWithinJumping}{}\label{exr:CompareWithinJumping}

{[}\emph{Dataset}: \texttt{Jumping}{]} \citet{hebert2023effect} recorded double-legged jumping distance for \(80\) healthy people, when they wore shoes and were barefoot (Table~\ref{tab:Jumping}).

\begin{enumerate}
\def\labelenumi{\arabic{enumi}.}
\tightlist
\item
  What do the differences mean (as given in the table)?
\item
  Create a plot of the data.
\item
  Create a numerical summary table for the data.
\end{enumerate}

\end{exercise}

\begin{table} \centering \centering\caption{\label{tab:Jumping}Jumping distances for athletes, with and without shoes (the first four and the last four observations).}

\fontsize{8}{10}\selectfont
\begin{tabular}{ccc}
\toprule
\multicolumn{3}{c}{\textbf{Distance (in cm)}} \\
\cmidrule(l{3pt}r{3pt}){1-3}
\textbf{With shoes} & \textbf{Barefoot} & \textbf{Difference}\\
\midrule
$42.73$ & $42.23$ & $\phantom{0}0.50$\\
$41.00$ & $39.47$ & $\phantom{0}1.53$\\
$27.37$ & $30.40$ & $\phantom{0}\llap{$-{}$}3.03$\\
$46.80$ & $36.60$ & $10.20$\\
$\vdots$ & $\vdots$ & $\vdots$\\
\bottomrule
\end{tabular} \quad\quad 
\begin{tabular}{ccc}
\toprule
\multicolumn{3}{c}{\textbf{Distance (in cm)}} \\
\cmidrule(l{3pt}r{3pt}){1-3}
\textbf{With shoes} & \textbf{Barefoot} & \textbf{Difference}\\
\midrule
$\vdots$ & $\vdots$ & $\vdots$\\
$32.73$ & $33.90$ & $\phantom{0}\llap{$-{}$}1.17$\\
$56.50$ & $55.10$ & $\phantom{0}1.40$\\
$33.57$ & $32.07$ & $\phantom{0}1.50$\\
$27.77$ & $33.57$ & $\phantom{0}\llap{$-{}$}5.80$\\
\bottomrule
\end{tabular}
\end{table}

\captionsetup{font=normalsize}

\begin{EOCanswerBox}{iconmonstr-check-mark-14-240.png}
\textbf{Answers to \emph{Quick review} questions:} \textbf{1.} True. \textbf{2.} False; a case-profile plot can be used for \emph{two or more} within-individual comparisons. \textbf{3.} False; use whatever numerical summaries are appropriate. \textbf{4.} True.

\end{EOCanswerBox}

\chapter{Comparing quantitative data between individuals}\label{BetweenQuantData}

\index{Quantitative data!compare \textit{between} individuals}

\begin{cols}
\begin{col}{0.52\textwidth}

\begin{objectivesBox}{iconmonstr-target-4-240.png}
So far, you have learnt to ask an RQ, design a study, collect the data, describe the data, and summarise data.
\textbf{In this chapter}, you will learn to:

\begin{itemize}\tightlist
  \item
   compare quantitative data between individuals using the appropriate graphs.
  \item
  compare quantitative data between individuals in summary tables.
\end{itemize}
\end{objectivesBox}

\end{col}

\begin{col}{0.03\textwidth}
~
\end{col}

\begin{col}{0.45\textwidth}

\includegraphics[width=0.95\linewidth]{14-CompareQuant_files/figure-latex/unnamed-chunk-11-1} 
\end{col}
\end{cols}

\section{Introduction}\label{introduction-2}

Relational RQs compare groups. This chapter considers how to compare \emph{quantitative} variables in different groups. Graphs are useful this purpose, and a table of the numerical summaries usually is produced also.

\section{Numerical summary: difference between means}\label{CompareQuantTables}

\index{Quantitative data!comparing \textit{between} individuals!summary tables}\index{Summary table!comparing two means}\index{Mean!difference between}\index{Difference between means}

When comparing quantitative variables in different groups, the data should be summarised for each group. If two groups are being compared, the \emph{difference} between the means and/or medians of the two groups must also be computed. If more than two groups are being compared, the \emph{differences} between one of the group means/medians (the first, or the benchmark, or the initial situation as the reference level\index{Reference level}) and the other group means/medians are also usually computed.

\begin{example}[Numerical summary table]
\protect\hypertarget{exm:GorillaSummarytable}{}\label{exm:GorillaSummarytable}\citet{wright2021chest} recorded the number of chest-beats by gorillas (Table~\ref{tab:GorillaDataTable}), for gorillas under \(20\)~years old (`younger') and \(20\)~years and over (`older'). A summary of the data can be tabulated as in Table~\ref{tab:GorillaTable}. Notice that no standard deviation or sample size is provided for the \emph{difference}; these make no sense.
\end{example}

\begin{table}
\centering
\caption{\label{tab:GorillaDataTable}The chest-beating rate of gorillas (in beats per $10$\,\text{h}).}
\centering
\fontsize{8}{10}\selectfont
\begin{tabular}[t]{ccccccccccccc}
\toprule
\multicolumn{7}{c}{\textbf{Younger }} & \multicolumn{6}{c}{\textbf{Older}} \\
\cmidrule(l{3pt}r{3pt}){1-7} \cmidrule(l{3pt}r{3pt}){8-13}
$0.7$ & $1.3$ & $1.5$ & $1.7$ & $1.8$ & $3.0$ & $4.4$ & $0.0$ & $0.2$ & $0.4$ & $0.8$ & $1.1$ & $4.0$\\
$0.9$ & $1.5$ & $1.5$ & $1.7$ & $2.6$ & $4.1$ & $4.4$ & $0.1$ & $0.3$ & $0.6$ & $0.9$ & $1.6$ & \\
\bottomrule
\end{tabular}
\end{table}

\begin{table}
\centering\centering
\caption{\label{tab:GorillaTable}A numerical summary of the gorillas data.}
\centering
\fontsize{8}{10}\selectfont
\begin{tabular}[t]{lccc}
\toprule
\multicolumn{1}{c}{\textbf{ }} & \multicolumn{1}{c}{\textbf{Mean}} & \multicolumn{1}{c}{\textbf{Standard deviation}} & \multicolumn{1}{c}{\textbf{Sample}} \\
\textbf{ } & \textbf{(in beats per 10 h)} & \textbf{(in beats per 10 h)} & \textbf{size}\\
\midrule
Younger & $2.22$ & $1.270$ & $14$\\
Older & $0.91$ & $1.131$ & $11$\\
\midrule
\em{Difference} & \em{$1.31$} & \em{} & \em{}\\
\bottomrule
\end{tabular}
\end{table}

\section{Graphs for the comparison}\label{GraphOneQualOneQuant}

\index{Quantitative data!comparing \textit{between} individuals!graphs}\index{Graphs!comparing \textit{between} individuals}\index{Software output!graphs}

When a \emph{quantitative} variable is measured or observed in different groups (i.e., between individuals), the distribution of each variable can be graphed separately. However, to \emph{compare} the quantitative variable in the groups, appropriate graphs include:

\begin{itemize}
\tightlist
\item
  \emph{back-to-back stemplots} (Sect.~\ref{BackToBackStem}), which are best for small amounts of data (and only possible for comparing \emph{two groups}).
\item
  \emph{2-D dot charts} (Sect.~\ref{TwoDDot}), which are the best choice for small to moderate amounts of data.
\item
  \emph{boxplots} (Sect.~\ref{Boxplot}), which are the best choice except for small amounts of data.
\end{itemize}

These situations have one quantitative variable being compared in different groups (defined by \emph{one qualitative variable}).

\subsection{Back-to-back stemplot}\label{BackToBackStem}

\index{Graphs!back-to-back stemplot}

Back-to-back stemplots are two stemplots (Sect.~\ref{StemAndLeafPlots}) sharing the same stems; one group has the leaves emerging left-to-right from the stem, and the second group has the leaves emerging right-to-left from the stem. Back-to-back stemplots can only be used when \emph{two} groups are being compared. Again, one advantage of using stemplots over other plots is that the original data are retained. Disadvantages are that only two groups can be compared, and not all data work well with stemplots.

\begin{example}[Back-to-back stemplots]
\protect\hypertarget{exm:GorillaData}{}\label{exm:GorillaData}A back-to-back stemplot for comparing the chest-beating rate of gorillas (Fig.~\ref{fig:GorillasDoubleStem}) has the leaves for younger gorillas right-to-left, and the leaves for older gorillas left-to-right, sharing the same stems. The younger gorillas have a faster chest-beating rate in general. One older gorilla has a much faster rate than the other older gorillas (a potential outlier).
\end{example}

\begin{figure}[hbtp]

{\centering \includegraphics[width=0.8\linewidth]{14-CompareQuant_files/figure-latex/GorillasDoubleStem-1} 

}

\caption{Stemplot for the chest-beating rate for gorillas.}\label{fig:GorillasDoubleStem}
\end{figure}

\subsection{2-D dot charts}\label{TwoDDot}

\index{Graphs!dot chart!comparing quantitative data}\index{Graphs!dot chart!two-dimensional}

A two-dimensional (2-D) dot chart places a dot for each observation, separated for each level of the qualitative variable (also see Sect.~\ref{DotChartsOneQual}). Any number of groups can be compared.

\begin{importantBox}{iconmonstr-warning-8-240.png}
The axis displaying the counts (or percentages) \emph{need not start from zero}, since the distance from the axis to the these numbers \emph{do not} visually imply any quantity of interest. Rather, how the dots \emph{compare} in the groups is the main feature of interest.

\end{importantBox}

\begin{example}[Boxplots]
\protect\hypertarget{exm:Dotchart2DGorillas}{}\label{exm:Dotchart2DGorillas}For the chest-beating data seen in Example~\ref{exm:GorillaData}, a dot chart is shown in Fig.~\ref{fig:TwoDDotchart}. Many observations are the same, so some points would be \emph{overplotted}\index{Overplotting} if points were not \emph{stacked}\index{Overplotting!stacking} (left panel), or \emph{jittered}\index{Overplotting!jittering} (right panel).
\end{example}

\begin{figure}[hbtp]

{\centering \includegraphics[width=1\linewidth]{14-CompareQuant_files/figure-latex/TwoDDotchart-1} 

}

\caption{Two variations of a 2-D dot chart for the chest-beating data to avoid overplotting: stacking (left) and jittering (right).}\label{fig:TwoDDotchart}
\end{figure}

\subsection{Boxplots}\label{Boxplot}

\index{Graphs!boxplot}

A boxplot is a picture of the quantiles (Sect.~\ref{VariationIQR}) for each group, drawn side-by-side on the same plot (and so are sometimes called \emph{parallel} boxplots or \emph{side-by-side} boxplots). Any number of groups can be compared using a boxplot. Sometimes, the mean of each groups is added to the boxplot using, for example, a solid dot.

The distribution for each group is summarised by five numbers: the minimum value; the first quartile (\(Q_1\)); the median (\(Q_2\)); the third quartile (\(Q_3\)); and the maximum value. Outliers, identified using the IQR rule (Sect.~\ref{OutliersIQRrule}), are usually shown too. The values of \(Q_1\), the median, and \(Q_3\) for each group can be used to compare the distributions. Different software may use different rules for computing quartiles, and hence may produce slightly different boxplots.

\begin{importantBox}{iconmonstr-warning-8-240.png}
The axis displaying these five numbers \emph{need not start from zero}, since the distance from the axis to the these numbers \emph{do not} visually imply any quantity of interest. Rather, the boxes display the values of these five numbers for each group \emph{relative} to each other, which is of interest.

\end{importantBox}

\begin{importantBox}{iconmonstr-warning-8-240.png}
Boxplots summarise data with only five numbers (sometimes called the five-number summary)\index{Five-number summary}, so details of the distributions are lost. For this reason, boxplots are excellent for \emph{comparing} distributions, but histograms are better for displaying the distribution of a single quantitative variable.

\end{importantBox}

\begin{example}[Boxplots]
\protect\hypertarget{exm:BoxplotGorillas}{}\label{exm:BoxplotGorillas}The boxplot for the chest-beating data (Example~\ref{exm:GorillaData}) is shown in Fig.~\ref{fig:BoxplotGorillas}. No outliers are identified for younger gorillas; one large outlier is identified for the older gorillas. The boxplot shows a distinct difference between the chest-beating rates of older and younger gorillas.
\end{example}

\begin{figure}[hbtp]

{\centering \includegraphics[width=0.5\linewidth]{14-CompareQuant_files/figure-latex/BoxplotGorillas-1} 

}

\caption{The boxplot for the chest-beating data.}\label{fig:BoxplotGorillas}
\end{figure}

The detail of the boxplots are explained in Fig.~\ref{fig:BoxplotGorillasExplain}. Firstly, focus on just the boxplot for the \emph{younger} gorillas (i.e., the left box).\index{Quartiles} Boxplots have five horizontal lines.

\begin{enumerate}
\def\labelenumi{\arabic{enumi}.}
\tightlist
\item
  \emph{Top line}: the \emph{fastest} chest-beating rate (largest value) is~\(4.4\) per~\(10\,\text{h}\).
\item
  \emph{Second line from top}: \(75\)\% of observations are smaller than about~\(3\) per~\(10\,\text{h}\), represented by the line at the top of the central box. This is the \emph{third quartile} (\(Q_3\)).
\item
  \emph{Middle line}: \(50\)\% of observations are smaller than about~\(1.7\) per~\(10\,\text{h}\), represented by the line inside the central box. This is the median value, the \emph{second quartile} (\(Q_2\)).
\item
  \emph{Second line from bottom}: \(25\)\% of observations are smaller than about~\(1.5\) per~\(10\,\text{h}\), represented by the line at the bottom of the central box. This is the \emph{first quartile} (\(Q_1\)).
\item
  \emph{Bottom line}: the \emph{slowest} chest-beating rate (smallest value) is~\(0.7\) per~\(10\,\text{h}\).
\end{enumerate}

\begin{figure}[hbtp]

{\centering \includegraphics[width=1\linewidth]{14-CompareQuant_files/figure-latex/BoxplotGorillasExplain-1} 

}

\caption{Explaining the boxplots for the chest-beating data.}\label{fig:BoxplotGorillasExplain}
\end{figure}

The box for the \emph{older} gorillas (Fig.~\ref{fig:BoxplotGorillas}, right box) is slightly different: one observation is identified with a point, \emph{above} the top line. Computer software identifies this observation as an \emph{extreme outlier} using the IQR rule (Sect.~\ref{OutliersIQRrule}), and has plotted this point separately.

The values of \(Q_1\), the median and \(Q_3\) are all substantially larger for the younger gorillas, suggesting that younger gorillas have, in general, faster chest-beating rates.

\begin{example}[Boxplots]
\protect\hypertarget{exm:BoxplotsHorizontal}{}\label{exm:BoxplotsHorizontal}Boxplots can be plotted horizontally too, which leaves space for long labels of the qualitative variable. In Fig.~\ref{fig:CementBuildBoxplot} (based on \citet{data:Silva2016:rootcanal}), the three dental cements are very different regarding their push-out forces.
\end{example}

\begin{figure}[hbtp]

{\centering \includegraphics[width=0.85\linewidth]{14-CompareQuant_files/figure-latex/CementBuildBoxplot-1} 

}

\caption{Comparing three push-out values for three dental cements.}\label{fig:CementBuildBoxplot}
\end{figure}

\section{Example: water access}\label{WaterAccessCompareQuant}

\citet{lopez2022farmers} recorded data about access to water in three rural communities in Cameroon (Sect.~\ref{WaterAccessQuant}). The study could be used to determine associations to the incidence of diarrhoea in young children (\(85\)~households had children under~\(5\) years of age).

The graphs (Fig.~\ref{fig:WaterAccessCompareQuantFigs}) and summary (Table~\ref{tab:WaterAccessCompareQuantTabs}) show that households in which diarrhoea was found in the last two weeks in children had older household coordinators, more people in the household, and more children under~\(5\) years of age in the household. These may be expected: older female coordinators probably have more children, hence more children in the household under~\(5\), and so more children (and so people) are in the household in general.

\begin{figure}[hbtp]

{\centering \includegraphics[width=1\linewidth]{14-CompareQuant_files/figure-latex/WaterAccessCompareQuantFigs-1} 

}

\caption{Three plots for the water access data in 85 households ($59$ household reported no diarrhoea in children under\ $5$ years of age; $26$ reported diarrhoea in children under\ $5$ years of age).}\label{fig:WaterAccessCompareQuantFigs}
\end{figure}

\begin{table}
\centering
\caption{\label{tab:WaterAccessCompareQuantTabs}A summary of the quantitative variables in the water-access study, according to whether diarrhoea had been observed in the last two weeks in children under $5$ years of age, for those household with children under $5$ years of age.}
\centering
\fontsize{8}{10}\selectfont
\begin{tabular}[t]{rccccc}
\toprule
\textbf{} & \textbf{$n$} & \textbf{Mean} & \textbf{Median} & \textbf{Std dev.} & \textbf{IQR}\\
\midrule
\addlinespace[0.3em]
\multicolumn{6}{l}{\textbf{Woman coordinator's age (in years)}}\\
\hspace{1em}\em{All households with children} & \em{$85$} & \em{$40.2$} & \em{$37.0$} & \em{$13.90$} & \em{$28.00$}\\
\hspace{1em}Incidents of diarrhoea & $26$ & $45.0$ & $46.5$ & $14.04$ & $28.50$\\
\hspace{1em}No incidents of diarrhoea & $59$ & $38.1$ & $35.0$ & $13.44$ & $22.50$\\
\hspace{1em}Difference &  & $\phantom{0}6.8$ &  &  & \\
\addlinespace[0.3em]
\multicolumn{6}{l}{\textbf{Household size}}\\
\hspace{1em}\em{All households with children} & \em{$85$} & \em{$\phantom{0}8.4$} & \em{$\phantom{0}7.0$} & \em{$\phantom{0}4.93$} & \em{$\phantom{0}6.00$}\\
\hspace{1em}Incidents of diarrhoea & $26$ & $10.5$ & $\phantom{0}8.5$ & $\phantom{0}6.51$ & $\phantom{0}7.75$\\
\hspace{1em}No incidents of diarrhoea & $59$ & $\phantom{0}7.5$ & $\phantom{0}6.0$ & $\phantom{0}3.78$ & $\phantom{0}4.50$\\
\hspace{1em}Difference &  & $\phantom{0}2.9$ &  &  & \\
\addlinespace[0.3em]
\multicolumn{6}{l}{\textbf{Children under 5 in household}}\\
\hspace{1em}\em{All households with children} & \em{$85$} & \em{$\phantom{0}2.2$} & \em{$\phantom{0}2.0$} & \em{$\phantom{0}1.56$} & \em{$\phantom{0}2.00$}\\
\hspace{1em}Incidents of diarrhoea & $26$ & $\phantom{0}2.8$ & $\phantom{0}2.0$ & $\phantom{0}2.01$ & $\phantom{0}1.75$\\
\hspace{1em}No incidents of diarrhoea & $59$ & $\phantom{0}1.9$ & $\phantom{0}2.0$ & $\phantom{0}1.26$ & $\phantom{0}1.00$\\
\hspace{1em}Difference &  & $\phantom{0}0.8$ &  &  & \\
\bottomrule
\end{tabular}
\end{table}

\clearpage

\section{Chapter summary}\label{Chap14-Summary}

Quantitative data can be compared between different groups (between individuals comparisons) using a back-to-back stemplot, boxplot or \(2\)-D dot chart. A summary table should show the numerical summaries for the levels of the quantitative variable, and the between-group differences.

\section{Quick review questions}\label{Chap14-QuickReview}

Are the following statements \emph{true} or \emph{false}?

\begin{enumerate}
\def\labelenumi{\arabic{enumi}.}
\item
  A boxplot is an appropriate graph for comparing a quantitative variable in two \emph{or more} groups. \tightlist
\item
  A back-to-back stemplot is an appropriate graph for comparing a quantitative variable in two \emph{or more} groups.
\item
  A case-profile plot is an appropriate graph for comparing a quantitative variable in two \emph{or more} groups.
\item
  When comparing a quantitative variable in two groups, the difference between the two sample sizes should be included.
\end{enumerate}

\section{Exercises}\label{CompareQuantData-Exercises}

\hyperref[Answers]{Answers to odd-numbered exercises} are given at the end of the book.

\captionsetup{font=small}

\begin{exercise}
\protect\hypertarget{exr:BoxplotsProjectCosts}{}\label{exr:BoxplotsProjectCosts}

\citet{data:Hale2009:ProjectDelivery} studied two different engineering project delivery methods (Fig.~\ref{fig:AISfemalesportEng}, left panel): Design/Build and Design/Bid/Build. The grey, horizontal line is where the projected costs are the same as the actual cost.

\begin{enumerate}
\def\labelenumi{\arabic{enumi}.}
\tightlist
\item
  What does the plot reveal about the two methods?
\item
  What is the median for each method (approximately)?
\item
  What is the IQR for each method (approximately)?
\end{enumerate}

\end{exercise}

\begin{exercise}
\protect\hypertarget{exr:GraphsAIS}{}\label{exr:GraphsAIS}

{[}\emph{Dataset}: \texttt{AISsub}{]} \citet{data:Telford1991:sexsportsize} studied athletes at the \emph{Australian Institute of Sport} (AIS). Numerous physical and blood measurements were taken from high performance athletes. Figure~\ref{fig:AISfemalesportEng} (right panel) compares the heights of females in two similar sports: basketball and netball. (Netball was derived from basketball.)

\begin{enumerate}
\def\labelenumi{\arabic{enumi}.}
\tightlist
\item
  What does the plot reveal about the heights of the females in each sport?
\item
  What is the median for each sport (approximately)?
\item
  What is the IQR for each sport (approximately)?
\end{enumerate}

\end{exercise}

\begin{figure}[hbtp]

{\centering \includegraphics[width=0.9\linewidth]{14-CompareQuant_files/figure-latex/AISfemalesportEng-1} 

}

\caption{Left: cost increases for two different building project delivery methods: Design/Build and Design/Bid/Build (the grey, horizontal line is where the projected costs are the same as the actual cost). Right: the heights of female basketball and netball players attending the AIS.}\label{fig:AISfemalesportEng}
\end{figure}

\begin{exercise}
\protect\hypertarget{exr:NumericalQuantMatchingHistogramsAndBoxplots}{}\label{exr:NumericalQuantMatchingHistogramsAndBoxplots}

Consider the histograms and boxplots in Fig.~\ref{fig:MatchHistoBox}.

\begin{enumerate}
\def\labelenumi{\arabic{enumi}.}
\tightlist
\item
  Match the histogram with the corresponding boxplot.
\item
  For which datasets would the mean and standard deviation be the appropriate numerical summary? For which datasets would the median and IQR be the appropriate numerical summary?
\end{enumerate}

\end{exercise}

\begin{figure}[hbtp]

{\centering \includegraphics[width=1\linewidth]{14-CompareQuant_files/figure-latex/MatchHistoBox-1} 

}

\caption{Match the histogram with the boxplot.}\label{fig:MatchHistoBox}
\end{figure}

\begin{exercise}
\protect\hypertarget{exr:NumericalJellyfish}{}\label{exr:NumericalJellyfish}

\citeauthor{others:lunn:cida} \citetext{\citeyear{others:lunn:cida}; \citealp{data:hand:handbook}} compared the dimensions of jellyfish at two sites at Hawkesbury River, NSW (Dangar Island; Salamander Bay) to determine the difference between the jellyfish at each site. A histogram of the breadth of jellyfish at Dangar Island Bay is shown in Fig.~\ref{fig:JellyfishHist} (left panel).

\begin{figure}[hbtp]

{\centering \includegraphics[width=0.9\linewidth]{14-CompareQuant_files/figure-latex/JellyfishHist-1} 

}

\caption{Left: a histogram of the breadth of jellyfish at Dangar Island. Right: a boxplot of the breadth of jellyfish at two sites.}\label{fig:JellyfishHist}
\end{figure}

\begin{enumerate}
\def\labelenumi{\arabic{enumi}.}
\tightlist
\item
  Two students are arguing about the median breadth.\\
  \strut ~~\\
  Student~1 says: \emph{The bars in the histogram have heights of \(10\),~\(2\),~\(4\),~\(2\) and~\(4\). When these numbers are put in order, they are: \(2\),~\(2\),~\(4\),~\(4\), and~\(10\). The median breadth is the median of these numbers, so the median breadth is the middle one: \(4\,\text{mm}\) is the median.}\\
  \strut ~~\\
  Student~2 responds: \emph{You have the correct answer, but for the wrong reason! There are five bars, and the middle bar is the third bar. Since the third bar has a height of~\(4\), the median breadth is~\(4\,\text{mm}\).}\\
  \strut ~~\\
  Which student, if either, is correct?
\item
  Describe the histogram.
\item
  A boxplot comparing the breadths of jellyfish at Dangar Island and Salamander Bay is shown in Fig.~\ref{fig:JellyfishHist} (right panel). Describe and compare the breadths of the jellyfish.
\item
  What is the median breadth for the jellyfish at each location?
\item
  Which box in the boxplot represents the Dangar Island jellyfish (in Fig.~\ref{fig:JellyfishHist}, left panel)?
\end{enumerate}

\end{exercise}

\begin{exercise}
\protect\hypertarget{exr:NumericalQuantConstructionWorkerProductivity}{}\label{exr:NumericalQuantConstructionWorkerProductivity}

\citet{data:Gatti2013:WorkforceStrain} studied the productivity of construction workers, recording (among other variables) the rate at which concrete panels could be installed by workers. Data for three different female workers in the study are shown in Table~\ref{tab:PanelsTable}, gathered over four installation periods of \(50\,\text{mins}\) each.

\begin{enumerate}
\def\labelenumi{\arabic{enumi}.}
\tightlist
\item
  Compute the IQR for each worker.
\item
  Construct the boxplot for comparing the three workers.
\item
  Draw the approximate histograms for each worker.
\item
  What do you learn about the workers?
\end{enumerate}

\end{exercise}

\begin{table}
\centering
\caption{\label{tab:PanelsTable}The productivity of three workers installing concrete panels (in panels per minute).}
\centering
\fontsize{8}{10}\selectfont
\begin{tabular}[t]{lccc}
\toprule
\textbf{ } & \textbf{Worker 1} & \textbf{Worker 2} & \textbf{Worker 3}\\
\midrule
Mean & $1.24$ & $1.73$ & $1.36$\\
Minimum & $0.59$ & $1.13$ & $0.86$\\
\addlinespace
1st quartile & $0.88$ & $1.51$ & $1.16$\\
Median & $1.35$ & $1.70$ & $1.38$\\
3rd quartile & $1.49$ & $1.91$ & $1.58$\\
\addlinespace
Maximum & $1.88$ & $3.00$ & $2.17$\\
\bottomrule
\end{tabular}
\end{table}

\begin{exercise}
\protect\hypertarget{exr:GreenBuilding}{}\label{exr:GreenBuilding}

In a study of the temperature in offices, \citet{data:Paul2008:Comfort} compared the temperature in three offices (during working hours) at Charles Sturt University (Australia; CSU); the data are summarised in Table~\ref{tab:OfficeTemps}.

\begin{enumerate}
\def\labelenumi{\arabic{enumi}.}
\tightlist
\item
  Compute the IQR for the temperatures in each office.
\item
  Construct the boxplot for comparing the temperatures in the three offices.
\item
  Draw the approximate histograms for each office.
\item
  What do you learn about the offices?
\end{enumerate}

\end{exercise}

\begin{table}
\centering
\caption{\label{tab:OfficeTemps}A summary of the temperature (in degrees C) in three offices at CSU during working hours. Office\ A is on the ground floor; Offices\ B and\ C are on the top floor.}
\centering
\fontsize{8}{10}\selectfont
\begin{tabular}[t]{lccc}
\toprule
\textbf{ } & \textbf{Office A} & \textbf{Office B} & \textbf{Office C}\\
\midrule
Mean & $24.1$ & $25.3$ & $25.7$\\
Minimum & $16.4$ & $15.9$ & $20.1$\\
$Q_1$ & $22.8$ & $23.8$ & $24.6$\\
Median & $24.4$ & $25.5$ & $26.1$\\
$Q_3$ & $25.5$ & $26.9$ & $27.2$\\
Maximum & $27.4$ & $31.0$ & $30.3$\\
\bottomrule
\end{tabular}
\end{table}

\begin{exercise}
\protect\hypertarget{exr:CompareQuantExercisesNHANES}{}\label{exr:CompareQuantExercisesNHANES}

{[}\emph{Dataset}: \texttt{NHANES}{]} Consider this RQ:

\begin{quote}
Among Americans, is the mean direct HDL cholesterol different for current smokers and non-smokers?
\end{quote}

Data to answer this RQ are available from the American \emph{National Health and Nutrition Examination Survey} (\textsc{nhanes}) \citep{data:NHANES:Rpackage}.

\begin{enumerate}
\def\labelenumi{\arabic{enumi}.}
\tightlist
\item
  What would be an appropriate graph to display the comparison?
\item
  Use the software output (Fig.~\ref{fig:NHANESTwoSample})\index{Software output!comparing two means} to construct an appropriate table showing the numerical summary relevant to the RQ.
\end{enumerate}

\end{exercise}



\begin{figure}[hbtp]

{\centering \includegraphics[width=1\linewidth]{jamovi/NHANES/NHANES-DirectHDL-Smoke-Descriptives} 

}

\caption{Software output for the \textsc{nhanes} data.}\label{fig:NHANESTwoSample}
\end{figure}

\begin{exercise}
\protect\hypertarget{exr:FacePlantSummary}{}\label{exr:FacePlantSummary}

{[}\emph{Dataset}: \texttt{ForwardFall}{]} \citet{data:Wojcik:ForwardFall} compared the lean-forward angle in younger and older women. An elaborate set-up was constructed to measure this angle, using a harness. Consider the RQ:

\begin{quote}
Among healthy women, what is difference between the mean lean-forward angle for younger women compared to older women?
\end{quote}

The data are shown in Table~\ref{tab:FacePlant}.

\begin{enumerate}
\def\labelenumi{\arabic{enumi}.}
\tightlist
\item
  What is an appropriate graph to display the comparison?
\item
  Construct an appropriate numerical summary from the software output (Fig.~\ref{fig:FallFowardTTestjamovi}).
\end{enumerate}

\end{exercise}

\begin{table}
\centering
\caption{\label{tab:FacePlant}Lean-forward angles (in degrees) for older women ($n = 10$) and younger women ($n = 5$).}
\centering
\fontsize{8}{10}\selectfont
\begin{tabular}[t]{cccccccccc}
\toprule
\multicolumn{5}{c}{\textbf{Younger women }} & \multicolumn{5}{c}{\textbf{Older women }} \\
\cmidrule(l{3pt}r{3pt}){1-5} \cmidrule(l{3pt}r{3pt}){6-10}
$29$ & $34$ & $33$ & $27$ & $28$ & $18$ & $15$ & $23$ & $13$ & $12$\\
$32$ & $31$ & $34$ & $32$ & $27$ &  &  &  &  & \\
\bottomrule
\end{tabular}
\end{table}

\begin{figure}[hbtp]

{\centering \includegraphics[width=0.95\linewidth]{jamovi/FallForward/FallForward-Descriptives} 

}

\caption{Software output for the lean-forward angles data.}\label{fig:FallFowardTTestjamovi}
\end{figure}

\begin{exercise}
\protect\hypertarget{exr:QuantCompareSpeedSignage}{}\label{exr:QuantCompareSpeedSignage}

{[}\emph{Dataset}: \texttt{Speed}{]} \citet{ma2019impacts} studied adding additional signage to reduce vehicle speeds on freeway exit ramps. At one site (Ningxuan Freeway), speeds were recorded for \(38\)~vehicles before the extra signage was added, and then for \(41\) different vehicles after the extra signage was added (Table~\ref{tab:SignageHT}).

\begin{table} \centering \centering\caption{\label{tab:SignageHT}Vehicle speeds (in km.h$^{-1}$) before and after adding extra signage.}

\fontsize{8}{10}\selectfont
\begin{tabular}[t]{rrrrr}
\toprule
\multicolumn{5}{c}{\textbf{Speeds before signage added}} \\
\cmidrule(l{3pt}r{3pt}){1-5}
$\phantom{0}90.0$ & $108.0$ & $127.1$ & $102.9$ & $\phantom{0}86.4$\\
$\phantom{0}83.1$ & $102.9$ & $\phantom{0}72.0$ & $113.7$ & $\phantom{0}83.1$\\
$\phantom{0}93.9$ & $108.0$ & $\phantom{0}80.0$ & $108.0$ & \\
$113.7$ & $\phantom{0}83.1$ & $\phantom{0}86.4$ & $\phantom{0}93.9$ & \\
$120.0$ & $\phantom{0}72.0$ & $\phantom{0}80.0$ & $\phantom{0}98.2$ & \\
$108.0$ & $\phantom{0}98.2$ & $\phantom{0}90.0$ & $108.0$ & \\
$\phantom{0}98.2$ & $102.9$ & $\phantom{0}98.2$ & $108.0$ & \\
$\phantom{0}90.0$ & $108.0$ & $\phantom{0}90.0$ & $102.9$ & \\
$\phantom{0}90.0$ & $120.0$ & $102.9$ & $102.9$ & \\
\bottomrule
\end{tabular} \enskip 
\begin{tabular}[t]{rrrrr}
\toprule
\multicolumn{5}{c}{\textbf{Speeds after signage added}} \\
\cmidrule(l{3pt}r{3pt}){1-5}
$\phantom{0}98.2$ & $\phantom{0}98.2$ & $\phantom{0}93.9$ & $102.9$ & $\phantom{0}69.7$\\
$102.9$ & $\phantom{0}93.9$ & $\phantom{0}93.9$ & $\phantom{0}90.0$ & $113.7$\\
$\phantom{0}93.9$ & $\phantom{0}98.2$ & $120.0$ & $\phantom{0}98.2$ & $102.9$\\
$\phantom{0}98.2$ & $\phantom{0}67.5$ & $\phantom{0}93.9$ & $108.0$ & $\phantom{0}72.0$\\
$\phantom{0}80.0$ & $\phantom{0}98.2$ & $\phantom{0}77.1$ & $\phantom{0}86.4$ & $\phantom{0}80.0$\\
$\phantom{0}86.4$ & $\phantom{0}93.9$ & $\phantom{0}86.4$ & $113.7$ & \\
$102.9$ & $\phantom{0}98.2$ & $\phantom{0}98.2$ & $\phantom{0}72.0$ & \\
$\phantom{0}83.1$ & $\phantom{0}63.5$ & $108.0$ & $\phantom{0}98.2$ & \\
$\phantom{0}98.2$ & $\phantom{0}83.1$ & $\phantom{0}74.5$ & $\phantom{0}93.9$ & \\
\bottomrule
\end{tabular}
\end{table}

The researchers are hoping that the addition of extra signage will \emph{reduce} the mean speed of the vehicles. The RQ is:

\begin{quote}
At this freeway exit, how much is the mean vehicle speed \emph{reduced} after extra signage is added?
\end{quote}

\begin{enumerate}
\def\labelenumi{\arabic{enumi}.}
\tightlist
\item
  Using the software output in Fig.~\ref{fig:SpeedjamoviCI}, summarise the data numerically, then construct a suitable summary table.
\item
  Produce a boxplot of the data (use a computer if necessary).
\end{enumerate}

\end{exercise}

\begin{figure}[hbtp]

{\centering \includegraphics[width=1\linewidth]{jamovi/Speed/Speed-Descriptives} 

}

\caption{Software output for the speed data.}\label{fig:SpeedjamoviCI}
\end{figure}

\begin{exercise}
\protect\hypertarget{exr:CompareQuantDeceleration}{}\label{exr:CompareQuantDeceleration}

{[}\emph{Dataset}: \texttt{Deceleration}{]} \citet{ma2019impacts} studied adding additional signage to reduce vehicle speeds on freeway exit ramps. At one site (Ningxuan Freeway), speeds were recorded at various points on the freeway exit for \(38\)~vehicles before the extra signage was added, and then for \(41\)~vehicles after the extra signage was added.

From this data, the \emph{deceleration} of each vehicle was determined (Table~\ref{tab:SignageSummaryData}) as the vehicle left the~\(120\,\text{km}\).h\textsuperscript{\(-1\)} speed zone and approached the~\(80\,\text{km}\).h\textsuperscript{\(-1\)} speed zone. The RQ is:

\begin{quote}
At this freeway exit, what is the difference between the mean vehicle deceleration, comparing the times before the extra signage is added and after extra signage is added?
\end{quote}

In this context, the researchers are hoping that the extra signage might cause cars to slow down \emph{faster} (i.e., they will decelerate more, on average, after adding the extra signage).

\begin{enumerate}
\def\labelenumi{\arabic{enumi}.}
\tightlist
\item
  Using the software output in Fig.~\ref{fig:DecelerationjamoviCI}, summarise the data numerically, then construct a suitable summary table.
\item
  Produce a boxplot of the data (use a computer if necessary).
\item
  What does a \emph{negative} deceleration value represent?
\end{enumerate}

\end{exercise}

\begin{table} \centering \centering\caption{\label{tab:SignageSummaryData}Vehicle deceleration  (in m.s$^{-2}$) before and after adding extra signage.}

\fontsize{8}{10}\selectfont
\begin{tabular}[t]{rrrrr}
\toprule
\multicolumn{5}{c}{\textbf{Deceleration before signage added}} \\
\cmidrule(l{3pt}r{3pt}){1-5}
$0.108$ & $\phantom{0}\llap{$-{}$}0.062$ & $0.023$ & $0.043$ & $0.044$\\
$0.064$ & $\phantom{0}0.063$ & $0.028$ & $0.096$ & $0.048$\\
$0.002$ & $\phantom{0}0.107$ & $0.080$ & $0.151$ & \\
$0.029$ & $\phantom{0}0.081$ & $0.061$ & $0.114$ & \\
$0.167$ & $\phantom{0}0.102$ & $0.031$ & $0.154$ & \\
$0.042$ & $\phantom{0}0.054$ & $0.071$ & $0.107$ & \\
$0.113$ & $\phantom{0}0.003$ & $0.113$ & $0.173$ & \\
$0.053$ & $\phantom{0}0.042$ & $0.126$ & $0.084$ & \\
$0.035$ & $\phantom{0}0.070$ & $0.084$ & $0.126$ & \\
\bottomrule
\end{tabular} \enskip 
\begin{tabular}[t]{rrrrr}
\toprule
\multicolumn{5}{c}{\textbf{Deceleration after signage added}} \\
\cmidrule(l{3pt}r{3pt}){1-5}
$0.134$ & $\phantom{0}0.093$ & $0.057$ & $0.147$ & $0.046$\\
$\llap{$-{}$}0.113$ & $0.095$ & $\phantom{0}0.076$ & $0.089$ & $0.007$\\
$\llap{$-{}$}0.052$ & $0.054$ & $0.167$ & $\phantom{0}0.074$ & $0.063$\\
$0.093$ & $0.118$ & $0.076$ & $0.085$ & $\phantom{0}0.087$\\
$0.080$ & $0.154$ & $0.107$ & $0.079$ & $0.096$\\
$\phantom{0}0.044$ & $0.076$ & $0.044$ & $0.119$ & \\
$0.043$ & $\phantom{0}0.113$ & $0.175$ & $0.014$ & \\
$0.064$ & $0.083$ & $\phantom{0}0.085$ & $0.093$ & \\
$0.074$ & $0.064$ & $0.037$ & $\phantom{0}0.095$ & \\
\bottomrule
\end{tabular}
\end{table}

\begin{figure}[hbtp]

{\centering \includegraphics[width=1\linewidth]{jamovi/Deceleration/Deceleration-Descriptives} 

}

\caption{Software output for the deceleration data.}\label{fig:DecelerationjamoviCI}
\end{figure}

\begin{exercise}
\protect\hypertarget{exr:QuantCompareTyping}{}\label{exr:QuantCompareTyping}

{[}\emph{Dataset}: \texttt{Typing}{]} The \texttt{Typing} dataset contains information about the typing speed and accuracy for students, from an online typing test \citep{pinet2022typing}. The four variables are: typing speed (\texttt{mTS}), typing accuracy (\texttt{mAcc}), age (\texttt{Age}), and sex (\texttt{Sex}) for \(1\,301\) students.

\begin{enumerate}
\def\labelenumi{\arabic{enumi}.}
\tightlist
\item
  Produce appropriate numerical summaries for the quantitative variables.
\item
  Produce appropriate numerical summaries for \emph{comparing} the quantitative variables for different values of the qualitative variable.
\item
  What do you learn from these numerical summaries?
\end{enumerate}

\end{exercise}

\begin{exercise}
\protect\hypertarget{exr:QuantCompareDental}{}\label{exr:QuantCompareDental}

{[}\emph{Dataset}: \texttt{Dental}{]} \citet{data:woodward:dental} recorded the sugar consumption and the average number of decayed, missing or filled teeth (DMFT) in \(29\)~industrialised countries and \(61\)~non-industrialised countries.

\begin{enumerate}
\def\labelenumi{\arabic{enumi}.}
\tightlist
\item
  Produce appropriate numerical summaries for the two quantitative variables.
\item
  Produce appropriate numerical summaries for \emph{comparing} the two quantitative variables for industrialised countries and non-industrialised countries.
\item
  What do you learn from these numerical summaries?
\end{enumerate}

\end{exercise}

\begin{exercise}
\protect\hypertarget{exr:QuantCompareSnakesConfounding}{}\label{exr:QuantCompareSnakesConfounding}

{[}\emph{Dataset}: \texttt{Snakes}{]} Some Mexican garter snakes (\emph{Thamnophis melanogaster}) live in habitats with no crayfish, while some live in habitats with crayfish and use crayfish as a food source. \citet{manjarrez2017morphological} were interested in whether the snakes in these regions were different:

\begin{quote}
For female Mexican garter snakes, is the mean snout--vent length (SVL) different for those in regions with crayfish and without crayfish?
\end{quote}

Two different groups of snakes are studied (so the study uses a between-individuals comparison\index{Comparison!between individuals}). (The data are shown in Table~\ref{tab:SnakesDataTableTest}.) Boxplots of the data are shown in Fig.~\ref{fig:BoxplotCrayfish}.

\begin{enumerate}
\def\labelenumi{\arabic{enumi}.}
\tightlist
\item
  Describe the boxplot displaying the SVL for the two regions, for \emph{all} crayfish (left panel). Compare the means for the two regions.
\item
  Describe the boxplot displaying the SVL for the two regions, for \emph{female} crayfish (centre panel). Compare the means for the two regions.
\item
  Describe the boxplot displaying the SVL for the two regions, for \emph{male} crayfish (right panel). Compare the means for the two regions.
\item
  How would you describe the variable `Sex of the snake': extraneous, confounding, lurking, response or explanatory?
\end{enumerate}

\end{exercise}

\begin{figure}[hbtp]
\includegraphics[width=1\linewidth]{14-CompareQuant_files/figure-latex/BoxplotCrayfish-1} \caption{The snount--vent length (SVL) for Mexican garter snakes, living in crayfish or non-crayfish regions. The solid dots represent the means.}\label{fig:BoxplotCrayfish}
\end{figure}

\captionsetup{font=normalsize}

\begin{EOCanswerBox}{iconmonstr-check-mark-14-240.png}
\textbf{Answers to \emph{Quick review} questions:} \textbf{1.} True. \textbf{2.} False; only compares two groups. \textbf{3.} False; a case-profile plot is appropriate for \emph{within}-individual changes. \textbf{4.} False; difference between sample sizes is meaningless.

\end{EOCanswerBox}

\chapter{Comparing qualitative data between individuals}\label{CompareQualData}

\index{Qualitative data!comparing \textit{between} individuals}

\begin{cols}
\begin{col}{0.52\textwidth}

\begin{objectivesBox}{iconmonstr-target-4-240.png}
So far, you have learnt to ask an RQ, design a study, collect the data, describe the data and summarise the data.
\textbf{In this chapter}, you will learn to:

\begin{itemize}\tightlist
  \item
  compare qualitative data between groups of individuals using the appropriate graphs.
  \item
  compare qualitative data between groups of individuals using the difference in proportions, odds ratios and summary tables.
\end{itemize}
\end{objectivesBox}

\end{col}

\begin{col}{0.03\textwidth}
~
\end{col}

\begin{col}{0.45\textwidth}

\includegraphics[width=0.95\linewidth]{15-CompareQual_files/figure-latex/unnamed-chunk-10-1} 
\end{col}
\end{cols}

\section{Introduction}\label{CompareQual-Intro}

Relational RQs compare groups. This chapter considers how to compare \emph{qualitative} variables in different groups. Graphs are useful for this purpose, and a table including odds, odds ratios and proportions is usually produced also.

\section{Two-way tables}\label{QualitativeTwoWaytables}

\index{Two-way tables}

When more than one qualitative variable is recorded for each individual, the data can be collated into a table. When \emph{two} qualitative variables are cross-tabulated, the resulting table is called a \emph{two-way table}.\index{Two-way tables} The categories for each variable should be \emph{exhaustive}\index{Exhaustive} (cover all levels) and \emph{mutually exclusive}\index{Mutually exclusive} (observations belong to one and only one level). Usually, the levels of the explanatory variable are in the rows of the table.

\begin{example}[Two-way tables]
\protect\hypertarget{exm:SmallKidneyStones}{}\label{exm:SmallKidneyStones}To compare two treatments for kidney stones, \citet{data:Charig:stones} collected data from \(700\)~UK patients on two qualitative variables:

\begin{itemize}
\tightlist
\item
  the treatment method (`A' or~`B'), the explanatory variable.
\item
  the result of the procedure (`success' or~`failure'), the response variable.
\end{itemize}

Both variables are \emph{qualitative} with two \emph{levels}, and each treatment was used on \(350\) patients. Treatment~A was used from 1972--1980, and Treatment~B from 1980--1985; that is, treatments were \emph{not randomly allocated}, and so \emph{confounding} may be present. For this reason, the researchers also recorded the \emph{size} of the kidney stone (`small' or `large') as one possible confounding variable. Firstly, consider just the \emph{small stones} \citep{julious1994confounding}, displayed in the two-way table in Table~\ref{tab:KS-Small}.
\end{example}



\begin{table}
\centering\centering
\caption{\label{tab:KS-Small}\emph{Counts} for two procedures with \emph{small} kidney stones.}
\centering
\fontsize{8}{10}\selectfont
\begin{tabular}[t]{>{}lcc>{}c}
\toprule
\textbf{ } & \textbf{Success} & \textbf{Failure} & \textbf{Total}\\
\midrule
\textbf{Method A} & $\phantom{0}81$ & $\phantom{0}6$ & \textbf{$\phantom{0}87$}\\
\textbf{Method B} & $234$ & $36$ & \textbf{$270$}\\
\midrule
\textbf{\textbf{Total}} & \textbf{$315$} & \textbf{$42$} & \textbf{\textbf{$357$}}\\
\bottomrule
\end{tabular}
\end{table}

\section{Summary tables by rows and columns}\label{RowPercentages}

\index{Two-way tables!summary by rows}\index{Two-way tables!summary by columns}

Each variable in a two-way table can be analysed separately, using percentages or proportions (Sect.~\ref{QualitativeProportionsPercentages}) or odds (Sect.~\ref{QualOdds}). For example, the two variables in Table~\ref{tab:KS-Small} (Method; Result) can be analysed separately. For overall results:

\begin{itemize}
\tightlist
\item
  the proportion of procedures that were successful is \(315/357 = 0.882\) (or \(88.2\)\%).
\item
  the odds that a procedure was successful is \(315/42 = 7.5\); that is, there were~\(7.5\) times as many successful procedures as unsuccessful procedures.
\end{itemize}

However, to \emph{compare} Methods~A and~B, the proportions (or percentages) and odds of successful results need to be computed for each row separately.

\begin{example}[Small kidney stones]
\protect\hypertarget{exm:SummaryTableCompareQual}{}\label{exm:SummaryTableCompareQual}The data in Table~\ref{tab:KS-Small} can be summarised by computing proportions or percentages by \emph{row}. Each row refers to a different method, so row percentages will compute success percentages for the two methods.

For the small kidney stones (Table~\ref{tab:KS-Small}), the \emph{row percentages} (Table~\ref{tab:KidneyRowColLATEX}, left table) give the percentage of successes for each \emph{Method}, since the rows represent the counts for Methods~A and~B.\index{Proportions} \emph{Row} proportions (or percentages) allow the proportions (or percentages) \emph{within the rows} (i.e., for each Method) to be compared:

\begin{itemize}
\tightlist
\item
  with Method~A, \(81 \div 87 = 0.931\) (or~\(93.1\)\%) of operations in the sample were successful.
\item
  with Method~B, \(234\div 270 = 0.867\) (or~\(86.7\)\%) of operations in the sample were successful.
\end{itemize}

For small kidney stones, Method~A is slightly more successful~(\(93.1\)\%) than Method~B~(\(86.7\)\%) in the \emph{sample}. These percentages are collated in\index{Percentages} Table~\ref{tab:KidneyRowColLATEX} (left table).

Odds can also be computed:\index{Odds}

\begin{itemize}
\tightlist
\item
  with Method~A, the odds of success is \(81\div6 = 13.5\); there are \(13.5\) times as many successful procedures than failures for Method~A.
\item
  with Method~B, the odds of success is \(234\div36 = 6.5\); there are \(6.5\) times as many successful procedures than failures for Method~B.
\end{itemize}

The odds of a success is far greater for Method~A than Method~B in the sample.
\end{example}







\begin{table} \centering \centering\caption{\label{tab:KidneyRowColLATEX}Two procedures with \emph{small} kidney stones. Left: \emph{row} percentages. Right: \emph{column} percentages (from Table~\ref{tab:KS-Small}). Proportions could be used rather than percentages.}

\fontsize{8}{10}\selectfont
\begin{tabular}[t]{>{}lccc}
\toprule
\textbf{\textbf{ }} & \textbf{Success} & \textbf{Failure} & \textbf{Total}\\
\midrule
\textbf{Method A} & $93.1$ & $\phantom{0}6.9$ & $100.0$\\
\textbf{Method B} & $86.7$ & $13.3$ & $100.0$\\
\bottomrule
\end{tabular} \qquad 
\begin{tabular}[t]{lrr}
\toprule
\textbf{ } & \textbf{Success} & \textbf{Failure}\\
\midrule
\textbf{Method A} & $25.7$ & $14.3$\\
\textbf{Method B} & $74.3$ & $85.7$\\
\midrule
\textbf{Total} & $100.0$ & $100.0$\\
\bottomrule
\end{tabular}
\end{table}

Rather than comparing \emph{methods} (in the rows), the procedure \emph{results} can be compared (i.e., the columns).

\begin{example}[Comparing by column]
\protect\hypertarget{exm:KidneyStonesSmallColums}{}\label{exm:KidneyStonesSmallColums}For the small kidney stones (Table~\ref{tab:KS-Small}), the \emph{column percentages} (Table~\ref{tab:KidneyRowColLATEX}, right table) give the percentage of successes within each column (i.e., for successes and for failures), since the columns contain the procedure results. \emph{Column} percentages (or proportions) allow the percentages (or proportions) within \emph{columns} to be compared:

\begin{itemize}
\tightlist
\item
  the proportion of the \emph{successful} procedures from Method~A is \(81 \div 315 = 0.257\) (or~\(25.7\)\%).
\item
  the proportion of the \emph{failed} procedures from Method~A is \(234\div 315 = 0.143\) (or~\(14.3\)\%).
\end{itemize}

Odds can also be computed:

\begin{itemize}
\tightlist
\item
  the odds of a \emph{success} coming from Method~A is \(81/234 = 0.346\); there are \(0.346\)~times as many Method~A procedures than Method~B procedures among the successes.
\item
  the odds of \emph{failure} coming from Method~A is \(6/36 = 0.167\); there are \(0.167\)~times as many Method~A procedures than Method~B procedures among the failures.
\end{itemize}

The odds of a success being a Method~A procedure is quite different from the odds of a success being a Method~B procedure.

Comparing rows (i.e., using row percentages and row odds) seems more intuitive than column proportions here: they compare the success percentages and odds for each method.
\end{example}

\section{Graphs for the comparison}\label{QualitativeCompareGraphs}

\index{Qualitative data!comparing \textit{between} individuals!graphs}\index{Software output!graphs}

When a \emph{qualitative} variable is compared across different groups (i.e., comparing between individuals), options for plotting include:

\begin{itemize}
\tightlist
\item
  \emph{stacked bar charts} (Sect.~\ref{StackedBarCharts}).
\item
  \emph{side-by-side bar charts} (Sect.~\ref{SideBySideBarCharts}).
\item
  \emph{dot charts} (Sect.~\ref{TwoWayCountsDotCharts}).
\end{itemize}

\subsection{Stacked bar charts}\label{StackedBarCharts}

\index{Graphs!stacked bar chart}

The data can be graphed by using a bar for each level of one variable, and \emph{stacking} the bars for the levels of the second variable. Bars indicate the counts (or percentages) in each category. The levels can be on the horizontal or vertical axis, but placing the level names on the vertical axis often makes for easier reading, and room for long labels.

\begin{importantBox}{iconmonstr-warning-8-240.png}
The axis displaying the counts (or percentages) should \emph{start from zero}, since the height of the bars visually implies the frequency of those observations (see Example~\ref{exm:VerticalTruncation}).

\end{importantBox}

\begin{example}[Stacked bar charts]
\protect\hypertarget{exm:BarStacked}{}\label{exm:BarStacked}For the small kidney-stone data in Example~\ref{exm:SmallKidneyStones}, a stacked bar chart can be created by producing a bar for each method, and \emph{stacking} the successes and failures for each method (Fig.~\ref{fig:QualGraphsStones}, top left panel).

Rather than using \emph{numbers}, the \emph{percentages} separately within each group can be used too (Fig.~\ref{fig:QualGraphsStones}, bottom left panel). This makes comparing the \emph{relative} proportions easier.
\end{example}

\begin{figure}[hbtp]

{\centering \includegraphics[width=1\linewidth]{15-CompareQual_files/figure-latex/QualGraphsStones-1} 

}

\caption{Six plots for the small kidney-stone data. Top plots: displaying the numbers for each method. Bottom plots: displaying the percentages for each method. Left: stacked bar chart. Centre: side-by-side bar charts. Right: dot charts.}\label{fig:QualGraphsStones}
\end{figure}

\subsection{Side-by-side bar charts}\label{SideBySideBarCharts}

\index{Graphs!side-by-side bar chart}

Instead of stacking the success and failures bars on top of each other, these bars can be placed \emph{side-by-side} for each method. Bars indicate the counts (or percentages) in each category. The levels can be on the horizontal or vertical axis, but placing the level names on the vertical axis often makes for easier reading, and room for long labels.

\begin{importantBox}{iconmonstr-warning-8-240.png}
The axis displaying the counts (or percentages) should \emph{start from zero}, since the height of the bars visually implies the frequency of those observations (see Example~\ref{exm:VerticalTruncation}).

\end{importantBox}

\begin{example}[Side-by-side bar charts]
\protect\hypertarget{exm:BarSideBySide}{}\label{exm:BarSideBySide}For the small kidney-stone data in Example~\ref{exm:SmallKidneyStones}, a side-by-side bar chart can be created by producing two bars for each method (one for failures; one for successes), and placing these side-by-side (Fig.~\ref{fig:QualGraphsStones}, centre panels). Again, numbers or percentages within each method can be graphed.
\end{example}

\subsection{Dot charts}\label{TwoWayCountsDotCharts}

\index{Graphs!dot chart!comparing qualitative data}

Instead of bars, dots (or other symbols) can be used in place of the bars in a side-by-side bar chart to create a dot chart.

\clearpage

\begin{importantBox}{iconmonstr-warning-8-240.png}
The axis displaying the counts (or percentages) should \emph{start from zero}, since the distance of the dots from the axis visually implies the frequency of those observations (see Example~\ref{exm:VerticalTruncation}).

\end{importantBox}

\begin{example}[Dot charts]
\protect\hypertarget{exm:BarSideBySide2}{}\label{exm:BarSideBySide2}For the data in Example~\ref{exm:SmallKidneyStones}, a dot chart can be created by placing plotting symbols for each result (one for failures; one for successes) side-by-side for each method (Fig.~\ref{fig:QualGraphsStones}, right panels). Again, numbers or percentages can be used.
\end{example}

\subsection{Other variations}\label{OtherVariations}

Many variations of these charts are possible, by making different choices:

\begin{itemize}
\tightlist
\item
  using a stacked bar chart, side-by-side bar chart, or dot chart.
\item
  using percentages or counts on one axis. (The percentages can be percentages of the total, or within the total for each level of the variable, as in the bottom plots in Fig.~\ref{fig:QualGraphsStones}.)
\item
  using the counts (or percentage) on either the horizontal or vertical axis.
\item
  deciding which variable can be used as the first division of the data.
\end{itemize}

The guiding principle remains: \emph{the purpose of a graph is to display the information in the clearest, simplest possible way, to facilitate understanding the message(s) in the data}.

Using a computer to create graphs is recommended, and using a computer makes it easy to try different variations to find the graph that best displays the message in the data.

\section{Numerical summary: difference between proportions}\label{DiffProportions}

\index{Difference between proportions}\index{Software output!comparing two proportions}\index{Summary table!comparing two proportions}\index{Summary table!comparing two odds}

The difference between the success-rates of the two methods for the small kidney-stone data (Table~\ref{tab:KS-Small}) can be summarised using the difference between the respective proportions:

\begin{itemize}
\tightlist
\item
  for \emph{Method~A}, the \emph{sample} proportion of successful procedures is \(\hat{p}_A = 0.931\).
\item
  for \emph{Method~B}, the \emph{sample} proportion of successful procedures is \(\hat{p}_B = 0.867\).
\end{itemize}

The \emph{difference} between these proportions is~\(\hat{p}_A - \hat{p}_B = 0.064\) (i.e., the success rate is higher for Method~A). The difference between the proportions is a \emph{statistic}, and the (unknown) difference between the population proportions (i.e., \(p_A - p_B\)) is a \emph{parameter}.

\section{Numerical summary: odds ratios}\label{OddsRatios}

\index{Odds ratio}

The small kidney-stone data (Table~\ref{tab:KS-Small}) also can be summarised using the odds of success for each method:

\begin{itemize}
\tightlist
\item
  for \emph{Method~A}, the odds of success are~\(13.5\) (\(13.5\)~\emph{times} as many successes as failures).
\item
  for \emph{Method~B}, the odds of success are~\(6.5\) (\(6.5\)~\emph{times} as many successes as failures).
\end{itemize}

The odds of success for Method~A and Method~B are very different. In the sample, the odds of success for Method~A is many \emph{times} greater than for Method~B.\spacex  In fact, in the sample, the odds of success for Method~A is \(13.5\div 6.5 = 2.08\) \emph{times} the odds of a success for Method~B.\spacex This value is the \emph{odds ratio} (OR). The sample OR is a \emph{statistic}, and the (unknown) population OR is a \emph{parameter}. There is no commonly-used symbol for odds ratios.

\begin{definition}[Odds Ratio (OR)]
\protect\hypertarget{def:OddsRatio}{}\label{def:OddsRatio}The \emph{odds ratio} (often written OR) is the ratio of the odds of a result of interest in one group, compared to the odds of the \emph{same} result in a \emph{different} group: \[
\text{Odds ratio (OR)} = 
\frac{\text{Odds of a result in Group A}}
{\text{Odds of the same result in Group B}}.
\]
\end{definition}

\begin{example}[Odds ratios]
\protect\hypertarget{exm:InterpretingOdds}{}\label{exm:InterpretingOdds}For the small kidney-stone data, the odds of a success for Method~A is \(81\div6 = 13.5\). The odds of a success for Method~B is \(234\div 36 = 6.5\). The OR is then computed as \(13.5\div 6.5 = 2.08\). The odds have been computed \emph{with the rows}.

This means that the odds of a success for Method~A is about~\(2.08\) times the odds of a success for Method~B.
\end{example}

Most software computes the OR from a two-way table by using the values in the \emph{first} row and \emph{first} column on the \emph{top} of the fractions when computing the odds and the odds ratio. In Example~\ref{exm:InterpretingOdds}, for instance, the odds for both methods were computed with the Column~1 values on the top of the fraction (\(81\) and~\(234\)), and the OR comparing the \emph{rows} was computed with the Row~1 odds (\(13.5\)) on top of the fraction.

However, the OR could also be computed using the odds within the columns (i.e., comparing the \emph{columns}), rather than within the rows.

\begin{softwareBox}{iconmonstr-laptop-4-240.png}
The OR can be interpreted in \emph{either} of these ways (i.e., both are correct):\index{Odds ratio!interpreting}\index{Software output!comparing two odds (odds ratio)}

\begin{itemize}
\tightlist
\item
  the \emph{odds} in each column compares Row~1 counts (top) to Row~2 counts (bottom). The \emph{OR} then compares the Column~1 odds (top) to the Column~2 odds (bottom).
\item
  the \emph{odds} in each row compares Column~1 counts to Column~2 counts. The \emph{OR} then compares the Row~1 odds to the Row~2 odds.
\end{itemize}

Odds and ORs are computed with the \emph{first row} and \emph{first column} values on the \emph{top} of the fraction. While both are correct, the levels of the explanatory variable are usually the rows of the table (as in Table~\ref{tab:KS-Small}), so usually the \emph{second} interpretation makes more sense (as in Example~\ref{exm:InterpretingOdds}).

\end{softwareBox}

The OR compares the odds of the same result (e.g., success) in two groups (e.g., Method~A and Method~B). This means a \(2\times 2\) table can be summarised with one number: the OR.

When interpreting ORs:

\begin{itemize}
\tightlist
\item
  ORs \emph{greater than}~\(1\) mean the odds of the result is \emph{larger} for the group on top of the fraction compared to the group on the bottom.
\item
  ORs \emph{equal to}~\(1\) mean the odds of the result is the \emph{same} for both groups (on the top and the bottom of the fraction).
\item
  ORs \emph{less than}~\(1\) mean the odds of the result is \emph{smaller} for the group on the top of the fraction compared to the group on the bottom.
\end{itemize}

The numerical summary information for comparing qualitative variables can be collated in a table.\index{Qualitative data!compare \textit{between} individuals!summary tables} The data should be summarised by one of the qualitative variables, producing proportions (or percentages) and odds for the other. The summary table also requires the differences between the proportions (or percentages) and the odds ratio.

\begin{example}[Numerical summary table]
\protect\hypertarget{exm:GorillaComparisonTable}{}\label{exm:GorillaComparisonTable}For the small kidney-stone data, the summary of the data can be tabulated as in Table~\ref{tab:KidneySmallSum}, using percentages and odds.
\end{example}

\begin{table}
\centering
\caption{\label{tab:KidneySmallSum}Numerical summary of the small kidney-stone data: odds and percentage of a successful procedure.}
\centering
\fontsize{8}{10}\selectfont
\begin{tabular}[t]{>{}lccc}
\toprule
\textbf{ } & \textbf{Percentage success} & \textbf{Odds of success} & \textbf{Sample size}\\
\midrule
\textbf{Method A} & $93.1$ & $13.50$ & $\phantom{0}87$\\
\textbf{Method B} & $86.7$ & $\phantom{0}6.50$ & $270$\\
\midrule
\em{\textbf{}} & \em{\llap{Difference:\ \ }$\phantom{0}6.4$} & \em{\llap{OR:\ }$\phantom{0}2.08$} & \em{}\\
\bottomrule
\end{tabular}
\end{table}

\section{Example: large kidney stones}\label{KidneyExample}

The data in Table~\ref{tab:KS-Small} are for procedures on \emph{small} kidney stones. Data were also recorded for the \emph{large} kidney stones (Table~\ref{tab:KStonesNumbersLargeAll}, left table). As for small kidney stones, the \emph{success proportions} can be computed for both methods:

\begin{itemize}
\tightlist
\item
  for \emph{Method~A}, the success proportion for \emph{large} kidney stones: \(192/263 = 0.730\).
\item
  for \emph{Method~B}, the success proportion for \emph{large} kidney stones: \(55/80 = 0.688\).
\end{itemize}

For large kidney stones, then, \emph{Method~A} has a higher success proportion than Method~B, just as with the small kidney stones.



So, could the data for small (Table~\ref{tab:KS-Small}) and large kidney stones (Table~\ref{tab:KStonesNumbersLargeAll}, left table) be combined, to produce a single two-way table of just Method and Result (Table~\ref{tab:KStonesNumbersLargeAll}, right table)? From this table of small and large stones combined:

\begin{itemize}
\tightlist
\item
  for \emph{Method~A}, the success proportion for \emph{all} kidney stones: \(273/350 = 0.780\).
\item
  for \emph{Method~B}, the success proportion for \emph{all} kidney stones: \(289/350 = 0.826\).
\end{itemize}





\begin{table} \centering \centering\caption{\label{tab:KStonesNumbersLargeAll}The kidney stones data. Left: numbers for \emph{large} stones only. Right: numbers for \emph{all} kidney stones combined, without separating by the size of the kidney stone.}

\fontsize{8}{10}\selectfont
\begin{tabular}{lccc}
\toprule
\multicolumn{4}{c}{Large stones only} \\
\cmidrule(l{3pt}r{3pt}){1-4}
\textbf{ } & \textbf{Success} & \textbf{Failure} & \textbf{Total}\\
\midrule
\textbf{Method A} & $192$ & $71$ & $263$\\
\textbf{Method B} & $\phantom{0}55$ & $25$ & $\phantom{0}80$\\
\bottomrule
\end{tabular} \qquad\qquad 
\begin{tabular}{>{}lccc}
\toprule
\multicolumn{4}{c}{Large and small stones combined} \\
\cmidrule(l{3pt}r{3pt}){1-4}
\textbf{ } & \textbf{Success} & \textbf{Failure} & \textbf{Total}\\
\midrule
\textbf{Method A} & $273$ & $77$ & $350$\\
\textbf{Method B} & $289$ & $61$ & $350$\\
\bottomrule
\end{tabular}
\end{table}

When all kidney stones are combined, \emph{Method~A} has a \emph{lower} success proportion than Method~B. To summarise:

\begin{itemize}
\tightlist
\item
  \emph{Method~A} is more successful for \emph{small} stones (\(0.931\) vs~\(0.867\)).
\item
  \emph{Method~A} is more successful for \emph{large} stones (\(0.730\) vs~\(0.688\)).
\item
  \emph{Method~B} is more successful for \emph{all} stones combined (\(0.780\) vs~\(0.826\)).
\end{itemize}

That seems strange: Method~A performs better for small \emph{and} for large kidney stones, but Method~B performs better when combining all kidney stones. The explanation is that the \emph{size of the stone} is a \emph{confounding variable}\index{Variables!confounding} (Fig.~\ref{fig:SimpsonRulesStones}). Size is associated with both the method (small stones are treated more often with Method~B) \emph{and} with the result (small stones have a higher success proportion for \emph{both} methods). Method~B was used more often on smaller kidney stones, for which a success is more likely (due to their smaller size).

This confounding could have been avoided by randomly allocating a treatment method to patients. However, random allocation was not possible in this observational study, so the researchers used a different method to manage confounding: \emph{recording} the size of the kidney stones to use in the analysis (see Sect.~ \ref{ManagingConfounding}).\index{Confounding}\index{Variables!lurking}\index{Variables!confounding}\index{Confounding!analysis}

In this example, incorporating information about a potential confounder (the size of the kidney stone) is important, otherwise the wrong (opposite) conclusion is reached: Method~B would be incorrectly considered better if the size of the stones was ignored, when the better method really is Method~A.

This is called\index{Simpson's paradox} \emph{Simpson's paradox}.\index{Simpson's paradox} If the size of the kidney stone had not been recorded, size would be a \emph{lurking variable}, and the incorrect conclusion would have been reached.

\begin{figure}[hbtp]

{\centering \includegraphics[width=0.8\linewidth]{15-CompareQual_files/figure-latex/SimpsonRulesStones-1} 

}

\caption{The size of the stones is associated with the success percentage and method.}\label{fig:SimpsonRulesStones}
\end{figure}

\section{Example: water access}\label{WaterAcessQualCompare}

\citet{lopez2022farmers} recorded data about access to water for three rural communities in Cameroon (see Sects.~\ref{WaterAccessQuant} and~\ref{WaterAccessQual}). The study could be used to determine associations to the incidence of diarrhoea in young children (\(85\)~households had children under~\(5\)). A cross-tabulation (Table~\ref{tab:WaterAcessQualCrosstab}) shows the relationship with keeping livestock; the numerical summary table (Table~\ref{tab:WaterAcessQualCompareTable}) may suggest a difference in the percentage of children with diarrhoea in households that do and do not keep livestock. The comparison in Fig.~\ref{fig:WaterAccessQualCompare} includes some categories with small sample sizes, so the percentages shown may not be precise estimates\index{Precision} of the population values.

As usual, the data come from one of countless possible samples, but the RQ is about the population, so making a definitive decision about the population is difficult.

\begin{table}
\centering
\caption{\label{tab:WaterAcessQualCrosstab}Cross-tabulation of having livestock in the household, and children under $5$ years of age having diarrhoea in the household in the last two weeks.}
\centering
\fontsize{8}{10}\selectfont
\begin{tabular}[t]{>{}lcc}
\toprule
\multicolumn{1}{c}{\textbf{ }} & \multicolumn{1}{c}{\textbf{No diarrhoea reported}} & \multicolumn{1}{c}{\textbf{Diarrhoea reported}} \\
\textbf{ } & \textbf{in children} & \textbf{in children}\\
\midrule
\textbf{Household does not have livestock} & $17$ & $\phantom{0}3$\\
\textbf{Household has livestock} & $42$ & $23$\\
\bottomrule
\end{tabular}
\end{table}

\begin{table}
\centering
\caption{\label{tab:WaterAcessQualCompareTable}Numerical summary of the water-access data: odds and percentage of children with diarrhoea in the last two weeks (comparing those without livestock to those with).}
\centering
\fontsize{8}{10}\selectfont
\begin{tabular}[t]{lccc}
\toprule
\multicolumn{1}{c}{\textbf{ }} & \multicolumn{1}{c}{\textbf{Percentage children}} & \multicolumn{1}{c}{\textbf{Odds children}} & \multicolumn{1}{c}{\textbf{Sample}} \\
\textbf{ } & \textbf{having diarrhoea} & \textbf{having diarrhoea} & \textbf{size}\\
\midrule
Household does not have livestock & $\phantom{0}15.0$ & $0.176$ & $20$\\
Household has livestock & $\phantom{0}35.4$ & $0.548$ & $65$\\
\midrule
\em{} & \em{\llap{Difference:\ }$-20.4$} & \em{\llap{OR:\ }$0.322$} & \em{}\\
\bottomrule
\end{tabular}
\end{table}

\begin{figure}[hbtp]

{\centering \includegraphics[width=1\linewidth]{15-CompareQual_files/figure-latex/WaterAccessQualCompare-1} 

}

\caption{Percentage of children with and without diarrhoea in the last two weeks, by water source (left) and how often the water vessel was cleaned (right).}\label{fig:WaterAccessQualCompare}
\end{figure}

\section{Chapter summary}\label{CompareQual-Summary}

Qualitative data can be compared between different groups (between-individuals comparisons) using a stacked bar chart, side-by-side bar chart or a dot chart. The data can be displayed in a two-way table, then summarised numerically by comparing proportions (or percentages) and odds. The odds ratio (OR) and the difference between the proportions (or percentages) can be used to compare the two different groups.

\section{Quick review questions}\label{CompareQual-QuickReview}

\citet{data:Alley2017:SocialMedia} examined social media use (Table~\ref{tab:SocialMedia}), using a representative sample of Queenslanders at least~\(18\) years of age (from the \(2013\) \emph{Queensland Social Survey}).

Are the following statements \emph{true} or \emph{false}?

\begin{enumerate}
\def\labelenumi{\arabic{enumi}.}
\item
  The \emph{sample proportion} of \emph{urban} residents who use social media is \(416/984 = 0.423\).\tightlist 
\item
  The \emph{sample proportion} of \emph{rural} residents who use social media is \(89/167 = 0.533\).
\item
  The \emph{sample odds} of \emph{urban} residents who use social media is \(416/568 = 0.732\).
\item
  The \emph{sample odds} of \emph{rural} residents who use social media is \(78/89 = 0.876\).
\item
  The \emph{sample OR} of using social media, comparing \emph{urban} to \emph{rural} residents is \(1.365/1.141 = 1.196\).
\item
  The \emph{sample difference between the proportions} using social media, comparing \emph{urban} to \emph{rural} residents, is \(0.577 -  0.533 = 0.044\).
\end{enumerate}

\begin{table}
\centering
\caption{\label{tab:SocialMedia}The number of Queenslanders using and not using social media (SM) in rural and urban locations in 2013 in a sample.}
\centering
\fontsize{8}{10}\selectfont
\begin{tabular}[t]{>{}lcc>{}c}
\toprule
\textbf{ } & \textbf{Doesn't use SM} & \textbf{Does use SM} & \textbf{Total}\\
\midrule
\textbf{Urban residents} & $568$ & $416$ & \textbf{$984$}\\
\textbf{Rural} & $\phantom{0}89$ & $\phantom{0}78$ & \textbf{$167$}\\
\bottomrule
\end{tabular}
\end{table}

\section{Exercises}\label{CompareQualData-Exercises}

\hyperref[Answers]{Answers to odd-numbered exercises} are given at the end of the book.

\captionsetup{font=small}

\begin{exercise}
\protect\hypertarget{exr:NumericalQual1}{}\label{exr:NumericalQual1}Suppose the sample OR has a value of one. What will be value of the difference between the sample proportions? Explain.
\end{exercise}

\begin{exercise}
\protect\hypertarget{exr:NumericalQual2}{}\label{exr:NumericalQual2}Suppose the sample OR (Row~1 divided by Row~2) has a value \emph{smaller} than one. Will the difference between the sample proportions (Row~1 minus Row~2) be a positive or a negative value? Explain carefully.
\end{exercise}

\begin{exercise}
\protect\hypertarget{exr:NumericalQualHangovers}{}\label{exr:NumericalQualHangovers}

\citet{data:Kochling2019:Hangover} studied hangovers and recorded, among other information, when people vomited after consuming alcohol. Table~\ref{tab:VomitTable} shows how many people vomited after consuming beer followed by wine, and how many people vomited after consuming only wine.

\begin{enumerate}
\def\labelenumi{\arabic{enumi}.}
\tightlist
\item
  Compute the \emph{row proportions}. What do these mean?
\item
  Compute the \emph{column percentages}. What do these mean?
\item
  Compute the \emph{overall percentage} of drinkers who vomited.
\item
  Compute the \emph{sample odds} that a wine-only drinker vomited.
\item
  Compute the \emph{sample odds} that a beer-then-wine drinker vomited.
\item
  Compute the \emph{sample OR}, comparing the odds of vomiting for wine-only drinkers to beer-then-wine drinkers.
\item
  Compute the \emph{sample OR}, comparing the odds of vomiting for beer-then-wine drinkers to wine-only drinkers.
\item
  Compute the difference between the \emph{sample proportions} of people vomiting, comparing beer-then-wine drinkers to wine-only drinkers.
\item
  What do the data suggest about the relationship?
\end{enumerate}

\end{exercise}

\begin{table}
\centering
\caption{\label{tab:VomitTable}How many people vomited and did not vomit, by type of alcohol consumed.}
\centering
\fontsize{8}{10}\selectfont
\begin{tabular}[t]{>{}lcc}
\toprule
\textbf{ } & \textbf{Beer then wine} & \textbf{Wine only}\\
\midrule
\textbf{Vomited} & $\phantom{0}6$ & $\phantom{0}6$\\
\textbf{Didn't vomit} & $62$ & $22$\\
\bottomrule
\end{tabular}
\end{table}

\begin{exercise}
\protect\hypertarget{exr:Wallabies}{}\label{exr:Wallabies}

\citet{data:Stirrat2008:wallabies} recorded the sex of adult and young wallabies at the East Point Reserve, Darwin. In December~1993, \(91\)~males and \(188\)~female \emph{adult} wallabies were recorded, and \(13\)~male and \(22\)~female \emph{young} wallabies were recorded.

\begin{enumerate}
\def\labelenumi{\arabic{enumi}.}
\tightlist
\item
  Create the two-way table of counts.
\item
  For \emph{adult} wallabies, what \emph{proportion} of adult wallabies were males?
\item
  For \emph{adult} wallabies, what are the \emph{odds} that a female was observed?
\item
  For \emph{young} wallabies, what \emph{percentage} of wallabies were males?
\item
  For \emph{young} wallabies, what are the \emph{odds} that a female was observed?
\item
  What is the OR of observing an adult wallaby, comparing females to males?
\item
  What is the difference between the sample proportions of females wallabies, comparing adults to young?
\item
  Create a summary table.
\item
  Sketch a graph to display the data.
\item
  What do the data suggest about the relationship?
\end{enumerate}

\end{exercise}

\begin{exercise}
\protect\hypertarget{exr:OddsAugustRainfall}{}\label{exr:OddsAugustRainfall}

{[}\emph{Dataset}: \texttt{EmeraldAug}{]} The \emph{Southern Oscillation Index} (SOI) is a standardised measure of the air pressure difference between Tahiti and Darwin, shown to be related to rainfall in some parts of the world \citep{climate:stone:1996}, and especially Queensland, Australia \citep{climate:stone:1992, mypapers:Dunn:bootstrap:2001}.

The rainfall at Emerald (Queensland) was recorded for Augusts between~1889 and~2002 inclusive \citep{mypapers:dunnsmyth:glms}, for months when the monthly average SOI was positive and non-positive (zero or negative); see Table~\ref{tab:SOItable}.

\begin{enumerate}
\def\labelenumi{\arabic{enumi}.}
\tightlist
\item
  Compute the \emph{percentage} of Augusts with no rainfall.
\item
  Compute the \emph{percentage} of Augusts with no rainfall, in Augusts with a \emph{non-positive SOI}.
\item
  Compute the \emph{percentage} of Augusts with no rainfall, in Augusts with a \emph{positive SOI}.
\item
  Compute the \emph{odds} of no August rainfall.
\item
  Compute the \emph{odds} of no August rainfall, in Augusts with a \emph{non-positive SOI}.
\item
  Compute the \emph{odds} of no August rainfall, in Augusts with a \emph{positive SOI}.
\item
  Compute the \emph{OR} of no August rainfall, comparing Augusts with \emph{non-positive SOI} to Augusts with a \emph{positive SOI}.
\item
  Interpret this OR.
\item
  Create a summary table.
\item
  Sketch a graph to display the data.
\end{enumerate}

\end{exercise}

\begin{table}
\centering
\caption{\label{tab:SOItable}The SOI, and whether rainfall was recorded in Augusts between 1889 and 2002.}
\centering
\fontsize{8}{10}\selectfont
\begin{tabular}[t]{>{}lcc}
\toprule
\textbf{ } & \textbf{Non-positive SOI} & \textbf{Positive SOI}\\
\midrule
\textbf{No rainfall recorded} & $14$ & $\phantom{0}7$\\
\textbf{Rainfall recorded} & $40$ & $53$\\
\bottomrule
\end{tabular}
\end{table}

\begin{exercise}
\protect\hypertarget{exr:OddsBackpacks}{}\label{exr:OddsBackpacks}

\citet{haselgrove2008perceived} asked boys and girls in Western Australia about back pain from carrying school bags (Table~\ref{tab:BagsTable}).

\begin{enumerate}
\def\labelenumi{\arabic{enumi}.}
\tightlist
\item
  Compute the \emph{percentage} of boys reporting back pain from carrying school bags.
\item
  Compute the \emph{percentage} of girls reporting back pain from carrying school bags.
\item
  Among the boys, compute the \emph{odds} of reporting back pain from carrying school bags.
\item
  Among the girls, compute the \emph{odds} of reporting back pain from carrying school bags.
\item
  Compute the \emph{odds} of a child reporting back pain.
\item
  Compute the \emph{OR} of reporting back pain, comparing boys to girls.
\item
  Interpret this OR.
\item
  Create a summary table.
\item
  Sketch a graph to display the data.
\end{enumerate}

\end{exercise}

\begin{table}
\centering
\caption{\label{tab:BagsTable}The number of boys and girls reporting back pain from carrying school bags.}
\centering
\fontsize{8}{10}\selectfont
\begin{tabular}[t]{>{}lcc}
\toprule
\textbf{ } & \textbf{Males} & \textbf{Females}\\
\midrule
\textbf{No back pain} & $330$ & $226$\\
\textbf{Back pain} & $280$ & $359$\\
\bottomrule
\end{tabular}
\end{table}

\begin{exercise}
\protect\hypertarget{exr:AVquestions}{}\label{exr:AVquestions}Using the information in Table~\ref{tab:AVtable2}, create a stacked bar chart to \emph{compare} the responses to the three questions.
\end{exercise}

\begin{exercise}
\protect\hypertarget{exr:Roadkill}{}\label{exr:Roadkill}

\citet{data:Russell2009:Possums} studied road-kill possums in the northern suburbs of Sydney (Table~\ref{tab:PossumRoadkillNumber}).

\begin{enumerate}
\def\labelenumi{\arabic{enumi}.}
\tightlist
\item
  Identify the two variables, and classify them as nominal or ordinal.
\item
  Sketch some graphs to display the data.
\item
  What is the main message in the data? What graph shows this best?
\end{enumerate}

\end{exercise}

\begin{table}
\centering
\caption{\label{tab:PossumRoadkillNumber}The number of possums found as road kill, by sex and season.}
\centering
\fontsize{8}{10}\selectfont
\begin{tabular}[t]{>{}lccc}
\toprule
\textbf{ } & \textbf{Unknown sex} & \textbf{Male} & \textbf{Female}\\
\midrule
\textbf{Autumn} & $75$ & $25$ & $21$\\
\textbf{Winter} & $74$ & $27$ & $22$\\
\textbf{Spring} & $71$ & $10$ & $18$\\
\textbf{Summer} & $58$ & $10$ & $12$\\
\bottomrule
\end{tabular}
\end{table}

\begin{exercise}
\protect\hypertarget{exr:SkippingBreakfast}{}\label{exr:SkippingBreakfast}

The data in Table~\ref{tab:SkipBreakfast} come from a study of Iranian children aged \(6\)--\(18\) years old \citep{data:kelishadi2017:snack}.

\begin{enumerate}
\def\labelenumi{\arabic{enumi}.}
\tightlist
\item
  Compute the \emph{proportion} of females who skipped breakfast.
\item
  Compute the \emph{proportion} of males who skipped breakfast.
\item
  Compute the \emph{odds} of a female skipping breakfast.
\item
  Compute the \emph{odds} of a male skipping breakfast.
\item
  Compute the \emph{OR} comparing the odds of skipping breakfast, comparing females to males.
\item
  Interpret this OR.
\item
  Construct a summary table.
\end{enumerate}

\end{exercise}

\begin{table}
\centering
\caption{\label{tab:SkipBreakfast}The number of Iranian children aged $6$ to $18$ who skip and do not skip breakfast.}
\centering
\fontsize{8}{10}\selectfont
\begin{tabular}[t]{>{}lcc>{}c}
\toprule
\textbf{ } & \textbf{Skips breakfast} & \textbf{Doesn't skip breakfast} & \textbf{Total}\\
\midrule
\textbf{Females} & $2\,383$ & $4\,257$ & \textbf{$6\,640$}\\
\textbf{Males} & $1\,944$ & $4\,902$ & \textbf{$6\,846$}\\
\bottomrule
\end{tabular}
\end{table}

\begin{exercise}
\protect\hypertarget{exr:CoffeeGreenTea}{}\label{exr:CoffeeGreenTea}

\citet{yonekura2020daily} studied Japanese women and their coffee drinking habits (Table~\ref{tab:CoffeeGT}).

\begin{enumerate}
\def\labelenumi{\arabic{enumi}.}
\tightlist
\item
  Compute the \emph{proportion} of coffee drinkers who are smokers.
\item
  Compute the \emph{proportion} of non-coffee drinkers who are smokers.
\item
  Compute the \emph{odds} of a coffee drinker being a smoker.
\item
  Compute the \emph{odds} of a non-coffee drinker being a smoker.
\item
  Compute the \emph{OR} comparing the odds of being a smoker, comparing coffee drinkers to non-coffee drinkers.
\item
  Interpret this OR.
\item
  Construct a summary table.
\end{enumerate}

\end{exercise}

\begin{table}
\centering
\caption{\label{tab:CoffeeGT}The number of Japanese women who smoked, and drank at least one cup of coffee per day.}
\centering
\fontsize{8}{10}\selectfont
\begin{tabular}[t]{>{}lcc}
\toprule
\textbf{ } & \textbf{Smokers} & \textbf{Non-smokers}\\
\midrule
\textbf{Coffee drinkers} & $10$ & $66$\\
\textbf{Non-coffee drinkers} & $\phantom{0}2$ & $84$\\
\bottomrule
\end{tabular}
\end{table}

\begin{exercise}
\protect\hypertarget{exr:Dispatchers}{}\label{exr:Dispatchers}

\citet{oostema2018emergency} studied how well emergency dispatchers recognised signs of stroke (Table~\ref{tab:DispatcherTab}).

\begin{enumerate}
\def\labelenumi{\arabic{enumi}.}
\tightlist
\item
  Sketch a side-by-side or stacked bar chart to display the data. \tightlist
\item
  Of the \emph{female} patients, what \emph{percentage} had stroke symptoms suspected by the dispatcher?
\item
  Of the \emph{male} patients, what \emph{percentage} had stroke symptoms suspected by the dispatcher?
\item
  For \emph{female} patients, what are the \emph{odds} they had stroke symptoms suspected by the dispatcher?
\item
  For \emph{male} patients, what are the \emph{odds} they had stroke symptoms suspected by the dispatcher?
\item
  What is the \emph{OR} that a patient had stroke symptoms suspected by the dispatcher, comparing \emph{females} to \emph{males}?
\item
  What is the \emph{OR} that a patient had stroke symptoms suspected by the dispatcher, comparing \emph{males} to \emph{females}?
\item
  Construct a numerical summary table.
\end{enumerate}

\end{exercise}

\begin{table}
\centering
\caption{\label{tab:DispatcherTab}The number of strokes suspected and missed by dispatchers.}
\centering
\fontsize{8}{10}\selectfont
\begin{tabular}[t]{>{}lcc}
\toprule
\textbf{ } & \textbf{Suspected stroke} & \textbf{Missed stroke}\\
\midrule
\textbf{Female patient} & $97$ & $67$\\
\textbf{Male patient} & $39$ & $43$\\
\bottomrule
\end{tabular}
\end{table}

\begin{exercise}
\protect\hypertarget{exr:SoccerHeaders}{}\label{exr:SoccerHeaders}

Soccer is a unique in that one aspect is `the purposeful use of the unprotected head for controlling and advancing the ball' \citep{kirkendall2001heading}. Some researchers suspect that repeatedly `heading' the ball may impair brain function. \citet{kirkendall2001heading} studied (p.~157)

\begin{quote}
\ldots whether long-term or chronic neuropsychological dysfunction (i.e., concussion) was present in collegiate soccer players
\end{quote}

Data were collected from~\(240\) college students for two variables:

\begin{itemize}
\tightlist
\item
  the student type, where each student was classified as a `soccer player' (\(63\)~students), a `non-soccer athlete' (\(96\) students), or a `non-athlete' (\(81\)~students).
\item
  the number of head concussions, where each student was asked about the number of head concussions they had experienced; `zero' (\(158\)~students), `one' (\(45\)~students), or `two or more' (\(37\)~students) concussions.
\end{itemize}

Use the study data (Table~\ref{tab:SoccerTable}) to answer the following questions.

\begin{table}
\centering
\caption{\label{tab:SoccerTable}The number of concussions experienced by college students.}
\centering
\fontsize{8}{10}\selectfont
\begin{tabular}[t]{>{}lccc>{}c}
\toprule
\multicolumn{1}{c}{\textbf{ }} & \multicolumn{3}{c}{\textbf{Number of concussions}} \\
\cmidrule(l{3pt}r{3pt}){2-4}
\textbf{ } & \textbf{0} & \textbf{1} & \textbf{2 or more} & \textbf{Total}\\
\midrule
\textbf{Soccer players} & $\phantom{0}45$ & $\phantom{0}5$ & $13$ & \textbf{$\phantom{0}63$}\\
\textbf{Non-soccer athletes} & $\phantom{0}68$ & $25$ & $\phantom{0}3$ & \textbf{$\phantom{0}96$}\\
\textbf{Non-athletes} & $\phantom{0}45$ & $15$ & $21$ & \textbf{$\phantom{0}81$}\\
\midrule
\textbf{\textbf{Total}} & \textbf{$158$} & \textbf{$45$} & \textbf{$37$} & \textbf{\textbf{$240$}}\\
\bottomrule
\end{tabular}
\end{table}

\begin{enumerate}
\def\labelenumi{\arabic{enumi}.}
\tightlist
\item
  Classify the two variables as nominal or ordinal.
\item
  Compute the percentage of college students in the \emph{sample} who have received exactly one concussion.
\item
  Among the \emph{non-athletes}, compute the odds of receiving two or more concussions. Interpret what this means.
\item
  Among the \emph{soccer players}, compute the odds of receiving two or more concussions. Interpret what this means.
\item
  Compute the OR comparing the odds of a non-athlete player receiving two or more concussions to the odds of a soccer player receiving two or more concussions.
\item
  Create a table of \emph{column} percentages. What do these tell you?
\item
  Create a table of \emph{row} percentages. What do these tell you?
\item
  Which one of these tables is probably more sensible, and why?
\end{enumerate}

\end{exercise}

\begin{exercise}
\protect\hypertarget{exr:PLHomeAway}{}\label{exr:PLHomeAway}{[}\emph{Dataset}: \texttt{PremierL}{]} In the 2019/2020 Premier League season, Chelsea had \(4\)~wins from \(10\)~games at home, and \(7\)~wins from \(11\)~wins away from home. What is the OR of a win (comparing home games and away games)?
\end{exercise}

\captionsetup{font=normalsize}

\begin{EOCanswerBox}{iconmonstr-check-mark-14-240.png}
\textbf{Answers to \emph{Quick review} questions:} \textbf{1.} True. \textbf{2.} False. \textbf{3.} True. \textbf{4.} True. \textbf{5.} False. \textbf{6.} False.

\end{EOCanswerBox}

\chapter{Correlations between quantitative variables}\label{TwoQuant}

\index{Quantitative data!correlation}

\begin{cols}
\begin{col}{0.52\textwidth}

\begin{objectivesBox}{iconmonstr-target-4-240.png}
So far, you have learnt to ask an RQ, design a study, collect the data, describe the data, and summarise data.
\textbf{In this chapter}, you will learn to:

\begin{itemize}\tightlist
  \item
  describe the relationships between two quantitative variables.
  \item
  compute and interpret correlation coefficients and $R^2$.
\end{itemize}
\end{objectivesBox}

\end{col}

\begin{col}{0.03\textwidth}
~
\end{col}

\begin{col}{0.45\textwidth}

\includegraphics[width=0.95\linewidth]{16-Connections-Two-Quant_files/figure-latex/unnamed-chunk-9-1} 
\end{col}
\end{cols}

\section{Introduction}\label{TwoQuant-Intro}

Correlational RQs ask about the relationship between two quantitative variables.\index{Research question!correlational} Scatterplots are useful for this purpose, and the relationship is usually described numerically using a correlation coefficient or \(R^2\).

\section{Graphs for the relationship}\label{Scatterplots}

\index{Quantitative data!correlation!graphs}\index{Graphs!scatterplot}\index{Software output!graphs}

Scatterplots display the relationship between \emph{two quantitative variables}. Conventionally, and when appropriate, the response variable (denoted~\(y\)) is shown on the vertical axis, and the explanatory variable (denoted~\(x\)) is shown on the horizontal axis.\index{Response variable}\index{Explanatory variable} Two quantitative variables are measured on each individual, and a point is placed on the scatterplot for each individual (unit of analysis) to indicate the values of the two variables. In some cases, which variable is denoted~\(x\) and which is~\(y\) is not important (e.g., see Exercise \ref{exr:CorTestDogs}.)

As with any graph, describing the message in the graph is important, because the purpose of a graph is to display the information in the clearest, simplest possible way.

\begin{example}[Red-deer data]
\protect\hypertarget{exm:RedDeer}{}\label{exm:RedDeer}\citet{data:Holgate1965:StraightLine} examined the relationship between the age of \(n = 78\) male red deer and the weight of their molars. The data (Table~\ref{tab:RedDeerData}) comprises two \emph{quantitative} variables, and both measurements are made on the same individuals (i.e., male red deer).

The scatterplot (Fig.~\ref{fig:RedDeerScatter}) shows one dot for each deer (individual). The response variable is the molar weight, which is on the vertical axis and denoted~\(y\). The explanatory variable is the deer age, which is on the horizontal axis and denoted~\(x\).

For instance, one deer is just over \(4\)~years of age (so \(x\) has a value a bit larger than~\(4\)), and has a molar weight of~\(2.42\,\text{g}\) (so that \(y = 2.42\)). This is the first deer listed in Table~\ref{tab:RedDeerData}.
\end{example}

\begin{table} \centering \centering\caption{\label{tab:RedDeerData}Molar weight and age
of male red deer: the first five and the last five observations are shown.}

\fontsize{8}{10}\selectfont
\begin{tabular}{cc}
\toprule
\multicolumn{1}{c}{\textbf{Age}} & \multicolumn{1}{c}{\textbf{Molar weight}} \\
\textbf{(in years)} & \textbf{(in g)}\\
\midrule
$\phantom{0}4.4$ & $2.42$\\
$\phantom{0}4.4$ & $4.45$\\
$\phantom{0}4.4$ & $5.24$\\
$\phantom{0}4.4$ & $3.19$\\
$\phantom{0}4.4$ & $3.90$\\
$\vdots$ & $\vdots$\\
\bottomrule
\end{tabular} \qquad 
\begin{tabular}{cc}
\toprule
\multicolumn{1}{c}{\textbf{Age}} & \multicolumn{1}{c}{\textbf{Molar weight}} \\
\textbf{(in years)} & \textbf{(in g)}\\
\midrule
$\vdots$ & $\vdots$\\
$12.4$ & $2.72$\\
$12.8$ & $1.71$\\
$13.4$ & $2.14$\\
$13.4$ & $2.76$\\
$14.4$ & $1.57$\\
\bottomrule
\end{tabular}
\end{table}



\begin{figure}[hbtp]

{\centering \includegraphics[width=0.7\linewidth]{16-Connections-Two-Quant_files/figure-latex/RedDeerScatter-1} 

}

\caption{A plot of the red-deer data. The indicated point is the first observation in Table~\ref{tab:RedDeerData}, where \(x = 4.4\) and \(y = 2.42\).}\label{fig:RedDeerScatter}
\end{figure}

\section{Describing scatterplots}\label{UnderstandingScatterplots}

\index{Graphs!scatterplot}

The purpose of a graph is to facilitate understanding of the data. For a scatterplot, the \emph{form}, \emph{direction}, and \emph{variation in the relationship} (or the \emph{strength of the relationship}) are described.

\begin{enumerate}
\def\labelenumi{\arabic{enumi}.}
\tightlist
\item
  \emph{Form}: the overall \emph{form} or structure of the relationship (e.g., linear; curved upwards; etc.).\index{Graphs!scatterplot!form}
\item
  \emph{Direction}: the \emph{direction} of the relationship (sometimes not relevant if the relationship is non-linear):\index{Graphs!scatterplot!direction}

  \begin{itemize}
  \tightlist
  \item
    a \emph{positive} association exists if \emph{high} values of one variable accompany \emph{high} values of the other variable, in general.
  \item
    a \emph{negative} association exists if \emph{high} values of one variable accompany \emph{low} values of the other variable, in general.
  \end{itemize}
\item
  \emph{Variation}: the amount of \emph{variation} in the relationship.\index{Graphs!scatterplot!variation}\index{Graphs!scatterplot!strength} A small amount of variation in the response variable for given values of the explanatory variable means the relationship is strong; a lot of variation in the response variable for given values of the explanatory variable means the relationship is weak. Describing the variation can be difficult; an objective, numerical way to do so is explained in Sect.~\ref{CorrelationR2}.
\end{enumerate}

Anything unusual or noteworthy should also be discussed. These features explain the \emph{type} of relationship (\emph{form}; \emph{direction}), and the \emph{strength} of that relationship (\emph{variation}). Examples are shown in Fig.~\ref{fig:ScatterplotDescriptionExamples}.

\begin{importantBox}{iconmonstr-warning-8-240.png}
The axes do not need to \emph{start from zero}, since the distance of the dots from the axes visually do not imply any quantity of interest.

\end{importantBox}

\begin{figure}[hbtp]

{\centering \includegraphics[width=1\linewidth]{16-Connections-Two-Quant_files/figure-latex/ScatterplotDescriptionExamples-1} 

}

\caption{Some example scatterplots. The dark lines show the overall relationship between the variables.}\label{fig:ScatterplotDescriptionExamples}
\end{figure}

\begin{example}[Scatterplots]
\protect\hypertarget{exm:DescribeScatterplotsDeer}{}\label{exm:DescribeScatterplotsDeer}For the red-deer data (Fig.~\ref{fig:RedDeerScatter}), the relationship is approximately linear (form) with a negative direction (\emph{older} deer generally have \emph{lighter} teeth); the \emph{variation} is, perhaps, moderate.
\end{example}

\begin{example}[Describing scatterplots]
\protect\hypertarget{exm:DescribeScatterplots}{}\label{exm:DescribeScatterplots}\citet{data:Tager:FEV} (cited by \citet{BIB:data:FEV}) measured the lung capacity of children in Boston (using forced expiratory volume, FEV, in litres). The scatterplot (Fig.~\ref{fig:FEVscatter}) is curved (\emph{form}), where older children have larger FEVs, in general (\emph{direction}). The \emph{variation} in FEV gets larger for taller youth.
\end{example}

\begin{figure}[hbtp]

{\centering \includegraphics[width=1\linewidth]{16-Connections-Two-Quant_files/figure-latex/FEVscatter-1} 

}

\caption{FEV plotted against height for children in Boston.}\label{fig:FEVscatter}
\end{figure}

\section{\texorpdfstring{Numerical summary: correlation coefficient and \(R^2\)}{Numerical summary: correlation coefficient and R\^{}2}}\label{CorrelationR2}

\subsection{Correlation coefficients}\label{CorrCoefficients}

\index{Graphs!scatterplot!form}\index{Correlation}\index{Correlation coefficient (Pearson)}

In general, summarising the relationship between two quantitative variables is difficult, because the possible relationships vary greatly (consider the variety in the scatterplots shown in Fig.~\ref{fig:ScatterplotDescriptionExamples}). However, if we focus only on approximately \emph{linear} relationships, the best way to numerically summarise the relationship between the variables is to use a \emph{correlation coefficient}. Both quantitative variables can also be numerically summarised individually.

\begin{definition}[Correlation coefficient]
\protect\hypertarget{def:CorrelationCoefficient}{}\label{def:CorrelationCoefficient}The Pearson correlation coefficient measures the \emph{strength} and \emph{direction} of the \emph{linear} relationship between two quantitative variables. Its value is always between~\(-1\) and~\(+1\).
\end{definition}

Pearson correlation coefficients only apply if both of these are true:

\begin{itemize}
\tightlist
\item
  the form is approximately \emph{linear}.
\item
  the variation in the values of~\(y\) is reasonably constant for all values of~\(x\).
\end{itemize}

Hence, checking the scatterplot first is important.

Only the \emph{Pearson} correlation coefficient is discussed in this book (and usually referred to as the `correlation coefficient'), but other correlation coefficients also exist (such as the \emph{Spearman}\index{Correlation coefficient!Spearman} or \emph{Kendall} correlation coefficients)\index{Correlation coefficient!Kendall}, which may be used for increasing-only or decreasing-only \emph{non-linear} relationships).

\begin{importantBox}{iconmonstr-warning-8-240.png}
The Pearson correlation coefficient only makes sense if the relationship is approximately linear.

\end{importantBox}

In the \emph{population}, the unknown value of the correlation coefficient is denoted~\(\rho\) (`rho'); in the \emph{sample}, the value of the correlation coefficient is denoted \(r\). As usual, \(r\) (the \emph{statistic}) is an estimate of \(\rho\) (the \emph{parameter}), and the value of~\(r\) is likely to be different in every sample (that is, \emph{sampling variation} exists).

\begin{pronounceBox}{iconmonstr-microphone-7-240.png}

The symbol \(\rho\) is the Greek letter `rho', pronounced `row', as in `row your boat'.

\end{pronounceBox}

The values of~\(\rho\) and~\(r\) are \emph{always} between~\(-1\) and~\(+1\). The \emph{sign} indicates whether the relationship has a positive or negative linear association, and the \emph{value} of the correlation coefficient describes the \emph{strength} of the relationship, as follows.

\begin{itemize}
\tightlist
\item
  \(r = -1\) indicates a \emph{perfect, negative} relationship.\index{Correlation coefficient (Pearson)!positive} By `perfect', we mean that each value of \(x\) always produces the same value of~\(y\); the negative value means \emph{larger} values of~\(y\) are associated with \emph{smaller} values of~\(x\).
\item
  values of~\(r\) between~\(-1\) and~\(0\) indicates a \emph{negative} relationship.\index{Correlation coefficient (Pearson)!negative} Each value of~\(x\) produces a range of values of~\(y\), and \emph{larger} values of~\(y\) are associated with \emph{smaller} values of~\(x\) (in general).
\item
  \(r = 0\) indicates \emph{no linear relationship} between the variables:\index{Correlation coefficient (Pearson)!zero} knowing how the value of~\(x\) changes tells us nothing about how the corresponding value of~\(y\) changes. The best prediction of~\(y\) for \emph{any} value of~\(x\) would be the mean of~\(y\); i.e., the value of~\(\bar{y}\).
\item
  values of~\(r\) between~\(0\) and~\(+1\) indicates a \emph{positive} relationship. Each value of~\(x\) produces a range of values of~\(y\), and \emph{larger} values of~\(y\) are associated with \emph{larger} values of~\(x\) (in general).
\item
  \(r = +1\) indicates a \emph{perfect, positive} relationship. By `perfect', we mean that each value of~\(x\) always produces the same value of~\(y\); the positive value means \emph{larger} values of~\(y\) are associated with \emph{larger} values of~\(x\).
\end{itemize}

Almost all values of~\(r\) seen in practice are between the extremes of \(r = -1\) and \(r = +1\). Guessing the values of the correlation coefficient from a scatterplot is very difficult.

\begin{example}[Correlation coefficients]
\protect\hypertarget{exm:CorrCoefForExamples}{}\label{exm:CorrCoefForExamples}

Numerous example scatterplots were shown in Sect.~\ref{UnderstandingScatterplots}. A correlation coefficient is not relevant for Plots~C,~D or~E, as those relationships are not linear. For the others:

\begin{itemize}
\tightlist
\item
  \emph{Plot~A}: the correlation coefficient is \emph{positive}, and reasonably close to one.
\item
  \emph{Plot~B}: the correlation coefficient is \emph{negative}, but not near~\(-1\).
\item
  \emph{Plot~F}: the correlation coefficient is close to zero.
\end{itemize}

\end{example}

\begin{example}[Correlation coefficients]
\protect\hypertarget{exm:Correlations}{}\label{exm:Correlations}\citet{leuchtenberger2022effects} and \citet{Nishizaki2022SanddollarData} explored the relationship between water temperature and fertilisation rates for sand dollars (Fig.~\ref{fig:SanddollarsPlot}). The correlation coefficient is \(r = -0.71\) (left panel), which might suggest that \emph{higher} temperatures result in \emph{lower} fertilisation rates. However, a \emph{curved} relationship is apparent (right panel), and so the relationship is more complex: the fertilisation rate increases up to about \(18\)\textsuperscript{o}C, and then starts falling again.

A Pearson correlation coefficient is not suitable for describing the relationship.
\end{example}

\begin{figure}[hbtp]

{\centering \includegraphics[width=0.9\linewidth]{16-Connections-Two-Quant_files/figure-latex/SanddollarsPlot-1} 

}

\caption{Water temperature vs fertilisation rates for sand dollars. Left: an inappropriate linear relationship. Right: the appropriate curved relationship.}\label{fig:SanddollarsPlot}
\end{figure}

Formulas exist to compute the value of~\(r\), but are tedious to use manually. We will use software output to obtain values of~\(r\).\index{Software output!correlation}

\begin{example}[Correlation coefficients]
\protect\hypertarget{exm:CorrelationsDeer}{}\label{exm:CorrelationsDeer}For the red-deer data (Fig.~\ref{fig:RedDeerScatter}), the relationship is approximately linear, and the software output (Fig.~\ref{fig:RedDeerCorrelationjamovi}) shows that \(r = -0.584\). The value of~\(r\) is \emph{negative} because, in general, \emph{older} deer~(\(x\)) are associated with \emph{smaller} weight molars~(\(y\)). The relationship may be described as `moderately strong' perhaps.
\end{example}

\begin{example}[Correlation coefficients]
\protect\hypertarget{exm:LungCapCor}{}\label{exm:LungCapCor}\citet{data:Tager:FEV} studied the lung capacity (forced expiratory volume; FEV) of children in Boston \citep{BIB:data:FEV}. The scatterplot in Fig.~\ref{fig:FEVscatter} is not linear, so a correlation coefficient is inappropriate.
\end{example}

\begin{figure}[hbtp]

{\centering \includegraphics[width=0.5\linewidth]{jamovi/RedDeer/RedDeer-Correlation} 

}

\caption{ Software output for correlation for the red-deer data.}\label{fig:RedDeerCorrelationjamovi}
\end{figure}

\index{Software output!correlation}

\subsection{\texorpdfstring{R-squared (\(R^2\))}{R-squared (R\^{}2)}}\label{Rsquared}

\index{R@$R^2$}

While~\(r\) describes the strength and direction of the linear relationship, knowing exactly what the value \emph{means} is tricky. Interpretation is easier using~\(R^2\): the square of the value of~\(r\).

\begin{definition}[$R^2$]
\protect\hypertarget{def:R2}{}\label{def:R2}The value of~\(R^2\) is how much the unexplained variation in the values of~\(y\) is reduced (usually expressed as a percentage) due to using the extra information in the values of~\(x\).
\end{definition}

\begin{pronounceBox}{iconmonstr-microphone-7-240.png}
\(R^2\) is pronounced `\(r\)-squared'.

\end{pronounceBox}

The value of~\(R^2\) is \emph{never} negative, and is usually multiplied by~\(100\) and expressed as a percentage.

\begin{softwareBox}{iconmonstr-laptop-4-240.png}
The value of~\(R^2\) is never negative! However, you need to be careful using your calculator. On most calculators, entering \texttt{-0.5\^{}2} returns an answer of~\texttt{-0.25}. The calculator interprets your input as meaning \texttt{-(0.5\^{}2)}.

Use brackets; \texttt{(-0.5)\^{}2} gives the correct answer of~\texttt{0.25} (or~\(25\)\%).

\end{softwareBox}

\begin{example}[Values of $R$-squared]
\protect\hypertarget{exm:R2Deer}{}\label{exm:R2Deer}For the red-deer data (Fig.~\ref{fig:RedDeerScatter}), the value of~\(R^2\) is \(R^2 = (-0.584)^2 = 0.341\), usually written as a percentage: \(34.1\)\%. The value of~\(R^2\) is positive, even though the value of~\(r\) is negative.

This means a reduction of about~\(34.1\)\% in the unexplained variation of the molar weights, due to using the information in the age of the deer (see Example~\ref{exm:ReductionR2}). The rest of the variation in molar weights is due to chance, and to extraneous variables such as weight, diet, amount of exercise, genetics, etc.
\end{example}

\(R^2\) measures the reduction in the unexplained variation in values of~\(y\) because the value of~\(x\) is known. If the values of~\(x\) were unknown, the best summary of the~\(y\)-values is the mean of the~\(y\)-values (i.e., \(\bar{y}\)). However, if a relationship exists between the values of~\(x\) and~\(y\) then better estimates of the value of~\(y\) could be made by knowing the value of~\(x\). That means that less variation should be left unexplained.

When expressed as a percentage, \(R^2\) measures how much the unexplained variation reduces due to our knowledge of the linear relationship. If \(R\)-squared is zero, then the amount of unexplained variation has not reduced at all, and exploring the relationship between~\(x\) and~\(y\) has no value.

\begin{example}[Unknown variation in $y$]
\protect\hypertarget{exm:ReductionR2}{}\label{exm:ReductionR2}\index{R@$R^2$!meaning} For the red-deer data, the unexplained variation in the values of~\(y\) (molar weight), without knowing anything about the age of the deer, is the variation in the \emph{distances from the mean} to each observation (Fig.~\ref{fig:RedDeerR2}, left panels). Effectively, the unexplained variation is the standard deviation of the molar weights (\(s = 0.7263\)).

If the age of the deer~(\(x\)) is used, the unexplained variation in the values of~\(y\) is now the variation in the \emph{distances from the line explaining the relationship} to each observation (Fig.~\ref{fig:RedDeerR2}, right panels). The distances are shorter, in general, showing a decrease in the \emph{unexplained} variation. Effectively, the unexplained variation is the standard deviation of the distances from the line to the observations (\(s = 0.5895\)).

Hence, the reduction in the \emph{square} of the standard deviations is \((0.7263^2 - 0.5895^2)/0.7263^2 = 0.341\), or \(34.1\)\%. This is the value of~\(R^2\).
\end{example}

\begin{figure}[hbtp]

{\centering \includegraphics[width=0.95\linewidth]{16-Connections-Two-Quant_files/figure-latex/RedDeerR2-1} 

}

\caption{The unexplained variation for the red-deer data. Left panels: when no information about the age of the deer is used, the mean (the horizontal grey regression line in the top panel) is the best summary of the molar weight. Right panels: when information about the age of the deer is used (as shown by the grey line in the top panel), the distances are shorter in general. $R^2$ is a measure of how much smaller.}\label{fig:RedDeerR2}
\end{figure}

\section{Numerical summary tables}\label{ConnectionSummaryTables}

\index{Quantitative data!correlation!summary tables}\index{Summary table!correlations}

In general, numerically summarising the relationship between two quantitative variables is difficult because of the many types of possible relationships (Sect.~\ref{UnderstandingScatterplots}). However, for \emph{linear} relationships, both quantitative variables can be summarised, and the correlation coefficient can be given (Table~\ref{tab:RDSummaryTable}).

\begin{table}
\centering\centering
\caption{\label{tab:RDSummaryTable}A numerical summary of the red-deer data.}
\centering
\fontsize{8}{10}\selectfont
\begin{tabular}[t]{lcccc}
\toprule
\textbf{ } & \textbf{Mean} & \textbf{Standard deviation} & \textbf{Sample size} & \textbf{Correlation}\\
\midrule
Age (in years) & $\phantom{0}7.7$ & $\phantom{0}2.34$ & $\phantom{0}78$ & $\phantom{0}\llap{$-{}$}0.584$\\
Molar weight (in g) & $\phantom{0}3.0$ & $\phantom{0}0.73$ & $\phantom{0}78$ & \\
\bottomrule
\end{tabular}
\end{table}

\section{Example: removal efficiency}\label{ScatterplotsRemoval-Efficiency}

In wastewater treatment facilities, air from biofiltration is passed through a membrane and dissolved in water, and is transformed into harmless by-products. The removal efficiency~\(y\) (in~\%) may depend on the inlet temperature (in~\textsuperscript{o}C;~\(x\)). \citet{chitwood2001treatment} asked:

\begin{quote}
In treating biofiltation wastewater, is the removal efficiency linearly associated with the inlet temperature?
\end{quote}

A scatterplot of \(n = 32\) observations \citep{DevoreBerk2007} suggests an approximately linear relationship (Fig.~\ref{fig:ScatterplotRemovalEfficiency}, left panel). The direction is positive: larger inlet temperatures are associated with a larger removal efficiency, in general. The variation is always hard to describe in words, but is perhaps `reasonably small'.

A more precise way to measure the strength of the linear association is to use the correlation coefficient. Using software output (Fig.~\ref{fig:ScatterplotRemovalEfficiency}, right panel), \(r = 0.891\), and so \(R^2 = (0.891)^2 = 79.4\)\%. This means that about~\(79.4\)\% of the variation in removal efficiency can be explained by knowing the inlet temperature.

\begin{figure}[hbtp]

{\centering \includegraphics[width=0.52\linewidth]{16-Connections-Two-Quant_files/figure-latex/ScatterplotRemovalEfficiency-1} \includegraphics[width=0.02\linewidth]{OtherImages/SPACER} \includegraphics[width=0.41\linewidth]{jamovi/Removal/RemovalCorrelation} 

}

\caption{The relationship between removal efficiency and inlet temperature. Left: scatterplot. Right: software output.}\label{fig:ScatterplotRemovalEfficiency}
\end{figure}

\section{Chapter summary}\label{TwoQuant-Summary}

A \emph{scatterplot} displays the relationship between two quantitative variables (the response denoted~\(y\); the explanatory denoted~\(x\)). The relationship is described by the \emph{form} (linear, or otherwise), the \emph{direction} of the relationship (sometimes not relevant if the graph is not linear), and the \emph{variation} in the relationship (or the \emph{strength} of the relationship).

Linear relationships are measured numerically using the correlation coefficient and~\(R^2\). \emph{Correlation coefficients} (denoted~\(r\) in the sample; \(\rho\) in the population) are always between~\(-1\) and~\(+1\). \emph{Positive} values denote \emph{positive} relationships between the two variables: as the values of one variable get larger, the values of the other tend to get larger too. \emph{Negative} values denote \emph{negative} relationships between the two variables: as the values of one variable get larger, the values of the other tend to get \emph{smaller}. Values close to~\(-1\) or~\(+1\) are very strong relationships; values near zero shows very little linear relationship between the values of the two variables.

Sometimes, \(R^2\) is used to describe the relationship: it measures how much the unexplained variation in the values of \(y\) is reduced due to using the extra information in the values of \(x\).

\section{Quick review questions}\label{TwoQuant-QuickReview}

A study of onion growth \citep{meadplant} produced the scatterplot shown in Fig.~\ref{fig:MeanScatter}.

\begin{figure}[hbtp]

{\centering \includegraphics{16-Connections-Two-Quant_files/figure-latex/MeanScatter-1} 

}

\caption{Onion yield plotted against planting density.}\label{fig:MeanScatter}
\end{figure}

Are the following statements \emph{true} or \emph{false}?

\begin{enumerate}
\def\labelenumi{\arabic{enumi}.}
\item
  The \(x\)-variable is `planting density'.\tightlist
\item
  The best description for the \emph{form} of the relationship is `curved'.
\item
  The best description for the \emph{direction} of the relationship is `negative'.
\item
  The best description for the \emph{variation} in the relationship is `small'.
\end{enumerate}

\section{Exercises}\label{TwoQuant-Exercises}

\hyperref[Answers]{Answers to odd-numbered exercises} are given at the end of the book.

\captionsetup{font=small}

\begin{exercise}
\protect\hypertarget{exr:CorrelationExerciseDrawNegR}{}\label{exr:CorrelationExerciseDrawNegR}

Draw a scatterplot with:

\begin{enumerate}
\def\labelenumi{\arabic{enumi}.}
\tightlist
\item
  a negative correlation coefficient, with~\(r\) very close to (but not equal to)~\(-1\).
\item
  a positive correlation coefficient, with~\(r\) very close to (but not equal to)~\(+1\).
\item
  a correlation coefficient very close to~\(0\).
\end{enumerate}

\end{exercise}

\begin{exercise}
\protect\hypertarget{exr:MatchCorrelations}{}\label{exr:MatchCorrelations}Estimate the correlation coefficients from scatterplots in Fig.~\ref{fig:MatchScatter}, when appropriate. (You can only give very rough estimates!)
\end{exercise}

\begin{figure}[hbtp]

{\centering \includegraphics[width=1\linewidth]{16-Connections-Two-Quant_files/figure-latex/MatchScatter-1} 

}

\caption{Four plots: estimate the correlation coefficients.}\label{fig:MatchScatter}
\end{figure}

\begin{exercise}
\protect\hypertarget{exr:TwoQuantExercisesPeas}{}\label{exr:TwoQuantExercisesPeas}{[}\emph{Dataset}: \texttt{Peas}{]} \citet{hacisalihoglu2021characterization} studied the nutritional content of peas (\emph{Pisum sativum}), and measured the quantities of various minerals. In these plots, it does not matter which of the pair of variables is used on the horizontal axis and which is used on the vertical axis. From Fig.~\ref{fig:PeasScatter} (\emph{left} panel), estimate the value of~\(r\).
\end{exercise}

\begin{exercise}
\protect\hypertarget{exr:TwoQuantExercisesPeas2}{}\label{exr:TwoQuantExercisesPeas2}{[}\emph{Dataset}: \texttt{Peas}{]} \citet{hacisalihoglu2021characterization} studied of the nutritional content of peas (\emph{Pisum sativum}), and measured the quantities of various minerals. In these plots, it does not matter which of the pair of variables is used on the horizontal axis and which is used on the vertical axis. From Fig.~\ref{fig:PeasScatter} (\emph{right} panel), estimate the value of~\(r\).
\end{exercise}

\begin{figure}[hbtp]

{\centering \includegraphics[width=0.9\linewidth]{16-Connections-Two-Quant_files/figure-latex/PeasScatter-1} 

}

\caption{The relationship between some minerals in pea seeds.}\label{fig:PeasScatter}
\end{figure}

\begin{exercise}
\protect\hypertarget{exr:TwoQuantExercisesLimeTrees}{}\label{exr:TwoQuantExercisesLimeTrees}

{[}\emph{Dataset}: \texttt{Lime}{]} \citet{data:ForestBiomass2017} measured the diameter and the age of \(385\)~small-leaved lime trees (Fig.~\ref{fig:LimeTreesScatter}).

\begin{enumerate}
\def\labelenumi{\arabic{enumi}.}
\tightlist
\item
  What does each point on the scatterplot represent?
\item
  Describe the scatterplot.
\item
  Would a correlation coefficient be appropriate? Explain.
\end{enumerate}

\end{exercise}

\begin{figure}[hbtp]

{\centering \includegraphics[width=0.95\linewidth]{16-Connections-Two-Quant_files/figure-latex/LimeTreesScatter-1} 

}

\caption{The age and foliage biomass of small-leaved lime trees grown in Russia ($n = 385$). The solid line on the right panel displays the linear relationship.}\label{fig:LimeTreesScatter}
\end{figure}

\begin{exercise}
\protect\hypertarget{exr:TwoQuantExercisesBoneQuality}{}\label{exr:TwoQuantExercisesBoneQuality}

{[}\emph{Dataset}: \texttt{BoneQuality}{]} \citet{kim2022similarities} measured numerous data from \(969\)~South Korean subjects, including \(517\)~females. A scatterplot of the height and age for females is shown in Fig.~\ref{fig:HeightAgeScatter}.

\begin{enumerate}
\def\labelenumi{\arabic{enumi}.}
\tightlist
\item
  What does each point on the scatterplot represent?
\item
  Describe the relationship.
\item
  Would a correlation coefficient be appropriate? Explain.
\item
  Does the scatterplot suggest that people become less tall, on average, as they age? Explain.
\end{enumerate}

\end{exercise}

\begin{figure}[hbtp]

{\centering \includegraphics[width=0.55\linewidth]{16-Connections-Two-Quant_files/figure-latex/HeightAgeScatter-1} 

}

\caption{The age and height of South Korean females ($n = 517$). The solid line shows the linear relationship.}\label{fig:HeightAgeScatter}
\end{figure}

\begin{exercise}
\protect\hypertarget{exr:TwoQuantExercisesSoftdrink}{}\label{exr:TwoQuantExercisesSoftdrink}

{[}\emph{Dataset}: \texttt{SDrink}{]} \citet{others:Montgomery:regressionanalysis} examined the time taken to deliver soft drinks to vending machines.

\begin{enumerate}
\def\labelenumi{\arabic{enumi}.}
\tightlist
\item
  Describe the relationship (Fig.~\ref{fig:MandibleGestationPlot}, left panel).
\item
  What does each point represent?
\item
  Would a correlation coefficient be appropriate? Explain.
\end{enumerate}

\end{exercise}

\begin{exercise}
\protect\hypertarget{exr:TwoQuantExercisesMandible}{}\label{exr:TwoQuantExercisesMandible}{[}\emph{Dataset}: \texttt{Mandible}{]} \citet{data:royston:mandible} examined the mandible length and gestational age for \(167\)~foetuses from the \(12\)th~week of gestation onward. Describe the relationship (Fig.~\ref{fig:MandibleGestationPlot}, right panel).
\end{exercise}

\begin{figure}[hbtp]

{\centering \includegraphics[width=1\linewidth]{16-Connections-Two-Quant_files/figure-latex/MandibleGestationPlot-1} 

}

\caption{Two scatterplots. Left: the time taken to deliver soft drinks to vending machines. Right: the relationship between gestational age and mandible length. In both plots, the solid grey line displays the linear relationship.}\label{fig:MandibleGestationPlot}
\end{figure}

\begin{exercise}
\protect\hypertarget{exr:TwoQuantExercisesGorillas}{}\label{exr:TwoQuantExercisesGorillas}{[}\emph{Dataset}: \texttt{Gorillas}{]} \citet{wright2021chest} recorded the chest-beating rate of \(25\)~gorillas, and the gorillas' size (measured by the breadth of the gorillas' backs). Describe the relationship (Fig.~\ref{fig:GorillaWindmillPlot}, left panel).
\end{exercise}

\begin{figure}[hbtp]

{\centering \includegraphics[width=1\linewidth]{16-Connections-Two-Quant_files/figure-latex/GorillaWindmillPlot-1} 

}

\caption{Two scatterplots. Left: chest beating in gorillas. Right: the relationship between DC output and wind speed.}\label{fig:GorillaWindmillPlot}
\end{figure}

\begin{exercise}
\protect\hypertarget{exr:TwoQuantExercisesWindmill}{}\label{exr:TwoQuantExercisesWindmill}{[}\emph{Dataset}: \texttt{Windmill}{]} \citet{data:joglekar:lackoffit} examined the relationship between direct current (DC) generated by a windmill and wind speed \citep{data:hand:handbook}. Describe the relationship (Fig.~\ref{fig:GorillaWindmillPlot}, right panel).
\end{exercise}

\begin{exercise}
\protect\hypertarget{exr:SoilCN}{}\label{exr:SoilCN}

{[}\emph{Dataset}: \texttt{SoilCN}{]} \citet{lambie2021microbial} recorded the percentage carbon (C) and the percentage nitrogen (N) in \(28\)~irrigated farming plots (Fig.~\ref{fig:IrrigatedCNStudents}, left panel).

\begin{enumerate}
\def\labelenumi{\arabic{enumi}.}
\tightlist
\item
  Describe the relationship.
\item
  Does it matter which variable is~\(x\) and which is~\(y\)? Explain.
\item
  What does each point represent?
\end{enumerate}

\end{exercise}

\begin{figure}[hbtp]

{\centering \includegraphics[width=1\linewidth]{16-Connections-Two-Quant_files/figure-latex/IrrigatedCNStudents-1} 

}

\caption{Left: the percentage N and percentage C in irrigated plots. Right: the weight of students in Week\ 1 and Week\ 12 of the university semester.}\label{fig:IrrigatedCNStudents}
\end{figure}

\begin{exercise}
\protect\hypertarget{exr:StudentWts}{}\label{exr:StudentWts}

{[}\emph{Dataset}: \texttt{StudentWt}{]} The weights of students starting at university (Week~1) and in Week~12 are shown in Fig~\ref{fig:IrrigatedCNStudents} (right panel).

\begin{enumerate}
\def\labelenumi{\arabic{enumi}.}
\tightlist
\item
  Describe the relationship.
\item
  What does each point represent?
\end{enumerate}

\end{exercise}

\begin{exercise}
\protect\hypertarget{exr:TwoQuantExercisesONI}{}\label{exr:TwoQuantExercisesONI}{[}\emph{Dataset}: \texttt{Cyclones}{]} The relationship \citep{mypapers:dunnsmyth:glms} between the number of cyclones~\(y\) in the Australian region each year from 1969 to 2005, and a climatological index called the \emph{Ocean Niño Index} (ONI,~\(x\)), is shown in Fig.~\ref{fig:ONIcyclonesCI}.

From software, \(r = -0.682\). What is the value of~\(R^2\)? What does it mean?
\end{exercise}

\begin{figure}[hbtp]

{\centering \includegraphics{16-Connections-Two-Quant_files/figure-latex/ONIcyclonesCI-1} 

}

\caption{The number of cyclones in the Australian region each year from 1969 to 2005, and the ONI for October, November, December.}\label{fig:ONIcyclonesCI}
\end{figure}

\captionsetup{font=normalsize}

\begin{EOCanswerBox}{iconmonstr-check-mark-14-240.png}
\textbf{Answers to \emph{Quick review} questions:} \textbf{1.} True. \textbf{2.} True. \textbf{3.} True. \textbf{4.} True.

\end{EOCanswerBox}

\chapter{More details about tables and graphs}\label{SummariseComments}

\begin{cols}
\begin{col}{0.52\textwidth}

\begin{objectivesBox}{iconmonstr-target-4-240.png}
So far, you have learnt to ask an RQ, design a study, collect the data, describe the data, and summarise data.
\textbf{In this chapter}, you will learn to:

\begin{itemize}\tightlist
  \item
  construct clear and informative graphs.
  \item
  construct clear and informative tables.
\end{itemize}
\end{objectivesBox}

\end{col}

\begin{col}{0.03\textwidth}
~
\end{col}

\begin{col}{0.45\textwidth}

\includegraphics[width=0.95\linewidth]{17-SummaryComments_files/figure-latex/unnamed-chunk-8-1} 
\end{col}
\end{cols}

\section{Introduction}\label{MoreTablesGraphsIntro}

A summary of the data is important for understanding the data, and for planning the direction of the analysis. In this chapter, we make some general comments for constructing graphs and tables. Always remember:

\begin{importantBox}{iconmonstr-warning-8-240.png}
The purpose of a graph and a table is to display the information in the clearest, simplest possible way, to facilitate understanding the message(s) in the data.

\end{importantBox}

\section{More details about preparing graphs}\label{GraphsConstructing}

\index{Graphs!preparing}\index{Software output!graphs}\index{Graphs!using software}

Helping readers to understand the data is the goal of producing a graph. You should be able to sketch graphs by hand, but \emph{usually software is used to produce graphs}.\index{Computers and software} Using a computer makes it easy to try different graphs, to change features of graphs, and to produce the best graph possible. When creating graphs, ensure you:

\begin{itemize}
\tightlist
\item
  \emph{do} make graphs clear and well-labelled.
\item
  \emph{do} add informative titles and axis labels.
\item
  \emph{do} add units of measurement where necessary.
\item
  \emph{do} add informative captions \emph{below} the figure.
\item
  \emph{do} add units of measurement and axis labels where appropriate.
\item
  \emph{do} make sure text and details are easy to read.
\item
  \emph{do} ensure the axis scales are appropriate.
\item
  \emph{do} add any necessary explanations.
\item
  \emph{do} make it easy for readers to easily make the important comparisons, as far as possible.
\item
  \emph{do not} add artificial third dimensions, or other `chart junk' \citep{su2008s}; see Sect.~\ref{ThirdDimensions}.
\item
  \emph{do not} add optical illusions, such as an artificial third dimension.
\item
  \emph{do not} use distracting colours and fonts; only use different colours and fonts if necessary, and explain that purpose if it is not clear.
\item
  \emph{do not} make errors.
\end{itemize}

Some specific problems to be aware of are discussed in the subsections that follows.

\subsection{Avoid unnecessary third dimensions}\label{ThirdDimensions}

Graphs should focus on clear communication. One barrier to clear communication is using an unnecessary third dimension. This is poor: such graphs can be misleading and hard to read \citep{siegrist1996use}.

\begin{example}[Two- and three-dimensional plots]
\protect\hypertarget{exm:Bar2D3D}{}\label{exm:Bar2D3D}In the \textsc{nhanes} study \citep{data:NHANES3:Data}, the age and sex of each participant were recorded. Using Fig.~\ref{fig:Bar3D2D} (left panel), can you easily determine if more females or more males are present in each age group?

The artificial third dimension makes determining the heights of the bars hard. In contrast, a side-by-side bar graph (Fig.~\ref{fig:Bar3D2D}, right panel) makes it clear whether each age group has more females or more males.\index{Graphs!side-by-side bar chart}
\end{example}



\begin{figure}[hbtp]

{\centering \includegraphics[width=1\linewidth]{17-SummaryComments_files/figure-latex/Bar3D2D-1} 

}

\caption{Two plots of the \textsc{nhanes} participants, divided by age group and sex. Left: a three-dimensional bar chart. Right: a side-by-side bar chart.}\label{fig:Bar3D2D}
\end{figure}

\subsection{Avoid overplotting}\label{Overplotting}

Some plots, such as dot charts and scatterplots, may suffer from \emph{overplotting}:\index{Overplotting} when multiple observations have the same (or nearly the same) values, and these cannot be distinguished on the graph. Overplotting can especially be a problem when plotting \emph{discrete} quantitative data. In many cases (such as dot charts), points can be \emph{jittered}\index{Overplotting!jittering} by adding a small amount of randomness to the observations, or \emph{stacked}; see Example~\ref{exm:Dotchart2DGorillas}.\index{Overplotting!stacking} Jittering is the best option for scatterplots. Overplotted points can change readers' impression of the data, since some observations are obscured and are effectively `lost' to the reader.

\subsection{Take care when truncating axes}\label{TruncatingAxes}

One common optical illusion occurs when the frequency (or percentage) axis does not start at zero. This is a problem in graphs where the distance represented visually implies the frequency of those observations, as with the count (or percentage) axis in bar charts, dot charts, or histograms. This is \emph{not} a problem in, for example, boxplots and scatterplots, where the distance of points from zero do not visually imply any quantity of interest.

Sometimes, the axes may be truncated intentionally so the differences are easier to see. In these cases, the reader should be alerted that the axes have been truncated for this reason.

\begin{example}[Truncating is not appropriate]
\protect\hypertarget{exm:VerticalTruncationOK}{}\label{exm:VerticalTruncationOK}Consider data recording the number of lung cancer cases in Fredericia in various age groups \citep{data:andersen:1977}.

Figure~\ref{fig:BarchartTruncated1} (left panel) shows a good bar chart with the count (vertical) axis starting at zero; the counts in each age group look similar. In contrast, if the vertical axis starts at \(9\), the counts look very different (Fig.~\ref{fig:BarchartTruncated1}, right panel) for two age categories, suggesting large difference between the number of lung cancer cases. The graph is visually misleading when the graph does not start at a count of zero, since the height of the bars from the axis visually implies the frequency of those observations.
\end{example}

\begin{figure}[hbtp]

{\centering \includegraphics[width=0.9\linewidth]{17-SummaryComments_files/figure-latex/BarchartTruncated1-1} 

}

\caption{The same data presented in two bar charts, without truncating the vertical axis (left) and truncating the vertical axis (right).}\label{fig:BarchartTruncated1}
\end{figure}

\begin{example}[Truncating is appropriate]
\protect\hypertarget{exm:VerticalTruncation}{}\label{exm:VerticalTruncation}Consider data recording the body temperature of \(n = 130\) people (\citet{data:mackowiak:bodytemp}, \citet{data:Shoemaker1996:Temperature}). A histogram of the data (Fig.~\ref{fig:HistosTemp}, left panel) clearly shows the distribution of body temperatures.

The vertical axis, displaying the counts, must start at zero since the bar heights visually imply a quantity of interest. However, the horizontal axis starts at~\(35.5\)\textsuperscript{o}C, which does not create any problems as the \emph{distances} from a temperature of~\(0\)\textsuperscript{o}C do not visually imply any quantity of interest.\index{Graphs!histogram}

In contrast, starting the horizontal axis at a temperature of~\(0\)\textsuperscript{o}C (Fig.~\ref{fig:HistosTemp}, right panel) makes any details in the histogram difficult to see; the histogram is pointless.
\end{example}

\begin{figure}[hbtp]

{\centering \includegraphics[width=1\linewidth]{17-SummaryComments_files/figure-latex/HistosTemp-1} 

}

\caption{Two histograms displaying the body temperature of $130$ people. Left: a well-constructed histogram. Right: a poorly-constructed histogram.}\label{fig:HistosTemp}
\end{figure}

\subsection{Take care when using pie charts}\label{PieChartProblems}

\index{Graphs!bar chart}\index{Graphs!pie chart}

As noted in Sect.~\ref{PieCharts}, pie charts may be hard to read: humans compare \emph{lengths} (bar and dot charts) better than \emph{angles} (pie charts) \citep{data:Friel:Graphs}. Pie charts are also difficult to use with levels having zero or small counts.

\begin{example}[Pie charts with small counts]
\protect\hypertarget{exm:PieSmallCounts}{}\label{exm:PieSmallCounts}\citet{data:Solomon2002:ginkgo} studied the use of ginkgo for memory enhancement. Caregivers blinded\index{Blinding} to the treatment (ginkgo or placebo)\index{Placebo} reported the impact on subjects' memory. The bar chart (Fig.~\ref{fig:PieSmallCounts}, left panel), for subjects on the placebo, shows that four of the available categories had zero responses, and one had a very small number of responses (two). The pie chart (right) make the small category difficult to see, and the categories with zero counts impossible to see.
\end{example}

\begin{figure}[hbtp]

{\centering \includegraphics[width=0.9\linewidth]{17-SummaryComments_files/figure-latex/PieSmallCounts-1} 

}

\caption{Data with zeros and small counts are easy to see in a bar chart (left panel) and dot chart, but difficult to see in a pie chart (right panel).}\label{fig:PieSmallCounts}
\end{figure}

\section{More details about preparing tables}\label{TablesConstructing}

\index{Tables!preparing}

A computer is helpful for constructing tables. Using a computer also makes it easy to try different orientations or layouts. As with graphs, the purpose of tables is to help readers understand the data. When creating numerical summary tables, ensure you:

\begin{itemize}
\tightlist
\item
  \emph{do} make tables clear and well-labelled.
\item
  \emph{do} use clear and informative row and column labels (as necessary).
\item
  \emph{do} add units of measurement where necessary.
\item
  \emph{do} add informative captions \emph{above} the table.
\item
  \emph{do} add units of measurement and value labels where appropriate.
\item
  \emph{do} make sure text and details are easy to read.
\item
  \emph{do} round numbers appropriately (don't necessarily use all figures provided by software).
\item
  \emph{do} align numbers in the table by decimal point if possible, for easier reading and comparing.
\item
  \emph{do} construct the table to allow readers to easily make the important comparisons, as far as possible (space restriction may take precedence, for example).
\item
  \emph{do not} use distracting colours and fonts; only use different colours and fonts if necessary, and explain that purpose if it is not clear.
\item
  \emph{do not} use vertical lines (in general), and use \emph{very few} horizontal lines. Horizontal lines can be used to group columns (for example, see Table~\ref{tab:WaterAccessSummaryCommentsTable}).
\end{itemize}

\section{Example: water access}\label{WaterAcessSummary}

\citet{lopez2022farmers} recorded data about access to water in three rural communities in Cameroon (see Sects.~\ref{WaterAccessQual} and~\ref{WaterAccessQuant}). The study could be used to determine associations to the incidence of diarrhoea in young children (\(85\) households had children under~\(5\) years of age). Relationships between the incidence of diarrhoea and some other variables appear in Figs.~\ref{fig:WaterAccessQualCompare} and~\ref{fig:WaterAccessCompareQuantFigs}. A summary table of information can also be constructed (Table~\ref{tab:WaterAccessSummaryCommentsTable}).

In this table, note that:

\begin{itemize}
\tightlist
\item
  quantitative and qualitative variables are summarised differently, but appropriately.
\item
  units of measurements are given where appropriate (i.e., only for age).
\item
  numbers in columns are aligned for easier reading and comparing.
\end{itemize}

\begin{table}
\centering
\caption{\label{tab:WaterAccessSummaryCommentsTable}Numerical summary of the water-access data in $85$ households with children. `All households' are broken into those that reported, and did not report, diarrhoea in children under $5$ years of age in the last two weeks.}
\centering
\fontsize{8}{10}\selectfont
\begin{tabular}[t]{lcccccc}
\toprule
\multicolumn{1}{c}{\textbf{ }} & \multicolumn{2}{c}{\textbf{All households}} & \multicolumn{2}{c}{\textbf{Diarrhoea}} & \multicolumn{2}{c}{\textbf{No diarrhoea}} \\
\cmidrule(l{3pt}r{3pt}){2-3} \cmidrule(l{3pt}r{3pt}){4-5} \cmidrule(l{3pt}r{3pt}){6-7}
  & $n$ & Summary & $n$ & Summary & $n$ & Summary\\
\midrule
\textbf{Age (in years)}$^a$ & $85$ & $37.0$ \ $( 28.0 )$ & $59$ & $35.0$ \ $( 22.5 )$ & $26$ & $46.5$ \ $( 28.5 )$\\
\textbf{Household size}$^a$ & $85$ & $\phantom{0}7.0$ \ $( \phantom{0}6.0 )$ & $59$ & $\phantom{0}6.0$ \ $( \phantom{0}4.5 )$ & $26$ & $\phantom{0}8.5$ \ $( \phantom{0}7.8 )$\\
\textbf{Under $5$s in household}$^a$ & $85$ & $\phantom{0}2.0$ \ $( \phantom{0}2.0 )$ & $59$ & $\phantom{0}2.0$ \ $( \phantom{0}1.0 )$ & $26$ & $\phantom{0}2.0$ \ $( \phantom{0}1.8 )$\\
\addlinespace[0.3em]
\multicolumn{7}{l}{\textbf{Region$^b$}}\\
\hspace{1em}Mbeng & $26$ & $30.6$\% & $14$ & $53.8$\% & $12$ & $46.2$\%\\
\hspace{1em}Mbih & $28$ & $32.9$\% & $19$ & $67.9$\% & $\phantom{0}9$ & $32.1$\%\\
\hspace{1em}Ntsingbeu & $31$ & $36.5$\% & $26$ & $83.9$\% & $\phantom{0}5$ & $16.1$\%\\
\addlinespace[0.3em]
\multicolumn{7}{l}{\textbf{Water source$^b$}}\\
\hspace{1em}Tap & $\phantom{0}6$ & $\phantom{0}7.1$\% & $\phantom{0}5$ & $83.3$\% & $\phantom{0}1$ & $16.7$\%\\
\hspace{1em}Bore & $56$ & $65.9$\% & $40$ & $71.4$\% & $16$ & $28.6$\%\\
\hspace{1em}Well & $10$ & $11.8$\% & $\phantom{0}5$ & $50.0$\% & $\phantom{0}5$ & $50.0$\%\\
\hspace{1em}River & $13$ & $15.3$\% & $\phantom{0}9$ & $69.2$\% & $\phantom{0}4$ & $30.8$\%\\
\addlinespace[0.3em]
\multicolumn{7}{l}{\textbf{Education$^b$}}\\
\hspace{1em}Primary or less & $38$ & $44.7$\% & $27$ & $71.1$\% & $11$ & $28.9$\%\\
\hspace{1em}Secondary or higher & $47$ & $55.3$\% & $32$ & $68.1$\% & $15$ & $31.9$\%\\
\addlinespace[0.3em]
\multicolumn{7}{l}{\textbf{Has livestock$^b$}}\\
\hspace{1em}No & $20$ & $23.5$\% & $17$ & $85.0$\% & $\phantom{0}3$ & $15.0$\%\\
\hspace{1em}Yes & $65$ & $76.5$\% & $42$ & $64.6$\% & $23$ & $35.4$\%\\
\bottomrule
\multicolumn{7}{l}{\textsuperscript{a} Quantitative variables are summarised using medians and IQR.}\\
\multicolumn{7}{l}{\textsuperscript{b} Qualitative variables are summarised using counts and percentages.}\\
\end{tabular}
\end{table}

The table summarises the \emph{sample}, but RQs are about the \emph{population}. For example, one relational RQ could be:

\begin{quote}
Is the percentage of households with children under~\(5\) years of age having diarrhoea the same for households that do and do not keep livestock?
\end{quote}

Since the observed sample is one of countless possible samples that may have been selected, answering RQs about the population is not straightforward. In the observed sample, \(85.0\)\% of households that \emph{did not} keep livestock reported diarrhoea in children under~\(5\), while \(64.6\)\% of households that \emph{did} keep livestock reported diarrhoea in children under~\(5\). That is, a difference is seen \emph{in the sample}; but RQs ask about the \emph{population}.

Broadly, two possible reasons could explain why the \emph{sample} percentages of households reporting diarrhoea in children are different:

\begin{enumerate}
\def\labelenumi{\arabic{enumi}.}
\tightlist
\item
  \emph{The \textbf{population} percentages are the same}. The \emph{sample} percentages are \emph{different} simply because of the households selected in this particular sample. Another sample, with different households, might produce different sample percentages. \emph{Sampling variation explains the difference in the sample percentages}.
\item
  \emph{The \textbf{population} percentages are different}. The difference between the \emph{sample} percentages reflects this difference between the \emph{population} percentages.
\end{enumerate}

The difficulty is knowing which of these reasons (`hypotheses')\index{Hypotheses} is the most likely explanation for the difference between the sample percentages. This question is of prime importance as it answers the RQ.\spacex Tools for answering these questions are considered later in this book.

\section{Quick review questions}\label{SummaryCommentsQuickReview}

Are the following statements \emph{true} or \emph{false}?

\begin{enumerate}
\def\labelenumi{\arabic{enumi}.}
\item
  Graphs usually have their captions \emph{under} the figure. \tightlist
\item
  Graphs should use as many colours as possible.
\item
  Graphs should usually be carefully created using computer software.
\item
  Tables should have plenty of horizontal and vertical lines.
\item
  Tables usually have their captions \emph{under} the table.
\end{enumerate}

\section{Exercises}\label{SummariseCommentsExercises}

\hyperref[Answers]{Answers to odd-numbered exercises} are given at the end of the book.

\captionsetup{font=small}

\begin{exercise}
\protect\hypertarget{exr:Graphs123}{}\label{exr:Graphs123}

What would be the best graph for displaying the data for these situations?

\begin{enumerate}
\def\labelenumi{\arabic{enumi}.}
\item
  Researchers record the pH of water and the temperature of the water, in various creeks in the north island of New Zealand, to explore the relationship between pH and temperature.
\item
  Researchers measure the difference between each swimmers' fastest \(100\,\text{m}\) time and their fastest \(200\,\text{m}\) time. The researchers were interested in the average time \emph{difference}.
\item
  A research study examined the way in which students usually came to university (bus; private car; carpooling; etc.) and their program of study.
\end{enumerate}

\end{exercise}

\begin{exercise}
\protect\hypertarget{exr:Graphs456}{}\label{exr:Graphs456}

What would be the best graph for displaying the data for these situations?

\begin{enumerate}
\def\labelenumi{\arabic{enumi}.}
\item
  Researchers record the number of times a specific recycling bin is used each day at a shopping centre, over many days.
\item
  Researchers measure the difference between heart rate before and two hours after drinking a cup of coffee. The researchers were interested in the average increase in heart rate.
\item
  A research study recorded the diet of students (vegan; vegetarian; other) and the cost of groceries in the previous week, for many students. The researchers were exploring if there was any relationship between diet and cost of groceries.
\end{enumerate}

\end{exercise}

\begin{exercise}
\protect\hypertarget{exr:GraphsLimeTrees}{}\label{exr:GraphsLimeTrees}\citet{data:ForestBiomass2017} recorded these variables for \(385\)~lime trees in Russia: the foliage biomass (in~kg; the response variable); the tree diameter (in~cm; the explanatory variable); the age of the tree (in~years); and the origin of the tree (one of Coppice, Natural, or Planted).

The purpose of the study is to estimate the foliage biomass from the tree diameter, in the presence of some extraneous variables. What graphs would be useful?
\end{exercise}

\begin{exercise}
\protect\hypertarget{exr:GraphNitrogenInSoil}{}\label{exr:GraphNitrogenInSoil}\citet{data:Lane2002:GLMsoilscience} recorded the soil nitrogen after applying different fertiliser doses. The researchers recorded:

\begin{itemize}
\tightlist
\item
  the fertiliser dose, in kilograms of nitrogen per hectare;
\item
  the soil nitrogen, in kilograms of nitrogen per hectare; and
\item
  the fertiliser source; one of `inorganic' or `organic'.
\end{itemize}

What graphs would be useful for understanding the data?
\end{exercise}

\begin{exercise}
\protect\hypertarget{exr:GraphNoisyMiners}{}\label{exr:GraphNoisyMiners}\citet{data:Maron:eucthreshold} counted the number of noisy miners (an Australian bird) and eucalyptus trees in random quadrats. Critique the graph (not given by \citet{data:Maron:eucthreshold}!) of the data (Fig.~\ref{fig:MinerCrabPlot}, left panel).
\end{exercise}

\begin{figure}[hbtp]

{\centering \includegraphics[width=0.95\linewidth]{17-SummaryComments_files/figure-latex/MinerCrabPlot-1} 

}

\caption{Left: the number of noisy miners and the number of eucalyptus trees. Right: a scatterplot of the colour of female horseshoe crabs and the condition of their spines.}\label{fig:MinerCrabPlot}
\end{figure}

\begin{exercise}
\protect\hypertarget{exr:GraphHorseshoeCrabs}{}\label{exr:GraphHorseshoeCrabs}\citet{data:brockmann:crabs} recorded, among other variables, the colour of the carapace (`Light medium', `Medium', `Dark medium' or `Dark') and the condition of the carapace (`Both OK', `One OK', `None OK') of \(n = 173\) female horseshoe crabs. Critique the scatterplot (Fig.~\ref{fig:MinerCrabPlot}, right panel) used to explore the data.
\end{exercise}

\begin{exercise}
\protect\hypertarget{exr:GraphsMADRS}{}\label{exr:GraphsMADRS}

\citet{data:Danielsson2014:Depression} examined the change in \textsc{madrs} (a \emph{quantitative} scale measuring level of depression) and treatment group (whether each person was treated using: exercise; body awareness; or advice).

\begin{enumerate}
\def\labelenumi{\arabic{enumi}.}
\tightlist
\item
  What is the response variable?
\item
  What is the explanatory variable?
\item
  What graphs would be useful for exploring the data and the relationships of interest?
\end{enumerate}

\end{exercise}

\begin{exercise}
\protect\hypertarget{exr:GraphsSkewBar}{}\label{exr:GraphsSkewBar}A study of high-performance athletes at the \emph{Australian Institute of Sport} (AIS) \citep{data:Telford1991:sexsportsize} recorded numerous variables about athletes. A plot for the sports played by the athletes is shown in Fig.~\ref{fig:AISSportBarchart}. How would you describe the data: left skewed, right skewed, approximately symmetrical? Or something else?
\end{exercise}

\begin{figure}[hbtp]

{\centering \includegraphics{17-SummaryComments_files/figure-latex/AISSportBarchart-1} 

}

\caption{Sports played by athletes in the AIS study.}\label{fig:AISSportBarchart}
\end{figure}

\begin{exercise}
\protect\hypertarget{exr:GraphsTyping}{}\label{exr:GraphsTyping}{[}\emph{Dataset}: \texttt{Typing}{]} The \texttt{Typing} dataset \citep{pinet2022typing} contains four variables: typing speed (\texttt{mTS}), typing accuracy (\texttt{mAcc}), age (\texttt{Age}), and sex (\texttt{Sex}) for \(1\,301\)~students. Produce graphs necessary for understanding the data, making sure to explain what they reveal.

Does the mean typing speed or mean accuracy appear to differ by the age or sex of the student? What other questions could be useful to ask about the data?
\end{exercise}

\begin{exercise}
\protect\hypertarget{exr:WriteExercisesNHANES}{}\label{exr:WriteExercisesNHANES}{[}\emph{Dataset}: \texttt{NHANES}{]} Consider the \textsc{nhanes} data. In preparing a paper about this study, suppose Fig.~\ref{fig:ResultsPlot} and Tables~\ref{tab:ResultsTable} were produced. Critique these.
\end{exercise}

\begin{figure}
\begin{minipage}{0.45\textwidth}
\captionof{table}{A table of results\label{tab:ResultsTable}.}
\fontsize{8}{12}\selectfont
\begin{@empty}

\begin{tabular}{lll}
\toprule
  & \textbf{Mean} & \textbf{Std dev.}\\
\midrule
\textbf{Current smoker} & $206.6$ & $46$\\
\textbf{Current non-smoker} & $214.64$ & $48.7945$\\
\textbf{Difference} & $8.03$ & \\
\bottomrule
\end{tabular}
\end{@empty}
\end{minipage}
\hspace{0.1\textwidth}
\begin{minipage}{0.45\textwidth}%

\includegraphics[width=0.95\linewidth]{17-SummaryComments_files/figure-latex/unnamed-chunk-6-1} 
\captionof{figure}{A boxplot\label{fig:ResultsPlot}.}
\end{minipage}
\end{figure}

\captionsetup{font=normalsize}

\begin{EOCanswerBox}{iconmonstr-check-mark-14-240.png}
\textbf{Answers to \emph{Quick review} questions:} \textbf{1.} True. \textbf{2.} False. Use different colours only if they have a purpose (and explain that purpose if it is not clear). \textbf{3.} True. \textbf{4.} False: very few vertical lines (if any); minimum of horizontal lines. \textbf{5.} False.

\end{EOCanswerBox}

\part{Tools for answering RQs}\label{part-tools-for-answering-rqs}

\chapter{Probability}\label{Probability}

\begin{cols}
\begin{col}{0.52\textwidth}

\begin{objectivesBox}{iconmonstr-target-4-240.png}
So far, you have learnt to ask an RQ, design a study, and describe and summarise the data.
\textbf{In this chapter}, you will learn to:
\begin{itemize}\tightlist
  \item
  explain probability.
  \item
  identify and apply various approaches to computing probability.
  \item
  determine the probability of events described using \textbf{and}, \textbf{or} and \textbf{not} in simple situations.
  \item
  identify events that are independent.
  \item
  compute simple conditional probabilities.
\end{itemize}
\end{objectivesBox}

\end{col}

\begin{col}{0.03\textwidth}
~
\end{col}

\begin{col}{0.45\textwidth}

\includegraphics[width=0.95\linewidth]{18-Tools-Probability_files/figure-latex/unnamed-chunk-4-1} 
\end{col}
\end{cols}

\section{Introduction}\label{Chap19Intro}

This chapter briefly discusses \emph{probability}. \emph{Probability} quantifies the chance of observing a specific, unknown result (an `event'). Before discussing probability, some associated terms need defining.

\begin{definition}[Random procedure]
\protect\hypertarget{def:RandomProcedure}{}\label{def:RandomProcedure}\index{Random procedure} A \emph{random procedure} is a sequence of well-defined steps that (a)~can be repeated, in theory, indefinitely under essentially identical conditions; (b)~has well-defined results; and (c)~where results are unpredictable for any individual repetition.
\end{definition}

Using this definition, the result of rolling a die is a `random procedure', with possible results \largedice{1}, \largedice{2}, \largedice{3}, \largedice{4}, \largedice{5} and \largedice{6}. Similarly, tossing a coin is a random procedure with two possible results: \textbf{Heads} \Heads~or \textbf{Tails} \Tails.

\section{Sample spaces, events and probability}\label{SampleSpaceEvents}

A list of all mutually exclusive\index{Mutually exclusivr} possible results from one instance of a random procedure is the \emph{sample space}. A \emph{simple event} is any element of the sample space.\index{Event}

\begin{definition}[Sample space]
\protect\hypertarget{def:SampleSpace}{}\label{def:SampleSpace}\index{Sample space} The \emph{sample space} is a list of all possible and mutually exclusive (distinct) results after administering a random procedure once.
\end{definition}

\begin{definition}[Simple event]
\protect\hypertarget{def:SimpleEvent}{}\label{def:SimpleEvent}\index{Event!simple} A \emph{simple event} is a single element of the sample space.
\end{definition}

\begin{example}[Sample spaces]
\protect\hypertarget{exm:SampleSpaceDie}{}\label{exm:SampleSpaceDie}Consider rolling a fair, six-sided die (the random procedure). We do not know what face will be uppermost until we roll the die.

However, the \emph{sample space} for this procedure can be listed: \largedice{1}, \largedice{2}, \largedice{3}, \largedice{4}, \largedice{5} and~ \largedice{6}. These are all mutually exclusive\index{Mutually exclusive} (or distinct) results and cover all possible results (exhaustive)\index{Exhaustive} from a single roll. The sample space is \emph{discrete} (see Sect.~\ref{QuantData}).\index{Quantitative data!discrete}

The event `rolling a \largedice{1}' is a simple event.
\end{example}

Combinations of the elements in the sample space are usually of more interest than simple events. These are called \emph{compound events}.

\begin{definition}[Compound event]
\protect\hypertarget{def:CompoundEvent}{}\label{def:CompoundEvent}\index{Event!compound} A \emph{compound event} is any combination of simple events.
\end{definition}

\begin{example}[Events]
\protect\hypertarget{exm:Events}{}\label{exm:Events}Some \emph{events} that can be defined using the sample space in Example~\ref{exm:SampleSpaceDie} include:

\begin{itemize}
\tightlist
\item
  rolling a \largedice{4}. This \emph{simple event} includes one element of the sample space: \largedice{4}.
\item
  rolling an odd number. This \emph{compound event} includes three elements of the sample space: \largedice{1}, \largedice{3} and \largedice{5}.
\item
  rolling a number larger than \largedice{2}. This \emph{compound event} includes four elements of the sample space: \largedice{3}, \largedice{4}, \largedice{5} and \largedice{6}.
\end{itemize}

The sample space is \emph{discrete} (see Sect.~\ref{QuantData}).
\end{example}

\begin{example}[Sample spaces and events]
\protect\hypertarget{exm:SampleSpaceThrowing}{}\label{exm:SampleSpaceThrowing}Consider the distance you can throw a baseball (the random procedure). We do not know beforehand what distance your next throw will be, but the \emph{sample space} (i.e., the throwing distance) is a number greater than \(0\,\text{m}\). This sample space is \emph{continuous}.\index{Quantitative data!continuous}

Many \emph{compound events} can be defined using this sample space; for example:

\begin{itemize}
\tightlist
\item
  throwing more than~\(50\,\text{m}\).
\item
  throwing between~\(10\) and~\(40\,\text{m}\).
\end{itemize}

Because the sample space is continuous, throwing an \emph{exact} distance (such as \emph{exactly}~\(10\,\text{m}\)) is technically not possible (see Sect.~\ref{QuantData}).
\end{example}

Events are often defined using \textbf{and}, \textbf{or}, \textbf{not}. Consider two events called~\(A\) and~\(B\). Then, `\(A\) \textbf{and}~\(B\)' is the event comprising events only occurring in \emph{both} events~\(A\) and~\(B\). `\(A\) \textbf{or}~\(B\)' is the event comprising all events in~\(A\), all events in~\(B\), and events in both. The event `\textbf{not}~\(A\)' comprises all the events in the sample space that are \emph{not} in Event~\(A\).

\begin{example}[Defining events]
\protect\hypertarget{exm:ComplicatedEvents}{}\label{exm:ComplicatedEvents}Consider rolling a fair, six-sided die again (Example~\ref{exm:SampleSpaceDie}). Suppose these two (compound) events are defined:

\begin{itemize}
\tightlist
\item
  Event~\(A\) is `roll a number divisible by~\(2\)'.
\item
  Event~\(B\) is `roll a number divisible by~\(3\)'.
\end{itemize}

Event~\(A\) comprises the simple events `roll a \largedice{2}', `roll a \largedice{4}' and `roll a \largedice{6}'. Event~\(B\) comprises the simple events `roll a \largedice{3}' and `roll a \largedice{6}'.

Then, the Event~`\(A\) \textbf{and}~\(B\)' includes all events only occurring in both~\(A\) and in~\(B\); that is, `\(A\) \textbf{and}~\(B\)' comprises the single simple event `roll a \largedice{6}'.

Event~`\(A\) \textbf{or}~\(B\)' include the events in~\(A\), the events in~\(B\), and those in both; that is, `\(A\) \textbf{or}~\(B\)' comprises the four simple events `roll a \largedice{2}', `roll a \largedice{3}', `roll a \largedice{4}' and~ `roll a \largedice{6}'.

The event `\textbf{not}~\(A\)' comprises the three simple events `roll a \largedice{1}', `roll a \largedice{3}' and~ `roll a \largedice{5}'.
\end{example}

Using these definitions, a \emph{probability} can be defined.\index{Probability}

\begin{definition}[Probability]
\protect\hypertarget{def:Probability}{}\label{def:Probability}A \emph{probability} is a number between \(0\) and \(1\) inclusive (or between~\(0\)\% and~\(100\)\% inclusive) that quantifies the likelihood that a certain event will occur.
\end{definition}

A probability of~\(0\) (or~\(0\)\%) means the event is `impossible' (will \emph{never} occur), and a probability of~\(1\) (or~\(100\)\%) means that the event is \emph{certain} to happen (will \emph{always} occur). Most events have a probability between the extremes of~\(0\)\% and~\(100\)\%.

\begin{example}[Probabilities]
\protect\hypertarget{exm:Probabilities}{}\label{exm:Probabilities}

Consider these examples:

\begin{itemize}
\tightlist
\item
  the probability of receiving negative rainfall in London next year is~\(0\); it is impossible.
\item
  the probability of receiving some rain in London next year is~\(1\); it is certain.
\item
  the probability of receiving rain on 01~January next year in London is between~\(0\) and~\(1\) inclusive.
\end{itemize}

\end{example}

\section{Determining probabilities}\label{DetermineProbabilities}

Three different ways to think about probability are:

\begin{itemize}
\tightlist
\item
  the \emph{classical approach} (Sect.~\ref{ProbClassical}).
\item
  the \emph{relative frequency approach} (Sect.~\ref{ProbRelFreq}).
\item
  the \emph{subjective approach} (Sect.~\ref{ProbSubjective}).
\end{itemize}

These approaches help determine, or approximate, values for probabilities.

\subsection{Classical approach}\label{ProbClassical}

\index{Probability!classical approach}

What is the probability of rolling a \largedice{4} on a die? The sample space has six possible outcomes (see Example~\ref{exm:SampleSpaceDie}) that are \emph{equally likely} to occur (i.e., no reason exists to expect one event to occur more often than the others), and the event `rolling a \largedice{4}' comprises just \emph{one} of those events. Thus, \[
   \text{Probability of rolling a $\largedice{4}$}
   = \frac{\text{The number of rolls that are a $\largedice{4}$}}{\text{The number of possible events in the sample space}}
   = \frac{1}{6}.
\]

This approach to computing probabilities is called the \emph{classical} approach to probability, and is only appropriate when all events in the sample space are \emph{equally likely}.

\begin{definition}[Classical approach to probability]
\protect\hypertarget{def:ClassicalApproachToProbability}{}\label{def:ClassicalApproachToProbability}In the \emph{classical approach to probability}, the probability of an event occurring is the number of elements of the sample space included in the event, divided by the total number of elements in the sample space, \emph{when all outcomes are equally likely}.
\end{definition}

By this definition: \[
   \text{Prob. of an event}
    = 
    \frac{\text{Number of simple events in the event of interest}}{\text{Total number of possible equally-likely events}}.
\]

We can say that `the probability of rolling a \largedice{4} is \(1/6\)', or `the probability of rolling a \largedice{4} is \(0.1667\)'. The answer can also be expressed as a \emph{percentage}: `the probability of rolling a \largedice{4} is \(16.67\)\%'.\index{Percentages} The answer could also be interpreted as `the \emph{expected} proportion of rolls that are a \largedice{4} is \(0.1667\)'.\index{Proportions} That is, about~\(16.67\)\% of a very large number of future rolls are likely to be a \largedice{4}.

The probability of rolling a \largedice{4} is \(0.1667\), but any single roll of the die either \emph{will} or \emph{will not} produce a \largedice{4}, and we don't know which will occur.

\begin{example}[Probabilities for compound events]
\protect\hypertarget{exm:SimpleProb}{}\label{exm:SimpleProb}Consider rolling a standard six-sided die. With six equally-likely results (Example~\ref{exm:SampleSpaceDie}), the probability of rolling an even number is~\(3/6\), since there are three even numbers in the sample space.
\end{example}

\begin{example}[Describing probability]
\protect\hypertarget{exm:ProbabilityOutcomes}{}\label{exm:ProbabilityOutcomes}

Consider rolling a standard six-sided die.

\begin{itemize}
\tightlist
\item
  The \emph{probability} of rolling an even number is \(3 \div 6 = 0.5\).
\item
  The \emph{percentage} of rolls expected to be even is \(3 \div 6 \times 100 = 50\)\%.
\item
  The \emph{odds} of rolling an even number is \(3\div 3 = 1\).
\end{itemize}

\end{example}

\begin{example}[Probabilities]
\protect\hypertarget{exm:EventsAndProb}{}\label{exm:EventsAndProb}

Consider the probabilities of the events in Example~\ref{exm:Events}.

\begin{itemize}
\tightlist
\item
  The probability of rolling a \largedice{4} is~\(1/6\) (or about~\(0.1667\)).
\item
  The probability of rolling an odd number is~\(3/6\), or~\(1/2\) (or~\(0.5\)).
\item
  The probability of rolling a number larger than \largedice{2} is~\(4/6\), or~\(2/3\) (or about~\(0.6667\)).
\end{itemize}

\end{example}

\subsection{Relative frequency approach}\label{ProbRelFreq}

\index{Probability!relative frequency approach}

What is the probability that a newborn baby will be male? The sample space could be listed as: \emph{male} and \emph{non-male}. Since the sample space has two elements, the classical approach suggests the probability is \(1\div2 =  0.5\). However, this approach is appropriate \emph{only if} males and non-males are \emph{equally likely} to be born. But are they?

In 2021 in Australia, \(289\,603\) live births occurred, with \(148\,636\) male births, \(140\,944\) female births, and \(23\)~others (or `not stated'). The \emph{proportion} of males born in the 2021 sample is \(148\,636\div 289\,603 = 0.513\), or about~\(51.3\)\%. An \emph{estimate} of the probability that the next birth will be male is about~\(0.513\) (or~\(51.3\)\%), based on using past data.

This is the \emph{relative frequency} approach to calculating probabilities: based on past data. The relative frequency method can only ever produce an \emph{approximate} probability, as it is based on a limited number of past observations. An actual probability would require an infinite number of observations.

\begin{definition}[Relative frequency approach to probability]
\protect\hypertarget{def:RelativeFrequencyApproachToProbability}{}\label{def:RelativeFrequencyApproachToProbability}In the \emph{relative frequency approach to probability}, the probability of an event is \emph{approximately} the number of times the outcomes of interest has appeared in the past, divided by the number of `attempts' in the past. This produces an \emph{approximate} probability.
\end{definition}

\begin{example}[Relative frequency probability]
\protect\hypertarget{exm:RFProbability}{}\label{exm:RFProbability}Based on the earlier information, the \emph{odds} that a new baby will be a boy is \emph{approximately} \(0.513\div (1 - 0.513) = 1.053\)\index{Odds} (i.e., \(105.3\) boys per \(100\) girls). According to the \emph{Australian Bureau of Statistics} (ABS):

\begin{quote}
The sex ratio for all births registered in Australia generally fluctuates around~\(105.5\) male births per~\(100\) female births.
\end{quote}

This is close to the odds of~\(1.053\) found above.
\end{example}

\begin{importantBox}{iconmonstr-warning-8-240.png}
\emph{Probabilities} describe the likelihood that an event will occur \emph{before} the result is known. \emph{Odds} and \emph{proportions} can be used either \emph{before} or \emph{after} the result is known, provided the wording is correct.

For example, \emph{proportions} describe how often an event has occurred \emph{after} the result is known, and \emph{expected proportions} describe the likelihood that an event will occur in many repetitions \emph{before} the result is known.

\end{importantBox}

The following example may help explain.

\begin{example}[Probabilities, proportions and odds]
\protect\hypertarget{exm:ProbProportioOdds}{}\label{exm:ProbProportioOdds}\emph{Before} a fair coin is tossed:\index{Probability}\index{Proportions}\index{Odds}

\begin{itemize}
\tightlist
\item
  the \emph{probability} of throwing a \Heads~is \(1/2 = 0.5\) (or~\(50\)\%).
\item
  the \emph{expected proportion} of \Heads~for many coin tosses is~\(0.5\) (or \(50\)\%).
\item
  the \emph{odds} of throwing a \Heads~is \(1/1 = 1\).
\end{itemize}

If we have \emph{already} tossed a coin \(100\) times and found \(47\)~heads:

\begin{itemize}
\tightlist
\item
  the \emph{proportion} of \Heads~in the sample is \(47/100 = 0.47\) (or~\(47\)\%).
\item
  the odds that we \emph{threw} a \Heads~in the sample is \(47/53 = 0.887\).
\end{itemize}

The `probability that we just threw a \Heads' makes no sense, because the result is known.
\end{example}

\subsection{Subjective approach}\label{ProbSubjective}

\index{Probability!subjective approach}

Many probabilities cannot be computed using the classical or relative frequency approach; for example, what is the probability that your sporting team wins their next game? It may depend on how important you deem the injuries to key players, whether you think recent form is crucial, or if you believe in a substantial home ground advantage. In this case, only a \emph{subjective probability} can be given.

`Subjective' probabilities may be based on personal judgement or experience. They can also be given by considering some of the relevant issues that may impact the probability (and may, for example, be based on mathematical models that incorporate information from numerous inputs). Depending on how these other issues are considered and combined, different subjective probabilities may be given.

Weather forecasts are one example: they incorporate data from sea surface temperatures, local topography, air pressures, air temperatures and so on. Different models use different inputs, and may combine these inputs differently to produce different (subjective) forecast probabilities. Subjective probabilities are deductive probabilities (based on reasoning).

\begin{definition}[Subjective approach to probability]
\protect\hypertarget{def:SubjectiveApproachToProbability}{}\label{def:SubjectiveApproachToProbability}In the \emph{subjective approach to probability}, various factors are incorporated subjectively to determine the probability of an event occurring.
\end{definition}

\begin{example}[Subjective probability]
\protect\hypertarget{exm:SubjectiveProbElNino}{}\label{exm:SubjectiveProbElNino}During El Niño events, eastern Australia typically experiences drier-than-average winters and springs. The \emph{Australian Broadcasting Corporation}'s news website reported (on 23~May 2023) that the Australian \emph{Bureau of Meteorology} predicted a \(50\)\%~probability of an El Niño event in~2023, while the American \emph{National Oceanic and Atmospheric Administration} predicted a \(90\)\%~chance of an El Niño event in~2023.

Despite this, `{[}both{]} agencies are looking at the same part of the Pacific Ocean' to make their predictions. However, `the US and Australia base their probability on different criteria'. The probabilities are subjective probabilities, based on complex mathematical models.
\end{example}

\section{Independence of events}\label{Independence}

\index{Independence}

One important concept in probability is \emph{independence}. Two events are \emph{independent} if the probability of one event happening is the same, whether or not the other event has happened. For example, the probability of getting a \Heads~on a coin toss is the same whether you are sitting or not sitting: the result of the coin toss is \emph{independent} of whether you are seated.

\begin{definition}[Independence]
\protect\hypertarget{def:Independence}{}\label{def:Independence}Two events are \emph{independent} if the probability of one event is the same, whether or not the other event has happened.
\end{definition}

\begin{example}[Independence]
\protect\hypertarget{exm:IndependenceCards}{}\label{exm:IndependenceCards}Consider drawing two cards from a well-shuffled, standard pack (of \(52\)~cards), \emph{without} returning the first card. For the \emph{first} card, the sample space contains every card in the pack, and drawing any card is as equally likely as drawing any other. Since four cards are \textbf{Aces}, the probability of drawing an \textbf{Ace} on the first draw is \(4/52\) (using the classical approach).

If we drew an \textbf{Ace} for the first card, the probability of drawing an \textbf{Ace} for the \emph{second} card is \(3/51\) (\emph{three} \textbf{Aces} remain among the \(51\) remaining cards). Alternatively, if we \emph{don't} draw an \textbf{Ace} for the first card, the probability of drawing an \textbf{Ace} second time is \(4/51\) (\emph{four} \textbf{Aces} remain among the \(51\)~remaining cards).

That is, the probability of drawing an \textbf{Ace} for the second card \emph{depends} on whether an \textbf{Ace} was drawn for the first card. The two events `Drawing an \textbf{Ace} for the first card' and `Drawing an \textbf{Ace} for the second card' are \emph{not independent} events.
\end{example}

\begin{tipBox}{iconmonstr-info-6-240.png}
A `standard' pack of cards has \(52\)~cards, organised into four \emph{suits}: spades \(\spadesuit\), clubs \(\clubsuit\) (both black), hearts \(\heartsuit\) and diamonds \(\diamondsuit\) (both red).\index{Cards: standard pack} Each \emph{suit} has \(13\)~\emph{denominations}: \(2\), \(3\), \(4\), \(5\), \(6\), \(7\), \(8\), \(9\), \(10\), Jack~(J), Queen~(Q), King~(K), Ace~(A). The Ace, King, Queen and Jack cards are called \emph{picture cards}. (Most packs also contain two jokers, which are not considered part of a \emph{standard} pack.)

\end{tipBox}

\begin{importantBox}{iconmonstr-warning-8-240.png}
Random samples produce \emph{independent} units of analysis.\index{Sampling!random}\index{Units of analysis}

\end{importantBox}

\section{Conditional probability}\label{ConditionalProbability}

\index{Probability!conditional}

\emph{Conditional probability} refers to adjusting probabilities when extra information is known. For example, the probability of rolling a \largedice{1} is \(1/6\) using the classical approach, as the sample space has six equally-likely elements. However, if we are told that an \emph{odd number} is rolled, only three elements in the sample space need now be considered (rolls of \largedice{1}, \largedice{3}, \largedice{5}) rather than all six elements; other outcomes are impossible). So, the probability of rolling a \largedice{1} is \(1/3\). We say `the probability of rolling a \largedice{1}, \emph{given that the roll is an odd number}, is~\(1/3\)'.

\begin{example}[Conditional probability]
\protect\hypertarget{exm:ConditionalCards}{}\label{exm:ConditionalCards}Suppose someone draws a card from a pack of cards. The probability that the card is a~\(\clubsuit\) is \(13/52 = 1/4\), or~\(25\)\%.

However, if you are told that the card is a \emph{black} card, then the card must be either a~\(\clubsuit\) or~\(\spadesuit\). The probability that the card is a~\(\clubsuit\), \emph{given} that the card is black, is \(13/26 = 1/2\), or~\(50\)\%.
\end{example}

\begin{example}[Wearing sunglasses]
\protect\hypertarget{exm:Sunglasses}{}\label{exm:Sunglasses}\citet{data:Dexter2019:SunProtection} recorded the number of people at the foot of the Goodwill Bridge, Brisbane, who wore sunglasses between \(11\):\(30\)am to \(12\):\(30\)pm (Table~\ref{tab:SunglassesTableProb}). The probability of an observed person wearing sunglasses is \[
  \frac{126 + 123}{126 + 123 + 240 + 263} = 0.3311,
\] or about \(33.1\)\%.

Conditional probabilities can also be computed:

\begin{itemize}
\tightlist
\item
  \emph{if the observed person is female}, the probability that she is wearing sunglasses is \(126\div (240 + 126) = 0.3443\), or about~\(34.4\)\%.
\item
  \emph{if the observed person is male}, the probability that he is wearing sunglasses is \(123\div (263 + 123) = 0.3187\), or about~\(31.9\)\%.
\end{itemize}

These probabilities are close, but not exactly equal.

If the two events were \emph{independent}, then these two conditional probabilities would be the same: the probability of wearing sunglasses would be the same for females and males. In other words, the probability of wearing sunglasses did not depend on whether a female or a male was observed. We might say that wearing sunglasses is close to, but not exactly, independent of the sex of the person, in the \emph{sample}. We cannot be sure if wearing sunglasses is independent of the sex of the person in the \emph{population}.
\end{example}

\begin{table}
\centering
\caption{\label{tab:SunglassesTableProb}Females and males wearing sunglasses on the Goodwill Bridge, Brisbane.}
\centering
\fontsize{8}{10}\selectfont
\begin{tabular}[t]{lcc}
\toprule
\textbf{ } & \textbf{Female} & \textbf{Male}\\
\midrule
Not wearing sunglasses & $240$ & $263$\\
Wearing sunglasses & $126$ & $123$\\
\bottomrule
\end{tabular}
\end{table}

\section{Chapter summary}\label{ToolsProbabilitySummary}

A \emph{probability} is a number between \(0\) and \(1\) inclusive (or between~\(0\)\% and~\(100\)\% inclusive) that quantifies the likelihood of a certain event occurring. Three ways to think about probabilities are:

\begin{itemize}
\tightlist
\item
  the \emph{classical approach}, which requires all outcomes to be \emph{equally likely}.
\item
  the \emph{relative frequency} approach (which gives approximate probabilities).
\item
  the \emph{subjective approach} (deductive probabilities).
\end{itemize}

Two events are \emph{independent} if the probability of one event is the same, whether the other event has happened or not. Conditional probability incorporates extra information when the probability is computed.

\section{Quick review questions}\label{ToolsProbabilityQuickReview}

Suppose \emph{Event~\(A\)} is defined as `Rolling a \largedice{1} or a \largedice{2} on a fair die'. Also, suppose \emph{Event~\(B\)} is defined as `Rolling an even number'.

Are the following statements \emph{true} or \emph{false}?

\begin{enumerate}
\def\labelenumi{\arabic{enumi}.}
\item
  The best \emph{approach} to computing the probability of Event~\(A\) occurring is the \emph{classical} approach.\tightlist
\item
  The \emph{probability} of Event~\(A\) occurring is \(2/6\).
\item
  Rolling a \largedice{1} on the first roll is \emph{independent} of rolling a \largedice{1} on a second roll.
\item
  The \emph{probability} of~`\(A\) \textbf{and}~\(B\)' occurring is \(1/6\).
\item
  The \emph{probability} of~`\(A\) \textbf{or}~\(B\)' occurring is \(4/6\).
\item
  The \emph{probability} of `\textbf{not}~\(B\)' occurring is \(3/6\).
\item
  The \emph{odds} of `\textbf{not}~\(B\)' occurring is \(3/6\).
\item
  The probability of Event~\(B\) occurring, \emph{if} Event~\(A\) has already occurred, is \(1/2\).
\end{enumerate}

\section{Exercises}\label{ProbabilityExercises}

\hyperref[Answers]{Answers to odd-numbered exercises} are given at the end of the book.

\captionsetup{font=small}

\begin{exercise}
\protect\hypertarget{exr:ProbabilityMethod}{}\label{exr:ProbabilityMethod}

Which \emph{approach} is best used to estimate a probability in these situations?

\begin{enumerate}
\def\labelenumi{\arabic{enumi}.}
\tightlist
\item
  The probability that the stock market will rise next month.
\item
  The probability that a randomly-chosen person writes left-handed.
\end{enumerate}

\end{exercise}

\begin{exercise}
\protect\hypertarget{exr:ProbabilityMethodB}{}\label{exr:ProbabilityMethodB}

Which \emph{approach} is best used to estimate a probability in these situations?

\begin{enumerate}
\def\labelenumi{\arabic{enumi}.}
\tightlist
\item
  The probability that a \textbf{King} will be chosen from a pack of cards.
\item
  The probability that Paris receives more than~\(50\,\text{mm}\) of rain next May.
\end{enumerate}

\end{exercise}

\begin{exercise}
\protect\hypertarget{exr:ProbabilityAndOrNot}{}\label{exr:ProbabilityAndOrNot}

Consider drawing cards from a fair pack. \emph{Event~A} is `drawing a picture card', \emph{Event~B} is `drawing a \textbf{King} or \textbf{Ace}' and \emph{Event~C} is `drawing a \(\spadesuit\)'.

\begin{enumerate}
\def\labelenumi{\arabic{enumi}.}
\tightlist
\item
  What events are in `\(A\) \textbf{and}~\(B\)'? \tightlist
\item
  Compute the probability of `\(A\) \textbf{and}~\(B\)'.
\item
  What events are in `\(A\) \textbf{or}~\(B\)'? \tightlist
\item
  Compute the probability of `\(A\) \textbf{or}~\(B\)'.
\item
  What events are in `\(A\) \textbf{and}~\(C\)'? \tightlist
\item
  Compute the probability of `\(A\) \textbf{and}~\(C\)'.
\item
  What events are in `\textbf{not}~\(C\)'? \tightlist
\item
  Compute the probability of `\textbf{not}~\(C\)'.
\item
  Compute the probability of \(C\), if~\(A\) has already occurred.
\item
  Compute the probability of \(A\), if~\(C\) has already occurred.
\end{enumerate}

\end{exercise}

\begin{exercise}
\protect\hypertarget{exr:ProbabilityAndOrNot2}{}\label{exr:ProbabilityAndOrNot2}

Consider rolling a fair die. \emph{Event~A} is `rolling an \emph{even} number', \emph{Event~B} is `rolling an \emph{odd} number' and \emph{Event~C} is 'rolling a \largedice{2}.

\begin{enumerate}
\def\labelenumi{\arabic{enumi}.}
\tightlist
\item
  What events are in `\(A\) \textbf{and}~\(B\)'? \tightlist
\item
  Compute the probability of `\(A\) \textbf{and}~\(B\)'.
\item
  What events are in `\(A\) \textbf{or} \(B\)'? \tightlist
\item
  Compute the probability of `\(A\) \textbf{or}~\(B\)'.
\item
  What events are in `\(A\) \textbf{and}~\(C\)'? \tightlist
\item
  Compute the probability of `\(A\) \textbf{and}~\(C\)'.
\item
  What events are in `\textbf{not}~\(C\)'? \tightlist
\item
  Compute the probability of `\textbf{not}~\(C\)'.
\item
  Compute the probability of \(C\), if~\(A\) has already occurred.
\item
  Compute the probability of \(C\), if~\(B\) has already occurred.
\end{enumerate}

\end{exercise}

\begin{exercise}
\protect\hypertarget{exr:ProbabilityThreeEvents}{}\label{exr:ProbabilityThreeEvents}

Consider these three events about tossing two fair coins, say Coin~A and Coin~B: \emph{Event~1} is `toss a \Heads~on Coin~A'; \emph{Event~2} is `toss a \Tails~on Coin~A'; and \emph{Event~3} is `toss a \Heads~on Coin~B'.

\begin{enumerate}
\def\labelenumi{\arabic{enumi}.}
\tightlist
\item
  Are \emph{Event~1} and \emph{Event~2} independent events? \tightlist  
\item
  Are \emph{Event~1} and \emph{Event~3} independent events?
\item
  Compute the probability of \emph{Event~3}.
\item
  What is the probability of \emph{Event~3} occurring, if \emph{Event~1} has already occurred?
\item
  List the sample space for the random procedure.
\end{enumerate}

\end{exercise}

\begin{exercise}
\protect\hypertarget{exr:ProbabilityThreeEvents2}{}\label{exr:ProbabilityThreeEvents2}

Consider these three events about drawing one card from a fair pack: \emph{Event~1} is `draw a \textbf{Jack}'; \emph{Event~2} is `draw a \(\heartsuit\)'; and \emph{Event~3} is `draw a \(\clubsuit\)'.

\begin{enumerate}
\def\labelenumi{\arabic{enumi}.}
\tightlist
\item
  Compute the probability of \emph{Event~1}. \tightlist
\item
  Compute the probability of \emph{Event~1}, if \emph{Event~2} has occurred.
\item
  Compute the probability of \emph{Event~1}, if \emph{Event~2} has \emph{not} occurred.
\item
  Are \emph{Event~1} and \emph{Event~2} independent? Explain.
\item
  Compute the probability of \emph{Event~3}.
\item
  Compute the probability of \emph{Event~3}, if \emph{Event~2} has occurred.
\item
  Compute the probability of \emph{Event~3}, if \emph{Event~2} has \emph{not} occurred.
\item
  Are \emph{Event~2} and \emph{Event~3} independent? Explain.
\end{enumerate}

\end{exercise}

\begin{exercise}
\protect\hypertarget{exr:ProbabilityDie}{}\label{exr:ProbabilityDie}

Suppose I roll a standard six-sided die.

\begin{enumerate}
\def\labelenumi{\arabic{enumi}.}
\item
  What is the \emph{probability} that I will roll a number larger than \largedice{2}? \tightlist
\item
  What are the \emph{odds} of rolling a number smaller than \largedice{6}?
\item
  Suppose I toss a coin after rolling the die. Is the result from the coin toss \emph{independent} of what I rolled on the die?
\item
  What is the probability that I roll a number divisible by~\(2\) on the die?
\item
  What is the probability that I roll a number divisible by~\(2\) \textbf{and} divisible by~\(3\) on the die?
\item
  What is the probability of rolling a \largedice{2}, \emph{given} that the number is smaller than \largedice{4}?
\end{enumerate}

\end{exercise}

\begin{exercise}
\protect\hypertarget{exr:ProbabilityCards}{}\label{exr:ProbabilityCards}

Suppose you have a well-shuffled, standard pack of \(52\)~cards.

\begin{enumerate}
\def\labelenumi{\arabic{enumi}.}
\tightlist
\item
  What is the \emph{probability} that you will draw a \textbf{King}?
\item
  What are the \emph{odds} that you will draw a \textbf{King}?
\item
  What is the \emph{probability} that you will draw a picture card (\textbf{Ace}, \textbf{King}, \textbf{Queen} or \textbf{Jack})?
\item
  What are the \emph{odds} that you will draw a picture card (\textbf{Ace}, \textbf{King}, \textbf{Queen} or \textbf{Jack})?
\item
  Suppose I draw two cards from the pack. Are the events `Draw a \textbf{King} first' and `Draw a \textbf{Queen} second' independent events?
\item
  Suppose I draw one card from the pack (drawing the second without replacing the first), then roll a six-sided die. Are the events `Draw a \textbf{Jack} from the pack of cards' and `Roll a \largedice{5} on the die' independent events? Explain.
\item
  If I draw a picture card, what is the probability the card is a \textbf{King}?
\end{enumerate}

\end{exercise}

\begin{exercise}
\protect\hypertarget{exr:SampleSpaceCardsDiff}{}\label{exr:SampleSpaceCardsDiff}

Consider drawing a card from a standard, well-shuffled pack of cards. The first card is replaced, the pack reshuffled, and then a second card is drawn from the pack. The colour of the two cards is recorded (Black or Red).

\begin{enumerate}
\def\labelenumi{\arabic{enumi}.}
\tightlist
\item
  Write down the sample space.
\item
  Define Event~\(D\) as `the total number of black cards drawn, minus the total number of red cards drawn'. Find the probability that~\(D\) is zero.
\item
  Find the probability that~\(D\) is zero \textbf{or} one.
\item
  Is the colour of the card drawn first \emph{independent} of the colour of the card drawn second? Explain.
\end{enumerate}

\end{exercise}

\begin{exercise}
\protect\hypertarget{exr:SampleSpaceInfiniteCoins}{}\label{exr:SampleSpaceInfiniteCoins}

Consider the random process `tossing a coin'. Event~\(H\) is of interest: 'the number of tosses until the first \Heads~is thrown'.

\begin{cols}

\begin{col}{0.45\textwidth}

\begin{enumerate}
\def\labelenumi{\arabic{enumi}.}
\tightlist
\item
  What is the sample space?
\item
  Find the probability that \(H\) is one.
\end{enumerate}

\end{col}

\begin{col}{0.025\textwidth}
~

\end{col}

\begin{col}{0.50\textwidth}

\begin{enumerate}
\def\labelenumi{\arabic{enumi}.}
\setcounter{enumi}{2}
\tightlist
\item
  Find the probability that \(H\) is two.
\item
  Find the probability that \(H\) is one \textbf{or} two.
\end{enumerate}

\end{col}

\end{cols}

\end{exercise}

\begin{exercise}
\protect\hypertarget{exr:FirstNationStudents}{}\label{exr:FirstNationStudents}

\citet{mypaper:Dunn:GLM-IEE} tabulated information about Queensland school children (Table~\ref{tab:EdTable}).

\begin{enumerate}
\def\labelenumi{\arabic{enumi}.}
\tightlist
\item
  What is the probability that a randomly chosen student is a First Nations student?
\item
  What is the probability that a randomly chosen student is in a government school?
\item
  Is the sex of the student approximately independent of whether the student is a First Nations student, for students in government schools?
\item
  Is the sex of the student approximately independent of whether the student is a First Nations student, for students in non-government schools?
\item
  Is whether the student is a First Nations student approximately independent of the type of school, for female students?
\item
  Is whether the student is a First Nations student approximately independent of the type of school, for male students?
\item
  Based on the above, what can you conclude from the data?
\end{enumerate}

\end{exercise}

\begin{table}
\centering
\caption{\label{tab:EdTable}The number of First Nations and non-First Nations students in Queensland schools in 2019.}
\centering
\fontsize{8}{10}\selectfont
\begin{tabular}[t]{l>{\centering\arraybackslash}p{35mm}>{\centering\arraybackslash}p{35mm}}
\toprule
\textbf{ } & \textbf{Number of First Nations students} & \textbf{Number of non-First Nations students}\\
\midrule
\addlinespace[0.3em]
\multicolumn{3}{l}{\textit{Government schools}}\\
\hspace{1em}Females & $2\,540$ & $21\,219$\\
\hspace{1em}Males & $2\,734$ & $22\,574$\\
\addlinespace[0.3em]
\multicolumn{3}{l}{\textit{Non-government schools}}\\
\hspace{1em}Females & $\phantom{0}391$ & $\phantom{0}9\,496$\\
\hspace{1em}Males & $\phantom{0}362$ & $\phantom{0}9\,963$\\
\bottomrule
\end{tabular}
\end{table}

\begin{exercise}
\protect\hypertarget{exr:TwoWayTableProbs}{}\label{exr:TwoWayTableProbs}

\citet{data:kelishadi2017:snack} recorded whether Iranian children aged \(6\)--\(18\) years ate breakfast (Table~\ref{tab:SkipBreakfast2}). Find the \emph{probability} that a randomly chosen student is:

\begin{enumerate}
\def\labelenumi{\arabic{enumi}.}
\tightlist
\item
  A female student.
\item
  A female student who skipped breakfast.
\item
  A female student, \emph{if we already know} the child skipped breakfast.
\end{enumerate}

\end{exercise}

\begin{table}
\centering
\caption{\label{tab:SkipBreakfast2}The number of Iranian children aged $6$ to $18$ who skip and do not skip breakfast.}
\centering
\fontsize{8}{10}\selectfont
\begin{tabular}[t]{lcc>{}c}
\toprule
\textbf{ } & \textbf{Skips breakfast} & \textbf{Doesn't skip breakfast} & \textbf{Total}\\
\midrule
Females & $2\,383$ & $4\,257$ & \textbf{$\phantom{0}6\,640$}\\
Males & $1\,944$ & $4\,902$ & \textbf{$\phantom{0}6\,846$}\\
\midrule
\textbf{Total} & \textbf{$4\,327$} & \textbf{$9\,159$} & \textbf{\textbf{$13\,486$}}\\
\bottomrule
\end{tabular}
\end{table}

\begin{exercise}
\protect\hypertarget{exr:IndependentEvents}{}\label{exr:IndependentEvents}

Are these pairs of events likely to be \emph{independent} or \emph{not independent}? Explain.

\begin{enumerate}
\def\labelenumi{\arabic{enumi}.}
\tightlist
\item
  `I walk to work tomorrow morning', and `Rain is expected tomorrow morning'.
\item
  `A person smokes more than \(10\) cigarettes per week' and `A person gets lung cancer'.
\item
  `It rains today' and `I hose my garden today'.
\end{enumerate}

\end{exercise}

\begin{exercise}
\protect\hypertarget{exr:SensitivitySpecifity}{}\label{exr:SensitivitySpecifity}

In disease testing, two keys aspects of the test are:

\begin{itemize}
\tightlist
\item
  \emph{sensitivity}:\index{Sensitivity} the probability of a \emph{positive} test result among those \emph{with} the disease; and
\item
  \emph{specificity}:\index{Specificity} the probability of a \emph{negative} test result among those \emph{without} the disease.
\end{itemize}

Both are important for understanding how well a test works in practice. Ideally, a test would have high sensitivity and high specificity.

A certain test has a \emph{sensitivity} of~\(0.99\) and a \emph{specificity} of~\(0.98\). Consider a group of \(1\,000\)~people, \(100\) of whom (unknowingly) have the disease and \(900\) who (unknowingly) do not have the disease. All the people are given the test.

\begin{enumerate}
\def\labelenumi{\arabic{enumi}.}
\tightlist
\item
  Suppose the \(100\)~people who \emph{do} have a disease are tested. How many would be expected to return a positive test result?
\item
  Suppose the \(900\)~people who \emph{do not} have a disease are tested. How many would be expected to return a positive test result?
\item
  In total, how many positive tests would be expected from the \(1\,000\)~people?
\item
  Consider those people who return a positive test result. What is the probability that one of these people actually has the disease?
\end{enumerate}

\end{exercise}

\begin{exercise}
\protect\hypertarget{exr:CoinOutcomes}{}\label{exr:CoinOutcomes}

Explain \emph{why} the following argument is incorrect:

\begin{quote}
When I toss two coins, there are only three outcomes: a \textbf{Head} and a \textbf{Head}, a \textbf{Tail} and a \textbf{Tail}, or one of each. So the probability of obtaining two \textbf{Tails} must be one-third.
\end{quote}

\end{exercise}

\begin{exercise}
\protect\hypertarget{exr:AndrewsQuote}{}\label{exr:AndrewsQuote}On 13~October, 1997, the American television programme \emph{Nightline} interviewed Dr~Richard Andrews, director of California's \emph{Office of Emergency Services}, to discussed natural disasters being predicted. In the interview, Dr Andrews said (see \emph{Chance News} 6.12):

\begin{quote}
I listen to earth scientists talk about earthquake probabilities a lot and in my mind every probability is \(50\)--\(50\): either it will happen or it won't happen.
\end{quote}

Explain why Dr Andrews is incorrect when he says that `every probability is \(50\)--\(50\)'. Give an example to show why he must be incorrect.
\end{exercise}

\captionsetup{font=normalsize}

\begin{EOCanswerBox}{iconmonstr-check-mark-14-240.png}
\textbf{Answers to \emph{Quick review} questions:} \textbf{1.} True. \textbf{2.} True. \textbf{3.} True. \textbf{4.} True. \textbf{5.} True. \textbf{6.} True. \textbf{7.} False: odds are~\(1\). \textbf{8.} True.

\end{EOCanswerBox}

\chapter{Sampling variation}\label{SamplingVariation}

\begin{cols}
\begin{col}{0.52\textwidth}

\begin{objectivesBox}{iconmonstr-target-4-240.png}
So far, you have learnt to ask an RQ, design a study, describe and summarise the data, and understand probability.
\textbf{In this chapter}, you will learn to:

\begin{itemize}\tightlist
  \item
  explain what a sampling distribution describes.
  \item
  explain the difference between the variation between individuals and the variation in statistics.
  \item
  determine when a standard error is appropriate to use.
  \item
  explain the difference between standard errors and standard deviations.
\end{itemize}
\end{objectivesBox}

\end{col}

\begin{col}{0.03\textwidth}
~
\end{col}

\begin{col}{0.45\textwidth}

\includegraphics[width=0.95\linewidth]{19-Tools-SamplingVariation_files/figure-latex/unnamed-chunk-7-1} 
\end{col}
\end{cols}

\section{Introduction}\label{SamplingVariationIntro}

One of the most important ideas in research and statistics is that the sample being studied is only one of countless possible samples that could have been selected to study.

\begin{importantBox}{iconmonstr-warning-8-240.png}
Studying a sample leads to the following observations: \vspace{-2ex}

\begin{itemize}
\tightlist
\item
  every sample is likely to be different.
\item
  we observe just one of the many possible samples.
\item
  every sample is likely to yield a different value for the statistic (i.e., a different estimate\index{Estimate} for the parameter).
\item
  we observe just one of the many possible values for the statistic. \vspace{-2ex}
\end{itemize}

Since many values for the statistic are possible, the values of the statistic vary and have a distribution.

\end{importantBox}

In research, decisions need to be made about \emph{populations} based on \emph{samples}; that is, about \emph{parameters} based on \emph{statistics}. The challenge is that the decision must be made using only one of the many possible samples, and every sample is likely to be different. Each sample will produce a different value for the \emph{statistic}. This is called \emph{sampling variation}.

\begin{definition}[Sampling variation]
\protect\hypertarget{def:SamplingVariation}{}\label{def:SamplingVariation}\index{Sampling variation} \emph{Sampling variation} refers to how the sample estimates (statistics) vary from sample to sample, because every possible sample is different.
\end{definition}

Any distribution that describes how a statistic varies for all possible samples is called a \emph{sampling distribution}.

\begin{definition}[Sampling distribution]
\protect\hypertarget{def:SamplingDistribution}{}\label{def:SamplingDistribution}A \emph{sampling distribution} is the distribution of a statistic, showing how its value varies in all possible samples.
\end{definition}

\section{Sample proportions have a sampling distribution}\label{SamplingDistributionProportions}

Sample proportions, like all statistics, vary from sample to sample; that is, \emph{sampling variation} exists, so sample proportions have a \emph{sampling distribution}.

Consider a European roulette wheel (Fig.~\ref{fig:RouletteWheel}): a ball and wheel are spun, and the ball can land on any number on the wheel from~\(0\) to~\(36\) (inclusive). Using the classical approach to probability, the probability of the ball landing on an \emph{odd} number (an `\emph{odd-spin}') is \(p = 18/37 =  0.486\). This is the \emph{population proportion} (the parameter).

If the wheel is spun (say) \(15\)~times, the \emph{sample} proportion of odd-spins, denoted~\(\hat{p}\), will vary. But, \emph{how} does~\(\hat{p}\) vary from one set of~\(15\) spins to another set of \(15\)~spins? Can we describe \emph{how} the values of~\(\hat{p}\) vary across the possible samples?

\begin{figure}[hbtp]

{\centering \includegraphics[width=0.42\linewidth]{19-Tools-SamplingVariation_files/figure-latex/RouletteWheel-1} 

}

\caption{A European roulette wheel, with numbers\ $0$ to\ $36$. The ball landed on\ $32$.}\label{fig:RouletteWheel}
\end{figure}

Computer simulation can be used to demonstrate what happens if the wheel was spun, over and over again, for \(n = 15\) spins each time, and the proportion of odd-spins was recorded for each repetition. The proportion of odd spins~\(\hat{p}\) varies from sample to sample (sampling variation), as shown by the histogram (Fig.~\ref{fig:RouletteWheelHist}, top left panel). The \emph{shape} of the distribution is approximately bell shaped. We can see that, for many repetitions of \(15\)~spins, \(\hat{p}\) is rarely smaller than~\(0.2\), and rarely larger than~\(0.8\). That is, reasonable values to expect for~\(\hat{p}\) are between about~\(0.2\) and~\(0.8\).

If the wheel was spun (say) \(n = 25\) times (rather than~\(15\) times), \(\hat{p}\) again varies (Fig.~\ref{fig:RouletteWheelHist}, top right panel): the values of~\(\hat{p}\) vary from sample to sample. The same process can be repeated for many repetitions of (say) \(n = 100\) and \(n = 200\) spins (Fig.~\ref{fig:RouletteWheelHist}, bottom panels).

Notice that as the sample size~\(n\) increases, the variation in the sampling distribution gets smaller. With \(200\)~spins, for instance, observing a sample proportion smaller than~\(0.4\) or greater than~\(0.6\) seems highly unusual, but these are not uncommon at all for~\(15\) spins.

The sampling distributions can be described by a mean (called the \emph{sampling mean}) and a standard deviation (called the \emph{standard error}).

\begin{example}[Reasonable values for the sample proportion]
\protect\hypertarget{exm:ReasonableValuesp}{}\label{exm:ReasonableValuesp}Suppose a roulette wheel was spun \(100\)~times, and \(31\)~odd numbers were observed. The sample proportion is \(\hat{p} = 31/100 = 0.31\). From Fig.~\ref{fig:RouletteWheelHist} (bottom left panel), a sample proportion this low almost never occurs in a sample of~\(100\) rolls.

This is very unlikely to occur from a fair roulette wheel. Hence, we either observed something highly unusual, or the wheel is not fair (e.g., a problem exists with the wheel).
\end{example}

\begin{figure}[hbtp]

{\centering \includegraphics[width=1\linewidth]{19-Tools-SamplingVariation_files/figure-latex/RouletteWheelHist-1} 

}

\caption{Sampling distributions for the proportion of roulette wheel spins that show an odd number, for sets of rolls of varying sizes.}\label{fig:RouletteWheelHist}
\end{figure}

\begin{importantBox}{iconmonstr-warning-8-240.png}
The values of the sample proportion vary from sample to sample. The distribution of the possible values of the \emph{sample} proportion is called a \emph{sampling distribution}.

Under certain conditions, the sampling distribution of a sample proportion is described by an approximate bell-shaped distribution (formally called a \emph{normal distribution}). In general, the approximation gets better as the sample size increases. In addition, the possible values of~\(\hat{p}\) vary less as the sample size increases.

The mean of the sampling distribution is called the \emph{sampling mean}; the standard deviation of the sampling distribution is called the \emph{standard error} (Fig.~\ref{fig:StatisticVariesAcrossSamples}).

\end{importantBox}

\section{Sample means have a sampling distribution}\label{SamplingDistributionMeans}

The sample mean, like all statistics, varies from sample to sample; that is, \emph{sampling variation} exists, so sample means have a \emph{sampling distribution}.

Consider a European roulette wheel again (Sect.~\ref{SamplingDistributionProportions}). Rather than recording the sample proportion of odd-spins, suppose the \emph{sample mean} of the numbers was recorded. If the wheel is spun (say) \(15\)~times, the \emph{sample} mean of the spins~\(\bar{x}\) will vary from one set of \(15\)~spins to another. But \emph{how} does it vary?

Again, computer simulation can be used to demonstrate what could happen if the wheel was spun \(15\)~times, over and over again, and the mean of the numbers was recorded for each repetition. Clearly, the sample mean spin~\(\bar{x}\) can vary from sample to sample (sampling variation) for \(n = 15\) spins (Fig.~\ref{fig:RouletteWheelHistx}, top left panel).

When \(n = 15\), the sample mean~\(\bar{x}\) varies from sample to sample, and the \emph{shape} of the distribution again is approximately bell-shaped. If the wheel was spun more than \(15\)~times (say, \(n = 50\) times) something similar occurs (Fig.~\ref{fig:RouletteWheelHistx}, top right panel): the values of~\(\bar{x}\) vary from sample to sample, and the values have an approximate bell-shaped (normal) distribution. In fact, the values of~\(\bar{x}\) have a bell-shaped distribution for other numbers of spins also (Fig.~\ref{fig:RouletteWheelHistx}, bottom panels).

The sampling distributions can be described by a mean (called the \emph{sampling mean}) and a standard deviation (called the \emph{standard error}).

\begin{figure}[hbtp]

{\centering \includegraphics[width=1\linewidth]{19-Tools-SamplingVariation_files/figure-latex/RouletteWheelHistx-1} 

}

\caption{Sampling distributions for the mean of the numbers after a roulette wheel is spun a certain number of times.}\label{fig:RouletteWheelHistx}
\end{figure}

\begin{importantBox}{iconmonstr-warning-8-240.png}
The values of the sample mean vary from sample to sample. The distribution of the possible values of the \emph{sample} mean is called a \emph{sampling distribution}.

Under certain conditions, the sampling distribution of a sample mean is described by an approximate bell-shaped (or normal) distribution. In general, the approximation gets better as the sample size increases. In addition, the possible values of~\(\bar{x}\) vary less as the sample size increases.

The mean of the sampling distribution is called the \emph{sampling mean}; the standard deviation of the sampling distribution is called the \emph{standard error} (Fig.~\ref{fig:StatisticVariesAcrossSamples}).

\end{importantBox}

\begin{example}[Reasonable values for the sample mean]
\protect\hypertarget{exm:ReasonableValuesMean}{}\label{exm:ReasonableValuesMean}Suppose we spun a roulette wheel~\(100\) times, and the mean of the observed numbers was \(\bar{x} = 18.9\). From Fig.~\ref{fig:RouletteWheelHistx} (bottom left panel), a sample mean with this value does not look unusual at all; it would occur reasonably frequently. The evidence does not suggest a problem with the wheel.
\end{example}

As we have seen, each sample is likely to be different, so \emph{any} statistic is likely to vary from sample to sample. (The value of the \emph{population} parameter does not change.) This variation in the possible values of the observed sampling statistic is called \emph{sampling variation}.

\section{Sampling means and standard errors}\label{StandardErrors}

As seen in the previous two sections, the value of a sample statistic varies from sample to sample. The value of the sample statistic that is observed depends on which one of the countless samples happens to be observed

The possible values of the statistic that we could potentially observe have a \emph{distribution} (specifically, a \emph{sampling distribution}); see Fig.~\ref{fig:StatisticVariesAcrossSamples}. The \emph{mean} of this sampling distribution is called the \emph{sampling mean}. The sampling mean is the average value of all possible values of the statistic. Not all sampling distributions have a bell-shaped distribution.

\begin{definition}[Sampling mean]
\protect\hypertarget{def:SamplingMean}{}\label{def:SamplingMean}The \emph{sampling mean} is the mean of the sampling distribution of a statistic: the mean of the values of the statistic from all possible samples.
\end{definition}

The \emph{standard deviation} of this sampling distribution is called the \emph{standard error}. The standard error measures how the value of the statistic varies across all possible values of the statistic; see Fig.~\ref{fig:StatisticVariesAcrossSamples}. The standard error is a measure of how precisely the \emph{sample} statistic estimates the \emph{population} parameter. If every possible sample (of a given size) was found, and the statistic computed from each sample, the standard deviation of all these estimates is the \emph{standard error}.

\begin{definition}[Standard error]
\protect\hypertarget{def:StandardError}{}\label{def:StandardError}A \emph{standard error} is the standard deviation of the sampling distribution of a statistic: the standard deviation of the values of the statistic from all possible samples.
\end{definition}

\begin{figure}[hbtp]

{\centering \includegraphics[width=0.7\linewidth]{19-Tools-SamplingVariation_files/figure-latex/StatisticVariesAcrossSamples-1} 

}

\caption{Describing how the value of the sample statistic varies across all possible samples, when the sampling distribution has a normal distribution.}\label{fig:StatisticVariesAcrossSamples}
\end{figure}

Figures~\ref{fig:RouletteWheelHist} and~\ref{fig:RouletteWheelHistx} show that the variation in the values of the statistic get smaller for larger sample sizes. That is, the standard error gets \emph{smaller} as the sample sizes get \emph{larger}: sample statistics show less variation for larger~\(n\). This makes sense: \emph{larger} samples generally produce more precise estimates.\index{Precision} After all, that's the advantage of larger samples: all else being equal, larger samples produce more precise estimates (Sect.~\ref{PrecisionAccuracy}).

\begin{example}[Standard errors]
\protect\hypertarget{exm:StandardErrors}{}\label{exm:StandardErrors}In Fig.~\ref{fig:RouletteWheelHistx}, a sample of~\(250\) (i.e., \(250\)~spins) is unlikely to produce a sample mean larger than~\(20\), or smaller than~\(15\). However, in a sample of size~\(15\) (i.e., \(15\)~spins) sample means near~\(15\) and~\(20\) are quite commonplace.

In samples of size~\(100\), the variation in the mean spin is smaller than in samples of size~\(15\). Hence, the \emph{standard error} (the standard deviation of the sampling distributions) will be smaller for samples of size~\(250\) than for samples of size~\(15\).
\end{example}

For many sample statistics, \emph{the variation from sample to sample can be approximately described by a bell-shaped (normal) distribution} (the \emph{sampling distribution}) if certain conditions are met. Furthermore, the \emph{standard deviation of this sampling distribution is called the standard error}. The standard error is a special name given to the \emph{standard deviation} that describes the variation in the possible values of a statistic.

\begin{tipBox}{iconmonstr-info-6-240.png}
`Standard error' is an unfortunate label: it is not an \emph{error}, or even \emph{standard}. (For example, there is no such thing as a `\emph{non}-standard error'.)

\end{tipBox}

\section{Standard deviation and standard error}\label{standard-deviation-and-standard-error}

Even experienced researchers confuse the meaning and the usage of the terms \emph{standard deviation} and \emph{standard error} \citep{ko2014inappropriate}. Understanding the difference is important.

The \emph{standard deviation}, in general, quantifies the amount of variation in any quantity that varies. The \emph{standard error} only ever refers to the standard deviation that describes a sampling distribution.

Typically, in a research study, \emph{standard deviations} describe the variation in the individuals in a sample: how observations vary from \emph{individual to individual}. The \emph{standard error} is only used to describe how \emph{sample estimates} vary from sample to sample (i.e., to describe the precision of sample estimates).

The standard error \emph{is} a standard deviation, but specifically describes the variation in sampling distributions. \emph{Any} numerical quantity estimated from a sample (a \emph{statistic}) can vary from sample to sample, and so has sampling variation, a sampling distribution, and hence a standard error. (Not all sampling distributions are \emph{normal} distributions, however.)

\begin{importantBox}{iconmonstr-warning-8-240.png}
\emph{Any} quantity estimated from a sample (a statistic) has a standard error.

\end{importantBox}

\begin{tipBox}{iconmonstr-info-6-240.png}
The \emph{standard error} is often abbreviated to `SE' or `s.e.'. For example, the `standard error of the sample mean' is usually written \(\text{s.e.}(\bar{x})\), and the `standard error of the sample proportion' is usually written \(\text{s.e.}(\hat{p})\).

\end{tipBox}

Parameters do not vary from sample to sample, so \emph{do not} have a sampling distribution (or a standard error).

\section{Chapter summary}\label{SamplingVariationSummary}

A \emph{sampling distribution} describes how all possible values of a statistic vary from sample to sample. Under certain circumstances, the sampling distribution often can be described by a \emph{bell-shaped (or normal) distribution}. The standard deviation of this normal distribution is called a \emph{standard error}. The standard error is the name specifically given to the standard deviation that describes the variation in the statistic \emph{across all possible samples}.

\section{Quick review questions}\label{SamplingVariationQuickReview}

Are the following statements \emph{true} or \emph{false}?

\begin{enumerate}
\def\labelenumi{\arabic{enumi}.}
\item
  The phrase `the standard error of the population proportion' is illogical.\tightlist  
\item
  The sample size \emph{does not} have a standard error?
\item
  Sampling variation refers to how sample sizes vary.
\item
  Sampling distributions describe how parameters vary.
\item
  Statistics do not vary from sample to sample.
\item
  Populations are numerically summarised using parameters
\item
  The \emph{standard deviation} is a type of \emph{standard error} in a specific situation.
\item
  Sampling distributions are always \emph{normal} distributions.
\item
  Sampling variation measures the amount of variation in the individuals in the sample.
\item
  The standard error measures the size of the error when we use a sample to estimate a population.
\item
  In general, a standard deviation measures the amount of variation.
\end{enumerate}

\section{Exercises}\label{SamplingVariationExercises}

\hyperref[Answers]{Answers to odd-numbered exercises} are given at the end of the book.

\captionsetup{font=small}

\begin{exercise}
\protect\hypertarget{exr:StdErrorOrStdDeviationA}{}\label{exr:StdErrorOrStdDeviationA}

In the following scenarios, would a \emph{standard deviation} or a \emph{standard error} be the appropriate way to measure the amount of variation? Explain.

\begin{enumerate}
\def\labelenumi{\arabic{enumi}.}
\tightlist
\item
  Researchers are studying the spending habits of customers. They would like to measure the variation in the amount spent by shoppers per transaction at a supermarket.
\item
  Researchers are studying the time it takes for inner-city office workers to travel to work each morning. They would like to determine the precision with which their estimate (a mean of \(47\,\text{mins}\)) has been measured.
\end{enumerate}

\end{exercise}

\begin{exercise}
\protect\hypertarget{exr:StdErrorOrStdDeviationB}{}\label{exr:StdErrorOrStdDeviationB}

In the following scenarios, would a \emph{standard deviation} or a \emph{standard error} be the appropriate way to measure the amount of variation? Explain.

\begin{enumerate}
\def\labelenumi{\arabic{enumi}.}
\tightlist
\item
  A study examined the effect of taking a pain-relieving drug on children. The researchers want to describe the sample they used in the study, including a description of how the ages of the children in the study vary.
\item
  A study estimated the proportion of children aged under~\(14\) who owned a mobile (cell) phone. The researchers want to report this estimate, indicating the precision of that estimate.
\end{enumerate}

\end{exercise}

\begin{exercise}
\protect\hypertarget{exr:HasStandardErrorA}{}\label{exr:HasStandardErrorA}

Which of the following have a \emph{standard error}?

\begin{enumerate}
\def\labelenumi{\arabic{enumi}.}
\tightlist
\item
  The population proportion.
\item
  The sample median.
\item
  The sample IQR.
\end{enumerate}

\end{exercise}

\begin{exercise}
\protect\hypertarget{exr:HasStandardErrorB}{}\label{exr:HasStandardErrorB}

Which of the following have a \emph{standard error}?

\begin{enumerate}
\def\labelenumi{\arabic{enumi}.}
\tightlist
\item
  The sample standard deviation.
\item
  The population odds.
\item
  The sample odds ratio.
\end{enumerate}

\end{exercise}

\begin{exercise}
\protect\hypertarget{exr:RouletteWheelA}{}\label{exr:RouletteWheelA}

Consider spinning a European roulette wheel.

\begin{enumerate}
\def\labelenumi{\arabic{enumi}.}
\tightlist
\item
  Suppose the wheel was spun \(15\)~times (Fig.~\ref{fig:RouletteWheelHistx}, top left panel), and the mean spin was~\(22.1\). What would you conclude about the wheel?
\item
  Suppose the wheel was spun \(250\)~times (Fig.~\ref{fig:RouletteWheelHistx}, bottom right panel), and the mean spin was~\(22.1\). What would you conclude about the wheel?
\item
  Suppose the wheel was spun \(50\)~times (Fig.~\ref{fig:RouletteWheelHistx}, top right panel), and the mean spin was~\(22.1\). What would you conclude about the wheel?
\item
  Suppose the wheel was spun \(50\)~times (Fig.~\ref{fig:RouletteWheelHistx}, top right panel), and the mean spin was~\(24.0\). What would you conclude about the wheel?
\end{enumerate}

\end{exercise}

\begin{exercise}
\protect\hypertarget{exr:RouletteWheelB}{}\label{exr:RouletteWheelB}

Consider spinning a European roulette wheel.

\begin{enumerate}
\def\labelenumi{\arabic{enumi}.}
\tightlist
\item
  Suppose the wheel was spun \(15\)~times (Fig.~\ref{fig:RouletteWheelHist}, top left panel), and the proportion of spins showing an odd number was~\(0.44\). What would you conclude about the wheel?
\item
  Suppose the wheel was spun \(15\)~times (Fig.~\ref{fig:RouletteWheelHist}, top left panel), and the proportion of spins showing an odd number was~\(0.13\). What would you conclude about the wheel?
\item
  Suppose the wheel was spun \(15\)~times (Fig.~\ref{fig:RouletteWheelHist}, top left panel), and the proportion of spins showing an odd number was~\(0.65\). What would you conclude about the wheel?
\item
  Suppose the wheel was spun \(200\)~times (Fig.~\ref{fig:RouletteWheelHist}, bottom right panel), and the proportion of spins showing an odd number was~\(0.65\). What would you conclude about the wheel?
\end{enumerate}

\end{exercise}

\begin{exercise}
\protect\hypertarget{exr:QuoteStdError}{}\label{exr:QuoteStdError}A research article \citep{nagele2003misuse} made this statement:

\begin{quote}
\ldots{} authors often {[}incorrectly{]} use the standard error of the mean (SEM) to describe the variability of their sample\ldots{}
\end{quote}

What is wrong with using the standard error of the mean to describe the sample? How would you explain the difference between the \emph{standard error} and the \emph{standard deviation}?
\end{exercise}

\captionsetup{font=normalsize}

\begin{EOCanswerBox}{iconmonstr-check-mark-14-240.png}
\textbf{Answers to \emph{Quick review} questions:} \textbf{1.} True. \textbf{2.} True. \textbf{3.} False. \textbf{4.} False. \textbf{5.} False. \textbf{6.} True. \textbf{7.} False. \textbf{8.} False. \textbf{9.} False. \textbf{10.} False. \textbf{11.} True.

\end{EOCanswerBox}

\chapter{Models and normal distributions}\label{SamplingDistributions}

\begin{cols}
\begin{col}{0.52\textwidth}

\begin{objectivesBox}{iconmonstr-target-4-240.png}
So far, you have learnt to ask an RQ, design a study, describe and summarise the data, and understand sampling variation.
\textbf{In this chapter}, you will learn to:
\begin{itemize}\tightlist
  \item
  describe and draw normal distributions.
  \item
  use $z$-scores to compute probabilities related to normal distributions.
  \item
  work `backwards' from probabilities for normal distributions.
\end{itemize}
\end{objectivesBox}

\end{col}

\begin{col}{0.03\textwidth}
~
\end{col}

\begin{col}{0.45\textwidth}

\includegraphics[width=0.95\linewidth]{20-Tools-DistributionsAndModels_files/figure-latex/unnamed-chunk-9-1} 
\end{col}
\end{cols}

\section{Introduction}\label{DistributionsModelsIntro}

As seen in Chap.~\ref{SamplingVariation}, many different samples could be drawn from a population, and the value of the statistic varies from sample to sample. The challenge of research is that only one of these countless possible samples is observed. The distribution of possible values of the statistic that could be observed from all possible samples is a \emph{sampling distribution}.

\begin{importantBox}{iconmonstr-warning-8-240.png}
Remember: studying a sample leads to the following observations: \vspace{-2ex}

\begin{itemize}
\tightlist
\item
  every sample is likely to be different.
\item
  we observe just one of the many possible samples.
\item
  every sample is likely to yield a different value for the statistic.
\item
  we observe just one of the many possible values for the statistic. \vspace{-2ex}
\end{itemize}

Since many values for the statistic are possible, the possible values of the statistic vary (called \emph{sampling variation}) and have a \emph{distribution} (called a \emph{sampling distribution}).

\end{importantBox}

As seen in Chap.~\ref{SamplingVariation}, sampling distributions often have a \emph{normal distribution} (or bell-shaped distribution).\index{Distributions}\index{Normal distribution} That is, the normal distribution\index{Model} is often used to describe the \emph{sampling distribution}.\index{Sampling distribution} We now study normal distributions, as they appear in many places in research.

\section{Normal distributions: examples}\label{DistributionsExample}

\index{Normal distribution!examples for data}

In Chap.~\ref{SamplingVariation}, we saw that the proportion of odd spins in \(15\)~spins of a roulette wheel could vary; similarly, the mean spin from \(15\)~spins could vary (Fig.~\ref{fig:RouletteWheelHistPropMean}). In both cases, these sampling distributions had a rough \emph{normal distribution} shape. This is true for larger numbers of spins also (Figs.~\ref{fig:RouletteWheelHist} and~\ref{fig:RouletteWheelHistx}).

\begin{figure}[hbtp]

{\centering \includegraphics[width=0.9\linewidth]{20-Tools-DistributionsAndModels_files/figure-latex/RouletteWheelHistPropMean-1} 

}

\caption{Sampling distributions for the proportion of odd spins (left), and the mean of the numbers after $15$ roulette wheel spins (right) are approximate normal distributions. The solid lines are theoretical normal distributions.}\label{fig:RouletteWheelHistPropMean}
\end{figure}

The \emph{histograms} in Fig.~\ref{fig:RouletteWheelHistPropMean} are based on results from a limited number of simulations. The solid lines shown in Fig.~\ref{fig:RouletteWheelHistPropMean} are actual \emph{normal distributions}, and represent how the histogram might appear theoretically if we used an infinite number of simulations. The normal distributions are \emph{models}\index{Model} for what might occur in the \emph{population}, so normal distributions are also called \emph{normal models}. Since the models represent \emph{populations}, the mean of the model is denoted~\(\mu\) and the standard deviation is denoted~\(\sigma\).

A \emph{model}\index{Model} is a theoretical or ideal concept. A model skeleton isn't \(100\)\%~accurate and certainly not exactly like \emph{your} skeleton; nonetheless, it suitably approximates reality. None of us probably have a skeleton \emph{exactly} like the model, but the model is still useful and helpful. Likewise, a sampling distribution may not have \emph{exactly} a normal shape, but the model is still useful and helpful. The model is a way of describing a \emph{theoretical} distribution in the population. A model is a simple (but not overly simple) approximation to reality.

The histograms in Fig.~\ref{fig:RouletteWheelHistPropMean} are not \emph{exactly} normal distributions, but are very close to normal distributions, and certainly close enough for most purposes. Many, but not all, sampling distributions have approximate normal distributions.

Sampling distributions represent theoretical distributions of sample \emph{statistics}, not the distribution of sample \emph{data}. When the sampling distribution is a normal distribution, the mean of the distribution is called the \emph{sampling mean} and the standard deviation is called the \emph{standard error}.

Apart from their use in modelling theoretical sampling distributions, some quantitative variables have approximate normal distributions too, when the distribution of the data in the \emph{population} can be approximately modelled by a normal distribution.

\begin{example}[Normal distributions of data]
\protect\hypertarget{exm:NormalExamples}{}\label{exm:NormalExamples}Some quantitative variables have approximate normal distributions. Figure~\ref{fig:HistogramDBPPossums} (left panel) shows the diastolic blood pressure of \(398\)~Americans \citep{data:Willems1997:CHD, data:Schorling1997:smoking}. Figure~\ref{fig:HistogramDBPPossums} (right panel) shows the weight of \(83\)~male Leadbeater's possums \citep{data:Williams2022:Possums}.
\end{example}

\begin{figure}[hbtp]

{\centering \includegraphics[width=1\linewidth]{20-Tools-DistributionsAndModels_files/figure-latex/HistogramDBPPossums-1} 

}

\caption{Two normal distributions. Left: diastolic blood pressure of a sample of $398$ Americans. Right: the weight of a sample of $83$ male Leadbeater's possums.  The solid lines are the approximate normal model for the variable in the population.}\label{fig:HistogramDBPPossums}
\end{figure}

\section{Normal distributions and the 68--95--99.7 rule}\label{NormalDistribution}

Normal distributions have a shape that is symmetric about the mean, with a bell shape. Half the values are greater than the mean, and half the values are less than the mean. The total probability represented by a normal distribution is one (or~\(100\)\%). For example, every sample will produce a sample proportion between~\(0\) and~\(1\) and so is represented somewhere in Fig.~\ref{fig:RouletteWheelHistPropMean} (left panel); every American has a diastolic blood pressure and so is represented somewhere in Fig.~\ref{fig:HistogramDBPPossums} (left panel); every male Leadbeater's possum has a weight and so is represented somewhere in Fig.~\ref{fig:HistogramDBPPossums} (right panel).

In theory, no upper limit or lower limit exists for a variable modelled using a normal distribution. In practice, this is rarely true, but usually never presents a problem. Consider the normal distributions in Fig.~\ref{fig:HistogramDBPPossums}, for example. The normal distribution shown for the diastolic blood pressure (left panel) has no lower or upper limit in theory, but all practical values of diastolic blood pressure are captured by that part of the normal distribution shown. The normal distribution implies almost no-one has a diastolic blood pressure below~\(40\,\text{mm}\)~Hg or above~\(130\,\text{mm}\)~Hg.

One of the most important properties of normal distributions is the \emph{68--95--99.7 rule} (sometimes called the \emph{empirical rule}).

\begin{definition}[The $68$--$95$--$99.7$ rule]
\protect\hypertarget{def:EmpiricalRule}{}\label{def:EmpiricalRule}For any quantity modelled by a normal distribution:\index{68@$68$--$95$--$99.7$ rule}

\begin{itemize}
\tightlist
\item
  \emph{approximately}~\(68\)\% of values lie within~\(1\) standard deviation of the mean.
\item
  \emph{approximately}~\(95\)\% of values lie within~\(2\) standard deviations of the mean.
\item
  \emph{approximately}~\(99.7\)\% of values lie within~\(3\) standard deviations of the mean.
\end{itemize}

These properties are true for \emph{all} normal distributions, whatever the quantity, whatever the value of the mean, and whatever the value of the standard deviation (Fig.~\ref{fig:EmpiricalRuleDiagram}).
\end{definition}

\begin{figure}[hbtp]

{\centering \includegraphics[width=1\linewidth]{20-Tools-DistributionsAndModels_files/figure-latex/EmpiricalRuleDiagram-1} 

}

\caption{The $68$--$95$--$99.7$ rule. The shaded regions correspond to the central $68$\%, $95$\% and $99.7$\%.}\label{fig:EmpiricalRuleDiagram}
\end{figure}

\begin{example}[Heights of females]
\protect\hypertarget{exm:HeightsFemales}{}\label{exm:HeightsFemales}Suppose the heights of Australian adult females in the population can be \emph{modelled} with a normal distribution having a mean of \(\mu = 162\,\text{cm}\), and a standard deviation of \(\sigma = 7\,\text{cm}\), and follow a normal distribution (Fig.~\ref{fig:EmpiricalRuleHts}). Using the \(68\)--\(95\)--\(99.7\) rule, approximately~\(68\)\% of Australian women will be between \(162 - 7 = 155\,\text{cm}\) and \(162 + 7 = 169\,\text{cm}\) tall using this model. Similarly, approximately~\(95\)\% of Australian women will be between \(162 - (2\times 7) = 148\,\text{cm}\) and \(162 + (2\times 7) = 176\,\text{cm}\) tall.
\end{example}

\begin{figure}[hbtp]

{\centering \includegraphics[width=0.75\linewidth]{20-Tools-DistributionsAndModels_files/figure-latex/EmpiricalRuleHts-1} 

}

\caption{A model for the height of adult Australian females in the population.}\label{fig:EmpiricalRuleHts}
\end{figure}

These regions under the normal curve are probabilities, are often called areas, and are sometimes expressed as percentages.

\section{\texorpdfstring{Standardising (\(z\)-scores)}{Standardising (z-scores)}}\label{zScores}

\index{z@$z$-score|(}

Since the \(68\)--\(95\)--\(99.7\) rule (Def.~\ref{def:EmpiricalRule}) applies for all normal distributions, the percentages in the rule only depend on how many standard deviations~(\(\sigma\)) a value~(\(x\)) is from the mean~(\(\mu\)). This information can be used to learn more about how values are distributed in a normal distribution.

For example, suppose heights of Australian adult females can be modelled with a normal distribution having a mean of \(\mu = 162\,\text{cm}\), and a standard deviation of \(\sigma = 7\,\text{cm}\) (Example~\ref{exm:HeightsFemales}). Using this model, the proportion of Australian adult women \emph{taller} than~\(169\,\text{cm}\) can be determined.

From a picture (Fig.~\ref{fig:HtsExer1}, left panel), \(162 + 7 = 169\,\text{cm}\) is one standard deviation \emph{above} the mean. Since~\(68\)\% of values are within one standard deviation of the mean,~\(32\)\% are outside that range (some shorter; some taller). Hence,~\(16\)\% are taller than one standard deviation above the mean, so the answer is about~\(16\)\%. (Another~\(16\)\% are shorter than one standard deviation \emph{below} the mean, or less than \(162 - 7 = 155\,\text{cm}\) in height.)

Again, the percentages only depend on how many standard deviations~(\(\sigma\)) the value~(\(x\)) is from the mean~(\(\mu\)), and not the actual values of~\(\mu\) and~\(\sigma\).

\begin{figure}[hbtp]

{\centering \includegraphics[width=0.9\linewidth]{20-Tools-DistributionsAndModels_files/figure-latex/HtsExer1-1} 

}

\caption{Left: what proportion of Australian adult females are taller than $169\,\text{cm}$? Right: what proportion of Australian adult females are shorter than $148$\,\text{cm}?}\label{fig:HtsExer1}
\end{figure}

\begin{example}[The $68$--$95$--$99.7$ rule]
\protect\hypertarget{exm:HeightsExer2}{}\label{exm:HeightsExer2}Consider again the heights of Australian adult females. Using this model, what proportion are \emph{shorter} than~\(148\,\text{cm}\)?

Again, drawing a picture is helpful (Fig.~\ref{fig:HtsExer1}, right panel). Since \(162 - (2\times 7) = 148\), \(148\,\text{cm}\) is two standard deviations \emph{below} the mean. Since \(95\)\% of values are within two standard deviation of the mean,~\(5\)\% are outside that range (half smaller, half larger; see Fig.~\ref{fig:HtsExer1}, right panel), so that~\(2.5\)\% are \emph{shorter} than~\(148\,\text{cm}\). (Another~\(2.5\)\% are \emph{taller} than \(162 + 14 = 176\,\text{cm}\).)
\end{example}

Again, the percentages only depend on how many standard deviations~(\(\sigma\)) the value~(\(x\)) is from the mean~(\(\mu\)). The number of standard deviations that an observation is from the mean is called a \emph{\(z\)-score}. A \(z\)-score is computed using \[
   z = \frac{ x - \mu}{\sigma},
\] where~\(\sigma\) is the standard deviation quantifying the variation in the \(x\)-values. Converting values to \(z\)-scores is called \emph{standardising}.

\begin{definition}[$z$-score]
\protect\hypertarget{def:zScore}{}\label{def:zScore}A \emph{\(z\)-score} measures how many standard deviations a value~\(x\) is from the mean. In symbols: \begin{equation}
   z = \frac{x - \mu}{\sigma},
   \label{eq:zscores}
\end{equation} where~\(\mu\) is the mean of the distribution, and~\(\sigma\) is the standard deviation of the distribution (measuring the variation in the \(x\)-values).
\end{definition}

The \(z\)-score is also called the \emph{standardised value} or \emph{standard score}. Note that:

\begin{itemize}
\tightlist
\item
  \(z\)-scores are negative for observations \emph{below} the mean.
\item
  \(z\)-scores are positive for observations \emph{above} the mean.
\item
  \(z\)-scores have no units (that is, not measured in kg, or cm, etc.).
\end{itemize}

\begin{example}[$z$-scores]
\protect\hypertarget{exm:HeightsExer3}{}\label{exm:HeightsExer3}Consider the model for the heights of Australian adult females again. From earlier, the \(z\)-score for a height of~\(169\,\text{cm}\) is \[
   z = \frac{x-\mu}{\sigma} = \frac{169 - 162}{7} = 1,
\] one standard deviation \emph{above} the mean. Similarly, the \(z\)-score for a height of~\(148\,\text{cm}\) is \[
   z = \frac{x-\mu}{\sigma} = \frac{148 - 162}{7} = -2,
\] two standard deviations \emph{below} the mean.
\end{example}

\begin{example}[The $68$--$95$--$99.7$ rule]
\protect\hypertarget{exm:EmpiricalRuleZ}{}\label{exm:EmpiricalRuleZ}

Consider the model for the heights of Australian adult females: a normal distribution, mean \(\mu = 162\,\text{cm}\), standard deviation \(\sigma = 7\,\text{cm}\) (Fig.~\ref{fig:HtsEmpirical}). Using this model:

\begin{itemize}
\tightlist
\item
  a height of \(162\,\text{cm}\) is zero standard deviations from the mean: \(z = 0\).
\item
  \(155\,\text{cm}\) is one standard deviation \emph{below} the mean: \(z = -1\).
\item
  \(169\,\text{cm}\) is one standard deviation \emph{above} the mean: \(z = 1\).
\item
  \(148\,\text{cm}\) and \(176\,\text{cm}\) correspond to \(z = -2\) and \(z = 2\) respectively.
\item
  \(141\,\text{cm}\) and \(183\,\text{cm}\) correspond to \(z = -3\) and \(z = 3\) respectively.
\end{itemize}

\end{example}

\begin{figure}[hbtp]

{\centering \includegraphics[width=0.75\linewidth]{20-Tools-DistributionsAndModels_files/figure-latex/HtsEmpirical-1} 

}

\caption{The $68$--$95$--$99.7$ rule and the heights of Australian adult females.}\label{fig:HtsEmpirical}
\end{figure}

\section{\texorpdfstring{Approximating areas (percentages) using the \(68\)--\(95\)--\(99.7\) rule}{Approximating areas (percentages) using the 68--95--99.7 rule}}\label{ApproxProbs}

\index{Normal distribution!approximating percentages|(}

As seen above, the \(68\)--\(95\)--\(99.7\) rule can be used to approximate percentages under normal distributions. The rule can even be used for values that do not exactly align with~\(1\),~\(2\) or~\(3\) standard deviations from the mean.

Suppose again that heights of Australian adult females can be modelled with a normal distribution with a mean of \(\mu = 162\,\text{cm}\), and a standard deviation of \(\sigma = 7\,\text{cm}\) (Fig.~\ref{fig:HtsEmpirical}). To find the proportion of women \emph{shorter} than \(145\,\text{cm}\), first draw the situation (Fig.~\ref{fig:HtsExer3}). Proceeding as before, we ask `How many standard deviations from the mean is~\(145\,\text{cm}\)?' Using Equation~\eqref{eq:zscores}, \(145\,\text{cm}\) corresponds to a \(z\)-score of \begin{equation}
   z = \frac{145 - 162}{7} = -2.4285...
   \label{eq:zscore214}
\end{equation} which is about~\(2.43\) standard deviations \emph{below} the mean.

\begin{figure}[hbtp]

{\centering \includegraphics[width=0.75\linewidth]{20-Tools-DistributionsAndModels_files/figure-latex/HtsExer3-1} 

}

\caption{What proportion of Australian adult females are shorter than $145\,\text{cm}$?}\label{fig:HtsExer3}
\end{figure}

What percentage of observations are less than this \(z\)-score? This case is not covered by the \(68\)--\(95\)--\(99.7\) rule, though the rule can be used to make \emph{rough estimates}.

About~\(2.5\)\% of observations are less than~\(2\) standard deviations below the mean; that is, about~\(2.5\)\% of women are shorter than~\(148\,\text{cm}\). So the percentage of females shorter than~\(145\)\,\text{cm} (that is, even shorter than~\(148\,\text{cm}\), and so further into the tail of the distribution) will be \emph{smaller} than~\(2.5\)\%. While we don't know the probability exactly, it will be smaller than~\(2.5\)\%.

Percentages found this way are very approximate, but often sufficient. However, more accurate percentages are found using tables compiled for this very purpose (Appendices~\ref{ZTablesNEG} and \ref{ZTablesPOS}). We now learn to use these tables. \index{Normal distribution!approximating percentages|)}

\section{Exact areas (percentages) using tables}\label{ExactAreasUsingTables}

\index{Normal distribution!using tables|(}

Areas under normal distributions can be found using online tables, or hard copy tables., for any \(z\)-score. The online tables are easier to use, but only the hard-copy tables are explained in this book (see the online version of this book for the online tables, and instructions for using the online tables). The tables (Appendices~\ref{ZTablesNEG} and~\ref{ZTablesPOS}) work with \(z\)-scores to two decimal places, so consider the \(z\)-score from Sect.~\ref{ApproxProbs} as \(z = -2.43\).

Using Appendix~\ref{ZTablesNEG}, find \(-2.4\) in the \emph{left} margin of the table (Fig.~\ref{fig:TablesExampleLaTeX}), and find the second decimal place (in this case, \(3\)) in the \emph{top} margin of the table: where these intersect is the area (or probability) \emph{less than} the \(z\)-score of \(-2.43\); that is, the probability of finding a \(z\)-score less than \(z = -2.43\) is \(0.0075\), or about \(0.75\)\%.

\begin{figure}[hbtp]

{\centering \includegraphics[width=0.5\linewidth]{TablesExampleLaTeX4} 

}

\caption{Using the $z$-score tables. When $z = -2.43$, the area to the left is $0.0075$.}\label{fig:TablesExampleLaTeX}
\end{figure}

\begin{importantBox}{iconmonstr-warning-8-240.png}
Our tables always give the area to the \emph{left} of the \(z\)-score.

\end{importantBox}

Either the hard-copy or online tables gives an answer of~\(0.75\)\%. This is consistent with the rough answer using the \(68\)--\(95\)--\(99.7\) rule: a value less than~\(2.5\)\%. \index{Normal distribution!using tables|)}

\section{\texorpdfstring{Examples using \(z\)-scores}{Examples using z-scores}}\label{ZScoreForestry}

The general approach to computing probabilities from normal distributions is:

\begin{itemize}
\tightlist
\item
  \emph{draw a diagram}, and mark on the value(s) of interest.
\item
  \emph{shade} the required region of interest.
\item
  \emph{compute} the \(z\)-score(s) using Equation~\eqref{eq:zscores}.
\item
  \emph{use} the tables in Appendices~\ref{ZTablesNEG} and~\ref{ZTablesPOS} to compute corresponding areas (percentages).
\item
  \emph{deduce} the answer.
\end{itemize}

This approach can be used to answer more complicated questions involving normal distributions.

\begin{example}[Normal distributions]
\protect\hypertarget{exm:NormalTrees}{}\label{exm:NormalTrees}Mechanised forest harvesting systems were simulated by \citet{data:Aedo1997:softwood}, and the diameters of a specific type of tree were modelled using:

\begin{itemize}
\tightlist
\item
  a normal distribution, with
\item
  a mean of \(\mu = 8.8\) inches, and
\item
  a standard deviation of \(\sigma = 2.7\) inches.
\end{itemize}

Using this model, what is the probability that a randomly-chosen tree has a diameter \emph{greater} than~\(5\)~inches?

Following the steps identified earlier:

\begin{itemize}
\tightlist
\item
  \emph{draw} the appropriate normal curve, and mark on~\(5\)~inches (Fig.~\ref{fig:ZDBH1}, left panel).
\item
  \emph{shade} the region `greater than~\(5\)~inches' (Fig.~\ref{fig:ZDBH1}, centre panel).
\item
  \emph{compute} the \(z\)-score using Equation~\eqref{eq:zscores}: \(\displaystyle z = (5 - 8.8)/2.7 = -1.41\) to two decimal places.
\item
  \emph{use} tables: the probability of a tree diameter \emph{shorter} than \(5\)~inches is~\(0.0793\). (Remember: the tables always give area \emph{less} than the value of~\(z\).)
\item
  \emph{deduce} the answer (Fig.~\ref{fig:ZDBH1}, right panel): since the \emph{total} area under the normal distribution is one (or~\(100\)\%), the probability of a tree diameter \emph{greater} than~\(5\)~inches is \(1 - 0.0793 = 0.9207\), or about~\(92\)\%.
\end{itemize}

A randomly-chosen tree has a probability of~\(92\)\% of having a diameter \emph{greater} than~\(5\)~inches.
\end{example}

\begin{figure}[hbtp]

{\centering \includegraphics[width=1\linewidth]{20-Tools-DistributionsAndModels_files/figure-latex/ZDBH1-1} 

}

\caption{What proportion of tree diameters are greater than $5$ inches?}\label{fig:ZDBH1}
\end{figure}

\begin{importantBox}{iconmonstr-warning-8-240.png}
Our normal-distribution tables \emph{always} provide area to the \emph{left} of the \(z\)-score. Drawing a picture of the situation is important: it helps visualise getting the area requested from the area the tables provide. Remember: the \emph{total} area under the normal distribution is one (or~\(100\)\%).

\end{importantBox}

\begin{example}[Normal distributions]
\protect\hypertarget{exm:NormalTreesDiagrams}{}\label{exm:NormalTreesDiagrams}

These scenarios can be displayed on a diagram as shown in Fig.~\ref{fig:MatchDiagrams} (recall \(\mu = 8.8\) inches).

\begin{enumerate}
\def\labelenumi{\arabic{enumi}.}
\tightlist
\item
  Tree diameters between~\(3\)~and~\(5\)~inches: Diagram~A.
\item
  Tree diameters greater than~\(11\)~inches: Diagram~B.
\item
  Tree diameters between~\(5\)~and~\(11\)~inches: Diagram~C.
\item
  Tree diameters less than~\(11\)~inches: Diagram~D.
\end{enumerate}

\end{example}

\begin{figure}[hbtp]

{\centering \includegraphics[width=0.95\linewidth]{20-Tools-DistributionsAndModels_files/figure-latex/MatchDiagrams-1} 

}

\caption{Scenarios with their corresponding diagrams.}\label{fig:MatchDiagrams}
\end{figure}

\begin{example}[Normal distributions]
\protect\hypertarget{exm:NormalTrees2}{}\label{exm:NormalTrees2}Using the model for tree diameters in Example~\ref{exm:NormalTrees}, what is the probability that a tree has a diameter \emph{between}~\(5\)~and~\(11\)~inches?

First, \emph{draw} the situation, and \emph{shade} `between \(5\)~and \(10\)~inches' (Fig.~\ref{fig:MatchDiagrams}, Diagram~C). Then, \emph{compute} the \(z\)-scores for \emph{both} tree diameters:

\begin{itemize}
\tightlist
\item
  \makebox[17mm][l]{For $5$ inches:} \(\quad  z = (5 - 8.8)/2.7 = -1.41\) (i.e., below the mean).
\item
  \makebox[17mm][l]{For $11$ inches:} \(\quad z = (11 - 8.8)/2.7 = 0.81\) (i.e., above the mean).
\end{itemize}

The tables in Appendices~\ref{ZTablesNEG} and~\ref{ZTablesPOS} can then be used to find the area to the \emph{left} of \(z = -1.41\) (which is~\(0.0793\)), and also to find the area to the \emph{left} of \(z = 0.81\) (which is~\(0.791\)). However, neither of these provide the area \emph{between} \(z = -1.41\) and \(z = 0.81\).

Looking carefully at the areas from the tables and the area sought, the required area is the \emph{area} between the two \(z\)-scores (Fig.~\ref{fig:ZDBH3}): \(0.7910 - 0.0793 = 0.7117\). The probability that a tree has a diameter between~\(5\) and~\(11\)~inches is about~\(0.7117\), or about~\(71\)\%.
\end{example}

\index{z@$z$-score|)}

\begin{figure}[hbtp]

{\centering \includegraphics[width=1\linewidth]{20-Tools-DistributionsAndModels_files/figure-latex/ZDBH3-1} 

}

\caption{What proportion of tree diameters are between $5$ and $11$ inches? Left: the hatched area is the area to the left of $z = -1.41$. Right: the shaded area is the area to the left of $z = 0.81$. Neither give us the area we seek directly.}\label{fig:ZDBH3}
\end{figure}

\section{Unstandardising: working backwards}\label{Unstandardising}

\index{Normal distribution!using tables backwards}\index{Unstandardising formula}

Using the model for tree diameters in Example~\ref{exm:NormalTrees} again, different types of questions can be asked too. Suppose we needed to identify the diameters of the \emph{smallest}~\(3\)\% of trees.

This is a different type of problem than before; previously, the \emph{tree diameter} was known, so a \(z\)-score could be computed, and hence a probability (Fig.~\ref{fig:WorkingWithZ}). However, here the \emph{probability} is known, and a tree diameter is sought. That is, working `backwards' is necessary (Fig.~\ref{fig:WorkingWithZ}), so the \(z\)-tables need to be used `backwards' too.

\begin{figure}[hbtp]

{\centering \includegraphics[width=0.9\linewidth]{20-Tools-DistributionsAndModels_files/figure-latex/WorkingWithZ-1} 

}

\caption{Working with $z$-scores. In the tables, the areas (probabilities) are in the body of the table, and the $z$-scores are in the margins of the table.}\label{fig:WorkingWithZ}
\end{figure}

Drawing a rough diagram of the situation again is very helpful (Fig.~\ref{fig:DBHBackwards}). We can only mark the approximate location of the required score, but this is sufficient. Then, tables must be used to determine the corresponding \(z\)-score. Since the required value will be smaller than the mean, the \(z\)-score will be negative (to the \emph{left} of the mean).

\begin{figure}[hbtp]

{\centering \includegraphics[width=0.7\linewidth]{20-Tools-DistributionsAndModels_files/figure-latex/DBHBackwards-1} 

}

\caption{Tree diameters: the smallest\ $3$\% is shaded. The approximate location of the required $z$-score is drawn.}\label{fig:DBHBackwards}
\end{figure}

As before (Sect.~\ref{ExactAreasUsingTables}), online tables or hard copy tables can be used (and again the online tables are easier to use). Only the hard-copy tables are explained in this book (see the online version of this book for the online tables, and instructions for their use).

When the \(z\)-scores (in the \emph{margins} of the tables in Appendices \ref{ZTablesNEG} and \ref{ZTablesPOS}) were known, the \emph{areas} were found in the \emph{body} of the table. If the area, or probability (in the \emph{body} of the table) is known, the corresponding \(z\)-score can be found (in the \emph{margins} of the table). Using the hard copy tables (Fig.~\ref{fig:TablesExampleLaTeXBackwards}), locate the area of \(0.0300\) (or as close as possible) in the \emph{body} of the table, then read the \(z\)-score from the margins of the table.

\begin{figure}[hbtp]

{\centering \includegraphics[width=0.9\linewidth]{TablesExampleLaTeXBackwards3} 

}

\caption{Using the $z$-tables backwards. When $z = -1.88$, the area to the left is $0.0301$, which is the closest we can get to $0.03$ (or $3$\%).}\label{fig:TablesExampleLaTeXBackwards}
\end{figure}

Using hard copy tables, the closest value in the \emph{body} of the table to \(0.0300\) (or \(3\)\%) is \(0.0301\). This corresponds to a \(z\)-score of \(z = -1.88\) (from the \emph{margins} of the table). Sometimes, the exact area can be found in the body of the table, but often the closest value in the body of the table must be used. (The online tables give a slightly more precise value of \(z = -1.881\).)

\begin{importantBox}{iconmonstr-warning-8-240.png}
Our tables always give the area to the \emph{left} of the \(z\)-score.

\end{importantBox}

Using either the hard-copy or online tables, the appropriate \(z\)-value is about~\(-1.88\) standard deviations \emph{below} the mean; that is, \(z = -1.88\) (Fig.~\ref{fig:DBHBackwards}). The \(z\)-score can be converted to an observation value \(x\) using the \emph{unstandardising} formula:\footnote{This is found by re-arranging Equation~\eqref{eq:zscores}.} \[
    x = \mu + z\sigma.
\] Using this unstandardising formula: \begin{align*}
    x &= \mu + (z\times\sigma) \\
        &= 8.8 + (-1.88 \times 2.7) = 3.724;
\end{align*} that is, about \(3\)\% of trees have diameters less than about~\(3.72\) inches.

\begin{definition}[Unstandardising formula]
\protect\hypertarget{def:UnstandardisingFormula}{}\label{def:UnstandardisingFormula}When the \(z\)-score is known, the corresponding value of the observation~\(x\) is \begin{equation}
    x = \mu + z\sigma.
  \label{eq:UnstandardisingFormula}
\end{equation} This is called the \emph{unstandardising formula}.
\end{definition}

\begin{example}[Normal distributions backwards]
\protect\hypertarget{exm:LargestPC}{}\label{exm:LargestPC}Using the model for tree diameters in Example~\ref{exm:NormalTrees} again, suppose now the diameters of the \emph{largest}~\(25\)\% of trees needs to be identified.

The situation can be drawn (Fig.~\ref{fig:DBHBackwards2}). Since an area is given, we need to work `backwards', so the \(z\)-tables need to be used `backwards' too. The \emph{largest}~\(25\)\% implies large trees, so required diameter is larger than the mean (so corresponds to a positive \(z\)-score).

The tables work with the area to the \emph{left} of the value of interest, which is~\(75\)\% (Fig.~\ref{fig:DBHBackwards2}). Using either the hard-copy or online tables, the appropriate \(z\)-value is \(z = 0.674\). Then, the \(z\)-score can be converted to an observation value~\(x\) using the \emph{unstandardising} formula: \begin{align*}
    x &= \mu + (z\times\sigma) \\
        &= 8.8 + (0.674 \times 2.7) = 10.621.
\end{align*} That is, about~\(25\)\% of trees have diameters larger than about~\(10.6\)~inches.
\end{example}

\begin{figure}[hbtp]

{\centering \includegraphics[width=0.65\linewidth]{20-Tools-DistributionsAndModels_files/figure-latex/DBHBackwards2-1} 

}

\caption{Tree diameters: the largest $25$\% is the same as the smallest $75$\%.}\label{fig:DBHBackwards2}
\end{figure}

\section{Example: methane production}\label{example-methane-production}

\citet{huhtanen2016effects} modelled the retention time of food in sheep, using a normal distribution with the mean retention time as \(\mu = 42.5\,\text{h}\), and the standard deviation as \(\sigma = 3.68\,\text{h}\). We can draw this normal distribution (Fig.~\ref{fig:RetentionTime}), and then apply the \(68\)--\(95\)--\(99.7\) rule:

\begin{itemize}
\tightlist
\item
  about~\(68\)\% of retention times are between~\(38.82\) and~\(46.18\,\text{h}\).
\item
  about~\(95\)\% of retention times are between~\(35.14\) and~\(49.86\,\text{h}\).
\item
  about~\(99.7\)\% of retention times are between~\(31.46\) and~\(53.54\,\text{h}\).
\end{itemize}

\begin{figure}[hbtp]

{\centering \includegraphics[width=0.65\linewidth]{20-Tools-DistributionsAndModels_files/figure-latex/RetentionTime-1} 

}

\caption{Retention times of food in sheep.}\label{fig:RetentionTime}
\end{figure}

\begin{example}[Working with the normal distribution]
\protect\hypertarget{exm:Methane1}{}\label{exm:Methane1}Using this model, what proportion of sheep have a retention time \emph{less than} \(40\,\text{h}\)?
\end{example}

A retention time of~\(40\,\text{h}\) corresponds to a \(z\)-score of (Fig.~\ref{fig:RetentionPlots}, top left panel): \[
   z = \frac{40 - 42.5}{3.68} = -0.68.
\] This is a \emph{negative} number, since \(40\,\text{h}\) is \emph{below} the mean. Using the tables in Appendices~\ref{ZTablesNEG} and~\ref{ZTablesPOS} (that give the \emph{area to the left} of the \(z\)-score), the area to the left of \(z = -0.68\) is~\(0.2483\), or about~\(24.8\)\%. About~\(24.8\)\% of sheep have a retention time \emph{less} than~\(40\,\text{h}\).

\begin{example}[Working with the normal distribution]
\protect\hypertarget{exm:Methane2}{}\label{exm:Methane2}What proportion of sheep have a retention time \emph{greater than}~\(48\,\text{h}\) (two days)?
\end{example}

A retention time of~\(48\,\text{h}\) corresponds to a \(z\)-score of~\(1.49\). Using the normal distribution tables, the area to the \emph{left} of this \(z\)-score is~\(0.9319\), so the area to the \emph{right} of this \(z\)-score is~\(0.0681\) (Fig.~\ref{fig:RetentionPlots}, top right panel).

\begin{example}[Working with the normal distribution]
\protect\hypertarget{exm:Methane3}{}\label{exm:Methane3}What proportion of sheep have a retention time \emph{between}~\(40\) and~\(48\,\text{h}\)?
\end{example}

\begin{figure}[hbtp]

{\centering \includegraphics[width=1\linewidth]{20-Tools-DistributionsAndModels_files/figure-latex/RetentionPlots-1} 

}

\caption{Plots for retention times in sheep.}\label{fig:RetentionPlots}
\end{figure}

A retention time of~\(40\,\text{h}\) corresponds to \(z = -0.68\) and, using the normal distribution tables, the area to the \emph{left} of \(z = -0.68\) is~\(0.2483\) (Fig.~\ref{fig:RetentionPlots}, bottom left panel; hatched area). But this is not the area that we seek. From earlier, the area to the \emph{left} of \(z = 1.49\) is~\(0.9319\) (Fig.~\ref{fig:RetentionPlots}, bottom left panel; shaded region). But this is not the area we seek either. From the two areas that we know, we \emph{can} find the area that we seek (Fig.~\ref{fig:RetentionPlots}, bottom left panel):

\begin{itemize}
\tightlist
\item
  \(48\,\text{h}\) corresponds to \(z = 1.49\); the area to the \emph{left} of this \(z\)-score is~\(0.9319\).
\item
  \(40\,\text{h}\) corresponds to \(z = -0.68\); the area to the \emph{left} of this \(z\)-score is~\(0.2483\).
\item
  the \emph{difference} between these two \emph{areas} is sought, which is \(0.9319 - 0.2483 = 0.6836\).
\end{itemize}

So the proportion is about~\(0.684\) (or~\(68.4\)\%).

\begin{example}[Working with the normal distribution]
\protect\hypertarget{exm:Methane4}{}\label{exm:Methane4}Consider the~\(35\)\% of sheep with the \emph{shortest} retention times. What are these retention times?
\end{example}

The time we seek must be \emph{smaller} than the mean if it defines the \emph{shortest}~\(35\)\% of retention times. We don't know \emph{exactly} where to draw the retention time that this corresponds to on the diagram; it's just somewhere to the left of the mean (Fig.~\ref{fig:RetentionPlots}, bottom right panel).

This time, \emph{we know the area to the left}, but we do not know the value (or \(z\)-score). This a `backwards problem', and we need to find the \(z\)-score `backwards' (Sect.~\ref{Unstandardising}). From the hard copy tables, a \(z\)-score of \(z = -0.39\) has an area to the left of~\(0.3483\), which is as close as we can get. (The online tables are more precise: \(z = -0.385\).)

We know the \(z\)-score, so the retention value is found using the unstandardising formula: \[
  x = \mu + (z \times \sigma) 
    =  42.5 + (-0.385\times 3.68) = 41.0832.
\] The retention time is about~\(41.1\,\text{h}\).

\section{Chapter summary}\label{DistributionModelsSummary}

A \emph{model} is a way of describing the theoretical distribution of some quantitative quantity. One common model is a \emph{normal model} or \emph{normal distribution}, which is a bell-shaped distribution with a theoretical mean~\(\mu\) and a theoretical standard deviation~\(\sigma\). Probabilities can be computed from normal distributions using \emph{\(z\)-scores}, the \(68\)--\(95\)--\(99.7\) rule, or tables.

\section{Quick review questions}\label{DistributionModelsQuickReview}

Consider again the model for tree diameters in Example~\ref{exm:NormalTrees} \citep{data:Aedo1997:softwood}: a normal distribution with \(\mu = 8.8\) inches, and \(\sigma = 2.7\) inches.

Are the following statements \emph{true} or \emph{false}?

\begin{enumerate}
\def\labelenumi{\arabic{enumi}.}
\item
  A tree diameter of \(10.2\)~inches corresponds to a \(z\)-score of \((10.2 - 8.8)/2.7 = 0.519\). \tightlist  
\item
  The probability that a tree has a diameter \emph{less} than~\(10.2\)~inches is about~\(0.70\).
\item
  The probability that a tree has a diameter \emph{greater} than~\(10.2\)~inches is about~\(0.70\).
\item
  A tree diameter of \(6\)~inches corresponds to a \(z\)-score of \(1.04\).
\item
  The probability that a tree has a diameter \emph{less} than \(6\)~inches is \(0.15\).
\item
  The probability that a tree has a diameter \emph{greater} than \(6\)~inches is \(0.85\).
\end{enumerate}

\section{Exercises}\label{SamplingDistributionsExercises}

\hyperref[Answers]{Answers to odd-numbered exercises} are given at the end of the book.

\captionsetup{font=small}

\begin{exercise}
\protect\hypertarget{exr:Statements}{}\label{exr:Statements}

Are the following statements \emph{true} or \emph{false}?

\begin{enumerate}
\def\labelenumi{\arabic{enumi}.}
\item
  The unstandardising formula can be used to compute probabilities. \tightlist
\item
  About~\(68\)\% of observations are within two standard deviations of the mean.
\item
  Positive \(z\)-scores correspond to values larger than the mean.
\item
  A \(z\)-score tells us how many standard deviations a value is away from the mean.
\end{enumerate}

\end{exercise}

\begin{exercise}
\protect\hypertarget{exr:StatementsB}{}\label{exr:StatementsB}

Are the following statements \emph{true} or \emph{false}?

\begin{enumerate}
\def\labelenumi{\arabic{enumi}.}
\item
  A \(z\)-score larger than~\(4\) is impossible.
\item
  A \(z\)-score of zero is located at the mean value of the population.
\item
  About~\(5\)\% of observations are less than two standard deviations below the mean.
\item
  A \(z\)-score of zero means a calculation error has been made.
\end{enumerate}

\end{exercise}

\begin{exercise}
\protect\hypertarget{exr:BasiczA}{}\label{exr:BasiczA}

Determine the probability that an observation is \emph{less} than the following \(z\)-scores.

\begin{cols}

\begin{col}{0.4\textwidth}

\begin{enumerate}
\def\labelenumi{\arabic{enumi}.}
\tightlist
\item
  \(z = 1.84\).
\item
  \(z = -2.09\).
\end{enumerate}

\end{col}

\begin{col}{0.05\textwidth}
~

\end{col}

\begin{col}{0.5\textwidth}

\begin{enumerate}
\def\labelenumi{\arabic{enumi}.}
\setcounter{enumi}{2}
\tightlist
\item
  \(z = -5.34\).
\item
  \(z = 4.25\)
\end{enumerate}

\end{col}

\end{cols}

\end{exercise}

\begin{exercise}
\protect\hypertarget{exr:BasiczB}{}\label{exr:BasiczB}

Determine the probability that an observation is \emph{greater} than the following \(z\)-scores.

\begin{cols}

\begin{col}{0.4\textwidth}

\begin{enumerate}
\def\labelenumi{\arabic{enumi}.}
\tightlist
\item
  \(z = -0.48\).
\item
  \(z = 1.03\).
\end{enumerate}

\end{col}

\begin{col}{0.05\textwidth}
~

\end{col}

\begin{col}{0.5\textwidth}

\begin{enumerate}
\def\labelenumi{\arabic{enumi}.}
\setcounter{enumi}{2}
\tightlist
\item
  \(z = -4.00\).
\item
  \(z = 0.00\)
\end{enumerate}

\end{col}

\end{cols}

\end{exercise}

\begin{exercise}
\protect\hypertarget{exr:SamplingDistributionsGrowthChartA}{}\label{exr:SamplingDistributionsGrowthChartA}Growth charts released by the \emph{World Health Organisation} \citep{who2006length} showed that girls aged five-years-old with a height of \(100\,\text{cm}\) are said to have a \(z\)-score of \(z = –2\). What does this mean?
\end{exercise}

\begin{exercise}
\protect\hypertarget{exr:SamplingDistributionsGrowthChartB}{}\label{exr:SamplingDistributionsGrowthChartB}Growth charts released by the \emph{World Health Organisation} \citep{who2006length} showed that girls aged five-years old with a height of \(120\,\text{cm}\) are said to have a \(z\)-score of \(z = +2\). What does this mean?
\end{exercise}

\begin{exercise}
\protect\hypertarget{exr:SamplingDistributionsIQForwards}{}\label{exr:SamplingDistributionsIQForwards}

IQ scores are designed to have a mean of~\(100\) and a standard deviation of~\(15\). Match the diagram in Fig.~\ref{fig:IQMatchDiagramsForwards} with the meaning.

\begin{cols}

\begin{col}{0.4\textwidth}

\begin{enumerate}
\def\labelenumi{\arabic{enumi}.}
\tightlist
\item
  IQs greater than~\(110\).
\item
  IQs between~\(90\) and~\(115\).
\end{enumerate}

\end{col}

\begin{col}{0.05\textwidth}
~

\end{col}

\begin{col}{0.5\textwidth}

\begin{enumerate}
\def\labelenumi{\arabic{enumi}.}
\setcounter{enumi}{2}
\tightlist
\item
  IQs less than~\(110\).
\item
  IQs greater than~\(85\).
\end{enumerate}

\end{col}

\end{cols}

\end{exercise}

\begin{figure}[hbtp]

{\centering \includegraphics[width=1\linewidth]{20-Tools-DistributionsAndModels_files/figure-latex/IQMatchDiagramsForwards-1} 

}

\caption{Match the diagram with the description.}\label{fig:IQMatchDiagramsForwards}
\end{figure}

\begin{exercise}
\protect\hypertarget{exr:SamplingDistributionsIQBackwards}{}\label{exr:SamplingDistributionsIQBackwards}

IQ scores are designed to have a mean of~\(100\) and a standard deviation of~\(15\). Match the diagram in Fig.~\ref{fig:IQMatchDiagramsBackwards} with the meaning.

\begin{cols}

\begin{col}{0.4\textwidth}

\begin{enumerate}
\def\labelenumi{\arabic{enumi}.}
\tightlist
\item
  The \emph{largest}~\(25\)\% of IQ scores.
\item
  The \emph{smallest}~\(10\)\% of IQ scores.
\end{enumerate}

\end{col}

\begin{col}{0.05\textwidth}
~

\end{col}

\begin{col}{0.5\textwidth}

\begin{enumerate}
\def\labelenumi{\arabic{enumi}.}
\setcounter{enumi}{2}
\tightlist
\item
  The \emph{largest}~\(70\)\% of IQ scores.
\item
  The \emph{smallest}~\(60\)\% of IQ scores.
\end{enumerate}

\end{col}

\end{cols}

\end{exercise}

\begin{figure}[hbtp]

{\centering \includegraphics[width=1\linewidth]{20-Tools-DistributionsAndModels_files/figure-latex/IQMatchDiagramsBackwards-1} 

}

\caption{Match the diagram with the description.}\label{fig:IQMatchDiagramsBackwards}
\end{figure}

\begin{exercise}
\protect\hypertarget{exr:SamplingDistributionsEmpiricalA}{}\label{exr:SamplingDistributionsEmpiricalA}The \(68\)--\(95\)--\(99.7\) rule states that \emph{approximately}~\(68\)\% of observations are within one standard deviation of the mean. Use the tables in Appendices~\ref{ZTablesNEG} and \ref{ZTablesPOS} to compute a more precise value for the percentage of observations within one standard deviation of the mean. Comment.
\end{exercise}

\begin{exercise}
\protect\hypertarget{exr:SamplingDistributionsEmpiricalB}{}\label{exr:SamplingDistributionsEmpiricalB}The \(68\)--\(95\)--\(99.7\) rule states that \emph{approximately}~\(95\)\% of observations are within two standard deviations of the mean. Use the tables in Appendices~\ref{ZTablesNEG} and \ref{ZTablesPOS} to compute a more precise value for the percentage of observations within two standard deviations of the mean. Comment.
\end{exercise}

\begin{exercise}
\protect\hypertarget{exr:SamplingDistributionsTrees}{}\label{exr:SamplingDistributionsTrees}

Consider again the study by \citet{data:Aedo1997:softwood} (Example~\ref{exm:NormalTrees}), who studied the diameter of trees in certain forests. The tree diameters can be modelled as having a normal distribution, with a mean of \(\mu = 8.8\) inches, and a standard deviation of \(\sigma = 2.7\) inches. Using this model, answer these questions.

\begin{enumerate}
\def\labelenumi{\arabic{enumi}.}
\tightlist
\item
  What is the probability that a tree will have a diameter \emph{less than}~\(8\)~inches?
\item
  What is the probability that a tree will have a diameter \emph{greater than}~\(9\)~inches?
\item
  What is the probability that a tree will have a diameter \emph{between}~\(7\) and~\(10\)~inches?
\item
  The largest~\(15\)\% of trees have what diameters?
\item
  The smallest~\(25\)\% of trees have what diameters?
\end{enumerate}

\end{exercise}

\begin{exercise}
\protect\hypertarget{exr:CornSeeds}{}\label{exr:CornSeeds}

\citet{pasha2016effect} simulated methods for coating corn seeds (with fertiliser and crop protection chemicals, etc.). The seed diameter was modelled with a normal distribution, with mean~\(7.5\,\text{mm}\) and standard deviation of~\(0.225\,\text{mm}\). Using this model, answer these questions.

\begin{enumerate}
\def\labelenumi{\arabic{enumi}.}
\item
  What is the probability that a seed has a diameter of more than \(8\,\text{mm}\)?\tightlist  
\item
  What is the probability that a seed has a diameter less than \(7.1\,\text{mm}\)?
\item
  What is the probability that a seed has a diameter between \(7.5\) and \(8\,\text{mm}\)?
\item
  What is the diameter of the smallest \(30\)\% of seeds?
\item
  What is the diameter of the largest \(90\)\% of the seeds?
\end{enumerate}

\end{exercise}

\begin{exercise}
\protect\hypertarget{exr:SamplingDistributionsGestationLength}{}\label{exr:SamplingDistributionsGestationLength}

\citet{snowden2018causal} studied factors influencing preterm births. They modelled the gestation length of healthy babies with a normal distribution, having a mean of~\(40\)~weeks, and a standard deviation of~\(1.64\)~weeks. Using this model, answer these questions.

\begin{enumerate}
\def\labelenumi{\arabic{enumi}.}
\tightlist
\item
  What proportion of births are \emph{longer} than \(39\)~weeks (that is, nine months)?
\item
  In Australia, a premature birth is defined as a birth occurring before \(37\) weeks. What proportion of births are expected to be premature?
\item
  According to \emph{Health Direct}, `Babies born between~\(32\) and~\(37\) weeks may need care in a special care nursery'. What proportion of healthy births would be expected to be born between~\(32\) and~\(37\) weeks gestation?
\item
  How long is the gestation length for the \emph{longest}~\(5\)\% of pregnancies?
\item
  How long is the gestation length for the \emph{shortest}~\(10\)\% of pregnancies?
\end{enumerate}

\end{exercise}

\begin{exercise}
\protect\hypertarget{exr:SamplingDistributionsBridgesTrucks}{}\label{exr:SamplingDistributionsBridgesTrucks}

A new method for evaluating bridge loads \citep{obrien2018probabilistic} used a simulation to compare the new method to an existing method. For the simulation, they modelled the gross vehicle mass (GVM) of trucks as having a normal distribution, with a mean of~\(13\)~tonnes and a standard deviation of~\(1.3\)~tonnes.

The Isuzu F-Series trucks in 2025 were rated as having a GVM between~\(10.7\) and~\(26.0\) tonnes (depending on the configuration).

\begin{enumerate}
\def\labelenumi{\arabic{enumi}.}
\tightlist
\item
  What is the \(z\)-score for the lower limit of~\(10.7\) tonnes?
\item
  What is the \(z\)-score for the upper limit of~\(26.0\) tonnes?
\item
  What does a negative \(z\)-score mean in this context?
\end{enumerate}

\end{exercise}

\begin{exercise}
\protect\hypertarget{exr:SamplingDistributionsIQs}{}\label{exr:SamplingDistributionsIQs}IQ scores are designed to have a mean of~\(100\) and a standard deviation of~\(15\). Mensa is a society for people with a high IQ; specifically, for people who have `attained a score within the upper two percent of the general population' (Mensa webpage: \url{https://www.mensa.org/}). What IQ score is needed to join Mensa?
\end{exercise}

\begin{exercise}
\protect\hypertarget{exr:SamplingDistributionsIQsMilitary}{}\label{exr:SamplingDistributionsIQsMilitary}IQ scores are designed to have a mean of~\(100\) and a standard deviation of~\(15\). \citet{data:Zagorsky2016:Blondes} reports that the US Military must `reject all military recruits whose IQ is in the bottom~\(10\)\% of the population' (\citet{data:Zagorsky2016:Blondes}, p.~403). What IQs scores lead to a rejection from the US military?
\end{exercise}

\begin{exercise}
\protect\hypertarget{exr:SamplingDistributionsChargingEVs}{}\label{exr:SamplingDistributionsChargingEVs}A study of the impact of charging electric vehicles (EVs) on electricity demands \citep{affonso2018probabilistic} modelled the \emph{time} at which people began charging their EVs at home. Based on a survey \citep{us20112009}, they modelled the time at which EVs began charging as having a mean of~\(5\):\(30\)pm, with a standard deviation of~\(2.28\,\text{h}\). For this model:

\begin{enumerate}
\def\labelenumi{\arabic{enumi}.}
\tightlist
\item
  What is the probability that an EV will begin charging after~\(9\)pm?
\item
  What is the probability that an EV will begin charging before~\(5\)pm?
\item
  What is the probability that an EV will begin charging between~\(5\)pm and~\(6\)pm?
\item
  \(30\)\% of the EVs begin charging after what time?
\item
  The earliest~\(15\)\% of charging begins when?
\end{enumerate}

\emph{Hint:} this question is easier if you convert times into `minutes after~\(5\):\(30\)'.
\end{exercise}

\captionsetup{font=normalsize}

\begin{EOCanswerBox}{iconmonstr-check-mark-14-240.png}
\textbf{Answers to \emph{Quick review} questions:} \textbf{1.} True. \textbf{2.} True. \textbf{3.} False: \(1 - 0.70 = 0.30\). \textbf{4.} False: \(z = -1.04\). \textbf{5.} True. \textbf{6.} True.

\end{EOCanswerBox}

\part{Analysis}\label{part-analysis}

\chapter{Introducing inference}\label{IntroducingInference}

\index{Hypothesis testing}\index{Confidence intervals}

\begin{cols}
\begin{col}{0.52\textwidth}

\begin{objectivesBox}{iconmonstr-target-4-240.png}
So far, you have learnt to ask an RQ, design a study, describe and summarise the data, and model sampling variation.
\textbf{In this chapter}, you will be introduced to the two big ideas in inference: \emph{confidence intervals} and \emph{hypothesis testing}.
You will learn to:
\begin{itemize}\tightlist
  \item
  explain the purpose of a confidence interval (CI).
  \item
  explain the purpose of hypothesis testing.
\end{itemize}
\end{objectivesBox}

\end{col}

\begin{col}{0.03\textwidth}
~
\end{col}

\begin{col}{0.45\textwidth}

\includegraphics[width=0.95\linewidth]{21-Inference-Intro_files/figure-latex/unnamed-chunk-4-1} 
\end{col}
\end{cols}

After posing an RQ (Chap.~\ref{RQs}), a study is designed (Chaps.~\ref{ResearchDesign}--\ref{Interpretation}) to gather the evidence to answer the RQ (Chap.~\ref{CollectingDataProcedures}). Then the data are classified (Chap.~\ref{DescribingVars}) and summarised accordingly (Chaps.~\ref{SummariseQualData} to~\ref{SummariseComments}) in preparation for answering the RQ.

This part introduces \emph{analysis}: where the data are used to answer the RQ about the population. Answering the RQ (which is about a \emph{parameter}) is difficult, since we only study one of the countless possible samples, and hence observe only one of the countless possible values for the \emph{statistic}. The variation in the values of the statistics from sample to sample is called \emph{sampling variation} (Chap.~\ref{SamplingVariation}).

Analysis provides the tools for learning about a population parameter, based on observing one of the numerous possible values of a sample statistic. The appropriate type of analysis depends upon the number and types of variables, and the purpose of the RQ (Sect.~\ref{TwoPurposesOfRQs}):

\begin{itemize}
\tightlist
\item
  \emph{confidence intervals} answer estimation RQs, where the interest is in how precisely a \emph{statistic} estimates a \emph{parameter} (Chaps.~\ref{CIOneProportion} to~\ref{OneMeanConfInterval}; \ref{AnalysisPaired} to \ref{AnalysisOddsRatio}; Sect.~\ref{RegressionCI}).\index{Confidence intervals}
\item
  \emph{hypothesis tests} answer decision-making RQs, where \emph{decisions} are required about a \emph{parameter} based on the value of the \emph{statistic} (Chaps.~\ref{TestOneProportion} to~\ref{TestOneMean}; \ref{AnalysisPaired} to \ref{AnalysisOddsRatio}; Sects.~\ref{CorrelationTesting} and~\ref{RegressionHT}.)\index{Hypothesis testing}
\end{itemize}

Different scenarios require different confidence intervals and hypothesis tests; those discussed in this book are shown in Table~\ref{tab:OverviewTable}.





























\begin{table}
\centering
\caption{\label{tab:OverviewTable}Confidence intervals and hypothesis tests for different situations.}
\centering
\fontsize{8}{10}\selectfont
\begin{tabular}[t]{lcc}
\toprule
\multicolumn{1}{c}{\textbf{ }} & \multicolumn{1}{c}{\textbf{Estimation RQs}} & \multicolumn{1}{c}{\textbf{Decision-making RQs}} \\
\multicolumn{1}{c}{\textbf{ }} & \multicolumn{1}{c}{\textbf{(for forming}} & \multicolumn{1}{c}{\textbf{(for conducting}} \\
\textbf{} & \textbf{confidence intervals)} & \textbf{hypothesis tests)}\\
\midrule
\addlinespace[0.3em]
\multicolumn{3}{l}{\textit{Descriptive RQs}}\\
\hspace{1em}Single proportions & Chap.~\ref{CIOneProportion} & Chap.~\ref{TestOneProportion}\\
\hspace{1em}Single means & Chap.~\ref{OneMeanConfInterval} & Chap.~\ref{TestOneMean}\\
\addlinespace[0.3em]
\multicolumn{3}{l}{\textit{Repeated-measures RQs}}\\
\hspace{1em}Mean differences (paired data) & Chap.~\ref{AnalysisPaired} & Chap.~\ref{AnalysisPaired}\\
\addlinespace[0.3em]
\multicolumn{3}{l}{\textit{Relational RQs}}\\
\hspace{1em}Comparing two means & Chap.~\ref{AnalysisTwoMeans} & Chap.~\ref{AnalysisTwoMeans}\\
\hspace{1em}Comparing two odds or proportions & Chap.~\ref{AnalysisOddsRatio} & Chap.~\ref{AnalysisOddsRatio}\\
\addlinespace[0.3em]
\multicolumn{3}{l}{\textit{Correlational RQs}}\\
\hspace{1em}Correlation & Sect.~\ref{CorrelationCI} & Sect.~\ref{CorrelationTesting}\\
\hspace{1em}Regression & Sect.~\ref{RegressionCI} & Sect.~\ref{RegressionHT}\\
\bottomrule
\end{tabular}
\end{table}

\chapter{Confidence intervals: one proportion}\label{CIOneProportion}

\begin{cols}
\begin{col}{0.52\textwidth}

\begin{objectivesBox}{iconmonstr-target-4-240.png}
So far, you have learnt to ask an RQ, design a study, describe and summarise the data, and model sampling variation.
\textbf{In this chapter}, you will learn to:

\begin{itemize}\tightlist
  \item
  identify situations where estimating a proportion is appropriate.
  \item
  form confidence intervals for one proportion.
  \item
  determine whether the conditions for using the confidence intervals apply in a given situation.

\end{itemize}
\end{objectivesBox}

\end{col}

\begin{col}{0.03\textwidth}
~
\end{col}

\begin{col}{0.45\textwidth}

\includegraphics[width=0.95\linewidth]{22-CIs-OneProportion_files/figure-latex/unnamed-chunk-8-1} 
\end{col}
\end{cols}

\section{Introduction}\label{CIOnePIntro}

Suppose a fair, six-sided die is rolled \(25\)~times. What proportion of the rolls will produce an even number? That is, what will be the value of the \emph{sample proportion} of numbers that are even? Of course, no-one knows, because the proportion of rolls that will be even will not be the same for every sample of \(25\)~rolls. The value of the sample proportion (the statistic) \emph{varies} from sample to sample: \emph{sampling variation} exists.

\section{\texorpdfstring{Sampling distribution for \(\hat{p}\): for \(p\) known}{Sampling distribution for \textbackslash hat\{p\}: for p known}}\label{SamplingDistributionKnownp}

\index{Sampling distribution!one proportion (known\ $p$)}

As seen in Chap.~\ref{SamplingVariation}, sample statistics often vary with a normal distribution (whose standard deviation is called the \emph{standard error}). However, being more specific about the details of the normal distribution (such as the values of its mean and standard deviation) is useful.

\begin{importantBox}{iconmonstr-warning-8-240.png}
Remember: studying a sample leads to the following observations: \vspace{-2ex}

\begin{itemize}
\tightlist
\item
  every sample is likely to be different.
\item
  we observe just one of the many possible samples.
\item
  every sample is likely to yield a different value for the statistic.
\item
  we observe just one of the many possible values for the statistic. \vspace{-2ex}
\end{itemize}

Since many values for the sample proportion are possible, the values of the sample proportion vary (called \emph{sampling variation}) and have a \emph{distribution} (called a \emph{sampling distribution}).

\end{importantBox}

To better understand the sampling distribution for the proportion of even numbers in \(25\)~rolls of a die, statistical theory could be used, or thousands of repetitions of a sample of~\(25\) rolls could be performed, or a computer could \emph{simulate} many samples of \(25\)~rolls (like we did for a roulette wheel in Sect.~\ref{SamplingDistributionProportions}).

Here, the \emph{population proportion} of even rolls is \(p = 0.5\) (using the classical approach to probability: three of the six faces of the die are even). Each sample of \(n = 25\) rolls produces a \emph{sample} proportion, denoted by~\(\hat{p}\), which varies from sample to sample.

\begin{tipBox}{iconmonstr-info-6-240.png}
\(p\) refers to the \emph{population} proportion, and~\(\hat{p}\) refers to a \emph{sample} proportion.

\end{tipBox}

The sample proportions would be expected to vary around \(p = 0.5\) (the \emph{population proportion}): some values of \(\hat{p}\) larger than~\(p\), and some smaller than~\(p\). The value of the sample proportion in~\(25\) rolls could be \emph{very} small or \emph{very} high by chance, but we wouldn't expect to see that very often. The sample proportions exhibit sampling variation, and the \emph{amount} of sampling variation is quantified using a \emph{standard error}.

Suppose a fair die was rolled \(25\)~times, and this random procedure\index{Random procedure} was repeated \emph{thousands} of times, and the proportion of even rolls was recorded for every one of those thousands of sets of \(25\)~rolls. These thousands of sample proportions \(\hat{p}\) (one from every sample of \(n = 25\)~rolls) could be shown using a histogram (Fig.~\ref{fig:RollDiceHistFigCI}).

\begin{figure}[hbtp]

{\centering \includegraphics[width=0.7\linewidth]{22-CIs-OneProportion_files/figure-latex/RollDiceHistFigCI-1} 

}

\caption{The proportion of rolls that are even, $\hat{p}$, is not the same for every sample of $25$ rolls; it varies around a mean of $p = 0.5$. The solid line is the normal distribution used to model the sampling distribution.}\label{fig:RollDiceHistFigCI}
\end{figure}

The shape of the histogram is roughly a normal distribution. The sampling distribution for \(\hat{p}\) will always have an approximately normal distribution when certain conditions are met: see Sect.~\ref{ValidityProportions}. The mean of this distribution is called the \emph{sampling mean}, and the standard deviation for this sampling distribution is called the \emph{standard error}, denoted~\(\text{s.e.}(\hat{p})\) (see Fig.~\ref{fig:NormalDieTheoryCI}).

More specifically, the \emph{values} of the mean and standard deviation of the normal distribution in Fig.~\ref{fig:RollDiceHistFigCI} can be determined:

\begin{itemize}
\tightlist
\item
  the \emph{sampling mean} has the value of \(p = 0.5\) (i.e., the average value of \(\hat{p}\) is~\(0.5\)).
\item
  the standard deviation, called the \emph{standard error} \(\text{s.e.}(\hat{p})\), has the value~\(0.1\). (The source of this number will be revealed soon, in Equation~\eqref{eq:StdErrorExampleDieCI}.)
\end{itemize}

This distribution is the \emph{sampling distribution}, whose standard deviation is called a \emph{standard error}. A picture of this normal distribution can be drawn (Fig.~\ref{fig:NormalDieTheoryCI}). While we still don't know \emph{exactly} what the next roll will produce, we have some idea of \emph{how} the sample proportion varies in samples of \(25\)~rolls. For instance, values of~\(\hat{p}\) less than~\(0.2\), or greater than~\(0.8\) are unlikely to be observed from a fair die (with \(p = 0.5\)) in \(25\)~rolls.

\begin{figure}[hbtp]

{\centering \includegraphics[width=1\linewidth]{22-CIs-OneProportion_files/figure-latex/NormalDieTheoryCI-1} 

}

\caption{The sampling distribution is an approximate normal distribution with mean \ $0.5$ and standard error\ $0.1$; it is a model for how the proportion of even rolls varies when a die is rolled $25$ times.}\label{fig:NormalDieTheoryCI}
\end{figure}

More generally, the sampling distribution of~\(\hat{p}\) is described as follows.

\begin{definition}[Sampling distribution of a sample proportion with $p$ known]
\protect\hypertarget{def:SamplingDistPropCI}{}\label{def:SamplingDistPropCI}

When the value of~\(p\) is \emph{known}, the \emph{sampling distribution of the sample proportion} is (when certain conditions are met; Sect.~\ref{ValidityProportions}) described by

\begin{itemize}
\tightlist
\item
  an approximate normal distribution,
\item
  centred around the sampling mean whose value is~\(p\),
\item
  with a standard deviation (called the \emph{standard error} of~\(\hat{p}\)), whose value is \begin{equation}
   \text{s.e.}(\hat{p}) = \sqrt{\frac{ p \times (1 - p)}{n}},
   \label{eq:StdErrorPknownCI}
  \end{equation} where~\(n\) is the size of the sample used to compute~\(\hat{p}\), and~\(p\) is the population proportion.
\end{itemize}

\end{definition}

\begin{importantBox}{iconmonstr-warning-8-240.png}
The parameter~\(p\) and the statistic~\(\hat{p}\) are both \emph{proportions}. However, the \emph{average value} of the sample proportions from all possible samples can be described by a \emph{sampling mean}, whose value is~\(p\). The sampling mean of the sampling distribution is the `average' value of all possible sample proportions,~\(\hat{p}\).

\end{importantBox}

For the die example, where \(n = 25\)~rolls and \(p = 0.5\), using Equation~\eqref{eq:StdErrorPknownCI} gives: \begin{equation} 
    \text{s.e.} (\hat{p}) = \sqrt{\frac{0.5 \times (1 - 0.5)}{25}} = 0.1.
   \label{eq:StdErrorExampleDieCI}
\end{equation} This standard error is the standard deviation of the normal distribution in Fig.~\ref{fig:RollDiceHistFigCI}.

In practice the value of~\(p\) is almost always unknown. This situation is studied from Sect.~\ref{SamplingDistributionUnknownp} onwards.

\section{\texorpdfstring{Sampling intervals for \(\hat{p}\)}{Sampling intervals for \textbackslash hat\{p\}}}\label{CIpKnownp}

\index{Sampling interval}

Since the possible values of the sample proportions~\(\hat{p}\) can be described by an approximate \emph{normal distribution}, the \(68\)--\(95\)--\(99.7\) rule (Def.~\ref{def:EmpiricalRule}) applies.\index{68@$68$--$95$--$99.7$ rule}

For example, in Fig.~\ref{fig:NormalDieTheoryCI} (where the sampling mean is~\(0.5\) and the standard error is \(0.1\)), about~\(68\)\% of the time a sample of~\(25\) rolls will have a value of~\(\hat{p}\) between~\(0.5\) give-or-take \emph{one} standard deviation (that is, give-or-take~\(0.1\)).

So, about~\(68\)\% of the time, the proportion of even rolls in a sample of~\(25\) rolls will lie between \(0.5 - 0.1 =  0.4\) and \(0.5 + 0.1 =  0.6\). Similarly, about~\(95\)\% of the time, the proportion of even rolls will be between~\(0.5\) give-or-take \((2\times0.1\)), or between~\(0.3\) and~\(0.7\).

These intervals tell us what values of~\(\hat{p}\) are likely to be observed in samples of size~\(25\). Most of the time (i.e., approximately~\(95\)\% of the time), the value of~\(\hat{p}\) is expected to be between~\(0.30\) and~\(0.70\) (Fig.~\ref{fig:CIrelationshipsSI}).

Formally, the sample proportion~\(\hat{p}\) is likely to lie within the interval \[
   p \pm \big(\text{multiplier} \times \text{s.e.}(\hat{p})\big),
\] where \(\text{s.e.}(\hat{p})\) is the \emph{standard error of the sample proportion} (calculated using Equation~\eqref{eq:StdErrorPknownCI}). The \emph{multiplier} is a \(z\)-score, whose value depends on how confident we wish to be that the interval contains the value of~\(\hat{p}\). For a~\(95\)\% interval, the multiplier is \emph{approximately}~\(2\), based on the \(68\)--\(95\)--\(99.7\) rule: approximately~\(95\)\% of observations are within \emph{two} standard deviations of the value of~\(p\) (the mean of the normal distribution in Fig.~\ref{fig:NormalDieTheoryCI}). That is, the \emph{approximate}~\(95\)\% sampling interval is: \[
  p \pm (2 \times \text{s.e.}(\hat{p}) ).
\] An exact value for the multiplier (i.e., a \(z\)-score) can be found using the tables in Appendices~\ref{ZTablesNEG} and \ref{ZTablesPOS}. \emph{Any} level of confidence can be used (but different multipliers are then needed). This interval is called a \emph{sampling interval}.

\begin{importantBox}{iconmonstr-warning-8-240.png}
The symbol~`\(\pm\)' means `plus or minus', or (colloquially) `give-or-take'.

\end{importantBox}

\begin{figure}[hbtp]

{\centering \includegraphics[width=0.73\linewidth]{22-CIs-OneProportion_files/figure-latex/CIrelationshipsSI-1} 

}

\caption{A known value of $p$ produces a range of $\hat{p}$ values.}\label{fig:CIrelationshipsSI}
\end{figure}

\section{\texorpdfstring{Sampling distribution for \(\hat{p}\): for \(p\) unknown}{Sampling distribution for \textbackslash hat\{p\}: for p unknown}}\label{SamplingDistributionUnknownp}

\index{Sampling distribution!one proportion (for a CI)}

In the die example (Sects.~\ref{SamplingDistributionKnownp} and~\ref{CIpKnownp}), the value of~\(p\) was known. However, usually the value of~\(p\) (the \emph{parameter}) is \emph{unknown}; after all, the reason for taking a sample is to \emph{estimate} the unknown value of~\(p\). When the value of~\(p\) is unknown, the standard error is computed using the best available estimate of~\(p\), which is~\(\hat{p}\).

\begin{definition}[Sampling distribution of a sample proportion with $p$ unknown]
\protect\hypertarget{def:DEFSamplingDistributionPhat}{}\label{def:DEFSamplingDistributionPhat}

When the value of~\(p\) is \emph{unknown}, the \emph{sampling distribution of the sample proportion} is (when certain conditions are met; Sect.~\ref{ValidityProportions}) described by

\begin{itemize}
\tightlist
\item
  an approximate normal distribution,
\item
  centred around the sampling mean, whose value is~\(p\),
\item
  with a standard deviation (called the \emph{standard error} of~\(\hat{p}\)) whose value is \begin{equation}
   \text{s.e.}(\hat{p}) = \sqrt{\frac{ \hat{p} \times (1 - \hat{p})}{n}},
   \label{eq:stderrorphat}
  \end{equation} where \(n\) is the size of the sample used to compute~\(\hat{p}\), and~\(\hat{p}\) is the sample proportion. In general, the approximation gets better as the sample size gets larger.
\end{itemize}

\end{definition}

\begin{importantBox}{iconmonstr-warning-8-240.png}
When computing the standard error for a proportion, take care! Make sure you use a \emph{proportion} in the formula, not a \emph{percentage} (i.e.,~ \(0.5\) rather than~\(50\)\%). Also: don't forget to take the square root.

\end{importantBox}

\section{\texorpdfstring{Confidence intervals for \(p\)}{Confidence intervals for p}}\label{ConfIntPUnknownP}

\index{Confidence intervals!one proportion|(}

Let's pretend for the moment that the population proportion of even rolls on a die is \emph{unknown} (simply to demonstrate ideas). An \emph{estimate} of the population proportion of even rolls could be found by rolling a die \(n = 25\) times, and computing~\(\hat{p}\) (an estimate of~\(p\)). Suppose~\(11\) of the \(n = 25\) rolls produce an even number, so \(\hat{p} = 11/25 = 0.44\). The best estimate of~\(p\) is therefore \(\hat{p} = 0.44\). We might expect the (unknown) value of~\(p\) to be a little smaller than this estimate~\(\hat{p}\), or a little larger.

Using Def.~\ref{def:DEFSamplingDistributionPhat}, the sample proportions vary with an approximate normal distribution around~\(p\) (whose value is unknown), with a standard deviation (the standard error) of \[
  \text{s.e.}(\hat{p}) = \sqrt{\frac{ 0.44 \times (1 - 0.44)}{25}} = 0.09927739.
\] Previously, the sampling distribution was used to construct a sampling interval that was likely to contain the unknown value of~\(\hat{p}\). However, here the value of~\(\hat{p}\) is known, so an interval is created that is likely to contain the unknown value of~\(p\) that produced the observed value of \(\hat{p}\) (Fig.~\ref{fig:CIrelationshipsCI}).

\begin{figure}[hbtp]

{\centering \includegraphics[width=0.75\linewidth]{22-CIs-OneProportion_files/figure-latex/CIrelationshipsCI-1} 

}

\caption{The sampling distribution for\ $\hat{p}$: many values of $p$ may have produced the observed value of $\hat{p}$.}\label{fig:CIrelationshipsCI}
\end{figure}

The unknown value of~\(p\) could be a little smaller or a little larger than the value of~\(\hat{p}\); the interval is the value of \(\hat{p}\), give-or-take a little. More formally: \[
  \hat{p} \pm \big(\text{multiplier}\times\text{s.e.}(\hat{p})\big)
\] for a suitable multiplier. This interval for~\(p\) is called a \emph{confidence interval} (or a CI). The multiplier is a \(z\)-score, and the \(68\)--\(95\)--\(99.7\)\index{68@$68$--$95$--$99.7$ rule} rule gives approximate values for the multipliers. The give-or-take amount, called the \emph{margin of error}, is \(\left(\text{multiplier}\times\text{s.e.}(\hat{p})\right)\).\index{Margin of error}

Using the approximate multiplier of~\(2\) (from the \(68\)--\(95\)--\(99.7\) rule), the approximate~\(95\)\%~CI is \[
   0.44 \pm (2 \times 0.099277), \quad\text{or $0.44\pm 0.1986$};
\] that is, the margin of error is~\(0.1986\). Computing the two values, the interval is from \begin{align*}
                 0.44 - 0.1986 &\qquad\text{(which is $0.241$)}\\
  \text{to}\quad 0.44 + 0.1986 &\qquad\text{(which is $0.639$)}.
\end{align*} The interval, from~\(0.241\) to~\(0.639\), is an interval containing values of~\(p\) that could have reasonably produced the observed value of~\(\hat{p} = 0.44\) (Fig.~\ref{fig:pProducingpHatLATEX}). We can say the interval~\(0.241\) to~\(0.639\) has a~\(95\)\% chance of straddling the unknown value of the population proportion~\(p\).

\begin{definition}[Confidence interval for $p$]
\protect\hypertarget{def:ConfidenceIntervalp}{}\label{def:ConfidenceIntervalp}A \emph{confidence interval} (CI) for the unknown value of the population proportion~\(p\) is \begin{equation}
  \hat{p} \pm \big( \text{multiplier} \times \text{s.e.}(\hat{p})\big), 
  \label{eq:CIp}
\end{equation} where \(\big( \text{multiplier} \times \text{s.e.}(\hat{p})\big)\) is the \emph{margin of error}, and \(\text{s.e.}(\hat{p})\) is the \emph{standard error} of~\(\hat{p}\) (see Equation~\eqref{eq:stderrorphat}), where~\(\hat{p}\) is the sample proportion. For an \emph{approximate}~\(95\)\%~CI, the multiplier is~\(2\).
\end{definition}

\begin{figure}[hbtp]

{\centering \includegraphics[width=1\linewidth]{22-CIs-OneProportion_files/figure-latex/pProducingpHatLATEX-1} 

}

\caption{The CI gives an interval containing values of $p$ that may have produced the observed value of $\hat{p}$. Here, the CI is $0.241$ to $0.639$, shown as the thick black horizontal line under the plots.}\label{fig:pProducingpHatLATEX}
\end{figure}

In general, we do not know if the computed CI contains the actual value of~\(p\), since the value of~\(p\) is usually unknown. However, in this contrived example, the CI \emph{does} happen to straddle the value of \(p = 0.5\).

\begin{tipBox}{iconmonstr-info-6-240.png}
In this case, we know the value of the population parameter: \(p = 0.5\). Usually we do \emph{not} know the value of the parameter. After all, that's why we take a sample: to \emph{estimate} the unknown value of the population proportion.

\end{tipBox}

Suppose \emph{thousands} of people rolled a die \(25\)~times, and \emph{each} person found~\(\hat{p}\) for their sample, and hence computed the CI for their sample of \(25\)~rolls. Every sample of \(25\)~rolls could produce a different estimate~\(\hat{p}\), and so a different value for \(\text{s.e.}(\hat{p})\), and hence a different~\(95\)\%~CI.\spacex However, \emph{about~95\% of these thousands of CIs from those thousands of samples would straddle the true proportion~\(p\)}.

Since we usually don't know the value of~\(p\), and since we usually only have one sample (and hence one CI), in general \emph{we never know whether the CI computed from the single sample we have straddles the value of~\(p\) or not}.

Again, let's allow the computer to simulate the situation. Suppose the process of recording the sample proportion of even numbers in \(n = 25\) rolls is repeated fifty times, and for each of those fifty sets of \(25\)~rolls a CI is produced (Fig.~\ref{fig:RollDiceCIFig}). About~\(95\)\% of those~\(95\)\%~CIs straddle the value \(p = 0.5\) (shown as solid lines), but some do not (shown as dashed lines). Of course, value of~\(p\) is rarely known, so we never know if the CI computed from our single sample contains~\(p\) or not.

\begin{figure}[hbtp]

{\centering \includegraphics[width=1\linewidth]{22-CIs-OneProportion_files/figure-latex/RollDiceCIFig-1} 

}

\caption{About $95$\% of CIs contain the population proportion. In the $50$ samples, three produced a CI that did not straddle $p = 0.5$. In practice, we only have one sample.}\label{fig:RollDiceCIFig}
\end{figure}

\begin{definition}[Confidence interval (CI)]
\protect\hypertarget{def:ConfidenceInterval}{}\label{def:ConfidenceInterval}A CI is an interval which contains the unknown value of the parameter a given percentage of the time (over repeated sampling).

Informally: a \emph{confidence interval} (CI) is an interval likely to contain the unknown value of the parameter.
\end{definition}

In general, higher confidence means wider intervals (Fig.~\ref{fig:CIWidthsMany}), since wider intervals are needed to be \emph{more} certain that the interval contains the value of \({p}\) that produced the observed value of \(\hat{p}\).

\begin{figure}[hbtp]

{\centering \includegraphics[width=0.65\linewidth]{22-CIs-OneProportion_files/figure-latex/CIWidthsMany-1} 

}

\caption{To have greater confidence that the interval straddles the value of the population proportion, the interval needs to be wider, for any given sample size.}\label{fig:CIWidthsMany}
\end{figure}

\begin{tipBox}{iconmonstr-info-6-240.png}
Using the \(68\)--\(95\)--\(99.7\) rule produces \emph{approximate} multipliers and hence \emph{approximate} CIs. Exact multipliers (and hence exact CIs), which are \(z\)-scores, can be found using the tables in Appendices~\ref{ZTablesNEG} and \ref{ZTablesPOS}, or software.\index{Software output} Except for small sample sizes, the approximate CIs are generally close to the exact CIs.

\end{tipBox}

\section{Interpretation of a CI}\label{CIInterpretationP}

\index{Confidence intervals!interpretation}

The \emph{correct} interpretation (see Def.~\ref{def:ConfidenceInterval}) of a~\(95\)\%~CI is the following:

\begin{quote}
If the same size samples were repeatedly taken many times, and the~\(95\)\% CI computed for each sample,~\(95\)\% of these CIs formed would contain the value of the parameter.
\end{quote}

In Sect.~\ref{ConfIntPUnknownP}, the CI was interpreted as giving a range of values of~\(p\) that could reasonably be expected to produce the observed value of~\(\hat{p}\). The CI can also be seen as having a~\(95\)\% chance of straddling the unknown value of the parameter. These are close to the correct interpretation.

Commonly, the CI is interpreted as having a~\(95\)\% chance of containing the value of population parameter~\(p\) (even though the CI either \emph{does} or \emph{does not} contain the value of~\(p\)). This is like a convenience that captures the essence of the correct interpretation. More details on interpreting a CI are given in Sect.~\ref{CIInterpretation}.

\section{Statistical validity conditions}\label{ValidityProportions}

\index{Statistical validity (for inference)!one proportion}

The CIs formed in this chapter assume the sampling distribution is approximately a normal distribution (and so, for example, the \(68\)--\(95\)--\(99.7\) rule can be applied). This is only true if certain conditions are met. If these conditions are met (so that the sampling distribution has an approximate normal distribution), the CI is called \emph{statistically valid}. Whenever a CI is formed, the relevant statistical validity conditions need to be checked.

If the statistical validity conditions are not met, an alternative method\index{Non-parametric statistics} \citep{conover2003practical} or resampling methods\index{Resampling methods} may be used \citep{efron2021computer}.

\begin{definition}[Statistical validity]
\protect\hypertarget{def:StatisticalValidity}{}\label{def:StatisticalValidity}A result is \emph{statistically valid} if the conditions for the underlying mathematical calculations to be approximately correct are met, such as the sampling distribution having an approximate normal distribution.
\end{definition}

\begin{example}[Statistical validity analogy]
\protect\hypertarget{exm:StatisticalValidityAnalogy}{}\label{exm:StatisticalValidityAnalogy}Suppose your doctor asks you to get a blood test, after fasting (refraining from eating) for~\(12\,\text{h}\) before your test.

The next day, you have a big breakfast, lunch at a café, and then have your blood test. Your blood is analysed, and your doctor is sent the results of the blood test.

Since you did not fast, the results may or may not be valid. The doctor can learn \emph{something}, but not as much as if you had followed instructions. Similarly, if the conditions for computing the CI are not met, the calculations still produce a CI, but the results may be slightly unreliable.
\end{example}

The CI for a single proportion is \emph{statistically valid} if \emph{both} of these are true:

\begin{itemize}
\tightlist
\item
  the number of individuals of interest exceeds~\(5\).
\item
  the number of individuals \emph{not} of interest exceeds~\(5\).
\end{itemize}

The value of~\(5\) here is a rough figure; some books give other values (such as~\(10\)). The units of analysis are also assumed to be \emph{independent} (e.g., ideally from a simple random sample).

These conditions ensure that the sampling distribution of~\(\hat{p}\) has an approximate normal distribution. If these conditions are not met, the normal model may not be a good approximation to the sampling distribution (so, for example, using the \(68\)--\(95\)--\(99.7\) rule may be inappropriate) and so the CI may also be slightly unreliable.

\begin{example}[Statistical validity]
\protect\hypertarget{exm:DiceStatValidity}{}\label{exm:DiceStatValidity}For the die-throwing example in Sect.~\ref{ConfIntPUnknownP}, \(11\)~even rolls and \(14\)~odd rolls were observed. Both exceed~\(5\), so the CI is statistically valid.
\end{example}

\begin{example}[Statistical validity conditions]
\protect\hypertarget{exm:StatisticalValidityPHat}{}\label{exm:StatisticalValidityPHat}Consider a situation where \(p = 0.1\) is the population proportion for some result of interest.

A sample of size \(n = 10\) is taken, with one positive result: \(\hat{p} = 0.1\). The statistical validity conditions \emph{are not} satisfied, and the sampling distribution is not well modelled by a normal distribution (Fig.~\ref{fig:StatisticalValidityPHat}, left panel). Using a normal distribution to model the sampling distribution would be silly.

In contrast, assume a sample of size \(n = 150\) is taken, with~\(15\) positive results: \(\hat{p} = 0.1\) again. However, in this case, the statistical validity conditions \emph{are} satisfied, and the sampling distribution is well modelled by a normal distribution (Fig.~\ref{fig:StatisticalValidityPHat}, right panel).
\end{example}

\index{Confidence intervals!one proportion|)}

\begin{figure}[hbtp]

{\centering \includegraphics[width=0.9\linewidth]{22-CIs-OneProportion_files/figure-latex/StatisticalValidityPHat-1} 

}

\caption{Two proposed sampling distributions. The sampling distribution from many simulated samples is shown in the histogram; the normal model is shown by the solid lines. Left: when the statistical validity conditions are not met. Right: when the statistical validity conditions are met.}\label{fig:StatisticalValidityPHat}
\end{figure}

\section{Example: female coffee drinkers}\label{Female-Coffee-Drinkers}

\citet{data:Kelpin2018:AlcoholCoffee} studied \(360\)~female college students in the United States, and found that~\(61\) drank coffee daily. The parameter is~\(p\), the unknown \emph{population} proportion of female college students in the United States that drink coffee daily.

The sample size is \(n = 360\), and the \emph{sample} proportion of daily coffee drinkers is \(\hat{p} = 61/360 = 0.16944\). Of course, the sample proportion varies from sample to sample, so the sample proportion has \emph{sampling variation}, measured by the \emph{standard error}: \[
  \text{s.e.}(\hat{p})
               = \sqrt{ \frac{ 0.16944 \times (1 - 0.16944)}{360}}
               = 0.01977.
\] An approximate~\(95\)\%~CI is \(0.16944 \pm (2 \times 0.01977)\), or \(0.16944 \pm 0.03954\) (i.e., the \emph{margin of error} is \(0.03954\)).\index{Margin of error} Equivalently, the approximate~\(95\)\%~CI is from~\(0.130\) to~\(0.209\), after rounding appropriately. We write:

\begin{quote}
The sample proportion of female US college students who drink coffee daily is \(\hat{p} = 0.169\) (\(n = 360\)), with an approximate~\(95\)\%~CI from~\(0.130\) to~\(0.209\).
\end{quote}

That is, values for~\(p\) that may have led to this value of \(\hat{p} = 0.1694\) are between~\(0.130\) and~\(0.209\) with \(95\)\% confidence. (This CI may or may not contain the true proportion~\(p\).) This CI is \emph{statistically} valid, since \(61\)~in the sample drink coffee, and~\(299\) do not (and both exceed five).

\begin{importantBox}{iconmonstr-warning-8-240.png}
Many decimal places are used in the working, but final answers are rounded.

\end{importantBox}

\section{Chapter summary}\label{Chap20-Summary}

To compute a confidence interval (CI) for a proportion, compute the sample proportion,~\(\hat{p}\), and identify the sample size~\(n\) used to compute~\(\hat{p}\). Then compute the standard error, which quantifies how much the value of~\(\hat{p}\) varies across all possible samples: \[
  \text{s.e.}(\hat{p})
  =
  \sqrt{\frac{ \hat{p} \times (1-\hat{p})}{n}}.
\] The \emph{margin of error} is (multiplier\({}\times{}\)standard error), where the multiplier is~\(2\) for an approximate~\(95\)\%~CI (from the \(68\)--\(95\)--\(99.7\) rule). Then the CI is: \[
   \hat{p} \pm \left( \text{multiplier}\times\text{standard error} \right).
\] The statistical validity conditions should also be checked.

\begin{tipBox}{iconmonstr-info-6-240.png}
You must use \emph{proportions} in these formulas, \textbf{not} \emph{percentages}; that is, use values between~\(0\) and~\(1\) (like~\(0.169\) rather than~\(16.9\)\%).

\end{tipBox}

\section{Quick review questions}\label{Chap24-QuickReview}

Are the following statements \emph{true} or \emph{false}?

\begin{enumerate}
\def\labelenumi{\arabic{enumi}.}
\item
  \(p\) is a \emph{parameter}. \tightlist  
\item
  The value of~\(p\) will vary from sample to sample.
\item
  The \emph{standard error} refers to the sampling variation in \(p\).
\item
  Suppose \(n = 50\) and \(\hat{p} = 0.4\); then the standard error of \(\hat{p}\) is \(0.06928\).
\end{enumerate}

\section{Exercises}\label{CIOneProportionExercises}

\hyperref[Answers]{Answers to odd-numbered exercises} are given at the end of the book.

\captionsetup{font=small}

\begin{exercise}
\protect\hypertarget{exr:CIOneProportionHiccups}{}\label{exr:CIOneProportionHiccups}

\citet{data:Lee2016:Hiccups} found that \(708\) of \(864\)~patients examined with hiccups were male in their sample.

\begin{enumerate}
\def\labelenumi{\arabic{enumi}.}
\tightlist
\item
  Compute the sample proportion of people with hiccups who are male.
\item
  Find an approximate~\(95\)\%~CI for the proportion of people with hiccups who are male.
\item
  Check if the statistical validity conditions are met or not.
\item
  Draw a sketch of how the sample proportion varies for samples of size~\(864\).
\end{enumerate}

\end{exercise}

\begin{exercise}
\protect\hypertarget{exr:CIParamedic}{}\label{exr:CIParamedic}

\citet{lord2009impact} studied how paramedics administer pain medication, and found that \(791\) of patients reporting pain did \emph{not} receive pain relief, out of~\(1\,766\) patients in the study who initially reported pain.

\begin{enumerate}
\def\labelenumi{\arabic{enumi}.}
\tightlist
\item
  Compute the sample proportion of patients who did not receive pain medication.
\item
  Find an approximate~\(95\)\%~CI for the proportion of patients who did not receive pain medication.
\item
  Check if the statistical validity conditions are met or not.
\item
  Draw a sketch of how the sample proportion varies for samples of size~\(1\,766\).
\end{enumerate}

\end{exercise}

\begin{exercise}
\protect\hypertarget{exr:CIzA}{}\label{exr:CIzA}For an approximate \(95\)\% CI, the multiplier (from the \(68\)--\(95\)--\(99.7\) rule) is~\(2\). Use Appendices~\ref{ZTablesNEG} and \ref{ZTablesPOS} to find the \emph{exact} value for the multiplier.
\end{exercise}

\begin{exercise}
\protect\hypertarget{exr:CIzB}{}\label{exr:CIzB}Use Appendices~\ref{ZTablesNEG} and \ref{ZTablesPOS} to find the \emph{exact} value for the multiplier to create a \(99\)\% CI.
\end{exercise}

\begin{exercise}
\protect\hypertarget{exr:CIOneProportionSnacking}{}\label{exr:CIOneProportionSnacking}

\citet{data:Mann12017:UniStudents} studied the eating habits of university students in Canada. They found that~\(8\) students out of~\(154\) met the recommendation for eating a sufficient number of servings of grains each day.

\begin{enumerate}
\def\labelenumi{\arabic{enumi}.}
\tightlist
\item
  Find an approximate~\(95\)\%~CI for the population proportion of Canadian students that meet the recommendation for eating a sufficient number of servings of grains each day.
\item
  Check if the statistical validity conditions are met or not.
\item
  Draw a sketch of how the sample proportion varies for samples of size~\(154\).
\item
  Would these results be likely to apply to US university students? Explain.
\end{enumerate}

\end{exercise}

\begin{exercise}
\protect\hypertarget{exr:KoalasCrossingRoads}{}\label{exr:KoalasCrossingRoads}

\citet{data:Dexter2018:Koalas} found that~\(18\) of the \(n = 51\) koalas studied in a certain area over~\(30\) months had crossed at least one road during that time. The parameter is~\(p\), the unknown \emph{population} proportion of koalas that had crossed at least one road over the~\(30\) months.

\begin{enumerate}
\def\labelenumi{\arabic{enumi}.}
\tightlist
\item
  Find an approximate~\(95\)\%~CI for the proportion of koalas that had crossed the road at least once in the~\(30\) months.
\item
  Check if the statistical validity conditions are met or not.
\item
  Draw a sketch of how the sample proportion varies for samples of size~\(51\).
\end{enumerate}

\end{exercise}

\begin{exercise}
\protect\hypertarget{exr:CIOneProportionSaltIntake}{}\label{exr:CIOneProportionSaltIntake}\citet{data:Sutherland:SaltIntake} studied salt intake in the United Kingdom, and found that~\(2\,182\) out of the~\(6\,882\) people sampled in 2007 `generally added salt at the table'. Find an approximate~\(95\)\%~CI for the population proportion of Britons that generally add salt at the table.
\end{exercise}

\begin{exercise}
\protect\hypertarget{exr:CITurbines}{}\label{exr:CITurbines}A study of turbine failures \citep{MyersBook, NelsonLifeData} ran~\(42\) turbines for around~\(3\,000\,\text{h}\), and found that nine developed fissures (small cracks). Find a~\(95\)\%~CI for the true proportion of turbines that would develop fissures after~\(3\,000\,\text{h}\) of use. Are the statistical validity conditions satisfied?

The study also ran~\(39\) turbines for around~\(400\,\text{h}\), and found that zero developed fissures. Find a~\(95\)\%~CI for the true proportion of turbines that would develop fissures after~\(400\,\text{h}\) of use. Are the statistical validity conditions satisfied?
\end{exercise}

\begin{exercise}
\protect\hypertarget{exr:CanadianEnergyDrinks}{}\label{exr:CanadianEnergyDrinks}\citet{data:Hammond2018:Drinks} studied young Canadians aged \(12\)--\(24\), and found~\(365\) of the~\(1\,516\) respondents reported sleeping difficulties after consuming energy drinks. Find a~\(95\)\%~CI for the true proportion of young Canadians who experience sleeping difficulties after consuming energy drinks. Are the statistical validity conditions satisfied?
\end{exercise}

\begin{exercise}
\protect\hypertarget{exr:OnePropCIAlcohol}{}\label{exr:OnePropCIAlcohol}\citet{mclaughlin2010alcohol} studied the proportion of alcohol-associated calls to the ambulance service over four years in a midwestern American town. Of the \(1\,014\) calls received over the four years, \(500\) were received on the weekend (Saturday and Sunday). Find an approximate \(95\)\% CI for the true proportion of alcohol-related calls that occur on the weekend.
\end{exercise}

\begin{exercise}
\protect\hypertarget{exr:OnePropCIAI}{}\label{exr:OnePropCIAI}\citet{oca2023bias} used three different AI chatbots to produce recommendations for ophthalmologist in the \(20\) largest cities in the USA. ChatGPT made \(44\) recommendations, of which \(14\) were female. Find an approximate \(95\)\% CI for the true proportion of female ophthalmologists recommended in those \(20\)~cities.
\end{exercise}

\begin{exercise}
\protect\hypertarget{exr:OnePropCICheating}{}\label{exr:OnePropCICheating}ChatGPT was launched in 2022. \citet{lee2024cheating} studied the impact on cheating for Californian high-school students in 2023. Students were asked to respond to this question (among many others):

\begin{quote}
In the past month, how often have you used an Artificial Intelligence or digital device (e.g.~ChatGPT, smart phone) as an unauthorised aid during an assessment, school assignment, or homework.
\end{quote}

Options were `Never', `Once', `\(2\) to~\(3\) times' and `\(4\)~or more times'.

In private high schools, \(13\) of~\(202\) students reported using AI in this manner at least once. Find an approximate \(95\)\% CI for the true proportion of students using ChaptGPT in this manner in 2023.
\end{exercise}

\begin{exercise}
\protect\hypertarget{exr:OnePropCIWearHats}{}\label{exr:OnePropCIWearHats}{[}\emph{Dataset}: \texttt{HatSunglasses}{]} \citet{data:Dexter2019:SunProtection} recorded the number of people at the foot of the Goodwill Bridge, Brisbane, who wore hats between \(11\):\(30\)am to \(12\):\(30\)pm. Of the \(752\)~people observed, \(101\) wore hats. Find an approximate \(95\)\% CI for the true proportion of people wearing hats at this time at the foot of the Goodwill Bridge.
\end{exercise}

\begin{exercise}
\protect\hypertarget{exr:OnePropCIWearSunglasses}{}\label{exr:OnePropCIWearSunglasses}{[}\emph{Dataset}: \texttt{HatSunglasses}{]} \citet{data:Dexter2019:SunProtection} recorded the number of people at the foot of the Goodwill Bridge, Brisbane, who wore sunglasses between \(11\):\(30\)am to \(12\):\(30\)pm. Of the \(752\)~people observed, \(249\) wore sunglasses. Find an approximate \(95\)\% CI for the true proportion of people wearing sunglasses at this time at the foot of the Goodwill Bridge.
\end{exercise}

\captionsetup{font=normalsize}

\begin{EOCanswerBox}{iconmonstr-check-mark-14-240.png}
\textbf{Answers to \emph{Quick review} questions:} \textbf{1.} True. \textbf{2.} False. \textbf{3.} False. \textbf{4.} True.

\end{EOCanswerBox}

\chapter{Confidence intervals: one mean}\label{OneMeanConfInterval}

\begin{cols}
\begin{col}{0.52\textwidth}

\begin{objectivesBox}{iconmonstr-target-4-240.png}
So far, you have learnt to ask an RQ, design a study, classify and summarise the data, and construct a confidence interval for one proportion.
\textbf{In this chapter}, you will learn to

\begin{itemize}\tightlist
  \item
  identify situations where estimating a mean is appropriate.
  \item
  form confidence intervals for one mean.
  \item
  determine whether the conditions for using the confidence intervals apply in a given situation.
\end{itemize}
\end{objectivesBox}

\end{col}

\begin{col}{0.03\textwidth}
~
\end{col}

\begin{col}{0.45\textwidth}

\includegraphics[width=0.95\linewidth]{23-CIs-OneMean_files/figure-latex/unnamed-chunk-5-1} 
\end{col}
\end{cols}

\section{Introduction}\label{CIOneMeanIntro}

\index{Confidence intervals!one mean|(}

Consider rolling a fair, six-sided die \(n = 25\) times. Suppose we are interested in the \emph{mean} of the numbers that are rolled. Since every face of the die is equally likely to appear on any one roll, the population mean of all possible rolls is \(\mu = 3.5\) (in the middle of the numbers on the faces of the die, so this is also the \emph{median}).

What will be the sample mean of the numbers in the \(25\)~rolls? We don't know, as the sample mean varies from sample to sample (\emph{sampling variation}).

\begin{importantBox}{iconmonstr-warning-8-240.png}
Remember: studying a sample leads to the following observations: \vspace{-2ex}

\begin{itemize}
\tightlist
\item
  every sample is likely to be different.
\item
  we observe just one of the many possible samples.
\item
  every sample is likely to yield a different value for the statistic.
\item
  we observe just one of the many possible values for the statistic. \vspace{-2ex}
\end{itemize}

Since many values for the sample mean are possible, the values of the sample mean vary (called \emph{sampling variation}) and have a \emph{distribution} (called a \emph{sampling distribution}).

\end{importantBox}

\section{\texorpdfstring{Sampling distribution for \(\bar{x}\): for \(\sigma\) known}{Sampling distribution for \textbackslash bar\{x\}: for \textbackslash sigma known}}\label{SamplingDistSampleMeanSigmaKnown}

\index{Sampling distribution!one mean ($\sigma$ known)}

Suppose thousands of people made one set of \(25\)~rolls each, and computed the mean for their sample.\index{Mean!of a sample} Then, every person would have a sample mean for their sample, and we could produce a histogram of all these sample means (Fig.~\ref{fig:RollDiceHistMeanFig}). The mean for any single sample of \(n = 25\) rolls will sometimes be higher than \(\mu = 3.5\), and sometimes lower than \(\mu = 3.5\), but often close to~\(3.5\). Sample means larger than~\(4.5\), or smaller than~\(2.5\), would occur rarely.

\begin{figure}[hbtp]

{\centering \includegraphics[width=0.6\linewidth]{23-CIs-OneMean_files/figure-latex/RollDiceHistMeanFig-1} 

}

\caption{Rolling dice: the mean of 25 rolls, for thousands of repetitions.  The solid line is the normal distribution used to model the sampling distribution.}\label{fig:RollDiceHistMeanFig}
\end{figure}

From Fig.~\ref{fig:RollDiceHistMeanFig}, the sample means vary with an approximate normal distribution (as with the sample proportions). This normal distribution does \emph{not} describe the data; it describes how the \emph{values of the sample means vary across all possible samples}. Under certain conditions (Sect.~\ref{ValiditySampleMean}), the values of the sample mean vary with a normal distribution, and this normal distribution has a mean and a standard deviation.

The mean of this sampling distribution (the \emph{sampling mean}) has the value \(\mu\). The standard deviation of this sampling distribution (the \emph{standard error of the sample means}) is denoted \(\text{s.e.}(\bar{x})\). When the \emph{population} standard deviation~\(\sigma\) is \emph{known}, the value of the standard error happens to be \[
  \text{s.e.}(\bar{x}) = \frac{\sigma}{\sqrt{n}}.
\] In summary, the values of the sample means have a \emph{sampling distribution} described by:

\begin{itemize}
\tightlist
\item
  an approximate normal distribution,
\item
  with a sampling mean whose value is~\(\mu\), and
\item
  a standard deviation, called the standard error, of \(\text{s.e.}(\bar{x}) = \sigma/\sqrt{n}\).
\end{itemize}

However, since the \emph{population} standard deviation is rarely ever known, we will focus on the case where the value of~\(\sigma\) is unknown (and estimated by the \emph{sample} standard deviation,~\(s\)).

\section{\texorpdfstring{Sampling distribution for \(\bar{x}\): for \(\sigma\) unknown}{Sampling distribution for \textbackslash bar\{x\}: for \textbackslash sigma unknown}}\label{SamplingDistSampleMean}

\index{Sampling distribution!one mean}

Since the value of the population standard deviation~\(\sigma\) is almost never known, the sample standard deviation~\(s\) is used to estimate of the standard error of the mean: \(\text{s.e.}(\bar{x}) = s/\sqrt{n}\). With this information, the \emph{sampling distribution of the sample mean} can be described.

\begin{definition}[Sampling distribution of a sample mean for $\sigma$ unknown]
\protect\hypertarget{def:DEFSamplingDistributionXbarCI}{}\label{def:DEFSamplingDistributionXbarCI}

When the \emph{population} standard deviation is unknown, the \emph{sampling distribution of the sample mean} is (when certain conditions are met; Sect.~\ref{ValiditySampleMean}) described by:

\begin{itemize}
\tightlist
\item
  an approximate normal distribution,
\item
  centred around a sampling mean whose value is \(\mu\),
\item
  with a standard deviation (called the \emph{standard error of the mean}), denoted~\(\text{s.e.}(\bar{x})\), whose value is \begin{equation}
   \text{s.e.}(\bar{x}) = \frac{s}{\sqrt{n}},
   \label{eq:stderrorxbarCI}
  \end{equation} where~\(n\) is the size of the sample, and~\(s\) is the sample standard deviation of the observations.
\end{itemize}

\end{definition}

\begin{importantBox}{iconmonstr-warning-8-240.png}
A mean or a median may be appropriate for describing the \emph{data}. However, the \emph{sampling distribution} for the sample mean (under certain conditions) has a \emph{normal distribution}. Hence, the mean is appropriate for describing the sampling distribution, even if not for describing the data.

\end{importantBox}

\section{\texorpdfstring{Confidence intervals for \(\mu\)}{Confidence intervals for \textbackslash mu}}\label{OneMeanCI}

In practice, we do not know the value of~\(\mu\). After all, that's why we take a sample: to \emph{estimate} the value of the unknown population mean. Suppose, then, we did not know the value of~\(\mu\) (the parameter) for the die-rolling situation, but we have an \emph{estimate}: the value of~\(\bar{x}\), the sample mean (the statistic). The value of~\(\bar{x}\) may be a bit smaller than~\(\mu\), or a bit larger than~\(\mu\) (but we don't know which, since we do not know the value of~\(\mu\)). In other words, the values of~\(\mu\) that may have produced the observed value~\(\bar{x}\) may be less than the value of~\(\bar{x}\), or greater than the value of~\(\bar{x}\).

Since the values of~\(\bar{x}\) vary from sample to sample (\emph{sampling variation}) with an approximate normal distribution (Def.~\ref{def:DEFSamplingDistributionXbarCI}), the \(68\)--\(95\)--\(99.7\) rule could be used to construct an approximate \(95\)\%~interval for the plausible values of~\(\mu\) that may have produced the observed values of the sample mean.\index{68@$68$--$95$--$99.7$ rule} This is a \emph{confidence interval} (or a CI).

A CI for the population mean is an interval surrounding a sample mean. In general, a CI for~\(\mu\) is \[
   \bar{x} \pm \overbrace{\big(\text{multiplier}\times\text{s.e.}(\bar{x})\big)}^{\text{The `margin of error'}}.
\] For an approximate \(95\)\%~CI, the multiplier is about~\(2\) (since about~\(95\)\% of values are within two standard deviations of the mean, from the \(68\)--\(95\)--\(99.7\) rule).

\begin{definition}[Confidence interval for $\mu$]
\protect\hypertarget{def:ConfidenceIntervalmu}{}\label{def:ConfidenceIntervalmu}A \emph{confidence interval} (CI) for the unknown value of the population mean~\(\mu\) is \begin{equation}
  \bar{x} \pm \big( \text{multiplier} \times \text{s.e.}(\bar{x})\big), 
  \label{eq:CImu}
\end{equation} where \(\big( \text{multiplier} \times \text{s.e.}(\hat{p})\big)\) is the \emph{margin of error}, and \(\text{s.e.}(\bar{x})\) is the \emph{standard error} of~\(\bar{x}\) (see Equation~\eqref{eq:stderrorxbarCI}), where~\(\bar{x}\) is the sample mean, and~\(n\) is the sample size. For an \emph{approximate} \(95\)\%~CI, the multiplier is~\(2\).
\end{definition}

CIs are often \(95\)\%~CIs, but any level of confidence can be used (with the appropriate multiplier). In this book, a multiplier of~\(2\) is used when \emph{approximate} \(95\)\%~CIs are created manually, and otherwise software is used.\index{Software output} Commonly, CIs are computed at \(90\)\%,~\(95\)\% and~\(99\)\% confidence levels.

\begin{tipBox}{iconmonstr-info-6-240.png}
In Chap.~\ref{CIOneProportion}, the multiplier was a \(z\)-score, and approximate values for the multiplier were found using the \(68\)--\(95\)--\(99.7\) rule.

However, when computing the CI for a sample mean, the multiplier is \emph{not} a \(z\)-score. The multiplier would be a \(z\)-score if the value of the \emph{population} standard deviation was known (e.g., the situation in Sect.~\ref{SamplingDistSampleMeanSigmaKnown}). When~\(\sigma\) is unknown (almost always), and the \emph{sample} standard deviation is used instead, the multiplier is a \(t\)-score (Sect.~\ref{Tscores}).\index{Test statistic!t@$t$-score}

The values of~\(t\)- and \(z\)-multipliers are \emph{very} similar, and (except for small sample sizes) using an approximate multiplier of~\(2\) is reasonable for computing \emph{approximate} \(95\)\%~CIs in either case.

\end{tipBox}

Pretend for the moment that the value of \(\mu\) was unknown, and we tossed a die \(25\)~times, and found \(\bar{x} = 3.2\) and \(s = 2.5\). Then, \[
   \text{s.e.}(\bar{x}) = \frac{s}{\sqrt{n}} = \frac{2.5}{\sqrt{25}} = 0.5.
\] Hence, the sample means vary with an approximate normal distribution, centred around the unknown value of~\(\mu\), with a standard deviation of \(\text{s.e.}(\bar{x}) = 0.5\) (Fig.~\ref{fig:DiceMeanNormal}).

\begin{figure}[hbtp]

{\centering \includegraphics[width=0.95\linewidth]{23-CIs-OneMean_files/figure-latex/DiceMeanNormal-1} 

}

\caption{The sampling distribution is an approximate normal distribution with mean $3.5$ and standard error $0.5$;  it is a model of how the mean roll varies when a die is rolled $25$ times.}\label{fig:DiceMeanNormal}
\end{figure}

Our estimate of~\(\bar{x} = 3.2\) may be a bit smaller than the value of~\(\mu\), or a bit larger than the value of~\(\mu\); that is, the value of~\(\mu\) is~\(\bar{x}\), give-or-take a bit. A range of~\(\mu\) values that are likely to straddle~\(\bar{x}\) is given by a CI.\spacex An \emph{approximate} \(95\)\%~CI is (using Equation~\eqref{eq:CImu}) from \begin{align*}
                 3.2 - (2 \times 0.5) &\qquad\text{(which is $2.2$)}\\
  \text{to}\quad 3.2 + (2 \times 0.5) &\qquad\text{(which is $4.2$)}.
\end{align*} Hence, values of~\(\mu\) between~\(2.2\) to~\(4.2\) could reasonably have produced a sample mean of \(\bar{x} = 3.2\). Using software, the exact \(95\)\%~CI is from~\(2.17\) to~\(4.23\), the same as the approximate CI to one decimal place.\index{Software output!one mean}

\section{Statistical validity conditions}\label{ValiditySampleMean}

\index{Statistical validity (for inference)!one mean}

As with any CI, the underlying mathematics requires certain conditions to be met so that the results are statistically valid (i.e., the sampling distribution is sufficiently like a normal distribution).

The CI for a single mean is \emph{statistically valid} if \emph{either} of these is true:

\begin{itemize}
\tightlist
\item
  \(n \ge 25\). (If the distribution of the data is highly skewed, the sample size may need to be larger.)
\item
  \(n < 25\), \emph{and} the sample data come from a population with a normal distribution.
\end{itemize}

The sample size of~\(25\) is a rough figure, and some books give other values (such as~\(30\)).

This condition ensures that the \emph{sampling distribution of the sample means has an approximate normal distribution} (so that, for example, the \(68\)--\(95\)--\(99.7\) rule can be used).\index{68@$68$--$95$--$99.7$ rule} Provided the sample size is larger than about~\(25\), this will be approximately true \emph{even if} the distribution of the individuals in the population do not have a normal distribution. That is, when \(n \ge 25\) the sample means generally have an approximate normal distribution, even if the data themselves do not follow a normal distribution. The units of analysis are also assumed to be \emph{independent} (e.g., ideally from a simple random sample).

If the statistical validity conditions are not met, other methods (e.g., non-parametric methods\index{Non-parametric statistics} \citep{bauer1972constructing}; resampling methods\index{Resampling methods} \citep{efron2021computer}) may be used.

\begin{importantBox}{iconmonstr-warning-8-240.png}
When \(n \ge 25\) approximately, the \emph{data} do not have to have a normal distribution. The \emph{sample means} need to have a normal distribution, which is approximately true if the statistical validity conditions are true.

\end{importantBox}

\begin{example}[Statistical validity]
\protect\hypertarget{exm:DiceConditions}{}\label{exm:DiceConditions}In the die example (Sect.~\ref{OneMeanCI}), where \(n = 25\), the CI is statistically valid.
\end{example}

The second statistical validity condition requires the \emph{population} to have a normal distribution. Knowing this is obviously difficult; we do not have access to the whole population. All we can reasonably do is to identify (from the sample) whether the population is likely to be non-normal (when the CI would be not valid).

\begin{example}[Statistical validity]
\protect\hypertarget{exm:AssumptionsCT}{}\label{exm:AssumptionsCT}\citet{data:silverman:CT} examine exposure to radiation for CT scans in the abdomen for \(n = 17\) patients \citep{data:zou:fluoroscopy}. As the sample size is `small' (less than~\(25\)), the \emph{population data} must have a normal distribution for a CI for~\(\mu\) to be statistically valid.

A histogram of the total radiation dose received using the \emph{sample} data (Fig.~\ref{fig:CTscanHistogram}) suggests this is very unlikely. Even though the histogram is from \emph{sample} data, it seems improbable that the data in the sample would have come from a \emph{population} with a normal distribution.

A CI for the mean of these data will probably \emph{not} be statistically valid. Other methods (such as resampling methods\index{Resampling methods}, which are beyond the scope of this book) are needed to compute a CI for the mean.
\end{example}

\begin{figure}[hbtp]

{\centering \includegraphics{23-CIs-OneMean_files/figure-latex/CTscanHistogram-1} 

}

\caption{The radiation doses from CT scans for $17$ people.}\label{fig:CTscanHistogram}
\end{figure}

\index{Confidence intervals!one mean|)}

\section{Example: cadmium in peanuts}\label{Cadmium-In-Peanuts}

\citet{data:Blair2017:Peanuts} studied peanuts gathered from a variety of regions in the United States over various times (perhaps a representative sample). They found the sample mean cadmium concentration was \(\bar{x} = 0.076\,\text{ppm}\) with a standard deviation of \(s = 0.0460\,\text{ppm}\), from a sample of \(290\)~peanuts. The parameter is~\(\mu\), the population mean cadmium concentration in peanuts.

Every sample of \(n = 290\) peanuts is likely to produce a different sample mean, so \emph{sampling variation} in \(\bar{x}\)~exists and can be measured using the standard error: \[
    \text{s.e.}(\bar{x}) = \frac{s}{\sqrt{n}} = \frac{0.0460}{\sqrt{290}} = 0.002701\,\text{ppm}.
\] The approximate \(95\)\%~CI~is \(0.0768 \pm (2 \times 0.002701)\), or \(0.0768 \pm 0.00540\), which is from~\(0.0714\) to~\(0.0822\,\text{ppm}\). (The \emph{margin of error} is \(0.00540\).)\index{Margin of error} We write:

\begin{quote}
The sample mean cadmium concentration of peanuts is \(0.0768\)\,\text{ppm} (\(n = 290\)), with an approximate \(95\)\%~CI from~\(0.0714\) to~\(0.0822\)\,\text{ppm}.
\end{quote}

If we repeatedly took samples of size~\(290\) from this population, about~\(95\)\% of the \(95\)\%~CIs would contain the population mean (\emph{our} CI may or may not contain the value of~\(\mu\)). The plausible values of~\(\mu\) that could have produced \(\bar{x} = 0.0768\) are between~\(0.0714\) and~\(0.0822\,\text{ppm}\). Alternatively, we are about~\(95\)\% confident that the CI of~\(0.0714\) to~\(0.0822\,\text{ppm}\) straddles the population mean.

Since the sample size is larger than~\(25\), the CI is statistically valid.

\section{Chapter summary}\label{chapter-summary}

To compute a confidence interval (CI) for a mean, compute the sample mean,~\(\bar{x}\), and identify the sample size~\(n\). Then compute the standard error, which quantifies how much the value of~\(\bar{x}\) varies across all possible samples: \[
  \text{s.e.}(\bar{x})
  =
  \frac{s}{\sqrt{n}},
\] where~\(s\) is the sample standard deviation. The \emph{margin of error} is (multiplier\({}\times{}\)standard error), where the multiplier is~\(2\) for an approximate \(95\)\%~CI (from the \(68\)--\(95\)--\(99.7\) rule). Then the CI is: \[
   \bar{x} \pm \left( \text{multiplier}\times\text{standard error} \right).
\] The statistical validity conditions should also be checked.

\section{Quick review questions}\label{Chap25-QuickReview}

Are the following statements \emph{true} or \emph{false}?

\begin{enumerate}
\def\labelenumi{\arabic{enumi}.}
\item
  The value of \(\bar{x}\) varies from sample to sample. \tightlist  
\item
  A CI for~\(\mu\) is never statistically valid if the histogram of the \emph{data} has a non-normal distribution.
\item
  A sample of data produces \(s = 8\) and \(n = 20\); the standard error of the mean is \(1.7889\).
\item
  When the sample size is less than~\(25\), the standard error is not statistically valid.
\end{enumerate}

\section{Exercises}\label{OneMeanConfIntervalExercises}

\hyperref[Answers]{Answers to odd-numbered exercises} are given at the end of the book.

\captionsetup{font=small}

\begin{exercise}
\protect\hypertarget{exr:OneMeanCIBears}{}\label{exr:OneMeanCIBears}

\citet{bartareau2017estimating} studied American black bears, and found the mean weight of the \(n = 185\) male bears was \(\bar{x} = 84.9\,\text{kg}\), with a standard deviation of \(s = 51.1\,\text{kg}\).

\begin{enumerate}
\def\labelenumi{\arabic{enumi}.}
\tightlist
\item
  Define the \emph{parameter} of interest.
\item
  Compute the standard error of the mean.
\item
  Draw a picture of the approximate sampling distribution for~\(\bar{x}\).
\item
  Compute the approximate \(95\)\%~CI.
\item
  Write a conclusion.
\item
  Is the CI statistically valid?
\end{enumerate}

\end{exercise}

\begin{exercise}
\protect\hypertarget{exr:BagsCI}{}\label{exr:BagsCI}

\citet{data:Dianat2014:schoolbags} studied the weight of the school bags of a sample of \(586\)~children in Grades \(6\)--\(8\) in Tabriz, Iran. The mean weight was \(\bar{x} = 2.8\,\text{kg}\) with a standard deviation of \(s = 0.94\,\text{kg}\).

\begin{enumerate}
\def\labelenumi{\arabic{enumi}.}
\tightlist
\item
  Define the \emph{parameter} of interest.
\item
  Compute the standard error of the mean.
\item
  Draw a picture of the approximate sampling distribution for~\(\bar{x}\).
\item
  Compute the approximate \(95\)\%~CI.
\item
  Write a conclusion.
\item
  Is the CI statistically valid?
\end{enumerate}

\end{exercise}

\begin{exercise}
\protect\hypertarget{exr:CIOneMeanLungCapacityInChildren}{}\label{exr:CIOneMeanLungCapacityInChildren}{[}\emph{Dataset}: \texttt{LungCap}{]} \citet{data:Tager:FEV} studied the lung capacity of children in East Boston. They measured the forced expiratory volume (FEV) of a sample of \(n = 45\) eleven-year-old girls. For these children, the mean lung capacity was \(\bar{x} = 2.85\) litres and the standard deviation was \(s = 0.43\) litres \citep{BIB:data:FEV}. Find an approximate \(95\)\%~CI for the population mean lung capacity of eleven-year-old females from East Boston.
\end{exercise}

\begin{exercise}
\protect\hypertarget{exr:CIOneMeanEnvironmentalPollution}{}\label{exr:CIOneMeanEnvironmentalPollution}\citet{data:Taylor2013:Lead} studied lead smelter emissions near children's public playgrounds. They found the mean lead concentration at one playground (Memorial Park, Port Pirie, in South Australia) was \(6\,956.41\,\ensuremath{\mu}\text{g}.\,\text{m}^{-2}\), with a standard deviation of \(7\,571.74\,\ensuremath{\mu}\text{g}.\,\text{m}^{-2}\), from a sample of \(n = 58\) wipes taken over a seven-day period. (As a reference, the Western Australian Government recommends a maximum of \(400\,\ensuremath{\mu}\text{g}.\,\text{m}^{-2}\).)

Find an approximate \(95\)\%~CI for the mean lead concentration at this playground. Would these results apply to playgrounds in other parts of Australia?
\end{exercise}

\begin{exercise}
\protect\hypertarget{exr:CIOneMeanToothbrushing}{}\label{exr:CIOneMeanToothbrushing}\citet{data:Macgregor1985:ToothbrushinghYoungAdults} studied the brushing time for \(60\)~young adults (aged \(18\)--\(22\) years old), and found the mean tooth brushing time was~\(33.0\,\text{s}\), with a standard deviation of~\(12.0\,\text{s}\). Find an approximate \(95\)\%~CI for the mean brushing time for young adults.
\end{exercise}

\begin{exercise}
\protect\hypertarget{exr:CIOneMeanBloodLoss}{}\label{exr:CIOneMeanBloodLoss}

\citet{data:Williams2007:BloodLoss} asked paramedics (\(n = 199\)) to estimate the amount of blood loss on four different surfaces. When the actual amount of blood spill on concrete was~\(1\,000\,\text{mL}\), the mean guess was~\(846.4\,\text{mL}\) (with a standard deviation of~\(651.1\,\text{mL}\)).

\begin{enumerate}
\def\labelenumi{\arabic{enumi}.}
\tightlist
\item
  What is the approximate \(95\)\%~CI for the mean guess of blood loss?
\item
  Do you think the participants are good at estimating the amount of blood loss on concrete?
\item
  Is this CI statistically valid?
\end{enumerate}

\end{exercise}

\begin{exercise}
\protect\hypertarget{exr:OneMeanCINHANESInterpret}{}\label{exr:OneMeanCINHANESInterpret}

{[}\emph{Dataset}: \texttt{NHANES}{]} Using data from the \textsc{nhanes} study \citep{data:NHANES3}, the approximate \(95\)\%~CI for the mean direct HDL cholesterol is~\(1.356\) to~\(1.374\,\text{mmol}\)/L.\spacex Which (if any) of these interpretations are acceptable? Explain \emph{why} are the other interpretations are incorrect.

\begin{enumerate}
\def\labelenumi{\arabic{enumi}.}
\tightlist
\item
  In the \emph{sample}, about \(95\)\%~of individuals have a direct HDL concentration between \(1.356\) to \(1.374\,\text{mmol}\)/L.
\item
  In the \emph{population}, about \(95\)\%~of individuals have a direct HDL concentration between \(1.356\) to \(1.374\,\text{mmol}\)/L.
\item
  About \(95\)\%~of the \emph{samples} are between~\(1.356\) to~\(1.374\,\text{mmol}\)/L.
\item
  About \(95\)\%~of the \emph{populations} are between~\(1.356\) to~\(1.374\,\text{mmol}\)/L.
\item
  The \emph{population} mean varies so that it is between~\(1.356\) to~\(1.374\,\text{mmol}\)/L about \(95\)\% of the time.
\item
  We are about \(95\)\%~sure that \emph{sample} mean is between~\(1.356\) to~\(1.374\,\text{mmol}\)/L.
\item
  It is plausible that the \emph{sample} mean is between~\(1.356\) to~\(1.374\,\text{mmol}\)/L.
\end{enumerate}

\end{exercise}

\begin{exercise}
\protect\hypertarget{exr:OneMeanStdError}{}\label{exr:OneMeanStdError}

\citet{data:Grabosky2016:Trees} describe the diameter of \emph{Quercus bicolor} trees planted in a lawn as having a mean of~\(25.8\,\text{cm}\), with a standard error of~\(0.64\,\text{cm}\), from a sample of~\(19\) trees. Which (if either) of the following is correct?

\begin{enumerate}
\def\labelenumi{\arabic{enumi}.}
\tightlist
\item
  About~\(95\)\% of the trees in the \emph{sample} will have a diameter between \(25.8 - (2\times 0.64)\) and \(25.8 + (2\times 0.64)\,\text{cm}\) (using the \(68\)--\(95\)--\(99.7\) rule).
\item
  About~\(95\)\% of these types of trees in the \emph{population} will have a diameter between \(25.8 - (2\times 0.64)\) and \(25.8 + (2\times 0.64)\,\text{cm}\) (using the \(68\)--\(95\)--\(99.7\) rule)?
\end{enumerate}

\end{exercise}

\begin{exercise}
\protect\hypertarget{exr:ChewingTime}{}\label{exr:ChewingTime}

\citet{watanabe1995estimation} studied \(n = 30\) five-year-old children, and found the mean time for the children to eat a cookie was~\(61.3\,\text{s}\), with a standard deviation of~\(29.4\,\text{s}\).

\begin{enumerate}
\def\labelenumi{\arabic{enumi}.}
\tightlist
\item
  What is an approximate \(95\)\%~CI for the population mean time for a five-year-old child to eat a cookie?
\item
  Is the CI statistically valid?
\end{enumerate}

\end{exercise}

\begin{exercise}
\protect\hypertarget{exr:CIPizzas}{}\label{exr:CIPizzas}

{[}\emph{Dataset}: \texttt{PizzaSize}{]} In~2011, \emph{Eagle Boys Pizza} ran a campaign that claimed (among many other claims) that \emph{Eagle Boys} pizzas were `Real size \(12\)-inch large pizzas' in an effort to out-market \emph{Dominos Pizza}. \emph{Eagle Boys} made the data behind the campaign publicly available \citep{mypapers:Dunn:PizzaSize}. A summary of the diameters of a sample of~\(125\) of \emph{Eagle Boys} large pizzas is shown in Fig.~\ref{fig:PizzaCIjamovi}.

\begin{enumerate}
\def\labelenumi{\arabic{enumi}.}
\tightlist
\item
  What do~\(\mu\) and~\(\bar{x}\) represent in this context? \tightlist 
\item
  Write down the \emph{values} of~\(\mu\) and~\(\bar{x}\).
\item
  Write down the \emph{values} of~\(\sigma\) and~\(s\).
\item
  Compute the value of the standard error of the mean \(\text{s.e.}(\bar{x})\).
\item
  Explain the difference in \emph{meaning} between~\(s\) and~\(\text{s.e.}(\bar{x})\) here.
\item
  If someone else takes a sample of~\(125\) \emph{Eagle Boys} pizzas, will the sample mean be \(11.486\)~inches again (as in this sample)? Why or why not?
\item
  Draw a picture of the approximate sampling distribution for~\(\bar{x}\).
\item
  Compute an approximate \(95\)\%~CI for the mean pizza diameter.
\item
  Write a statement that communicates your \(95\)\%~CI for the mean pizza diameter.
\item
  What are the \emph{statistical} validity conditions? Is the computed CI statistically valid?
\item
  Do you think that, on average, the pizzas do have a mean diameter of \(12\)~inches in the population, as Eagle Boys claim? Explain.
\end{enumerate}

\end{exercise}

\begin{exercise}
\protect\hypertarget{exr:CIMatchesPerBox}{}\label{exr:CIMatchesPerBox}

Claire and Jake were wondering about the mean number of matches in a box. The boxes contain this statement:

\begin{quote}
An average of \(45\)~matches per box.
\end{quote}

They purchased a carton containing \(n = 25\)~boxes of matches, and Jake counted the number of matches in \emph{one} of those \(25\)~boxes. The box contained \(44\)~matches.

`Oh wow. Just wow.' said Jake. `They lie. There's only~\(44\) in this box.'

\begin{enumerate}
\def\labelenumi{\arabic{enumi}.}
\tightlist
\item
  What is Jake's misunderstanding? \tightlist
\item
  Then, they counted the number of matches in \emph{each} of the \(n = 25\) boxes, and found the mean number of matches per box was~\(44.9\) matches, and the standard deviation was~\(0.124\). Jake notes that the mean is \(44.9\)~matches per box, and says: `You can't have \(0.9\)~of a match. That's dumb.' How would you respond?
\item
  `Wow!' said Jake. `The claim is \(45\)~matches per box on average, but the mean really is~\(44.9\)! They're liars!' What is Jake's misunderstanding?
\item
  `Come on, Jake,' said Claire. `As if the mean will be \emph{exactly}~\(45\) in a sample every single time. Let's work out the confidence interval.' Why does Claire think a CI is needed? What will it tell them?
\item
  What is an approximate \(95\)\%~CI for the mean for Claire's sample?
\item
  `Aha,' said jake; `I told you so! They \emph{are} absolutely lying! Your confidence interval doesn't even include their mean of~\(45\)! 'The manufacturer \emph{must} be lying!' Is Jake correct? Why or why not? What does the CI \emph{mean}?
\item
  In this scenario, what does~\(\bar{x}\) represent? What is the \emph{value} of~\(\bar{x}\)?
\item
  In this scenario, what does~\(\mu\) represent? What is the \emph{value} of~\(\mu\)?
\end{enumerate}

\end{exercise}



\begin{figure}[hbtp]

{\centering \includegraphics[width=0.4\linewidth]{jamovi/PizzaDiameter/PizzaDiameters-jamovi} 

}

\caption{Summary statistics for the diameter of \emph{Eagle Boys} large pizzas.}\label{fig:PizzaCIjamovi}
\end{figure}

\captionsetup{font=normalsize}

\begin{EOCanswerBox}{iconmonstr-check-mark-14-240.png}
\textbf{Answers to \emph{Quick review} questions:} \textbf{1.} True \textbf{2.} False. \textbf{3.} True. \textbf{4.} False.

\end{EOCanswerBox}

\chapter{More details about CIs}\label{AboutCIs}

\begin{cols}
\begin{col}{0.52\textwidth}

\begin{objectivesBox}{iconmonstr-target-4-240.png}
So far, you have learnt to ask an RQ, design a study, classify and summarise the data, and construct some confidence intervals.
\textbf{In this chapter}, you will learn more about forming \emph{confidence intervals}.
You will learn to
\begin{itemize}
  \item
  communicate confidence intervals.
  \item
  interpret confidence intervals.
\end{itemize}
\end{objectivesBox}

\end{col}

\begin{col}{0.03\textwidth}
~
\end{col}

\begin{col}{0.45\textwidth}

\includegraphics[width=0.95\linewidth]{24-CIs-More_files/figure-latex/unnamed-chunk-4-1} 
\end{col}
\end{cols}

\section{General comments}\label{CIGeneralComments}

The previous chapters discussed forming confidence intervals (CIs) for estimating a population proportion, and for estimating a population mean. CIs will also be studied (Chaps.~\ref{AnalysisPaired} to~\ref{AnalysisOddsRatio}) in other contexts. This chapter discusses some principles that apply to CIs in general:

\begin{itemize}
\tightlist
\item
  statistical validity (Sect.~\ref{ValidityCIs}).
\item
  writing conclusions (Sect.~\ref{CIWritingConclusions}).
\item
  interpreting CIs (Sect.~\ref{CIInterpretation}).
\end{itemize}

CIs are formed for an unknown value of a \emph{population} parameter (such as the population proportion~\(p\)), using the best estimate: the value of the \emph{sample} statistic (such as the sample proportion~\(\hat{p}\)). When the sampling distribution of the statistic has an approximate normal distribution (and not all sampling distribution do have a normal distribution), CIs have the form \[
  \text{statistic} \pm (\text{multiplier} \times \text{standard error}),
\] where \((\text{multiplier} \times \text{standard error})\) is called the \emph{margin of error}.\index{Margin of error}

When the sampling distribution has an approximate normal distribution, the \emph{approximate} \(95\)\%~CI uses a \emph{multiplier} of~\(2\) (from the \(68\)--\(95\)--\(99.7\) rule).\index{68@$68$--$95$--$99.7$ rule} To compute CIs other than \(95\)\% CIs (such as \(99\)\% CIs), and for \emph{exact} \(95\)\% CIs, software is used.\index{Software output}

\begin{importantBox}{iconmonstr-warning-8-240.png}
\emph{Confidence intervals (CIs)} tell us about the unknown value of the \emph{population parameter}, based on what we learn from one of the countless possible sample statistics.

\end{importantBox}

\section{More details about statistical validity}\label{ValidityCIs}

\index{Confidence intervals!statistical validity}\index{68@$68$--$95$--$99.7$ rule}

When CIs are computed, \emph{statistical validity conditions} must be true to ensure the mathematics behind the calculations are sound. For instance, many CIs assume the sampling distribution has a normal distribution (so that, for example, the \(68\)--\(95\)--\(99.7\) rule can be used); the statistical validity conditions state the conditions under which the sampling distribution has an approximate normal distribution. If these conditions are \emph{not} met, the sampling distribution may not be close to an approximate normal distribution, so the \(68\)--\(95\)--\(99.7\) rule (on which the CI is based) may not be appropriate, and the CI itself may be inappropriate. Of course, if the statistical validity conditions are close to being satisfied, then the resulting CI will still be reasonably useful.

Besides checking the statistical validity conditions, the \emph{internal validity}\index{Internal validity} and \emph{external validity}\index{External validity} of the study should be discussed (Fig.~\ref{fig:ValiditiesCI}). In addition, CIs also require that the sample size is less than about~\(10\)\% of the population size; this is almost always the case.

\begin{figure}[hbtp]

{\centering \includegraphics[width=0.9\linewidth]{24-CIs-More_files/figure-latex/ValiditiesCI-1} 

}

\caption{Four types of validities for studies.}\label{fig:ValiditiesCI}
\end{figure}

\section{More details about writing conclusions}\label{CIWritingConclusions}

\index{Confidence intervals!writing conclusions}

When reporting a CI, include:

\begin{enumerate}
\def\labelenumi{\arabic{enumi}.}
\tightlist
\item
  the CI (including units of measurement, if relevant).
\item
  the level of confidence for the CI (typically, a~\(95\)\%~CI).
\item
  the value of the statistic (the parameter estimate) and the sample size.
\end{enumerate}

If the CI is an \emph{approximate} CI (e.g., based on using an approximate multiplier of~\(2\) from the \(68\)--\(95\)--\(99.7\) rule), this should also be clear.

\begin{example}[Writing conclusions]
\protect\hypertarget{exm:CIWritingConclusions}{}\label{exm:CIWritingConclusions}In Sect.~\ref{Cadmium-In-Peanuts}, the mean cadmium level of peanuts was estimated. The conclusion was:

\begin{quote}
The sample mean cadmium concentration of peanuts is \(0.0768\,\text{ppm}\) (\(n = 290\)), with an approximate \(95\)\%~CI from~\(0.0714\) to~\(0.0822\,\text{ppm}\).
\end{quote}

Each of the three elements above are given.

\begin{enumerate}
\def\labelenumi{\arabic{enumi}.}
\tightlist
\item
  The CI: \(0.0714\) to~\(0.0822\,\text{ppm}\).
\item
  The level of confidence for the CI: \(95\)\%.
\item
  The value of the statistic: \(\bar{x} = 0.0768\,\text{ppm}\).
\end{enumerate}

In addition, the CI is flagged as an \emph{approximate} \(95\)\%~CI.
\end{example}

\section{More details about interpreting CIs}\label{CIInterpretation}

\index{Confidence intervals!interpretation}

Interpreting CIs correctly takes care. The \emph{correct} interpretation (Def.~\ref{def:ConfidenceInterval}) of a \(95\)\%~CI is:

\begin{quote}
The CI is an interval which contains the unknown value of the parameter \(95\)\%~of the time (over repeated sampling).
\end{quote}

That is, if we \emph{repeated} the process (of selecting a sample of a given size, then computing the CI) numerous times, \(95\)\%~of those CIs formed would contain the value of the parameter. This is the idea shown in Fig.~\ref{fig:RollDiceCIFig}.

In practice, this definition is unsatisfying, since we only ever have \emph{one} sample, not \emph{many} samples. Furthermore, since the value of the parameter is unknown (after all, the reason for taking a sample was to \emph{estimate} the value of the parameter), we don't know if the CI from \emph{our} single sample straddles the population parameter or not.

Two reasonable alternative interpretations for a \(95\)\%~CI are below.

\begin{quote}
\begin{itemize}
\tightlist
\item
  The \(95\)\%~CI gives a range of values of the parameter that could reasonably (with \(95\)\%~confidence) have produced the observed value of the statistic.
\item
  There is a \(95\)\%~chance that the \(95\)\%~CI straddles the unknown value of the parameter.
\end{itemize}
\end{quote}

These alternatives are adequate and common interpretations.

Frequently, the CI is described as having a \(95\)\%~chance of containing the population parameter. This is not strictly correct (the CI either \emph{does} or \emph{does not} contain the value of the population parameter), but is a common and a brief paraphrase for the correct interpretation above.

I use this analogy: most people say the sun rises in the east. This is incorrect; the sun doesn't \emph{rise} at all. People \emph{say} the sun rises in the east as a convenient way to explain that we see the sun each morning in the east as the earth rotates. Similarly, most people say a CI is an interval with a certain chance of containing the value of the population parameter, as a convenient way to explain the CI.

\begin{example}[Interpreting CIs]
\protect\hypertarget{exm:CIWritingConclusionsInterpret}{}\label{exm:CIWritingConclusionsInterpret}In Example~\ref{exm:CIWritingConclusions}, the approximate \(95\)\%~CI for the cadmium concentration in peanuts was from \(0.0714\) to \(0.0822\,\text{ppm}\). The correct interpretation is:

\begin{quote}
If many samples of \(290\)~peanuts were taken, and the approximate \(95\)\%~CI computed for each one, about \(95\)\%~of those CIs would contain the population mean.
\end{quote}

Our CI may or may not include the value of~\(\mu\), however. We might say:

\begin{quote}
This \(95\)\%~CI (from~\(0.0714\) to~\(0.0822\,\text{ppm}\)) has a \(95\)\%~chance of straddling the actual value of~\(\mu\).
\end{quote}

Alternatively, we could say:

\begin{quote}
The range of values of~\(\mu\) that could plausibly (with \(95\)\%~confidence) have produced \(\bar{x} = 0.0768\) is between~\(0.0714\) to~\(0.0822\,\text{ppm}\).
\end{quote}

In practice, the CI is usually interpreted as:

\begin{quote}
There is a~\(95\)\% chance that the population mean level of cadmium in peanuts is between~\(0.0714\) to~\(0.0822\,\text{ppm}\).
\end{quote}

This last statement is not strictly correct, but is commonly-used, and sufficient for our use.
\end{example}

\section{Chapter summary}\label{AboutCIsSummary}

\emph{Confidence intervals} (or CIs) tell us about the unknown \emph{population parameter}, based on what we learn from one the countless possible sample statistics. CIs give an interval in which a parameter is likely to lie over repeated sampling. Since we only ever have one sample, two reasonable alternative interpretations for a \(95\)\%~CI are:

\begin{quote}
\begin{itemize}
\tightlist
\item
  the \(95\)\%~CI gives a range of values of the parameter that could reasonably (with \(95\)\%~confidence) have produced our observed value of the statistic.
\item
  there is a \(95\)\%~chance that our \(95\)\%~CI straddles the value of the parameter.
\end{itemize}
\end{quote}

We never know if the CI from \emph{our} single sample includes the population parameter or not. When reporting a CI, include:

\begin{enumerate}
\def\labelenumi{\arabic{enumi}.}
\tightlist
\item
  the CI (including units of measurement, if relevant);
\item
  the level of confidence for the CI (typically, a \(95\)\%~CI); and
\item
  the value of the statistic (the parameter estimate) and the sample size.
\end{enumerate}

\section{Quick review exercises}\label{Chap26-QuickReview}

Are the following statements \emph{true} or \emph{false}?

\begin{enumerate}
\def\labelenumi{\arabic{enumi}.}
\item
  CIs \emph{always} have \(95\)\%~confidence. \tightlist  
\item
  Statistical validity concerns the \emph{generalisability} of the results.
\item
  CIs always include the value of the \emph{population} parameter.
\item
  All else being equal, a \(95\)\%~CI is \emph{wider} than a \(90\)\%~CI.
\item
  The `multiplier times the standard error' is called the \emph{margin of error}.
\item
  We are fairly sure (but \emph{not certain}) that the CI includes the value of the statistic.
\end{enumerate}

\section{Exercises}\label{AboutCIsExercises}

\hyperref[Answers]{Answers to odd-numbered exercises} are given at the end of the book.

\captionsetup{font=small}

\begin{exercise}
\protect\hypertarget{exr:AboutCIsInterpretationP}{}\label{exr:AboutCIsInterpretationP}

\citet{hirst1962epidemiology} computed a \(95\)\%~CI to estimate the proportion of trees with apple scab, and found \(\hat{p} = 0.314\) and \(\text{s.e.}(\hat{p}) = 0.091\). What would be \emph{wrong} with the following conclusions?

\begin{enumerate}
\def\labelenumi{\arabic{enumi}.}
\tightlist
\item
  An approximate \(95\)\%~CI for the sample proportion is between~\(0.223\) and~\(0.405\).
\item
  This CI means we are \(95\)\%~confident that between~\(22.3\) and~\(40.5\) trees are infected with apple scab.
\end{enumerate}

\end{exercise}

\begin{exercise}
\protect\hypertarget{exr:AboutCIsInterpretationP2}{}\label{exr:AboutCIsInterpretationP2}

\citet{data:Fayet2017:Snacks} studied the snacking habits of Australian children. In~2007 (for which \(n = 3\,637\)), the CI for the proportion of children snacking (`an eating occasion that occurred between meals based on time of day'; p.~1) was \(0.981\pm 0.003\) in~2007. What would be \emph{wrong} with the following conclusion?

\begin{itemize}
\tightlist
\item
  An approximate \(95\)\%~CI for the sample proportion of snacks (in~2007) is \(0.981\pm 0.003\).
\end{itemize}

\end{exercise}

\begin{exercise}
\protect\hypertarget{exr:AboutCIsInterpretationMean}{}\label{exr:AboutCIsInterpretationMean}

\citet{data:Guirao2017:amputees} studied how far amputees could walk in two minutes following a femoral (leg) implant. After \(14\)~months, the sample of ten amputees walked a mean of \(122.5\,\text{m}\); the \(95\)\%~CI was computed as~\(96.4\,\text{m}\) to~\(148.6\,\text{m}\). What would be \emph{wrong} with the following conclusions?

\begin{enumerate}
\def\labelenumi{\arabic{enumi}.}
\tightlist
\item
  Approximately \(95\)\% of the amputees walked between~\(96.4\) and~\(1\,488.6\,\text{m}\) in two minutes.
\item
  The \(95\)\%~CI for the sample mean distance walked in two minutes was between~\(96.4\) and~\(1\,488.6\,\text{m}\).
\end{enumerate}

\end{exercise}

\begin{exercise}
\protect\hypertarget{exr:AboutCIsInterpretationMean2}{}\label{exr:AboutCIsInterpretationMean2}

A study of sodium intake in Thailand found the \(95\)\%~CI for the mean daily sodium intake for subjects with a secondary school education was~\(3\,565\) to~\(3\,903\,\text{mg}\). What would be wrong with the following conclusions?

\begin{enumerate}
\def\labelenumi{\arabic{enumi}.}
\tightlist
\item
  This CI means that approximately~\(95\)\% of the subjects had a daily sodium intake between~\(3\,565\) to~\(3\,903\,\text{mg}\).
\item
  A \(95\)\%~CI for the sample mean daily sodium intake is between~\(3\,565\) to~\(3\,903\,\text{mg}\).
\end{enumerate}

\end{exercise}

\begin{exercise}
\protect\hypertarget{exr:CIPossums}{}\label{exr:CIPossums}In discussing the weight of adult male Leadbeater's possums, \citet{data:Williams2022:Possums} state (p.~170):

\begin{quote}
The average adult male Leadbeater's possum weighed~\(137\,\text{g}\) (\(95\)\%~CI = \(135\,\text{g}\), \(139\,\text{g}\)), with~\(90\)\% of weights between~\(122\) and~\(153\,\text{g}\).
\end{quote}

Figure~\ref{fig:CIWidthsMany} indicates that a \emph{higher} value for the confidence level means \emph{wider} CIs, since wider intervals are needed to be \emph{more} certain that the interval contains the value of the parameter that produced the value of the statistic.

In light of this, explain why the \(90\)\%~interval is \emph{wider} than the \(95\)\%~interval in the above quote.
\end{exercise}

\captionsetup{font=normalsize}

\begin{EOCanswerBox}{iconmonstr-check-mark-14-240.png}
\textbf{Answers to \emph{Quick review} questions:} \textbf{1.} False. \textbf{2.} False. \textbf{3.} False. \textbf{4.} True. \textbf{5.} True. \textbf{6.} False (CI \emph{must} contain value of statistic).

\end{EOCanswerBox}

\chapter{Making decisions}\label{MakingDecisions}

\begin{cols}
\begin{col}{0.52\textwidth}

\begin{objectivesBox}{iconmonstr-target-4-240.png}
So far, you have learnt to ask an RQ, design a study, describe and summarise the data, and form confidence intervals.
\textbf{In this chapter}, you will learn to:

\begin{itemize}\tightlist
  \item
  state the two broad reasons that might explain the difference between the values of the statistic and parameter.
  \item
  explain how decisions are made in research.
\end{itemize}
\end{objectivesBox}

\end{col}

\begin{col}{0.03\textwidth}
~
\end{col}

\begin{col}{0.45\textwidth}

\includegraphics[width=0.95\linewidth]{25-Tools-DecisionMaking_files/figure-latex/unnamed-chunk-9-1} 
\end{col}
\end{cols}

\section{Introduction: drawing cards}\label{NeedForDecisionMaking}

\index{Decision making}

Suppose I produce a pack of cards, and shuffle them well. The event of interest is `the number of red cards when I draw \(25\)~cards from the pack, with replacement'. (`With replacement' means that, after drawing a card, I place the card back into the pack, and reshuffle before drawing a new card; each draw is then from an identical, complete pack of \(52\)~cards.) The pack of cards can be considered the \emph{population}. In a standard pack, the proportion of red cards is \(p = 0.5\) for each draw, because sampling is with replacement. \(\hat{p}\) is the proportion of red cards in the \emph{sample} of \(n = 25\)~cards.

Suppose the sample of \(25\)~cards produces \(\hat{p} = 1\); that is, \emph{all} \(n = 25\) cards in the sample are red cards (Fig.~\ref{fig:Draw15Cards}). What should you conclude? How likely is it that this would happen by chance from a fair pack? Is this evidence that the pack of cards is somehow unfair, poorly shuffled, or manipulated?

\begin{figure}[hbtp]

{\centering \includegraphics[width=0.75\linewidth]{25-Tools-DecisionMaking_files/figure-latex/Draw15Cards-1} 

}

\caption{How likely is it to draw (with replacement) $25$ red cards in a row from a fair pack?}\label{fig:Draw15Cards}
\end{figure}

Of course, the sample of \(25\)~cards is just one of \emph{countless} possible samples that could have been chosen to study. Different samples comprise different cards, and the sample proportion depends on which cards are drawn for the studied sample. This leads to one of the most important observations about sampling.

\begin{importantBox}{iconmonstr-warning-8-240.png}
Studying a sample leads to the following observations: \vspace{-2ex}

\begin{itemize}
\tightlist
\item
  every sample is likely to be different.
\item
  we observe just one of the many possible samples.
\item
  every sample is likely to yield a different value for the sample statistic.
\item
  we observe just one of the many possible values for the statistic. \vspace{-2ex}
\end{itemize}

Since many values for the sample are possible, the possible values of the statistic vary (this is called \emph{sampling variation})\index{Sampling variation} and have a \emph{distribution} (this is called a \emph{sampling distribution}).\index{Sampling distribution}

\end{importantBox}

In research, decisions need to be made about parameters, based on just one of many possible values of the statistic. Sensible decisions \emph{can} be made (and \emph{are} made) about parameters based on statistics. To do this though, the process of \emph{how} decisions are made needs to be articulated, which is the purpose of this chapter.

In the cards example, obtaining \(25\)~reds cards out of~\(25\) (i.e., \(\hat{p} = 1\)) seems very unlikely from. a fair pack; you would probably conclude that the pack is somehow unfair, or that I was cheating somehow. But importantly, \emph{how} did you reach that decision? Your unconscious decision-making process may have followed these steps.\index{Decision making}

\begin{enumerate}
\def\labelenumi{\arabic{enumi}.}
\tightlist
\item
  You \emph{assumed}, quite reasonably, that I used a standard, well-shuffled pack of cards, where half the cards are red and half the cards are black. That is, you assumed the \emph{population proportion} of red cards really was \(p = 0.5\).
\item
  Based on that assumption, you \emph{expected} about half the cards in the sample of~\(25\) to be red (i.e., expected~\(\hat{p}\) to be about~\(0.5\)). You wouldn't necessarily have expected \emph{exactly} half red cards (because of sampling variation), but you expected the value of \(\hat{p}\) to be close to~\(0.5\).
\item
  You then \emph{observed} that \emph{all} \(25\)~cards were red. That is, you observed \(\hat{p} = 1\)\ldots{} which seems rather high.
\item
  You were expecting \(\hat{p} = 0.5\) approximately, but instead observed \(\hat{p} = 1\). What you observed (`all red cards') was not at all like what you were expecting (`about half red cards'); the sample \emph{contradicts} what you were expecting (from a fair pack). This suggests your assumption of a fair pack is probably wrong.
\end{enumerate}

Of course, finding \(25\)~red cards in a row is \emph{possible}\ldots{} just \emph{extremely unlikely}. For this reason, you would probably conclude that this is persuasive evidence that the pack is not fair.

\section{Making decisions: hypotheses}\label{making-decisions-hypotheses}

Two reasons could explain why the value of the \emph{sample} proportion of red cards in \(25\)~cards (\(\hat{p} = 1\)) is not exactly equal to the value of the population proportion (\(p = 0.5\)).

\begin{enumerate}
\def\labelenumi{\arabic{enumi}.}
\tightlist
\item
  \emph{The \textbf{population} proportion of red cards really is \(p = 0.5\)}, and the value of the \emph{sample} proportion \(\hat{p}\) is not equal to \(0.5\) only due to \emph{sampling variation}. That is, we just happen to have---by chance---one of those samples where the the value of \(\hat{p}\) is very large and not equal to \(p\).
\item
  \emph{The \textbf{population} proportion of red cards really isn't \(p = 0.5\)}, and this is simply reflected in the observed sample proportion.
\end{enumerate}

These two possible explanations (`statistical hypotheses') have special names.\index{Hypotheses!statistical}

\begin{enumerate}
\def\labelenumi{\arabic{enumi}.}
\tightlist
\item
  The first explanation is the \emph{null hypothesis}, denoted \(H_0\).\index{Parameter}\index{Hypotheses!null} This hypothesis proposes that \emph{the population proportion is \(0.5\)}; the value of the sample proportion is \emph{not} \(0.5\) due to \emph{sampling variation}.
\item
  The second explanation is the \emph{alternative hypothesis}, denoted \(H_1\).\index{Hypotheses!alternative} This hypothesis proposes that the population proportion is really not \(0.5\) at all, which is reflected in the value of the sample proportion.
\end{enumerate}

\emph{How} do we decide which of these explanations is supported by the data?\index{Evidence-based research}

The usual approach to decision-making in science begins by assuming the null hypothesis (the sampling-variation explanation) is true. Then the data are examined to see if persuasive evidence exists to change our mind (and support the alternative hypothesis). Remember: conclusions drawn about the \emph{population} from the \emph{sample} can never be certain, since the sample studied is just one of many possible samples that could have been taken (and every sample is likely to be different).

\begin{importantBox}{iconmonstr-warning-8-240.png}
The onus is on the data to refute the null hypothesis. That is, the null hypothesis is retained unless persuasive evidence suggests otherwise.

\end{importantBox}

\section{Making decisions: the process}\label{DecisionMaking}

The ideas in Sect.~\ref{NeedForDecisionMaking} suggest a formal process of decision-making in research (Fig.~\ref{fig:DecisionFlow2}).

\begin{enumerate}
\def\labelenumi{\arabic{enumi}.}
\tightlist
\item
  \emph{Make an assumption about the value of the parameter}. By doing so, we assume that \emph{sampling variation} explains any discrepancy between the value of the observed statistic and this assumed value of the parameter.
\item
  \emph{Define the expectations of the statistic}. Based on the assumption made about the parameter, describe what values of the \emph{statistic} might reasonably be observed from all the possible samples from the population (due to sampling variation).
\item
  \emph{Evaluate the observations}. Take a sample (one of the many possible samples), and compute the observed sample statistic from these data; then compare this to what was expected.
\item
  \emph{Make a decision}:

  \begin{itemize}
  \tightlist
  \item
    if the observed value of the \emph{sample statistic} is \emph{unlikely} to have been observed by chance, the statistic (i.e., the evidence) \emph{contradicts} the assumption about the \emph{parameter}, so the assumption is probably (but not certainly) \emph{wrong}.
  \item
    if the observed value of the \emph{sample statistic} could easily be explained by chance, the statistic (i.e., the evidence) is \emph{consistent with} the assumption about the \emph{parameter}, so the assumption may be (but is not certainly) \emph{correct}.
  \end{itemize}
\end{enumerate}

\begin{figure}[hbtp]

{\centering \includegraphics[width=1\linewidth]{25-Tools-DecisionMaking_files/figure-latex/DecisionFlow2-1} 

}

\caption{A way to make decisions.}\label{fig:DecisionFlow2}
\end{figure}

This approach is similar to how we unconsciously make decisions every day. For example, suppose I ask my son to brush his teeth \citep{data:Budgett:RandomizationTest}. Later, I wish to decide if he really did. The decision-making process may proceed as follows.

\begin{enumerate}
\def\labelenumi{\arabic{enumi}.}
\tightlist
\item
  \emph{Assumption}: I \emph{assume} my son brushed his teeth (because I asked him to).
\item
  \emph{Expectation}: based on my assumption, I would \emph{expect} to find his toothbrush is damp.
\item
  \emph{Observation}: when I check later, I observe a \emph{dry} toothbrush, which is unexpected.
\item
  \emph{Decision}: the evidence \emph{contradicts} what I expected to find based on my assumption, so my assumption is probably \emph{false}: he probably \emph{didn't} brush his teeth.
\end{enumerate}

I may have made the wrong decision: he may have brushed his teeth, then dried his toothbrush with a hair dryer. However, based on the evidence, he likely has not brushed his teeth.

The situation may have ended differently: when I check later, suppose I observe a \emph{damp} toothbrush. Then, the evidence seems \emph{consistent} with what I expected if he brushed his teeth (my assumption), so my assumption is probably \emph{true}; he probably did brush his teeth. Again, I may be wrong: he may have rinsed his toothbrush under a tap. Nonetheless, I don't have evidence that he didn't brush his teeth.

Similar logic underlies most decision-making in science.\footnote{Other ways exist to make decisions, such as using prior knowledge. For example, if my son had a reputation for wetting his toothbrush under the tap instead of brushing his teeth, that information can be incorporated into the decision-making. This is called a \emph{Bayesian approach}.}

\section{Making decisions: the steps}\label{MakingDecisionsInResearch}

\index{Decision making!steps}

Let's think about each step in the decision-making process (Fig.~\ref{fig:DecisionFlow2}) individually:

\begin{itemize}
\tightlist
\item
  making an \emph{assumption} about the parameter (Sect.~\ref{Assumption}).
\item
  describing the \emph{expectations} of the statistic (Sect.~\ref{ExpectationOf}).
\item
  taking the sample \emph{observations} (Sect.~\ref{Observation}).
\item
  making a \emph{decision} (Sect.~\ref{MakeDecision}).
\end{itemize}

\subsection{Making an assumption about the parameter}\label{Assumption}

\index{Decision making!assumption}

The initial assumption is that sampling variation explains any discrepancy between the values of the parameter and the statistic. This assumption about the value of the parameter is called the \emph{null hypothesis},\index{Hypotheses!null} denoted~\(H_0\).

The null hypothesis is always that sampling variation explains the difference between the observed value of the statistic and the assumed value of the parameter. Depending on the RQ and the context, this may mean:

\begin{itemize}
\tightlist
\item
  the parameter value has not changed (e.g., for descriptive or repeated-measures RQs).\index{Research question!descriptive}\index{Research question!repeated-measures} The value of the statistic might show a change, but only due to sampling variation.
\item
  the value of some parameter is the same in all the groups being compared (e.g., for relational RQs).\index{Research question!relational} The values of the statistic are not exactly the same due to sampling variation.
\item
  there is no relationship between the variables, as measured by some parameter (e.g., for correlational RQs).\index{Research question!correlational} The value of the statistic is not exactly this value due to sampling variation.
\end{itemize}

In other words, the null hypothesis is the `no change, no difference, no relationship' position.

Using this idea, a reasonable assumption can be made about the parameter.\index{Parameter} For example, when comparing the mean of two groups, we would initially assume \emph{no difference} between the \emph{population} means: any difference between the \emph{sample} means would be attributed to sampling variation.

\begin{example}[Assumptions about the population]
\protect\hypertarget{exm:Assumptions}{}\label{exm:Assumptions}Many dental associations recommend brushing teeth for two minutes. \citet{data:Macgregor1979:BrushingDurationKids} recorded the toothbrushing time for \(85\) uninstructed schoolchildren from England to assess compliance with these guidelines.

We initially \emph{assume} the \emph{population} mean toothbrushing time is two minutes (\(H_0\):~\(\mu = 2\)). If the \emph{sample} mean is not two minutes, the null hypothesis explains this discrepancy as sampling variation. A sample can then be obtained to determine if the sample mean is consistent with, or contradicts, this assumption.
\end{example}

\subsection{Describing the expectations of the statistic}\label{ExpectationOf}

\index{Decision making!expectation}\index{Statistic}

Based on the assumed value for the parameter, we then determine \emph{what values to expect from the statistic} from all the possible samples we could select (of which we only select one). Since many samples are possible, and every sample is likely to be different (sampling variation), the observed value of the statistic depends on which one of the countless possible samples we obtain. To know what values of the statistic are expected, the \emph{sampling distribution} needs to be described.

Think about the cards in Sect.~\ref{NeedForDecisionMaking}. In a fair pack, \emph{half} the cards are red in the \emph{population} (the pack of cards), so the population proportion is assumed to be \(p = 0.5\). In a \emph{sample} of \(25\)~cards, what values could be reasonably expected for the \emph{sample} proportion~\(\hat{p}\) of red cards (the statistic)? If samples of size \(n = 25\) were repeatedly taken from a fair pack with \(p = 0.5\), the sample proportion of red cards would vary from hand to hand, of course. But \emph{how} would~\(\hat{p}\) vary from sample to sample?

Suppose we simulated only ten hands of \(n = 25\) cards each, using a computer; Fig.~\ref{fig:RollDice10} shows the sample proportion of red cards from each repetition. Naturally, the value of \(\hat{p}\) varies.

\begin{figure}[hbtp]

{\centering \includegraphics[width=0.75\linewidth]{25-Tools-DecisionMaking_files/figure-latex/RollDice10-1} 

}

\caption{Ten hands of $25$ cards: the sample proportion of cards in each hand that is red (shown on the right-hand side) varies from hand to hand.}\label{fig:RollDice10}
\end{figure}

The distribution of the sample statistic is called the \emph{sampling distribution} (Sect.~\ref{SamplingVariationIntro}). The sampling distribution for \(\hat{p}\) is given in Sect.~\ref{SamplingDistributionKnownp} when the value of \(p\) is known (as assumed here): an approximate normal distribution, with mean~\(p = 0.5\), and a standard deviation (the \emph{standard error}) of \[
  \text{s.e.}(\hat{p}) 
  =
  \sqrt{\frac{p\times(1 - p)}{n} }
  =
  \sqrt{\frac{0.5\times(1 - 0.5)}{25} }
  =
  0.1.
\] A picture of this sampling distribution can be drawn (Fig.~\ref{fig:SamplingDistCards}). Values of \(\hat{p}\) larger than~\(0.8\) or smaller than~\(0.2\) would appear very unlikely.

\begin{figure}[hbtp]

{\centering \includegraphics[width=0.85\linewidth]{25-Tools-DecisionMaking_files/figure-latex/SamplingDistCards-1} 

}

\caption{The sampling distribution for $\hat{p}$, the sample proportion of red cards in $25$ cards.}\label{fig:SamplingDistCards}
\end{figure}

\subsection{Evaluating the sample observations}\label{Observation}

\index{Decision making!observations}

While many samples are possible, we only observe \emph{one} of those countless possible samples. From our sample of \(25\)~cards, all cards are red (Fig.~\ref{fig:Draw15Cards}), and so the value of the statistic is \(\hat{p} = 25/25 = 1\). Assuming \(p = 0.5\), is this value of~\(\hat{p}\) unusual, or not unusual? From Fig.~\ref{fig:SamplingDistCards}, the value \(\hat{p} = 1\) is \emph{very} unusual: it would not be expected in a sample of \(25\)~cards.

\subsection{Making a conclusion}\label{MakeDecision}

\index{Decision making!decision}

Observing \(25\)~red cards out of \(25\)~cards from a fair pack is highly unusual, so the chance that our specific, randomly-chosen sample produced \(\hat{p} = 1\) is incredibly unlikely. So you could reasonably conclude that finding \(\hat{p} = 1\) almost never occurs, \emph{if the assumption of a fair pack was true}.

But since we \emph{did} find \(\hat{p} = 1\), the assumption of a fair pack is probably wrong. That is, there is persuasive evidence that the assumption is wrong, and that the pack of cards is not fair.\index{Evidence-based research} The decision-making process is shown in Fig.~\ref{fig:DecisionFlowCards}.

\begin{figure}[hbtp]

{\centering \includegraphics[width=1\linewidth]{25-Tools-DecisionMaking_files/figure-latex/DecisionFlowCards-1} 

}

\caption{A way to make decisions for the cards example.}\label{fig:DecisionFlowCards}
\end{figure}

What if we had observed \(18\)~red cards in a hand of \(25\)~cards, a sample proportion of \(\hat{p} = 18/25 = 0.72\)? The conclusion is not quite so obvious then: Fig.~\ref{fig:SamplingDistCards} shows that \(\hat{p} = 0.72\) is unlikely, but \(\hat{p} = 0.72\) (and even larger values) certainly do occur.

What if \(15\)~red cards were found in the~\(25\) (i.e., \(\hat{p} = 0.6\))? Figure~\ref{fig:SamplingDistCards} shows that \(\hat{p} = 0.6\) could reasonably be observed, since there are many possible samples that lead to \(\hat{p} = 0.6\), or even larger values. This would not seem unusual at all, and is certainly not persuasive evidence to change our mind. Many of the possible samples produce values of~\(\hat{p}\) near~\(0.6\).

This process of decision-making is similar to the process used in research. This process will be studied in coming chapters.

\section{Example: brushing teeth}\label{ToolsDecisionMakingExample}

Many dental associations recommend brushing teeth for two minutes (i.e., for \(120\,\text{s}\)). \citet{data:Macgregor1979:BrushingDurationKids} recorded the toothbrushing time for \(85\)~uninstructed schoolchildren from England to assess compliance with these guidelines. Of course, every possible sample of \(85\) children in England will include different children, and so produce various sample mean brushing times~\(\bar{x}\). Even if the \emph{population} mean toothbrushing time really was \(120\,\text{s}\) (i.e., \(\mu = 120\)), the \emph{sample} mean probably wouldn't be exactly \(120\,\text{s}\), because of this sampling variation.

To begin, \emph{assume} the population mean toothbrushing time is \(\mu = 120\); that is, \(H_0\) is \(\mu = 120\). \emph{If} this is true, we then could describe what values of the sample statistic \(\bar{x}\) could be \emph{expected} from all possible samples by describing the sampling distribution of~\(\bar{x}\): how sample means are likely to vary for samples of size \(85\) when \(\mu = 120\).

The study found the mean time spent brushing was \(60.3\,\text{s}\), with a standard deviation of \(23.8\,\text{s}\). Using the ideas in Chap.~\ref{OneMeanConfInterval}, and in Sect.~\ref{SamplingDistSampleMean} specifically, the sampling distribution of \(\bar{x}\) has an approximate normal distribution, with mean \(\mu = 120\) (from \(H_0\)) and a standard deviation of \(\text{s.e.}(\bar{x}) = 23.8/\sqrt{85} = 2.58\,\text{s}\) (shown in Fig.~\ref{fig:SamplingDistBrushing}).

A sample mean of \(\bar{x} = 60.3\) seems incredibly unlikely if \(\mu = 120\). This suggests that the sample evidence \emph{contradicts} the assumption that \(\mu = 120\), and so the mean toothbrushing time in the population is very unlikely to be \(120\,\text{s}\).

\begin{figure}[hbtp]

{\centering \includegraphics[width=0.85\linewidth]{25-Tools-DecisionMaking_files/figure-latex/SamplingDistBrushing-1} 

}

\caption{The sampling distribution for $\bar{x}$, the mean toothbrushing time in schoolchildren from England. A sample mean of $60.3\,\text{s}$ seems very unlikely.}\label{fig:SamplingDistBrushing}
\end{figure}

\section{Chapter summary}\label{ToolsDecisionMakingSummary}

Making decisions about parameters based on a statistic is challenging: only one of the many possible samples is observed. Since every sample is likely to be different, different values of the sample statistic are possible. In this chapter, though, a process for making decisions has been studied (Fig.~\ref{fig:DecisionFlow2}).

Decisions are often made by making an \emph{assumption} about the parameter, which leads to an \emph{expectation} of what values of the statistic are reasonably possible. We can then make \emph{observations} about our sample, and then make a \emph{decision} about whether the sample data support or contradict the initial assumption.

\section{Quick review questions}\label{ToolsDecisionMakingQuickReview}

Are the following statements \emph{true} or \emph{false}?

\begin{enumerate}
\def\labelenumi{\arabic{enumi}.}
\item
  Parameters describe \emph{populations}.\tightlist  
\item
  Both \(\bar{x}\) and \(\mu\) are \emph{statistics}.
\item
  The value of a statistic is likely to be \emph{same} in every sample.
\item
  \emph{Sampling variation} describes how the value of a \emph{statistic} varies from sample to sample.
\item
  An initial assumption is made about the \emph{sample statistic}.
\item
  If the sample results seem inconsistent with what was expected, then the assumption about the population is probably true.
\item
  In the sample, we know exactly what value of~\(\hat{p}\) to expect.
\item
  Hypotheses are made about the population.
\end{enumerate}

\section{Exercises}\label{MakingDecisionsExercises}

\hyperref[Answers]{Answers to odd-numbered exercises} are given at the end of the book.

\captionsetup{font=small}

\begin{exercise}
\protect\hypertarget{exr:MakingDecisionsDice}{}\label{exr:MakingDecisionsDice}

While playing a die-based game, your opponent rolls a \largedice{6} ten times in a row.

\begin{enumerate}
\def\labelenumi{\arabic{enumi}.}
\tightlist
\item
  Do you think there is a problem with the die?
\item
  Explain how you came to this decision.
\end{enumerate}

\end{exercise}

\begin{exercise}
\protect\hypertarget{exr:MakingDecisionsCoin}{}\label{exr:MakingDecisionsCoin}

A friend tosses a coin, and obtains \Heads~two times in a row.

\begin{enumerate}
\def\labelenumi{\arabic{enumi}.}
\tightlist
\item
  Do you think there is a problem with the coin?
\item
  Explain how you came to this decision.
\end{enumerate}

\end{exercise}

\begin{exercise}
\protect\hypertarget{exr:MakingDecisionsJuryService}{}\label{exr:MakingDecisionsJuryService}Since my wife and I have been married, I have been called to jury service four times. The latest notice reads: `Your name has been selected at random from the electoral roll'.

In the same time, my wife has \emph{never} been called to jury service. Do you think the selection process really is `at random'? Explain.
\end{exercise}

\begin{exercise}
\protect\hypertarget{exr:MakingDecisionsClaim}{}\label{exr:MakingDecisionsClaim}

In a 2012 advertisement, an Australian pizza company claimed that their \(12\)-inch pizzas were `real \(12\)-inch pizzas' \citep{mypapers:Dunn:PizzaSize}.

\begin{enumerate}
\def\labelenumi{\arabic{enumi}.}
\tightlist
\item
  What is a reasonable assumption to make to test this claim?
\item
  The claim is supported by a sample of \(125\) pizzas, which gave the sample mean pizza diameter as \(\bar{x} = 11.48\) inches. What are the two reasons why the sample mean is not exactly \(12\)-inches?
\item
  Does the claim appear to be supported by, or contradicted by, the data? Explain.
\item
  Would your conclusion change if the sample mean was \(\bar{x} = 11.25\)~inches? Explain.
\item
  Does your answer depend on the sample size? For example, is observing a sample mean of \(11.25\)~inches from a sample of size \(n = 10\) equivalent to observing a sample mean of \(11.25\)~inches from a sample of size \(n = 125\)? Explain.
\end{enumerate}

\end{exercise}

\begin{exercise}
\protect\hypertarget{exr:MakingDecisionsOlder}{}\label{exr:MakingDecisionsOlder}

Suppose that \(36\)\% of all students at a certain large university are aged over~\(30\). A student takes a sample of \(n = 40\)~students from the School of Arts to determine if students in that school are somehow different from the general university population in terms of age.

\begin{enumerate}
\def\labelenumi{\arabic{enumi}.}
\tightlist
\item
  What is the null hypothesis?
\item
  If the student researcher finds \(13\)~students in the sample aged over~\(30\), does this present persuasive evidence to change your mind? Explain.
\item
  If the student researcher finds three students in the sample aged over~\(30\), does this present persuasive evidence to change your mind? Explain.
\end{enumerate}

\end{exercise}

\captionsetup{font=normalsize}

\begin{EOCanswerBox}{iconmonstr-check-mark-14-240.png}
\textbf{Answers to \emph{Quick review} questions:} \textbf{1.} True. \textbf{2.} False. \textbf{3.} False. \textbf{4.} True. \textbf{5.} False. \textbf{6.} False. \textbf{7.} False. \textbf{8.} True.

\end{EOCanswerBox}

\chapter{Hypothesis tests: one proportion}\label{TestOneProportion}

\begin{cols}
\begin{col}{0.52\textwidth}

\begin{objectivesBox}{iconmonstr-target-4-240.png}
So far, you have learnt to ask an RQ, design a study, classify and summarise the data, and form confidence intervals.
\textbf{In this chapter}, you will learn to:
  
\begin{itemize}\tightlist
  \item
  identify situations where conducting a test for a proportion is appropriate.
  \item
  conduct hypothesis tests for one sample proportion, using a $z$-test.
  \item
  determine whether the conditions for using these methods apply in a given situation.
\end{itemize}
\end{objectivesBox}

\end{col}

\begin{col}{0.03\textwidth}
~
\end{col}

\begin{col}{0.45\textwidth}

\includegraphics[width=0.95\linewidth]{26-Testing-OneProportion_files/figure-latex/unnamed-chunk-5-1} 
\end{col}
\end{cols}

\section{Introduction: rolling dice}\label{ProportionTestIntro}

\index{Hypothesis testing!one proportion|(}

\captionsetup[wrapfigure]{margin=8pt}
\begin{wrapfigure}[5]{R}{.35\textwidth} % The first optional input is the number of lines allowed for the image to be placed in
  \centering%
  \vspace{-30pt}% This removes some white space
  \includegraphics[width=.32\textwidth]{OtherImages/SmiffyDice-Rotated.png}%
  \caption{The packaging (Photo: P.\ K.\ Dunn).}\label{fig:DodgyDice}
\end{wrapfigure}

When in a toy store\index{Toy store} one day (for my children, of course), I saw `loaded dice'\index{Loaded dice} for sale (Fig.~\ref{fig:DodgyDice}).~ The packaging claimed \textsc{one loaded \& one normal}. I bought two sets! However, there was no indication as to \emph{which} die was the loaded die. How could I determine which of the dice was loaded? That is, how could I make a \emph{decision} about which die was loaded?

For a die that is \emph{not} loaded, the population proportion of rolling any face of the die is \(p = 1/6\). So, for example, the population proportion of rolls that show a \largedice{1} is \(p = 1/6\), using the classical approach to probability.\index{Probability!classical approach} In any \emph{sample} of rolls, however, the proportion of rolls showing a \largedice{1} would vary due to sampling variation, but would be approximately \(\hat{p} = 1/6\) with a fair die.

Suppose I rolled one die a certain number of times (say, \(n = 50\)~times), then determined the value of the sample proportion~\(\hat{p}\), the sample proportion of rolls that show a \largedice{1}. It is unlikely that the value of~\(\hat{p}\) will be \emph{exactly} \(1/6\) (the population proportion). If the observed value of \(\hat{p}\) was not exactly \(1/6\), two possible reasons could explain this discrepancy between the value of the statistic and the assumed value of the parameter (Chap.~\ref{MakingDecisions}):

\begin{itemize}
\tightlist
\item
  I was rolling the \emph{fair} die (with \(p = 1/6\)), and the discrepancy between the values of the \emph{population} and \emph{sample} proportions was simply due to sampling variation.
\item
  I was rolling the \emph{loaded} die (with \(p \ne 1/6\)), and the discrepancy between the values of the \emph{population} and \emph{sample} proportion simply reflected this.
\end{itemize}

If I observed an unusually small or unusually large sample proportion of rolls that showed a \largedice{1}, I would suspect that I had the loaded die: I was observing something unusual if I had rolled the fair die. This is exactly the decision-making process seen in Chap.~\ref{MakingDecisions}.

More formally then, the decision-making process (Chap.~\ref{MakingDecisions}) could proceed as follows.

\begin{itemize}
\tightlist
\item
  Make an \emph{assumption} about the parameter: assume I have a fair die, so that \(p = 1/6\), where \(p\) is the population proportion of rolls that show a \largedice{1}.
\item
  Describe the \emph{expectations} of the statistic: describe what value of the \emph{sample} proportion \(\hat{p}\) could reasonably be expected from a fair die in \(50\)~rolls.
\item
  Evaluate the sample \emph{observations}: roll the die \(50\)~times to find a value of \(\hat{p}\) and compare to what was expected.
\item
  Make a \emph{decision} based on what is observed in the sample.
\end{itemize}

Using this decision-making process (Fig.~\ref{fig:DecisionFlowDice}), I could decide if the die I had rolled seemed to be the fair die (based on rolling a \largedice{1}; the die may be loaded in a different way, of course). For one specific die, I am asking the decision-making RQ:

\begin{quote}
For this die, is the population proportion of rolls that show a \largedice{1} equal to~\(1/6\)?
\end{quote}

Answering a decision-making RQ such as this requires a \emph{hypothesis test}.

\begin{figure}[hbtp]

{\centering \includegraphics[width=1\linewidth]{26-Testing-OneProportion_files/figure-latex/DecisionFlowDice-1} 

}

\caption{A way to make decisions for the dice example.}\label{fig:DecisionFlowDice}
\end{figure}

\begin{tipBox}{iconmonstr-info-6-240.png}
\(p\) refers to the \emph{population} proportion, and~\(\hat{p}\) refers to a \emph{sample} proportion.

\end{tipBox}

\section{\texorpdfstring{Rolling dice: the sampling distribution of \(\hat{p}\)}{Rolling dice: the sampling distribution of \textbackslash hat\{p\}}}\label{SamplingDistributionKnownpHT}

\index{Sampling distribution!one proportion (for testing)}

When a fair, six-sided die is rolled \(50\)~times, what proportion of the rolls will produce a \largedice{1}? That is, what will be the value of the \emph{sample proportion} \(\hat{p}\)? Of course, no-one knows, because the sample proportion will not be the same for every sample of \(50\)~rolls. The sample proportion \emph{varies} from sample to sample: \emph{sampling variation} exists and is described by the \emph{sampling distribution}.

\begin{importantBox}{iconmonstr-warning-8-240.png}
Remember: studying a sample leads to the following observations: \vspace{-2ex}

\begin{itemize}
\tightlist
\item
  every sample is likely to be different.
\item
  we observe just one of the many possible samples.
\item
  every sample is likely to yield a different value for the statistic.
\item
  we observe just one of the many possible values for the statistic. \vspace{-2ex}
\end{itemize}

Since many values for the sample proportion are possible, the values of the sample proportion vary (called \emph{sampling variation}) and have a \emph{distribution} (called a \emph{sampling distribution}).

\end{importantBox}

The sampling distribution of~\(\hat{p}\) was described in Def.~\ref{def:SamplingDistPropCI} (and repeated in Def.~\ref{def:SamplingDistPropHT} below). The sample proportions are described by

\begin{itemize}
\tightlist
\item
  an approximate normal distribution,
\item
  centred around the \emph{sampling mean}, with a value of \(p = 1/6\) (assumed, from \(H_0\)),
\item
  with a standard deviation, called the \emph{standard error} \(\text{s.e.}(\hat{p})\), of \begin{equation}
     \text{s.e.}(\hat{p}) 
     = \sqrt{ \frac{p\times(1 - p)}{n} }
     = \sqrt{\frac{ \frac{1}{6} \times \left(1 - \frac{1}{6}\right)}{50}}
     = 0.0527.
    \label{eq:StdErrorExampleDieHT}
  \end{equation}
\end{itemize}



\begin{definition}[Sampling distribution of a sample proportion with $p$ known]
\protect\hypertarget{def:SamplingDistPropHT}{}\label{def:SamplingDistPropHT}

For a known value of \(p\), the \emph{sampling distribution of the sample proportion} is (when certain conditions are met; Sect.~\ref{ValidityProportions}) described by

\begin{itemize}
\tightlist
\item
  an approximate normal distribution,
\item
  centred around the sampling mean whose value is~\(p\),
\item
  with a standard deviation (called the \emph{standard error} of~\(\hat{p}\)), denoted \(\text{s.e.}(\hat{p})\), whose value is \begin{equation}
   \text{s.e.}(\hat{p}) = \sqrt{\frac{ p \times (1 - p)}{n}},
   \label{eq:StdErrorPknownHT}
  \end{equation} where~\(n\) is the size of the sample used to compute~\(\hat{p}\), and~\(p\) is the population proportion.
\end{itemize}

\end{definition}

A picture of this normal distribution can be drawn (Fig.~\ref{fig:NormalDieTheoryHT}); the standard error is the standard deviation of the normal distribution in Fig.~\ref{fig:NormalDieTheoryHT}. While we still don't know \emph{exactly} what values of \(\hat{p}\) the next set of \(n = 50\)~rolls will produce, we have some idea of \emph{how} the sample proportion varies in samples of \(50\)~rolls. For instance, values of~\(\hat{p}\) greater than about~\(0.35\) are unlikely to be observed from a fair die (with \(p = 1/6\)).



\begin{figure}[hbtp]

{\centering \includegraphics[width=1\linewidth]{26-Testing-OneProportion_files/figure-latex/NormalDieTheoryHT-1} 

}

\caption{The sampling distribution is an approximate normal distribution; it shows a model of how the proportion of rolls showing a \largedice{1} varies, when a die is rolled \(50\) times. The cross represents the observed sample proportion, \(\hat{p} = 0.38\).}\label{fig:NormalDieTheoryHT}
\end{figure}

\section{Rolling dice: making a decision}\label{TestpObsDecision}

Figure~\ref{fig:NormalDieTheoryHT} shows what values of the sample proportion \(\hat{p}\) are expected when a fair die is rolled. Step~3 of the decision-making process (Fig.~\ref{fig:DecisionFlowDice}) is to roll the die.

When I rolled the die, a \largedice{1} appeared \(19\)~times in my \(50\)~rolls, a sample proportion of \[
\hat{p} = \frac{19}{50} = 0.38.
\] In this unusual or unexpected? Locating this value of \(\hat{p}\) on the sampling distribution in Fig.~\ref{fig:NormalDieTheoryHT} shows that a sample proportion of \(\hat{p} = 0.38\) is \emph{highly} unusual from a fair die with \(p = 1/6\). More specifically, since the sampling distribution has a normal distribution, a \(z\)-score can be computed: \[
  z 
  = \frac{\text{statistic} - \text{mean of the distribution}}{\text{std dev. of the distribution}}
  = \frac{0.38 - (1/6)}{0.05270}
  = 4.05,
\] which is a \emph{very} large \(z\)-score (based on the \(68\)--\(95\)--\(99.7\) rule).\index{68@$68$--$95$--$99.7$ rule} Using a fair die, observing \(\hat{p} = 0.38\) would almost never occur. But I \emph{did} observe \(\hat{p} = 0.38\), which suggests that the die I was rolling was \emph{not} the fair die.

I concluded that the die I was rolling was loaded (that is, \(p \ne 1/6\)). I may be incorrect (after all, it is not \emph{impossible} to observe \(\hat{p} = 0.38\)), but the evidence is certainly persuasive. Using the decision-making process, a decision has been made about the die.

The process described above is called \emph{hypothesis testing}.\index{Hypothesis testing} Hypothesis testing is used to make decisions about a population after observing just one of the countless possible samples. Formally, the hypothesis test above proceeds as described in the following sections.

\section{Assumptions: hypotheses}\label{TestpObsDecisionHypothesis}

\index{Test statistic!z@$z$-score}

\textbf{Step~1} in the decision-making process is to make an assumption about the parameter. For the die example, the parameter is~\(p\), the population proportion of rolls that show a \largedice{1}. The assumption is that \(p = 1/6\). This is called the \emph{null hypothesis},\index{Hypotheses!null} denoted by \(H_0\): \[
  \text{$H_0$: } p = 1/6.
\] The null hypothesis states the value of~\(p\) is \(1/6\); in other words, if the sample proportion~\(\hat{p}\) is not equal to \(1/6\), the discrepancy is explained by sampling variation. The null hypothesis is always the `sampling variation' explanation for the discrepancy between the values of the statistic and the parameter (Sect.~\ref{Assumption}).

The other explanation for why the value of the sample proportion \(\hat{p}\) is not equal to \(1/6\) is called the \emph{alternative hypothesis} (denoted \(H_1\)),\index{Hypotheses!alternative} that the population proportion is \emph{not} \(1/6\), and this is the cause of the discrepancy between the values of the statistic and the parameter: \[
  \text{$H_1$: } p \ne 1/6.
\] These two hypotheses offer different explanations for the discrepancy between the values of the population proportion (the parameter) and the sample proportion (the statistic). The null hypothesis \(H_0\) states that \(p = 1/6\) and the discrepancy is due to sampling variation. The alternative hypothesis \(H_1\) states that \(p \ne 1/6\), which explains the discrepancy.

Here, the RQ here is open to the value of \(p\) being smaller \emph{or} larger than~\(1/6\); that is, two possibilities are considered. Hence, we write \(p\ne 1/6\), which is called a \emph{two-tailed} alternative hypothesis. Alternative hypotheses like \(p > 1/6\) (the population proportion is \emph{larger} than \(1/6\)) or \(p < 1/6\) (the population proportion is \emph{smaller} than \(1/6\)) are \emph{one-tailed} hypothesis.

\begin{importantBox}{iconmonstr-warning-8-240.png}
The form of the alternative hypothesis (either one- or two-tailed) depends on what the research question asks, \emph{not the data}.

\end{importantBox}

\section{\texorpdfstring{Expectations: sampling distribution for \(\hat{p}\)}{Expectations: sampling distribution for \textbackslash hat\{p\}}}\label{TestpObsDecisionSamplingDist}

\textbf{Step~2} in the decision-making process is to describe what values of the statistic (i.e., \(\hat{p}\)) could be expected under the assumption about the parameter (i.e., \emph{when the null hypothesis is true}). Hypothesis testing \emph{always} begins by assuming the null hypothesis is true.

\begin{importantBox}{iconmonstr-warning-8-240.png}
The decision-making process begins by assuming the \emph{null hypothesis} is true. Thus, \emph{the onus is on the data to refute the null hypothesis, the initial assumption}.

That is, the null hypothesis is retained unless persuasive evidence emerges to change our mind.

\end{importantBox}

Effectively, this step requires describing the sampling distribution of the statistic. For the die example, the sampling distribution for \(\hat{p}\) is (see Def.~\ref{def:SamplingDistPropHT}):

\begin{itemize}
\tightlist
\item
  an approximate normal distribution,
\item
  centred around the sampling mean whose value is~\(p = 1/6\),
\item
  with a standard deviation, whose value is \(\text{s.e.}(\hat{p}) = 0.05270\dots\)
\end{itemize}

Drawing the picture of the sampling distribution (like Fig.~\ref{fig:NormalDieTheoryHT}) using this information is not necessary, but may be helpful.

\section{\texorpdfstring{Observations: \(z\)-score}{Observations: z-score}}\label{TestpObsDecisiontestStat}

\textbf{Step~3} in the decision-making process is to evaluate the observations. As noted above, a \largedice{1} was observed in~\(19\) of the \(50\)~rolls, so \(\hat{p} = 0.38\). Since the sampling distribution has a normal distribution, the corresponding \(z\)-score was computed as \(z = 4.05\), which very large.

In hypothesis testing, the \(z\)-score is called the \emph{test statistic}.\index{Test statistic}\index{Test statistic!z@$z$-score} The test statistic measures how far, in relative terms, the sample proportion is from the assumed value of the parameter.

\section{\texorpdfstring{Decision: \(P\)-value}{Decision: P-value}}\label{TestpObsDecisionPvalues}

\textbf{Step~4} of the decision-making process is to use the information to make a decision: is the sample statistic \emph{consistent} with what was expected under the assumption that \(p = 1/6\), or does it \emph{contradict} what was expected?

For the die example, the decision is reasonably easy: \(z = 4.05\) is \emph{very} large and \emph{very} unlikely to be observed if \(p = 1/6\). This means the sample evidence \emph{contradicts} what was expected if the assumption was true: persuasive evidence exists that the die is loaded.

More generally, evidence is evaluated using a \(P\)-value.\index{P@$P$-values} \(P\)-values refer to the area \emph{more extreme} than the calculated test statistic in the sampling distribution.

For this situation, where the sampling distribution has a normal distribution, \(P\)-values refer to the area \emph{more extreme} than the calculated \(z\)-score (the statistic)\index{Statistic} in the normal distribution; that is, the area in the \emph{tails} of the distribution (see Fig.~\ref{fig:OnePropTestP}). This is a way to measure how unusual the calculated \(z\)-score is.

For \emph{two-tailed} alternative hypotheses, the \(P\)-value is the combined area in the lower and upper tails that correspond to the positive \emph{and} negative values of the test statistic. For \emph{one-tailed} alternative hypotheses, the \(P\)-value is the area in one tail only. Clearly, since the \(P\)-value is a probability, its value is always between~\(0\) and~\(1\).

Since the sampling distribution has a normal distribution in this example, \(P\)-values can be approximated using the \(68\)--\(95\)--\(99.7\) rule and a diagram (Sect.~\ref{ApproxProbs}; Sect.~\ref{OnePropTestP6895997}), or more precisely using the \(z\)-tables in Appendices~\ref{ZTablesNEG} and \ref{ZTablesPOS} (Sect.~\ref{ZScoreForestry}; Sect.~\ref{OnePropTestPTables}). \(P\)-values are also reported by software for most statistical tests.

\subsection{\texorpdfstring{Approximating \(P\)-values using the \(68\)--\(95\)--\(99.7\) rule}{Approximating P-values using the 68--95--99.7 rule}}\label{OnePropTestP6895997}

\index{68@$68$--$95$--$99.7$ rule}\index{P@$P$-values!estimating using $68$--$95$--$99.7$ rule}

The \(68\)--\(95\)--\(99.7\) rule can be used to determine \emph{approximate} \(P\)-values. To demonstrate, suppose the computed \(z\)-score was \(z = 1\). Then, the two-tailed \(P\)-value is the shaded tail-area in Fig.~\ref{fig:OnePropTestP} (top left panel): about~\(32\)\%, based on the \(68\)--\(95\)--\(99.7\) rule. The two-tailed \(P\)-value would be the same if \(z = -1\) (as both tails are of interest). The \emph{one-tailed} \(P\)-value would be the area in one-tail (Fig.~\ref{fig:OnePropTestP}, bottom left panel): about~\(16\)\%, based on the \(68\)--\(95\)--\(99.7\) rule.

As another example, suppose the calculated \(z\)-score was \(z = -2\). Then, the two-tailed \(P\)-value is the shaded area shown in Fig.~\ref{fig:OnePropTestP} (top right panel): about~\(5\)\%, based on the \(68\)--\(95\)--\(99.7\) rule. The two-tailed \(P\)-value would be the same if \(z = 2\). The \emph{one-tailed} \(P\)-value would be the area in one tail only (Fig.~\ref{fig:OnePropTestP}, bottom right panel): about~\(2.5\)\%, based on the \(68\)--\(95\)--\(99.7\) rule.

\begin{figure}[hbtp]

{\centering \includegraphics[width=1\linewidth]{26-Testing-OneProportion_files/figure-latex/OnePropTestP-1} 

}

\caption{The two-tailed $P$-value is the combined area in the two tails of the distribution; the one-tailed $P$-value is the area in one tail only. Top left panel: if $z = 1$ (or $z = -1$), the two-tailed $P$-value is approximately $0.16$. Top right panel: if $z = 2$ (or $z = -2$), the two-tailed $P$-value is approximately $0.05$. The corresponding one-tailed $P$-values are half the two-tailed $P$-values, and are shown in the bottom panels.}\label{fig:OnePropTestP}
\end{figure}

Of course, calculated \(z\)-scores are unlikely to be exactly \(z = 1\) or \(z = -2\). Suppose the \(z\)-score is a little \emph{larger} than \(z = 1\); say \(z = 1.2\). Then, the two-tailed area will be a little \emph{smaller} than the tail area when \(z = 1\) (Fig.~\ref{fig:OnePropTestP2}, left panel). The two-tailed \(P\)-value is a little \emph{smaller} than~\(0.32\).

Similarly, suppose the \(z\)-score is not quite equal to \(z = -2\); say \(z = -1.9\). Then, the two-tailed area will be a little \emph{larger} than the tail area when \(z = -2\) (Fig.~\ref{fig:OnePropTestP2}, right panel). The two-tailed \(P\)-value is a little \emph{larger} than~\(0.05\).

\begin{figure}[hbtp]

{\centering \includegraphics[width=0.95\linewidth]{26-Testing-OneProportion_files/figure-latex/OnePropTestP2-1} 

}

\caption{The two-tailed $P$-value for $z$-scores not aligned with the $68$--$95$--$99.7$ rule. Left panel: when $z = 1.2$ (or $z = -1.2$). Right panel: when $z = 1.9$ (or $z = -1.9$).}\label{fig:OnePropTestP2}
\end{figure}

\subsection{\texorpdfstring{More precise \(P\)-values using tables}{More precise P-values using tables}}\label{OnePropTestPTables}

\index{P@$P$-values!using tables}

Using the tables of areas under normal distributions (Appendices~\ref{ZTablesNEG} and \ref{ZTablesPOS}), more precise \(P\)-values can be found using the ideas from Sect.~\ref{ExactAreasUsingTables}. For instance (see Fig.~\ref{fig:OnePropTestP2}):

\begin{itemize}
\tightlist
\item
  for \(z = 1.2\): the area to the \emph{left} of \(z = -1.2\) is~\(0.1151\), and the area to the \emph{right} of \(z = 1.2\) is~\(0.1151\), so the \emph{two-tailed} \(P\)-value is \(0.1151 + 0.1151 = 0.2302\). This is a little smaller than~\(0.32\), as estimated above.
\item
  for \(z = 1.9\): the area to the \emph{left} of \(z = -1.9\) is~\(0.0287\), and the area to the \emph{right} of \(z = 1.9\) is~\(0.0287\), so the \emph{two-tailed} \(P\)-value is \(0.0287 + 0.0287 = 0.0574\). This is a little larger than~\(0.05\), as estimated above.
\end{itemize}

In this die-rolling example, where \(z = 4.05\), the tail area is \emph{very} small (using Appendices~\ref{ZTablesNEG} and~\ref{ZTablesPOS}), and zero to four decimal places. \(P\)-values are never exactly zero, so we write \(P < 0.0001\) (that is, the \(P\)-value is \emph{less than}~\(0.0001\)).

\(P\)-values tells us the probability of observing the sample statistic (or a value even more extreme), assuming the null hypothesis is true. In the die-rolling example, the \(P\)-value is the probability of observing the value of \(\hat{p} = 0.38\) (or a more extreme value), just through sampling variation if \(p = 1/6\). Then (see Fig.~\ref{fig:PvaluesBigSmall}):

\begin{itemize}
\tightlist
\item
  `big' \(P\)-values mean the sample statistic (i.e., \(\hat{p}\)) could reasonably have occurred through sampling variation in one of the many possible samples, if the assumption made about the parameter (stated in \(H_0\)) was true; the data \emph{do not} contradict the assumption in~\(H_0\). There \emph{is no} persuasive evidence to support the alternative hypothesis.
\item
  `small' \(P\)-values mean the sample statistic (i.e., \(\hat{p}\)) is \emph{unlikely} to have occurred through sampling variation in one of the many possible samples, if the assumption made about the parameter (stated in~\(H_0\)) was true; the data \emph{do} contradict the assumption in \(H_0\). There \emph{is} persuasive evidence to support the alternative hypothesis.
\end{itemize}

\begin{figure}[hbtp]

{\centering \includegraphics[width=1\linewidth]{26-Testing-OneProportion_files/figure-latex/PvaluesBigSmall-1} 

}

\caption{The strength of evidence: $P$-values. As the $z$-score becomes larger, the $P$-value becomes smaller, and it is more likely that the evidence contradicts the null hypothesis.}\label{fig:PvaluesBigSmall}
\end{figure}

What is meant by `small' and `big' in this context? What represents persuasive evidence to support the alternative hypothesis? A \(P\)-value smaller than~\(5\)\% (or~\(0.05\)) is usually considered `small', and persuasive evidence to support the alternative hypothesis. In contrast, a \(P\)-value larger than~\(5\)\% (or~\(0.05\)) is usually considered `big', and \emph{not} persuasive evidence to support the alternative hypothesis.

\begin{importantBox}{iconmonstr-warning-8-240.png}
The value of~\(0.05\) given here is \emph{arbitrary}, and in some disciplines the distinction is made when \(P = 0.01\) or \(P = 0.10\) instead.

\end{importantBox}

Rather than having an arbitrary boundary between `big' and `small', a more sensible approach is to qualify the strength of the evidence that supports the alternative hypotheses (discussed further in Sect.~\ref{AboutPvalues}).

In this die-rolling example, where the \(P\)-value is \emph{very} small, the data contradict the null hypothesis that \(p = 1/6\): the evidence supports the alternative hypothesis that \(p \ne 1/6\). This suggests that the die is very likely \emph{not} fair.

\begin{importantBox}{iconmonstr-warning-8-240.png}
\emph{Be careful interpreting the results!} We cannot be \emph{sure} that the die is unfair. \emph{A small \(P\)-value is not proof that the die is loaded.} The die may be fair but, due to sampling variation, the sample we observed may simply have produced an unusually high proportion of rolls that show a \largedice{1} by chance.

The result is interpreted as `there is strong evidence that the die is unfair'. Remember: \emph{the onus is on the data to refute the null hypothesis, the initial assumption}.

\end{importantBox}

\begin{example}[Interpreting $P$-values]
\protect\hypertarget{exm:PvaluesInterpret}{}\label{exm:PvaluesInterpret}In the die example, suppose we found the two-tailed \(P\)-value as~\(0.26\). This is `large' (i.e., much larger than~\(0.05\)). Then the observed value of~\(\hat{p}\) could easily be explained by chance, and is \emph{not} persuasive evidence to support the alternative hypothesis (that the die is unfair). There is no persuasive evidence that~\(p\) is not~\(1/6\).
\end{example}

Note that a different value for \(\text{s.e.}(\hat{p})\) is required to produce the CI (see Def.~\ref{def:DEFSamplingDistributionPhat}).

\section{Writing conclusions}\label{OnePropTestCommunicate}

In general, communicating the results of any hypothesis test requires:

\begin{itemize}
\tightlist
\item
  an answer to the RQ, worded in terms of how much evidence exists to support the \emph{alternative} hypothesis.
\item
  a summary of the evidence used to reach that conclusion (such as the \(z\)-score and \(P\)-value, including if the \(P\)-value is one- or two-tailed).
\item
  sample summary information (see Chap.~\ref{CIOneProportion}), summarising the data used to make the decision (which usually includes a CI for the parameter).
\end{itemize}

So for the die-rolling example, write:

\begin{quote}
The sample provides strong evidence (\(z = 4.05\); two-tailed \(P < 0.001\)) that the proportion of rolls that show a \largedice{1} is not~\(1/6\) (\(\hat{p} = 0.38\); approx.~\(95\)\% CI: \(0.243\) to~\(0.517\); \(n = 50\) rolls) in the population.
\end{quote}

This statement includes the three necessary components:

\begin{itemize}
\tightlist
\item
  an answer to the RQ: `The sample provides very strong evidence\ldots{} that the population proportion is not~\(1/6\)'.
\item
  the evidence used to reach the conclusion: `\(z = 4.05\); two-tailed \(P < 0.001\)'.
\item
  sample summary information (including a CI).
\end{itemize}

\begin{importantBox}{iconmonstr-warning-8-240.png}
Since the \emph{null} hypothesis is initially assumed to be true, \emph{the onus is on the evidence to refute the null hypothesis}. That is, we retain the null hypothesis unless there is persuasive evidence to stop doing so. Hence, conclusions are worded in terms of how strongly the evidence (i.e., sample data) supports the alternative hypothesis.

The alternative hypothesis \emph{may} or \emph{may not} be true, but we report how strongly the evidence (data) supports the alternative hypothesis. Conclusions are \emph{not} worded in terms of how much evidence supports the null hypothesis.

\end{importantBox}

\section{Process overview}\label{OnePropTestOverview}

Let's recap the decision-making process, in this context of rolling a \largedice{1} (Fig.~\ref{fig:DecisionFlowDieRoll}):

\begin{enumerate}
\def\labelenumi{\arabic{enumi}.}
\tightlist
\item
  \emph{Assumption}. Write the \emph{null hypothesis} and \emph{alternative hypothesis} about the \emph{parameter} (based on the RQ), where~\(p\) is the population proportion of rolls that are a \largedice{1}:

  \begin{itemize}
  \tightlist
  \item
    \(H_0\): \(p = 1/6\) (i.e., sampling variation explains the discrepancy between~\(p\) and~\(\hat{p}\)).
  \item
    \(H_1\): \(p \ne 1/6\) (this is a two-tailed alternative hypothesis).
  \end{itemize}
\item
  \emph{Expectation}. The sampling distribution describes what values to reasonably expect from the sample statistic across all possible samples, \emph{if} the null hypothesis is true. In this situation, the sampling distribution has an approximate normal distribution.
\item
  \emph{Observation}. Compute the \(z\)-score (\(z = 4.05\)), a measure of the discrepancy between the assumed population value, and the observed sample value. This is a very large value.
\item
  \emph{Decision}. Determine if the data are consistent with the assumption, by computing the \(P\)-value. Here, the two-tailed \(P\)-value is (much) less than~\(0.0001\), so strong evidence exists that~\(p\) is \emph{not}~\(1/6\).
\end{enumerate}

\begin{figure}[hbtp]

{\centering \includegraphics[width=1\linewidth]{26-Testing-OneProportion_files/figure-latex/DecisionFlowDieRoll-1} 

}

\caption{The decison-making process for the die-rolling data.}\label{fig:DecisionFlowDieRoll}
\end{figure}

\section{Statistical validity conditions}\label{ValidityProportionsTest}

\index{Statistical validity (for inference)!one proportion}

The hypothesis test conducted in this chapter assumes the sampling distribution is approximately a normal distribution (and so, for example, the \(68\)--\(95\)--\(99.7\) rule can be applied). This is only true if certain conditions are met.

The \emph{statistical validity conditions} for a test for a single proportion is that the \emph{expected} number of individuals in the group of interest (i.e, \(n\times p\)) and in the group \emph{not} of interest (i.e., \(n\times (1 - p)\)) both exceed five; that is:

\begin{itemize}
\tightlist
\item
  both \(n\times p > 5\) \emph{and} \(n\times (1 - p) > 5\).
\end{itemize}

The value of~\(5\) here is a rough figure; some books give other values (such as~\(10\)). This condition ensures that the \emph{sampling distribution of the sample proportions has an approximate normal distribution} (so that, for example, the \(68\)--\(95\)--\(99.7\) rule can be used). The units of analysis are also assumed to be \emph{independent} (e.g., from a simple random sample). For a test for one proportions, these conditions are similar to those for the CI for one proportion (Sect.~\ref{CIOneProportion}).

If the statistical validity conditions are not met, other similar options include using a binomial test\index{Non-parametric statistics} \citep{conover2003practical}.

\begin{example}[Statistical validity]
\protect\hypertarget{exm:StatisticalValidityDice}{}\label{exm:StatisticalValidityDice}The hypothesis test regarding the dice is statistically valid. Firstly, \(n\times p = 50 \times (1/6) = 8.666\dots\) (i.e., expect about \(8.7\)~rolls to show a \largedice{1}), and \(n\times (1 - p) = 41.666\dots\) (i.e., expect about \(41.7\)~rolls to \emph{not} show a \largedice{1}). \emph{Both} comfortably exceed five, so the normal distribution will be a good approximation for the sampling distribution. This is what we observe from a computer simulation (Fig.~\ref{fig:StatValidp}, left panel).
\end{example}

\begin{example}[Statistical validity]
\protect\hypertarget{exm:StatisticalValidityDice2}{}\label{exm:StatisticalValidityDice2}Suppose the die was rolled \(10\)~times rather than \(50\)~times. Then, \(n\times p = 10 \times (1/6) = 1.666\dots\) and \(n\times (1 - p) = 10 \times (1 - 1/6) = 8.333\dots\). These do not \emph{both} exceed five, so the normal distribution may be a poor approximation for the sampling distribution.

This is what we observe from simulating the situation (Fig.~\ref{fig:StatValidp}, right panel). The normal model is poor: the simulation shows that the sample proportions are not even symmetrically distributed.
\end{example}



\begin{figure}[hbtp]

{\centering \includegraphics[width=0.95\linewidth]{26-Testing-OneProportion_files/figure-latex/StatValidp-1} 

}

\caption{The sampling distributions for two situations for rolling a die. Left: for sets of 50 rolls, the sampling distribution does have an approximate normal distribution. Right: for sets of \(10\) rolls, the sampling distribution does not have a normal distribution. The solid lines show the approximate normal distributions, and the histograms show the simulated distribution of the sample proportions over many sets of rolls. The solid dots are the value \(p = 1/6\), the population proportion of rolls that show a \largedice{1}. \index{Hypothesis testing!one proportion|)}}\label{fig:StatValidp}
\end{figure}

\section{Example: rolling the other die}\label{OneProportiontestRollOtherDie}

In \(50\) rolls of the \emph{other} die, I found a \largedice{1} on \(7\)~rolls, so that \(\hat{p} = 7/50 = 0.14\). To determine if this die appears loaded, the hypotheses are the same as before: \[
  \text{$H_0$: } p = 1/6  \qquad\text{and}\qquad  \text{$H_1$: } p \ne 1/6.
\] Following the procedures above (check!) and using the same hypotheses, \(z = -0.506\) and (using tables) the two-tailed \(P\)-value is \(2\times 0.3061 = 0.6122\). This means that the sample result was not unusual if \(p = 1/6\), and is certainly not persuasive evidence to support the alternative hypothesis. There is \emph{no evidence} to suggest the second die is loaded.

This all implies the first die was the loaded die. Now I need to decide how to distinguish the two dice so I can tell which is which\dots

\begin{importantBox}{iconmonstr-warning-8-240.png}
\emph{A large \(P\)-value does not prove that the die is fair!} It only means that the proportions of rolls that produce a \largedice{1} is not unusual\ldots{} but perhaps the die is loaded in some other way (i.e., to produce more-than-expected rolls of a \largedice{5}).

\emph{A large \(P\)-value does not necessarily mean that the die is fair!} The die may indeed be loaded to produce a larger-than-expected numbers of rolls that show a \largedice{1}, but (due to sampling variation) the sample we observed simply did not provide evidence to make that conclusion.

The result is interpreted in terms of how much evidence exists to support the alternative hypothesis. The onus is on the data (i.e., evidence) to refute the assumption made in the null hypothesis.

\end{importantBox}

\section{Example: dominance of birds}\label{OneProportiontestBirds}

\citet{barve2017elevational} compared two types of birds (male green-backed tits; male cinereous tits) to see which was more behaviourally dominant over winter. If the species were equally-dominant, then about~\(50\)\% of the interactions would be won by each species. If we define~\(p\) as the proportion of interactions won by green-backed tits, then we would expect \(p = 0.50\). However, in the \(45\)~interactions observed between the two species, green-backed tits won \(37\)~interactions (i.e., \(\hat{p} = 37/45 = 0.82222\)). A discrepancy exists between the sample proportion (\(\hat{p} = 0.8222\)) and the expected population proportion \(p = 0.50\).

Of course, different sample of \(45\)~interactions would produce different values of~\(\hat{p}\). To test if the population proportion of interaction wins could be equally shared, the hypotheses are: \[
   \text{$H_0$: } p = 0.5\quad\text{and}\quad\text{$H_1$: } p \ne 0.5 \text{ (two-tailed)}.
\] The test is statistically valid, since both \(n\times p = 45\times 0.5 = 22.5\) and \(n\times (1 - p) = 22.5\) exceed five. The \emph{standard error} is \[
   \text{s.e.}(\hat{p}) 
   = \sqrt{\frac{p \times (1 - p)}{n}} 
   = \sqrt{\frac{0.50 \times (1 - 0.50)}{45}} 
   = 0.0745356...
\] Then, the value of the \emph{test statistic} is: \[
   z 
   = \frac{\hat{p} - p}{\text{s.e.}(\hat{p})}
   = \frac{0.82222 - 0.50}{0.0745356}
   = 4.322.
\] This is a \emph{very} large \(z\)-score, so the \(P\)-value will be very small, using the \(68\)--\(95\)--\(99.7\) rule, or using tables. This is persuasive evidence to support the alternative hypothesis. We write:

\begin{quote}
\emph{Very} strong evidence exists in the sample (\(P < 0.0001\); \(z = 4.325\)) that the interactions were not won equally by each species (\(\hat{p} = 0.8222\) won by green-backed tits; approx.~\(95\)\% CI: \(0.708\) to~\(0.936\); \(n = 45\)) in the population.
\end{quote}

Note that a different value for \(\text{s.e.}(\hat{p})\) is required to produce the CI (see Def.~\ref{def:DEFSamplingDistributionPhat}).

\section{Chapter summary}\label{Chap28Summary}

These steps are used to test a hypothesis about a population proportion \(p\).

\begin{itemize}
\tightlist
\item
  Write the null hypothesis (\(H_0\); the sampling variation explanation) and the alternative hypothesis (\(H_1\)); initially \emph{assume} the value of~\(p\) in the null hypothesis to be true.
\item
  Describe the \emph{sampling distribution}, which describes what to \emph{expect} from the sample statistic across all possible samples, based on this assumption: under certain statistical validity conditions, the sample mean varies with:

  \begin{itemize}
  \tightlist
  \item
    an approximate normal distribution,
  \item
    with sampling mean, whose value is the value of~\(p\),
  \item
    with a standard deviation of \(\displaystyle \text{s.e.}(\hat{p}) = \sqrt{\frac{p \times (1 - p)}{n}}\), where~\(p\) is the hypothesised value given in the null hypothesis, and~\(n\) is the sample size.
  \end{itemize}
\item
  Compute the value of the \emph{test statistic}: \[
   z = \frac{ \hat{p} - p}{\text{s.e.}(\hat{p})}.
  \]
\item
  Compute an approximate \emph{\(P\)-value} using the \(68\)--\(95\)--\(99.7\) rule, or using tables. Use the \(P\)-value to make a decision, and write a conclusion.
\item
  Check the statistical validity conditions.
\end{itemize}

\section{Quick review questions}\label{Chap31-QuickReview}

A study of diseases in Native Americans \citep{kizer2006digestive} found \(381\)~obese or overweight patients in \(449\)~patients. Across all the US population, the percentage obese or overweight was~\(65\)\%. The researchers wanted to determine if the percentage of obesity/overweight Native Americans was \emph{greater} than that of the general population.

Are the following statements \emph{true} or \emph{false}?

\begin{enumerate}
\def\labelenumi{\arabic{enumi}.}
\item
  The sample size is \(n = 381\). \tightlist
\item
  The value of the \emph{sample} proportion is~\(\hat{p} = 381\).
\item
  The \emph{null} hypothesis is~\(H_0\): \(p = 0.65\).
\item
  The \emph{alternative} hypothesis is~\(H_0\): \(p = 0.8486\).
\item
  We initially assume the \emph{population} proportion of overweight/obese Native Americans is~\(0.65\).
\item
  The \emph{alternative} hypothesis is \emph{one}-tailed.
\item
  In a one-sample test of proportion, the \(z\)-score is always large.
\item
  The value of the \(z\)-score for this example is \(8.82\).
\item
  We have evidence to support the alternative hypothesis in this example.
\item
  We always accept the \emph{null} hypothesis.
\end{enumerate}

\section{Exercises}\label{OneProportionTestExercises}

\hyperref[Answers]{Answers to odd-numbered exercises} are given at the end of the book.

\captionsetup{font=small}

\begin{exercise}
\protect\hypertarget{exr:sepPWhy1}{}\label{exr:sepPWhy1}Explain \emph{why} the standard error is computed using~\(p\) for hypothesis testing, but using~\(\hat{p}\) for CIs.
\end{exercise}

\begin{exercise}
\protect\hypertarget{exr:sepPWhy2}{}\label{exr:sepPWhy2}Explain why describing the sampling distribution is difficult if we \emph{assume} \(p \ne 1/6\).
\end{exercise}

\begin{exercise}
\protect\hypertarget{exr:OneProportionTestExplainA}{}\label{exr:OneProportionTestExplainA}In the die example, the observed proportion is \(0.38\). Could we simply state that the proportion clearly is not \(1/6 = 0.1666\)? Explain.
\end{exercise}

\begin{exercise}
\protect\hypertarget{exr:OneProportionTestExplainB}{}\label{exr:OneProportionTestExplainB}Explain why we compute \(\text{s.e.}(\hat{p})\) and not~\(\text{s.e.}(p)\).
\end{exercise}

\begin{exercise}
\protect\hypertarget{exr:OneProportionTestExercisesDodgyA}{}\label{exr:OneProportionTestExercisesDodgyA}

What is wrong with the following statement, after testing \(H_0\): \(p = 0.25\):

\begin{quote}
There is very strong evidence that the sample proportion is greater than~\(0.25\).
\end{quote}

\end{exercise}

\begin{exercise}
\protect\hypertarget{exr:OneProportionTestExercisesDodgyB}{}\label{exr:OneProportionTestExercisesDodgyB}

Explain what is wrong with this statement from \citet{davis2024higher}, that appears under their Table~2:

\begin{quote}
One proportion \(z\)-test with \(H_0 = 0.076\), the proportion of UDT in our sample\ldots{}
\end{quote}

\end{exercise}

\begin{exercise}
\protect\hypertarget{exr:OneProportionTestExercisesPlacebos}{}\label{exr:OneProportionTestExercisesPlacebos}

The study of herbal medicines is complicated, as \emph{blinding} subjects\index{Blinding!individuals} is difficult: placebos\index{Placebo} are often easily identifiable by eye, by taste, or by smell.

\citet{loyeung2018experimental} studied if subjects could identify potential placebos at a \emph{better} rate than just guessing. The \(81\) subjects were each presented with a choice of five different supplements, four of which were placebos. Subjects were asked to select which one was the legitimate herbal supplement based on the \emph{taste}; \(50\)~subjects correctly selected the true herbal supplement.

\begin{enumerate}
\def\labelenumi{\arabic{enumi}.}
\tightlist
\item
  If the subjects were selecting the true herbal supplement randomly, what proportion of subjects would be expected to select the correct supplement as the true herbal medicine?
\item
  Write the hypotheses for addressing the aims of the study.
\item
  Is this a one- or two-tailed test? Explain.
\item
  Sketch the \emph{sampling distribution} of the sample proportion, assuming \(H_0\) is correct, for \(n = 81\).
\item
  Is there evidence that people can identify the true supplement by taste?
\item
  Are the statistical validity conditions satisfied?
\end{enumerate}

\end{exercise}

\begin{exercise}
\protect\hypertarget{exr:POnePropTestMeasles}{}\label{exr:POnePropTestMeasles}

\citet{kim2004sero} studied the measles-rubella vaccination-rates in Korea, comparing the proportion of children susceptible to measles with the \emph{World Health Organization} target proportion (for children aged~\(5\) to~\(9\) years old: \(10\)\%).

The aim was to test if the proportion of Korean children susceptible to measles in the \emph{population} was \(10\)\%~or \emph{lower} (i.e., better). In the study, \(55\)~children out of~\(972\) were susceptible to measles.

\begin{enumerate}
\def\labelenumi{\arabic{enumi}.}
\tightlist
\item
  Compute the sample proportion~\(\hat{p}\) of children susceptible to measles.
\item
  Write the hypotheses for the test. Is the test one- or two-tailed?
\item
  Compute the standard error for the test.
\item
  Compute the \(z\)-score and determine the \(P\)-value.
\item
  Write a conclusion.
\item
  Are the statistical validity conditions satisfied?
\end{enumerate}

\end{exercise}

\begin{exercise}
\protect\hypertarget{exr:OneProportionTestTurtleSex}{}\label{exr:OneProportionTestTurtleSex}\citet{streeting2022optimising} studied western saw-shelled turtles. When eggs were incubated at \(27\)\textsuperscript{o}C, they observed that \(29\)~males and \(44\)~females hatched. Are the proportions of male and female turtles that hatch at this temperature equal?
\end{exercise}

\begin{exercise}
\protect\hypertarget{exr:OneProportionTestExercisesEPL}{}\label{exr:OneProportionTestExercisesEPL}{[}\emph{Dataset}: \texttt{PremierL}{]} In the 2019/2020 English Premier League (EPL), the home team won \(91\)~games, and the away team won \(67\)~games. (Another \(50\)~games were draws.)

Use the \(n = 158\)~games with a result to determine if there is evidence that the home team wins more often than~\(50\)\% (i.e., that there is a home-side advantage).
\end{exercise}

\begin{exercise}
\protect\hypertarget{exr:OneProportionTestExercisesPedalMachines}{}\label{exr:OneProportionTestExercisesPedalMachines}\citet{maeda2013introducing} introduced pedal machines on the first floor of the Joyner Library for use by students at East Carolina University (ECU) to increase activity in library users. At ECU, \(60.2\)\%~of all students were females (i.e., in the population). Students were observed using the machine on \(589\)~occasions, of which~\(295\) times were by females

Is there evidence that the proportion of female users of the machines was \emph{lower} than the overall female proportion at the university? What would you conclude?
\end{exercise}

\begin{exercise}
\protect\hypertarget{exr:OneProportionTestExercisesCasinos}{}\label{exr:OneProportionTestExercisesCasinos}\citet{koenen1995analysis} found that \(88\)~of the \(357\)~visitors to Las Vegas casinos in 1995 were smokers. At the time, \(25.5\)\%~of the general US population were smokers (based on data from the US \emph{National Center for Health Statistics}). Is the proportion of smokers among casino-goers the same as for the general US population?
\end{exercise}

\begin{exercise}
\protect\hypertarget{exr:OneProportionBreadfruitPasta}{}\label{exr:OneProportionBreadfruitPasta}\citet{nochera2019development} developed gluten-free pasta made from breadfruit. In the study sample, \(57\)~of the \(71\)~participants stated that they liked the pasta. Do the researchers have sufficient evidence to claim that the `majority of people like breadfruit pasta'?
\end{exercise}

\begin{exercise}
\protect\hypertarget{exr:OneProportionTestExercisesCTS}{}\label{exr:OneProportionTestExercisesCTS}Carpal Tunnel Syndrome (CTS) is a painful condition in the wrists. \citet{boltuch2020palmaris} were interested in whether `a relationship exists between the palmaris tendon {[}and{]} carpal tunnel syndrome (CTS)' (p.~493). The palmaris longus (PL) tendon is visually absent in about~\(15\)\% of the population. The researchers found PL was visually absent in~\(33\) of~\(516\) CTS wrists in their sample. Is there evidence to suggest that rate of PL absence is \emph{different} in CTS cases, compared to the general population?
\end{exercise}

\begin{exercise}
\protect\hypertarget{exr:OneProportionTestExercisesBorers}{}\label{exr:OneProportionTestExercisesBorers}\citet{siegfried2014estimating} studied resistance of some commercial corn varieties to the European corn borer. Borers were collected from corn in Iowa and Nebraska.

Researchers aimed to estimate the frequency of resistance to the toxin in the corn. By mating borers collected from the field with various resistant laboratory individuals, they could determine what proportion of resistant individuals to expect in the second generation offspring. In one study of \(n = 172\) second-generation individuals, \(24\)~were found to be resistant. The theoretical expectation was that \(1\)-in-\(16\) of the second-generation borers would be resistant if the field borers were resistant. Perform a hypothesis test to determine if the data suggest that the field borers were resistant (that is, if the population proportion is~\(1/16\)) as expected.
\end{exercise}

\begin{exercise}
\protect\hypertarget{exr:OneProportionTestExercisesLEDlights}{}\label{exr:OneProportionTestExercisesLEDlights}\citet{davidovic2019drivers} studied street-light preferences of drivers. Drivers were asked to conduct a series of manoeuvres under \(3\,000\)K~LED light and then under \(4\,000\)K~LED lights. They were then asked to decide which street light they preferred. Out of the \(52\)~subjects, \(29\)~preferred the \(3\,000\)K~LED lights. Is there evidence that the choice between the two street lights is random, or is there evidence of a preference for one over the other?
\end{exercise}

\begin{exercise}
\protect\hypertarget{exr:OneProportionTestExercisesCoinSpin}{}\label{exr:OneProportionTestExercisesCoinSpin}The euro was introduced as a currency on 01~January 1999. According to a report by the \emph{New Scientist}, students in Poland spun a Belgian one-euro coin \(250\)~times, and found \(140\)~heads (as reported by \citet{data:Gelman2002:DiceCoins}). This resulted in an `accusation of bias' in the \emph{New Scientist} article. However, every set of \(250\)~spins can produce a different proportion of heads, so perhaps the results is just due to randomness. Does this sample of \(250\)~spins suggest that the one-euro Belgian coin is biased?
\end{exercise}

\begin{exercise}
\protect\hypertarget{exr:OneProportionTestExercisesBirths}{}\label{exr:OneProportionTestExercisesBirths}

As noted in Sect.~\ref{ProbRelFreq}, the \emph{Australian Bureau of Statistics} (ABS) stated that:

\begin{quote}
The sex ratio for all births registered in Australia generally fluctuates around \(105.5\)~male births per \(100\)~female births.
\end{quote}

(This statistic does not use births registered as `other' or `not stated'.)

\begin{enumerate}
\def\labelenumi{\arabic{enumi}.}
\tightlist
\item
  The value of~\(105.5\) is effectively a population odds ratio of male-to-female births. Show that this is equivalent to the population proportion of male births as~\(0.51338\) (not including `other' or `not stated').
\item
  In~2021, there were \(148\,636\)~male births and \(140\,944\)~female births. Compute the \emph{sample} proportion of male births in~2021 (to five decimal places).
\item
  Conduct a test to determine if the 2021~data appear different to the long-term proportion.
\end{enumerate}

\end{exercise}

\captionsetup{font=normalsize}

\begin{EOCanswerBox}{iconmonstr-check-mark-14-240.png}
\textbf{Answers to \textit{Quick review} questions:} \textbf{1.} False. \textbf{2.} False; \(\hat{p} = 381/449 = 0.84855\). \textbf{3.} True. \textbf{4.} False. \textbf{5.} True. \textbf{6.} True. \textbf{7.} False. \textbf{8.} True. \textbf{9.} True. \textbf{10.} False.

\end{EOCanswerBox}

\chapter{Hypothesis tests: one mean}\label{TestOneMean}

\begin{cols}
\begin{col}{0.52\textwidth}

\begin{objectivesBox}{iconmonstr-target-4-240.png}
You have learnt to ask an RQ, design a study, classify and summarise the data, construct confidence intervals, and perform a hypothesis test for one proportion.
\textbf{In this chapter}, you will learn to:
  
\begin{itemize}\tightlist
  \item
  identify situations where conducting a test for a mean is appropriate.
  \item
  conduct hypothesis tests for one sample mean, using a $t$-test.
  \item
  determine whether the conditions for using these methods apply in a given situation.
\end{itemize}
\end{objectivesBox}

\end{col}

\begin{col}{0.03\textwidth}
~
\end{col}

\begin{col}{0.45\textwidth}

\includegraphics[width=0.95\linewidth]{27-Testing-OneMean_files/figure-latex/unnamed-chunk-7-1} 
\end{col}
\end{cols}

\section{Introduction: body temperatures}\label{BodyTemperature}

\index{Hypothesis testing!one mean|(}

The average internal body temperature is commonly believed to be \(37.0\)\textsuperscript{o}C (\(98.6\)\textsuperscript{o}F). This value is based on data over~\(150\) years old \citep{data:Wunderlich:BodyTemp}. Since then, the methods for measuring internal body temperature have changed substantially:

\begin{quote}
Thermometers used by Wunderlich were cumbersome, had to be read in situ, and, when used for axillary measurements {[}i.e., under the armpit{]}\ldots{} required \(15\) to~\(20\,\text{mins}\) to equilibrate. Today's thermometers are smaller and more reliable and equilibrate more rapidly. In addition, the mouth and rectum have replaced the axilla {[}armpit{]} as the preferred sites for monitoring body temperature.

\VA{--- \citet{data:mackowiak:bodytemp}, p.~1579}{}
\end{quote}

For this reason, the reported internal body temperature (recorded by newer instruments, in different locations) may have changed since the 1860s. Therefore, we could ask:

\begin{quote}
Is the \emph{population} mean internal body temperature equal to \(37.0\)\textsuperscript{o}C?
\end{quote}

A \emph{decision} is sought about the value of the \emph{population} mean body temperature. Presumably, the intended population is all people, though the population, in practice, may depend on what population is represented by the available data.

The population mean internal body temperature will never be known: the internal body temperature of every person alive would need to be measured, and even those not yet born. A \emph{sample} must be studied.

Define the parameter as \(\mu\), the population mean internal body temperature (in \textsuperscript{o}C). A \emph{sample} of people can be used to determine if evidence exists that the \emph{population} mean internal body temperature is not \(37.0\)\textsuperscript{o}C,\index{Mean!of a population} using the decision-making process (Sect.~\ref{DecisionMaking}).

\section{Assumptions: hypotheses}\label{TestpObsDecisionHypotheses}

\textbf{Step~1} of the decision-making process is to \emph{assume} a value for the parameter. The established claim is that the population mean internal body temperature is \(37.0\)\textsuperscript{o}C, so we assume this value. This assumption becomes the null hypothesis: \[
   \text{$H_0$: } \mu = 37.0.
\] If the \emph{sample} mean is not \(37.0\)\textsuperscript{o}C, this hypothesis proposes that the discrepancy is due to sampling variation.

The RQ asks if the \emph{population} mean internal body temperate \(\mu\) is \emph{equal} to \(37.0\)\textsuperscript{o}C, or if it has \emph{changed}. The RQ does not specifically ask if \(\mu\) is smaller than \(37.0\)\textsuperscript{o}C, or larger than \(37.0\)\textsuperscript{o}C. This means the alternative hypothesis is two-tailed: \[
   \text{$H_1$: } \mu \ne 37.0.
\]

\section{\texorpdfstring{Expectations: sampling distribution for \(\bar{x}\)}{Expectations: sampling distribution for \textbackslash bar\{x\}}}\label{SamplingDistSampleMeanHT}

\index{Sampling distribution!one mean}

\textbf{Step~2} of the decision-making process is to describe what values of the statistic (in this case, the sample mean \(\bar{x}\)) can be expected if the value of \(\mu\) is assumed to be \(37.0\) (the value specified in \(H_0\)). In other words, the \emph{sampling distribution} of \(\bar{x}\) needs to be described.

The sample mean \emph{varies} from sample to sample, and varies with a normal distribution (whose standard deviation is called the \emph{standard error}) under certain conditions (given in Sect.~\ref{ValiditySampleMeanTest}). The sampling distribution of~\(\bar{x}\) was described in Sect.~\ref{SamplingDistSampleMean}, and repeated below.

\begin{definition}[Sampling distribution of a sample mean]
\protect\hypertarget{def:DEFSamplingDistributionXbarHT}{}\label{def:DEFSamplingDistributionXbarHT}

When the \emph{population} standard deviation is unknown, the \emph{sampling distribution of the sample mean} is (when certain conditions are met; Sect.~\ref{ValiditySampleMean}) described by:

\begin{itemize}
\tightlist
\item
  an approximate normal distribution,
\item
  centred around a sampling mean whose value is \(\mu\),
\item
  with a standard deviation (called the \emph{standard error of the mean}), denoted~\(\text{s.e.}(\bar{x})\), whose value is \begin{equation}
   \text{s.e.}(\bar{x}) = \frac{s}{\sqrt{n}},
   \label{eq:stderrorxbarHT}
  \end{equation} where~\(n\) is the size of the sample, and~\(s\) is the sample standard deviation of the observations.
\end{itemize}

\end{definition}

The mean of this sampling distribution---the \emph{sampling mean}---has the value \(\mu\). The standard deviation of this sampling distribution is called the \emph{standard error of the sample means}, denoted \(\text{s.e.}(\bar{x})\). When the \emph{population} standard deviation~\(\sigma\) is \emph{unknown}, the value of the standard error happens to be (see Equation~\eqref{eq:stderrorxbarHT}) \[
  \text{s.e.}(\bar{x}) = \frac{s}{\sqrt{n}}.
\]

\citet{data:mackowiak:bodytemp} gathered body-temperature data for \(n = 130\) people, collated by \citet{data:Shoemaker1996:Temperature} (Table~\ref{tab:DataBodyTemp}; Fig.~\ref{fig:BodyTempHist}). The data all come from

\begin{quote}
\ldots{} volunteers participating in Shigella vaccine trials conducted at the University of Maryland Center for Vaccine Development, Baltimore\ldots{}

\VA{--- \citet{data:mackowiak:bodytemp}, p.~1578}{}
\end{quote}

Hence, the population for the study (and RQ) should be redefined accordingly. From software output (Fig.~\ref{fig:BodyTempjamovi}), the \emph{sample} mean is \(\bar{x} = 36.8052\)\textsuperscript{o}C\index{Mean!of a sample} and the \emph{sample} standard deviation is \(s = 0.4073\)\textsuperscript{o}C. Using this value of \(s\), the sampling distribution of~\(\bar{x}\) can be described, if~\(\mu\) really was~\(37.0\):

\begin{itemize}
\tightlist
\item
  an approximate normal distribution,
\item
  with a sampling mean whose value is \(\mu = 37.0\) (from~\(H_0\)),
\item
  with a standard deviation of \(\text{s.e.}(\bar{x}) = s/\sqrt{n} = 0.4073/\sqrt{130} = 0.0357\) (as in the output).
\end{itemize}

\begin{figure}
\begin{minipage}{0.42\textwidth}
\captionof{table}{The body temperature data: the first nine and last nine of the $130$ ordered observations\label{tab:DataBodyTemp}.}
\fontsize{8}{12}\selectfont
\begin{@empty}

\begin{tabular}{cccc}
\toprule
\multicolumn{4}{c}{\textbf{Body temperature (in ${}^{\circ}$C)}} \\
\cmidrule(l{3pt}r{3pt}){1-4}
$35.72$ & $36.17$ & $\vdots$ & $37.39$\\
$35.94$ & $36.17$ & $37.28$ & $37.44$\\
$36.06$ & $36.22$ & $37.28$ & $37.72$\\
$36.11$ & $36.28$ & $37.33$ & $37.78$\\
$36.17$ & $\vdots$ & $37.33$ & $38.22$\\
\bottomrule
\end{tabular}
\end{@empty}
\end{minipage}
\hspace{0.05\textwidth}
\begin{minipage}{0.51\textwidth}%
\centering

\includegraphics[width=0.95\linewidth]{27-Testing-OneMean_files/figure-latex/unnamed-chunk-5-1} 
\caption{The histogram of the body temperature data\label{fig:BodyTempHist}.}
\end{minipage}
\end{figure}

\begin{figure}[hbtp]

{\centering \includegraphics[width=0.65\linewidth]{jamovi/BodyTemp/BodyTemp-Summary} 

}

\caption{The software output summary for the body temperature data.}\label{fig:BodyTempjamovi}
\end{figure}

A picture of this sampling distribution (Fig.~\ref{fig:BodyTempSamplingDist}) shows how the sample mean varies when \(n = 130\), for all possible samples when \(\mu = 37.0\). For example, the value of~\(\bar{x}\) will be \emph{larger} than~\(37.0357\)\textsuperscript{o}C about~\(16\)\% of the time (using the \(68\)--\(95\)--\(99.7\) rule) if~\(\mu\) really is~\(37.0\).

\begin{figure}[hbtp]

{\centering \includegraphics[width=0.95\linewidth]{27-Testing-OneMean_files/figure-latex/BodyTempSamplingDist-1} 

}

\caption{The distribution of sample mean body temperatures, if the population mean is $37.0^\circ$C and $n = 130$. The grey vertical lines are\ $1$,\ $2$ and\ $3$ standard deviations from the mean.}\label{fig:BodyTempSamplingDist}
\end{figure}

\section{\texorpdfstring{Observations: \(t\)-score}{Observations: t-score}}\label{Tscores}

\index{Test statistic!t@$t$-score}

\textbf{Step~3} of the decision-making process is to evaluate the observations. Locating \(\bar{x} = 36.8052\) on the sampling distribution (Fig.~\ref{fig:BodyTempSamplingDistT}) shows that this observed sample mean is, relatively speaking, \emph{extremely} low: a sample mean this low is very unlikely to occur in any sample of \(n = 130\) when \(\mu = 37.0\). How many standard deviations is~\(\bar{x}\) away from \(\mu = 37.0\)? Compute: \[
  \frac{\text{statistic} - \text{mean of the distribution}}{\text{std dev. of the distribution}}   
  =
  \frac{36.8052 - 37.0}{0.035724} = -5.453.
\] This is like a \(z\)-score: it measures the number of standard deviations that the value is from the mean. However, it is not a \(z\)-score; it is a \(t\)-score. Both \(t\)- and \(z\)-scores measure \emph{the number of standard deviations that a value is from the mean}. Here the value is a \(t\)-score, because the \emph{population} standard deviation~\(\sigma\) is unknown, and the \emph{sample} standard deviation is used instead to compute \(\text{s.e.}(\bar{x})\).

\begin{tipBox}{iconmonstr-info-6-240.png}
Like \(z\)-scores, \(t\)-scores measure the number of standard deviations that a value is from the mean of the distribution.

\end{tipBox}

\begin{figure}[hbtp]

{\centering \includegraphics[width=0.85\linewidth]{27-Testing-OneMean_files/figure-latex/BodyTempSamplingDistT-1} 

}

\caption{The sample mean of $\bar{x} = 36.8041^\circ$C is very unlikely to be observed in any sample of size $n = 130$, if $\mu = 37.0^\circ$C.\spacex The standard deviation of the distribution is $\text{s.e.}(\bar{x}) = 0.035724$.}\label{fig:BodyTempSamplingDistT}
\end{figure}

The calculation is therefore: \[
   t = \frac{36.8052 - 37.0}{0.035724} = -5.453.
\] The observed sample mean is \emph{more than five standard deviations below the population mean}, which is highly unusual based on the \(68\)--\(95\)--\(99.7\) rule (Fig.~\ref{fig:BodyTempSamplingDistT}). This is very persuasive evidence that \(\mu\) is not \(37.0\).

In general, when the sampling distribution has an approximate normal distribution and the \emph{sample} standard deviation is used to compute the standard error, the \(t\)-score is \begin{equation}
   t 
   = 
   \frac{\text{sample statistic} - \text{mean of the sampling distribution}}
        {\text{standard error of the sampling distribution}}
   =
   \frac{\bar{x} - \mu}{\text{s.e.}(\bar{x})}.
   \label{eq:tscore}
\end{equation}

\section{\texorpdfstring{Decision: \(P\)-value}{Decision: P-value}}\label{Pvalues}

As seen in Sect.~\ref{TestpObsDecisionPvalues}, a \(P\)-value quantifies how unusual the observed sample statistic is, after assuming \(H_0\) is true. Since \(t\)-scores and \(z\)-scores are very similar, the \(P\)-value can be \emph{approximated} using the \(68\)--\(95\)--\(99.7\) rule and a diagram, or \emph{approximated} using \(z\)-tables (Appendices~\ref{ZTablesNEG} and~\ref{ZTablesPOS}). Usually, however, software is used to compute the \(P\)-value.\index{Software output!one mean} \(t\)-scores and \(z\)-scores with the same value produce almost the same \(P\)-values, except for small sample sizes.

Both methods produce approximate \(P\)-values only, since the approximations are based on using \(z\)-scores rather than \(t\)-scores. Usually, software is used to determine precise \(P\)-values for \(t\)-scores (Fig.~\ref{fig:BodyTempTestjamovi}).\index{Software output!one mean} The output (under the heading \texttt{p}) shows that the \(P\)-value is indeed very small: less than~\(0.001\) (written \(P < 0.001\)).

\begin{tipBox}{iconmonstr-info-6-240.png}
Some software reports a \(P\)-value of~\texttt{0.000}, which really means that the \(P\)-value is zero to three decimal places. Since \(P\)-values can never be exactly zero, we should write \(P < 0.001\): that is, the \(P\)-value is \emph{smaller} than~\(0.001\).

\end{tipBox}

This \(P\)-value means that, if \(\mu = 37.0\), a sample mean as low as~\(36.8052\) would be \emph{very} unusual to observe (from a sample size of \(n = 130\)). And yet, we did. Using the decision-making process, this implies that the initial assumption (i.e., \(H_0\)) is contradicted by the data: we observed something extremely unlikely if \(\mu = 37.0\). That is, there is very persuasive evidence that the \emph{population} mean body temperature is \emph{not}~\(37.0\)\textsuperscript{o}C.

\begin{figure}[hbtp]

{\centering \includegraphics[width=0.65\linewidth]{jamovi/BodyTemp/BodyTempTtest} 

}

\caption{Software output for conducting the $t$-test for the body temperature data.}\label{fig:BodyTempTestjamovi}
\end{figure}

\begin{importantBox}{iconmonstr-warning-8-240.png}
For \emph{one-tailed tests}, the \(P\)-value is \emph{half} the value of the two-tailed \(P\)-value.

\end{importantBox}

As seen in Sect.~\ref{TestpObsDecisionPvalues}, \(P\)-values measure the probability of observing the sample statistic (or something more extreme), assuming the population parameter is the value given in~\(H_0\). For the body-temperature data then, where \(P < 0.001\), the \(P\)-value is \emph{very} small, so \emph{very strong evidence} exists that the population mean body temperature is not~\(37.0\)\textsuperscript{o}C.

\section{Writing conclusions}\label{writing-conclusions}

Communicating the results of any hypothesis test requires an \emph{answer to the RQ}, a summary of the \emph{evidence} used to reach that conclusion (such as the \(t\)-score and \(P\)-value, stating if the \(P\)-value is one- or two-tailed), and some \emph{sample summary information} (including a CI). For the body-temperature example, write:

\begin{quote}
The sample provides very strong evidence (\(t = -5.45\); two-tailed \(P < 0.001\)) that the population mean body temperature is \emph{not}~\(37.0\)\textsuperscript{o}C (\(\bar{x} = 36.81\); \(95\)\% CI: \(36.73\) to~\(36.88\)\textsuperscript{o}C; \(n = 130\)).
\end{quote}

This statement contains the three components.

\begin{enumerate}
\def\labelenumi{\arabic{enumi}.}
\tightlist
\item
  The \emph{answer to the RQ}: the sample provides very strong evidence that the population mean body temperature is not~\(37.0\)\textsuperscript{o}C. The alternative hypothesis is two-tailed, so the conclusion is that the population mean body temperature is \emph{not equal to} ~\(37.0\)\textsuperscript{o}C.
\item
  The \emph{evidence} used to reach the conclusion: \(t = -5.45\); two-tailed \(P < 0.001\).
\item
  Some \emph{sample summary information}: the sample mean (with the CI) and the sample size.
\end{enumerate}

The test is about the \emph{mean} internal body temperature; \emph{individuals} have internal body temperatures ranging from~\(35.722\)\textsuperscript{o}C to~\(38.222\)\textsuperscript{o}C.\index{Range}

The difference between the value of \(37.0\)\textsuperscript{o}C and the sample mean of \(36.81\)\textsuperscript{o}C is small in absolute terms, and is probably of little practical importance for most applications.\index{Practical importance} Notice that the CI does \emph{not} include the value of \(\mu = 37.0\).

\section{Process overview}\label{TestSummary}

Let's recap the decision-making process for this body temperatures (Fig.~\ref{fig:DecisionFlowTemps}) example:

\begin{enumerate}
\def\labelenumi{\arabic{enumi}.}
\tightlist
\item
  \emph{Assumption}. Write the \emph{null hypothesis} about the parameter (based on the RQ): \(H_0\): \(\mu = 37.0\). In addition, write the \emph{alternative hypothesis}: \(H_1\): \(\mu \ne 37.0\). (This alternative hypothesis is two-tailed.)
\item
  \emph{Expectation}. The \emph{sampling distribution} describes what to expect from the statistic \emph{if} the null hypothesis is true. The sampling distribution is an approximate normal distribution.
\item
  \emph{Observation}. Compute the \(t\)-score: \(t = -5.45\). The \(t\)-score can be computed by software, or using the general equation in Equation~\eqref{eq:tscore}.
\item
  \emph{Decision}. Determine if the data are \emph{consistent} with the assumption, by computing the \(P\)-value. Here, the \(P\)-value is much smaller than~\(0.001\). The \(P\)-value can be computed by software, or approximated using the \(68\)--\(95\)--\(99.7\) rule. The \emph{conclusion} is that there is very strong evidence that~\(\mu\) is not~\(37.0\).
\end{enumerate}

\begin{figure}[hbtp]

{\centering \includegraphics[width=1\linewidth]{27-Testing-OneMean_files/figure-latex/DecisionFlowTemps-1} 

}

\caption{The decison-making process for the body-temperature data.}\label{fig:DecisionFlowTemps}
\end{figure}

\section{Statistical validity conditions}\label{ValiditySampleMeanTest}

\index{Statistical validity (for inference)!one mean}

All hypothesis tests have underlying conditions to be met so that the results are statistically valid. For a test of one mean, this means that the sampling distribution must have an approximate normal distribution so that \(P\)-values can be found.

The test for a single mean is \emph{statistically valid} if \emph{either} of these is true:

\begin{itemize}
\tightlist
\item
  when \(n \ge 25\). (If the distribution of the data is highly skewed, the sample size may need to be larger.)
\item
  when \(n < 25\), \emph{and} the sample data come from a \emph{population} with a normal distribution.
\end{itemize}

The sample size of~\(25\) is a rough figure; some books give other values (such as~\(30\)).

This condition ensures that the \emph{distribution of the sample means has an approximate normal distribution} (so that, for example, the \(68\)--\(95\)--\(99.7\) rule can be used). Provided the sample size is larger than about \(25\), this will be approximately true \emph{even if} the distribution of the individuals in the population does not have a normal distribution. That is, when \(n \ge 25\) the sample means generally have an approximate normal distribution, even if the data themselves do not have a normal distribution. The units of analysis are also assumed to be \emph{independent} (e.g., from a simple random sample).

If the statistical validity conditions are not met, other similar options include a sign test\index{Sign test} or a Wilcoxon signed-rank test\index{Wilcoxon signed ranks test} \citep{conover2003practical}, or using resampling methods \citep{efron2021computer}.

\begin{example}[Statistical validity]
\protect\hypertarget{exm:StatisticalValidityTemps}{}\label{exm:StatisticalValidityTemps}The hypothesis test regarding body temperature is statistically valid since the sample size is larger than~\(25\) (\(n = 130\)). (The data, as displayed in Fig.~\ref{fig:BodyTempHist}, do \emph{not} need to come from a population with a normal distribution.)
\end{example}

\index{Hypothesis testing!one mean|)}

\section{Example: student IQs}\label{IQstudents}

Standard IQ scores are designed to have a mean in the general population of~\(\mu = 100\). Researchers at Griffith University (GU) asked:

\begin{quote}
For students at Griffith University, is the mean IQ higher than~\(100\)?
\end{quote}

The parameter is \(\mu\), the population mean IQ for students at GU.\spacex

To answer this RQ, \citet{reilly2022gender} studied \(n = 224\) students at Griffith University (GU), finding a sample mean IQ of~\(111.19\) and a standard deviation of~\(14.21\). Is this evidence that GU students have a \emph{higher} mean IQ than the general population? The hypotheses are: \[
   \text{$H_0$: $\mu = 100 \qquad \text{and} \qquad H_1$: $\mu > 100$.}
\] This test is \emph{one-tailed}, since the RQ asks if the mean IQ of GU students is \emph{greater} than~\(100\), the one-tailed \(P\)-value will be in the tail corresponding to \emph{larger} IQ scores (i.e., to the right of the mean). (Writing \(H_0\): \(\mu\le 100\) is also correct (and equivalent), though the test still proceeds as though \(\mu = 100\), the largest option permitted by \(\mu\le100\).)

We do not have the original data, but the summary information is sufficient: \(\bar{x} = 111.19\) with \(s = 14.21\) from a sample of size \(n = 224\). The \emph{sample} mean is higher than \(100\), but since sample means vary, the difference may be just due to sampling variation. The sample means vary with a normal distribution, with mean~\(100\) and a standard deviation of \[
   \text{s.e.}(\bar{x}) = \frac{s}{\sqrt{n}} = \frac{14.21}{\sqrt{224}} = 0.94945.
\] The \(t\)-score is \[
   t = \frac{\bar{x} - \mu}{\text{s.e.}(\bar{x})} = \frac{111.19 - 100}{0.94945} = 11.786.
\]

This \(t\)-score is \emph{huge}: a sample mean as large as~\(111.19\) would be highly unlikely to occur in any sample of size \(n = 224\), simply by sampling variation, if the population mean really was~\(100\). Since the alternative hypothesis is \emph{one-tailed}, and \(\mu > 100\) specifically, the \(P\)-value is the area in the right-side tail of the distribution only (Fig.~\ref{fig:IQSamplingDistribution}); it will be extremely small. This is very persuasive evidence to support the alternative hypothesis.

\begin{figure}[hbtp]

{\centering \includegraphics[width=0.95\linewidth]{27-Testing-OneMean_files/figure-latex/IQSamplingDistribution-1} 

}

\caption{The sampling distribution for the IQ data. The RQ is one-tailed so the $P$-value is the area in one tail.}\label{fig:IQSamplingDistribution}
\end{figure}

We conclude:

\begin{quote}
Very strong evidence exists in the sample (\(t = 11.78\); one-tailed \(P < 0.001\)) that the population mean IQ in students at Griffith University is greater than~\(100\) (mean \(111.19\); \(95\)\% CI: \(109.29\) to~\(113.09\); \(n = 224\)).
\end{quote}

The test is about the \emph{mean} IQ; \emph{individual} students may have IQs less than \(100\).

Since the sample size is much larger than~\(25\), this conclusion is \emph{statistically valid}. The sample is not a random sample from the population of all GU students (the students are mostly first-year, undergraduate psychological science students). However, these students may be somewhat representative of all GU students. In any case, the results probably apply to first-year, undergraduate psychological science students at GU.

The difference between the general population IQ of~\(100\) and the sample mean IQ of GU students is only small: about~\(11\) IQ units (less than one standard deviation). Possibly, this difference has very little practical importance,\index{Practical importance} even though the statistical evidence suggests that the difference cannot be explained by chance.

IQ scores are designed to have a standard deviation of \(\sigma = 15\) in the general population. If this applies for university students too (and we do not know if it does), the standard error is \(\text{s.e.}(\bar{x}) = \sigma/\sqrt{n} = 15/\sqrt{130} = 1.0022\), and the test-statistic is then a \(z\)-score: \[
  z = \frac{\bar{x} - \mu}{\text{s.e.}(\bar{x})} = \frac{111.19 - 100}{1.0022} = 11.87.
\] The conclusions do not change: the \(P\)-value is still extremely small.

\section{Chapter summary}\label{Chap27-Summary}

These steps are used to test a hypothesis about a population mean \(\mu\).

\begin{itemize}
\tightlist
\item
  Write the null hypothesis (\(H_0\)) and the alternative hypothesis (\(H_1\)); initially \emph{assume} the value of \(\mu\) in the null hypothesis to be true.
\item
  Describe the \emph{sampling distribution}, which describes what to \emph{expect} from the sample mean based on this assumption: under certain statistical validity conditions, the sample mean varies with:

  \begin{itemize}
  \tightlist
  \item
    an approximate normal distribution,
  \item
    with sampling mean whose value is the value of \(\mu\) (from \(H_0\)), and
  \item
    having a standard deviation of \(\displaystyle \text{s.e.}(\bar{x}) =\frac{s}{\sqrt{n}}\).
  \end{itemize}
\item
  Compute the value of the \emph{test statistic}: \[
   t = \frac{ \bar{x} - \mu}{\text{s.e.}(\bar{x})},
  \] where \(\mu\) is the hypothesised value given in the null hypothesis.
\item
  The \(t\)-value is like a \(z\)-score, and so an approximate \emph{\(P\)-value} can be estimated using the \(68\)--\(95\)--\(99.7\) rule or tables, or found using software. Use the \(P\)-value to make a decision, and write a conclusion.
\item
  Check the statistical validity conditions.
\end{itemize}

\section{Quick review questions}\label{Chap32-QuickReview}

The usual recommendation for a safe gap between travelling vehicles in traffic (a `headway') is \emph{at least} \(1.9\,\text{s}\) (often rounded to~\(2\,\text{s}\) for the public). \citet{majeed2014field} studied \(n = 28\) streams of traffic in Birmingham, Alabama found the mean headway was \(1.1915\,\text{s}\), with a standard deviation of~\(0.231\,\text{s}\). The researchers wanted to test if the mean headway in Birmingham was \emph{less than} the recommended~\(1.9\,\text{s}\).

Are the following statements \emph{true} or \emph{false}?

\begin{enumerate}
\def\labelenumi{\arabic{enumi}.}
\item
  The standard error of the mean is \(0.231\,\text{s}\). \tightlist
\item
  The null hypothesis is `The sample mean headway is \(1.9\,\text{s}\)'.
\item
  The alternative hypothesis `The population mean is less than \(1.9\,\text{s}\)'.
\item
  The test is \emph{one-tailed}.
\item
  The value of the test statistic is \(t = -16.23\).
\item
  The one-tailed \(P\)-value is very small.
\item
  There is no evidence to support the \emph{alternative} hypothesis.
\end{enumerate}

\section{Exercises}\label{TestOneMeanAnswerExercises}

\hyperref[Answers]{Answers to odd-numbered exercises} are given at the end of the book.

\captionsetup{font=small}

\begin{exercise}
\protect\hypertarget{exr:OneTSpeed}{}\label{exr:OneTSpeed}

\citet{azwari2021evaluating} studied driving speeds in Malaysia, and recorded the speeds of vehicles on various roads. One RQ was whether the mean speed of cars on one particular road was the posted speed limit of~\(90\,\text{km}.\text{h}^{-1}\), or whether it was \emph{higher}.

The researchers recorded the speed of \(n = 400\) vehicles on this road, and found the mean and standard deviation of the speeds of individual vehicles were \(\bar{x} = 96.56\) and \(s = 13.874\,\text{km}.\text{h}^{-1}\).

\begin{enumerate}
\def\labelenumi{\arabic{enumi}.}
\tightlist
\item
  Define the parameter of interest.
\item
  Write the statistical hypotheses.
\item
  Compute the standard error of the sample mean.
\item
  Sketch the sampling distribution of the sample mean for \(n = 400\).
\item
  Compute the test statistic, a \(t\)-score.
\item
  Determine the \(P\)-value, and write a conclusion.
\item
  Is the test statistically valid?
\end{enumerate}

\end{exercise}

\begin{exercise}
\protect\hypertarget{exr:TestOneMeanExercisesSlalom}{}\label{exr:TestOneMeanExercisesSlalom}

A competitive slalom competitor completed \(n = 30\) attempts on a \(38.8\,\text{m}\) kayak slalom course to assess the accuracy of a GPS tracking system \citep{macdermidValidityReliabilityGlobal2022}. The trials produced a mean distance, recorded by the GPS, as \(36.54\,\text{m}\) with a standard deviation of \(2.07\,\text{m}\).

\begin{enumerate}
\def\labelenumi{\arabic{enumi}.}
\tightlist
\item
  Define the parameter of interest.
\item
  Write the statistical hypotheses.
\item
  Compute the standard error of the sample mean.
\item
  Sketch the sampling distribution of the sample mean for \(n = 30\).
\item
  Compute the test statistic, a \(t\)-score.
\item
  Determine the \(P\)-value, and write a conclusion.
\item
  Is the test statistically valid?
\end{enumerate}

\end{exercise}

\begin{exercise}
\protect\hypertarget{exr:TestOneMeanExercisesAutomatedVehicles}{}\label{exr:TestOneMeanExercisesAutomatedVehicles}\citet{data:greenlee2018:vehicles} conducted a study of human--automation interaction with automated vehicles. They were interested in whether the average mental demand of `drivers' of automated vehicles was \emph{higher} than the average mental demand for ordinary tasks.

In the study, the \(n = 22\) participants `drove' (in a simulator) an automated vehicle for~\(40\,\text{mins}\). While driving, the drivers monitored the road for hazards. The researchers assessed the `mental demand' placed on these drivers, where scores over~\(50\) `typically indicate substantial levels of workload' (p.~471). For the sample, the mean score was~\(84.00\) with a standard deviation of~\(22.05\).

Is there evidence of a `substantial workload' associated with monitoring roadways while `driving' automated vehicles?
\end{exercise}

\begin{exercise}
\protect\hypertarget{exr:TestOneMeanWaterTemp}{}\label{exr:TestOneMeanWaterTemp}Health departments recommend that hot water be stored at \(60\)\textsuperscript{o}C or higher, to kill \emph{legionella} bacteria. \citet{alary1991risk} studied \(n = 178\) Quebec homes with electric water heaters to see if the mean water temperature was less than \(60\)\textsuperscript{o}C (i.e., at risk).

The mean temperature was~\(56.6\)\textsuperscript{o}C, with a standard error of~\(0.4\)\textsuperscript{o}C.\spacex Is there evidence the mean water temperature in Quebec is too low to kill \emph{legionella} bacteria?
\end{exercise}

\begin{exercise}
\protect\hypertarget{exr:TestOneMeanExercisesCherryRipes}{}\label{exr:TestOneMeanExercisesCherryRipes}

{[}\emph{Dataset}: \texttt{CherryRipe}{]} A \emph{Cherry Ripe} is a popular Australian chocolate bar. In~2017,~2018 and~2019, I sampled some \emph{Cherry Ripe} Fun Size bars. The packaging claimed that the Fun Size bars weigh~\(14\,\text{g}\) (on average).

\begin{enumerate}
\def\labelenumi{\arabic{enumi}.}
\tightlist
\item
  Use the software output (Fig.~\ref{fig:CherryRipes201720182019}) to determine if the mean weight is~\(14\,\text{g}\) or not.
\item
  Explain the difference in the meaning of \texttt{SD} and \texttt{SE} in this context.
\end{enumerate}

\end{exercise}



\begin{figure}[hbtp]

{\centering \includegraphics[width=0.6\linewidth]{jamovi/CherryRipe/CherryRipe-Descriptives} 

}

\caption{Software output for the \emph{Cherry Ripes} data.}\label{fig:CherryRipes201720182019}
\end{figure}

\begin{exercise}
\protect\hypertarget{exr:TestOneMeanBloodLoss}{}\label{exr:TestOneMeanBloodLoss}(This study was also seen in Exercise~\ref{exr:CIOneMeanBloodLoss}.) \citet{data:Williams2007:BloodLoss} asked \(n = 199\) paramedics to estimate the amount of blood on four different surfaces. When the actual amount of blood spilt on concrete was~\(\,1000\,\text{mL}\), the mean guess was~\(846.4\,\text{mL}\) (with \(s = 651.1\,\text{mL}\)).

Is there evidence that the mean guess is~\(1\,000\,\text{mL}\) (the true amount)? Is this test statistically valid?
\end{exercise}

\begin{exercise}
\protect\hypertarget{exr:TestOneMeanExercisesSleep}{}\label{exr:TestOneMeanExercisesSleep}\citet{lin2021sleep} compared the average sleep times of Taiwanese pre-school children to the recommendation (of \emph{at least}~\(10\,\text{h}\) per night). Using the summary of the data for weekend sleep-times (Table~\ref{tab:SleepingSummary}), do girls get \emph{less than} \(10\,\text{h}\) of sleep per night, on average? Do boys?
\end{exercise}

\begin{table}
\centering
\caption{\label{tab:SleepingSummary}Summary information for the Taiwanese pre-schoolers sleep times (in h).}
\centering
\fontsize{8}{10}\selectfont
\begin{tabular}[t]{lccc}
\toprule
\textbf{ } & \textbf{Sample size} & \textbf{Sample mean} & \textbf{Sample std dev.}\\
\midrule
Boys & $47$ & $8.50$ & $0.48$\\
Girls & $39$ & $8.64$ & $0.37$\\
\bottomrule
\end{tabular}
\end{table}

\begin{exercise}
\protect\hypertarget{exr:TestOneMeanQualityControl}{}\label{exr:TestOneMeanQualityControl}{[}\emph{Dataset}: \texttt{LHconc}{]} \citet{feng2017application} assessed the accuracy of two instruments from a clinical laboratory, by comparing the reported luteotropichormone (LH) concentrations to known, pre-determined values using \(n = 36\) samples. Use hypothesis tests to determine how the instruments perform, for both high- and mid-level LH concentrations (using the information in Table~\ref{tab:QualityControlData}).
\end{exercise}

\begin{table} \centering \centering\caption{\label{tab:QualityControlData}The quality-control data: LH levels (in mIU.mL$^{-1}$) for two instruments (only the first four of $36$\ observations shown).}

\fontsize{8}{10}\selectfont
\begin{tabular}{rcc}
\toprule
\multicolumn{1}{c}{\textbf{ }} & \multicolumn{2}{c}{\textbf{Instrument 1}} \\
\cmidrule(l{3pt}r{3pt}){2-3}
\textbf{ } & \textbf{High level} & \textbf{Mid level}\\
\midrule
 & $61.63$ & $18.36$\\
 & $63.11$ & $18.77$\\
 & $66.88$ & $18.98$\\
 & $62.56$ & $17.97$\\
 & $\vdots$ & $\vdots$\\
\midrule
\textbf{Mean} & $64.31$ & $19.24$\\
\textbf{Std deviation} & $\phantom{0}1.70$ & $\phantom{0}0.59$\\
\textbf{Target} & $64.22$ & $19.01$\\
\bottomrule
\end{tabular} \quad\quad 
\begin{tabular}{rcc}
\toprule
\multicolumn{1}{c}{\textbf{ }} & \multicolumn{2}{c}{\textbf{Instrument 2}} \\
\cmidrule(l{3pt}r{3pt}){2-3}
\textbf{ } & \textbf{High level} & \textbf{Mid level}\\
\midrule
 & $62.64$ & $19.12$\\
 & $64.36$ & $19.07$\\
 & $66.06$ & $19.58$\\
 & $65.39$ & $19.35$\\
 & $\vdots$ & $\vdots$\\
\midrule
\textbf{Mean} & $64.97$ & $19.40$\\
\textbf{Std deviation} & $\phantom{0}1.03$ & $\phantom{0}0.41$\\
\textbf{Target} & $65.05$ & $19.45$\\
\bottomrule
\end{tabular}
\end{table}

\begin{exercise}
\protect\hypertarget{exr:PizzasHT}{}\label{exr:PizzasHT}

{[}\emph{Dataset}: \texttt{PizzaSize}{]} (This study was also seen in Exercise~\ref{exr:CIPizzas}.) In 2011, \emph{Eagle Boys Pizza} ran a campaign that claimed that \emph{Eagle Boys} pizzas were `Real size \(12\)-inch large pizzas' \citep{mypapers:Dunn:PizzaSize}. \emph{Eagle Boys} made the data from the campaign publicly available. Using the summary of the diameters of a sample of~\(125\) of their large pizzas (Fig.~\ref{fig:PizzaSoftwareHTjamovi}), test the company's claim:

\begin{quote}
For \emph{Eagle Boys}' pizzas, is mean diameter actually~\(12\) inches, or not?
\end{quote}

\begin{enumerate}
\def\labelenumi{\arabic{enumi}.}
\tightlist
\item
  What is the parameter of interest?
\item
  Write down the values of~\(\bar{x}\) and~\(s\).
\item
  Determine the value of the standard error of the mean.
\item
  Write the hypotheses to test if the mean pizza diameter is~\(12\)~inches.
\item
  Is the alternative hypothesis one- or two-tailed? Why?
\item
  Draw the normal distribution that shows how the \emph{sample mean pizza diameter} would vary by chance, \emph{even if} the population mean diameter was \(12\)~inches.
\item
  Compute the \(t\)-score for testing the hypotheses.
\item
  What is the approximate \(P\)-value using the \(68\)--\(95\)--\(99.7\) rule?
\item
  Write a conclusion: do pizzas have a mean diameter of~\(12\)~inches, as claimed?
\item
  Is it reasonable to assume the \emph{statistical} validity conditions are satisfied?
\end{enumerate}

\end{exercise}



\begin{figure}[hbtp]

{\centering \includegraphics[width=0.4\linewidth]{jamovi/PizzaDiameter/PizzaDiameters-jamovi} 

}

\caption{Summary statistics for the diameter of \emph{Eagle Boys} large pizzas.}\label{fig:PizzaSoftwareHTjamovi}
\end{figure}

\begin{exercise}
\protect\hypertarget{exr:OneMeanHtExerciseSleepTime}{}\label{exr:OneMeanHtExerciseSleepTime}\citet{saxvig2021sleep} studied the length of sleep each night for a `large and representative sample of Norwegian adolescents' (p.~1) aged~\(16\) and~\(17\)~years of age. The recommendation is for adolescents to have at least~\(8\,\text{h}\) of sleep each night.

In the sample of \(n = 3\,972\) individuals, the mean amount of sleep on schools days was \(6\,\text{h}\)~\(43\,\text{mins}\) (i.e., \(403\,\text{mins}\)), with a standard deviation of \(87\,\text{mins}\). On non-school days, the mean amount of sleep was \(8\,\text{h}\)~\(38\,\text{mins}\) (i.e., \(518\,\text{mins}\)), with a standard deviation of~\(98\,\text{mins}\).

Do Norwegian adolescents appear to meet the guidelines of having `\emph{at least}~\(8\,\text{h}\)' sleep each night on school days? On non-school days?
\end{exercise}

\captionsetup{font=normalsize}

\begin{EOCanswerBox}{iconmonstr-check-mark-14-240.png}
\textbf{Answers to \textit{Quick review} questions:} \textbf{1.} False: \(0.0436\). \textbf{2.} False: \emph{population} mean. \textbf{3.} True. \textbf{4.} True. \textbf{5.} True. \textbf{6.} True. \textbf{7.} False.

\end{EOCanswerBox}

\chapter{More details about hypothesis testing}\label{MoreAboutTests}

\begin{cols}
\begin{col}{0.52\textwidth}

\begin{objectivesBox}{iconmonstr-target-4-240.png}
You have learnt to ask an RQ, design a study, classify and summarise the data, construct confidence intervals, and conduct hypothesis tests.
\textbf{In this chapter}, you will learn more about \emph{hypothesis tests}.
You will learn to:
\begin{itemize}\tightlist
  \item
  understand the process of hypothesis testing.
  \item
  communicate the results of hypothesis tests.
  \item
  interpret $P$-values.
\end{itemize}
\end{objectivesBox}

\end{col}

\begin{col}{0.03\textwidth}
~
\end{col}

\begin{col}{0.45\textwidth}

\includegraphics[width=0.95\linewidth]{28-Testing-More_files/figure-latex/unnamed-chunk-6-1} 
\end{col}
\end{cols}

\section{Introduction}\label{Chap28-Intro}

In Chaps.~\ref{TestOneProportion} and~\ref{TestOneMean}, hypothesis tests for one proportion and one mean were studied. Later chapters discuss hypothesis tests in other contexts, too. However, the general approach to hypothesis testing is the same for \emph{any} hypothesis test. This chapter discusses some general ideas in hypothesis testing:

\begin{itemize}
\tightlist
\item
  stating the \emph{assumptions} and forming hypotheses (Sect.~\ref{AboutHypotheses}).
\item
  describing the \emph{expectations} of the statistic using the sampling distribution (Sect.~\ref{SamplingDistributionsExpectation}).
\item
  evaluating \emph{observations} and the test statistic (Sect.~\ref{TestStatistic}).
\item
  quantifying the \emph{consistency} between the values of the statistic and parameter using \(P\)-values (Sect.~\ref{AboutFindingPvalues}).
\item
  interpreting \(P\)-values (Sect.~\ref{AboutPvalues}).
\item
  how conclusions can go wrong (Sect.~\ref{TypeErrors}).
\item
  wording \emph{conclusions} (Sect.~\ref{WordingConclusion}).
\item
  practical importance and statistical significance (Sect.~\ref{PracticalSignificance}).
\item
  statistical validity in hypothesis testing (Sect.~\ref{ValidityHTs}).
\end{itemize}

\section{More details about hypotheses and assumptions}\label{AboutHypotheses}

Two \emph{statistical} hypotheses are stated about the population parameter: the null hypothesis~\(H_0\), and the alternative hypothesis~\(H_1\). The null hypothesis is assumed to be true, and retained unless persuasive evidence exists to change our mind.

\begin{tipBox}{iconmonstr-info-6-240.png}
The word \emph{hypothesis} means `a possible explanation'.\index{Hypotheses}

\emph{Scientific hypotheses}\index{Hypotheses!scientific} refer to potential \emph{scientific} explanations that can be tested by collecting data. For example, an engineer may hypothesise that replacing sand with glass in the manufacture of concrete will produce desirable characteristics \citep{devaraj2021exploring}. Scientific hypotheses lead to research questions.

\emph{Statistical hypotheses}\index{Hypotheses!statistical} refer to statements made about a parameter that may explain the value of a sample statistic. The statistical hypotheses are the foundation of the logic of hypothesis testing. One of the statistical hypotheses usually align with the scientific hypothesis.

This book discusses forming \emph{statistical hypotheses}.

\end{tipBox}

\subsection{Null hypotheses}\label{HypothesisNull}

\index{Hypotheses!null}

Statistical hypotheses \emph{are always about a parameter}. Hypothesising, for example, that the \emph{sample} mean body temperature (in Chap.~\ref{TestOneMean}) is equal to~\(37.0\)\textsuperscript{o}C is silly: the \emph{sample} mean clearly is~\(36.8052\)\textsuperscript{o}C for the sample taken, and its value will vary from sample to sample anyway. The RQ is about the unknown \emph{population}: the \textbf{P} in \textbf{P}OCI stands for \textbf{P}opulation.\index{POCI}

The \emph{null hypothesis}~\(H_0\) proposes that \emph{sampling variation} is why the value of the statistic (such as the sample mean) is not the same as the assumed value of the parameter (such as the population mean). Every sample is different, and the observed data is from just one of the many possible samples. The value of the \emph{statistic} will vary from sample to sample; the statistic may not be equal to the \emph{parameter}, just because of the random sample obtained and sampling variation.

\begin{definition}[Null hypothesis]
\protect\hypertarget{def:NullHypothesis}{}\label{def:NullHypothesis}The \emph{null hypothesis} proposes that \emph{sampling variation} explains the discrepancy between the proposed value of the parameter, and the observed value of the statistic.\index{Sampling variation}
\end{definition}

Null hypotheses always contain an `equals', because (as part of the decision-making process) a specific value must be assumed for the parameter, so we can describe what we might expect from the sample. For example: the population mean \emph{equals}~\(100\), is \emph{less than or equal to}~\(100\) (\(\mu\le100\)), or is \emph{more than or equal to}~\(100\) (\(\mu\ge100\)).

The null hypothesis always assumes the discrepancy between the statistic and the assumed value of the parameter is due to sampling variation. This may mean, for example:

\begin{itemize}
\tightlist
\item
  there is \emph{no change} in the value of the parameter compared to an established or accepted value (for descriptive RQs), such as in the body-temperature example in Chap.~\ref{TestOneMean}.
\item
  there is \emph{no change} in the value of the parameter for the units of analysis (i.e., for repeated-measures RQs).
\item
  there is \emph{no difference} between the value of the parameter in two (or more) groups (i.e., for relational RQs).
\item
  there is \emph{no relationship} between the variables, as measured by some parameter (for correlational RQs).
\end{itemize}

\begin{importantBox}{iconmonstr-warning-8-240.png}
The \emph{null hypothesis} always has the form `no difference, no change, no relationship' regarding the population parameter. It is the `sampling variation' explanation for the discrepancy between the value of the parameter and the value of the statistic.

\end{importantBox}

\begin{importantBox}{iconmonstr-warning-8-240.png}
Defining the parameter carefully is important!

\end{importantBox}

\subsection{Alternative hypotheses}\label{HypothesisAlternative}

\index{Hypotheses!alternative}

The alternative hypothesis~\(H_1\) (or~\(H_a\)) offers another possible reason why the value of the statistic (such as the sample proportion) is not the same as the proposed value of the parameter (such as the population proportion): the value of the parameter really is not the value claimed in the null hypothesis.

\begin{definition}[Alternative hypothesis]
\protect\hypertarget{def:AltHypothesis}{}\label{def:AltHypothesis}The \emph{alternative hypothesis} proposes that the discrepancy between the proposed value of the parameter and the observed value of the statistic cannot be explained by \emph{sampling variation}. It proposes that the value of the parameter is not the value claimed in the null hypothesis.
\end{definition}

Alternative hypotheses can be \emph{one-tailed} or \emph{two-tailed}.\index{Hypotheses!one-tailed}\index{Hypotheses!two-tailed} A \emph{two}-tailed alternative hypothesis means, for example, that the population mean could be either smaller \emph{or} larger than what is claimed. A \emph{one}-tailed alternative hypothesis admits only one of those two possibilities. Most (but certainly not all) hypothesis tests are two-tailed.

The decision about whether the alternative hypothesis is one- or two-tailed depends on what the RQ asks (\emph{not} by looking at the data). \emph{The RQ and hypotheses should (in principle) be formed before the data are obtained}, or at least before looking at the data if the data are already collected.

The idea of hypothesis testing is the same whether the alternative hypothesis is one- or two-tailed: based on the data and the statistic, a decision is to be made about whether the data provides persuasive evidence to support the alternative hypothesis.

\begin{example}[Alternative hypotheses]
\protect\hypertarget{exm:AltHypothesisBodyTemp}{}\label{exm:AltHypothesisBodyTemp}For the body-temperature study (Chap.~\ref{TestOneMean}), the alternative hypothesis is \emph{two-tailed} (i.e., \(H_1\): \(\mu \ne 37.0\)): the RQ asks if the population mean is~\(37.0\)\textsuperscript{o}C or \emph{not}. Two possibilities are considered: that \(\mu\) could be either larger \emph{or} smaller than~\(37.0\).

A \emph{one-tailed alternative hypothesis} would be appropriate if the RQ asked `Is the \emph{population} mean internal body temperature \emph{greater} than~\(37.0\)\textsuperscript{o}C?' (i.e., \(H_1\): \(\mu > 37.0\)), or `Is the \emph{population} mean internal body temperature \emph{smaller} than~\(37.0\)\textsuperscript{o}C?' (i.e., \(H_1\): \(\mu < 37.0\)). One-tailed RQs such as these would only be asked if there were good scientific reasons to suspect a difference in one direction specifically.
\end{example}

\begin{importantBox}{iconmonstr-warning-8-240.png}

Important points about forming hypotheses:

\begin{itemize}
\tightlist
\item
  hypotheses always concern a \emph{population} parameter.
\item
  hypotheses emerge from the RQ (not the data).
\item
  null hypothesis always have the form `no difference, no change, no relationship' (i.e., sampling variation explains the discrepancy between the values of the parameter and statistic).
\item
  null hypotheses always contain an `equals'.
\item
  alternative hypotheses may be one- or two-tailed, depending on the RQ.
\end{itemize}

\end{importantBox}

\section{More details about sampling distributions and expectations}\label{SamplingDistributionsExpectation}

\index{Sampling distribution}

The \emph{sampling distribution} describes, approximately, how the value of the statistic (such as~\(\hat{p}\) or~\(\bar{x}\)) varies across all possible samples, when \(H_0\) is true; it describes the \emph{sampling distribution}.\index{Sampling variation}\index{Sampling distribution} Some sampling distributions have an approximate normal distribution.\index{Normal distribution}

When the sampling distribution is described by a normal distribution, the \emph{mean} of the normal distribution (the sampling mean) is the parameter value given in the \emph{assumption} (\(H_0\)), and the \emph{standard deviation} of the normal distribution is called the \emph{standard error}.\index{Standard error} However, \emph{not all sampling distributions are normal distributions}.

The variation in the sampling distribution (as measured by the standard error) depends on the sample size. For example, suppose \(p\) is defined as the probability of rolling a \largedice{1} on a die. In one roll, finding a sample proportion of \(\hat{p} = 1\), is not unreasonable. However, in \(20\,000\)~rolls, a sample proportion of \(\hat{p} = 1\) would be \emph{incredibly} unlikely for a fair die.

\section{More details about observations and the test statistic}\label{TestStatistic}

The sampling distribution describes what values the statistic can take over all possible samples of a given size. When the sampling distribution has an approximate normal distribution, the observed value of the \emph{test statistic}\index{Test statistic} is \[
   \text{test statistic} = 
   \frac{\text{value of sample statistic} - \text{centre of the sampling distribution}}
        {\text{standard deviation of the sampling distribution (i.e., standard error)}}.
\] The `standard deviation of the sampling distribution' is called the standard error of the statistic. This is called a `\emph{test statistic}', since the calculation is based on sample data (so it is a \emph{statistic}) and used in a hypothesis \emph{test}. This test statistic may be a \(z\)-score or a \(t\)-score. Other test statistics, when the sampling distribution is not described by a normal distribution, are used too (as in Chap.~\ref{AnalysisOddsRatio}).

\begin{tipBox}{iconmonstr-info-6-240.png}
\index{Test statistic!t@$t$-score}\index{Test statistic!z@$z$-score} For sampling distributions with an approximate normal distribution, a \(t\)-score and \(z\)-score both measure the number of standard deviations that a value is from the mean: \[
   \frac{\text{a value that varies} - \text{mean of the distribution}}
        {\text{standard deviation of the distribution}}.
\] Then:

\begin{itemize}
\tightlist
\item
  if the quantity that varies is an \emph{individual} observation \(x\), the measure of variation is the standard deviation of the individual observations.
\item
  if the quantity that varies is a \emph{sample statistic}, the measure of variation is a \emph{standard error}, which measures the variation in a sample statistic.
\end{itemize}

\emph{When conducting hypothesis tests about means}, the test statistic is a \(t\)-score if the measure of variation uses a \emph{sample} standard deviation.

\end{tipBox}

\section{\texorpdfstring{More details about finding \(P\)-values}{More details about finding P-values}}\label{AboutFindingPvalues}

\index{P@$P$-values}

When the sampling distribution has an approximate normal distribution, \(P\)-values can be \emph{approximated} (using the \(68\)--\(95\)--\(99.7\) rule or tables), as demonstrated in Sect.~\ref{TestpObsDecisionPvalues}. The \(P\)-value is the area \emph{more extreme} than the calculated \(z\)- or \(t\)-score (i.e., in the \emph{tails} of the distribution). The \(68\)--\(95\)--\(99.7\) rule can be used to approximate this tail area (when the sampling distribution has an approximate normal distribution).

\begin{importantBox}{iconmonstr-warning-8-240.png}
A lower-case~\(p\) or upper-case~\(P\) can be used to denote a \(P\)-value. We use an upper-case~\(P\), since we use~\(p\) to denote a population proportion.

\end{importantBox}

For \emph{two-tailed} tests, the \(P\)-value is the \emph{combined} area in the left and right tails.\index{Hypotheses!one-tailed}\index{Hypotheses!two-tailed}\index{P@$P$-values!one-tailed}\index{P@$P$-values!two-tailed} For \emph{one-tailed} tests, the \(P\)-value is the area in just the left or right tail (as appropriate, according to the alternative hypothesis; see Sect.~\ref{IQstudents}).

\begin{importantBox}{iconmonstr-warning-8-240.png}
If the sampling distribution has an approximate normal distribution, the one-tailed \(P\)-value is half the value of the two-tailed \(P\)-value.

\end{importantBox}

\begin{softwareBox}{iconmonstr-laptop-4-240.png}
Some software always reports two-tailed \(P\)-values.\index{Computers and software!statistical}

\end{softwareBox}

More accurate approximations of the \(P\)-value can be found using tables. Precise \(P\)-values are found using the \(P\)-values from software output.

\section{\texorpdfstring{More details about interpreting \(P\)-values}{More details about interpreting P-values}}\label{AboutPvalues}

\index{P@$P$-values!interpretation}\index{Hypothesis testing!interpretation}

Understanding \(P\)-values requires care.

\begin{definition}[$P$-value]
\protect\hypertarget{def:Pvalue}{}\label{def:Pvalue}A \(P\)-value is the likelihood of observing the sample statistic (or something more extreme) over repeated sampling, under the assumption that the null hypothesis about the population parameter is true.
\end{definition}

Since the null hypothesis is initially assumed true, \emph{the onus is on the data to present evidence to contradict the null hypothesis}. That is, the null hypothesis is retained unless persuasive evidence suggests otherwise.

\begin{importantBox}{iconmonstr-warning-8-240.png}
Conclusions are \emph{always} about the parameters. \(P\)-values tell us about the unknown \emph{parameters}, based on the data from one of the many possible values of the \emph{statistic}.

\end{importantBox}

A `big' \(P\)-value means that the sample statistic (such as~\(\hat{p}\)) could reasonably have occurred through sampling variation in one of the many possible samples, if the assumption made about the parameter (stated in~\(H_0\)) was true. A `small' \(P\)-value means that the sample statistic (such as~\(\hat{p}\)) is unlikely to have occurred through sampling variation in one of the many possible samples, if the assumption made about the parameter (stated in~\(H_0\)) was true. `Small' \(P\)-values provide persuasive evidence to support the alternative hypothesis.

Commonly, a \(P\)-value smaller than~\(5\)\% (or~\(0.05\)) is considered `small' but this is \emph{arbitrary}, and sometimes the threshold is discipline-dependent. More reasonably, \(P\)-values should be interpreted as giving varying degrees of evidence in support of the alternative hypothesis (Table~\ref{tab:PvaluesInterpretation}), but these too are only guidelines.

\begin{importantBox}{iconmonstr-warning-8-240.png}
The threshold for a `small' \(P\)-value is very commonly~\(0.05\), but this is arbitrary and not universal. There is nothing special about the value~\(0.05\), and there is very little difference in the meaning of a \(P\)-value of~\(0.051\) and a \(P\)-value of~\(0.049\).

\end{importantBox}

\begin{table}
\centering
\caption{\label{tab:PvaluesInterpretation}A guideline for interpreting \(P\)-values. \(P\)-values should be interpreted in context, and indicate the strength of evidence to support the alternative hypothesis.}
\centering
\fontsize{8}{10}\selectfont
\begin{tabular}[t]{rl}
\toprule
\textbf{If the $P$-value is...} & \textbf{Write the conclusion as...}\\
\midrule
Larger than $0.10$ & \emph{Insufficient} evidence to support $H_1$\\
Between $0.05$ and $0.10$ & \emph{Slight} evidence to support $H_1$\\
Between $0.01$ and $0.05$ & \emph{Moderate} evidence to support $H_1$\\
Between $0.001$ and $0.01$ & \emph{Strong} evidence to support $H_1$\\
Smaller than $0.001$ & \emph{Very strong} evidence to support $H_1$\\
\bottomrule
\end{tabular}
\end{table}

Identifying a \(P\)-value of~\(0.05\) as `small' (and hence providing `persuasive evidence' to support~\(H_1\)) is arbitrary; it means that, if~\(H_0\) is true, there is a~\(1\)-in-\(20\) chance that the value of the statistic (or a value more extreme) would be observed due to sampling variation. In many situations, the evidence must be more persuasive than this.

To appreciate the concept of a~\(0.05\) (or a~\(1\)-in-\(20\)) chance:

\begin{itemize}
\tightlist
\item
  the probability of throwing~\(5\) or more \Heads~in a row using a fair coin is about~\(0.063\).
\item
  the probability of drawing a black Ace from a pack of cards is about \(0.038\).
\item
  the probability of rolling two or more consecutive throws of a \largedice{6} is about~\(0.033\).
\end{itemize}

These events are improbable, without being essentially impossible.

\(P\)-values are commonly used in research, but must be used and interpreted correctly \citep{greenland2016statistical}. Specifically:

\begin{itemize}
\tightlist
\item
  a \(P\)-value \emph{is not} the probability that the null hypothesis is true.
\item
  a \(P\)-value \emph{does not prove} anything (only one possible sample was studied).
\item
  a big \(P\)-value \emph{does not} mean the null hypothesis~\(H_0\) is true, or that~\(H_1\) is false.
\item
  a small \(P\)-value \emph{does not} mean the null hypothesis~\(H_0\) is false, or that~\(H_1\) is true.
\item
  a small \(P\)-value \emph{does not} mean the results are practically important (Sect.~\ref{PracticalSignificance}).
\item
  a small \(P\)-value does not necessarily mean a large difference between the statistic and parameter; it means that the difference (whether large or small) could not reasonably be attributed to \emph{sampling variation} (chance).
\end{itemize}

\begin{importantBox}{iconmonstr-warning-8-240.png}
\(P\)-values are never \emph{exactly} zero. Some software reports very small \(P\)-values as `\(P < 0.001\)' (i.e., the \(P\)-value is smaller than \(0.001\)).\index{Computers and software!statistical} Some software reports very small \(P\)-values as `\(P = 0.000\)' (i.e., zero to three decimal places). In either case, we should still write \(P < 0.001\).

Some software only reports two-tailed \(P\)-values.

\end{importantBox}

\begin{importantBox}{iconmonstr-warning-8-240.png}
Sometimes the results of a study are reported as being \emph{statistically significant}.\index{Statistical significance} This usually means that the \(P\)-value is less than~\(0.05\), though a different \(P\)-value is sometimes used as the `threshold', so check!

To avoid confusion, the word `significant' should be avoided in writing about research unless `statistical significance' is actually meant. In other situations, consider using words like `substantial'.

\end{importantBox}

\section{More details about how conclusions can go wrong}\label{TypeErrors}

In hypothesis testing, a decision is made about a \emph{population} using \emph{sample} information. Since the observed sample is just one of countless possible samples that could have been observed, making an incorrect conclusion is always a possibility.

Two mistakes can be made when making a conclusion:

\begin{itemize}
\tightlist
\item
  \emph{incorrectly} concluding that evidence supports the alternative hypothesis. Of course, the researchers \emph{do not know they are incorrect}, but the possibility of making this mistake is always present. This is a \emph{false positive}, or a \emph{Type~I error}.\index{Type\ I error}
\item
  \emph{incorrectly} concluding there is \emph{no} evidence to support the alternative hypothesis. Of course, the researchers \emph{do not know they are incorrect}, but the possibility of making this mistake is always present. This is a \emph{false negative}, or a \emph{Type~II error}.\index{Type\ II error}
\end{itemize}

Ideally, neither of these errors would be made; however, sampling variation means that neither can ever be completely eliminated. In practice, hypothesis testing begins by assuming the null hypothesis is true, and hence places the onus on the data to provide persuasive evidence in favour of the alternative hypothesis. This means researchers usually prioritise minimising the chance of a Type~I error.

A Type~I error is like declaring an innocent person guilty (recall: innocence is presumed in the judicial system). Similarly, a Type~II error is like declaring a guilty person innocent. The law generally sees a Type~I error as more grievous than a Type~II error, just as in research. In general, larger sample sizes\index{Sample size} reduce the probability of making Type~I and Type~II errors.

In medical contexts, the similar concepts of \emph{sensitivity} and \emph{specificity} are often used rather than the terms \emph{Type~I} and \emph{Type~II errors}. \emph{Sensitivity}\index{Sensitivity} is the probability of a \emph{positive} test result among those \emph{with} the disease, and \emph{specificity}\index{Specificity} is the probability of a \emph{negative} test result among those \emph{without} the disease. High sensitivity is associated with a low chance of Type~II error, and higher specificity is associated with a low chance of a Type~I.

\begin{example}[Type I errors]
\protect\hypertarget{exm:ExampleTypeITypeII}{}\label{exm:ExampleTypeITypeII}For the body-temperature example (Chap.~\ref{TestOneMean}), the conclusion was that the sample provided very strong evidence that the population mean body temperature was \emph{not}~\(37.0\)\textsuperscript{o}C. However, in truth, the mean internal body may not have changed, and is still~\(37.0\)\textsuperscript{o}C; that is, the null hypothesis actually is true, but we incorrectly decided it was probably not true.

This would be a Type I error: we \emph{incorrectly} concluded that the evidence supported the alternative hypothesis. Of course, since the value of~\(\mu\) is unknown, we do not know if we have made a Type~I error or not.
\end{example}

\section{More details about writing conclusions}\label{WordingConclusion}

\index{Hypothesis testing!writing conclusions} In general, communicating the result of a hypothesis test requires stating:

\begin{enumerate}
\def\labelenumi{\arabic{enumi}.}
\tightlist
\item
  the \emph{answer} to the RQ.
\item
  the \emph{evidence} used to reach that conclusion (such as the \(t\)-score and \(P\)-value, clarifying if the \(P\)-value is \emph{one-tailed} or \emph{two-tailed}).
\item
  \emph{sample summary statistics} (such as sample means, with CIs and sample sizes).
\end{enumerate}

Since we initially assume the null hypothesis is true, conclusions are worded (in context) in terms of how strongly the evidence supports the alternative hypothesis.

\begin{importantBox}{iconmonstr-warning-8-240.png}

Since the null hypothesis is initially assumed to be true, the onus is on the data to provide evidence in support of the alternative hypothesis: the null hypothesis is retained unless persuasive evidence suggests otherwise. Hence, conclusions are always worded in terms of how much evidence supports the \emph{alternative} hypothesis.

We \emph{do not} say whether the evidence supports the null hypothesis; the null hypothesis is already assumed to be true. Even if the current sample presents no evidence to contradict the assumption, future evidence may emerge. That is:

\begin{quote}
`No evidence of a difference' is \emph{not} the same as `evidence of no difference'.
\end{quote}

\end{importantBox}

\begin{example}[No evidence of a difference]
\protect\hypertarget{exm:NoEvidenceOfADifference}{}\label{exm:NoEvidenceOfADifference}Suppose, when we tested if the mean internal body temperature remained \(37.0\)\textsuperscript{o}C (Chap.~\ref{TestOneMean}), that we found \emph{no evidence} that the temperature had changed. This \emph{does not} provide evidence that the mean internal body temperature is \(37.0\)\textsuperscript{o}C. It just means that the sample provided no evidence to change our initial \emph{assumption} that the mean internal body temperature is~\(37.0\)\textsuperscript{o}C.
\end{example}

\section{More details about practical importance, statistical significance}\label{PracticalSignificance}

\index{Practical importance}\index{Statistical significance}

Hypothesis tests assess \emph{statistical significance}, which answers the question: `Can sampling variation reasonably explain the discrepancy between the value of the statistic and the assumed value of the parameter?' Even very small discrepancies between the statistic and the parameter can be \emph{statistically} different if the sample size is sufficiently large.

In contrast, \emph{practical importance} answers the question: `Is the discrepancy between the values of the statistic and the parameter of any importance \emph{in practice}?' Whether a result is of practical importance depends upon the context: what the data are being used for. `Practical importance' and `statistical significance' are separate issues.

\begin{example}[Practical importance]
\protect\hypertarget{exm:PracticalImportance}{}\label{exm:PracticalImportance}In the body-temperature study (Sect.~\ref{BodyTemperature}), very strong evidence exists that the mean body temperature had changed (`statistical significance'). But the change was so small that, for most purposes, it has no practical importance. In other (e.g., medical) situations, it \emph{may} have practical importance.
\end{example}

\begin{example}[Practical importance]
\protect\hypertarget{exm:PracticalImportanceHerbal}{}\label{exm:PracticalImportanceHerbal}\citet{maunder2020effectiveness} studied the use of herbal medicines for weight loss, and found that the intervention (p.~891)

\begin{quote}
\ldots{} resulted in a statistically significant weight loss compared to placebo, although this was not considered clinically significant.
\end{quote}

This means that the difference in mean weight loss between the placebo and intervention groups was unlikely to be explained by chance (\(P < 0.001\); i.e., `statistical significant'),\index{Statistical significance} but the difference was so small that it was unlikely to be of any use in practice (`practical importance'). In this context, the researchers decided that a weight loss of at least~\(2.5\,\text{kg}\) was of practical importance. However, in the study, the sample mean weight loss was~\(1.61\,\text{kg}\).
\end{example}

\section{More details about statistical validity}\label{ValidityHTs}

\index{Statistical validity (for inference)}

When performing hypothesis tests, \emph{statistical validity conditions} must be true to ensure that the mathematics behind computing the \(P\)-value is sound. For instance, the statistical validity conditions may ensure that the sampling distribution is sufficiently like a normal distribution for the \(68\)--\(95\)--\(99.7\) rule to apply.

If the statistical validity conditions are \emph{not} met, the \(P\)-values (and hence conclusions) may be inappropriate or only approximately correct.

\section{Chapter summary}\label{Chap28-Summary}

Hypothesis testing formalises the decision-making process. Starting with an \emph{assumption} about a parameter of interest, a description of what values the statistic might take is produced (the sampling distribution): this describes what values the statistic is \emph{expected} to take over all possible samples. This sampling distribution is often a normal distribution.

The statistic (the \emph{sample estimate}) is then \emph{observed}, and a \emph{test statistic} is computed to quantify the discrepancy between the values of the parameter (given in~\(H_0\)) and statistic. Using a \(P\)-value, a decision is made about whether the sample evidence supports or contradicts the initial assumption, and hence a \emph{conclusion} is made. When the sampling distribution is an approximate normal distribution, the test statistic is a \(t\)-score or \(z\)-score,\index{z@$z$-score} and \(P\)-values can often be approximated using the \(68\)--\(95\)--\(99.7\) rule.

\section{Quick review questions}\label{Chap33-QuickReview}

Are the following statements \emph{true} or \emph{false}?

\begin{enumerate}
\def\labelenumi{\arabic{enumi}.}
\item
  When a \(P\)-value is very small, a very large difference \emph{must} exist between the statistic and parameter. \tightlist
\item
  The alternative hypothesis is one-tailed if the sample statistic is larger than the hypothesised population parameter.
\item
  When the sampling distribution has an approximate normal distribution, the standard deviation of this normal distribution is called the \emph{standard error}.
\item
  Both \(z\)-scores and \(t\)-scores can be test statistics.
\item
  \(P\)-values can never be exactly zero.
\item
  A \(P\)-value is the probability that the null hypothesis is true.
\end{enumerate}

Select the correct answer:

\begin{enumerate}
\def\labelenumi{\arabic{enumi}.}
\setcounter{enumi}{6}
\tightlist
\item
  What is wrong (if anything) with this null hypothesis: \(H_0 = 37\)?

  \begin{enumerate}
  \def\labelenumii{\alph{enumii}.}
  \tightlist
  \item
    There is nothing wrong.
  \item
    The value of \(37\) is probably a sample value.
  \item
    No parameter is given.
  \item
    This is the alternative (not the null) hypothesis.
  \end{enumerate}
\end{enumerate}

\section{Exercises}\label{MoreAboutTestsExercises}

\hyperref[Answers]{Answers to odd-numbered exercises} are given at the end of the book.

\captionsetup{font=small}

\begin{exercise}
\protect\hypertarget{exr:MoreAboutExercisesApproximatingPValues}{}\label{exr:MoreAboutExercisesApproximatingPValues}

Assuming the statistical validity conditions are satisfied, use the \(68\)--\(95\)--\(99.7\) rule to approximate the \emph{two}-tailed \(P\)-value if:

\begin{cols}

\begin{col}{0.4\textwidth}

\begin{enumerate}
\def\labelenumi{\arabic{enumi}.}
\tightlist
\item
  the \(t\)-score is~\(3.4\).
\item
  the \(t\)-score is~\(-2.9\).
\end{enumerate}

\end{col}

\begin{col}{0.05\textwidth}
~

\end{col}

\begin{col}{0.5\textwidth}

\begin{enumerate}
\def\labelenumi{\arabic{enumi}.}
\setcounter{enumi}{2}
\tightlist
\item
  the \(z\)-score is~\(-2.1\).
\item
  the \(t\)-score is~\(-6.7\).
\end{enumerate}

\end{col}

\end{cols}

\end{exercise}

\begin{exercise}
\protect\hypertarget{exr:MoreAboutExercisesApproximatingPValues2}{}\label{exr:MoreAboutExercisesApproximatingPValues2}

Assuming the statistical validity conditions are satisfied, use the \(68\)--\(95\)--\(99.7\) rule to approximate the \emph{two}-tailed \(P\)-value if:

\begin{cols}

\begin{col}{0.4\textwidth}

\begin{enumerate}
\def\labelenumi{\arabic{enumi}.}
\tightlist
\item
  the \(z\)-score is~\(1.05\).
\item
  the \(t\)-score is~\(-1.3\).
\end{enumerate}

\end{col}

\begin{col}{0.05\textwidth}
~

\end{col}

\begin{col}{0.5\textwidth}

\begin{enumerate}
\def\labelenumi{\arabic{enumi}.}
\setcounter{enumi}{2}
\tightlist
\item
  the \(t\)-score is~\(6.7\).
\item
  the \(t\)-score is~\(0.1\).
\end{enumerate}

\end{col}

\end{cols}

\end{exercise}

\begin{exercise}
\protect\hypertarget{exr:MoreAboutExercisesApproximatingPValuesOneTailed}{}\label{exr:MoreAboutExercisesApproximatingPValuesOneTailed}Consider the test statistics in Exercise \ref{exr:MoreAboutExercisesApproximatingPValues}. Use the \(68\)--\(95\)--\(99.7\) rule to approximate the \emph{one}-tailed \(P\)-values in each case.
\end{exercise}

\begin{exercise}
\protect\hypertarget{exr:MoreAboutExercisesApproximatingPValuesOneTailed2}{}\label{exr:MoreAboutExercisesApproximatingPValuesOneTailed2}Consider the test statistics in Exercise \ref{exr:MoreAboutExercisesApproximatingPValues2}. Use the \(68\)--\(95\)--\(99.7\) rule to approximate the \emph{one}-tailed \(P\)-values in each case.
\end{exercise}

\begin{exercise}
\protect\hypertarget{exr:MoreAboutTestsInterpretingResults}{}\label{exr:MoreAboutTestsInterpretingResults}Suppose a hypothesis test results in a \(P\)-value of~\(0.0501\). What would we conclude? What if the \(P\)-value was~\(0.0499\)? Comment.
\end{exercise}

\begin{exercise}
\protect\hypertarget{exr:MoreAboutTestsInterpretingResults2}{}\label{exr:MoreAboutTestsInterpretingResults2}Suppose a hypothesis test results in a \(P\)-value of~\(0.011\). What would we conclude? What if the \(P\)-value was~\(0.009\)? Comment.
\end{exercise}

\begin{exercise}
\protect\hypertarget{exr:MoreAboutTestsInterpretingHypotheses}{}\label{exr:MoreAboutTestsInterpretingHypotheses}

Consider the study to determine if the mean body temperature (Chap.~\ref{TestOneMean}) was~\(37.0\)\textsuperscript{o}C, where \(\bar{x} = 36.8052\)\textsuperscript{o}C.\spacex Explain \emph{why} each of these sets of hypotheses are incorrect.

\begin{cols}

\begin{col}{0.4\textwidth}

\begin{enumerate}
\def\labelenumi{\arabic{enumi}.}
\tightlist
\item
  \(H_0\): \(\bar{x} = 37.0\); \(H_1\): \(\bar{x} \ne 37.0\).
\item
  \(H_0\): \(\mu = 37\); \(H_1\): \(\mu > 37\).
\item
  \(H_0\): \(\mu = 37\); \(H_1\): \(\mu = 36.8052\).
\end{enumerate}

\end{col}

\begin{col}{0.05\textwidth}
~

\end{col}

\begin{col}{0.5\textwidth}

\begin{enumerate}
\def\labelenumi{\arabic{enumi}.}
\setcounter{enumi}{3}
\tightlist
\item
  \(H_0\): \(\bar{x} = 36.8052\); \(H_1\): \(\bar{x} > 36.8052\).
\item
  \(H_0\): \(\mu = 36.8052\); \(H_1\): \(\mu \ne 36.8052\).
\item
  \(H_0\): \(\mu > 37.0\); \(H_1\): \(\bar{x} > 37.0\).
\end{enumerate}

\end{col}

\end{cols}

\end{exercise}

\begin{exercise}
\protect\hypertarget{exr:MoreAboutTestsInterpretingHypotheses2}{}\label{exr:MoreAboutTestsInterpretingHypotheses2}

Consider the study to determine if a die was loaded (Chap.~\ref{TestOneProportion}) by studying the proportion of rolls that showed a \largedice{1}, and where \(\hat{p} = 0.41\). Explain \emph{why} each of these sets of hypotheses are incorrect.

\begin{cols}

\begin{col}{0.4\textwidth}

\begin{enumerate}
\def\labelenumi{\arabic{enumi}.}
\tightlist
\item
  \(H_0\): \(\hat{p} = 1/6\); \(H_1\): \(\hat{p} \ne 1/6\).
\item
  \(H_0 = 1/6\); \(H_1 \ne 1/6\).
\item
  \(H_0\): \(p = 1/6\); \(H_1\): \(\hat{p} = 0.41\).
\end{enumerate}

\end{col}

\begin{col}{0.05\textwidth}
~

\end{col}

\begin{col}{0.5\textwidth}

\begin{enumerate}
\def\labelenumi{\arabic{enumi}.}
\setcounter{enumi}{3}
\tightlist
\item
  \(H_0\): \(\hat{p} = 1/6\); \(H_1\): \(\hat{p} = 0.41\).
\item
  \(H_0\): \(p = 1/6\); \(H_1\): \(p > 1/6\).
\item
  \(H_0\): \(p = 1/6\); \(H_1\): \(p = 0.41\).
\end{enumerate}

\end{col}

\end{cols}

\end{exercise}

\begin{exercise}
\protect\hypertarget{exr:MoreAboutTestsConclusions}{}\label{exr:MoreAboutTestsConclusions}

The recommended daily energy intake for women is~\(7\,725\)~kJ (for a particular cohort, in a particular country; \citet{data:Altman1991:PracticalStats}). The daily energy intake for \(11\)~women was measured to see if this is being adhered to. The RQ was `Is the population mean daily energy intake~\(7\,725\)~kJ?'

The test produced \(P = 0.018\). What, if anything, is wrong with these conclusions after completing the hypothesis test?

\begin{enumerate}
\def\labelenumi{\arabic{enumi}.}
\tightlist
\item
  There is moderate evidence (\(P = 0.018\)) that the energy intake is not meeting the recommended daily energy intake.
\item
  There is moderate evidence (\(P = 0.018\)) that the sample mean energy intake is not meeting the recommended daily energy intake.
\item
  There is moderate evidence (\(P = 0.018\)) that the population energy intake is not meeting the recommended daily energy intake.
\item
  The study proves that the population energy intake is not meeting the recommended daily energy intake (\(P = 0.018\)).
\item
  There is some evidence that the population energy intake is not meeting the recommended daily energy intake (\(P < 0.018\)).
\end{enumerate}

\end{exercise}

\begin{exercise}
\protect\hypertarget{exr:MoreAboutTestsBatteries}{}\label{exr:MoreAboutTestsBatteries}

{[}\emph{Dataset}: \texttt{Battery}{]} A study compared ALDI batteries to another brand of battery. In one test (comparing the time taken for \(1.5\,\text{V}\) AA batteries to reach \(1.1\,\text{V}\)), the ALDI brand battery took~\(5.73\,\text{h}\), and the other brand (Energizer) took~\(5.44\,\text{h}\)~\citep{mypapers:Dunn:BatteryData}.

\begin{enumerate}
\def\labelenumi{\arabic{enumi}.}
\tightlist
\item
  What is the null hypothesis for the test?
\item
  The \(P\)-value for comparing these two means is about \(P = 0.70\). What does this mean?
\item
  Is this difference likely to be of any practical importance? Explain.
\item
  What would be a correct conclusion for ALDI to report from the study? Explain.
\item
  What else would be useful to know when comparing the two brands of batteries?
\end{enumerate}

\end{exercise}

\begin{exercise}
\protect\hypertarget{exr:MoreAboutTestsConsistency}{}\label{exr:MoreAboutTestsConsistency}An ecologist was compared the proportion of female and male dingoes kept in zoos that showed signs of mange (a skin disease). She finds `no statistically significant' difference between the proportions of female and male dingoes with evidence of mange.

Which of these statements is \emph{consistent} with this conclusion?

\begin{enumerate}
\def\labelenumi{\arabic{enumi}.}
\tightlist
\item
  The difference in proportions is \(0.27\) and \(P = 0.36\). \tightlist 
\item
  The difference in proportions is \(0.27\) and \(P = 0.0001\).
\item
  The difference in proportions is \(0.04\) and \(P = 0.36\).
\item
  The difference in proportions is \(0.04\) and \(P = 0.0001\).
\end{enumerate}

How would the other statements be interpreted then?
\end{exercise}

\begin{exercise}
\protect\hypertarget{exr:MoreAboutTestsConsistency2}{}\label{exr:MoreAboutTestsConsistency2}The study of body temperatures (Chap.~\ref{TestOneMean}) also compared the mean internal body temperatures for females and males \citep{data:mackowiak:bodytemp}. The study concludes that there is moderate evidence of a difference between the mean temperatures of females and males.

Which of these statements is \emph{consistent} with this conclusion?

\begin{enumerate}
\def\labelenumi{\arabic{enumi}.}
\tightlist
\item
  The difference between the mean temperatures is~\(0.289\)\textsuperscript{o}C and \(P = 0.024\). \tightlist 
\item
  The difference between the mean temperatures is~\(2.89\)\textsuperscript{o}C and \(P = 0.024\). \tightlist 
\item
  The difference between the mean temperatures is~\(0.289\)\textsuperscript{o}C and \(P = 0.39\). \tightlist 
\item
  The difference between the mean temperatures is~\(2.89\)\textsuperscript{o}C and \(P = 0.39\). \tightlist 
\end{enumerate}

How would the other statements be interpreted then?
\end{exercise}

\captionsetup{font=normalsize}

\begin{EOCanswerBox}{iconmonstr-check-mark-14-240.png}
\textbf{Answers to \textit{Quick review} questions:} \textbf{1.} False. \textbf{2.} False. \textbf{3.} True. \textbf{4.} True. \textbf{5.} True. \textbf{6.} False. \textbf{7.} c.~No parameter is given; perhaps \(H_0\): \(\mu = 37\).

\end{EOCanswerBox}

\chapter{Mean differences (paired data): CIs and tests}\label{AnalysisPaired}

\index{Research question!relational}\index{Mean difference}

\begin{cols}
\begin{col}{0.52\textwidth}

\begin{objectivesBox}{iconmonstr-target-4-240.png}
You have learnt to ask an RQ, design a study, classify and summarise the data, construct confidence intervals, and conduct hypothesis tests.
\textbf{In this chapter}, you will learn to:
\begin{itemize}\tightlist
  \item
  identify situations where mean differences are appropriate.
  \item
  construct confidence intervals for a mean difference.
  \item
  conduct hypothesis tests for the mean difference with paired data.
  \item
  determine whether the conditions for using these methods apply in a given situation.
\end{itemize}
\end{objectivesBox}

\end{col}

\begin{col}{0.03\textwidth}
~
\end{col}

\begin{col}{0.45\textwidth}

\includegraphics[width=0.95\linewidth]{29-CIsTesting-MeanDifference_files/figure-latex/unnamed-chunk-11-1} 
\end{col}
\end{cols}

\section{Introduction: six-minute walk test}\label{PairedIntro}

The six-minute walk test (6MWT) measures how far subjects can walk in six minutes, and is used as a simple, low-cost evaluation of fitness and other health-related measures. The recommended setting for the test is usually a walkway of at least~\(30\,\text{m}\). \citet{saiphoklang2022comparison} measured the 6MWT distance when the same subjects used \emph{both}~\(20\,\text{m}\) and~\(30\,\text{m}\)~walkways.

The comparison is \emph{within} individuals (Sect.~\ref{RQsRepeatedMeasures});\index{Comparison!within individuals} this is a \emph{repeated-measures} study. Each subject has a \emph{pair} of 6MWT measurements, and the study produced \emph{paired data} (Table~\ref{tab:Data6MWT}), the topic of this chapter.\index{Data!paired}\index{Study types!paired}

\begin{table} \centering \centering\caption{\label{tab:Data6MWT}The six-minute walk test (6MWT) distance, for walkways of $20\,\text{m}$\ and $30\,\text{m}$\ length. These are the first five and the last five of the $50$ total observations. (A negative difference means the $20\,\text{m}$\ distance is greater than the $30\,\text{m}$\ distance.)}

\fontsize{8}{10}\selectfont
\begin{tabular}[t]{cccc}
\toprule
\multicolumn{1}{c}{\textbf{ }} & \multicolumn{3}{c}{\textbf{Distance walked (in m)}} \\
\cmidrule(l{3pt}r{3pt}){2-4}
\textbf{Person} & \textbf{$20\,\text{m}$\ w'way} & \textbf{$30\,\text{m}$\ w'way} & \textbf{Diff.}\\
\midrule
$1$ & $272.1$ & $281.6$ & $\phantom{0}\phantom{0}9.5$\\
$2$ & $425.3$ & $454.4$ & $\phantom{0}29.0$\\
$3$ & $338.2$ & $330.0$ & $\phantom{0}\phantom{0}\llap{$-{}$}8.2$\\
$4$ & $240.0$ & $271.0$ & $\phantom{0}31.0$\\
$5$ & $518.3$ & $555.3$ & $\phantom{0}37.0$\\
$\vdots$ & $\vdots$ & $\vdots$ & $\vdots$\\
\bottomrule
\end{tabular} \quad 
\begin{tabular}[t]{cccc}
\toprule
\multicolumn{1}{c}{\textbf{ }} & \multicolumn{3}{c}{\textbf{Distance walked (in m)}} \\
\cmidrule(l{3pt}r{3pt}){2-4}
\textbf{Person} & \textbf{20\,\text{m}\ w'way} & \textbf{30\,\text{m}\ w'way} & \textbf{Diff.}\\
\midrule
$\vdots$ & $\vdots$ & $\vdots$ & $\vdots$\\
$46$ & $245.2$ & $245.2$ & $53.2$\\
$47$ & $184.4$ & $184.4$ & $34.0$\\
$48$ & $400.0$ & $400.0$ & $13.9$\\
$49$ & $344.8$ & $344.8$ & $39.4$\\
$50$ & $285.3$ & $285.3$ & $12.6$\\
\bottomrule
\end{tabular}
\end{table}

\begin{importantBox}{iconmonstr-warning-8-240.png}
Some differences are \emph{negative}. This does \emph{not} mean a negative distance. Since the differences are computed as the \(30\,\text{m}\)~distance minus the \(20\,\text{m}\)~distance, a negative difference means the \(20\,\text{m}\)~distance is a larger value than the \(30\,\text{m}\)~distance.

\end{importantBox}

\section{Paired data}\label{PairedData}

\index{Data!paired}

The data in Table~\ref{tab:SoilCN} are \emph{paired}.\index{Data!paired} Computing the \emph{differences} or \emph{changes} between the pairs of observations makes sense, since the values for each pair belong to the same unit of analysis (the same person, in this case).

Pairing data, when appropriate, is useful because individuals can vary substantially, and pairing means that extraneous variables\index{Variables!extraneous} (potentially, \emph{confounding} variables)\index{Variables!confounding} are held constant for those paired observations. For example, each pair of measurements in Table~\ref{tab:SoilCN} are recorded for the same person, so both measurements are recorded for someone of the same age, same sex, and with the same physical attributes.

Pairing is a form of blocking\index{Blocking} (Sect.~\ref{ManagingConfounding}). Pairing is a good design strategy when the individuals in the pair are the same, or are very similar, for many extraneous variables. (For example, the pair may comprise two different people, of the same sex, with similar age, height and weight.) Pairing often involves taking two measurements from the \emph{same} individuals, as in Table~\ref{tab:SoilCN}.

\begin{definition}[Paired data]
\protect\hypertarget{def:PairedData}{}\label{def:PairedData}\emph{Paired data} occurs when the outcome is compared for two different, distinct situations for each unit of analysis.
\end{definition}

Paired studies appear in many situations; for example, when:

\begin{itemize}
\tightlist
\item
  heart rate is measured for each twin in a pair (the twin-pair is the `individual'), one of whom exercises regularly and one who does not. Pairing the twins is reasonable, given the shared genetics (and probably childhood environments also). The \emph{difference} between the hearts rates of the twins can be recorded for each pair.
\item
  the body temperature of dogs (the `individuals') is measured using \emph{both} rectal and ear thermometers for each dog. The \emph{difference} between the two recorded temperatures from the thermometers for each dog is recorded.
\item
  blood pressure is recorded from some individuals (Group~A) after receiving Drug~A, and from another group of individuals (Group~B) after receiving Drug~B. Each person in Group~A is matched with someone in Group~B of the same sex, similar age and similar weight (e.g., in one of the pairs, both individuals are male, about \(30\)~years-of-age, and weighing about \(95\,\text{kg}\)). The \emph{difference} between the blood pressure measurements for the individual in Group~A and the matched person in Group~B is recorded for each pair.
\item
  the number of campers is recorded at many national parks (the `individuals') on the first weekend in summer, and on the first weekend in winter. The \emph{difference} in camper numbers for each national park between these time points is recorded.
\end{itemize}

Many of these examples can be extended to beyond two measurements. For instance, temperatures can be compared on each dog using three different types of thermometers. In this chapter, only \emph{pairs} of measurements are studied, and only for quantitative variables.

\section{Summarising the data}\label{SummarisingPairedCI}

For the 6MWT study, the distance is measured for the same subjects for two different walkway distances. Each subject receives two measurements, and the \emph{difference} between the distances walked for each individual is computed.\index{Mean difference}

Since the data are paired, an appropriate graph is a histogram of the differences (Sect.~\ref{HistoDiffPlot}); specifically, \(30\,\text{m}\)~distance minus the \(20\,\text{m}\)~distance. A boxplot comparing 6MWT distance for both walkway lengths (that is, \emph{not} pairing the data) shows the distribution of distances, and the median distances, are very similar (Fig.~\ref{fig:ComparePairedBoxplotsHistogram}, left panel). Any difference in individuals' 6MWT distances is difficult to see and detect. In addition, linking the \(20\,\text{m}\) and \(30\,\text{m}\) distances that belong together for each individual patient is not possible.

However, using a histogram of the differences\index{Graphs!histogram of differences} makes the individuals' differences easier to see (Fig.~\ref{fig:ComparePairedBoxplotsHistogram}, right panel). The histogram also makes it easy to see that some subjects walked further with a \(20\,\text{m}\)~walkway, and some further for a \(30\,\text{m}\)~walkway. Individually graphing the distances for both walkway distances may also be useful too (e.g., using two histograms), but a graph of the differences is \emph{crucial}, as the RQ is about those differences. A case-profile plot (Sect.~\ref{CaseProfilePlot}) is also appropriate, but is difficult to read for these data because sample size is large (a line is needed for each of the \(50\)~units of analysis).



\begin{figure}[hbtp]

{\centering \includegraphics[width=0.95\linewidth]{29-CIsTesting-MeanDifference_files/figure-latex/ComparePairedBoxplotsHistogram-1} 

}

\caption{Plots of the 6MWT data. Left: graphing the data \emph{incorrectly} as unpaired. Right: a histogram of 6MWT distances changes (\(30\,\text{m}\)~walkway distance \emph{minus} \(20\,\text{m}\)~walkway distance; the vertical grey line represents no change in distance).}\label{fig:ComparePairedBoxplotsHistogram}
\end{figure}

The 6MWT distances for each walkway length can be summarised individually (the first two rows of Table~\ref{tab:SMWTSummary}) using the methods of Chap.~\ref{OneMeanConfInterval}, using software (Fig.~\ref{fig:SMWTNumericalOutput}). All statistics are slightly different for the two walkway distances; in particular, the mean~\(30\,\text{m}\) walkway distance is slightly larger. However, since the RQ is about the difference between the distances, a numerical summary of the \emph{differences} is essential (third row of Table~\ref{tab:SMWTSummary}, based on Fig.~\ref{fig:SMWTNumericalOutput}). Notice that the third row of information is computed from the values in the \textbf{Diff.} column in Table~\ref{tab:SoilCN}, not by (for instance) finding the difference between the standard deviations in the first two rows.

\begin{table}
\centering
\caption{\label{tab:SMWTSummary}The numerical summary of the 6MWT data. (The differences are the $30\,\text{m}$\ distances minus the $20\,\text{m}$\ differences.)}
\centering
\fontsize{8}{10}\selectfont
\begin{tabular}[t]{lccccc}
\toprule
\multicolumn{1}{c}{\textbf{ }} & \multicolumn{1}{c}{\textbf{ }} & \multicolumn{1}{c}{\textbf{ }} & \multicolumn{1}{c}{\textbf{Standard}} & \multicolumn{1}{c}{\textbf{Standard}} & \multicolumn{1}{c}{\textbf{Sample}} \\
\textbf{ } & \textbf{Mean} & \textbf{Median} & \textbf{deviation} & \textbf{error} & \textbf{size}\\
\midrule
20m walkway distance (in m) & $337.82$ & $351.0$ & $71.801$ & $10.154$ & $50$\\
30m walkway distance (in m) & $359.85$ & $371.4$ & $77.250$ & $10.925$ & $50$\\
\midrule
\em{Difference (in m)} & \em{$\phantom{0}22.03$} & \em{$\phantom{0}17.0$} & \em{$22.039$} & \em{$\phantom{0}3.117$} & \em{$50$}\\
\bottomrule
\end{tabular}
\end{table}

\begin{figure}[hbtp]

{\centering \includegraphics[width=0.64\linewidth]{jamovi/SMWT/SMWT-NumericalSummary} \includegraphics[width=1\linewidth]{jamovi/SMWT/SMWT-PairedT} 

}

\caption{The 6MWT data: numerical summary software output for each group (top), and the CI and test results (bottom).}\label{fig:SMWTNumericalOutput}
\end{figure}

The differences (i.e., the \textbf{Diff.}~column in Table~\ref{tab:Data6MWT}) can be treated like a single sample of data (Table~\ref{tab:PairedNotation}), with the notation adapted accordingly:

\begin{itemize}
\tightlist
\item
  \(\mu_d\): the mean \emph{difference} in the \emph{population} (in m).
\item
  \(\bar{d}\): the mean \emph{difference} in the \emph{sample} (in m).
\item
  \(s_d\): the \emph{sample} standard deviation of the \emph{differences} (in m).
\item
  \(n\): the number of \emph{differences}.
\end{itemize}

\begin{table}
\centering
\caption{\label{tab:PairedNotation}The notation used for mean differences (paired data) compared to the notation used for one sample mean.}
\centering
\fontsize{8}{10}\selectfont
\begin{tabular}[t]{lcc}
\toprule
\textbf{ } & \textbf{One sample mean} & \textbf{Mean difference}\\
\midrule
The observations: & Values: $x$ & Differences: $d$\\
\addlinespace
Population mean: & $\mu$ & $\mu_d$\\
\addlinespace
Sample mean: & $\bar{x}$ & $\bar{d}$\\
\addlinespace
Standard deviation: & $s$ & $s_d$\\
\addlinespace
Standard error of $\bar{x}$: & $\displaystyle\text{s.e.}(\bar{x}) = \frac{s}{\sqrt{n}}$ & $\displaystyle\text{s.e.}(\bar{d}) = \frac{s_d}{\sqrt{n}}$\\
\addlinespace
Sample size: & Number of \emph{observations}: $n$ & Number of \emph{differences}: $n$\\
\bottomrule
\end{tabular}
\end{table}

\section{\texorpdfstring{Confidence intervals for \(\mu_d\)}{Confidence intervals for \textbackslash mu\_d}}\label{MeanDiffCI}

\index{Sampling distribution!paired quantitative data}\index{Confidence intervals!paired quantitative data|(}\index{Sampling distribution!mean difference}\index{Confidence intervals!mean difference|(}

The data in Table~\ref{tab:Data6MWT} can be used to answer this repeated-measures, estimation RQ:

\begin{quote}
For Thai patients with chronic obstructive pulmonary disease, what is the mean difference between the 6MWT distance when subjects use a~\(20\,\text{m}\) walkway and a~\(30\,\text{m}\) walkway?
\end{quote}

Every possible sample of \(n = 50\) subjects comprises different people, and hence produces different 6MWT distances for~\(20\,\text{m}\) and~\(30\,\text{m}\) walkways. For this reason, the 6MWT distance summaries in Table~\ref{tab:SMWTSummary} include standard errors. Since the 6MWT distance varies from sample to sample for each person, the \emph{differences} between the distances for each person varies from sample to sample too, and also have a \emph{sampling distribution}.

\begin{definition}[Sampling distribution of a sample mean difference]
\protect\hypertarget{def:DEFSamplingDistributionDbar}{}\label{def:DEFSamplingDistributionDbar}The \emph{sampling distribution of a sample mean difference} is (when certain conditions are met; Sect.~\ref{ValiditySampleMeanDiff}) described by:

\begin{itemize}
\tightlist
\item
  an approximate normal distribution,
\item
  centred around the \emph{sampling mean} whose value is the population mean \emph{difference}~\(\mu_d\),
\item
  with a standard deviation, called the standard error of the difference, of \(\displaystyle\text{s.e.}(\bar{d}) = \frac{s_d}{\sqrt{n}}\),
\end{itemize}

where~\(n\) is the number of differences, and~\(s_d\) is the standard deviation of the individual differences in the sample.
\end{definition}

For the 6MWT data, the sample mean differences~\(\bar{d}\) are described by (Fig.~\ref{fig:SMWTSamplingDist}):

\begin{itemize}
\tightlist
\item
  approximate normal distribution,
\item
  with a sampling mean whose value is~\(\mu_{{d}}\),
\item
  with a \emph{standard error} of \begin{equation}
  \text{s.e.}(\bar{d}) =  \frac{22.039}{\sqrt{50}} = 3.117.
  \label{eq:StdErrorDifferences}
  \end{equation}
\end{itemize}

\begin{figure}[hbtp]

{\centering \includegraphics[width=1\linewidth]{29-CIsTesting-MeanDifference_files/figure-latex/SMWTSamplingDist-1} 

}

\caption{The sampling distribution is a normal distribution; it describes how the sample mean difference between the 6MWT distances varies in samples of size $n = 50$.}\label{fig:SMWTSamplingDist}
\end{figure}

The CI for the mean difference has the same form as for a single mean (Chap.~\ref{OneMeanConfInterval}). The \(95\)\%~confidence interval (CI) for~\(\mu_d\) is \[
  \bar{d} \pm \big(\text{multiplier} \times\text{s.e.}(\bar{d})\big).
\] As usual when the sampling distribution has an approximate normal distribution, an approximate \(95\)\%~CI uses the approximate multiplier of~\(2\) (from the \(68\)--\(95\)--\(99.7\) rule). This is the same as the CI for~\(\bar{x}\) if the differences are treated as the data.

For the 6MWT data, the approximate \(95\)\% CI is: \[
  22.03 \pm (2 \times 3.117),
\] or \(22.03\pm 6.234\,\text{m}\) (so the \emph{margin of error} is~\(6.234\,\text{m}\)). Equivalently, the CI is from \(22.03 - 6.234 = 15.796\,\text{m}\), up to \(22.03 + 6.234 = 28.264\,\text{m}\). We write:

\begin{quote}
The mean difference in the 6MWT distances when using a~\(20\,\text{m}\) and~\(30\,\text{m}\) walkway is~\(22.03\,\text{m}\) (\(\text{s.e.} = 3.117\); \(n = 50\)), with an approximate \(95\)\%~CI from~\(15.80\,\text{m}\) to~\(28.26\,\text{m}\), further for a \(30\,\text{m}\)~walkway.
\end{quote}

The CI means that the reasonable values for the population mean difference in 6MTW distances are between~\(15.80\,\text{m}\) and~\(28.26\,\text{m}\). Alternatively, we are \(95\)\%~confident that the population mean difference between the 6MWT distances is between~\(15.80\,\text{m}\) and~\(28.26\,\text{m}\) (further for \(30\,\text{m}\)~walkway). A difference of this magnitude probably has practical importance.\index{Practical importance} Also notice that the \emph{direction} of the difference is given: `further for \(30\,\text{m}\)~walkway'.

Statistical software\index{Software output!mean differences} produces \emph{exact} \(95\)\%~CIs, which may be slightly different from the \emph{approximate} \(95\)\%~CI (recall: the \(68\)--\(95\)--\(99.7\) rule gives \emph{approximate} multipliers). For the 6MWT data, the \emph{approximate} and \emph{exact} \(95\)\%~CIs are the same to one decimal place (Fig.~\ref{fig:SMWTNumericalOutput}). We write:

\begin{quote}
The mean difference in the 6MWT distances when using a~\(20\,\text{m}\) and~\(30\,\text{m}\) walkway is \(22.03\,\text{m}\) (\(\text{s.e.} = 3.117\); \(n = 50\)), with a \(95\)\%~CI from~\(15.76\,\text{m}\) to~\(28.29\,\text{m}\) further for a \(30\,\text{m}\)~walkway. \index{Confidence intervals!paired quantitative data|)} \index{Confidence intervals!mean difference|)}
\end{quote}

\section{\texorpdfstring{Hypothesis tests for \(\mu_d\): \(t\)-test}{Hypothesis tests for \textbackslash mu\_d: t-test}}\label{MeanDiffTest}

\index{Sampling distribution!paired quantitative data}\index{Hypothesis testing!paired quantitative data|(}\index{Sampling distribution!mean difference}\index{Hypothesis testing!mean difference|(}

The data in Table~\ref{tab:Data6MWT} can be used to answer this repeated-measures, decision-making RQ:

\begin{quote}
For Thai patients with chronic obstructive pulmonary disease, is there a mean increase in 6MWT distance using a~\(30\,\text{m}\) walkway compared to a \(20\,\text{m}\) walkway?
\end{quote}

In Sect.~\ref{PairedIntro}, the differences were defined as the \(30\,\text{m}\)~distance minus the \(20\,\text{m}\)~distance, which is consistent with the wording in this RQ. This RQ asks if the mean walking distance is, in general, a smaller value when subjects use a~\(20\,\text{m}\) walkway compared to a~\(30\,\text{m}\) walkway (that is how \emph{positive} differences eventuate). The \emph{parameter} is the \emph{population mean difference} in 6MWT, \(\mu_d\). Note that the RQ is worded as one-tailed.

The \emph{null} hypothesis is that `there is \emph{no mean change} in 6MWT, in the population':

\begin{itemize}
\tightlist
\item
  \(H_0\): \(\mu_d = 0\).
\end{itemize}

This hypothesis, which we initially \emph{assume} to be true, postulates that the mean reduction may not be zero in the sample, due to sampling variation.

Since the RQ asks specifically if the mean distance is \emph{smaller} for a \(20\,\text{m}\) walkway, the alternative hypothesis is \emph{one-tailed} (Sect.~\ref{AboutHypotheses}). According to how the differences have been defined, the alternative hypothesis is:

\begin{itemize}
\tightlist
\item
  \(H_1\): \(\mu_d > 0\) (i.e., one-tailed).
\end{itemize}

This hypothesis says that the mean change in the population is \emph{greater than} zero, because of the wording of the RQ, and because of how the differences were defined. If the differences were defined in the opposite way (as `the \(20\,\text{m}\)~distance minus the \(30\,\text{m}\)~distance') then the alternative hypothesis would be \(\mu_d < 0\), which has the same \emph{meaning}.

The sampling distribution, as described in Sect.~\ref{def:DEFSamplingDistributionDbar}, still applies, where \(\mu_d\) is assumed to be the value given in \(H_0\) (see Fig.~\ref{fig:SMWTSamplingDistHT}):

\begin{itemize}
\tightlist
\item
  an approximate normal distribution,
\item
  centred around the \emph{sampling mean} whose value is the population mean \emph{difference}~\(\mu_d = 0\) (from \(H_0\)),
\item
  with a standard deviation of \(\displaystyle\text{s.e.}(\bar{d}) = 3.117\) (from Equation~\eqref{eq:StdErrorDifferences}).
\end{itemize}

The sample mean difference can be located on the sampling distribution by computing the \(t\)-score:\index{Hypothesis testing!mean difference}\index{Test statistic!t@$t$-score} \[
t
= \frac{\bar{d} - \mu_{d}}{\text{s.e.}(\bar{d})}
= \frac{22.026 - 0}{3.117} = 7.07,
\] following the ideas in Equation~\eqref{eq:tscore}. Software displays the same \(t\)-score (Fig.~\ref{fig:SMWTNumericalOutput}). This is a \emph{huge} \(t\)-score.

\begin{figure}[hbtp]

{\centering \includegraphics[width=1\linewidth]{29-CIsTesting-MeanDifference_files/figure-latex/SMWTSamplingDistHT-1} 

}

\caption{The sampling distribution is a normal distribution; it describes how the sample mean difference between the 6MWT distances varies in samples of size $n = 50$.}\label{fig:SMWTSamplingDistHT}
\end{figure}

A \(P\)-value determines if the sample data are consistent with the assumption (Table~\ref{tab:PvaluesInterpretation}). Since \(t = 7.07\), and since \(t\)-scores are like \(z\)-scores, the \emph{one}-tailed \(P\)-value will be very small (based on the \(68\)--\(95\)--\(99.7\) rule).\index{68@$68$--$95$--$99.7$ rule} Software (Fig.~\ref{fig:SMWTNumericalOutput}) reports that the \emph{two}-tailed \(P\)-value is less than~\(0.0001\). Hence, the \emph{one}-tailed \(P\)-value is less than \(0.0001/2 = 0.00005\).

\begin{importantBox}{iconmonstr-warning-8-240.png}
The software clarifies \emph{how} the differences have been computed. At the left of the output (Fig.~\ref{fig:SMWTNumericalOutput}), the order implies the differences are found as \texttt{Dist30} (the \(30\,\text{m}\) walk distance) minus \texttt{Dist20} (the \(20\,\text{m}\) walk distance), the same as our definition.

\end{importantBox}

The one-tailed \(P\)-value is less than~\(0.00005\), suggesting very strong evidence (Table~\ref{tab:PvaluesInterpretation}) to support \(H_1\). A conclusion requires an \emph{answer to the RQ}, a summary of the \emph{evidence} leading to that conclusion, and some \emph{summary statistics}:

\begin{quote}
Very strong evidence exists in the sample (paired \(t = 7.07\); one-tailed \(P < 0.0005\)) of a mean reduction in 6MWT for a \(20\,\text{m}\) walkway compared to a \(30\,\text{m}\) walkway (mean reduction: \(22.03\,\text{m}\); \(95\)\% CI: \(15.76\,\text{m}\) to~\(28.29\,\text{m}\); \(n = 50\)).
\end{quote}

Note that the direction of the difference is provided.

\begin{importantBox}{iconmonstr-warning-8-240.png}
Saying `there is evidence of a difference' is insufficient. You must state \emph{which} measurement is, on average, higher (that is, what the differences \emph{mean}).

\end{importantBox}

\index{Hypothesis testing!paired quantitative data|)} \index{Hypothesis testing!mean difference|)}

\section{Statistical validity conditions}\label{ValiditySampleMeanDiff}

\index{Sampling distribution!mean difference}\index{Statistical validity (for inference)!mean differences}\index{Sampling distribution!paired quantitative data}

As with any CI and hypothesis test, these results apply under certain conditions. The conditions under which the results are statistically valid for paired data are similar to those for one sample mean, rephrased for differences.

The CI and test for a mean difference is \emph{statistically valid} if \emph{either} of these is true:

\begin{itemize}
\tightlist
\item
  when \(n \ge 25\). (If the distribution of the differences is highly skewed, the sample size may need to be larger.)
\item
  when \(n < 25\), \emph{and} the sample data come from a \emph{population} with a normal distribution.
\end{itemize}

The sample size of~\(25\) is a rough figure; some books give other values (such as~\(30\)).

This condition ensures that the \emph{distribution of the sample mean differences has an approximate normal distribution} (so that, for example, the \(68\)--\(95\)--\(99.7\) rule can be used). Provided the sample size is larger than about~\(25\), this will be approximately true \emph{even if} the distribution of the differences in the population does not have a normal distribution. That is, when \(n \ge 25\) the sample mean differences generally have an approximate normal distribution, even if the differences themselves don't have a normal distribution. The units of analysis are also assumed to be \emph{independent} (e.g., from a simple random sample).

If the statistical validity conditions are not met, other methods (e.g., non-parametric methods\index{Non-parametric statistics} \citep{conover2003practical}; resampling methods\index{Resampling methods} \citep{efron2021computer}) may be used. For paired qualitative data, McNemar's test can be used \citep{conover2003practical}.

\begin{example}[Statistical validity]
\protect\hypertarget{exm:StatisticalValidityWeightGain}{}\label{exm:StatisticalValidityWeightGain}For the 6MWT data, the sample size is \(n = 50\), so the results are statistically valid. Neither the differences \emph{in the population}, nor the distances \emph{in the population} for the individual walkway lengths, need to follow a normal distribution.
\end{example}

\section{Example: invasive plants}\label{PairedInvasivePlants}

Skypilot is an alpine wildflower native to the Colorado Rocky Mountains (USA). In recent years, a willow shrub (\emph{Salix}) has been encroaching on skypilot territory and, because willow often flowers early, \citet{kettenbach2017shrub} studied whether the willow may `negatively affect pollination regimes of resident alpine wildflower species' (p.~\(6\,965\)).

Data for both species was collected at \(n = 25\) different sites, so the data are \emph{paired} by site (Sect.~\ref{PairedIntro}).\index{Data!paired} The data are shown in Table~\ref{tab:FloweringData}. The parameter is~\(\mu_d\), the population mean \emph{difference} in day of first flowering for skypilot, less the day of first flowering for willow. A \emph{positive} value for the difference means that the skypilot values are larger, and hence that willow flowered first. The RQ is:

\begin{quote}
In the Colorado Rocky Mountains, is there a mean difference between first-flowering day for the native skypilot and encroaching willow?
\end{quote}

The hypotheses are \[
\text{$H_0$: $\mu_d = 0$}\quad\text{and}\quad\text{$H_1$: $\mu_d\ne 0$},
\] where the alternative hypothesis is two-tailed, and \(\mu_d\) is the mean difference between first-flowering day for the native skypilot and encroaching willow.

\begin{tipBox}{iconmonstr-info-6-240.png}
Explaining \emph{how} the differences are computed is important. The differences here are skypilot minus willow first-flowering days.

However, the differences could be computed as willow minus skypilot first-flowering days. \emph{Either is fine}, as long as you remain consistent. The \emph{meaning} of any conclusions will be the same.

\end{tipBox}

The data are summarised graphically in Fig.~\ref{fig:FloweringPlots} and numerically (Table~\ref{tab:FloweringSummaryHT}, after rounding) using software output (Fig.~\ref{fig:FloweringjamoviHT}).

\begin{figure}[hbtp]

{\centering \includegraphics[width=1\linewidth]{jamovi/Flowering/FloweringAll} 

}

\caption{Software output for the flowering-day data.}\label{fig:FloweringjamoviHT}
\end{figure}

\begin{table}
\centering
\caption{\label{tab:FloweringSummaryHT}The day of first flowering for encroaching willow and native skypilot.}
\centering
\fontsize{8}{10}\selectfont
\begin{tabular}[t]{lcccc}
\toprule
\textbf{ } & \textbf{Mean} & \textbf{Standard deviation} & \textbf{Standard error} & \textbf{Sample size}\\
\midrule
Willow (encroaching) & $189.40$ & $12.200$ & $2.440$ & $25$\\
Skypilot (native) & $190.76$ & $13.062$ & $2.612$ & $25$\\
\midrule
\em{Differences} & \em{$\phantom{0}\phantom{0}1.36$} & \em{$\phantom{0}4.698$} & \em{$0.940$} & \em{$25$}\\
\bottomrule
\end{tabular}
\end{table}

The standard error of the mean difference is \(\text{s.e.}(\bar{d}) = 0.9396\) (Fig.~\ref{fig:FloweringjamoviHT} or Table~\ref{tab:FloweringSummaryHT}). The sampling distribution for \(\bar{d}\) has a normal distribution, centred around \(\mu_d\) with a standard deviation of \(\text{s.e.}(\bar{d}) = 0.9396\).

The approximate \(95\)\%~CI for the mean difference is \[
1.36 \pm ( 2\times 0.9396),
\] or from \(-0.52\) to~\(3.24\)~days. The exact \(95\)\% CI (Fig.~\ref{fig:FloweringjamoviHT}) is \(-0.58\) to~\(3.30\)~days; the difference is because the approximate CI uses the \emph{approximate} multiplier of~\(2\) from the \(68\)--\(95\)--\(99.7\) rule.

The value of the test statistic (i.e., the \(t\)-score) is \begin{align*}
t 
= \frac{\bar{d} - \mu_d}{\text{s.e.}(\bar{d})}
= \frac{1.36 - 0}{0.9396} = 1.45,
\end{align*} as in the output. This is a relatively small value of~\(t\), so a large \(P\)-value is expected using the \(68\)--\(95\)--\(99.7\) rule. Indeed, the output shows that \(P = 0.161\): there is \emph{no evidence} of a mean difference in first-flowering day (i.e., the sample mean difference could reasonably be explained by sampling variation if \(\mu_d = 0\)).

Since \emph{positive} differences mean that willow flowers earlier, we write (using the exact CI):

\begin{quote}
No evidence exists (\(t = 1.45\); two-tailed \(P = 0.161\)) that the day of first-flowering is different for the encroaching willow and the native skypilot (mean difference: \(1.36\) days earlier for willow; approximate \(95\)\%~CI between \(0.52\)~days earlier for skypilot to \(3.24\)~days earlier for willow; \(n = 25\)).
\end{quote}

The CI is statistically valid since \(n = 25\).

\begin{importantBox}{iconmonstr-warning-8-240.png}
Be clear in your conclusion about \emph{how} the differences are computed. Make sure to interpret the test and CI consistently with how the differences are defined.

\end{importantBox}

\begin{importantBox}{iconmonstr-warning-8-240.png}
We \emph{do not} say whether the evidence supports the null hypothesis. We assume the null hypothesis is true, so we state how strong the evidence is to support the alternative hypothesis. The current sample presents no evidence to contradict the assumption (but future evidence may emerge).

\end{importantBox}

\section{Example: chamomile tea}\label{ChamomileTea-Paired}

\citet{rafraf2015effectiveness} studied patients with Type~2 diabetes mellitus (T2DM). They randomly allocated \(32\)~patients into a control group (who drank hot water), and another \(32\)~patients to receive chamomile tea (p.~164):\index{Control}\index{Blinding!researchers}

\begin{quote}
The study was blinded so that the allocation of the intervention or control group was concealed from the researchers and statistician {[}\ldots{]} The intervention group (\(n = 32\)) consumed one cup of chamomile tea {[}\ldots{]} three times a day immediately after meals (breakfast, lunch, and dinner) for \(8\)~weeks. The control group (\(n = 32\)) consumed an equivalent volume of warm water during the \(8\)-week period\ldots{}
\end{quote}

The total glucose (TG) was measured for each individual both \emph{before} the intervention and \emph{after} eight weeks on the intervention (a within-individuals comparison)\index{Comparison!within individuals}\}, in both the control and treatment groups (a between-individuals comparison)\index{Comparison!between individuals}\}. The data are not available, so no graphical summary of the data can be produced; however, the article gives a data summary (motivating Table~\ref{tab:TGsummaryTable}).



\begin{table}
\centering
\caption{\label{tab:TGsummaryTable}The total glucose (TG; in mg.dL\(^{-1}\)) for two groups: those who drank chamomile tea, and those who drank hot water (the control group). The \textbf{Reduction} columns summarise the reduction in TG for each group.}
\centering
\fontsize{8}{10}\selectfont
\begin{tabular}[t]{lcccccccc}
\toprule
\multicolumn{2}{c}{\textbf{ }} & \multicolumn{2}{c}{\textbf{Baseline}} & \multicolumn{2}{c}{\textbf{After 8 weeks}} & \multicolumn{3}{c}{\textbf{Reduction}} \\
\cmidrule(l{3pt}r{3pt}){3-4} \cmidrule(l{3pt}r{3pt}){5-6} \cmidrule(l{3pt}r{3pt}){7-9}
\textbf{ } & \textbf{$n$} & \textbf{Mean} & \textbf{Std dev.} & \textbf{Mean} & \textbf{Std dev.} & \textbf{Mean} & \textbf{Std dev.} & \textbf{Std error}\\
\midrule
Chamomile tea & $32$ & $203.00$ & $54.96$ & $164.37$ & $50.70$ & $38.62$ & $30.37$ & $\phantom{0}5.37$\\
Control & $32$ & $178.25$ & $53.06$ & $185.37$ & $52.59$ & $\phantom{0}\llap{$-{}$}7.12$ & $36.66$ & $\phantom{0}6.48$\\
\midrule
\em{Difference} & \em{} & \em{$\phantom{0}24.75$} & \em{} & \em{$\phantom{0}21.00$} & \em{} & \em{$45.74$} & \em{} & \em{}\\
\bottomrule
\end{tabular}
\end{table}

Is there a mean reduction in TG in either group? Estimates of the mean reduction in each group can be found by constructing a CI for each group. First, the standard errors for each reduction are needed:

\begin{itemize}
\tightlist
\item
  \makebox[31mm][l]{Tea-drinking group:} \(\text{s.e.}(\bar{d}) = 30.37/\sqrt{32} = 5.37\).
\item
  \makebox[31mm][l]{Control group:} \(\text{s.e.}(\bar{d}) = 36.66/\sqrt{32} = 6.48\).
\end{itemize}

Then the approximate \(95\)\% CIs are:

\begin{itemize}
\tightlist
\item
  \makebox[31mm][l]{Tea-drinking group:} \(38.62\pm (2\times 5.37)\), or from~\(27.88\) to~\(49.36\,\text{mg}.\text{dL}^{-1}\).
\item
  \makebox[31mm][l]{Control group:} \(-7.12\pm (2\times 6.48)\), or from~\(-20.08\) to~\(5.84\,\text{mg}.\text{dL}^{-1}\).
\end{itemize}

(A \emph{negative reduction} in TG means an \emph{increase} in TG.) The first CI suggests that the population mean difference is almost certainly larger than zero; the second suggests that a population mean difference of zero could reasonably have produced the sample data.

Of course, the sample mean differences in TG may be non-zero due to sampling variation. So, the following repeated-measures RQs can be asked:

\begin{quote}
\begin{itemize}
\tightlist
\item
  For patients with T2DM, is there a mean \emph{change} in TG after eight weeks drinking \emph{chamomile tea}?
\item
  For patients with T2DM, is there a mean \emph{change} in TG after eight weeks drinking \emph{hot water}?
\end{itemize}
\end{quote}

Then, the hypotheses are (where \(\mu_d\) represent the mean change in TG (in~mg.dL\textsuperscript{\(-1\)}) after eight weeks):

\begin{itemize}
\tightlist
\item
  \makebox[31mm][l]{Tea-drinking group:} \(H_0\):~\(\mu_d = 0\)\quad~vs~\(H_1\):~\(\mu_d \ne 0\).
\item
  \makebox[31mm][l]{Control group:} \(H_0\):~\(\mu_d = 0\)\quad~vs~\(H_1\):~\(\mu_d \ne 0\).
\end{itemize}

The two test statistics are: \[
  t_T = \frac{38.62 - 0}{5.37} = 7.19\qquad\text{and}\qquad t_W = \frac{-7.12 - 0}{6.48} = -1.10,
\] where the subscripts~\(T\) and~\(W\) refer to the tea and hot-water groups respectively. The \(t\)-score for the tea-drinking group is \emph{huge}, so the two-tailed \(P\)-value will be \emph{very small} using the \(68\)--\(95\)--\(99.7\) rule, and certainly smaller than~\(0.001\). This means that there is evidence that chamomile tea had an impact on the mean change in~TG.

In contrast, the \(t\)-score for the water-drinking group is \emph{small}, so the two-tailed \(P\)-value will be \emph{large} using the \(68\)--\(95\)--\(99.7\) rule, and certainly larger than~\(0.10\). This means there is no evidence that placebo\index{Placebo} treatment (hot water) had any impact on mean change in TG (as one might expect for a placebo).

We write:

\begin{quote}
There is very strong evidence (\(t = 7.19\); two-tailed \(P < 0.001\)) of a mean change in TG for the chamomile-drinking groups (mean reduction: \(38.62\,\text{mg}.\text{dL}^{-1}\); approx. \(95\)\%~CI: \(27.88\) to~\(49.36\,\text{mg}.\text{dL}^{-1}\); \(n = 32\)), but \emph{no} evidence (\(t = -1.10\); two-tailed \(P > 0.10\)) of a mean change in the hot-water drinking group (mean reduction: \(-7.12\,\text{mg}.\text{dL}^{-1}\); approx. \(95\)\%~CI: \(-20.08\) and~\(5.84\,\text{mg}.\text{dL}^{-1}\); \(n = 32\)).
\end{quote}

The intervals have a \(95\)\%~chance of straddling the population mean reduction in TG.\spacex The sample sizes are larger than~\(25\), so the results are statistically valid.

These hypothesis tests have allowed decisions to be made about each group individually. However, the two groups ultimately need to be \emph{compared}; this is considered in Sect.~\ref{ChamomileTea-TwoMeans}.

\section{Chapter summary}\label{Chap29-Summary}

To compute a confidence interval (CI) for a mean difference, compute the sample mean difference,~\(\bar{d}\), and identify the sample size~\(n\). Then compute the standard error, which quantifies how much the value of~\(\bar{d}\) varies across all possible samples: \[
\text{s.e.}(\bar{d})
=
\frac{ s_d }{\sqrt{n}},
\] where~\(s_d\) is the sample standard deviation. The \emph{margin of error} is (multiplier\({}\times{}\)standard error), where the multiplier is~\(2\) for an approximate \(95\)\%~CI (using the \(68\)--\(95\)--\(99.7\) rule). Then the CI is: \[
\bar{d} \pm \left( \text{multiplier}\times\text{standard error} \right).
\] The statistical validity conditions should also be checked.

These steps are used to test a hypothesis about a population mean difference \(\mu_d\).

\begin{itemize}
\tightlist
\item
  Write the null hypothesis (\(H_0\)) and the alternative hypothesis (\(H_1\)); initially \emph{assume} the value of~\(\mu_d\) in the null hypothesis to be true.
\item
  Describe the \emph{sampling distribution}, which describes what to \emph{expect} from the sample mean difference based on this assumption: under certain statistical validity conditions, the sample mean difference varies with:

  \begin{itemize}
  \tightlist
  \item
    an approximate normal distribution,
  \item
    with sampling mean whose value is the value of~\(\mu_d\) (from~\(H_0\)), and
  \item
    having a standard deviation of \(\displaystyle \text{s.e.}(\bar{d}) =\frac{s_d}{\sqrt{n}}\).
  \end{itemize}
\item
  Compute the value of the \emph{test statistic}: \[
  t = \frac{ \bar{d} - \mu}{\text{s.e.}(\bar{d})},
  \] where~\(\mu_d\) is the hypothesised value given in the null hypothesis.
\item
  The \(t\)-value is like a \(z\)-score, and so an approximate \emph{\(P\)-value} can be estimated using the \(68\)--\(95\)--\(99.7\) rule, or found using software. Use the \(P\)-value to make a decision, and write a conclusion.
\item
  Check the statistical validity conditions.
\end{itemize}

\section{Quick review questions}\label{Chap34-QuickReview}

\citet{bacho2019effects} compared joint pain in stroke patients receiving a supervised exercise treatment. The same participants (\(n = 34\)) were assessed \emph{before} and \emph{after} treatment. The mean \emph{improvement} in joint pain after \(13\)~weeks was~\(1.27\) (with a standard error of~\(0.57\)) measured using a standardised tool.

Are the following statements \emph{true} or \emph{false}?

\begin{enumerate}
\def\labelenumi{\arabic{enumi}.}
\item
  For paired data, the mean of the \emph{differences} is treated like the mean of a single variable.\tightlist  
\item
  An appropriate graph for displaying these data is a histogram of the differences.
\item
  The \emph{population} mean difference is denoted \(\mu_d\).
\item
  The standard error of the sample mean difference is denoted \(s_d\).
\item
  Only `before and after' studies can be paired. \tightlist
\item
  The null hypothesis is about the \emph{population} mean difference.
\item
  The value of the test statistic is \(2.23\).
\item
  The approximate value of the two-tailed \(P\)-value is very small.
\item
  The `test statistic' for this test is a \(t\)-score.
\end{enumerate}

\section{Exercises}\label{TestPairedMeansExercises}

\hyperref[Answers]{Answers to odd-numbered exercises} are given at the end of the book.

\captionsetup{font=small}

\begin{exercise}
\protect\hypertarget{exr:MeanDiffWhichPaired}{}\label{exr:MeanDiffWhichPaired}

Which (if any) of these scenarios are \emph{paired}?

\begin{enumerate}
\def\labelenumi{\arabic{enumi}.}
\tightlist
\item
  Heart rate is measured for each individual when sitting and when standing. (Some individuals have their heart rate recorded first while sitting, and some first while standing.) Each person receives two measurements, and the \emph{difference} in heart rate between sitting and standing is recorded.
\item
  The mean protein concentrations were compared in sea turtles before and after being rehabilitated \citep{data:March2018:turtles}.
\end{enumerate}

\end{exercise}

\begin{exercise}
\protect\hypertarget{exr:MeanDiffWhichPaired2}{}\label{exr:MeanDiffWhichPaired2}

Which (if any) of these scenarios are \emph{paired}?

\begin{enumerate}
\def\labelenumi{\arabic{enumi}.}
\tightlist
\item
  The mean HDL cholesterol concentration is recorded for a group of males and a group of females, and the means compared.
\item
  Heart rate was recorded for \(36\)~people, both before and after exercise, to determine how much the average heart rate increases.
\end{enumerate}

\end{exercise}

\begin{exercise}
\protect\hypertarget{exr:MeanDiffGDiffsA}{}\label{exr:MeanDiffGDiffsA}A group of primary school children was asked to complete a certain task on both a personal computer (PC) and using a tablet computer.

If the differences were defined as the time to complete the task on the PC, minus the time to complete the same task on a tablet (one difference for each child), what do the differences \emph{mean}?
\end{exercise}

\begin{exercise}
\protect\hypertarget{exr:MeanDiffGDiffsB}{}\label{exr:MeanDiffGDiffsB}Suppose water quality was recorded \(500\,\text{m}\) upstream and \(500\,\text{m}\) downstream of \(28\)~different copper mines.

If the differences were defined as the pH downstream minus the water pH upstream for each river, what do the differences \emph{mean}?
\end{exercise}

\begin{exercise}
\protect\hypertarget{exr:MeanDiffFlowering}{}\label{exr:MeanDiffFlowering}Suppose, in the example of Sect.~\ref{PairedInvasivePlants}, the differences were defined as the day of first flowering for willow, less the day of first flowering for skypilot. Write down, and interpret the meaning of, the approximate \(95\)\%~CI for the mean difference in first-flowering times.
\end{exercise}

\begin{exercise}
\protect\hypertarget{exr:MeanDiffTea}{}\label{exr:MeanDiffTea}Suppose, in the example of Sect.~\ref{ChamomileTea-Paired}, the differences were defined as \emph{increase} in total glucose (TG). Write down, and interpret the meaning of the approximate \(95\)\%~CI for the mean increase in TG for the tea-drinking group.
\end{exercise}

\begin{exercise}
\protect\hypertarget{exr:MeanDiffGrowingSquash}{}\label{exr:MeanDiffGrowingSquash}

{[}\emph{Dataset}: \texttt{Fruit}{]} \citet{mukherjee2019diversity} studied the effect of rainfall on growing Chayote squash (\emph{Sechium edule}). They compared the size of the fruit in a year with normal rainfall (2015) compared to a dry year (2014) on \(24\) farms:

\begin{quote}
For Chayote squash grown in Bangalore, what is the mean difference in fruit weight between a normal and dry year?
\end{quote}

Ten fruits were gathered from each farm in both years, and the average (mean) weight of the fruit recorded for the farm. Since the same farms are used in both years, the data are \emph{paired} (Table~\ref{tab:FruitsData}). Data is missing for Farm~20 in the dry year (2014), so there are \(n = 23\) differences.

\begin{table} \centering \centering\caption{\label{tab:FruitsData}The average weight of fruits (in g) in two different years, from $24$ farms. One observation is missing for Farm 20. The change is computed as the normal year (2015) minus dry year (2014).}

\fontsize{8}{10}\selectfont
\begin{tabular}[t]{cccc}
\toprule
\multicolumn{1}{c}{\textbf{ }} & \multicolumn{3}{c}{\textbf{Average fruit weight (kg)}} \\
\cmidrule(l{3pt}r{3pt}){2-4}
\textbf{Farm} & \textbf{Dry} & \textbf{Normal} & \textbf{Change}\\
\midrule
$1$ & $367.75$ & $371.05$ & $\phantom{0}\phantom{0}3.30$\\
$2$ & $238.25$ & $218.85$ & $\phantom{0}\llap{$-{}$}19.40$\\
$3$ & $271.25$ & $217.55$ & $\phantom{0}\llap{$-{}$}53.70$\\
$4$ & $286.27$ & $221.70$ & $\phantom{0}\llap{$-{}$}64.57$\\
$5$ & $259.20$ & $268.95$ & $\phantom{0}\phantom{0}9.75$\\
$\vdots$ & $\vdots$ & $\vdots$ & $\vdots$\\
\bottomrule
\end{tabular} \quad\quad 
\begin{tabular}[t]{cccc}
\toprule
\multicolumn{1}{c}{\textbf{ }} & \multicolumn{3}{c}{\textbf{Average fruit weight (kg)}} \\
\cmidrule(l{3pt}r{3pt}){2-4}
\textbf{Farm} & \textbf{Dry} & \textbf{Normal} & \textbf{Change}\\
\midrule
$\vdots$ & $\vdots$ & $\vdots$ & $\vdots$\\
$20$ & --- & $223.70$ & ---\\
$21$ & $257.50$ & $258.75$ & $\phantom{0}\phantom{0}\phantom{0}1.25$\\
$22$ & $230.70$ & $248.95$ & $\phantom{0}\phantom{0}18.25$\\
$23$ & $260.50$ & $155.95$ & $\phantom{0}\llap{$-{}$}104.55$\\
$24$ & $231.85$ & $219.30$ & $\phantom{0}\phantom{0}\llap{$-{}$}12.55$\\
\bottomrule
\end{tabular}
\end{table}

\begin{figure}[hbtp]

{\centering \includegraphics[width=0.9\linewidth]{jamovi/Fruit/FruitWeight} 

}

\caption{Software output for the fruit data.}\label{fig:FruitDescriptivesjamovi}
\end{figure}

\begin{enumerate}
\def\labelenumi{\arabic{enumi}.}
\tightlist
\item
  What is the \emph{unit of analysis}?\index{Units of analysis} What is the \emph{units of observation}?\index{Units of observation}
\item
  What is the advantage of using the same \(24\)~farms twice each?
\item
  Construct a suitable graph to display the differences.
\item
  Create a numerical summary table for the data (use Fig.~\ref{fig:FruitDescriptivesjamovi}).
\item
  What is the parameter? Carefully describe what it means.
\item
  Write down the hypotheses.
\item
  Sketch the sampling distribution.
\item
  Compute the \(t\)-score.
\item
  Determine the \(P\)-value.
\item
  Construct an approximate \(95\)\%~CI for the mean difference in fruit weight.
\item
  Are the test and the CI statistically valid?
\item
  Write a conclusion.
\end{enumerate}

\end{exercise}

\begin{exercise}
\protect\hypertarget{exr:TestPairedMeansCaptopril}{}\label{exr:TestPairedMeansCaptopril}

{[}\emph{Dataset}: \texttt{Captopril}{]} In a study of hypertension \citep{data:hand:handbook, data:macgregor:essential}, \(n = 15\) patients were given a drug (Captopril) and their systolic blood pressure measured (in mm~Hg) immediately before and two hours after being given the drug.

The aim is to see if there is evidence of a \emph{reduction} in blood pressure after taking Captopril. Use the data (Table~\ref{tab:CICaptoprilData}) and the software output (Fig.~\ref{fig:Captoriljamovi}) to answer these questions.

\begin{enumerate}
\def\labelenumi{\arabic{enumi}.}
\tightlist
\item
  Explain why it is probably more sensible to compute differences as the \emph{Before} minus the \emph{After} measurements. What do the differences \emph{mean} when computed this way?
\item
  What is the advantage of using the same patients for both the before and after measurements, rather than one group for before measurements and a different group of people for after measurements?
\item
  What is the parameter? Carefully describe what it means.
\item
  Construct a suitable graph to display the differences.
\item
  Write down the hypotheses.
\item
  Sketch the sampling distribution.
\item
  Write down the \(t\)-score.
\item
  Write down the \(P\)-value.
\item
  Write down the \emph{exact} \(95\)\%~CI using the computer output (Fig.~\ref{fig:Captoriljamovi}).
\item
  Compute an \emph{approximate} \(95\)\%~CI for the mean difference.
\item
  Why are the two CIs different?
\item
  Write a conclusion.
\item
  Are the CI and test statistically valid?
\end{enumerate}

\end{exercise}

\begin{figure}[hbtp]

{\centering \includegraphics[width=0.8\linewidth]{jamovi/CaptoprilAll/CaptoprilAll-PairedTOutput} 

}

\caption{Software output for the Captopril data.}\label{fig:Captoriljamovi}
\end{figure}

\begin{exercise}
\protect\hypertarget{exr:TestPairedMeansTasteOfBroccoli}{}\label{exr:TestPairedMeansTasteOfBroccoli}

People often struggle to eat the recommended intake of vegetables. \citet{data:Fritts2018:Vegetables} explored ways to increase vegetable intake in teens. Teens rated the taste of raw broccoli, and raw broccoli served with a specially-made dip.

Each teen (\(n = 100\)) had a \emph{pair} of measurements: the taste rating of the broccoli \emph{with} and \emph{without} dip. Taste was assessed using a `\(100\,\text{mm}\)~visual analogue scale', where a \emph{higher} score means a \emph{better} taste. In summary:

\begin{itemize}
\tightlist
\item
  for raw broccoli, the mean taste rating was~\(56.0\) (with a standard deviation of~\(26.6\));
\item
  for raw broccoli served with dip, the mean taste rating was~\(61.2\) (with a standard deviation of~\(28.7\)).
\end{itemize}

Because the data are paired, the \emph{differences} are the best way to describe the data. The mean difference in the ratings was~\(5.2\), with standard error of~\(3.06\).

\begin{enumerate}
\def\labelenumi{\arabic{enumi}.}
\tightlist
\item
  Construct a suitable numerical summary table.
\item
  What does a positive difference mean?
\item
  Perform a hypothesis test to see if the use of dip \emph{increases} the mean taste rating.
\item
  Compute the approximate \(95\)\%~CI for the mean difference in taste ratings.
\item
  Are the CI and test statistically valid?
\end{enumerate}

\end{exercise}

\begin{exercise}
\protect\hypertarget{exr:TestPairedMeansSmokingAndExercise}{}\label{exr:TestPairedMeansSmokingAndExercise}

\citet{data:Allen2018:Smoking} examined the effect of exercise on smoking. Men and women were assessed on their `intention to smoke', both before and after exercise for each subject (using two quantitative questionnaires). Smokers (`smoking at least five cigarettes per day') aged~\(18\) to~\(40\) were enrolled for the study. For the \(23\)~women in the study, the mean intention to smoke after exercise \emph{reduced} by~\(0.66\) (with a standard error of~\(0.37\)). (Larger values for `intention to smoke' mean a greater intent to smoke.)

\begin{enumerate}
\def\labelenumi{\arabic{enumi}.}
\tightlist
\item
  Perform a hypothesis test to determine if there is evidence of a population mean \emph{reduction} in intention-to-smoke for women after exercising.
\item
  Find an approximate \(95\)\% CI for the population mean reduction in intention to smoke for women after exercising.
\item
  Are the CI and test statistically valid?
\end{enumerate}

\end{exercise}

\begin{exercise}
\protect\hypertarget{exr:TestPairedMeansFerritin}{}\label{exr:TestPairedMeansFerritin}{[}\emph{Dataset}: \texttt{Ferritin}{]} In a study \citep{cressie1984use} conducted at the Adelaide Children's Hospital (p.~107; emphasis added):

\begin{quote}
\ldots{} a group of beta thalassemia patients {[}\ldots{]} were treated by a continuous infusion of desferrioxamine, in order to \emph{reduce} their ferritin content\ldots{}
\end{quote}

Using the data in Table~\ref{tab:FerritinTable}, conduct a hypothesis test to determine if there is evidence that the treatment reduces the ferritin content, as intended. Make sure to include a \(95\)\%~CI in the conclusion.
\end{exercise}

\begin{table} \centering \centering\caption{\label{tab:FerritinTable}The ferritin content (in $\,\ensuremath{\mu}\text{g}.\text{L}$) for $20$\ thalassemia patients at the Adelaide Children's Hospital.}

\fontsize{8}{10}\selectfont
\begin{tabular}{ccc}
\toprule
\textbf{Sept.} & \textbf{March} & \textbf{Reduction}\\
\midrule
$6630$ & $5100$ & $\phantom{0}1530$\\
$4590$ & $3510$ & $\phantom{0}1080$\\
$3510$ & $6600$ & $\phantom{0}\llap{$-{}$}3090$\\
$6375$ & $8000$ & $\phantom{0}\llap{$-{}$}1625$\\
\addlinespace
$2500$ & $2800$ & $\phantom{0}\phantom{0}\llap{$-{}$}300$\\
$1400$ & $2860$ & $\phantom{0}\llap{$-{}$}1460$\\
$4580$ & $3640$ & $\phantom{0}\phantom{0}940$\\
$6885$ & $9030$ & $\phantom{0}\llap{$-{}$}2145$\\
\addlinespace
$4200$ & $4420$ & $\phantom{0}\phantom{0}\llap{$-{}$}220$\\
$5600$ & $7910$ & $\phantom{0}\llap{$-{}$}2310$\\
\bottomrule
\end{tabular} \quad 
\begin{tabular}{ccc}
\toprule
\textbf{Sept.} & \textbf{March} & \textbf{Reduction}\\
\midrule
$5360$ & $6780$ & $\phantom{0}\llap{$-{}$}1420$\\
$6110$ & $7250$ & $\phantom{0}\llap{$-{}$}1140$\\
$5300$ & $6000$ & $\phantom{0}\phantom{0}\llap{$-{}$}700$\\
$3120$ & $4300$ & $\phantom{0}\llap{$-{}$}1180$\\
\addlinespace
$3300$ & $4680$ & $\phantom{0}\llap{$-{}$}1380$\\
$11400$ & $8500$ & $\phantom{0}2900$\\
$3100$ & $3735$ & $\phantom{0}\phantom{0}\llap{$-{}$}635$\\
$2800$ & $2730$ & $\phantom{0}\phantom{0}\phantom{0}70$\\
\addlinespace
$3500$ & $6600$ & $\phantom{0}\llap{$-{}$}3100$\\
$12700$ & $7000$ & $\phantom{0}5700$\\
\bottomrule
\end{tabular}
\end{table}

\begin{exercise}
\protect\hypertarget{exr:StressSurgeryHT}{}\label{exr:StressSurgeryHT}

{[}\emph{Dataset}: \texttt{Stress}{]} The concentration of beta-endorphins in the blood is a sign of stress. \citet{hoaglin2011exploring} measured the beta-endorphin concentration for \(19\)~patients about to undergo surgery \citep{data:hand:handbook}. Each patient had their beta-endorphin concentrations measured \(12\)--\(14\,\text{h}\) before surgery, and also \(10\,\text{mins}\) before surgery (in fmol.mL\textsuperscript{\(-1\)}).

A numerical summary (Table~\ref{tab:StressDescriptivesjamoviHT}) was produced from output.

\begin{enumerate}
\def\labelenumi{\arabic{enumi}.}
\tightlist
\item
  Use the output to test the RQ.
\item
  Use the software output in Fig.~\ref{fig:StressDescriptivesjamovi} to construct an \emph{approximate} \(95\)\%~CI for the \emph{increase} in beta-endorphin concentrations as surgery gets closer.
\item
  Use the software output in Fig.~\ref{fig:StressDescriptivesjamovi} to write down the \emph{exact} \(95\)\%~CI for the \emph{increase} in beta-endorphin concentrations as surgery gets closer.
\item
  Why is there a difference between the two CIs?
\item
  Are the CI and test statistically valid?
\end{enumerate}

\end{exercise}

\begin{table}
\centering
\caption{\label{tab:StressDescriptivesjamoviHT}The surgery-stress data (in fmol.mL$^{-1}$).}
\centering
\fontsize{8}{10}\selectfont
\begin{tabular}[t]{lcccc}
\toprule
\textbf{ } & \textbf{Sample mean} & \textbf{Standard deviation} & \textbf{Standard error} & \textbf{$n$}\\
\midrule
12--14 h before surgery & $\phantom{0}8.35$ & $\phantom{0}4.40$ & $1.01$ & $19$\\
10 min before surgery & $16.05$ & $12.51$ & $2.87$ & $19$\\
\midrule
\em{Increase} & \em{$\phantom{0}7.70$} & \em{$13.52$} & \em{$3.10$} & \em{$19$}\\
\bottomrule
\end{tabular}
\end{table}

\begin{figure}[hbtp]

{\centering \includegraphics[width=0.7\linewidth]{jamovi/Stress/StressDescriptives} 

}

\caption{Software output for the surgery-stress data.}\label{fig:StressDescriptivesjamovi}
\end{figure}

\begin{exercise}
\protect\hypertarget{exr:MeanDiffCOVIDCI}{}\label{exr:MeanDiffCOVIDCI}

A study of \(n = 213\) Spanish health students \citep{romero2020physical} measured (among other variables) the number of minutes of vigorous physical activity (PA) performed by students weekly \emph{before} and \emph{during} the \textsc{covid}-19 lockdown (from March~to April~2020 in Spain). Since the \emph{before} and \emph{during} lockdown were both measured on \emph{each} participant, the data are \emph{paired}. The data are summarised in Table~\ref{tab:COVIDsummaryTable}.

\begin{enumerate}
\def\labelenumi{\arabic{enumi}.}
\tightlist
\item
  Explain what the differences \emph{mean}.
\item
  Compute the standard error of the differences.
\item
  Perform a hypothesis test to determine if mean minutes of vigorous PA changed from before to during the lockdowns.
\end{enumerate}

\end{exercise}



\begin{table}
\centering
\caption{\label{tab:COVIDsummaryTable}Summary information for the \textsc{covid}-lockdown exercise data for \(n = 214\) Spanish students: weekly minutes of vigorous physical activity.}
\centering
\fontsize{8}{10}\selectfont
\begin{tabular}[t]{lcc}
\toprule
\textbf{ } & \textbf{Mean (mins)} & \textbf{Std dev. (mins)}\\
\midrule
Before lockdown & $28.47$ & $54.13$\\
During lockdown & $30.66$ & $30.04$\\
\midrule
\em{Increase} & \em{$\phantom{0}2.68$} & \em{$51.30$}\\
\bottomrule
\end{tabular}
\end{table}

\begin{exercise}
\protect\hypertarget{exr:MeanDiffStudentEatingHabits}{}\label{exr:MeanDiffStudentEatingHabits}

What happens when students start university? Many students will be responsible for their own meals for the first time, so some may forgo healthy foods for convenient, but less healthy, foods. Alternatively, some may not be able to afford sufficient or healthy food. Exercise regimes may also change.

\citet{levitsky2004freshman} recorded some students' weights as they began university, and then \emph{the same} students' weight some later time. They asked the RQ:

\begin{quote}
For Cornell University students, what is the \emph{mean weight change} in students after \(12\) weeks at university?
\end{quote}

The data collected to answer this RQ are shown in Table~\ref{tab:DataWeightChange} \citep{DASL:WeightChange}.

\begin{enumerate}
\def\labelenumi{\arabic{enumi}.}
\tightlist
\item
  Use the software output (Fig.~\ref{fig:WeightGainOutput}) to compute an \emph{approximate} \(95\)\%~CI for the weight \emph{gain} from Weeks~\(1\) to~\(12\).
\item
  Use the software output to write down an \emph{exact} \(95\)\%~CI for the weight \emph{gain} from Weeks~\(1\) to~\(12\).
\item
  Comment on the two CIs.
\item
  Are the CIs statistically valid?
\item
  Conduct a hypothesis tests to determine if there is a change in mean weight from Weeks~\(1\) to~\(12\).
\item
  Do you think the weight gain would be of practical importance?
\end{enumerate}

\begin{table} \centering \centering\caption{\label{tab:DataWeightChange}The student weight-change data, showing the weight of students in Week\ 1 at university, in Week\ 12, and the weight gain (all in kg). These are the first five and the last five of the $68$ total observations. (A negative weight gain means a weight loss.)}

\fontsize{8}{10}\selectfont
\begin{tabular}[t]{cccc}
\toprule
\multicolumn{1}{c}{\textbf{ }} & \multicolumn{3}{c}{\textbf{Weight (in kg)}} \\
\cmidrule(l{3pt}r{3pt}){2-4}
\textbf{Student} & \textbf{Week 1} & \textbf{Week 12} & \textbf{Weight gain}\\
\midrule
$1$ & $77.0$ & $75.6$ & $\phantom{0}\llap{$-{}$}1.4$\\
$2$ & $49.5$ & $50.0$ & $\phantom{0}0.5$\\
$3$ & $60.3$ & $61.2$ & $\phantom{0}0.9$\\
$4$ & $51.8$ & $53.6$ & $\phantom{0}1.8$\\
$5$ & $67.5$ & $69.8$ & $\phantom{0}2.3$\\
$\vdots$ & $\vdots$ & $\vdots$ & $\vdots$\\
\bottomrule
\end{tabular} \enskip 
\begin{tabular}[t]{cccc}
\toprule
\multicolumn{1}{c}{\textbf{ }} & \multicolumn{3}{c}{\textbf{Weight (in kg)}} \\
\cmidrule(l{3pt}r{3pt}){2-4}
\textbf{Student} & \textbf{Week 1} & \textbf{Week 12} & \textbf{Weight gain}\\
\midrule
$\vdots$ & $\vdots$ & $\vdots$ & $\vdots$\\
$64$ & $69.8$ & $71.1$ & $\phantom{0}1.3$\\
$65$ & $72.0$ & $72.4$ & $\phantom{0}0.4$\\
$66$ & $51.8$ & $53.6$ & $\phantom{0}1.8$\\
$67$ & $75.2$ & $76.5$ & $\phantom{0}1.3$\\
$68$ & $59.0$ & $59.0$ & $\phantom{0}0.0$\\
\bottomrule
\end{tabular}
\end{table}

\begin{figure}[hbtp]

{\centering \includegraphics[width=0.6\linewidth]{jamovi/WeightGain/WeightGain-Numericals} \includegraphics[width=1\linewidth]{jamovi/WeightGain/WeightGain-PairedT} 

}

\caption{The weight-gain data: software output.}\label{fig:WeightGainOutput}
\end{figure}

\end{exercise}

\begin{exercise}
\protect\hypertarget{exr:PairedCIExercisesAnorexia}{}\label{exr:PairedCIExercisesAnorexia}{[}\emph{Dataset}: \texttt{Anorexia}{]} Young girls with anorexia (\(n = 29\)) received cognitive behavioural treatment (\citet{data:hand:handbook}), and their weight before and after treatment were recorded. In summary:

\begin{itemize}
\tightlist
\item
  Before the treatment, the mean weight was \(82.69\)~pounds (\(s = 4.845\)~pounds);
\item
  After the treatment, the mean weight was \(85.70\)~pounds (\(s = 8.352\)~pounds).
\end{itemize}

The mean weight gain per girls was~\(3.01\)~pounds, with a standard deviation of~\(7.31\)~pounds. Find an approximate \(95\)\%~CI for the population mean weight gain. Do you think the treatment had any meaningful impact on the mean weight gain of the girls, based solely on these data?
\end{exercise}

\begin{exercise}
\protect\hypertarget{exr:PairedCIExercisesSoilN}{}\label{exr:PairedCIExercisesSoilN}

{[}\emph{Dataset}: \texttt{SoilCN}{]} \citet{lambie2021microbial} compared the percentage nitrogen (\%N) in soils from intensively-grazed irrigated and non-irrigated pastures. The researchers \emph{paired}\index{Comparison!within individuals} similar irrigated and non-irrigated sites (p.~338):

\begin{quote}
The irrigated and non-irrigated pairs within each site were within~\(100\,\text{m}\) of each other and were on the same soil, landform and usually the same farm with the same farm management\ldots{}
\end{quote}

One RQ in the study was:

\begin{quote}
For intensively grazed pastures sites, is there a mean reduction in percentage soil nitrogen (\%N) when sites are irrigated, compared to non-irrigated?
\end{quote}

The data are shown in Table~\ref{tab:SoilCN}. Use the data to answer the RQ.

\begin{table} \centering \centering\caption{\label{tab:SoilCN}The percentage total soil nitrogen (\%N) in irrigated and non-irrigated soils in $28$\ sites.}

\fontsize{8}{10}\selectfont
\begin{tabular}[t]{ccc}
\toprule
\multicolumn{1}{c}{\textbf{\%N:}} & \multicolumn{1}{c}{\textbf{\%N: not}} & \multicolumn{1}{c}{\textbf{\%N: reduction}} \\
\textbf{irrigated} & \textbf{irrigated} & \textbf{when irrigated}\\
\midrule
$\phantom{0}0.35$ & $\phantom{0}0.38$ & $\phantom{0}0.03$\\
$\phantom{0}0.42$ & $\phantom{0}0.43$ & $\phantom{0}0.01$\\
$\phantom{0}0.27$ & $\phantom{0}0.23$ & $\phantom{0}\llap{$-{}$}0.04$\\
$\phantom{0}0.18$ & $\phantom{0}0.24$ & $\phantom{0}0.06$\\
$\phantom{0}0.56$ & $\phantom{0}0.58$ & $\phantom{0}0.02$\\
$\phantom{0}0.34$ & $\phantom{0}0.26$ & $\phantom{0}\llap{$-{}$}0.08$\\
$\phantom{0}0.26$ & $\phantom{0}0.25$ & $\phantom{0}\llap{$-{}$}0.01$\\
$\phantom{0}0.58$ & $\phantom{0}0.44$ & $\phantom{0}\llap{$-{}$}0.14$\\
$\phantom{0}0.50$ & $\phantom{0}0.49$ & $\phantom{0}\llap{$-{}$}0.01$\\
$\phantom{0}0.47$ & $\phantom{0}0.55$ & $\phantom{0}0.08$\\
$\phantom{0}0.55$ & $\phantom{0}0.55$ & $\phantom{0}0.00$\\
$\phantom{0}0.41$ & $\phantom{0}0.45$ & $\phantom{0}0.04$\\
$\phantom{0}0.51$ & $\phantom{0}0.54$ & $\phantom{0}0.03$\\
$\phantom{0}0.47$ & $\phantom{0}0.56$ & $\phantom{0}0.09$\\
\bottomrule
\end{tabular} \enskip 
\begin{tabular}[t]{ccc}
\toprule
\multicolumn{1}{c}{\textbf{\%N:}} & \multicolumn{1}{c}{\textbf{\%N: not}} & \multicolumn{1}{c}{\textbf{\%N: reduction}} \\
\textbf{irrigated} & \textbf{irrigated} & \textbf{when irrigated}\\
\midrule
$\phantom{0}0.27$ & $\phantom{0}0.33$ & $\phantom{0}0.06$\\
$\phantom{0}0.29$ & $\phantom{0}0.31$ & $\phantom{0}0.02$\\
$\phantom{0}0.40$ & $\phantom{0}0.43$ & $\phantom{0}0.03$\\
$\phantom{0}0.26$ & $\phantom{0}0.26$ & $\phantom{0}0.00$\\
$\phantom{0}0.52$ & $\phantom{0}0.53$ & $\phantom{0}0.01$\\
$\phantom{0}0.30$ & $\phantom{0}0.41$ & $\phantom{0}0.11$\\
$\phantom{0}0.20$ & $\phantom{0}0.32$ & $\phantom{0}0.12$\\
$\phantom{0}0.30$ & $\phantom{0}0.30$ & $\phantom{0}0.00$\\
$\phantom{0}0.24$ & $\phantom{0}0.26$ & $\phantom{0}0.02$\\
$\phantom{0}0.49$ & $\phantom{0}0.67$ & $\phantom{0}0.18$\\
$\phantom{0}0.27$ & $\phantom{0}0.29$ & $\phantom{0}0.02$\\
$\phantom{0}0.44$ & $\phantom{0}0.47$ & $\phantom{0}0.03$\\
$\phantom{0}0.27$ & $\phantom{0}0.28$ & $\phantom{0}0.01$\\
$\phantom{0}0.40$ & $\phantom{0}0.50$ & $\phantom{0}0.10$\\
\bottomrule
\end{tabular}
\end{table}

\begin{figure}[hbtp]

{\centering \includegraphics[width=1\linewidth]{jamovi/SoilCN/SoilCN-Testing} \includegraphics[width=0.6\linewidth]{jamovi/SoilCN/SoilCN-Summary} 

}

\caption{Software output for the nitrogen data. In the top table, the difference is implied as non-irrigated minus irrigated.}\label{fig:Nitrogenjamovi}
\end{figure}

\end{exercise}

\begin{exercise}
\protect\hypertarget{exr:PairedCIJumping}{}\label{exr:PairedCIJumping}{[}\emph{Dataset}: \texttt{Jumping}{]} \citet{hebert2023effect} recorded double-legged jumping distance for \(80\) healthy people, when they wore both shoes and were barefoot (Exercise~\ref{exr:CompareWithinJumping}). Use the data to form a \(95\)\%~CI to estimate the mean distance people can jump further when barefoot.
\end{exercise}

\begin{exercise}
\protect\hypertarget{exr:PairedCIWCTennis}{}\label{exr:PairedCIWCTennis}{[}\emph{Dataset}: \texttt{WCTennis}{]} \citet{alberca2022sprint} recorded the push time (the time between a shot and resetting) for French wheelchair tennis players, while holding a racquet and not holding a racquet (Table~\ref{tab:WCTennis}; \citet{alberca2022sprintDATA}). Use the data to form a \(95\)\%~CI to estimate the mean difference between push times with and without a racquet.
\end{exercise}

\captionsetup{font=normalsize}

\begin{EOCanswerBox}{iconmonstr-check-mark-14-240.png}
\textbf{Answers to \emph{Quick review} questions:} \textbf{1.} True. \textbf{2.} True. \textbf{3.} True. \textbf{4.} False. \textbf{5.} False. \textbf{6.} True. \textbf{7.} True. \textbf{8.} True. \textbf{9.} True.

\end{EOCanswerBox}

\chapter{Comparing two means: CIs and tests}\label{AnalysisTwoMeans}

\index{Difference between means}

\begin{cols}
\begin{col}{0.52\textwidth}

\begin{objectivesBox}{iconmonstr-target-4-240.png}
You have learnt to ask an RQ, design a study, classify and summarise the data, construct confidence intervals, and conduct hypothesis tests.
\textbf{In this chapter}, you will learn to:
\begin{itemize}\tightlist
  \item
  identify situations where comparing two means is appropriate.
  \item
  construct confidence intervals for the difference between two means.
  \item
  conduct hypothesis tests for comparing two means.
  \item
  determine whether the conditions for using these methods apply in a given situation.
\end{itemize}
\end{objectivesBox}

\end{col}

\begin{col}{0.03\textwidth}
~
\end{col}

\begin{col}{0.45\textwidth}

\includegraphics[width=0.95\linewidth]{30-CIsTesting-TwoMeans_files/figure-latex/unnamed-chunk-6-1} 
\end{col}
\end{cols}

\section{Introduction: garter snakes}\label{TwoMeansHT-intro}

Some Mexican garter snakes (\emph{Thamnophis melanogaster}) live in habitats with no crayfish, while some live in habitats with crayfish and hence use crayfish as a food source. \citet{manjarrez2017morphological} were interested in whether the snakes in these two regions were different:

\begin{quote}
For female Mexican garter snakes, is the mean snout--vent length (SVL) different for those in regions with crayfish and without crayfish?
\end{quote}

Two different groups of snakes are studied, so this is a relational RQ (the study uses a between-individuals comparison\index{Comparison!between individuals}) with no intervention, and the data are shown in Table~\ref{tab:SnakesDataTableTest}.

\begin{table} \centering \centering\caption{\label{tab:SnakesDataTableTest}Snout--vent length (in cm) for female Mexican garter snakes living in  crayfish ($n = 35$) and non-crayfish ($n = 41$) regions.}

\fontsize{8}{10}\selectfont
\begin{tabular}[t]{ccccccccc}
\toprule
\multicolumn{9}{c}{\textbf{Non-crayfish region}} \\
\cmidrule(l{3pt}r{3pt}){1-9}
$52$ & $50$ & $51$ & $52$ & $44$ & $34$ & $39$ & $38$ & $44$\\
$48$ & $43$ & $35$ & $48$ & $43$ & $54$ & $48$ & $26$ & \\
$29$ & $44$ & $33$ & $48$ & $48$ & $43$ & $45$ & $50$ & \\
$40$ & $36$ & $26$ & $47$ & $48$ & $24$ & $50$ & $38$ & \\
$48$ & $44$ & $40$ & $38$ & $40$ & $36$ & $26$ & $26$ & \\
\bottomrule
\end{tabular} \quad\quad 
\begin{tabular}[t]{ccccccc}
\toprule
\multicolumn{7}{c}{\textbf{Crayfish region}} \\
\cmidrule(l{3pt}r{3pt}){1-7}
$51$ & $26$ & $19$ & $46$ & $49$ & $46$ & $18$\\
$16$ & $21$ & $22$ & $32$ & $34$ & $17$ & $32$\\
$38$ & $34$ & $20$ & $39$ & $40$ & $49$ & $20$\\
$44$ & $46$ & $56$ & $26$ & $38$ & $18$ & $52$\\
$40$ & $46$ & $47$ & $24$ & $20$ & $24$ & $45$\\
\bottomrule
\end{tabular}
\end{table}

\section{Summarising the data and error bar charts}\label{ErrorBarCharts}

A numerical summary \emph{must} summarise the difference between the means, because the RQ is about this difference. Both groups should be summarised too.\index{Mean!difference between} The information can be found using software (Fig.~\ref{fig:SnakesSummaryTestjamovi}),\index{Software output!comparing two means} and compiled into a table (Table~\ref{tab:SnakesNumericalTest}). The appropriate summary for graphically summarising the \emph{data} is (for example) a boxplot (Fig.~\ref{fig:SnakesErrorbar}, left panel).\index{Graphs!boxplot}

\begin{importantBox}{iconmonstr-warning-8-240.png}
No sample size or standard deviation is provided for the differences in Table~\ref{tab:SnakesNumericalTest}; these make no sense in the context of comparing two means.

\end{importantBox}

\begin{figure}[hbtp]

{\centering \includegraphics[width=1\linewidth]{jamovi/Snakes/Snakes-Test} \includegraphics[width=0.65\linewidth]{jamovi/Snakes/Snakes-Summary} 

}

\caption{Software output for the garter-snakes data.}\label{fig:SnakesSummaryTestjamovi}
\end{figure}

\begin{table}
\centering
\caption{\label{tab:SnakesNumericalTest}Numerical summaries of SVL (in cm) for female garter snakes in two regions.}
\centering
\fontsize{8}{10}\selectfont
\begin{tabular}[t]{lcccc}
\toprule
\textbf{ } & \textbf{Mean} & \textbf{Standard deviation} & \textbf{Sample size} & \textbf{Standard error}\\
\midrule
Non-crayfish region & $42.57$ & $\phantom{0}7.79$ & $41$ & $1.216$\\
Crayfish region & $34.17$ & $12.49$ & $35$ & $2.112$\\
\midrule
\em{Difference} & \em{$\phantom{0}8.39$} & \em{} & \em{} & \em{$2.437$}\\
\bottomrule
\end{tabular}
\end{table}

\begin{figure}[hbtp]

{\centering \includegraphics[width=0.95\linewidth]{30-CIsTesting-TwoMeans_files/figure-latex/SnakesErrorbar-1} 

}

\caption{Boxplot (left) and error bar chart (right) of SVL for female snakes in two regions.}\label{fig:SnakesErrorbar}
\end{figure}

Since two groups are being compared, subscripts are used to distinguish between the statistics for the two groups; say, Groups~\(1\) and~\(2\) in general (Table~\ref{tab:IndSampleNotationHT}). Using this notation, the \emph{parameter} in the RQ is the difference between population means: \(\mu_1 - \mu_2\). As usual, the population values are unknown, so this is estimated using the statistic \(\bar{x}_1 - \bar{x}_2\).

\begin{table}
\centering
\caption{\label{tab:IndSampleNotationHT}Notation used to distinguish the two independent groups.}
\centering
\fontsize{8}{10}\selectfont
\begin{tabular}[t]{lccc}
\toprule
\textbf{ } & \textbf{Group 1} & \textbf{Group 2} & \textbf{Comparing groups}\\
\midrule
Sample sizes: & $n_1$ & $n_2$ & \\
\addlinespace
Population means: & $\mu_1$ & $\mu_2$ & $\mu_1 - \mu_2$\\
Sample means: & $\bar{x}_1$ & $\bar{x}_2$ & $\bar{x}_1 - \bar{x}_2$\\
\addlinespace
Standard deviations: & $s_1$ & $s_2$ & \\
Standard errors: & $\displaystyle\text{s.e.}(\bar{x}_1) = \frac{s_1}{\sqrt{n_1}}$ & $\displaystyle\text{s.e.}(\bar{x}_2) = \frac{s_2}{\sqrt{n_2}}$ & $\displaystyle\text{s.e.}(\bar{x}_1 - \bar{x}_2)$\\
\bottomrule
\end{tabular}
\end{table}

For the garter-snakes data, define the differences as the mean for females snakes living in non-crayfish regions~(\(N\)), \emph{minus} the mean for female snakes in crayfish regions~(\(C\)): \(\mu_N - \mu_C\). This is the \emph{parameter}. By this definition, the differences refer to how much larger (on average) the SVL is for snakes living in non-crayfish regions.

\begin{importantBox}{iconmonstr-warning-8-240.png}
Here the difference is computed as the mean~SVL for snakes living in non-crayfish regions, \emph{minus} the mean~SVL for snakes living in crayfish regions. Computing the difference as the mean~SVL for snakes in crayfish regions, \emph{minus} non-crayfish regions is also correct.

You need to be clear about how the difference is computed, and be consistent throughout. The \emph{meaning} of the conclusions will be the same whichever direction is used.

\end{importantBox}

A useful way to compare the means of two (or more) groups is to display the CIs for the means of the groups being compared in an \emph{error bar chart}.\index{Graphs!error bar charts} Error bars charts display the expected variation \emph{in the sample means} from sample to sample, while boxplots display the variation \emph{in the individual observations}. For the garter-snakes data, the error bar chart (Fig.~\ref{fig:SnakesErrorbar}, right panel) shows the \(95\)\%~CI for each group; the mean appears as a dot.

The two CIs for the SVL are (using information from the bottom table in Fig.~\ref{fig:SnakesSummaryTestjamovi}):

\begin{itemize}
\tightlist
\item
  \makebox[34mm][l]{Crayfish regions:} \(34.171 \pm (2 \times 2.112)\), or from~\(29.94\) to~\(38.40\,\text{cm}\).
\item
  \makebox[34mm][l]{Non-crayfish regions:} \(42.566 \pm (2\times 1.216)\), or from~\(40.13\) to~\(45.00\,\text{cm}\).
\end{itemize}

However, the error bar chart, and these CIs, do not give a CI for the \emph{difference} between the two means, as relevant to the RQ.

\begin{example}[Error bar charts]
\protect\hypertarget{exm:ErrorBarCharts2}{}\label{exm:ErrorBarCharts2}\citet{data:ForestBiomass2017} studied the foliage biomass of small-leaved lime trees from three sources: coppices; natural; planted. Three graphical summaries are shown in Fig.~\ref{fig:LimeTreesBoxErrorbar}: a boxplot (showing the variation in \emph{individual} trees; left), an error bar chart (showing the variation in the \emph{sample means}; centre) on the same vertical scale as the boxplot, and the same error bar chart using a more appropriate scale for the error bar plot (right).
\end{example}

\begin{figure}[hbtp]

{\centering \includegraphics[width=1\linewidth]{30-CIsTesting-TwoMeans_files/figure-latex/LimeTreesBoxErrorbar-1} 

}

\caption{Boxplot (left) and error bar charts (centre; right) comparing the mean foliage biomass for small-leaved lime trees from three sources (C:\ Coppice; N:\ Natural; P:\ Planted). The centre panel shows an error bar chart using the same vertical scale as the boxplot; the dashed horizontal lines are the limits of the error bar chart on the right. The right error bar chart uses a more appropriate scale on the vertical axis. The solid dots show the mean of the distributions.}\label{fig:LimeTreesBoxErrorbar}
\end{figure}

\section{\texorpdfstring{Confidence intervals for \(\mu_1 - \mu_2\)}{Confidence intervals for \textbackslash mu\_1 - \textbackslash mu\_2}}\label{CIDiffBetweenMeans}

\index{Sampling distribution!comparing two means}\index{Confidence intervals!comparing two means|(}

Each sample will comprise different snakes, and give different SVLs. The sample means for each group will differ from sample to sample, and the \emph{difference} between the sample means will be different for each sample also. The \emph{difference} between the sample means varies from sample to sample, and so has a sampling distribution and a standard error.

\begin{definition}[Sampling distribution for the difference between two sample means]
\protect\hypertarget{def:DEFSamplingDistributionDiffMeans}{}\label{def:DEFSamplingDistributionDiffMeans}The \emph{sampling distribution of the difference between two sample means}~\(\bar{x}_1\) and~\(\bar{x}_2\) is (when the appropriate conditions are met; Sect.~\ref{ValidityTwoMeans}) described by:

\begin{itemize}
\tightlist
\item
  an approximate normal distribution,
\item
  centred around a sampling mean whose value is~\({\mu_1} - {\mu_2}\), the difference between the \emph{population} means,
\item
  with a standard deviation, called the standard error of the difference between the means, of \(\displaystyle\text{s.e.}(\bar{x}_1 - \bar{x}_2)\).
\end{itemize}

The standard error for the difference between the means is found using \[
\text{s.e.}(\bar{x}_1 - \bar{x}_2) = \sqrt{ \text{s.e.}(\bar{x}_1)^2 + \text{s.e.}(\bar{x}_2)^2},
\] though this value will often be \emph{given} (e.g., on computer output) rather than needing to be computed.
\end{definition}

For the garter-snakes data, the differences between the sample means will have:

\begin{itemize}
\tightlist
\item
  an approximate normal distribution,
\item
  centred around the sampling mean whose value is \(\mu_N - \mu_C\),
\item
  with a standard deviation, called the \emph{standard error} of the difference, of \(\text{s.e.}(\bar{x}_P - \bar{x}_C) = 2.437\).
\end{itemize}

The standard error of the difference between the means was computed using \[
  \text{s.e.}(\bar{x}_N - \bar{x}_C)
  = \sqrt{ \text{s.e.}(\bar{x}_N)^2 + \text{s.e.}(\bar{x}_C)^2}
  = \sqrt{ 1.216^2 + 2.1112^2 } = 2.437,
\] the same value shown in the \emph{second row} of the software output (Fig.~\ref{fig:SnakesSummaryTestjamovi}).

The sampling distribution describes how the values of \(\bar{x}_N - \bar{x}_C\) vary from sample to sample. Then, finding a \(95\)\%~CI for the difference between the mean SVLs is similar to the process used in Chap.~\ref{OneMeanConfInterval}, since the sampling distribution has an approximate normal distribution: \[
\text{statistic} \pm \big(\text{multiplier} \times\text{s.e.}(\text{statistic})\big).
\] When the statistic is \(\bar{x}_P - \bar{x}_C\), the approximate \(95\)\%~CI is \[
(\bar{x}_N - \bar{x}_C) \pm \big(2 \times \text{s.e.}(\bar{x}_N - \bar{x}_C)\big).
\] So, in this case, the approximate \(95\)\%~CI is \[
8.394 \pm (2 \times 2.437)
\] or \(8.394\pm 4.874\), after rounding appropriately. We write:

\begin{quote}
The difference between mean SVLs is~\(8.39\,\text{cm}\), shorter for those living in a crayfish region (mean: \(34.17\,\text{cm}\); s.e.: \(2.112\); \(n = 35\)) compared to those \emph{not} living in a crayfish region (mean: \(42.57\,\text{cm}\); s.e.: \(1.216\); \(n = 41\)), with an \emph{approximate} \(95\)\%~CI for the difference between mean SVLs from \(3.52\) to \(13.27\,\text{cm}\).
\end{quote}

The plausible values for the difference between the two population means SVLs are between~\(3.52\) to \(13.27\,\text{cm}\) (shorter for those living in crayfish regions).

\begin{importantBox}{iconmonstr-warning-8-240.png}
Giving the CI alone is insufficient; the \emph{direction} in which the differences were calculated must be given, so readers know which group had the higher mean.

\end{importantBox}

Output from software often shows two CIs for the difference between the two means (Fig.~\ref{fig:SnakesSummaryTestjamovi}). \emph{We will use the results from Welch's test (the second row)},\index{Welch's test} as this row of output is more general, and makes fewer assumptions, than the results in the first row. The information in the second row makes fewer assumptions, and is more widely applicable.

\begin{importantBox}{iconmonstr-warning-8-240.png}
Most software gives \emph{two} CIs: one assuming the standard deviations in the two groups are the same (Student's), and another \emph{not} assuming the standard deviations in the two groups are the same (Welch's).

We will use the information that does \emph{not} assume the standard deviations in the two groups are the same. In the software output in Fig.~\ref{fig:SnakesSummaryTestjamovi}, this is the second row of the top table (labelled `Welch's~\(t\)'). (The information in both rows are often similar anyway.)

\end{importantBox}

From the output, the \(95\)\%~CI for the difference is from~\(3.51\) to~\(13.28\,\text{cm}\). The \emph{approximate} CI and the \emph{exact} (from software) CIs are only slightly different, as the samples sizes are not too small. (Recall: the \(t\)-multiplier of~\(2\) is an approximation, based on the \(68\)--\(95\)--\(99.7\) rule.) \index{Confidence intervals!comparing two means|)}

\section{\texorpdfstring{Hypothesis tests for \(\mu_1 - \mu_2\): \(t\)-test}{Hypothesis tests for \textbackslash mu\_1 - \textbackslash mu\_2: t-test}}\label{hypothesis-tests-for-mu_1---mu_2-t-test}

\index{Hypothesis testing!comparing two means|(}

A hypothesis test can be used to decide if the SVL is different for the two regions. The parameter for the test is \(\mu_N - \mu_C\).

As always, the null hypothesis is the default `no difference, no change, no relationship' position; any difference between the parameter and statistic is due to sampling variation (Sect.~\ref{AboutHypotheses}). Hence, the null hypothesis is `no difference' between the population mean~SVL of the two groups:

\begin{itemize}
\tightlist
\item
  \(H_0\): \(\mu_N - \mu_C = 0\) (equivalent to \(\mu_N = \mu_C\)).
\end{itemize}

From the RQ, the alternative hypothesis is \emph{two}-tailed:

\begin{itemize}
\tightlist
\item
  \(H_1\): \(\mu_N - \mu_C\ne 0\) (equivalent to \(\mu_N \ne \mu_C\)).
\end{itemize}

The alternative hypothesis proposes that any difference between the \emph{sample} means is because a difference really exists between the \emph{population means}. The alternative hypothesis is two-tailed, based on the RQ.

The difference between the sample mean SVLs in the two groups depends on which one of the many possible samples is randomly obtained, \emph{even if} the difference between the means in the population is zero. The difference between the sample means is \(8.394\,\text{cm}\), but this difference will vary from sample to sample; that is, \emph{sampling variation} exists.

For the SVL~data, the sampling distribution of \(\bar{x}_N - \bar{x}_C\) can be described as (see Def.~\ref{def:DEFSamplingDistributionDiffMeans}):

\begin{itemize}
\tightlist
\item
  an approximate normal distribution,
\item
  centred around the sampling mean whose value is \({\mu_{N}} - {\mu_{C}} = 0\), the difference between the population means (from \(H_0\)),
\item
  with a standard deviation of \(\text{s.e.}(\bar{x}_N - \bar{x}_C) = 2.4368\).
\end{itemize}

\begin{softwareBox}{iconmonstr-laptop-4-240.png}
Most software gives \emph{two} hypothesis test results: one assuming the standard deviations in the two groups are the same, and another \emph{not} assuming the standard deviations in the two groups are the same.

We will use the information that does \emph{not} assume the standard deviations in the two groups are the same. In the software output in Fig.~\ref{fig:SnakesSummaryTestjamovi}, this is the second row of the bottom table (labelled `Welch's~\(t\)'). (The information in both rows are often similar anyway.)

\end{softwareBox}

The observed difference between sample means, relative to what was expected, is found by computing the test statistic; in this case, a \(t\)-score. The software output (Fig.~\ref{fig:SnakesSummaryTestjamovi}) gives the \(t\)-score, but the \(t\)-score can also be computed using the information in Table~\ref{tab:SnakesNumericalTest}:\index{Test statistic!t@$t$-score} \begin{align*}
t
&= 
\frac{\text{sample statistic} - \text{mean of sampling distribution (from $H_0$)}}
{\text{standard deviation of sampling distribution}}\\[6pt]
&= 
\frac{ (\bar{x}_P - \bar{x}_C) - (\mu_P - \mu_C)}
{\text{s.e.}(\bar{x}_P - \bar{x}_C)}
= \frac{8.39 - 0}{2.4368} = 3.44,
\end{align*} as in the software output.

A \(P\)-value determines if the sample statistic is consistent with the assumption (i.e., \(H_0\)). Since the \(t\)-score is large, the \(P\)-value will be small using the \(68\)--\(95\)--\(99.7\) rule\index{68@$68$--$95$--$99.7$ rule} (and less than~\(0.003\)). This is confirmed by the software (Fig.~\ref{fig:SnakesSummaryTestjamovi}): the two-tailed \(P\)-value is~\(0.0011\).

A small \(P\)-value suggests the observations are \emph{inconsistent} with the assumption of no difference (Table~\ref{tab:PvaluesInterpretation}), and the difference between the sample means could \emph{not} be reasonably explained by sampling variation, if \(\mu_N - \mu_C = 0\).

In conclusion, write:

\begin{quote}
Strong evidence exists in the sample (two independent samples \(t = 3.445\); two-tailed \(P = 0.0011\)) that the population mean SVL is different for female snakes living in crayfish regions (mean:~\(34.17\,\text{cm}\); \(n = 35\)) and non-crayfish regions (mean:~\(42.57\,\text{cm}\); \(n = 41\);~\(95\)\%~CI for the difference: \(3.51\) to~\(13.28\,\text{cm}\) longer for those in non-crayfish regions).
\end{quote}

The conclusion contains an \emph{answer to the RQ}, the \emph{evidence} leading to this conclusion (\(t = 3.44\); two-tailed \(P = 0.0011\)), and \emph{sample summary statistics}, including a CI. \index{Hypothesis testing!comparing two means|)}

\section{Statistical validity conditions}\label{ValidityTwoMeans}

\index{Statistical validity (for inference)!comparing two means}

As usual, these results apply under certain conditions. The CI and test for comparing two means is \emph{statistically valid} if \emph{either} of these is true:

\begin{itemize}
\tightlist
\item
  when \emph{both} samples have \(n \ge 25\). (If the distribution of a sample is highly skewed, the sample size for that sample may need to be larger.)
\item
  when one or both groups have \(25\)~or fewer observations, \emph{and} the \emph{populations} corresponding to the groups with samples sizes under~\(25\) have an approximate normal distribution.
\end{itemize}

The sample size of~\(25\) is a rough figure; some books give other values (such as~\(30\)).

This condition ensures that the \emph{distribution of the difference between sample means has an approximate normal distribution} (so that, for example, the \(68\)--\(95\)--\(99.7\) rule can be used). The histograms of the \emph{sample data} can be used to determine if normality of the \emph{populations} seems reasonable. The units of analysis are also assumed to be \emph{independent} (e.g., from a simple random sample).

If the statistical validity conditions are not met, other similar options include using a Mann-Whitney test\index{Mann-Whitney test} \citep{conover2003practical} or using resampling methods \citep{efron2021computer}.

\begin{example}[Statistical validity]
\protect\hypertarget{exm:StatisticalValidityReactionHT}{}\label{exm:StatisticalValidityReactionHT}For the garter-snakes data, both samples sizes exceed~\(25\) (\(41\) and~\(35\)), so the test is statistically valid. The data in each group do not need to be normally distributed, since both sample sizes are larger than~\(25\), and the data are not severely skewed (Fig.~\ref{fig:SnakesErrorbar}, left panel).
\end{example}

\section{Tests for comparing more than two means: ANOVA}\label{CompareManyMeans}

Often, more than two means need to be compared. This requires a different method, called \emph{analysis of variance}\index{Analysis of variance} (or \textsc{anova}). The details are beyond the scope of this book. In this section, a very brief overview of using a one-way \textsc{anova} is given, using an example. Importantly, this example shows that the basic principles of hypothesis testing from Chap.~\ref{MoreAboutTests} still apply.

\textsc{Anova} is a general tool that can be extended beyond just comparing more than two means, and used in many and varied context for the analysis of data.

\begin{example}[ANOVA]
\protect\hypertarget{exm:ANOVA}{}\label{exm:ANOVA}{[}\emph{Dataset}: \texttt{Lime}{]} \citet{data:ForestBiomass2017} studied the foliage biomass of small-leaved lime trees from three sources: coppices (\(C\)); natural (\(N\)); planted \(P\)); see Example~\ref{exm:ErrorBarCharts2}. A boxplot and error bar chart are shown in Fig.~\ref{fig:LimeTreesBoxErrorbar}. A numerical summary is shown in Table~\ref{tab:LimeTreesSummary} (based on the output in Fig.~\ref{fig:LimeTreesjamovi}).

To compare the mean foliage biomass of trees from the three sources, the null hypothesis is `no difference' between the population means: \[
  \text{$H_0$:}\ \mu_C = \mu_N = \mu_P.
\] The alternative hypothesis is that the three means are not all equal. This hypothesis encompasses many possibilities: for example, that the three means are \emph{all} different from each other, or that the first is different from the other two (which are the same). Because the alternative hypothesis encompasses many possibilities, we write: \[
  \text{$H_1$:}\ \text{Not all means are equal.}
\]

\begin{importantBox}{iconmonstr-warning-8-240.png}
For comparing more than two means, the alternative hypothesis \emph{is always two-tailed}.

\end{importantBox}

Performing an \textsc{anova} using software (Fig.~\ref{fig:LimeTreesjamovi}) gives \(P = 0.005\). (The \emph{test statistic} here is an \(F\)-score;\index{Test statistic!F@$F$-score} we don't discuss these further, but the \(F\)-score measures the overall difference between the three means.) The small \(P\)-value in this context means the same as usual (Sect.~\ref{AboutPvalues}): there is persuasive evidence to support the alternative hypothesis (that the three means are \emph{not} all equal).

While we know the means are not all the same, we do not know \emph{which} group means are different from which other group means. One option might be to compare all possible combinations of two groups (i.e., the means of groups~\(C\) and~\(N\); the means of groups~\(C\) and~\(P\); the means of groups~\(N\) and~\(P\)) using three separate two-sample \(t\)-tests. While this approach is possible, it increases the probability of declaring a false positive (i.e., of making a Type~I error; Sect.~\ref{TypeErrors}):\index{Type\ I error} \emph{incorrectly} declaring that a difference exists between two sets of means. The correct approach requires methods beyond this book.
\end{example}

\begin{figure}[hbtp]

{\centering \includegraphics[width=0.55\linewidth]{jamovi/Lime/LimeANOVA} 

}

\caption{Software output for testing hypotheses for the lime-trees data.}\label{fig:LimeTreesjamovi}
\end{figure}

\begin{table}
\centering
\caption{\label{tab:LimeTreesSummary}Foliage biomass of lime trees (in kg) from different origins.}
\centering
\fontsize{8}{10}\selectfont
\begin{tabular}[t]{>{}lcccc}
\toprule
\textbf{ } & \textbf{Mean} & \textbf{Standard deviation} & \textbf{Standard error} & \textbf{$n$}\\
\midrule
\textbf{Coppice} & $2.01$ & $2.29$ & $0.199$ & $133$\\
\textbf{Natural} & $1.48$ & $1.76$ & $0.129$ & $185$\\
\textbf{Planted} & $2.68$ & $3.32$ & $0.406$ & $\phantom{0}67$\\
\bottomrule
\end{tabular}
\end{table}

\section{Example: speed signage}\label{SpeedSignage}

To reduce vehicle speeds on freeway exit ramps, \citet{ma2019impacts} studied the impact of additional signage. At one site (Ningxuan Freeway), speeds were recorded for \(38\)~vehicles \emph{before} the extra signage was added, and then for \(41\) different vehicles \emph{after} the extra signage was added.

The researchers are hoping that the addition of extra signage will \emph{reduce} the mean speed of the vehicles. The RQ is:

\begin{quote}
At this freeway exit, does the mean vehicle speed \emph{reduce} after extra signage is added?
\end{quote}

The data are \emph{not} paired: different vehicles are measured before~(\(B\)) and after~(\(A\)) the extra signage is added. Define \(\mu\) as the mean speed (in km.h\textsuperscript{\(-1\)}) on the exit ramp, and the parameter as \(\mu_B - \mu_A\), the \emph{reduction} in the mean speed.

The data can be summarised (Table~\ref{tab:SignageSummary}) using the software output (Fig.~\ref{fig:SpeedjamoviCI2}), where \[
\text{s.e.}(\bar{x}_B - \bar{x}_A) 
= \sqrt{ \text{s.e.}(\bar{x}_B)^2 + \text{s.e.}(\bar{x}_A)^2} 
= \sqrt{ 2.140^2 + 2.051^2} = 2.965,
\] as in the output table (Row~2). A boxplot of the data is shown in Fig.~\ref{fig:SignageErrorBar} (left panel), and an error bar chart in Fig.~\ref{fig:SignageErrorBar} (right panel).

\begin{figure}[hbtp]

{\centering \includegraphics[width=1\linewidth]{jamovi/Speed/Speed-ALL} 

}

\caption{Software output for the speed data.}\label{fig:SpeedjamoviCI2}
\end{figure}

\begin{table}
\centering
\caption{\label{tab:SignageSummary}The signage data summary (in km.h$^{-1}$).}
\centering
\fontsize{8}{10}\selectfont
\begin{tabular}[t]{lccccc}
\toprule
\textbf{ } & \textbf{Mean} & \textbf{Median} & \textbf{Standard deviation} & \textbf{Standard error} & \textbf{Sample size}\\
\midrule
Before & $98.02$ & $98.2$ & $13.194$ & $2.140$ & $38$\\
After & $92.34$ & $93.9$ & $13.134$ & $2.051$ & $41$\\
\midrule
\em{Speed reduction} & \em{$\phantom{0}5.68$} & \em{} & \em{} & \em{$2.965$} & \em{}\\
\bottomrule
\end{tabular}
\end{table}

\begin{figure}[hbtp]

{\centering \includegraphics[width=0.9\linewidth]{30-CIsTesting-TwoMeans_files/figure-latex/SignageErrorBar-1} 

}

\caption{Boxplot (left) and error bar chart (right) showing the mean speed before and after the addition of extra signage, and the $95$\% CIs. The vertical scales on the two graphs are different.}\label{fig:SignageErrorBar}
\end{figure}

An approximate \(95\)\% CI for the difference between the mean speeds is \[
5.674 \pm (2 \times 2.9642),
\] or from~\(-0.25\) to~\(11.60\,\text{km}\).h\textsuperscript{\(-1\)}. (This is very similar to the \(95\)\% CI shown in Fig.~\ref{fig:SpeedjamoviCI2}.) The negative value is not a negative speed. Since the difference between the means is defined as a \emph{reduction}, this CI means that the \emph{reduction} in the populations mean speed is likely between \(-0.25\) to~\(11.64\,\text{km}\).h\textsuperscript{\(-1\)}. Since a negative reduction is an increase, this is more easily understood as the difference being located between a \(0.25\,\text{km}\).h\textsuperscript{\(-1\)} \emph{increase} before the signage was added to an~\(11.64\,\text{km}\).h\textsuperscript{\(-1\)} \emph{reduction} after the signage was added.

The hypotheses are:

\begin{itemize}
\tightlist
\item
  \(H_0\): \(\mu_B - \mu_A = 0\): there is no difference in the population mean speeds.
\item
  \(H_1\): \(\mu_B - \mu_A > 0\): the population mean speed has \emph{reduced} after the addition of signage.
\end{itemize}

The best estimate of the difference in \emph{population} means is the difference between the \emph{sample} means: \((\bar{x}_B - \bar{x}_A) = 5.68\). Since \(\text{s.e.}(\bar{x}_B - \bar{x}_A) = 2.965\), the \(t\)-score is \[
t
= \frac{(\bar{x}_B - \bar{x}_A) - (\mu_B - \mu_A)}{\text{s.e.}(\bar{x}_B - \bar{x}_{A})}
= \frac{5.674 - 0}{2.9642} = 1.91.
\] using Equation~\eqref{eq:tscore}. (Recall that \(\mu_B - \mu_A = 0\) is initially assumed, from the null hypothesis.)

Remembering that the alternative hypothesis is \emph{one-tailed}, the \(P\)-value (using the \(68\)--\(95\)--\(99.7\) rule) is larger than~\(0.025\), but smaller than~\(0.32\), so making a clear decision is difficult without using software. However, since the \(t\)-score is \emph{just} less than 2, we suspect that the \(P\)-value is likely to be closer to~\(0.025\) than to~\(0.32\).

From software, \(P = 0.0297\) (you cannot be this precise just using the \(68\)--\(95\)--\(99.7\) rule). Using Table~\ref{tab:PvaluesInterpretation}, this \(P\)-value provides moderate evidence of a reduction in mean speeds. We conclude:

\begin{quote}
Moderate evidence exists in the sample (\(t = 1.91\); one-tailed \(P = 0.030\)) that mean speeds have reduced after the addition of extra signage (mean reduction: \(5.67\,\text{km}\).h\textsuperscript{\(-1\)}; \(95\)\%~CI for the difference: \(-0.23\) to~\(11.6\,\text{km}\).h\textsuperscript{\(-1\)}; s.e.: \(2.96\,\text{km}\).h\textsuperscript{\(-1\)}). The before mean speed was \(98.02\,\text{km}\).h\textsuperscript{\(-1\)} (\(n = 38\); standard deviation: \(13.19\,\text{km}\).h\textsuperscript{\(-1\)}); the after mean speed was \(92.34\,\text{km}\).h\textsuperscript{\(-1\)} (\(n = 41\); standard deviation: \(13.13\,\text{km}\).h\textsuperscript{\(-1\)}).
\end{quote}

Whether the mean speed reduction of \(5.67\,\text{km}\).h\textsuperscript{\(-1\)} has \emph{practical importance} is a separate issue.\index{Practical importance} Using the validity conditions, the CI and the test are statistically valid.

\begin{tipBox}{iconmonstr-info-6-240.png}
Remember: the conclusion must make clear \emph{which} mean is larger!

\end{tipBox}

\section{Example: chamomile tea}\label{ChamomileTea-TwoMeans}

(This study was seen in Sect.~\ref{ChamomileTea-Paired}.) \citet{rafraf2015effectiveness} studied patients with Type~2 diabetes mellitus (T2DM). They randomly allocated \(32\)~patients into a control group (who drank hot water), and \(32\) to receive chamomile tea (\citet{rafraf2015effectiveness}).

The total glucose (TG) was measured for each individual in both groups, both before the intervention and after eight weeks on the intervention. Summary data are given in Table~\ref{tab:TGsummaryTable}. Evidence suggests that the chamomile tea group shows a mean reduction in TG (Sect.~\ref{ChamomileTea-Paired}), while the hot-water group shows no evidence of a reduction. That is, there appears to be a difference between the two groups regarding the \emph{change} in TG.\spacex However, the differences between the chamomile-tea and the hot-water groups may be due to the samples selected (i.e., sampling variation), so comparing the changes between the two groups is helpful.

The following relational RQ can be asked:

\begin{quote}
For patients with T2DM, is the mean reduction in TG \emph{greater} for the chamomile tea group compared to the hot water group?
\end{quote}

Notice the RQ is one-tailed; the aim of the study is to determine if the chamomile-tea drinking group performs \emph{better} (i.e., reduces the mean TG) than the control group.

This RQ is comparing two separate groups; specifically, comparing the \emph{differences} between the two groups. This study contains both \emph{within}-individuals comparisons (see Sect.~\ref{ChamomileTea-Paired}) and a \emph{between}-individuals comparison (this section); see Fig.~\ref{fig:TeaSummaryAnnotated}. This is equivalent to treating the \emph{differences} for both groups as the two separate sets of data in the two-sample analysis.

\begin{figure}[hbtp]

{\centering \includegraphics[width=1\linewidth]{OtherImages/ChamomileTea/ChamomileSummary-BetweenWithinGroups} 

}

\caption{The chamomile-tea study has two within-individuals comparisons, and a between-individuals comparison (comparing the differences in each group).}\label{fig:TeaSummaryAnnotated}
\end{figure}

The corresponding hypotheses are:

\[
  \text{$H_0$: $\mu_T - \mu_W = 0$ \qquad and\qquad $H_1$: $\mu_T - \mu_W > 0$}
\]

where~\(\mu\) refers to the mean \emph{reduction} in TG, \(T\) refers to the tea-drinking group, and \(W\) to the hot-water drinking group.

The parameter \(\mu_T - \mu_W\) is estimated by the statistic \(\bar{x}_T - \bar{x}_W = 45.74\,\text{mg}\).dL\textsuperscript{\(-1\)}. The standard error for the statistic was found as \(\text{s.e.}(\bar{x}_T - \bar{x}_W) = 8.42\) (using the information in Table~\ref{tab:TGsummaryTable}). Hence, the test statistic is: \[
  t 
  = \frac{(\mu_T - \mu_W) - (\bar{x}_T - \bar{x}_W)}{\text{s.e.}(\bar{x}_T - \bar{x}_W)}
  = \frac{45.75 - 0}{8.42} 
  = 5.43,
\] which is very large, so the \(P\) value will be very small (using the \(68\)--\(95\)--\(99.7\) rule), and certainly smaller than~\(0.001\).

We write:

\begin{quote}
There is very strong evidence (\(t = 5.43\); one-tailed \(P < 0.001\)) that the mean reduction in TG for the chamomile-tea drinking group (mean reduction: \(36.62\,\text{mg}\).dL\textsuperscript{\(-1\)}) is greater than the mean reduction in TG for the hot-water drinking group (mean reduction: \(-7.12\,\text{mg}\).dL\textsuperscript{\(-1\)}; difference between means: \(45.74\,\text{mg}\).dL\textsuperscript{\(-1\)}; approx. \(95\)\%~CI: \(28.64\) to~\(62.84\,\text{mg}\).dL\textsuperscript{\(-1\)}).
\end{quote}

Again, the sample sizes are larger than~\(25\), so the results are statistically valid.

\section{Chapter summary}\label{Chap30-Summary}

To compute a confidence interval (CI) for the difference between two means, compute the difference between the two sample means, \(\bar{x}_1 - \bar{x}_2\), and identify the sample sizes~\(n_1\) and~\(n_2\). Then compute the standard error, which quantifies how much the value of \(\bar{x}_1 - \bar{x}_2\) varies across all possible samples: \[
\text{s.e.}(\bar{x}_1 - \bar{x}_2)
=
\sqrt{ \text{s.e}(\bar{x}_1) + \text{s.e.}(\bar{x}_2)},
\] where \(\text{s.e.}(\bar{x}_1)\) and \(\text{s.e.}(\bar{x}_2)\) are the standard errors of Groups~\(1\) and~\(2\). The \emph{margin of error} is (multiplier\({}\times{}\)standard error), where the multiplier is~\(2\) for an approximate \(95\)\%~CI (using the \(68\)--\(95\)--\(99.7\) rule). Then the CI is: \[
(\bar{x}_1 - \bar{x}_2) \pm \left( \text{multiplier}\times\text{standard error} \right).
\] The statistical validity conditions should also be checked.

These steps are used to test a hypothesis about a difference between two population means \(\mu_1 - \mu_2\).

\begin{itemize}
\tightlist
\item
  Write the null hypothesis~(\(H_0\)) and the alternative hypothesis~(\(H_1\)); initially \emph{assume} the value of \((\mu_1 - \mu_2)\) in the null hypothesis to be true.
\item
  Describe the \emph{sampling distribution}, which describes what to \emph{expect} from the difference between the sample means based on this assumption: under certain statistical validity conditions, the difference between the sample means vary with:

  \begin{itemize}
  \tightlist
  \item
    an approximate normal distribution,
  \item
    with sampling mean whose value is the value of \((\mu_1 - \mu_2)\) (from \(H_0\)), and
  \item
    having a standard deviation of \(\displaystyle \text{s.e.}(\bar{x}_1 - \bar{x}_2)\).
  \end{itemize}
\item
  Compute the value of the \emph{test statistic}: \[
  t = \frac{ (\bar{x}_1 - \bar{x}_2) - (\mu_1 - \mu_2)}{\text{s.e.}(\bar{x}_1 - \bar{x}_2)},
  \] where \(\mu_1 - \mu_2\) is the hypothesised difference given in the null hypothesis.
\item
  The \(t\)-value is like a \(z\)-score, and so an approximate \emph{\(P\)-value} can be estimated using the \(68\)--\(95\)--\(99.7\) rule, or found using software. Use the \(P\)-value to make a decision, and write a conclusion.
\item
  Check the statistical validity conditions.
\end{itemize}

\textsc{Anova} is used to compare means for more than two groups.

\section{Quick review questions}\label{Chap35-QuickReview}

\citet{lee2016effect} studied iron levels in Koreans with Type~2 diabetes, comparing people on a vegan (\(n = 46\)) and a conventional (\(n = 47\)) diet for \(12\)~weeks. A summary of the data for iron levels are shown in Table~\ref{tab:VeganDiet}.

Are the following statements \emph{true} or \emph{false}?

\begin{enumerate}
\def\labelenumi{\arabic{enumi}.}
\item
  An appropriate graph for displaying the \emph{data} is a boxplot. \tightlist  
\item
  The difference between the means in the population is denoted \(\mu_V - \mu_C\), where~\(V\) represent the vegan diet, and~\(C\) represents the conventional diet.
\item
  The standard error of the difference between the sample means is denoted \(\text{s.e.}(\bar{x}_V) - \text{s.e.}(\bar{x}_C)\).
\item
  An error bar chart displays the variation in the \emph{data}.
\item
  The sample size is missing from the \emph{Difference} row, but the value is \(47 - 46 = 1\).
\item
  The standard deviation is missing from the \emph{Difference} row, but the value is \(0.4\).
\item
  The standard error for the difference cannpt be computed, as not enough information is given.
\item
  The two-tailed \(P\)-value for the comparison is \(P = 0.046\). This means that \emph{no evidence} that the population means are different.
\end{enumerate}

\begin{table}
\centering
\caption{\label{tab:VeganDiet}Comparing the iron levels (mg) for subjects using a vegan or conventional diet for $12$ weeks.}
\centering
\fontsize{8}{10}\selectfont
\begin{tabular}[t]{lccc}
\toprule
\textbf{ } & \textbf{Mean} & \textbf{Standard deviation} & \textbf{$n$}\\
\midrule
Vegan diet & $13.9$ & $2.3$ & $46$\\
Conventional diet & $15.0$ & $2.7$ & $47$\\
\midrule
\em{Difference} & \em{$\phantom{0}1.1$} & \em{} & \em{}\\
\bottomrule
\end{tabular}
\end{table}

\section{Exercises}\label{TestTwoMeansExercises}

\hyperref[Answers]{Answers to odd-numbered exercises} are given at the end of the book.

\captionsetup{font=small}

\begin{exercise}
\protect\hypertarget{exr:TwoSampleDiffsA}{}\label{exr:TwoSampleDiffsA}Suppose researchers are comparing the cell diameter of lymphocytes (a type of white blood cell) and tumour cells. Define the mean diameter of lymphocytes as~\(\mu_L\), and the mean diameter of tumour cells as~\(\mu_T\).

If the difference between the means were defined as \(\mu_L - \mu_T\), what does this \emph{mean}?
\end{exercise}

\begin{exercise}
\protect\hypertarget{exr:TwoSampleDiffsB}{}\label{exr:TwoSampleDiffsB}Suppose researchers are comparing the braking distance of cars using two different types of brake pads (Type~A and Type~B). Define the mean breaking distance for cars with Type~A brake pads as~\(\mu_A\), and mean breaking distance for cars with Type~B brake pads as~\(\mu_B\).

If the difference between the means were defined as \(\mu_B - \mu_A\), what does this \emph{mean}?
\end{exercise}

\begin{exercise}
\protect\hypertarget{exr:TwoMeansCISamplingDistSignage}{}\label{exr:TwoMeansCISamplingDistSignage}Sketch the sampling distribution for the difference between the mean speeds before and after adding extra signage (Sect.~\ref{SpeedSignage}).
\end{exercise}

\begin{exercise}
\protect\hypertarget{exr:TwoMeansCISamplingDistTea}{}\label{exr:TwoMeansCISamplingDistTea}Sketch the sampling distribution for the difference between reduction in mean TG for the tea-drinking and the hot-water drinking group (Sect.~\ref{ChamomileTea-TwoMeans}).
\end{exercise}

\begin{exercise}
\protect\hypertarget{exr:TwoMeansWhales}{}\label{exr:TwoMeansWhales}

\citet{agbayani2020growth} measured (among other variables) the length of gray whales (\emph{Eschrichtius robustus}) at birth. Are female gray whales longer than males, on average, in the population at birth? Summary information is shown in Table~\ref{tab:WhaleInfo}.

\begin{table}
\centering
\caption{\label{tab:WhaleInfo}Numerical summary of length of whales at birth (in m).}
\centering
\fontsize{8}{10}\selectfont
\begin{tabular}[t]{lccc}
\toprule
\textbf{ } & \textbf{Mean} & \textbf{Standard deviation} & \textbf{Sample size}\\
\midrule
Female & $4.66$ & $0.38$ & $26$\\
Male & $4.60$ & $0.30$ & $30$\\
\bottomrule
\end{tabular}
\end{table}

\begin{enumerate}
\def\labelenumi{\arabic{enumi}.}
\tightlist
\item
  Define the parameter, and write down its estimate. Carefully describe what it means.
\item
  Sketch an error bar chart.
\item
  Compute the standard error of the difference between the two means.
\item
  Compile a numerical summary table.
\item
  Compute the approximate \(95\)\%~CI.
\item
  Write the hypotheses to answer the RQ.
\item
  Compute the \(t\)-score, and approximate the \(P\)-value using the \(68\)--\(95\)--\(99.7\) rule.
\item
  Write a conclusion.
\item
  Are the CI and test statistically valid?
\end{enumerate}

\end{exercise}

\begin{exercise}
\protect\hypertarget{exr:TwoMeansNHANES}{}\label{exr:TwoMeansNHANES}

{[}\emph{Dataset}: \texttt{NHANES}{]} Earlier, the \textsc{nhanes} study (Exercise~\ref{exr:CompareQuantExercisesNHANES}) was used to summarise the data used to answer this RQ:

\begin{quote}
Among Americans, is the mean direct HDL cholesterol (in mmol.L\textsuperscript{\(-1\)}) different for current smokers and non-smokers?
\end{quote}

Use the software output in Fig.~\ref{fig:NHANESTest} to answer these questions.

\begin{enumerate}
\def\labelenumi{\arabic{enumi}.}
\tightlist
\item
  Define the parameter of interest, and write down its estimate. Carefully describe what it means.
\item
  Sketch an error bar chart.
\item
  Compile a numerical summary table.
\item
  Compute the approximate \(95\)\%~CI, and write a conclusion.
\item
  Write down the exact \(95\)\%~CI, and write a conclusion.
\item
  Write the hypotheses to answer the RQ.
\item
  Write down the standard error of the difference.
\item
  Write down the \(t\)-score and the \(P\)-value.
\item
  Write a conclusion.
\item
  Are the CI and test statistically valid?
\item
  Is the difference between the means likely to be of practical importance?
\end{enumerate}

\end{exercise}



\begin{figure}[hbtp]

{\centering \includegraphics[width=0.95\linewidth]{jamovi/NHANES/NHANES-DirectHDL-Smoke-ALL} 

}

\caption{Software output for the \textsc{nhanes} data.}\label{fig:NHANESTest}
\end{figure}

\begin{exercise}
\protect\hypertarget{exr:MeansIndSamplesExercisesEchinacea}{}\label{exr:MeansIndSamplesExercisesEchinacea}

\citet{data:barrett:echinacea} studied the effectiveness of echinacea to treat the common cold, and compared the mean duration of the cold for participants treated with echinacea or a placebo\index{Placebo} to determine if using echinacea \emph{reduced} the mean duration of symptoms. Participants were blinded to the treatment, and allocated to the groups randomly. A summary of the data is given in Table~\ref{tab:Echinacea}.

\begin{enumerate}
\def\labelenumi{\arabic{enumi}.}
\tightlist
\item
  What is the parameter? Carefully describe what it means.
\item
  Compute the standard error for the mean duration of symptoms for each group.
\item
  Compute the standard error for the difference between the means.
\item
  Sketch an error bar chart.
\item
  Compute an approximate \(95\)\%~CI for the \emph{difference} between the mean durations for the two groups.
\item
  Compute an approximate \(95\)\%~CI for the population mean duration of symptoms for those treated with echinacea.
\item
  Write the hypotheses to answer the RQ.
\item
  Compute the standard error of the difference.
\item
  Compute the \(t\)-score, and approximate the \(P\)-value using the normal distribution tables.
\item
  Write a conclusion.
\item
  Are the CI and test statistically valid?
\item
  Are the results likely to be of practical importance?
\end{enumerate}

\end{exercise}

\begin{table}
\centering
\caption{\label{tab:Echinacea}Numerical summary of duration (in days) of common cold symptoms, for blinded patients taking echinacea or a placebo.}
\centering
\fontsize{8}{10}\selectfont
\begin{tabular}[t]{lcccc}
\toprule
\textbf{ } & \textbf{Mean} & \textbf{Standard deviation} & \textbf{Standard error} & \textbf{Sample size}\\
\midrule
Placebo & $6.87$ & $3.62$ &  & $176$\\
Echinacea & $6.34$ & $3.31$ &  & $183$\\
\midrule
\em{Difference} & \em{$0.53$} & \em{} & \em{} & \em{}\\
\bottomrule
\end{tabular}
\end{table}

\begin{exercise}
\protect\hypertarget{exr:MeansIndSamplesExercisesCarpalTunnelSyndrome}{}\label{exr:MeansIndSamplesExercisesCarpalTunnelSyndrome}

Carpal tunnel syndrome (CTS) is pain experienced in the wrists. \citet{data:Schmid2012:splinting} compared two different treatments: night splinting, or gliding exercises.

Participants were \emph{randomly allocated} to one of the two groups. Pain intensity (measured using a quantitative visual analogue scale; \emph{larger} values mean \emph{greater} pain) were recorded after one week of treatment. The data are summarised in Table~\ref{tab:CarpalTunnel}.

\begin{enumerate}
\def\labelenumi{\arabic{enumi}.}
\tightlist
\item
  What is the parameter? Carefully describe what it means.
\item
  Compute the standard error for the mean pain intensity for each group.
\item
  Compute the standard error for the difference between the mean of the two groups.
\item
  Sketch an error bar chart.
\item
  Compute an approximate \(95\)\%~CI for the \emph{difference} in mean pain intensity for the treatments.
\item
  Compute an approximate \(95\)\%~CI for the population mean pain intensity for those treated with splinting.
\item
  Write the hypotheses to answer the RQ.
\item
  Compute the \(t\)-score, and approximate the \(P\)-value using the \(68\)--\(95\)--\(99.7\) rule.
\item
  Write a conclusion.
\item
  Are the CI and test statistically valid?
\end{enumerate}

\end{exercise}

\begin{table}
\centering
\caption{\label{tab:CarpalTunnel}Numerical summary of pain intensity for two different treatments of carpal tunnel syndrome.}
\centering
\fontsize{8}{10}\selectfont
\begin{tabular}[t]{lcccc}
\toprule
\textbf{ } & \textbf{Mean} & \textbf{Standard deviation} & \textbf{Standard error} & \textbf{Sample size}\\
\midrule
Exercise & $0.8$ & $1.4$ &  & $10$\\
Splinting & $1.1$ & $1.1$ &  & $10$\\
\midrule
\em{Difference} & \em{$0.3$} & \em{} & \em{} & \em{}\\
\bottomrule
\end{tabular}
\end{table}

\begin{exercise}
\protect\hypertarget{exr:TwoMeansDental}{}\label{exr:TwoMeansDental}

{[}\emph{Dataset}: \texttt{Dental}{]} \citet{data:woodward:dental} recorded the sugar consumption in industrialised (mean: \(41.8\,\text{kg}\)/person/y) and non-industrialised (mean: \(24.6\,\text{kg}\)/person/y) countries. The software output is shown in Fig.~\ref{fig:Dentaljamovi}.

\begin{enumerate}
\def\labelenumi{\arabic{enumi}.}
\tightlist
\item
  What is the parameter? Carefully describe what it means.
\item
  Write the hypotheses.
\item
  Using the software output (Fig.~\ref{fig:Dentaljamovi}), write down and interpret the CI.
\item
  Write a conclusion for the hypothesis test.
\item
  Is the test statistically valid?
\end{enumerate}

\end{exercise}

\begin{figure}[hbtp]

{\centering \includegraphics[width=0.95\linewidth]{jamovi/Dental/WoodwardWalker1994-ttest} 

}

\caption{Software output for the sugar-consumption data; the Groups refer to whether the country is industrialised (Yes) or not (No).}\label{fig:Dentaljamovi}
\end{figure}

\begin{exercise}
\protect\hypertarget{exr:Deceleration}{}\label{exr:Deceleration}{[}\emph{Dataset}: \texttt{Deceleration}{]} To reduce vehicle speeds on freeway exit ramps, \citet{ma2019impacts} studied using additional signage. At one site studied (Ningxuan Freeway), speeds were recorded at various points on the freeway exit for vehicles \emph{before} the extra signage was added, and then for different vehicles \emph{after} the extra signage was added.

In addition, the \emph{deceleration} of each vehicle was determined (Table~\ref{tab:SignageSummaryData}) as the vehicle left the \(120\,\text{km}\).h\textsuperscript{\(-1\)} speed zone and approached the \(80\,\text{km}\).h\textsuperscript{\(-1\)} speed zone. Use the data, and the summary in Table~\ref{tab:SignageSummaryDecHT}, to test the RQ:

\begin{quote}
At this freeway exit, is the mean vehicle deceleration the same before extra signage is added and after extra signage is added?
\end{quote}

Identify clearly the parameter of interest to understand how much the deceleration \emph{increased} after adding the extra signage. Remember to compute and interpret the CI for this parameter.
\end{exercise}

\begin{table}
\centering
\caption{\label{tab:SignageSummaryDecHT}The signage deceleration data summary (in m.s$^{-2}$).}
\centering
\fontsize{8}{10}\selectfont
\begin{tabular}[t]{lcccc}
\toprule
\textbf{ } & \textbf{Mean} & \textbf{Standard deviation} & \textbf{Standard error} & \textbf{Sample size}\\
\midrule
Before & $\phantom{0}0.0745$ & $0.0494$ & $0.00802$ & $38$\\
After & $\phantom{0}0.0765$ & $0.0521$ & $0.00814$ & $41$\\
\midrule
\em{Change} & \em{$\phantom{0}\llap{$-{}$}0.0020$} & \em{} & \em{$0.01143$} & \em{}\\
\bottomrule
\end{tabular}
\end{table}

\begin{exercise}
\protect\hypertarget{exr:FacePlant}{}\label{exr:FacePlant}

{[}\emph{Dataset}: \texttt{ForwardFall}{]} A study \citep{data:Wojcik:ForwardFall} compared the lean-forward angle in younger and older women (Table~\ref{tab:FacePlant}). An elaborate set-up was constructed to measure this lean-forward angle, using harnesses. Consider this RQ:

\begin{quote}
Among healthy women, is the mean lean-forward angle \emph{greater} for younger women compared to older women?
\end{quote}

Use the software output (Fig.~\ref{fig:FallFowardTTestTestjamovi}) to answer these questions:

\begin{enumerate}
\def\labelenumi{\arabic{enumi}.}
\tightlist
\item
  What is the parameter? Carefully describe what it means.
\item
  What is an appropriate graph to display the \emph{data}?
\item
  Construct an appropriate numerical summary from the software output (Fig.~\ref{fig:FallFowardTTestjamovi}).
\item
  Construct \emph{approximate} and \emph{exact} \(95\)\%~CIs. Explain any differences.
\item
  Is the test one- or two-tailed?
\item
  Write the statistical hypothesis.
\item
  Use the software output to conduct the hypothesis test.
\item
  Write a conclusion.
\item
  Are the CI and test statistically valid?
\end{enumerate}

\end{exercise}

\begin{figure}[hbtp]

{\centering \includegraphics[width=0.95\linewidth]{jamovi/FallForward/FallForwardTTestOutput-All} 

}

\caption{Software output for the face-plant data.}\label{fig:FallFowardTTestTestjamovi}
\end{figure}

\begin{exercise}
\protect\hypertarget{exr:BHADP}{}\label{exr:BHADP}

\citet{data:Becker1991:BHADP} compared the access to health promotion (HP) services for people with and without a disability in southwestern of the USA.\spacex `Access' was measured using the quantitative \emph{Barriers to Health Promoting Activities for Disabled Persons} (\textsc{bhadp}) scale. \emph{Higher} scores mean \emph{greater} barriers to health promotion services. The RQ is:

\begin{quote}
Is there a difference between the mean \textsc{bhadp} scores, for people with and without a disability, in southwestern USA?
\end{quote}

\begin{enumerate}
\def\labelenumi{\arabic{enumi}.}
\tightlist
\item
  What is the parameter? Carefully describe what it means.
\item
  Sketch an error bar chart.
\item
  Compute the standard error of the difference.
\item
  Compile a numerical summary table.
\item
  Compute the approximate \(95\)\%~CI, and write a conclusion.
\item
  Write down the hypotheses.
\item
  Compute the \(t\)-score.
\item
  Determine the \(P\)-value.
\item
  Write a conclusion.
\item
  Are the CI and test statistically valid?
\end{enumerate}

\end{exercise}

\begin{table}
\centering
\caption{\label{tab:BHADPSummary}The data summary for \textsc{bhadp} scores (no measurement units).}
\centering
\fontsize{8}{10}\selectfont
\begin{tabular}[t]{lcccc}
\toprule
\textbf{ } & \textbf{Sample mean} & \textbf{Standard deviation} & \textbf{Sample size} & \textbf{Standard error}\\
\midrule
Disability & $31.83$ & $7.73$ & $132$ & $0.67280$\\
No disability & $25.07$ & $4.80$ & $137$ & $0.41010$\\
\midrule
\em{Difference} & \em{$\phantom{0}6.76$} & \em{} & \em{} & \em{}\\
\bottomrule
\end{tabular}
\end{table}

\begin{exercise}
\protect\hypertarget{exr:TestTwoMeansBodyTemperature}{}\label{exr:TestTwoMeansBodyTemperature}{[}\emph{Dataset}: \texttt{BodyTemp}{]} Consider again the body temperature data from Sect.~\ref{BodyTemperature}. The researchers also recorded the gender of the patients, as they also wanted to compare the mean internal body temperatures for females and males.

Use the software output in Fig.~\ref{fig:BodyTempGenderjamoviHT} to perform this test and to construct an approximate \(95\)\%~CI appropriate for answering the RQ. Comment on the practical significance of your results.
\end{exercise}

\begin{figure}[hbtp]

{\centering \includegraphics[width=1\linewidth]{jamovi/BodyTemp/BodyTempTestGender} 

}

\caption{Software output for the body-temperature data.}\label{fig:BodyTempGenderjamoviHT}
\end{figure}

\begin{exercise}
\protect\hypertarget{exr:TestTwoMeansFitnessOfParamedics}{}\label{exr:TestTwoMeansFitnessOfParamedics}

\citet{data:chapman2007:MaleParamedics} compared `conventional' male paramedics in Western Australia with male `special-operations' paramedics. Some information comparing their physical profiles is shown in Table~\ref{tab:ParamedicsTest}.

\begin{enumerate}
\def\labelenumi{\arabic{enumi}.}
\tightlist
\item
  Compute the missing standard errors.
\item
  Compare the mean grip strength for the two groups of paramedics. (The \emph{standard error for the difference between the means} is \(3.30\).)
\item
  Compare the mean number of push-ups completed in one minute for the two groups of paramedics. (The \emph{standard error for the difference between the means} is \(4.0689\).)
\end{enumerate}

\end{exercise}

\begin{table}
\centering
\caption{\label{tab:ParamedicsTest}The physical profile of conventional ($n = 18$) and special operation ($n = 11$) paramedics in Western Australia.}
\centering
\fontsize{8}{10}\selectfont
\begin{tabular}[t]{lcc}
\toprule
\textbf{ } & \textbf{Conventional} & \textbf{Special Operations}\\
\midrule
\addlinespace[0.3em]
\multicolumn{3}{l}{\textbf{Grip strength (in kg)}}\\
\hspace{1em}Mean & $51$ & $56$\\
\hspace{1em}Standard deviation & $\phantom{0}8$ & $\phantom{0}9$\\
\hspace{1em}Standard error &  \vphantom{1} & \\
\addlinespace[0.3em]
\multicolumn{3}{l}{\textbf{Push-ups (per minutes)}}\\
\hspace{1em}Mean & $36$ & $47$\\
\hspace{1em}Standard deviation & $10$ & $11$\\
\hspace{1em}Standard error &  & \\
\bottomrule
\end{tabular}
\end{table}

\begin{exercise}
\protect\hypertarget{exr:TwoMeansCIExercisesAnorexia}{}\label{exr:TwoMeansCIExercisesAnorexia}

{[}\emph{Dataset}: \texttt{Anorexia}{]} Young girls (\(n = 29\)) with anorexia received cognitive behavioural treatment (\citet{data:hand:handbook}), while another \(n = 26\) young girls received a control treatment (the `standard' treatment). All girls had their weight recorded before and after treatment.

\begin{enumerate}
\def\labelenumi{\arabic{enumi}.}
\tightlist
\item
  Determine the mean \emph{gain} for individual girls using software.
\item
  Compute a CI for the mean weight gain for the girls in each group.
\item
  Compute a CI for the difference between the mean weight gains for the two treatment groups.
\item
  Conduct a test to determine if there is a difference between the mean weight gains for the two treatment groups.
\end{enumerate}

\end{exercise}

\begin{exercise}
\protect\hypertarget{exr:Coeliac}{}\label{exr:Coeliac}

Researchers studied the impact of a gluten-free diet on dental cavities \citep{khalaf2020caries}. Some summary information regarding the number decayed, missing and filled teeth (DMFT) is shown in Table~\ref{tab:CoeliacDMFTCI}. An \emph{exact} \(95\)\%~CI is given as for the difference is~\(-2.32\) to~\(2.76\).

\begin{enumerate}
\def\labelenumi{\arabic{enumi}.}
\item
  Using the \(68\)--\(95\)--\(99.7\) rule gives a slightly different CI.\spacex Why?
\item
  True or false: the difference is computed as the number of DMFT for coeliacs minus non-coeliacs. \tightlist 
\item
  True or false: one of the values for the CI is a negative value, which must be an error (as a negative number of DMFT is impossible).
\item
  We are \(95\)\%~confident that the difference between the population means is:

  \begin{itemize}
  \tightlist
  \item
    smaller for coeliacs;
  \item
    between~\(2.32\) higher for non-coeliacs to~\(2.76\) higher for coeliacs.
  \item
    between~\(2.76\) higher for non-coeliacs to~\(2.32\) higher for coeliacs.
  \end{itemize}
\end{enumerate}

\end{exercise}

\begin{table}
\centering
\caption{\label{tab:CoeliacDMFTCI}The summary of the number of DMFT for coeliacs and non-coeliacs.}
\centering
\fontsize{8}{10}\selectfont
\begin{tabular}[t]{lcccc}
\toprule
\textbf{ } & \textbf{Sample size} & \textbf{Mean} & \textbf{Standard deviation} & \textbf{Standard error}\\
\midrule
Coeliacs & $23$ & $8.39$ & $4.4$ & $0.92$\\
Non-coeliacs & $23$ & $8.17$ & $4.1$ & $0.86$\\
\midrule
\em{Difference} & \em{} & \em{$0.22$} & \em{} & \em{$1.30$}\\
\bottomrule
\end{tabular}
\end{table}

\begin{exercise}
\protect\hypertarget{exr:TwoMeansReactionTimes}{}\label{exr:TwoMeansReactionTimes}

{[}\emph{Dataset}: \texttt{ReactionTime}{]} \citet{data:Strayer2001:phones} examined the reaction times, while driving, for students from the University of Utah \citep{agresti2007statistics}. In one study, students were randomly allocated to one of two groups: one group \emph{used} a mobile phone while driving in a driving simulator, and one group \emph{did not use} a mobile phone while driving in a driving simulator. The reaction time for each student was measured. The data are shown in Table~\ref{tab:PhoneDataTable}.

Use the data to answer this RQ:

\begin{quote}
For students, what is the difference between the mean reaction time while driving when using a mobile phone and when \emph{not} using a mobile phone?
\end{quote}

\end{exercise}

\begin{table} \centering \centering\caption{\label{tab:PhoneDataTable}Reaction times (in milliseconds) for students using, and not using, mobile phones while driving.}

\fontsize{8}{10}\selectfont
\begin{tabular}[t]{ccccccc}
\toprule
\multicolumn{7}{c}{\textbf{Reaction time: using phone}} \\
\cmidrule(l{3pt}r{3pt}){1-7}
$636$ & $600$ & $609$ & $554$ & $578$ & $688$ & $527$\\
$623$ & $542$ & $559$ & $626$ & $560$ & $679$ & $536$\\
$615$ & $554$ & $595$ & $501$ & $525$ & $960$ & \\
$672$ & $543$ & $565$ & $574$ & $647$ & $558$ & \\
$601$ & $520$ & $573$ & $468$ & $456$ & $482$ & \\
\bottomrule
\end{tabular} \quad\quad 
\begin{tabular}[t]{ccccccc}
\toprule
\multicolumn{7}{c}{\textbf{Reaction time: not using phone}} \\
\cmidrule(l{3pt}r{3pt}){1-7}
$557$ & $506$ & $626$ & $436$ & $617$ & $539$ & $512$\\
$572$ & $648$ & $626$ & $642$ & $528$ & $523$ & $449$\\
$457$ & $485$ & $426$ & $476$ & $578$ & $479$ & \\
$489$ & $610$ & $585$ & $586$ & $472$ & $535$ & \\
$532$ & $444$ & $487$ & $565$ & $485$ & $603$ & \\
\bottomrule
\end{tabular}
\end{table}

\begin{exercise}
\protect\hypertarget{exr:TwoMeansCIExercisesBMIanova}{}\label{exr:TwoMeansCIExercisesBMIanova}

{[}\emph{Dataset}: \texttt{BMI}{]} \citet{johnson2021association} collected data from hospital outpatients at an Irish hospital. One RQ in the study concerns comparing the mean number of days per week that patients exercise for more than \(30\,\text{mins}\) (say, \(\mu\)) according to their smoking status: daily~(\(D\)), occasionally~(\(O\)) or not at all~(\(N\)).

Use the output (Fig.~\ref{fig:BMIjamovi}) to answer the questions that follow.

\begin{enumerate}
\def\labelenumi{\arabic{enumi}.}
\tightlist
\item
  Construct an error bar chart to summarise the data.
\item
  Construct a numerical summary table.
\item
  Perform a suitable hypothesis test, and answer the RQ,
\end{enumerate}

\end{exercise}

\begin{figure}[hbtp]

{\centering \includegraphics[width=0.55\linewidth]{jamovi/BMI/BMIANOVA} 

}

\caption{Software output for testing hypotheses for the BMI data.}\label{fig:BMIjamovi}
\end{figure}

\begin{EOCanswerBox}{iconmonstr-check-mark-14-240.png}
\textbf{Answers to \emph{Quick review} questions}: \textbf{1.} True. \textbf{2.} True. \textbf{3.} False: \(\text{s.e.}(\bar{x}_C - \bar{x}_V)\). \textbf{4.} False: variation for the sample means. \textbf{5.} False: sample size makes no sense. \textbf{6.} False: standard deviation makes no sense \textbf{7.} False: \(0.5197\). \textbf{8.} False: \emph{slight} evidence population means are different.

\end{EOCanswerBox}

\chapter{Comparing two odds or proportions: CIs and tests}\label{AnalysisOddsRatio}

\begin{cols}
\begin{col}{0.52\textwidth}

\begin{objectivesBox}{iconmonstr-target-4-240.png}
You have learnt to ask an RQ, design a study, classify and summarise the data, construct confidence intervals, and conduct hypothesis tests.
\textbf{In this chapter}, you will learn to:
\begin{itemize}\tightlist
  \item
  identify situations where comparing proportions or odds is appropriate.
  \item
  form confidence intervals for the difference between two proportions.
  \item
  form confidence intervals for odds ratios.
  \item
  conduct hypothesis tests for comparing two proportions.
  \item
  conduct hypothesis tests for comparing two odds.
  \item
  determine whether the conditions for using these methods apply in a given situation.
\end{itemize}
\end{objectivesBox}

\end{col}

\begin{col}{0.03\textwidth}
~
\end{col}

\begin{col}{0.45\textwidth}

\includegraphics[width=0.95\linewidth]{31-CIsTesting-OddsRatios_files/figure-latex/unnamed-chunk-30-1} 
\end{col}
\end{cols}

\section{Introduction: meals on-campus}\label{MealsOnCampus}

\citet{data:Mann12017:UniStudents} examined the relationship between where university students usually ate, and where the student lived, for students from two Canadian universities. The researchers cross-classified the \(n = 183\) students (the units of analysis) according to two \emph{qualitative} variables:

\begin{itemize}
\tightlist
\item
  where the student lived, \emph{with} their parents or \emph{not with} their parents.
\item
  where the student ate most meals, \emph{off}-campus or \emph{on}-campus.
\end{itemize}

Both variables are qualitative, so means are not appropriate for summarising the data. The data can be compiled into a two-way table of counts\index{Tables!two-way} (Table~\ref{tab:MealsDataTable}), also called a \emph{contingency table}.\index{Two-way tables} Both qualitative variables have two levels, so this is a \(2\times 2\) table. Every cell in the \(2\times 2\) table contains different students, so the comparison is \emph{between} individuals.

\begin{importantBox}{iconmonstr-warning-8-240.png}
The study has one sample of students, classified according to two variables (i.e., each student is placed into one of the four cells in the \(2\times 2\) table).

\end{importantBox}

\begin{table}
\centering
\caption{\label{tab:MealsDataTable}Where university students live and eat.}
\centering
\fontsize{8}{10}\selectfont
\begin{tabular}[t]{>{}lcc}
\toprule
\multicolumn{1}{c}{\textbf{ }} & \multicolumn{1}{c}{\textbf{Has most meals}} & \multicolumn{1}{c}{\textbf{Has most meals}} \\
\textbf{ } & \textbf{off-campus} & \textbf{on-campus}\\
\midrule
\textbf{Living with parents} & $\phantom{0}52$ & $\phantom{0}\phantom{0}2$\\
\textbf{Not living with parents} & $105$ & $\phantom{0}24$\\
\bottomrule
\end{tabular}
\end{table}

The \emph{proportion} of students who eat most meals off-campus can be compared between those who live with their parents and those who do \emph{not} live with their parents. Then, the parameter is the difference between the population proportions in each group.

Alternatively, the \emph{odds} of students who eat most meals off-campus can be compared between those who live with their parents and those who do \emph{not} live with their parents. Then, the parameter is the comparison of the odds in both groups, the \emph{odds ratio} (OR);\index{Odds ratio} specifically, the OR of eating most meals off-campus, comparing those living with parents to those not living with parents.

\begin{softwareBox}{iconmonstr-laptop-4-240.png}
The table can be constructed with either variable as the rows. However, software commonly compares \emph{rows}, so it makes sense to place the groups to be compared (i.e., the levels of the explanatory variable) in the rows of the table.

\end{softwareBox}

\section{Summarising data}\label{CIOddsRatiosSummaries}

\index{Odds}\index{Difference between proportions}\index{Odds ratio}

Since two groups are being compared, subscripts are used to distinguish between the two groups; say, Groups~\(1\) and~\(2\) in general (Table~\ref{tab:ORHT}). For this example, we use~\(N\) to refer to students \emph{not} living with their parents, and~\(L\) for students living with their parents.

\begin{table}
\centering
\caption{\label{tab:ORHT}Notation used to distinguish the two independent groups.}
\centering
\fontsize{8}{10}\selectfont
\begin{tabular}[t]{lccc}
\toprule
\textbf{ } & \textbf{Group 1} & \textbf{Group 2} & \textbf{Comparing groups}\\
\midrule
Sample sizes: & $n_1$ & $n_2$ & \\
\addlinespace
Sample odds: & $\text{Odds}_1$ & $\text{Odds}_2$ & $\text{Odds ratio (OR)} = \text{Odds}_1/\text{Odds}_2$\\
\addlinespace
Sample proportions: & $\hat{p}_1$ & $\hat{p}_2$ & $\hat{p}_1 - \hat{p}_2$\\
\addlinespace
Standard errors: & $\displaystyle\text{s.e.}(\hat{p}_1)$ & $\displaystyle\text{s.e.}(\hat{p}_2)$ & $\displaystyle\text{s.e.}(\hat{p}_1 - \hat{p}_2)$\\
\bottomrule
\end{tabular}
\end{table}

The parameter is either a difference between two population proportions, or a population OR. For example, the parameter could be the difference between population proportion of students eating most meals \emph{off}-campus, comparing students living with their parents, to students \emph{not} living with their parents. Alternatively (and equivalently), the parameter could be the population~OR of eating most meals \emph{off}-campus, comparing students living with their parents, to students \emph{not} living with their parents.

\begin{figure}[hbtp]

{\centering \includegraphics[width=0.65\linewidth]{jamovi/UniStudents/UniStudents-Chisq-zTest-All} 

}

\caption{Software output for comparing the odds and proportion of students eating most meals at home, for students living with and not with their parents}\label{fig:EatingSoftware}
\end{figure}

\begin{importantBox}{iconmonstr-warning-8-240.png}
Since software commonly compares \emph{rows} (for example, see the text under the bottom table in Fig.~\ref{fig:EatingSoftware}), it makes sense to place the groups to be compared (i.e., the explanatory variable) in the rows of the table.

Then, the difference between the two proportions are usually calculated as the Row~1 proportion minus the Row~2 proportion. Similarly, the odds then can be interpreted as\index{Odds ratio!interpreting}\index{Software output!comparing two odds (odds ratio)} comparing Column~1 counts to Column~2 counts, and the \emph{odds ratio} as comparing the Row~1 odds to the Row~2 odds.

\end{importantBox}

The RQ and the hypotheses can be written as comparing \emph{proportions} (Sect.~\ref{CompareTwoProportions}), comparing \emph{odds} (Sect.~\ref{CompareTwoOdds}), or about \emph{ORs}. With two qualitative variables, an appropriate numerical summary includes the odds and proportions (or percentages) for the outcome for both comparison groups, and the sample sizes (Table~\ref{tab:EatingNumericalSummary}).

To compare the \emph{proportions}, define the sample proportion of students eating most meals off-campus as~\(\hat{p}\), and write~\(\hat{p}_L\) for the proportion living with parents and~\(\hat{p}_N\) for the proportion \emph{not} living with parents. Then, \[
  \hat{p}_L = \frac{52}{52 + 2} = 0.96296
  \quad\text{and}\quad
  \hat{p}_N = \frac{105}{105 + 24} = 0.813953.
\] The \emph{difference} between the two proportions is \[
  \hat{p}_L - \hat{p}_N = 0.9630 - 0.8140 =  0.1490,
\] (as in the software output: Fig.~\ref{fig:EatingSoftware}). By this definition, the difference is how much greater the proportion eating most meals off-campus is for students \emph{living} with their parents, compared to students \emph{not living} with their parents.

\begin{tipBox}{iconmonstr-info-6-240.png}
Be clear about how differences are defined! Differences could be computed as:

\begin{itemize}
\tightlist
\item
  the proportion eating most meals off-campus for those living with their parents, \emph{minus} the proportion \emph{not} living with their parents. This measures how much greater the proportion is for those living with their parents; or
\item
  the proportion eating most meals off-campus for those \emph{not} living with their parents, \emph{minus} the proportion living with their parents. This measures how much greater the proportion is for those \emph{not} living with their parents.
\end{itemize}

Either is fine, provided you are \emph{consistent}, and \emph{clear} about how the differences are computed. The \emph{meaning} of any conclusions will be the same.

\end{tipBox}

To compare the \emph{odds}, first see that the odds of eating most meals \emph{off-campus} is:

\begin{itemize}
\tightlist
\item
  \(52 \div  2 = 26\) for students \emph{living with their parents} (Row~1 of Table~\ref{tab:MealsDataTable}).
\item
  \(105\div 24 = 4.375\) for students \emph{not living with their parents} (Row~2 of Table~\ref{tab:MealsDataTable}).
\end{itemize}

(Notice the numbers in the \emph{second} column are always on the bottom of the fraction.) So the \emph{OR} of eating most meals \emph{off-campus} (the \emph{first} column), comparing students living with parents to students \emph{not} living with parents (\emph{second} column), is \(26 \div 4.375 = 5.943\) (as in the software output: Fig.~\ref{fig:EatingGraphs}).

The numerical summary (Table~\ref{tab:EatingNumericalSummary}) shows the proportion and odds of eating most meals off-campus, comparing students living at home and those not living at home.

\begin{importantBox}{iconmonstr-warning-8-240.png}
The OR can be interpreted in \emph{either} of these ways (i.e., both are correct):\index{Odds ratio!interpreting}\index{Software output!comparing two odds (odds ratio)}

\begin{itemize}
\tightlist
\item
  the \emph{odds} compare Row~1 counts to Row~2 counts, for both columns. The \emph{OR} then compares the Column~1 odds to the Column~2 odds.
\item
  the \emph{odds} compare Column~1 counts to Column~2 counts. The \emph{OR} then compares the Row~1 odds to the Row~2 odds.
\end{itemize}

Odds and ORs are computed with the \emph{first row} and \emph{first column} values on the \emph{top} of the fraction. Since the explanatory variable is usually in the rows, the second is usually the most useful. In this case, both of the above approaches produces an OR of~\(5.943\).

\end{importantBox}

An appropriate graph is a side-by-side bar chart\index{Graphs!side-by-side bar chart} or a stacked bar chart\index{Graphs!stacked bar chart} (Fig.~\ref{fig:EatingGraphs}). The side-by-side bar is useful for comparing odds. For instance, in the two left-most bars in Fig.~\ref{fig:EatingGraphs} (left panel), the first bar is \(26\)~times as high as the second bar (and \(26\)~is the odds); in the two right-most bars, the first bar is \(4.375\)~times as high as the second bar (and \(4.375\)~is the odds). A stacked bar chart is useful for comparing proportions.

\begin{table}
\centering
\caption{\label{tab:EatingNumericalSummary}The odds and proportion of university students eating most meals off-campus.}
\centering
\fontsize{8}{10}\selectfont
\begin{tabular}[t]{lccc}
\toprule
\multicolumn{1}{c}{\textbf{ }} & \multicolumn{1}{c}{\textbf{Odds having most}} & \multicolumn{1}{c}{\textbf{Proportion having most}} & \multicolumn{1}{c}{\textbf{Sample}} \\
\textbf{ } & \textbf{meals off-campus} & \textbf{meals off-campus} & \textbf{size}\\
\midrule
Living with parents & $26.000$ & $\phantom{0}0.963$ & $\phantom{0}54$\\
Not living with parents & $\phantom{0}4.375$ & $\phantom{0}0.814$ & $129$\\
\midrule
\em{} & \em{\text{\llap{OR: }$\phantom{-}5.943$}} & \em{\text{\llap{Difference: }$\phantom{-}0.149$}} & \em{}\\
\bottomrule
\end{tabular}
\end{table}

\begin{figure}[hbtp]

{\centering \includegraphics[width=0.95\linewidth]{31-CIsTesting-OddsRatios_files/figure-latex/EatingGraphs-1} 

}

\caption{The student-eating data. Left: a side-by-side bar chart. Right: a stacked bar chart.}\label{fig:EatingGraphs}
\end{figure}

\section{Confidence intervals for comparing proportions}\label{CICompareProportions}

\index{Confidence intervals!comparing two proportions|(}\index{Sampling distribution!comparing two proportions}\index{Difference between proportions}

The sample proportions for each group will vary from sample to sample, and the \emph{difference} between the sample proportions will be different for each sample. Hence, the \emph{difference} between the two sample proportions has a sampling distribution and \emph{standard error}. Under certain conditions (Sect.~\ref{ValidityChiSq}), this sampling distribution has a normal distribution.

\begin{definition}[Sampling distribution for the difference between two sample proportions for a CI]
\protect\hypertarget{def:DEFSamplingDistributionDiffProportions}{}\label{def:DEFSamplingDistributionDiffProportions}When constructing a CI, the \emph{sampling distribution of the difference between two sample proportions}~\(\hat{p}_1\) and~\(\hat{p}_2\) is (when the appropriate conditions are met; Sect.~\ref{ValidityChiSq}) described by:

\begin{itemize}
\tightlist
\item
  an approximate normal distribution,
\item
  centred around a sampling mean whose value is \({p_1} - {p_2}\), the difference between the \emph{population} proportions,
\item
  with a standard deviation, called the standard error of the difference between the proportions, of \(\displaystyle\text{s.e.}(\hat{p}_1 - \hat{p}_2)\).
\end{itemize}

The standard error for the difference between the proportions is found using \begin{equation}
  \text{s.e.}(\hat{p}_1 - \hat{p}_2) = \sqrt{ \text{s.e.}(\hat{p}_1)^2 + \text{s.e.}(\hat{p}_2)^2},
  \label{eq:SEtwoproportionsCI}
\end{equation} though this value will often be \emph{given} (e.g., on computer output).
\end{definition}

For the student-eating data, the standard errors of the sample proportions for each group are computed using Equation~\eqref{eq:stderrorphat} as \begin{align*}
  \text{s.e.}(\hat{p}_L) &= \sqrt{\frac{0.962963 \times (1 - 0.962963)}{54}}   = 0.025700, \text{and}\\
  \text{s.e.}(\hat{p}_N) &= \sqrt{\frac{0.8139535\times (1 - 0.8139535)}{129}} = 0.034262.
\end{align*} The standard error of the difference between the proportions is then \[
\text{s.e.}(\hat{p}_L - \hat{p}_N)
= \sqrt{ \text{s.e.}(\hat{p}_L)^2 + \text{s.e.}(\hat{p}_N)^2}
= \sqrt{ 0.025700^2 + 0.034262^2 } = 0.042830.
\]

Thus, the differences between the sample proportions will have:

\begin{itemize}
\tightlist
\item
  an approximate normal distribution,
\item
  centred around the sampling mean whose value is \(p_L - p_N\),
\item
  with a standard deviation of \(\text{s.e.}(\hat{p}_L - \hat{p}_N) = 0.0428295\).
\end{itemize}

The sampling distribution describes how the values of \(\hat{p}_L - \hat{p}_N\) vary from sample to sample. Then, finding a \(95\)\%~CI for the difference between the proportions is similar to the process used previously, since the sampling distribution has an approximate normal distribution: \[
\text{statistic} \pm \big(\text{multiplier} \times\text{s.e.}(\text{statistic})\big).
\] When the statistic is \(\hat{p}_L - \hat{p}_N\), the approximate \(95\)\%~CI is \[
(\hat{p}_L - \hat{p}_N) \pm \big(2 \times \text{s.e.}(\hat{p}_L - \hat{p}_N)\big).
\] So, in this case, the approximate \(95\)\%~CI is \[
0.1490 \pm (2 \times 0.042830),
\] or \(0.149 \pm 0.0857\) after rounding (i.e., from~\(0.0633\) to~\(0.235\)). This approximate~CI is very similar to the (exact) CI from software (Fig.~\ref{fig:EatingGraphs}).\index{Software output!comparing two proportions} We write:

\begin{quote}
The difference between the proportions of students eating most meals at home is~\(0.1490\), higher for those living with their parents (\(0.963\); \(n = 52\)) than those not living with their parents (\(0.814\); \(n = 129\)), with the approximate \(95\)\%~CI from~\(0.0633\) to~\(0.235\).
\end{quote}

The plausible values for the difference between the two population proportions are between~\(0.063\) to~\(0.235\), larger for those living with parents.

\begin{importantBox}{iconmonstr-warning-8-240.png}
Giving the CI alone is insufficient; the \emph{direction} in which the differences were calculated must be given, so readers know which group had the higher proportion.

\end{importantBox}

\index{Confidence intervals!comparing two proportions|)}

\section{\texorpdfstring{Hypothesis tests for comparing proportions: \(z\)-test}{Hypothesis tests for comparing proportions: z-test}}\label{CompareTwoProportions}

\index{Hypothesis testing!comparing two proportions|(}\index{Difference between proportions}

To compare the two proportions using a hypothesis test, the two-tailed RQ is:

\begin{quote}
Is the \emph{population} proportion of students eating most meals off-campus the same for students \emph{living with} their parents and for students \emph{not living with} their parents?
\end{quote}

As usual, the population values are unknown, so the parameter \(p_L - p_N\) is estimated using the statistic \(\hat{p}_L - \hat{p}_N\).

Hypothesis testing always begins by assuming that the null hypothesis is true (Sect.~\ref{HypothesisNull}). In this context, that means assuming that the population proportion of eating most meals off-campus is the same in both groups:

\begin{itemize}
\tightlist
\item
  \(H_0\): \(p_L - p_N = 0\) (equivalent to \(p_L = p_N\)).
\end{itemize}

From the RQ, the alternative hypothesis is \emph{two}-tailed:

\begin{itemize}
\tightlist
\item
  \(H_1\): \(p_L - p_N \ne 0\) (equivalent to \(p_L \ne p_N\)).
\end{itemize}

Because the null hypothesis is assumed to be true, the proportions are assumed to have the same value for both groups. Hence, the data from the two groups can be combined to determine an overall (or common) proportion of students eating most meals off-campus: \begin{equation}
  \hat{p} = \frac{52 + 105}{52 + 105 + 2 + 24} = \frac{157}{183} = 0.85792.
  \label{eq:OverallP}
\end{equation} This is the overall proportion of students eating most meals off-campus, since we assumed no difference between students living with and not with their parents. Effectively, this proportion has been computed by summing the columns in Table~\ref{tab:MealsDataTable} and using this combined data to compute the proportion of students eating most meals off-campus.

\begin{importantBox}{iconmonstr-warning-8-240.png}
As with any hypothesis test, the null hypothesis is assumed to be true. For a test comparing two proportions, that implies the proportion in each group is the same, and so the standard errors are computed using the common (overall) proportion.

\end{importantBox}

The \emph{sample} proportions for the two groups (\(L\) and~\(N\)) will vary from sample to sample and so have a sampling distribution. The standard error of the sample proportion for each sample is computed using this common proportion~\(\hat{p}\), using the same idea as in Equation~\eqref{eq:stderrorphat}: \begin{align*}
  \text{s.e.}(p_L) &= \sqrt{ \frac{\hat{p}\times(1 - \hat{p})}{n_L}} = \sqrt{ \frac{0.85792\times(1 - 0.85792)}{54}} = 0.047511, \text{and}\\
  \text{s.e.}(p_N) &= \sqrt{ \frac{\hat{p}\times(1 - \hat{p})}{n_N}} = \sqrt{ \frac{0.85792\times(1 - 0.85792)}{129}} = 0.030739.
\end{align*}

\begin{importantBox}{iconmonstr-warning-8-240.png}
When computing the standard errors as part of a \emph{hypothesis test}, the common or overall proportion is used to compute the standard errors.

\end{importantBox}

The difference between the two proportions will vary from sample to sample too, and hence have a sampling distribution; under certain conditions (Sect.~\ref{ValidityChiSq}), this sampling distribution will have a normal distribution. The standard error of this sampling distribution for the \emph{difference} between the proportions is \[
  \text{s.e.}(\hat{p}_L - \hat{p}_N) = \sqrt{ \text{s.e.}(\hat{p}_L)^2 +  \text{s.e.}(\hat{p}_N)^2 }
  =
  \sqrt{ 0.047511^2 + 0.030739^2} = 0.056588,
\] which is similar to Equation~\eqref{eq:SEtwoproportionsCI}, except that a common proportion was used to compute \(\text{s.e.}(\hat{p}_L)\) and \(\text{s.e.}(\hat{p}_L)\).

\begin{definition}[Sampling distribution for the difference between two sample proportions for a hypothesis test]
\protect\hypertarget{def:DEFSamplingDistributionDiffProportionsHT}{}\label{def:DEFSamplingDistributionDiffProportionsHT}When conducting a hypothesis test, the \emph{sampling distribution of the difference between two sample proportions}~\(\hat{p}_1\) and~\(\hat{p}_2\) is (when the appropriate conditions are met; Sect.~\ref{ValidityChiSq}) described by:

\begin{itemize}
\tightlist
\item
  an approximate normal distribution,
\item
  centred around a sampling mean whose value is \({p_{1}} - {p_{2}}\), the difference between the \emph{population} proportions (from \(H_0\)),
\item
  with a standard deviation, called the standard error of the difference between the proportions, of \(\displaystyle\text{s.e.}(\hat{p}_1 - \hat{p}_2)\).
\end{itemize}

The standard error for the difference between the proportions is \[
  \text{s.e.}(\hat{p}_1 - \hat{p}_2) = \sqrt{ \text{s.e.}(\hat{p}_1)^2 +  \text{s.e.}(\hat{p}_2)^2 },
\] where \[
  \text{s.e.}(p_1) = \sqrt{ \frac{\hat{p}\times(1 - \hat{p})}{n_1}}
  \quad\text{and}\quad
  \text{s.e.}(p_2) = \sqrt{ \frac{\hat{p}\times(1 - \hat{p})}{n_2}},
\] where~\(\hat{p}\) is the common (overall) sample proportion.
\end{definition}

Since the sampling distribution has an approximate normal distribution, the test statistic is\index{Test statistic!z@$z$-score} \[
  z = \frac{ (\hat{p}_L - \hat{p}_N) - (p_L - p_N) }{\text{s.e.}(\hat{p}_L - \hat{p}_N)} \
  = \frac{ 0.14901 - 0}{0.056588} 
  = 2.633.
\] Since the sampling distribution has an approximate normal distribution, the approximate \(P\)-value can be computed from normal distribution tables (Sect.~\ref{ExactAreasUsingTables}), approximated using the \(68\)--\(95\)--\(99.7\) rule,\index{68@$68$--$95$--$99.7$ rule} or from software output (Fig.~\ref{fig:EatingSoftware}).\index{Software output!comparing two proportions} The two-tailed \(P\)-value reported by software (Fig.~\ref{fig:EatingSoftware}, under the column~\texttt{p}) is indeed small: \(0.008\) to three decimal places.

\begin{importantBox}{iconmonstr-warning-8-240.png}
The test statistic for tests involving proportions is a \(z\)-score and \emph{not} a \(t\)-score.

\end{importantBox}

A small \(P\)-value means strong evidence exists to supporting~\(H_1\): the evidence suggests a difference between the \emph{population} proportions. We write:

\begin{quote}
The \emph{sample} provides strong evidence (\(z = 2.63\); two-tailed \(P = 0.008\)) that the proportion of students in the \emph{population} of having most meals off-campus is different for students living with their parents (proportion: \(0.963\), \(n = 54\)) and students \emph{not} living with their parents (proportion: \(0.814\), \(n = 129\); difference: \(0.149\); approximate \(95\)\%~CI from~\(0.0633\) to~\(0.235\), higher for students living with their parents).
\end{quote}

The conclusion includes three components (Sect.~\ref{WordingConclusion}): the \emph{answer to the RQ}; the \emph{evidence} used to reach that conclusion (`\(z = 2.63\); two-tailed \(P = 0.008\)'); and some \emph{sample summary statistics} (including the \(95\)\%~CI for the difference between proportions). The conclusion makes clear which proportion is higher. \index{Hypothesis testing!comparing two proportions|)}

\section{Confidence intervals for comparing odds (for an odds ratio)}\label{CICompareOdds}

\index{Sampling distribution!odds ratio}\index{Odds ratio}\index{Confidence intervals!comparing two odds (odds ratio)|(}

A CI can be formed for the OR, as well as for the difference between two proportions. Every sample of students is likely to be different, and hence the odds of students eating off campus will vary from sample to sample (in both groups). Hence, the OR varies also from sample to sample. That is, \emph{sampling variation} exists, so the OR has a \emph{sampling distribution}.

However, the sampling distribution of the sample OR does \emph{not have a normal distribution}.\footnote{For those interested (this is \emph{optional}): the \emph{logarithm} of the OR has an approximate normal distribution under certain conditions.} For this reason, the CI for the OR will be taken directly from software output, and the sampling distribution is not discussed.

Software\index{Software output!comparing two odds (odds ratio)} (Fig.~\ref{fig:EatingGraphs}, right panel) gives the sample OR as~\(5.94\), and the (exact) \(95\)\%~CI as~\(1.35\) to~\(26.1\). The value of the OR is the same as the value computed manually.

We write:

\begin{quote}
The odds of students eating most meals off-campus is~\(5.94\), higher for students living with their parents (odds:~\(26.0\); \(n = 54\)) than for students \emph{not} living with their parents (odds:~\(4.38\); \(n = 129\)), with the \(95\)\%~CI from~\(1.35\) to~\(26.1\).
\end{quote}

There is a \(95\)\%~chance that this CI straddles the population OR.\spacex Notice that the \emph{meaning} of the OR is explained in the conclusions: the odds of eating most meals \emph{off}-campus, and comparing students living with parents to \emph{not} living with parents.

\emph{The CI for an OR is not symmetrical}, like the others we have seen;\footnote{For those interested (this is \emph{optional}): this is because the OR has no upper limit, but the lower limit of an OR is zero. The \emph{logarithm} of the limits of the CI form a symmetric interval.} that is, the sample OR of~\(5.94\) is not in the centre of the CI.

\begin{tipBox}{iconmonstr-info-6-240.png}
Interpreting and explaining ORs can be challenging, so care is needed!

\end{tipBox}

\index{Confidence intervals!comparing two odds (odds ratio)|)}

\section{\texorpdfstring{Hypothesis tests for comparing odds: \(\chi\)-test}{Hypothesis tests for comparing odds: \textbackslash chi-test}}\label{CompareTwoOdds}

\index{Odds ratio}\index{Hypothesis testing!odds ratio|(}

\subsection{Hypotheses}\label{TwoOddsHypotheses}

For the \(2\times 2\) table of counts in Table~\ref{tab:MealsDataTable}, odds can be compared rather than proportions:

\begin{quote}
Are the \emph{population odds} of students eating most meals off-campus the same for students \emph{living with} their parents and for students \emph{not living with} their parents?
\end{quote}

If the odds are the same in the two groups, this is equivalent to an OR of one. Hence, the RQ could also be written as

\begin{quote}
Is the \emph{population OR} of eating most meals off-campus, comparing students who live \emph{with their} parents to students \emph{not living with} their parents, equal to one?
\end{quote}

Either way, the \emph{parameter} is the population OR, and the null hypothesis is the `no difference, no change, no relationship' position:

\begin{itemize}
\tightlist
\item
  \(H_0\): The \emph{population}~OR is one, or (equivalently)\\
  \phantom{$H_0$:{}} The \emph{population} odds are the same in each group.
\end{itemize}

This hypothesis proposes that the \emph{sample} odds are not the same in the two groups only due to sampling variation. This is the initial \emph{assumption}. The alternative hypothesis is

\begin{itemize}
\tightlist
\item
  \(H_1\): The \emph{population}~OR is not one, or (equivalently)\\
  \phantom{$H_0$:{}} The \emph{population} odds are \emph{different} in each group.
\end{itemize}

\begin{importantBox}{iconmonstr-warning-8-240.png}
For comparing odds, the alternative hypothesis \emph{is always two-tailed}.

\end{importantBox}

In our example then:

\begin{itemize}
\tightlist
\item
  \(H_0\): The \emph{population} odds of eating most meals off-campus is the \emph{same} for students living with their parents and for students not living with their parents.
\item
  \(H_1\): The \emph{population} odds of eating most meals off-campus is \emph{different} for students living with their parents and for students not living with their parents.
\end{itemize}

As usual, the decision-making process starts by \emph{assuming} the null hypothesis is true: that the \emph{population} OR is one (i.e., the population odds in each group are equal).

\begin{importantBox}{iconmonstr-warning-8-240.png}
For two-way tables, RQs can be framed in terms of ORs, comparing odds, comparing proportions, or (for larger two-way table) using associations (or relationships).

For consistency: if the RQ is about odds, the hypotheses and conclusion should be about the odds; if the RQ is about proportions, the hypotheses and conclusion should be about the proportions; and so on.

\end{importantBox}

\subsection{Finding expected counts}\label{ExpectedValues}

Assuming the null hypothesis is true (which is the initial assumption made) means that the odds are the same in both groups (and the proportions are the same in both groups too). That is, the proportions of students eating most meals off-campus is the same for students \emph{living with} and \emph{not living with} their parents. Let's consider the implication.

From Table~\ref{tab:MealsDataTable}, \(157\)~students out of \(183\) ate most meals off-campus, so that \(157\div 183 = 0.8579\) of students in the entire sample ate most of their meals off-campus (which is the common proportion found in Equation~\eqref{eq:OverallP}).

If the proportion of students who eat most of their meals off-campus is the \emph{same} for those who live with their parents and those who don't, then we'd \emph{expect} \(0.8579\)~of students in \emph{both} groups to be eating most meals off-campus. In other words, the two \emph{conditional} probabilities\index{Probability!conditional} would be the same. In that case:

\begin{itemize}
\tightlist
\item
  we would \emph{expect} a proportion of~\(0.8579\) of the \(54\)~students who \emph{live with their parents} (i.e., \(0.8579\times 54 = 46.33\) students) to eat most meals off-campus.
\item
  we would \emph{expect} a proportion of~\(0.8579\) of the \(129\)~students who \emph{don't live with their parents} (i.e., \(0.8579\times 129 = 110.67\) students) to eat most meals off-campus.
\end{itemize}

In other words, the proportions (and hence the odds) of eating most meals off-campus is the same in each group. Those are the \emph{expected} counts\index{Expected counts} if the proportions (or odds) were exactly the same in each group (Table~\ref{tab:MealsDataTableExpected}), as assumed in~\(H_0\).

How close are the \emph{observed} counts (Table~\ref{tab:MealsDataTable}) to the \emph{expected} counts (Table~\ref{tab:MealsDataTableExpected})? For instance, \(46.33\) of the \(54\)~students who \emph{live with their parents} are \emph{expected} to eat most meals off-campus; yet we observed~\(52\); \(110.67\) of the \(129\)~students who \emph{don't live with their parents} are \emph{expected} to eat most meals off-campus; yet we observed~\(105\).

The observed and expected counts are similar, but not the exactly same. The difference between the observed and expected counts \emph{may} be explained by sampling variation (that is, the null hypothesis explanation).

The hypothesis test effectively compares the observed counts to the expected counts (assuming no relationship between the variables) over the whole \(2\times 2\) table.

\begin{importantBox}{iconmonstr-warning-8-240.png}
You \emph{do not} have to compute the expected counts\index{Expected counts} explicitly (software does it in the background, or explicitly if requested). However, seeing how the decision-making process works in this context is helpful.

\end{importantBox}

In previous hypothesis tests, the \emph{sampling distribution} had an approximate normal distribution. However, the sampling distribution of the OR is more complicated\footnote{For those interested: the \emph{logarithm} of the sample ORs have an approximate normal distribution, and hence a \emph{standard error}.} so will not be presented. We will use software output only to conduct the test.

\begin{table}
\centering
\caption{\label{tab:MealsDataTableExpected}Where university students live and eat: expected counts.}
\centering
\fontsize{8}{10}\selectfont
\begin{tabular}[t]{>{}lcc>{}c}
\toprule
\textbf{ } & \textbf{Most off-campus} & \textbf{Most on-campus} & \textbf{Total}\\
\midrule
\textbf{Living with parents} & $\phantom{0}46.328$ & $\phantom{0}\phantom{0}7.672$ & \textbf{$\phantom{0}54$}\\
\textbf{Not living with parents} & $110.672$ & $\phantom{0}18.328$ & \textbf{$129$}\\
\midrule
\textbf{\textbf{Total}} & \textbf{$157.000$} & \textbf{$\phantom{0}26.000$} & \textbf{\textbf{$183$}}\\
\bottomrule
\end{tabular}
\end{table}

\subsection{Computing the value of the test statistic}\label{TestStatObs}

The decision-making process compares what is \emph{expected} if the null hypothesis about the parameter is true (Table~\ref{tab:MealsDataTableExpected}) to what is \emph{observed} in the sample (Table~\ref{tab:MealsDataTable}). Previously, when the sampling distribution was a normal distribution, the test statistic was a \(t\)-score or a \(z\)-score. However, the sampling distribution for an OR does \emph{not} have a normal distribution, and so a different test statistic is needed.

In this context, the test-statistic is `chi-squared', written \(\chi^2\).\index{Test statistic!$\chi^2$-score} The \(\chi^2\)-score measures the overall size of the differences between the expected counts\index{Expected counts} and observed counts, over the entire \(2\times 2\) table.

\begin{pronounceBox}{iconmonstr-microphone-7-240.png}

The Greek letter \(\chi\) is pronounced `kie', as in \textbf{ki}te (\emph{not} `chi' as in \textbf{Chi}na or in \textbf{chi}n). The test statistic \(\chi^2\) is pronounced as `chi-squared'.

\end{pronounceBox}

From the software (Fig.~\ref{fig:EatingSoftware}), \(\chi^2 = 6.934\). But what does this value \emph{mean}? Is it `large' or `small'? The \(\chi^2\)-value, for \(2\times 2\) tables of counts, has an equivalent \(z\)-score, so that a \(P\)-value can be estimated using the \(68\)--\(95\)--\(99.7\) rule. The \(\chi^2\)-value is equivalent to \[
  z = \sqrt{\chi^2}\qquad\text{for a $2\times 2$ table of counts only}.
\] Here then, the \(\chi^2\)-value is equivalent to a \(z\)-score of \(\sqrt{6.934} = 2.633\). This is the \emph{same} \(z\)-score produced when comparing two proportions (Sec.~\ref{CompareTwoProportions}; Fig.~\ref{fig:EatingSoftware}), and hence the \(P\)-value will be the same also. Using the \(68\)--\(95\)--\(99.7\) rule, a small \(P\)-value is expected. The two-tailed \(P\)-value reported by software (Fig.~\ref{fig:EatingSoftware}, under the column~\texttt{p}) is indeed small: \(0.008\) to three decimals.

\begin{importantBox}{iconmonstr-warning-8-240.png}
Recall that \(\chi^2\)-tests always have \emph{two-tailed} alternative hypotheses, so two-tailed \(P\)-values are always reported.

\end{importantBox}

\subsection{Writing conclusions}\label{WritingConclusionChi2}

A very small \(P\)-value (\(0.008\) to three decimals) means strong evidence exists to supporting~\(H_1\): the evidence suggests a difference in the \emph{population} odds in the two groups. We write:

\begin{quote}
The \emph{sample} provides strong evidence (\(\chi^2 = 6.934\), \(n = 54\); two-tailed \(P = 0.008\)) that the odds in the \emph{population} of having most meals off-campus is different for students living with their parents (odds:~\(26\)) and students \emph{not} living with their parents (odds:~\(4.375\), \(n = 129\); OR: \(5.94\); \(95\)\%~CI from~\(1.35\) to~\(26.1\)).
\end{quote}

The conclusion includes three components (Sect.~\ref{WordingConclusion}): the \emph{answer to the RQ}; the \emph{evidence} used to reach that conclusion (`\(\chi^2 = 6.934\); two-tailed \(P = 0.008\)'); and some \emph{sample summary statistics} (including the \(95\)\%~CI for the OR).

The conclusion makes clear what the odds and the OR \emph{mean}. The odds are described as the `odds of having most meals off-campus', and the OR as then comparing these odds between `students living with their parents and students \emph{not} living with their parents'. \index{Hypothesis testing!odds ratio|)}

\section{Statistical validity conditions}\label{ValidityChiSq}

\index{Statistical validity (for inference)!odds ratio}

As usual, these results hold under certain conditions. The CIs and tests above are statistically valid if:

\begin{itemize}
\tightlist
\item
  all \emph{expected} counts are at least five.
\end{itemize}

Some books may give other (but similar) conditions.

The statistical validity condition refers to the \emph{expected} (not the \emph{observed}) counts. In some software, the \emph{expected} counts must be explicitly requested to see if this condition is satisfied (Fig.~\ref{fig:UniMealsTestExpectedjamovi}). The units of analysis are also assumed to be \emph{independent} (e.g., from a simple random sample).

If the statistical validity conditions are not met, other similar options include using a Fisher's exact test\index{Fisher's exact test} \citep{conover2003practical} or using resampling methods \citep{efron2021computer}.

\begin{figure}[hbtp]

{\centering \includegraphics[width=0.7\linewidth]{jamovi/UniStudents/UniStudents-Expected} 

}

\caption{The expected counts, as computed by software.}\label{fig:UniMealsTestExpectedjamovi}
\end{figure}

\begin{example}[Statistical validity]
\protect\hypertarget{exm:StatisticalValidityEatingHT}{}\label{exm:StatisticalValidityEatingHT}For the student-eating data, the smallest \emph{observed} count is~\(2\) (living with parents; most meals off-campus), but the smallest \emph{expected} count (see Table~\ref{tab:MealsDataTableExpected} or Fig.~\ref{fig:UniMealsTestExpectedjamovi}) is~\(7.67\), which is greater than five. This means the two analyses (comparing proportions; comparing odds) are both statistically valid. The size of the \emph{expected} counts is important for the statistical validity condition.
\end{example}

Usually, you do not compute these expected counts. However, a quick check for the statistical validity is to compute the \emph{smallest} expected counts, using \begin{equation}
  \frac{(\text{Smallest row total})\times(\text{Smallest column total})}{\text{Overall total}}.
  \label{eq:SmallestExpectedCount}
\end{equation} If this value is greater than five, the CIs and tests are statistically valid.

\section{\texorpdfstring{Hypothesis tests of independence more generally: \(\chi^2\)-tests}{Hypothesis tests of independence more generally: \textbackslash chi\^{}2-tests}}\label{CompareManyProportions}

Often a table of counts is larger than \(2\times 2\). In these situations, the RQ may not be able to be worded in terms of comparing proportions or odds. Instead, the hypotheses can be worded in terms of \emph{independence}, \emph{relationships} or \emph{associations} (but \emph{not} correlations) between the variables:

\begin{quote}
Is there a relationship (or association) between one qualitative variable and another qualitative variable?
\end{quote}

The RQ is answered using a \(\chi^2\)-test, by extending the ideas in Sect.~\ref{CompareTwoOdds}; \emph{\(z\)-tests and \(t\)-tests are not appropriate.}

\begin{example}[Two-way tables larger than $2\times 2$]
\protect\hypertarget{exm:ChiSqLarger}{}\label{exm:ChiSqLarger}

{[}\emph{Dataset}: \texttt{RipsID}{]} \citet{diez2023rip} studied Spanish people's knowledge of ocean rips (Table~\ref{tab:RipTableSummary}, left table). The table is a \(4\times 2\) two-way table. The rows are the age groups, as the age groups are being compared. The RQ is

\begin{quote}
Is there a relationship (or association) between age group and people's ability to correctly identify a rip?
\end{quote}

\end{example}

\begin{table} \centering \centering\caption{\label{tab:RipTableSummary}Identifying rips. Left: the data by age group. Right: a summary table. The ORs are relative to the $51$ to $65$ age group.}

\fontsize{8}{10}\selectfont
\begin{tabular}{lcc}
\toprule
\multicolumn{1}{c}{\textbf{ }} & \multicolumn{2}{c}{\textbf{Identifying rips}} \\
\cmidrule(l{3pt}r{3pt}){2-3}
\textbf{ } & \textbf{Correctly} & \textbf{Incorrectly}\\
\midrule
18 to 24 & $\phantom{0}41$ & $\phantom{0}5$\\
25 to 34 & $\phantom{0}47$ & $12$\\
35 to 50 & $106$ & $19$\\
51 to 65 & $\phantom{0}52$ & $\phantom{0}7$\\
\bottomrule
\end{tabular} \quad\quad 
\begin{tabular}{lcccc}
\toprule
\multicolumn{1}{c}{\textbf{ }} & \multicolumn{3}{c}{\textbf{Correctly identifying rips}} & \multicolumn{1}{c}{\textbf{ }} \\
\cmidrule(l{3pt}r{3pt}){2-4}
\textbf{ } & \textbf{Odds} & \textbf{OR} & \textbf{Percentage} & \textbf{$n$}\\
\midrule
18 to 24 & $8.200$ & $1.104$ & $89.1$ & $\phantom{0}46$\\
25 to 34 & $3.917$ & $0.527$ & $79.7$ & $\phantom{0}59$\\
35 to 50 & $5.579$ & $0.751$ & $84.8$ & $125$\\
51 to 65 & $7.429$ &  & $88.1$ & $\phantom{0}59$\\
\bottomrule
\end{tabular}
\end{table}

The odds and percentage of people in each age group that can correctly identify rips can be computed (Table~\ref{tab:RipTableSummary}, right table), but this is not always possible (e.g., for a \(3\times 4\) table). ORs compare \emph{pairs} of odds, and the ORs in Table~\ref{tab:RipTableSummary} (right table) are all relative to those in the~\(51\) to~\(65\) age group (hence, no OR is given for the \(51\) to~\(65\) age group, which is the \emph{reference level}).\index{Reference level} For example, the odds of someone aged~\(18\) to~\(24\) correctly identifying a rip is~\(1.104\) times the odds of someone aged~\(51\) to~\(65\) correctly identifying a rip.

For tables larger than \(2\times 2\) more generally, the hypothesis are usually worded in terms of associations or relationships (but \emph{not} correlations) between the variables:

\begin{itemize}
\tightlist
\item
  \(H_0\): In the \emph{population}, there \emph{is no association} between correctly identifying a rip and age group.
\item
  \(H_1\): In the \emph{population}, there \emph{is an association} between correctly identifying a rip and age group.
\end{itemize}

The test statistic is a \(\chi^2\)-value, which compares the observed and expected counts;\index{Expected counts} the expected counts are found in the same way as in Sect.~\ref{ExpectedValues}.

For two-way tables larger than \(2\times 2\), the parameter describing the association between the variables is the \(\chi^2\)-value. When no relationship exists in the sample, the observed and expected counts are the same, and \(\chi^2 = 0\). The larger the difference between the observed and expected counts, the larger the value of~\(\chi^2\). Sampling variation means that the observed counts will vary from sample to sample, so that~\(\chi^2\) may not be exactly zero, even if there is no association between the variables.

Software computes \(\chi^2 = 2.406\), and the two-tailed \(P\)-value as \(P = 0.492\) (Fig.~\ref{fig:Ripsjamovi}, left panel). This \(P\)-value means there is not persuasive evidence to support the alternative hypothesis:

\begin{quote}
The \emph{sample} provides no evidence (\(\chi^2 = 2.406\), \(n = 289\); two-tailed \(P = 0.492\)) of an association between age group and the ability to correctly identify a rip among Spanish people.
\end{quote}

\begin{importantBox}{iconmonstr-warning-8-240.png}
For hypothesis tests involving tables of counts larger than \(2\times 2\), the alternative hypothesis \emph{is always two-tailed}.

\end{importantBox}

\begin{figure}[hbtp]

{\centering \includegraphics[width=0.43\linewidth]{jamovi/RipsID/RipsID-chisq} \includegraphics[width=0.52\linewidth]{jamovi/RipsID/RipsID-Expected} 

}

\caption{Software output for the hypothesis test about knowledge of ocean rips.}\label{fig:Ripsjamovi}
\end{figure}

The statistical validity conditions are the same as in Sect.~\ref{ValidityChiSq}:\index{Statistical validity (for inference)!odds ratio} all \emph{expected} counts are at least five. Using Equation~\eqref{eq:SmallestExpectedCount}, \[
  \frac{(\text{Smallest row total})\times(\text{Smallest column total})}{\text{Overall total}}
  =
  \frac{46\times 43}{289} = 6.84
\] (as in Fig.~\ref{fig:Ripsjamovi}, right panel), which is larger than five. The test is statistically valid.

\section{Example: turtle nests}\label{TurtleNests}

The hatching success of loggerhead turtles on Mediterranean beaches is often compromised by fungi and bacteria. \citet{candan2021first} studied the odds of a nest being infected, comparing relocated nests (relocated due to the risk of tidal inundation), and non-relocated nests (Table~\ref{tab:TurtleNestDataTable}, left table). The researchers were interested in knowing:

\begin{quote}
For Mediterranean loggerhead turtles, are the odds of infections the same for natural and relocated nests?
\end{quote}

\begin{table} \centering \centering\caption{\label{tab:TurtleNestDataTable}The turtles data (left), and the numerical summary (right).}

\fontsize{8}{10}\selectfont
\begin{tabular}[t]{lcc}
\toprule
\multicolumn{1}{c}{\textbf{ }} & \multicolumn{1}{c}{\textbf{Not}} & \multicolumn{1}{c}{\textbf{ }} \\
\textbf{ } & \textbf{infected} & \textbf{Infected}\\
\midrule
Natural & $29$ & $10$\\
Relocated & $14$ & $\phantom{0}8$\\
\bottomrule
\end{tabular} \qquad 
\begin{tabular}[t]{lccc}
\toprule
\multicolumn{1}{c}{\textbf{ }} & \multicolumn{1}{c}{\textbf{Odds}} & \multicolumn{1}{c}{\textbf{Proportion}} & \multicolumn{1}{c}{\textbf{Sample}} \\
\textbf{ } & \textbf{infected} & \textbf{infected} & \textbf{size}\\
\midrule
\textbf{Natural} & $2.90$ & $0.744$ & $39$\\
\textbf{Relocated} & $1.75$ & $0.636$ & $22$\\
\midrule
\em{} & \em{\llap{OR: }$1.66$} & \em{\llap{Diff.: }$0.107$} & \em{}\\
\bottomrule
\end{tabular}
\end{table}

\begin{figure}[hbtp]

{\centering \includegraphics[width=0.5\linewidth]{31-CIsTesting-OddsRatios_files/figure-latex/TurtleNestsGraphs-1} 

}

\caption{Bar chart for the turtle-nesting data.}\label{fig:TurtleNestsGraphs}
\end{figure}

Since the RQ is written in terms of odds, the hypotheses should be written using odds also:

\begin{itemize}
\tightlist
\item
  \(H_0\): The odds of a nest being infected is \emph{the same} for natural and relocated nests.
\item
  \(H_1\): The odds of a nest being infected is \emph{not the same} for natural and relocated nests.
\end{itemize}

Here, \(N\) refers to \textbf{N}atural nests, and~\(R\) to \textbf{R}elocated nests. The parameter is the odds ratio of a nest being infected, comparing natural to relocated nests. (The equivalent hypotheses written in terms of proportions would be \(H_0\): \(p_N - p_R = 0\) and \(H_1\): \(p_N - p_R \ne 0\). The hypotheses could also be written in terms of associations.)

A graphical summary is shown in Fig.~\ref{fig:TurtleNestsGraphs}. A numerical summary table (Table~\ref{tab:TurtleNestDataTable}, right table) shows that the odds of natural nest being infected is \(1.66\)~times the odds of a relocated nest being infected. From the software output (Fig.~\ref{fig:TurtleNestsOutputjamovi}), the \(\chi^2\)-value is~\(0.777\). Since the table is a \(2\times 2\) table, the equivalent \(z\)-score can be found: \(z = \sqrt{0.777} = 0.88\). This \(z\)-scorw is very small, so expect a large \(P\)-value. (This is the value of the \(z\)-score shown in Fig.~\ref{fig:TurtleNestsOutputjamovi} for comparing two proportions.) The \(P\)-value is~\(0.378\) on the output (for both tests).

The smallest \emph{expected} count is \((22\times 18) / 61 = 6.49\), which exceeds five, so the test is statistically valid. Since the RQ and hypotheses were written in terms of odds, the conclusion is also written in terms of odds:

\begin{quote}
There is no evidence of a difference in the odds of infection (\(\chi^2\): \(0.777\); \(P\)-value: \(0.378\); OR: \(1.657\); \(95\)\%~CI: \(0.537\) to~\(5.12\)) between natural nests (odds: \(2.90\); \(n = 39\)) and relocated nests (odds: \(1.75\); \(n = 22\)).
\end{quote}

There is no evidence that relocating the nest (to protect them from tidal inundation) changes the risk of infection.

\begin{importantBox}{iconmonstr-warning-8-240.png}
We \emph{do not} say whether the evidence supports the null hypothesis. We assume the null hypothesis is true, so we state the strength of evidence to change our mind (and hence support the alternative hypothesis). The current sample presents no evidence to contradict the assumption, but future evidence may emerge.

\end{importantBox}

\begin{figure}[hbtp]

{\centering \includegraphics[width=0.6\linewidth]{jamovi/TurtleNests/TurtleNests-Chisq-pDiff} 

}

\caption{The software output for the turtle-nesting data.}\label{fig:TurtleNestsOutputjamovi}
\end{figure}

\section{Example: health of female burros}\label{HTCompareOddsBurros}

\citet{johnson1987demography} studied \(315\)~introduced female burros (donkeys) in the Mojave Desert (California) to understand management processes. One RQ was:

\begin{quote}
For these female burros, is the reproductive status of the burros related to their health?
\end{quote}

The data (Table~\ref{tab:BurrosData}, left table) are given in a \(3\times 3\) table of counts. The data are summarised using row proportions in Table~\ref{tab:BurrosData} (right table), and in a graph in Fig.~\ref{fig:BurrosChisqjamoviPlot} (left panel). Software output is shown in Fig.~\ref{fig:BurrosChisqjamoviPlot} (right panel).

\begin{table} \centering \centering\caption{\label{tab:BurrosData}Left: the health and reproductive status of female burros. Right: row proportions for the burro data (i.e., rows sum to one). Pregnant and lactating burros were counted with the lactating burros only.}

\fontsize{8}{10}\selectfont
\begin{tabular}{lcccc}
\toprule
\multicolumn{1}{c}{\textbf{ }} & \multicolumn{3}{c}{\textbf{Health: counts}} & \multicolumn{1}{c}{\textbf{ }} \\
\cmidrule(l{3pt}r{3pt}){2-4}
\textbf{ } & \textbf{Excellent} & \textbf{Fair} & \textbf{Poor} & \textbf{Total}\\
\midrule
Barren & $\phantom{0}16$ & $\phantom{0}21$ & $\phantom{0}38$ & $\phantom{0}75$\\
Pregnant & $\phantom{0}14$ & $\phantom{0}53$ & $\phantom{0}62$ & $129$\\
Lactating & $\phantom{0}\phantom{0}4$ & $\phantom{0}29$ & $\phantom{0}78$ & $111$\\
\bottomrule
\end{tabular} \quad\quad 
\begin{tabular}{lccc}
\toprule
\multicolumn{1}{c}{\textbf{ }} & \multicolumn{3}{c}{\textbf{Health: row proportions}} \\
\cmidrule(l{3pt}r{3pt}){2-4}
\textbf{ } & \textbf{Excellent} & \textbf{Fair} & \textbf{Poor}\\
\midrule
Barren & $0.213$ & $0.280$ & $0.507$\\
Pregnant & $0.109$ & $0.411$ & $0.481$\\
Lactating & $0.036$ & $0.261$ & $0.703$\\
\bottomrule
\end{tabular}
\end{table}

The hypotheses must be worded in terms of associations (or \emph{relationships}):

\begin{itemize}
\tightlist
\item
  \(H_0\): \emph{No association} exists between reproductive status and overall health.
\item
  \(H_1\): \emph{An association} exists between reproductive status and overall health.
\end{itemize}

From the software output (Fig.~\ref{fig:BurrosChisqjamoviPlot}, right panel), \(\chi^2 = 23.585\). Notice that a comparison of proportions is not possible for tables larger than \(2\times 2\). Software reports \(P < 0.001\), which suggests very strong evidence in the sample that an association exists between reproductive status and overall health.

\begin{figure}[hbtp]

{\centering \includegraphics[width=0.44\linewidth]{31-CIsTesting-OddsRatios_files/figure-latex/BurrosChisqjamoviPlot-1} \includegraphics[width=0.55\linewidth]{jamovi/Burros/BurrosChisquareTest} 

}

\caption{Left: a stacked bar chart for the burro-health data. Right: software output for the burro-health data.}\label{fig:BurrosChisqjamoviPlot}
\end{figure}

The conclusion could be written as

\begin{quote}
The sample provides very strong evidence (\(\chi^2 = 23.585\); \(P < 0.001\); \(3\times 3\) table) of an association between reproductive status and overall health of female burros (\(n = 315\)).
\end{quote}

Adding sample summary information to this conclusion is cumbersome. Instead, readers can be pointed to the numerical summary (Table~\ref{tab:BurrosData}, right table). Furthermore, CIs are not reported.

While we know there is an association between the variables, we can only speculate on the nature of the association (i.e., for which group(s) the \emph{population} proportions are different). Formal methods for doing so requires methods beyond this book, but Fig.~\ref{fig:BurrosChisqjamoviPlot} (left panel) suggests that lactating burros are far more likely to have poor health.

The smallest \emph{expected} value is \(75\times 34/315 = 8.1\), which exceeds~\(5\), so the results are statistically valid.

\section{Chapter summary}\label{TestsOddsRatio-Summary}

To compare a two-level qualitative variable between two groups, a CI can be formed for the difference between two proportions, or for an OR.

To compute a CI for the difference between two proportions, compute the difference between the two sample proportions, \(\hat{p}_1 - \hat{p}_2\), and identify the sample sizes~\(n_1\) and~\(n_2\). Then the standard error, which quantifies how much the value of \(\hat{p}_1 - \hat{p}_2\) varies across all possible samples, is \[
\text{s.e.}(\hat{p}_1 - \hat{p}_2)
=
\sqrt{ \text{s.e}(\hat{p}_1) + \text{s.e.}(\hat{p}_2)},
\] where \(\text{s.e.}(\hat{p}_1)\) and \(\text{s.e.}(\hat{p}_2)\) are the standard errors of Groups~\(1\) and~\(2\) (Equation~\eqref{eq:stderrorphat}). The \emph{margin of error} is (multiplier\({}\times{}\)standard error), where the multiplier is~\(2\) for an approximate \(95\)\%~CI (using the \(68\)--\(95\)--\(99.7\) rule). Then the CI is: \[
(\hat{p}_1 - \hat{p}_2) \pm \left( \text{multiplier}\times\text{standard error} \right).
\] Software is used to compute a CI for the OR, as the sampling distribution does not have a normal distribution.

These steps are used to test a hypothesis about a difference between two population proportions \(p_1 - p_2\).

\begin{itemize}
\tightlist
\item
  Write the null hypothesis~(\(H_0\)) and the alternative hypothesis~(\(H_1\)); initially \emph{assume} the value of \((p_1 - p_2)\) in the null hypothesis to be true.
\item
  Describe the \emph{sampling distribution}, which describes what to \emph{expect} from the difference between the sample proportions based on this assumption: under certain statistical validity conditions, the difference between the sample proportions vary with:

  \begin{itemize}
  \tightlist
  \item
    an approximate normal distribution,
  \item
    with sampling mean whose value is the value of \((p_1 - p_2)\) (from \(H_0\)), and
  \item
    having a standard deviation of \(\displaystyle \text{s.e.}(\hat{p}_1 - \hat{p}_2)\) computed using the \emph{common} proportion.
  \end{itemize}
\item
  Compute the value of the \emph{test statistic}: \[
  z = \frac{ (\hat{p}_1 - \hat{p}_2) - (p_1 - p_2)}{\text{s.e.}(\hat{p}_1 - \hat{p}_2)},
  \] where \(p_1 - p_2\) is the hypothesised difference given in the null hypothesis.
\item
  An approximate \emph{\(P\)-value} can be estimated using the \(68\)--\(95\)--\(99.7\) rule, or an exact \(P\)-value found using software. Use the \(P\)-value to make a decision, and write a conclusion.
\item
  Check the statistical validity conditions.
\end{itemize}

These steps are used to test a hypothesis for comparing two odds, or to test for a relationship between two qualitative variables more generally.

\begin{itemize}
\tightlist
\item
  Write the null hypothesis~(\(H_0\)) and the alternative hypothesis~(\(H_1\)); initially \emph{assume} no relationship between the two variables.
\item
  Find the value of the \emph{test statistic} (a \(\chi^2\)-score) on the software output. (For \(2\times 2\) tables only, the equivalent \(z\)-score is~\(\sqrt{\chi^2}\).)
\item
  A \emph{\(P\)-value} is found using software; use the \(P\)-value to make a decision, and write a conclusion.
\item
  Check the statistical validity conditions.
\end{itemize}

The statistical validity conditions should be checked: all \emph{expected} counts should exceed five.

\section{Quick review questions}\label{TestsOddsRatio-QuickReview}

\citet{chen2023triage} investigated the relationship between body temperature of patients admitted to hospital following a heart attack (HA), and a having a subsequent HA while in hospital (Table~\ref{tab:HAttackData}).

Are the following statements \emph{true} or \emph{false}?

\begin{enumerate}
\def\labelenumi{\arabic{enumi}.}
\item
  From the software output, the \(P\)-value is \(0.180\). \tightlist
\item
  The alternative hypothesis two-tailed.
\item
  There is \emph{no} evidence of a difference in odds of having an in-hospital HA, comparing patients with low and high body temperatures.
\item
  The CI is not statistically valid, because the CIs for the difference between the proportions has a \emph{negative} value.
\item
  The CI means that the sample~OR is likely to be between~\(0.330\) and~\(5.568\).
\item
  The \(\chi^2\)-value is \(0.180\).
\item
  Of patients with a low body temperature, \(4/27 = 0.148\) had an in-hospital HA.
\item
  The \emph{odds} that a patient with a low body temperature had an in-hospital HA is \(4/23 = 0.174\).
\end{enumerate}

Select the correct answer:

\begin{enumerate}
\def\labelenumi{\arabic{enumi}.}
\setcounter{enumi}{8}
\item
  The OR in the output is given as \(1.357\). What does this OR \emph{mean}?

  \begin{enumerate}
  \def\labelenumii{\alph{enumii}.}
  \tightlist
  \item
    The odds of having an in-hospital heart attack is \(1.357\) times greater for those with a low body temperature.
  \item
    The odds of having an in-hospital heart attack is \(1.357\) times smaller for those with a low body temperature.
  \item
    The proportion of patients having an in-hospital heart attack is \(1.357\) times greater for those with a low body temperature.
  \item
    The proportion of patients having an in-hospital heart attack is \(1.357\) times smaller for those with a low body temperature.
  \end{enumerate}
\end{enumerate}

\begin{figure}
\begin{minipage}{0.38\textwidth}
\captionof{table}{Body temperature of patients, and whether they experienced a heart attack in hospital\label{tab:HAttackData}.}
\fontsize{8}{12}\selectfont
\begin{@empty}

\begin{tabular}{lcc}
\toprule
\multicolumn{1}{c}{\textbf{ }} & \multicolumn{2}{c}{\textbf{In-hospital}} \\
\multicolumn{1}{c}{\textbf{ }} & \multicolumn{2}{c}{\textbf{heart attack}} \\
\cmidrule(l{3pt}r{3pt}){2-3}
\textbf{ } & \textbf{Yes} & \textbf{No}\\
\midrule
Low body temp. & $\phantom{0}4$ & $23$\\
High body temp. & $\phantom{0}5$ & $39$\\
\bottomrule
\end{tabular}
\end{@empty}
\end{minipage}
\hspace{0.05\textwidth}
\begin{minipage}{0.54\textwidth}%
\centering

\includegraphics[width=0.95\linewidth]{jamovi/HAttack/HAttack-jamovi} 
\caption{Software output for the heart-attack study.}\label{fig:HAttackjamovi}
\end{minipage}
\end{figure}

\section{Exercises}\label{TestsOddsRatioExercises}

\hyperref[Answers]{Answers to odd-numbered exercises} are given at the end of the book.

\captionsetup{font=small}

\begin{exercise}
\protect\hypertarget{exr:OddsSame}{}\label{exr:OddsSame}Consider the expected counts in Table~\ref{tab:MealsDataTableExpected}. Confirm that the \emph{odds} of having most meals off-campus is the same for students living with their parents, and for students not living with their parents.
\end{exercise}

\begin{exercise}
\protect\hypertarget{exr:OddsSame2}{}\label{exr:OddsSame2}Compute all four expected counts in Table~\ref{tab:HAttackData}. Confirm that the corresponding test may not be statistically valid.
\end{exercise}

\begin{exercise}
\protect\hypertarget{exr:OddsRatioCISamplingDistA}{}\label{exr:OddsRatioCISamplingDistA}Sketch the sampling distribution for the difference between the proportions of students eating most meals off-campus, for those living with parents minus those not living with parents. What is the sampling distribution for the equivalent OR?
\end{exercise}

\begin{exercise}
\protect\hypertarget{exr:OddsRatioCISamplingDistB}{}\label{exr:OddsRatioCISamplingDistB}Sketch the sampling distribution for the difference between the proportion of non-infected turtle nests, for natural nests minus relocated nests (in Sect.~\ref{TurtleNests}). What is the sampling distribution for the equivalent OR?
\end{exercise}

\begin{exercise}
\protect\hypertarget{exr:Chi2z}{}\label{exr:Chi2z}

Suppose an analysis of a \(2\times 2\) table of counts produces a value of \(\chi^2 = 10.66\).

\begin{enumerate}
\def\labelenumi{\arabic{enumi}.}
\tightlist
\item
  What would be the equivalent \(z\)-score for comparing the two proportions?
\item
  What would be the approximate \(P\)-value?
\end{enumerate}

\end{exercise}

\begin{exercise}
\protect\hypertarget{exr:Chi2zA}{}\label{exr:Chi2zA}

Suppose an analysis of a \(2\times 2\) table of counts produces a value of \(\chi^2 = 4.06\).

\begin{enumerate}
\def\labelenumi{\arabic{enumi}.}
\tightlist
\item
  What would be the equivalent \(z\)-score for comparing the two proportions?
\item
  What would be the approximate \(P\)-value?
\end{enumerate}

\end{exercise}

\begin{exercise}
\protect\hypertarget{exr:EVAdoption}{}\label{exr:EVAdoption}

{[}\emph{Dataset}: \texttt{EVPurchase}{]} \citet{egbue2017mass} studied the adoption of electric vehicles (EVs) by a group of professional Americans (Table~\ref{tab:EV10years}). Software output is shown in Fig.~\ref{fig:EVjamovi}.

\begin{enumerate}
\def\labelenumi{\arabic{enumi}.}
\tightlist
\item
  Based on the output, how is the difference between the two proportions defined?
\item
  Write the hypothesis for comparing the \emph{proportions} using this definition of the difference.
\item
  Use the software output to conduct the test.
\item
  Use the software output to write down the corresponding CI for the difference in proportions.
\item
  Based on the output, how is the OR defined?
\item
  Write the hypothesis for comparing the \emph{odds}, for those with and without post-graduate study.
\item
  Use the software output to conduct the test.
\item
  Use the software output to write down the corresponding CI for the OR.
\item
  Are the CIs and tests statistically valid?
\end{enumerate}

\end{exercise}

\begin{figure}
\begin{minipage}{0.32\textwidth}
\captionof{table}{Responses to `Would you purchase an electric vehicle in the next $10$ years?' by education\label{tab:EV10years}.}
\fontsize{8}{12}\selectfont\centering
\begin{@empty}

\begin{tabular}{lcc}
\toprule
\textbf{ } & \textbf{Yes} & \textbf{No}\\
\midrule
No post-grad & $24$ & $\phantom{0}8$\\
Post-grad study & $51$ & $29$\\
\bottomrule
\end{tabular}
\end{@empty}
\end{minipage}
\hspace{0.05\textwidth}
\begin{minipage}{0.60\textwidth}%
\centering

\includegraphics[width=0.98\linewidth]{jamovi/EVs/EVs-Tests-CI-both} 
\caption{Software output for the EV study.}\label{fig:EVjamovi}
\end{minipage}
\end{figure}

\begin{exercise}
\protect\hypertarget{exr:EthiopianFarmers}{}\label{exr:EthiopianFarmers}

\citet{meresa2023effect} investigated Ethiopian farmers' adoption of improved soil and water conservation structures on their farms (Table~\ref{tab:FarmersData}). Software output is shown in Fig.~\ref{fig:farmersjamoviHT}.

\begin{figure}
\begin{minipage}{0.32\textwidth}
\captionof{table}{Adoption of conservation practices by Ethiopian farmers, by farm size\label{tab:FarmersData}.}
\fontsize{8}{12}\selectfont\centering
\begin{@empty}

\begin{tabular}{lcc}
\toprule
\multicolumn{1}{c}{\textbf{ }} & \multicolumn{2}{c}{\textbf{Adopter?}} \\
\cmidrule(l{3pt}r{3pt}){2-3}
\textbf{ } & \textbf{No} & \textbf{Yes}\\
\midrule
$< 0.5$ ha (small) & $86$ & $61$\\
$\ge 0.5$ ha (large) & $43$ & $71$\\
\bottomrule
\end{tabular}
\end{@empty}
\end{minipage}
\hspace{0.05\textwidth}
\begin{minipage}{0.60\textwidth}%
\centering

\includegraphics[width=0.98\linewidth]{jamovi/Farmers/Farmers-Chi2-pDiff-CITest} 
\caption{Software output for the farming study.}\label{fig:farmersjamoviHT}
\end{minipage}
\end{figure}

\begin{enumerate}
\def\labelenumi{\arabic{enumi}.}
\tightlist
\item
  Based on the output, how is the difference between the two proportions defined?
\item
  Write the hypothesis for comparing the proportions, using this definition of the difference.
\item
  Use the software output to conduct the test.
\item
  Use the software output to write down the corresponding CI for the difference in proportions.
\item
  Based on the output, how is the OR defined?
\item
  Write the hypothesis for comparing the odds, for farmers with small and large farms.
\item
  Use the software output to conduct the test.
\item
  Use the software output to write down the corresponding CI for the OR.
\item
  Are the CIs and tests statistically valid?
\end{enumerate}

\end{exercise}

\begin{exercise}
\protect\hypertarget{exr:ORcrashes}{}\label{exr:ORcrashes}

{[}\emph{Dataset}: \texttt{CarCrashes}{]} \citet{wang2020driver} recorded information about car crashes in a rural, mountainous county in western China (Table~\ref{tab:CrashDataTableLATEX}).

\begin{enumerate}
\def\labelenumi{\arabic{enumi}.}
\tightlist
\item
  Sketch a suitable graph to display the data.
\item
  Compute the \emph{proportion} of crashes involving a pedestrian in~2011 (\(\hat{p}_{2011}\)), and in~2015 (\(\hat{p}_{2015}\)).
\item
  Compute the \emph{difference between the proportion} of crashes involving a pedestrian from~2011 to~2015, consistent with the definition used in the output (Fig.~\ref{fig:CarCrashjamovi}).
\item
  Compute the value of \(\text{s.e.}(\hat{p}_{2011} - \hat{p}_{2015})\), needed for constructing the CI.
\item
  Construct the \emph{approximate} \(95\)\% CI for the difference between the proportions.
\item
  Write down a \(95\)\%~CI for the difference between the proportions.
\item
  Interpret what this CI means.
\item
  Compute the \emph{odds} of crashes involving a pedestrian in~2011, and also in ~2015.
\item
  Compute the \emph{OR} of crashes involving a pedestrian, comparing~2011 to~2015.
\item
  Write down the CI for the OR.
\item
  Construct an appropriate numerical summary table for the data.
\item
  Compute the value of \(\text{s.e.}(\hat{p}_{2011} - \hat{p}_{2015})\), needed for conducting a hypothesis test.
\item
  Conduct a hypothesis test to determine if there is a difference between \(p_{2011}\) and \(p_{2015}\).
\item
  Conduct a hypothesis test to determine if there is a difference between the odds of a crash involving a pedestrian for~2011 and~2015.
\item
  Are the CIs and tests statistically valid?
\end{enumerate}

\end{exercise}

\begin{figure}
\begin{minipage}{0.32\textwidth}
\captionof{table}{Type of car crashes in different years\label{tab:CrashDataTableLATEX}.}
\fontsize{8}{12}\selectfont

\begin{tabular}{lcc}
\toprule
\multicolumn{1}{c}{\textbf{ }} & \multicolumn{1}{c}{\textbf{Involving}} & \multicolumn{1}{c}{\textbf{Involving}} \\
\textbf{ } & \textbf{pedestrians} & \textbf{vehicles}\\
\midrule
2011 & $15$ & $35$\\
2015 & $37$ & $85$\\
\bottomrule
\end{tabular}
\end{minipage}
\hspace{0.05\textwidth}
\begin{minipage}{0.60\textwidth}%
\centering

\begin{center}\includegraphics[width=0.99\linewidth]{jamovi/CarCrashes/CarCrashes-ALL} \end{center}
\caption{Software output for the car-crash study.}\label{fig:CarCrashjamovi}
\end{minipage}
\end{figure}

\begin{exercise}
\protect\hypertarget{exr:TestsOddsRatioScars}{}\label{exr:TestsOddsRatioScars}

{[}\emph{Dataset}: \texttt{ScarHeight}{]} \citet{data:Wallace2017:Sunburn} compared the heights of scars from burns received by people in Western Australia (Table~\ref{tab:ScarsData}). Software was used to analyse the data (Fig.~\ref{fig:ScarHeightRiskjamovi}).

\begin{enumerate}
\def\labelenumi{\arabic{enumi}.}
\tightlist
\item
  Sketch an appropriate graph to summarise the data.
\item
  Compute the \emph{proportion} of men having a smooth scar, and the \emph{proportion} of women.
\item
  Compute the \emph{difference between the proportions} of men and women having a smooth scar.
\item
  Compute the standard error for the difference between the proportions, needed for constructing a CI.
\item
  Compute the \emph{approximate} \(95\)\%~CI for the difference between the proportions.
\item
  Write down the \(95\)\%~CI for the difference between the proportions, using the software output.
\item
  Interpret what this CI means.
\item
  Compute the \emph{odds} of having a smooth scar for men, and for women.
\item
  Compute the \emph{OR} of having a smooth scar, comparing men to women.
\item
  Write down the CI for the OR of having a smooth scar, comparing men to women.
\item
  Compile a numerical summary table.
\item
  Compute the value of standard error of the difference between the proportions, needed for conducting a hypothesis test.
\item
  Conduct a hypothesis test to determine if there is a difference between the proportions for men and women.
\item
  Conduct a hypothesis test to determine if there is a difference between the odds for men and women.
\item
  Are the CIs and tests statistically valid?
\end{enumerate}

\end{exercise}

\begin{figure}
\begin{minipage}{0.32\textwidth}
\captionof{table}{Heights of scars for men and women\label{tab:ScarsData}.}
\fontsize{8}{12}\selectfont

\begin{tabular}{lcc}
\toprule
\multicolumn{1}{c}{\textbf{ }} & \multicolumn{1}{c}{\textbf{Smooth}} & \multicolumn{1}{c}{\textbf{Over 0\,mm,}} \\
\textbf{ } & \textbf{(0\,mm)} & \textbf{up to 1\,mm}\\
\midrule
Men & $216$ & $115$\\
Women & $\phantom{0}99$ & $\phantom{0}62$\\
\bottomrule
\end{tabular}
\end{minipage}
\hspace{0.05\textwidth}
\begin{minipage}{0.60\textwidth}%
\centering

\begin{center}\includegraphics[width=0.98\linewidth]{jamovi/ScarHeight/ScarHeight-No-Expected} \end{center}
\caption{Software output for the scar-height data.}\label{fig:ScarHeightRiskjamovi}
\end{minipage}
\end{figure}

\begin{exercise}
\protect\hypertarget{exr:PetBirdsTest}{}\label{exr:PetBirdsTest}

{[}\emph{Dataset}: \texttt{PetBirds}{]} \citet{data:Kohlmeier1992:BirdsCancer} examined people with lung cancer, and a matched set of controls who did not have lung cancer, and recorded the number in each group that kept pet birds. The data are shown in Table~\ref{tab:BirdsData}, and the software output in Fig.~\ref{fig:PetBirdsjamovi}.

Consider this RQ:

\begin{quote}
Are the odds of having a pet bird the same for people \emph{with} lung cancer (cases) and for people \emph{without} lung cancer (controls)?
\end{quote}

\begin{enumerate}
\def\labelenumi{\arabic{enumi}.}
\tightlist
\item
  Compute the difference between the proportions of people with pet birds, for those with and without lung cancer.
\item
  Compute the standard error needed to compute the CI for the difference in proportions.
\item
  Compute the standard error needed to conduct the hypothesis test to compare the proportions.
\item
  Explain \emph{why} the two standard errors have slightly different values.
\item
  Compute an approximate \(95\)\%~CI for the difference between the two proportions.
\item
  Write down the \(95\)\%~CI for the difference between the proportions using the output (Fig.~\ref{fig:PetBirdsjamovi}).
\item
  Interpret the CIs.
\item
  Conduct a hypothesis test to compare the two proportions.
\item
  Confirm that the OR in the output is correct.
\item
  Write down a \(95\)\%~CI for the OR, and interpret what it means.
\item
  Perform a hypothesis test to determine if the odds of having a pet bird is the same for people with and without lung cancer.
\item
  Are the CIs and tests statistically valid?
\item
  Explain why no cause-and-effect can be reached.
\end{enumerate}

\end{exercise}

\begin{table}
\centering
\caption{\label{tab:BirdsData}The pet bird data.}
\centering
\fontsize{8}{10}\selectfont
\begin{tabular}[t]{lcc>{}c}
\toprule
\multicolumn{1}{c}{\textbf{ }} & \multicolumn{1}{c}{\textbf{Adults with}} & \multicolumn{1}{c}{\textbf{Adults without}} \\
\textbf{ } & \textbf{lung cancer} & \textbf{lung cancer} & \textbf{Total}\\
\midrule
Did not keep pet birds & $141$ & $328$ & \textbf{$469$}\\
Kept pet birds & $\phantom{0}98$ & $101$ & \textbf{$199$}\\
\midrule
\textbf{Total} & \textbf{$239$} & \textbf{$429$} & \textbf{\textbf{$668$}}\\
\bottomrule
\end{tabular}
\end{table}

\begin{figure}[hbtp]

{\centering \includegraphics[width=0.495\linewidth]{jamovi/PetBirds/Pets-Test-Both} \includegraphics[width=0.495\linewidth]{jamovi/PetBirds/PetsCI-Both} 

}

\caption{Software output for the pet-birds data.}\label{fig:PetBirdsjamovi}
\end{figure}

\begin{exercise}
\protect\hypertarget{exr:TestsOddsRatioAugustRainfallInEmerald}{}\label{exr:TestsOddsRatioAugustRainfallInEmerald}

{[}\emph{Daatset}: \texttt{EmeraldAug}{]} The \emph{Southern Oscillation Index}~(SOI) is a standardised measure of the air pressure difference between Tahiti and Darwin, and is related to rainfall in some parts of the world \citep{climate:stone:1996}, and especially Queensland \citep{climate:stone:1992}.

The rainfall at Emerald (Queensland) was recorded for Augusts between~1889 and~2002 inclusive \citep{mypapers:dunnsmyth:glms}, where the monthly average~SOI was positive, and when the SOI was non-positive (zero or negative), as shown in Table~\ref{tab:SOItableAnalysis}.

\begin{enumerate}
\def\labelenumi{\arabic{enumi}.}
\tightlist
\item
  Compute the difference between the proportions of Augusts with rain, for months with a positive SOI compared to months with a non-positive SOI.
\item
  Compute the standard error needed to compute the CI for the difference in proportions.
\item
  Compute the standard error needed to conduct the hypothesis to compare the proportions.
\item
  Explain \emph{why} the two standard errors have slightly different values.
\item
  Compute an approximate \(95\)\%~CI for the difference between the two proportions.
\item
  Write down the \(95\)\%~CI for the difference between the proportions using the output (Fig.~\ref{fig:EmeraldRainjamoviOutput}).
\item
  Interpret the CIs.
\item
  Conduct a hypothesis test to compare the two proportions.
\item
  Confirm that the OR in the output is correct.
\item
  Write down a \(95\)\%~CI for the OR, and interpret what it means.
\item
  Perform a hypothesis test to determine if the odds of recoding rain is the same for Augusts with non-positive and positive SOI.
\item
  Are the CIs and tests statistically valid?
\end{enumerate}

\end{exercise}

\begin{table}
\centering
\caption{\label{tab:SOItableAnalysis}The SOI, and whether rainfall was recorded in Augusts between 1889 and 2002 inclusive.}
\centering
\fontsize{8}{10}\selectfont
\begin{tabular}[t]{lcc}
\toprule
\textbf{ } & \textbf{Rainfall recorded} & \textbf{No rainfall recorded}\\
\midrule
Positive SOI & $53$ & $\phantom{0}7$\\
Non-positive SOI & $40$ & $14$\\
\bottomrule
\end{tabular}
\end{table}

\begin{figure}[hbtp]

{\centering \includegraphics[width=0.65\linewidth]{jamovi/EmeraldRain/EmeraldRain-NoE-Both} 

}

\caption{Software output for the Emerald-rain data.}\label{fig:EmeraldRainjamoviOutput}
\end{figure}

\begin{exercise}
\protect\hypertarget{exr:TestOddsRatioSunglasses}{}\label{exr:TestOddsRatioSunglasses}

{[}\emph{Dataset}: \texttt{HatSunglasses}{]} \citet{data:Dexter2019:SunProtection} recorded the number of people at the foot of the Goodwill Bridge, Brisbane, who wore hats between \(11\):\(30\)am to \(12\):\(30\)pm. Of the \(366\)~females observed, \(22\) wore hats; of the \(386\)~males observed, \(79\) wore hats.

\begin{enumerate}
\def\labelenumi{\arabic{enumi}.}
\tightlist
\item
  Construct the two-way table for the data.
\item
  Compute the proportions of females and males wearing a hat, and hence the difference between the proportions.
\item
  Compute the odds of a female and the odds of a male wearing a hat, and hence the OR.
\item
  Compute an approximate \(95\)\%~CI for the difference between the proportions.
\item
  Write down the \(95\)\%~CI for the difference between the proportion (Fig.~\ref{fig:SunglassesOutput}).
\item
  Interpret the CIs.
\item
  Write down, then interpret, a \(95\)\%~CI for the OR.
\item
  Perform a hypothesis test to determine if the odds of wearing a hat is the same for females and males.
\item
  Write down the conclusion.
\item
  Are the CIs and tests statistically valid?
\end{enumerate}

\end{exercise}

\begin{figure}[hbtp]

{\centering \includegraphics[width=0.49\linewidth]{jamovi/HatSunglasses/HatSunglasses-CI-Both} \includegraphics[width=0.5\linewidth]{jamovi/HatSunglasses/HatSunglasses-Test-Both} 

}

\caption{Software output for the hats data.}\label{fig:SunglassesOutput}
\end{figure}

\begin{exercise}
\protect\hypertarget{exr:OddsRatiosCITurbinesHT}{}\label{exr:OddsRatiosCITurbinesHT}

{[}\emph{Dataset}: \texttt{Turbines}{]} A study of turbine failures \citep{MyersBook, NelsonLifeData} ran \(73\)~turbines for around \(1\,800\,\text{h}\), and found that seven developed fissures (small cracks). They also ran a different set of~\(42\) turbines for about~\(3\,000\,\text{h}\), and found that nine developed fissures.

\begin{enumerate}
\def\labelenumi{\arabic{enumi}.}
\tightlist
\item
  Construct the two-way table for the data.
\item
  Compute the difference between the proportions of fissures at \(1\,800\,\text{h}\) and \(3\,000\,\text{h}\), and hence the difference between the proportions.
\item
  Compute the odds of a fissure after \(1\,800\,\text{h}\) and after \(3\,000\,\text{h}\), and hence the OR.
\item
  Compute an approximate \(95\)\%~CI for the difference between the proportions.
\item
  Write down the \(95\)\%~CI for the difference between the proportions (Fig.~\ref{fig:TurbinesOutput}).
\item
  Interpret the CIs.
\item
  Write down, then interpret, a \(95\)\%~CI for the OR.
\item
  Test for a relationship.
\item
  Are the CIs and tests statistically valid?
\end{enumerate}

\end{exercise}

\begin{figure}[hbtp]

{\centering \includegraphics[width=0.6\linewidth]{jamovi/Turbines/Turbines-Both-CITest} 

}

\caption{Software output for the turbine data.}\label{fig:TurbinesOutput}
\end{figure}

\begin{exercise}
\protect\hypertarget{exr:TestOddsRatioBearTree}{}\label{exr:TestOddsRatioBearTree}

\citet{witmer2020preliminary} compared various types of repellents (including bear faeces) to prevent bears damaging trees in an Idaho forest. Part of the data are summarised in (Table~\ref{tab:BearsB12Data}, left table).

\begin{enumerate}
\def\labelenumi{\arabic{enumi}.}
\tightlist
\item
  Compute the odds of new damage for both repellents, and hence the OR.
\item
  Compute the proportion of trees with new damage for both repellents, and hence the difference between the proportions.
\item
  Write the hypothesis for conducting a hypothesis test involving proportions.
\item
  Write the hypothesis for conducting a hypothesis test involving odds.
\item
  Software gives \(\chi^2\) as~\(4.4850\). What is the equivalent \(z\)-score (e.g., for the test of proportions)? Would you expect a large or small \(P\)-value?
\item
  The \(P\)-value, from software, is \(P = 0.0342\). Write a conclusion, either using odds or proportions.
\item
  Is the analysis statistically valid?
\end{enumerate}

\end{exercise}

\begin{exercise}
\protect\hypertarget{exr:B12Deficiency}{}\label{exr:B12Deficiency}

{[}\emph{Dataset}: \texttt{B12Diet}{]} \citet{data:Gammon2012:B12} examined B12~deficiencies in `predominantly overweight/obese women of South Asian origin living in Auckland'. Some women were on a vegetarian diet and some were not (Table~\ref{tab:BearsB12Data}, right table). One RQ was:

\begin{quote}
Among this group of women, are the odds of being vitamin~B12 deficient different for women on a vegetarian diet compared to women on a non-vegetarian diet?
\end{quote}

\begin{enumerate}
\def\labelenumi{\arabic{enumi}.}
\tightlist
\item
  Compute the odds of B12 deficiency for both diets, and hence the OR.
\item
  Compute the proportion of women with B12 deficiency for both diets, and hence the difference between the proportions.
\item
  Write the hypothesis for conducting a hypothesis test involving proportions.
\item
  Write the hypothesis for conducting a hypothesis test involving odds.
\item
  Software gives \(\chi^2\) as~\(4.7067\). What is the equivalent \(z\)-score (e.g., for the test of proportions)? Would you expect a large or small \(P\)-value?
\item
  The \(P\)-value, from software, is \(P = 0.0305\). Write a conclusion, either using odds or proportions.
\item
  Is the analysis statistically valid?
\end{enumerate}

\end{exercise}

\begin{table} \centering \centering\caption{\label{tab:BearsB12Data}Left: the number of trees with new damage by bears, according to different repellents. Right: the number of vegetarian and non-vegetarian women who are (and are not) B12 deficient.}

\fontsize{8}{10}\selectfont
\begin{tabular}[t]{lcc}
\toprule
\multicolumn{1}{c}{\textbf{ \null  }} & \multicolumn{1}{c}{\textbf{New}} & \multicolumn{1}{c}{\textbf{No new}} \\
\textbf{ } & \textbf{damage} & \textbf{damage}\\
\midrule
Bear faeces & $\phantom{0}6$ & $69$\\
Control (water) & $15$ & $60$\\
\bottomrule
\end{tabular} \qquad\qquad 
\begin{tabular}[t]{lcc}
\toprule
\multicolumn{1}{c}{\textbf{ \null  }} & \multicolumn{1}{c}{\textbf{B12}} & \multicolumn{1}{c}{\textbf{Not B12}} \\
\textbf{ } & \textbf{deficient} & \textbf{deficient}\\
\midrule
Vegetarians & $\phantom{0}8$ & $\phantom{0}26$\\
Non-vegetarians & $\phantom{0}8$ & $\phantom{0}82$\\
\bottomrule
\end{tabular}
\end{table}

\captionsetup{font=normalsize}

\begin{exercise}
\protect\hypertarget{exr:DogsHT}{}\label{exr:DogsHT}

{[}\emph{Dataset}: \texttt{DogWalks}{]} \citet{naughton2024association} studied the difference between the activities of dogs kept in the city and on farms (Table~\ref{tab:DogWalkTable}). One RQ was:

\begin{quote}
For Northern Ireland dogs, is there an association between length of walks, and location?
\end{quote}

\begin{enumerate}
\def\labelenumi{\arabic{enumi}.}
\tightlist
\item
  Write down the hypotheses to answer this RQ.
\item
  Perform a hypothesis to answer the RQ, using the output in Fig.~\ref{fig:DogWalksjamovi}.
\item
  Write down the conclusion, in terms of odds, including a CI.
\item
  Write down the conclusion, in terms of proportions, including a CI.
\item
  Is the test statistically valid?
\end{enumerate}

\end{exercise}

\begin{table}
\centering
\caption{\label{tab:DogWalkTable}The length of walks for dogs, living in the city and farms. (`Varies' means usually short walks, but occasional longer walks.)}
\centering
\fontsize{8}{10}\selectfont
\begin{tabular}[t]{lcccc}
\toprule
\multicolumn{1}{c}{\textbf{ }} & \multicolumn{4}{c}{\textbf{Walk length (in mins)}} \\
\cmidrule(l{3pt}r{3pt}){2-5}
\textbf{ } & \textbf{Under $30$} & \textbf{$30$ to under $60$} & \textbf{$60$ to under $120$} & \textbf{Varies}\\
\midrule
City & $138$ & $\phantom{0}84$ & $\phantom{0}13$ & $264$\\
Farm & $\phantom{0}84$ & $102$ & $\phantom{0}33$ & $243$\\
\bottomrule
\end{tabular}
\end{table}

\begin{figure}[hbtp]

{\centering \includegraphics[width=0.45\linewidth]{jamovi/DogWalks/DogWalks-Test} 

}

\caption{Software output for the dog-walking data.}\label{fig:DogWalksjamovi}
\end{figure}

\begin{exercise}
\protect\hypertarget{exr:ComplianceHT}{}\label{exr:ComplianceHT}

{[}\emph{Dataset}: \texttt{Mumps}{]} \citet{soud2009isolation} studied the compliance of students with an isolation request following a large mumps outbreak in Kansas in 2006. One RQ was:

\begin{quote}
Is there an association between age group, and compliance with the isolation order?
\end{quote}

The data are shown in Table~\ref{tab:MumpsTable} and the software output in Fig.~\ref{fig:Mumpsjamovi}.

\begin{enumerate}
\def\labelenumi{\arabic{enumi}.}
\tightlist
\item
  Write down the hypotheses.
\item
  Compute the proportion of each age group that complied with the isolation request.
\item
  Compute the odds of each age group that complied with the isolation request.
\item
  Compute the relevant ORs (using `Older than~\(22\)' as the reference level), and interpret what these mean.
\item
  Determine the \(\chi^2\)-value and perform a hypothesis to answer the RQ.
\item
  Is the test statistically valid?
\end{enumerate}

\end{exercise}

\begin{figure}
\begin{minipage}{0.48\textwidth}
\captionof{table}{The compliance of students by age group\label{tab:MumpsTable}.}
\fontsize{8}{12}\selectfont
\begin{@empty}

\begin{tabular}{lcc}
\toprule
\textbf{ } & \textbf{Complied} & \textbf{Did not comply}\\
\midrule
$18$ to $19$ & $40$ & $10$\\
$20$ to $21$ & $37$ & $14$\\
Older than $22$ & $22$ & $\phantom{0}9$\\
\bottomrule
\end{tabular}
\end{@empty}
\end{minipage}
\hspace{0.05\textwidth}
\begin{minipage}{0.42\textwidth}%
\centering

\includegraphics[width=1\linewidth]{jamovi/Mumps/Mumps-Test} 
\caption{Software output for the compliance data.}\label{fig:Mumpsjamovi}
\end{minipage}%
\end{figure}

\begin{exercise}
\protect\hypertarget{exr:ShoppingBags}{}\label{exr:ShoppingBags}

{[}\emph{Dataset}: \texttt{ShoppingBags}{]} \citet{choon2017perception} studied \(400\)~residents of Klang Valley, Malaysia, to examine residents' approach to waste management. One RQ was:

\begin{quote}
For residents of Klang Valley, is age group associated with whether people bring their own bags when shopping?
\end{quote}

The data (Table~\ref{tab:BagsTableHT}) are given in a \(3\times 2\) table of counts. The software output is shown in Fig.~\ref{fig:BagsChisqjamovi}.

\begin{figure}
\begin{minipage}{0.48\textwidth}
\captionof{table}{Whether shoppers bring their own bags, and the shoppers age group\label{tab:BagsTableHT}.}
\fontsize{8}{12}\selectfont
\begin{@empty}

\begin{tabular}{lcc}
\toprule
\multicolumn{1}{c}{\textbf{ }} & \multicolumn{2}{c}{\textbf{Brings bags?}} \\
\cmidrule(l{3pt}r{3pt}){2-3}
\textbf{ } & \textbf{Yes} & \textbf{No}\\
\midrule
$30$ and under & $126$ & $138$\\
$31$ to $40$ & $\phantom{0}50$ & $\phantom{0}32$\\
Over $40$ & $\phantom{0}41$ & $\phantom{0}13$\\
\bottomrule
\end{tabular}
\end{@empty}
\end{minipage}
\hspace{0.05\textwidth}
\begin{minipage}{0.42\textwidth}%
\centering

\includegraphics[width=1\linewidth]{jamovi/ShoppingBags/ShoppingBags-Test} 
\caption{Software output for the shopping-bags data.}\label{fig:BagsChisqjamovi}
\end{minipage}%
\end{figure}

\begin{enumerate}
\def\labelenumi{\arabic{enumi}.}
\tightlist
\item
  Compute the odds of someone bringing a shopping bag, for each age group.
\item
  Compute the OR of bringing a shopping bag (using the `Over~\(40\)' age group as the reference level).
\item
  Compute the percentage of people bringing a shopping bag, for each age group.
\item
  Construct the hypotheses for testing for an association between the variables.
\item
  Use the software output to answer the research question.
\item
  Write a conclusion.
\item
  Is the test statistically valid.
\end{enumerate}

\end{exercise}

\begin{exercise}
\protect\hypertarget{exr:CrabShells}{}\label{exr:CrabShells}

{[}\emph{Dataset}: \texttt{CrabShells3}{]} Hermit crabs place sea anemones on their shells for protection. \citet{brooks1989hermit} studied the placement of the anemones:

\begin{quote}
Is there a relationship between the vertical and horizontal locations of anemones placed by hermit crabs on their shells?
\end{quote}

The data are shown in Table~\ref{tab:CrabShellData}, and output in Fig.~\ref{fig:CrabShellsChisqjamovi}.

\begin{enumerate}
\def\labelenumi{\arabic{enumi}.}
\tightlist
\item
  Perform a hypothesis test to answer the RQ using the \(3\times 3\) table (Fig.~\ref{fig:CrabShellsChisqjamovi}, top output).
\item
  Confirm that the statistical validity conditions are not met when using the \(3\times 3\) table.
\item
  Construct a \(2\times 2\) table, recording the location of the anemones as either `Central' or `Side' without distinguishing \emph{which} side. Hence, repeat the test using the \(2\times 2\) table (Fig.~\ref{fig:CrabShellsChisqjamovi}, bottom output). (These data are in the file \texttt{CrabShell2}.)
\end{enumerate}

\end{exercise}

\begin{figure}
\begin{minipage}{0.48\textwidth}
\captionof{table}{The location of anemones placed on shells by hermit crabs\label{tab:CrabShellData}.}
\fontsize{8}{12}\selectfont
\begin{@empty}

\begin{tabular}{lccc}
\toprule
\multicolumn{1}{c}{\textbf{ }} & \multicolumn{3}{c}{\textbf{Column}} \\
\cmidrule(l{3pt}r{3pt}){2-4}
\textbf{ } & \textbf{Side 1} & \textbf{Central} & \textbf{Side 2}\\
\midrule
\textbf{Row}: Side 1 & $\phantom{0}\phantom{0}2$ & $\phantom{0}\phantom{0}9$ & $\phantom{0}\phantom{0}9$\\
\textbf{Row}: Central & $\phantom{0}22$ & $\phantom{0}30$ & $\phantom{0}37$\\
\textbf{Row}: Side 2 & $\phantom{0}\phantom{0}1$ & $\phantom{0}\phantom{0}0$ & $\phantom{0}\phantom{0}2$\\
\bottomrule
\end{tabular}
\end{@empty}
\end{minipage}
\hspace{0.05\textwidth}
\begin{minipage}{0.42\textwidth}%
\centering

\includegraphics[width=1\linewidth]{jamovi/CrabShells/CrabShells3ChisquareTest} 
\includegraphics[width=1\linewidth]{jamovi/CrabShells/CrabShells2ChisquareTest} 
\caption{Software output for the $3\times 3$ table of crab-shell data (top output), and for the $2\times 2$ table of crab-shell data (bottom output).}\label{fig:CrabShellsChisqjamovi}
\end{minipage}%
\end{figure}

\captionsetup{font=normalsize}

\begin{EOCanswerBox}{iconmonstr-check-mark-14-240.png}
\textbf{Answers to \emph{Quick review} questions:} \textbf{1.} False. \textbf{2.} True. \textbf{3.} True. \textbf{4.} False. \textbf{5.} False. \textbf{6.} True. \textbf{7.} True. \textbf{8.} True. \textbf{9.} a.

\end{EOCanswerBox}

\chapter{Finding sample sizes for CIs}\label{EstimatingSampleSize}

\begin{cols}
\begin{col}{0.52\textwidth}

\begin{objectivesBox}{iconmonstr-target-4-240.png}
You have learnt to ask an RQ, design a study, classify and summarise the data, construct confidence intervals, and conduct hypothesis tests.
\textbf{In this chapter}, you will learn to:
\begin{itemize}\tightlist
  \item
  estimate the sample size for producing a CI of given width for a proportion, mean, mean difference, difference between two means, and difference between two proportions.
  \item
  explain issues relevant to estimating sample sizes.
\end{itemize}
\end{objectivesBox}

\end{col}

\begin{col}{0.03\textwidth}
~
\end{col}

\begin{col}{0.45\textwidth}

\includegraphics[width=0.95\linewidth]{32-Sample-Size-Estimation_files/figure-latex/unnamed-chunk-7-1} 
\end{col}
\end{cols}

\section{Introduction}\label{SampleSizeIntroduction}

\index{Sample size estimation}

A confidence interval (CI) is an interval which gives a range of values of the parameter that could plausibly have produced the observed value of the statistic.\index{Confidence intervals} All else being equal, a \emph{larger} sample size gives a \emph{more precise} estimate of the parameter (Sect.~\ref{PrecisionAccuracy});\index{Precision} that is, a \emph{narrower} CI. After all, that's why larger samples are preferred over smaller samples: they provide more \emph{precise} estimates.

\begin{example}[Impact of sample size on CIs]
\protect\hypertarget{exm:SampleSizeImpact}{}\label{exm:SampleSizeImpact}Suppose we wish to estimate an unknown proportion, and find that \(\hat{p} = 0.52\) from a sample of size \(n = 25\). The approximate \(95\)\%~CI is \(0.52 \pm 0.200\) (so the \emph{margin of error} is~\(0.200\)).

If the estimate of \(\hat{p} = 0.52\) was found from a sample of size \(n = 100\) (rather than \(n = 25\)), a more precise estimate should be expected. The approximate \(95\)\%~CI is \(0.52\pm 0.100\); the margin of error is~\(0.100\), so the estimate is indeed more precise.

If the estimate of \(\hat{p} = 0.52\) was found from a sample of size \(n = 400\), the approximate \(95\)\%~CI is \(0.52\pm 0.050\); the margin of error is~\(0.050\) (which is more precise again).

At each step, the sample size was four times as large, but the margin of error was halved.
\end{example}

Figure~\ref{fig:SampleSizeCIWidth} shows the approximate width of the CI for estimating a proportion, for various sample sizes (all else being equal). Observe that:

\begin{itemize}
\tightlist
\item
  greater precision (\emph{smaller} CI width) is obtained using \emph{larger} sample sizes.
\item
  for \emph{small} sample sizes (say, smaller than~\(15\)), precision greatly increases with small increases in the sample size.
\item
  for \emph{large} sample sizes (say, greater than~\(30\)), precision improves only slightly when the sample size is increased.
\end{itemize}

\begin{figure}[hbtp]

{\centering \includegraphics[width=0.95\linewidth]{32-Sample-Size-Estimation_files/figure-latex/SampleSizeCIWidth-1} 

}

\caption{The approximate width of a $95$\% CI for a proportion, when various size samples are used.}\label{fig:SampleSizeCIWidth}
\end{figure}

That is, improving precision gets more difficult as sample sizes get larger: large gains in precision are made by moderately increasing small sample sizes, but only small gains in precision are made by large increases in already-large sample sizes.

\begin{importantBox}{iconmonstr-warning-8-240.png}
Remember that the sample size is the number of \emph{units of analysis}.

\end{importantBox}

\section{General ideas}\label{SampleSizeIdeas}

If larger samples give more precise estimates, should the largest sample possible always be used? Not necessarily; using large samples also has disadvantages:

\begin{itemize}
\tightlist
\item
  as seen above, very large sample sizes only slightly improve precision.
\item
  studies with larger samples sizes take longer to complete.
\item
  studies with larger samples sizes are more expensive.
\item
  ethics committees aim to keep sample sizes as small as possible, so that:\index{Ethics}

  \begin{itemize}
  \tightlist
  \item
    the environment is impacted as little as possible.
  \item
    the fewest possible animals are harmed.
  \item
    the fewest possible people are harmed or inconvenienced.
  \item
    resources, time and money are not wasted.
  \end{itemize}
\end{itemize}

\begin{example}[The cost of research]
\protect\hypertarget{exm:Biochar}{}\label{exm:Biochar}\citet{farrar2021biochar} studied the residual effect of organic biochar compound fertilisers (BCFs) \emph{two years} after application. This study required planting turmeric in pots using soil previously treated with BCFs.

After the turmeric was grown, the concentration of potassium, phosphorus and nitrogen---as well as many trace minerals---was determined from the soil in \emph{every} pot. In addition, \emph{every} turmeric plant was analysed for the number of shoots, the leaf mass fraction, and foliar nutrient information.

Every pot that is used has a substantial cost, both in terms of time and money. Using more pots increases precision, but also increases costs and the time to complete the study.
\end{example}

Determining the sample size to use in a study is a trade-off between the advantages of increasing precision, and the challenges of cost, time, and remaining ethical (Chap.~\ref{Ethics}). In addition, \emph{how} the sample is obtained is important: random samples give more \emph{accurate} samples (Sect.~\ref{PrecisionAccuracy}) than non-random samples. That is, the sample size is not the only issue to consider; \emph{how} the sample is obtained (i.e., random; non-random) is also important.

For these reasons, researchers usually identify a margin of error that is meaningful (i.e., of practical importance)\index{Practical importance} in the context of their study, to help identify an appropruarte sample size.

\begin{example}[Practical importance in sample size calculations]
\protect\hypertarget{exm:SampleSizeMError}{}\label{exm:SampleSizeMError}In a weight-loss study, estimating the weight loss with a precision of~\(1\,\text{g}\) is far more precise than is necessary: a weight loss of~\(1\,\text{g}\) has no practical importance, but would require a massive sample size to detect.\index{Practical importance}

In contrast, the sample size needed to detect a mean weight loss with a precision of~\(50\,\text{kg}\) would be far smaller. However, a weight loss so great is of no practical importance either, as most people who are looking to lose weight are hoping to lose far less than~\(50\,\text{kg}\).

The researchers may decide that a weight loss of~\(5\,\text{kg}\) is sufficient to be of practical importance, and determine the sample size based on this value.
\end{example}

In this chapter, we learn how to compute the (approximate) minimum sample size needed to obtain a given precision (i.e., for a given \emph{margin of error}\index{Margin of error}) for a \(95\)\%~CI. The estimation of sample sizes for constructing a CI is studied for these situations:

\begin{itemize}
\tightlist
\item
  estimating a proportion in Sect.~\ref{SampleSizeProportions}.
\item
  estimating a mean in Sect.~\ref{SampleSizeOneMean}.
\item
  estimating a mean difference in Sect.~\ref{SampleSizeMeanDifferences}.
\item
  estimating a difference between two means in Sect.~\ref{SampleSizeDifferenceTwoMeans}.
\item
  estimating a difference between two proportions in Sect.~\ref{SampleSizeDifferenceTwoProportions}.
\end{itemize}

The formulas given in this chapter only apply for \emph{forming \(95\)\%~CIs}, and are very \emph{conservative}: they will probably give \emph{minimum} samples sizes that are a little \emph{too large}, but that is better than being too small.

To ensure that the required targets are met, the results from the sample size calculation should always be rounded up. In addition, sample sizes slightly larger than calculated are often used, to allow for \emph{drop-outs}:\index{Drop outs} animals or plants that die; people who can no longer be contacted; and so on.

\begin{importantBox}{iconmonstr-warning-8-240.png}
Always \emph{round up} the result of the sample size calculation.

\end{importantBox}

\section{Sample size for estimating one proportion}\label{SampleSizeProportions}

\index{Sample size estimation!one proportion}

In Sect.~\ref{Female-Coffee-Drinkers}, a CI was formed for the \emph{population} proportion of female college students in the United States that drink coffee daily \citep{data:Kelpin2018:AlcoholCoffee}. From a sample of \(n = 360\), the CI was \(0.1694 \pm 0.0395\) (i.e., the \emph{margin of error} is~\(0.0395\)), or from~\(0.130\) to~\(0.209\).

To obtain a more precise estimate (i.e., a narrower CI), a larger sample is needed, but how much larger? For instance, suppose we would like a CI with margin of error of~\(0.02\) (rather than~\(0.0395\)). What size sample is needed?

\begin{definition}[Sample size: proportion]
\protect\hypertarget{def:SampleSizeProportion}{}\label{def:SampleSizeProportion}Conservatively, the size of the \emph{simple random sample} needed \emph{for a \(95\)\%~CI for a proportion} with a specified margin of error is \emph{at least} \[
   \frac{1}{(\text{Margin of error})^2}.
\]
\end{definition}

For the coffee-drinking situation above, at least \(\displaystyle 1\div (0.02^2) = 2\ 500\) female college students in the US is needed. This is a substantial increase from the original sample size of~\(360\).

\begin{example}[Sample size calculations for one proportion]
\protect\hypertarget{exm:SampleSizep}{}\label{exm:SampleSizep}To estimate the population proportion of South Africans that smoke, within~\(0.07\) with \(95\)\%~confidence, at least \[
  \frac{1}{(\text{Margin of error})^2} { = \frac{1}{0.07^2}}
\] people are needed; \emph{at least} \(n = 204.08\) people. In practice, \emph{at least} \(205\)~people are needed to achieve this desired level of precision (that is, \emph{always round up} in sample size calculations).
\end{example}

\section{Sample size for estimating one mean}\label{SampleSizeOneMean}

\index{Sample size estimation!one mean}

Estimating a mean depends on the variation in the observations. If the data have a small amount of variation, estimating the mean requires a smaller sample size as most observations are similar.

\begin{definition}[Sample size: mean]
\protect\hypertarget{def:SampleSizeMean}{}\label{def:SampleSizeMean}Conservatively, the size of the \emph{simple random sample} needed \emph{for a \(95\)\%~CI for the mean} with a specified margin of error is \emph{at least} \[
   \left( \frac{2 \times s}{\text{Margin of error}}\right)^2,
\] where \(s\) is an estimate of the standard deviation in the population.
\end{definition}

The formula requires a value for the sample standard deviation, \(s\). But if we don't have a sample yet, how can we have a value for the standard deviation of the \emph{sample} to use? An approximate value for~\(s\) is used, which can come from:

\begin{itemize}
\tightlist
\item
  the value of~\(s\) from the results of a pilot study (Sect.~\ref{Protocols}).\index{Pilot study}
\item
  the results of a similar study, where the value~\(s\) there can be used (see Example~\ref{exm:SampleSizePeanuts}).
\end{itemize}

\begin{example}[Sample size estimation for one mean]
\protect\hypertarget{exm:SampleSizePeanuts}{}\label{exm:SampleSizePeanuts}Sect.~\ref{Cadmium-In-Peanuts} discusses a study about the mean cadmium concentrations in peanuts in the United States, where \(s = 0.0460\)~ppm \citep{data:Blair2017:Peanuts}.

To estimate the mean cadmium concentration in \emph{Canadian} peanuts, within~\(0.005\)~ppm with \(95\)\%~confidence, this value for~\(s\) can be used. Then: \[
   \left( \frac{2 \times 0.0460}{0.005}\right)^2 = 338.56;
\] we would need at least \(339\)~Canadian peanuts.
\end{example}

\section{Sample size for estimating a mean difference}\label{SampleSizeMeanDifferences}

\index{Sample size estimation!mean difference}

The ideas in the previous section also work for computing sample sizes for estimating \emph{mean differences}, since the differences can be treated like a single sample.

\begin{definition}[Sample size: mean difference]
\protect\hypertarget{def:SampleSizeMeanDiff}{}\label{def:SampleSizeMeanDiff}Conservatively, the size of the \emph{simple random sample} needed \emph{for a \(95\)\%~CI for the mean difference} with a specified margin of error is \emph{at least} \[
   \left( \frac{2 \times s_d}{\text{Margin of error}}\right)^2,
\] where~\(s_d\) is an estimate of the standard deviation of the population differences.
\end{definition}

Again, an approximate value for \(s_d\) can come from a pilot study (Sect.~\ref{Protocols}),\index{Pilot study} or from the results of a similar study.

\begin{example}[Sample size estimation for mean differences]
\protect\hypertarget{exm:SampleSizeWeightGain}{}\label{exm:SampleSizeWeightGain}In Sect.~\ref{MeanDiffCI}, a CI is computed for the difference between the distances walked in \(6\,\text{mins}\) (the six-minute walk test, 6MWT), using a \(20\,\text{m}\) and~\(30\,\text{m}\) walkway \citep{saiphoklang2022comparison}, for \(50\)~Thai patients. The \emph{approximate} \(95\)\%~CI is from~\(15.80\,\text{m}\) to~\(28.26\,\text{m}\), further for a \(30\,\text{m}\)~walkway (i.e., the margin of error is~\(6.234\,\text{m}\)).

Suppose we wanted to estimate the mean difference in the 6MWT distances for Malaysian patients; we could use the value of~\(s\) from this study (i.e., \(s = 22.03920\)). Also, suppose we wanted a precision of~\(4\,\text{m}\) (that is, the margin of error is~\(4\)). For this \emph{more precise} estimate, we would need a \emph{larger sample}. So compute: \[
   \left( \frac{2 \times 22.03920}{4}\right)^2 = 121.43;
\] we would need at least \(122\)~patients, after rounding up.
\end{example}

\section{Sample size for estimating a difference between two means}\label{SampleSizeDifferenceTwoMeans}

\index{Sample size estimation!difference between means}

A formula for computing sample sizes for estimating the \emph{difference between two means} is simple if we make two assumptions:

\begin{itemize}
\tightlist
\item
  the sample size in both groups being compared is the same.
\item
  the standard deviation in both groups being compared is the same.
\end{itemize}

Formulas are available for computing sample sizes without these restrictions, but are more complicated than that given here. Again, an approximate value for \(s\) can come from a pilot study (Sect.~\ref{Protocols}),\index{Pilot study} or from the results of a similar study.

\begin{definition}[Sample size: difference between two means]
\protect\hypertarget{def:SampleSizeDiffBetweenTwomeans}{}\label{def:SampleSizeDiffBetweenTwomeans}Conservatively, the size of the \emph{simple random sample} needed \emph{for a \(95\)\%~CI for the difference between two means} with a specified margin of error is \emph{at least} \[
   2\times \left( \frac{2 \times s}{\text{Margin of error}}\right)^2
\] for \emph{each} sample, where~\(s\) is an estimate of the common standard deviation in the population for both groups.
\end{definition}

\begin{example}[Sample size estimation for difference between means]
\protect\hypertarget{exm:SampleSizeSpeeds}{}\label{exm:SampleSizeSpeeds}In Sect.~\ref{SpeedSignage}, a CI is computed for difference between the mean speeds of cars before and after signage was added \citep{ma2019impacts}. Suppose we wanted to estimate the difference between the mean reaction times within~\(5\,\text{km}\).h\textsuperscript{\(-1\)}.

In Sect.~\ref{SpeedSignage}, the two groups (before and after signage added) produced standard deviations of~\(13.194\) and~\(13.134\) (which are very similar). We decide to use \(s = 13.15\) in the sample-size calculation as the common value of~\(s\): \[
   2 \times \left( \frac{2 \times 13.15}{5}\right)^2 = 55.335.
\] We would need to measure the speed of at least~\(56\) cars before signage was added, and another~\(56\) cars after the addition of signage (rounding up the result).
\end{example}

\section{Sample size for estimating a difference between proportions}\label{SampleSizeDifferenceTwoProportions}

\index{Sample size estimation!difference between proportions}

A formula for computing sample sizes for estimating the \emph{difference between two proportions} is simple if we assume the sample size in both groups being compared is the same. Formulas are available for computing sample sizes without this restriction, but are more complicated than that given here.

\begin{definition}[Sample size: difference between two proportions]
\protect\hypertarget{def:SampleSizeDiffBetweenTwoProportions}{}\label{def:SampleSizeDiffBetweenTwoProportions}Conservatively, the size of the \emph{simple random sample} needed \emph{for a \(95\)\%~CI for the difference between two proportions} with a specified margin of error is \emph{at least} \[
   \frac{2}{(\text{Margin of error})^2}
\] for \emph{each} sample.
\end{definition}

\begin{example}[Sample size estimation for difference between proportions]
\protect\hypertarget{exm:SampleSizeTurtleNests}{}\label{exm:SampleSizeTurtleNests}In Sect.~\ref{TurtleNests}, a CI is computed for difference between the proportion of infected turtles nests, comparing natural and relocated nests \citep{candan2021first}. Suppose we wanted to estimate the difference between the proportion of infected nests within~\(0.15\).

We compute: \[
   \frac{2}{0.15^2} = 88.89.
\] We would need to record data from at least~\(89\) natural nests and~\(89\) relocated nests.
\end{example}

\section{More details about these sample size calculations}\label{SampleSizeOtherIssues}

The above calculations form just one part of the information needed to make the final decision about the necessary sample size. For example, the \emph{cost} (time and money) of taking the samples has not been considered.

The calculations in this chapter assume a \emph{simple random sample} will be used, which is often unreasonable. Other, more complex, formulas are available for computing sample sizes for other random-sampling schemes (such as stratified samples). However, the above calculations give an approximate \emph{minimum} sample size required. In addition, the calculations in this chapter are only for producing \(95\)\%~CI.

In practice, researchers often start with a slightly larger sample than calculated to allow for drop-outs\index{Drop outs} (e.g., plants die, or people withdraw from the study).

\section{Example: emergency residential aged care}\label{SampleSizeExample}

\citet{dwyer2021residential} studied residential aged care residents in Australia needing emergency care and recorded, among other information, the average age of such residents (\(\bar{x} = 85\); \(s = 7.3\)) and the proportion of calls related to falls (\(\hat{p} = 0.156\)).

Suppose a similar study was to be conducted in New Zealand. The aim was to estimate the mean age of residents within \(2\) years of age, and the proportion of incidents related to falls within~\(0.10\).

Using the value of~\(s\) from Australia, the sample size required to meet the age requirement is at least \[
  n = \left(\frac{2\times s}{\text{Margin of error}}\right)^2 = \left(\frac{2\times 7.3}{2}\right)^2 = 53.29,
\] or at least \(54\)~residents (rounding up). The sample size required to meet the falls requirement is at least \[
  n = \frac{1}{(\text{Margin of error}^2)} = \frac{1}{0.1^2} = 100.
\] Since the same subjects are needed for both estimates, at least \(100\)~residents are needed.

\section{Chapter summary}\label{chapter-summary-1}

Estimating a sample size is a compromise between the precision of the estimate, and the need to remain ethical and reduce costs. All else being equal, making a sample size four times as large results in a CI half as wide. This means that large gains in precision are made by increasing small sample sizes, but only small gains are made by increasing already-large sample sizes.

\section{Quick review questions}\label{Chap30-QuickReview}

Are the following statements \emph{true} or \emph{false}?

\begin{enumerate}
\def\labelenumi{\arabic{enumi}.}
\item
  A \emph{larger} sample size produces a \emph{more accurate} estimate of the parameter, all else being equal. \tightlist  
\item
  A \emph{larger} sample size produces a \emph{more random} sample.
\item
  We should \emph{always} take the \emph{largest} possible sample size.
\end{enumerate}

\section{Exercises}\label{EstimatingSampleSizeExercises}

\hyperref[Answers]{Answers to odd-numbered exercises} are given at the end of the book.

\captionsetup{font=small}

\begin{exercise}
\protect\hypertarget{exr:SampleSizeNarrow1}{}\label{exr:SampleSizeNarrow1}To obtain a \emph{narrower} CI, is a larger or smaller sample size necessary (all else being equal)?
\end{exercise}

\begin{exercise}
\protect\hypertarget{exr:SampleSizeNarrow2}{}\label{exr:SampleSizeNarrow2}Does a \emph{narrow} CI imply a \emph{precise} estimate, or an \emph{accurate} estimate of the parameter?
\end{exercise}

\begin{exercise}
\protect\hypertarget{exr:SampleSizeMean1}{}\label{exr:SampleSizeMean1}

Suppose we need to estimate a population \emph{mean} (with \(95\)\% confidence), using \(s = 1\,\text{kg}\).

\begin{enumerate}
\def\labelenumi{\arabic{enumi}.}
\tightlist
\item
  What size sample is needed to estimate the population mean within~\(0.4\,\text{kg}\)?
\item
  What size sample is needed to estimate the population mean within~\(0.2\,\text{kg}\) (that is, the CI will be \emph{half} as wide as in the first calculation)?
\item
  What size sample is needed to estimate the population mean within~\(0.1\,\text{kg}\) (that is, the CI will be \emph{a quarter} as wide as in the first calculation)?
\item
  To get a CI \emph{half} as wide, how many \emph{times} more units of analysis are needed?
\item
  To get a CI \emph{a quarter} as wide, how many \emph{times} more units of analysis are needed?
\item
  Would a \emph{smaller} or \emph{larger} sample be needed to estimate the population mean within~\(0.4\,\text{kg}\), with \(99\)\% confidence? Explain.
\end{enumerate}

\end{exercise}

\begin{exercise}
\protect\hypertarget{exr:SampleSizeTwoMean2}{}\label{exr:SampleSizeTwoMean2}

Suppose we need to estimate a difference between two population \emph{means} (with \(95\)\%~confidence), using \(s = 8\,\text{cm}\).

\begin{enumerate}
\def\labelenumi{\arabic{enumi}.}
\tightlist
\item
  What size samples are needed to estimate the difference between the population means within~\(4\,\text{cm}\)?
\item
  What size samples are needed to estimate the difference between the population means within~\(2\,\text{cm}\) (that is, the CI will be \emph{half} as wide as in the first calculation)?
\item
  What size samples are needed to estimate the difference between the population means within~\(1\,\text{cm}\) (that is, the CI will be \emph{a quarter} as wide as in the first calculation)?
\item
  To get a CI \emph{half} as wide, how many \emph{times} more units of analysis are needed?
\item
  To get a CI \emph{a quarter} as wide, how many \emph{times} more units of analysis are needed?
\item
  Would a \emph{smaller} or \emph{larger} sample be needed to estimate the population mean within~\(4\,\text{cm}\), with \(99\)\% confidence? Explain.
\end{enumerate}

\end{exercise}

\begin{exercise}
\protect\hypertarget{exr:SampleSizePropEating}{}\label{exr:SampleSizePropEating}

\citet{data:Mann12017:UniStudents} studied of the eating habits of university students in Canada (Sect.~\ref{exr:CIOneProportionSnacking}). They estimated the proportion of Canadian students that ate a sufficient number of servings of grains each day.

Suppose we wished to repeat the study but for \emph{New Zealand} university students; that is, we seek an estimate of the population proportion of New Zealand students that eat a sufficient number of servings of grains each day (with \(95\)\% confidence).

\begin{enumerate}
\def\labelenumi{\arabic{enumi}.}
\tightlist
\item
  What size sample is needed to estimate the proportion within~\(0.01\)?
\item
  What size sample is needed to estimate the proportion within~\(0.02\)?
\item
  What size sample is needed to estimate the proportion within~\(0.10\)?
\item
  Do you think this study would be costly, in terms of time and money?
\end{enumerate}

\end{exercise}

\begin{exercise}
\protect\hypertarget{exr:SampleSizeOneProportionAustSmokers}{}\label{exr:SampleSizeOneProportionAustSmokers}

We wish to estimate the population proportion of Kenyans that smoke.

\begin{enumerate}
\def\labelenumi{\arabic{enumi}.}
\tightlist
\item
  Suppose we wish our \(95\)\%~CI to have a margin of error of~\(0.05\). How many Kenyans would need to be surveyed?
\item
  Suppose we wish our \(95\)\%~CI to have a margin of error of~\(0.025\); that is, we wish to \emph{halve} the width of the interval above. How many Kenyans would need to be surveyed?
\item
  How many \emph{times} as many Kenyans are needed to \emph{halve} the width of the CI?
\end{enumerate}

\end{exercise}

\begin{exercise}
\protect\hypertarget{exr:SampleSizeMeanLungCapacity}{}\label{exr:SampleSizeMeanLungCapacity}

\citet{data:Tager:FEV} measured the lung capacity of \(11\)-year-old girls in East Boston, using the \emph{forced expiratory volume} (FEV) of the children (Exercise~\ref{exr:CIOneMeanLungCapacityInChildren}). Suppose we wished to repeat the study, and find a \(95\)\%~CI for the mean FEV for \(11\)-year-old \emph{Australian} girls.

Since Australian and American children might be somewhat similar, we could use, as an approximation, the standard deviation from that study: \(s = 0.43\)\,\text{L}.

\begin{enumerate}
\def\labelenumi{\arabic{enumi}.}
\tightlist
\item
  What size sample is needed to estimate the mean within~\(0.02\,\text{L}\)?
\item
  What size sample is needed to estimate the mean within~\(0.05\,\text{L}\)?
\item
  What size sample is needed to estimate the mean within~\(0.10\,\text{L}\)?
\item
  Suppose we wished to find \(99\)\% (not~\(95\)\%) CI for the mean FEV for \(11\)-year-old \emph{Australian} girls, within~\(0.10\,\text{L}\). Would this sample size be \emph{larger} or \emph{smaller} than the sample size found for a \(95\)\%~CI (also within~\(0.10\,\text{L}\))?
\item
  Do you think this study would be costly, in terms of time and money?
\end{enumerate}

\end{exercise}

\begin{exercise}
\protect\hypertarget{exr:SampleSizeOneMeanBloodLossSampleSize}{}\label{exr:SampleSizeOneMeanBloodLossSampleSize}

\citet{data:Williams2007:BloodLoss} asked paramedics (\(n = 199\)) to estimate the amount of blood loss on four different surfaces. When the actual amount of blood spill on concrete was~\(1\,000\,\text{mL}\), the mean guess was~\(846.4\,\text{mL}\) (with a standard deviation of~\(651.1\,\text{mL}\)). For a different study:

\begin{enumerate}
\def\labelenumi{\arabic{enumi}.}
\tightlist
\item
  how many paramedics are needed to estimate the mean with a precision of~\(50\,\text{mL}\)?
\item
  how many paramedics are needed to estimate the mean with a precision of~\(25\,\text{mL}\)?
\item
  how many times greater does the sample size need to be to \emph{halve} the width of the margin of error?
\end{enumerate}

\end{exercise}

\begin{exercise}
\protect\hypertarget{exr:SampleSizeInvasivePlants}{}\label{exr:SampleSizeInvasivePlants}

Skypilot is an alpine wildflower native to the Colorado Rocky Mountains (USA). In recent years, a willow shrub has been encroaching on skypilot territory and, because willow often flowers early, \citet{kettenbach2017shrub} studied whether the willow may `negatively affect pollination regimes of resident alpine wildflower species' (p.~\(6\,965\)). Data for both species was collected at \(25\)~different sites, so the data are \emph{paired} by site. The `first-flowering day' is the number of days since the start of the year (e.g., January~\(12\) is `day~\(12\)') when flowers were first observed.

Suppose a similar paired study was to be conducted on skypilot growing in Sierra Nevada, California. Using the software output in Fig.~\ref{fig:Floweringjamovi}:

\begin{enumerate}
\def\labelenumi{\arabic{enumi}.}
\tightlist
\item
  determine the sample size needed to estimate the mean difference in first-flowering day within two days.
\item
  determine the sample size needed to estimate the mean difference in first-flowering day within three days.
\end{enumerate}

\end{exercise}

\begin{exercise}
\protect\hypertarget{exr:SampleSizeCaptopril}{}\label{exr:SampleSizeCaptopril}

\citet{data:macgregor:essential} studied treating hypertension with Captopril. Patients had their systolic blood pressure measured (in mm~Hg) immediately \emph{before} and two hours \emph{after} being given the drug. A pilot study showed that the difference between the two measurements had a standard deviation of about~\(9\,\text{mm}\) Hg.

\begin{enumerate}
\def\labelenumi{\arabic{enumi}.}
\tightlist
\item
  Determine the sample size needed to estimate the mean reduction in \emph{systolic} blood pressure within~\(2\,\text{mm}\) Hg.
\item
  Determine the sample size needed to estimate the mean reduction in \emph{diastolic} blood pressure within~\(1.5\,\text{mm}\) Hg.
\end{enumerate}

\end{exercise}

\begin{exercise}
\protect\hypertarget{exr:SampleSizeWhales}{}\label{exr:SampleSizeWhales}

\citet{agbayani2020growth} studied gray whales (\emph{Eschrichtius robustus}) and measured (among other variables) the length of whales at birth. Summary information is shown in Table~\ref{tab:WhaleInfo}. Suppose another research study wanted to study sperm whales, which have an approximately similar size.

\begin{enumerate}
\def\labelenumi{\arabic{enumi}.}
\tightlist
\item
  Determine the sample size needed to estimate the difference between the mean lengths for female and male sperm whales at birth, within~\(0.15\,\text{m}\).
\item
  Determine the sample size needed to estimate the difference between the mean lengths for female and male sperm whales at birth, within~\(0.10\,\text{m}\).
\item
  Determine the sample size needed to estimate the difference between the mean lengths for female and male \emph{goldfish} at birth, within~\(1\,\text{mm}\).
\end{enumerate}

\end{exercise}

\begin{exercise}
\protect\hypertarget{exr:SampleSizePneumonia}{}\label{exr:SampleSizePneumonia}

Suppose researchers are trialling a new drug to reduce the recovery time (compared to standard treatments) after contracting pneumonia. They conduct a pilot study, and find the standard deviation of the duration of the symptoms, in both groups, is about \(s = 1.25\) days.

\begin{enumerate}
\def\labelenumi{\arabic{enumi}.}
\tightlist
\item
  What size sample is needed to estimate the difference between the mean recovery times between the two treatments within~\(1\)~day.
\item
  What size sample is needed to estimate the difference between the mean recovery times between the two treatments within~\(0.5\)~days.
\end{enumerate}

\end{exercise}

\begin{exercise}
\protect\hypertarget{exr:SampleSizeHAttack}{}\label{exr:SampleSizeHAttack}Table~\ref{tab:HAttackData} summarises the data from a study of the incidents of in-hospital heart attacks for people admitted following an earlier heart attack. To estimate the difference between the proportion of patients having an in-hospital heart attack (between patients with a low body temperature and patients with a high body temperature) within~\(0.03\), what size samples are needed?
\end{exercise}

\begin{exercise}
\protect\hypertarget{exr:SampleSizeSunglasses}{}\label{exr:SampleSizeSunglasses}Exercise~\ref{exr:TestOddsRatioSunglasses} describes a study comparing the proportion of females and males who wore sunglasses in Brisbane, Australia \citep{data:Dexter2019:SunProtection}. Suppose we wished to make a similar comparison for people in Auckland, estimating the difference in the proportions within~\(0.07\). How many females and males would be needed?
\end{exercise}

\captionsetup{font=normalsize}

\begin{EOCanswerBox}{iconmonstr-check-mark-14-240.png}
\textbf{Answers to \textit{Quick review} questions:} \textbf{1.} False. \textbf{2.} False. \textbf{3.} False.

\end{EOCanswerBox}

\chapter{Correlation and regression: CIs and tests}\label{CorrelationRegression}

\begin{cols}
\begin{col}{0.52\textwidth}

\begin{objectivesBox}{iconmonstr-target-4-240.png}
So far, you have learnt about the research process, including analysing data using confidence intervals and conducting hypothesis tests.
\textbf{In this chapter}, you will learn to:
\begin{itemize}\tightlist
  \item
  produce confidence intervals for correlation coefficients.
  \item
  conduct hypothesis tests for correlation coefficients.
  \item
  produce and interpret linear regression equations.
  \item
  conduct hypothesis tests for the slope of a regression line.
  \item
  produce confidence intervals for the slope of a regression line.
  \item
  determine whether the conditions for using these methods apply in a given situation.
\end{itemize}
\end{objectivesBox}

\end{col}

\begin{col}{0.03\textwidth}
~
\end{col}

\begin{col}{0.45\textwidth}

\includegraphics[width=0.95\linewidth]{33-CorrelationRegression_files/figure-latex/unnamed-chunk-9-1} 
\end{col}
\end{cols}

\section{Introduction: sorghum yield and borers}\label{CorReg-Intro}

\index{Correlation}

So far, RQs about single variables (descriptive~RQs) and RQs for comparisons (relational and repeated-measures~RQs) have been studied. In this chapter, the relationship between two quantitative variables is studied (correlational~RQs) \emph{when that relationship is approximately linear}. The strength of the relationship (correlation) and the nature of that relationship (regression) are discussed.

For this chapter, consider this (one-tailed) RQ:

\begin{quote}
In sorghum crops (AG1090 hybrid) in Brazil, is a larger sugarcane borer infestation associated with smaller yields?
\end{quote}

\citet{da2024potential} recorded the borer infestation in sorghum crops, for \(n = 24\) crops over three years \citep{da2024potentialDATA}, shown in Table~\ref{tab:BorersData}. The data comprises two quantitative variables (Fig.~\ref{fig:BorersScatterjamovi2}, left panel).\index{Graphs!scatterplot}

\begin{table} \centering \centering\caption{\label{tab:BorersData}Sorghum AG1090 yield and sugarcane borer infestation; the first five and last five of $n = 24$ observations.}

\fontsize{8}{10}\selectfont
\begin{tabular}[t]{cc}
\toprule
\textbf{Infestation (\%)} & \textbf{Yield (kg.ha$^{-1}$)}\\
\midrule
$10.57$ & $2844.12$\\
$13.29$ & $3055.69$\\
$18.64$ & $3025.14$\\
$11.43$ & $3101.36$\\
$\phantom{0}5.53$ & $4539.90$\\
$\vdots$ & $\vdots$\\
\bottomrule
\end{tabular} \quad\quad 
\begin{tabular}[t]{cc}
\toprule
\textbf{Infestation (\%)} & \textbf{Yield (kg.ha$^{-1}$)}\\
\midrule
$\vdots$ & $\vdots$\\
$35.79$ & $1109.62$\\
$27.42$ & $1573.81$\\
$27.09$ & $1587.57$\\
$25.01$ & $2403.90$\\
$23.04$ & $3819.29$\\
\bottomrule
\end{tabular}
\end{table}

Knowing the amount of infestation provides some information about the yield: a moderate relationship between the variables seems evident. The relationship also seems somewhat \emph{linear}. The \emph{Pearson correlation coefficient} (Fig.~\ref{fig:BorersScatterjamovi2}, right panel)\index{Correlation coefficient (Pearson)} is \(r = -0.934\), so \(R^2 = (-0.934)^2 = 87.2\)\%. This means that the unexplained variation in yield reduces by \(78.8\)\% by knowing the amount of infestation.

\begin{importantBox}{iconmonstr-warning-8-240.png}
Recall that the \emph{sample} correlation coefficient is denoted by~\(r\), and the \emph{population} correlation coefficient is denoted by~\(\rho\).

\end{importantBox}

\begin{figure}[hbtp]

{\centering \includegraphics[width=0.54\linewidth]{33-CorrelationRegression_files/figure-latex/BorersScatterjamovi2-1} \includegraphics[width=0.45\linewidth]{jamovi/Borers/Borers-Correlation-CI} 

}

\caption{Sorghum yield against borer infestation. Left: scatterplot. Right: correlation output.}\label{fig:BorersScatterjamovi2}
\end{figure}

\section{\texorpdfstring{Correlation: CIs and tests for \(\rho\)}{Correlation: CIs and tests for \textbackslash rho}}\label{CorrelationCITest}

\subsection{\texorpdfstring{Correlation: CIs for \(\rho\)}{Correlation: CIs for \textbackslash rho}}\label{CorrelationCI}

\index{Confidence intervals!correlation coefficient|(}

The sorghum data in Table~\ref{tab:BorersData} is only one of the countless possible samples of sorghum crops that could have been studied. The value of~\(r\) (an estimate of \(\rho\), the \emph{parameter}) will vary from sample to sample; that is, the value of~\(r\) has a sampling distribution, and sampling variation exists.\index{Sampling variation} The sampling distribution of~\(r\), however, does \emph{not} have a normal distribution, so CIs for~\(\rho\) will be taken directly from software output (Fig.~\ref{fig:BorersScatterjamovi2}, right panel). For the sorghum data, the \(95\)\% CI for~\(\rho\) is from \(-0.971\) to~\(-0.851\). This CI is not symmetrical: the value of~\(r\) is not halfway between these limits. We write:

\begin{quote}
For soghum crops, the correlation coefficient between yield and infestation percentage is~\(-0.934\), with a \(95\)\% CI from \(-0.971\) to~\(-0.851\) (\(n = 24\)).
\end{quote}

In other words, a population with a correlation coefficient~\(\rho\) between \(-0.971\) and~\(-0.851\) could reasonably have produced a sample correlation coefficient of \(r = -0.934\) from a sample of size \(n = 24\).

\begin{example}[Correlation]
\protect\hypertarget{exm:CycloneCorrelationCI}{}\label{exm:CycloneCorrelationCI}The relationship between the number of cyclones~\(y\) in the Australian region each year from~1969 to~2005, and a unitless climatological index called the \emph{Ocean Niño Index} (ONI,~\(x\)), averaged over October, November and December, is shown in Fig.~\ref{fig:ONIcyclonesCorrelation} (left panel) \citep{mypapers:dunnsmyth:glms}.

The relationship has a \emph{negative} direction, so the value of~\(r\) is \emph{negative}. From the software output, \(r = -0.683\) with a \(95\)\% CI from \(-0.824\) to~\(-0.460\).
\end{example}

\index{Confidence intervals!correlation coefficient|)}

\begin{figure}[hbtp]

{\centering \includegraphics[width=0.54\linewidth]{33-CorrelationRegression_files/figure-latex/ONIcyclonesCorrelation-1} \includegraphics[width=0.45\linewidth]{jamovi/Cyclones/Cyclones-CorrelationCI} 

}

\caption{The number of cyclones in the Australian region each year from 1969 to 2005, and the ONI averaged over October, November, December. Left: scatterplot. Right: software output.}\label{fig:ONIcyclonesCorrelation}
\end{figure}

\subsection{\texorpdfstring{Correlation: hypothesis test for \(\rho\)}{Correlation: hypothesis test for \textbackslash rho}}\label{CorrelationTesting}

\index{Hypothesis testing!correlation coefficient|(}

A hypothesis test can also be conducted regarding~\(\rho\), the Pearson correlation coefficient in the \emph{population}.\index{Correlation} The null hypothesis is, as always, the `no difference, no change, no relationship' position which is, in this context:

\begin{itemize}
\tightlist
\item
  \(H_0\): \(\rho = 0\).
\end{itemize}

Clearly, the \emph{sample} correlation coefficient \(r\) for the data is not zero, and the RQ is effectively asking if sampling variation is the reason for this discrepancy between~\(r\) and the parameter~\(\rho\).

Since the RQ (in Sect.~\ref{CorReg-Intro}) is one-tailed (negative direction), the alternative hypothesis is:

\begin{itemize}
\tightlist
\item
  \(H_1\): \(\rho < 0\) \quad (\emph{one-tailed} test, based on the RQ).
\end{itemize}

As usual, initially \emph{assume} that \(\rho = 0\) (from~\(H_0\)), then describe what values of~\(r\) could be \emph{expected} using the \emph{sampling distribution}, under that assumption, across all possible samples. Then the \emph{observed} value of~\(r\) is compared to the values expected through sampling variation to determine if the value of~\(r\) supports or contradicts the assumption.

For a correlation coefficient, the sampling distribution of~\(r\) does not have a normal distribution.\footnote{For those interested: the value of~\(r\) only varies between \(-1\) and \(1\), so cannot have a normal distribution. A transformation of~\(r\) \emph{does} exist that has an approximate normal distribution and \emph{standard error}.} However, the output (Fig.~\ref{fig:BorersScatterjamovi2}, right panel)\index{Software output!correlation} contains the relevant two-tailed \(P\)-value for the test: less than~\(0.001\). Hence, the one-tailed \(P\)-value for the test is less than~\(0.0005\). \emph{Very strong evidence} exists to support~\(H_1\) (that the correlation in the population is negative).

We write:

\begin{quote}
The sample presents very strong evidence (one-tailed \(P < 0.0005\)) that the sorghum yield has a negative association with borer infestation percentage (\(r = -0.934\) with \(95\)\% CI from \(-0.971\) to \(-0.851\); \(n = 24\)) in the population.
\end{quote}

Notice the three features of writing conclusions again: an answer to the RQ, evidence to support the conclusion, and some sample summary information.

\begin{tipBox}{iconmonstr-info-6-240.png}
If the evidence suggests that the correlation coefficient is \emph{not zero} (in the population), this does \emph{not} necessarily mean a \emph{strong} correlation exists. The correlation may be weak in the population (as estimated by the value of~\(r\)), but evidence exists that the correlation is \emph{not zero} in the \emph{population}.

That is, the test is about statistical significance, not practical importance.\index{Practical importance}

\end{tipBox}

\begin{example}[Correlation]
\protect\hypertarget{exm:CycloneCorrelationTest}{}\label{exm:CycloneCorrelationTest}The relationship between the number of cyclones~\(y\) in the Australian region each year from~1969 to~2005, and the ONI~\(x\) is shown in Fig.~\ref{fig:ONIcyclonesCorrelation} (left panel). To test for a relationship, use \[
   \text{$H_0$: $\rho = 0$} \qquad\text{ against}\qquad \text{$H_0$: $\rho \ne 0$};
\] software reports that \(P < 0.0001\) (Fig.~\ref{fig:ONIcyclonesCorrelation}, right panel). There is very strong evidence of a relationship between the number of cyclones in the Australian region and the ONI (averaged over October, November and December).
\end{example}

\index{Hypothesis testing!correlation coefficient|)}

\section{Regression}\label{regression}

\subsection{Introducting regression}\label{Chap35-Intro}

\emph{Correlation} measures the \emph{strength} and \emph{direction} of the \emph{linear} relationship between two quantitative variables~\(x\) (an explanatory variable) and~\(y\) (a response variable). Sometimes, however, \emph{describing} the nature of the relationship is useful. This is called \emph{regression}.

The regression relationship is described mathematically using an \emph{equation}, and allows us to:

\begin{enumerate}
\def\labelenumi{\arabic{enumi}.}
\tightlist
\item
  \emph{Predict} the mean value of~\(y\) from a given value of~\(x\) (Sect.~\ref{RegressionForPrediction}).
\item
  \emph{Understand} the relationship between~\(x\) and~\(y\) (Sect.~\ref{RegressionForUnderstanding}).
\end{enumerate}

An example of a \emph{linear} regression equation, describing the linear relationship between the observed values of an explanatory variable~\(x\) and the observed values of a response variable~\(y\), is\index{Linear equations} \begin{equation}
  \hat{y} = -4  + (2\times x), \qquad\text{usually written}\qquad \hat{y} = -4  + 2x.
  \label{eq:ExampleRegressionEqn}
\end{equation} The notation~\(\hat{y}\) refers to the mean of all the \(y\)-values that could be observed for some given value of~\(x\). That is, for some value of~\(x\), many different values of~\(y\) could be observed, and~\(\hat{y}\) is the value that regression equation predicts as the \emph{mean} of all those possible values. This equation describes the connection between the values of~\(x\) and the corresponding average values of~\(y\).

\begin{importantBox}{iconmonstr-warning-8-240.png}
\(y\)~refers to the values of the response variable \emph{observed} from individuals. \(\hat{y}\) refers to \emph{predicted mean} value of~\(y\) for given values of~\(x\).

\end{importantBox}

\begin{pronounceBox}{iconmonstr-microphone-7-240.png}
\(\hat{y}\) is pronounced as `why-hat'; the `caret' above the~\(y\) is called a `hat'.

\end{pronounceBox}

More generally, the equation of a straight line is \[
   \hat{y} = b_0 + (b_1 \times x), \qquad\text{usually written}\qquad \hat{y} = b_0 + b_1 x,
\] where the values of~\(b_0\) and~\(b_1\) are unknown, and estimated from sample data. Notice that~\(b_1\) is the number multiplied by~\(x\). In Equation~\eqref{eq:ExampleRegressionEqn}, \(b_0 = -4\) and \(b_1 = 2\).

\begin{example}[Regression equations]
\protect\hypertarget{exm:RegressionGirlsHt}{}\label{exm:RegressionGirlsHt}A report on the growth of Australian children \citep{data:AustralianGirls} found an approximate linear relationship between the age (in years) \(x\) and height (in~cm)~\(y\) of girls aged between four and seven. The regression equation was approximately \[
   \hat{y} = 73 + 7x.
\] The regression equation is the same if written as \[
   \hat{y} = 7x + 73.
\] In both cases, \(b_0 = 73\) and \(b_1 = 7\). This regression equation describes the connection between ages~\(x\) and heights~\(y\) for girls aged between four and seven.
\end{example}

\subsection{Reviewing linear equations}\label{RegressionEquationsReview}

\index{Linear equations}\index{Regression}

To introduce, or revise, the idea of a linear equation, consider the (artificial) data in Fig.~\ref{fig:ExampleScatterLATEX} (left panel), with an explanatory variable~\(x\) and a response variable~\(y\).\index{Response variable}\index{Explanatory variable}\index{Graphs!scatterplot} In the graph, a sensible line is drawn on the graph that seems to capture the relationship between~\(x\) and~\(y\). (You may have drawn a slightly different, but similar, line.) The line describes the predicted mean values of~\(y\) (i.e., values of \(\hat{y}\)) for various values of~\(x\). The relationship is \emph{positive} and \emph{linear}.

A \emph{regression equation} specifies the line.\index{Regression!equation} In the regression equation \(\hat{y} = b_0 + b_1 x\), the numbers~\(b_0\) and~\(b_1\) are called \emph{regression coefficients},\index{Regression!coefficients} where

\begin{itemize}
\tightlist
\item
  \(b_0\) is the \emph{intercept} (or the \emph{\(y\)-intercept}),\index{Regression!equation!intercept}\index{Linear equations!intercept} whose value corresponds to the \emph{predicted} mean value of~\(y\) when \(x = 0\).
\item
  \(b_1\) is the \emph{slope},\index{Regression!equation!slope}\index{Linear equations!slope} whose value measures how much the value of~\(\hat{y}\) changes, on average, when the value of~\(x\) \emph{increases} by~\(1\).
\end{itemize}

We will use software to find the values of~\(b_0\) and~\(b_1\) (as the formulas are tedious to use). However, a rough approximation of the values of~\(b_0\) and~\(b_1\) can be obtained using the rough straight line drawn on the scatterplot (Fig.~\ref{fig:ExampleScatterLATEX}).

A rough approximation of the value of the \emph{intercept}~\(b_0\) is the value of~\(\hat{y}\) when \(x = 0\), from the drawn line. When \(x = 0\), the regression line suggests the value of~\(\hat{y}\) is about~\(2\) in Fig.~\ref{fig:ExampleScatterLATEX} (left panel); that is, the value of~\(b_0\) is approximately~\(2\).

A rough approximation of the \emph{slope}~\(b_1\) is found using\index{Regression!equation!rise over run|(}

\begin{equation}
     \frac{\text{Change in $\hat{y}$}}{\text{Corresponding \emph{increase} in $x$}}
   = \frac{\text{rise}}{\text{run}}
   \label{eq:RiseOverRun}
\end{equation}

from the drawn line. This approximation of the slope is the \emph{change} in the value of~\(\hat{y}\) (the `rise') divided by the corresponding \emph{increase} in the value of~\(x\) (the `run').

\begin{figure}[hbtp]

{\centering \includegraphics[width=1\linewidth]{33-CorrelationRegression_files/figure-latex/ExampleScatterLATEX-1} 

}

\caption{Estimating the regression equation from a line using an example scatterplot. Left: approximating $b_0$. Right: approximating $b_1$ using rise-over-run.}\label{fig:ExampleScatterLATEX}
\end{figure}

Consider what happens in Fig.~\ref{fig:ExampleScatterLATEX} (right panel) when the value of~\(x\) increases from~\(1\) to~\(5\) (a \emph{run} of \(5 - 1 = 4\)). The corresponding value of~\(y\) changes from about~\(5\) to about~\(17\), a \emph{rise} of \(17 - 5 = 12\). So, using Equation~\eqref{eq:RiseOverRun}, \[
   \frac{\text{rise}}{\text{run}} 
   = \frac{17 - 5}{5 - 1}
   = \frac{12}{4} 
   = 3.
\] The value of~\(b_1\) is about~\(3\). When using rise-over-run, better guesses for~\(b_1\) are found when using one value of~\(x\) near the left-side of the scatterplot, and another value of~\(x\) near the right-side of the scatterplot, but any two values can be used (try using \(x = 0\) and \(x = 3\)).

Combing the rough guesses for \(b_0\) and \(b_1\), the regression line is approximately \(\hat{y} = 2 + (3\times x)\), usually written \[
  \hat{y} = 2 + 3x.
\]

\begin{importantBox}{iconmonstr-warning-8-240.png}
The regression equation has~\(\hat{y}\) (not~\(y\)) on the left-hand side. That is, the equation \emph{predicts} the \emph{mean} values of~\(y\), which may not be equal to any of the observed values (which are denoted by~\(y\)).

A `good' regression equation would produce predicted values~\(\hat{y}\) close to the observed values~\(y\); that is, the line passes close to each point on the scatterplot.\index{Graphs!scatterplot}

\end{importantBox}

The \emph{intercept}~\(b_0\) has the same measurement units as the response variable.\index{Regression!equation!intercept} The measurement units for the \emph{slope}~\(b_1\) is the `measurement units of the response variable', per `measurement units of the explanatory variable'.\index{Regression!equation!intercept}

\begin{example}[Measurement units of regression parameters]
In Example~\ref{exm:RegressionGirlsHt}, the regression line for the relationship between the age of Australian girls~\(x\) (in years) and their height (in~cm)~\(y\) was \(\hat{y} = 73 + 7x\) (for girls aged between four and seven years).

In the equation, the intercept is \(b_0 = 73\,\text{cm}\) and the slope is \(b_1 = 73\,\text{cm}\)/y (the growth \emph{rate}).
\end{example}

\begin{example}[A rough approximation of the regression equation]
For the sorghum data, a rough estimate of the regression line can be drawn on a scatterplot to estimate~\(b_0\) and~\(b_1\) (Fig.~\ref{fig:BorersRiseRun}). The estimate of~\(b_0\) (the value of~\(\hat{y}\) when \(x = 0\)) is roughly~\(4\,800\,\text{kg}.\text{ha}^{-1}\).

The estimate of~\(b_1\) can be found using the rise-over-run idea. When \(x = 0\), the value of~\(\hat{y}\) (according to the drawn line) is about~\(4\,800\). At the other extreme of the plot, where \(x = 40\), the value of~\(\hat{y}\) is about~\(1\,000\). (Any two points on the line can be used, but using two points at each end gives better guesses of the slope.) So, as~\(x\) increases from~\(0\) to about~\(40\), the value of~\(\hat{y}\) \emph{reduces} from about~\(4\,800\) to about~\(1\,000\), a change of about~\(-3\,800\). That is, for a `run' of \(40 - 0 = 40\), the `rise' is \(4\,800 - 1\,000 = -3\,800\) (i.e., a drop of~\(3\,800\)), and so a rough estimate of the slope is \(-3\,800/40 = -95\).\index{Linear equations!rise over run} (The relationship is \emph{negative}, so the slope is \emph{negative}.)

The rough guess of the regression line is therefore \[
  \hat{y} = 4\,800 - 95x,
\] where~\(x\) is the infestation percentage, and~\(y\) is yield (in~\(\,\text{kg}.\text{ha}^{-1}\)). The rough guess of the intercept~\(b_0\) is~\(4\,800\,\text{kg}.\text{ha}^{-1}\), while the rough guess of the slope~\(b_1\) is~\(-95\,\text{kg}.\text{ha}^{-1}/\%\).
\end{example}

\begin{figure}[hbtp]

{\centering \includegraphics[width=1\linewidth]{33-CorrelationRegression_files/figure-latex/BorersRiseRun-1} 

}

\caption{Obtaining rough guesses for the regression equation for the sorghum data. Left: approximating $b_0$. Right: approximating $b_1$ using rise-over-run. The plus signs on the right plot indicate the points used to estimate the slope.}\label{fig:BorersRiseRun}
\end{figure}

\begin{example}[Estimating regression parameters]
\protect\hypertarget{exm:CycloneRegressionGuesses}{}\label{exm:CycloneRegressionGuesses}The relationship between the number of cyclones~\(y\) in the Australian region each year from~1969 to~2005, and the ONI~\(x\) is shown in Fig.~\ref{fig:ONIcyclonesCorrelation} (left panel).

To make a guess of the regression coefficients, a sensible line can be drawn through the data (Fig.~\ref{fig:ONIcyclones}). When the value of~\(x\) is zero, the predicted value of~\(y\) is about~\(12\), so~\(b_0\) is about~\(12\)~cyclones. Recall: the intercept is the predicted value of~\(y\) when \(x = 0\), which is \emph{not} at the left of the graph in Fig.~\ref{fig:ONIcyclones}.

To approximate the value of~\(b_1\), use the rise-over-run idea.\index{Linear equations!rise over run} When~\(x = -2\), the predicted mean value of~\(y\) is about~\(17\); when~\(x = 2\), the predicted mean value of~\(y\) is about~\(8\). The value of~\(x\) increases by \(2 - (-2) = 4\), while the value of~\(\hat{y}\) changes by \(7.5 - 17 = -9.5\) (a \emph{decrease} of about \(9.5\)). Hence, \(b_1\) is approximately \(-9.5/4 = -2.375\) cyclones per unit change in ONI. (You may get a slightly different value from a slightly different line.)

The relationship has a \emph{negative} direction, so the slope must be \emph{negative}. The regression line is approximately \(\hat{y} = 12 - 2.375x\).
\end{example}

\begin{figure}[hbtp]

{\centering \includegraphics[width=0.7\linewidth]{33-CorrelationRegression_files/figure-latex/ONIcyclones-1} 

}

\caption{The number of cyclones in the Australian region each year from 1969 to 2005, and the ONI averaged over October, November, December. An estimate of the reggression line is shown. The plus sign ${}+{}$ is located on the line where $x = 0$. The crosses ${}\times{}$ are located to find rise-over-run.}\label{fig:ONIcyclones}
\end{figure}

The\index{Regression!equation!rise over run|)} above method gives a crude approximation to the values of the intercept~\(b_0\) and the slope~\(b_1\). In practice, \emph{many} reasonable lines could be drawn through a scatterplot of data, each giving slightly different rough guesses for~\(b_0\) and~\(b_1\). However, one of those lines is the `best' line in some sense,\footnote{For those interested: the `line of best fit' is the line for which the sum of the \emph{squared} vertical distances between the observations~\(y\) and the predicted values~\(\hat{y}\) (i.e, the regression line) is as small as possible.} and is sometimes called the `line of best fit'.\index{Regression!equation!line of best fit}

\subsection{Regression: finding equations using software}\label{Regression-Software}

\index{Regression!equation!using software}\index{Software output!regression}

Software is almost always used to find the estimates for the intercept~\(b_0\) and slope~\(b_1\), as the formulas are complicated and tedious to use. For the sorghum data (Fig.~\ref{fig:BorersScatterjamovi2}), the relevant software output is shown in Fig.~\ref{fig:BorersRegression}.

In the output, the values of~\(b_0\) and~\(b_1\) are in the column labelled \texttt{Estimate}; the value of the sample \(y\)-intercept is \(b_0 = 4\,814.1\,\text{kg}.\text{ha}^{-1}\), and the value of the sample \emph{slope} is \(b_1 = -101.4\,\text{kg}.\text{ha}^{-1}/\%\). These are the values of the two \emph{regression coefficients}.\index{Regression!coefficients} The regression equation is, after rounding: \begin{equation}
    \hat{y} = 4\,814.1 + (-101.4\times x),
    \qquad\text{usually written as}\qquad  
    \hat{y} = 4\,814.1 - 101.4 x.
  \label{eq:BorersRegressionEqn}
\end{equation} These are close to the values obtained using the rough method in Sect.~\ref{RegressionEquationsReview} (which gave \(b_0 = 4\,800\) and \(b_1 = -95\) approximately).

\begin{tipBox}{iconmonstr-info-6-240.png}
The \emph{sign} of the slope \(b_1\) and the \emph{sign} of correlation coefficient \(r\) are always the same. For example, if the slope is negative, the correlation coefficient will be negative. However, the \emph{value} of the slope cannot be deduced solely from the value of the correlation coefficient, nor can the \emph{value} of the correlation coefficient be deduced solely from the value of the slope.

\end{tipBox}

\begin{figure}[hbtp]

{\centering \includegraphics[width=0.55\linewidth]{jamovi/Borers/Borers-Regression} 

}

\caption{The regression output for the sorghum data.}\label{fig:BorersRegression}
\end{figure}

\index{Software output!correlation}

\begin{example}[Regression coefficients]
\protect\hypertarget{exm:RegressionCoefficients}{}\label{exm:RegressionCoefficients}The regression equation for the cyclone data (Fig.~\ref{fig:ONIcyclones}) is found from the software output (Fig.~\ref{fig:CyclonesRegressionjamoviCI}) as \[
  \hat{y} = 12.1 - 2.23x,
\] where~\(x\) is the ONI (averaged over October, November, December) and~\(y\) is the number of cyclones; that is, \(b_0 = 12.1\)~cyclones and \(b_1 = -2.23\) cyclones per unit change in ONI. These values are close to the approximations made in Example~\ref{exm:CycloneRegressionGuesses} (\(b_0 = 12\) and \(b_1 = -2.375\) respectively).
\end{example}

\begin{figure}[hbtp]

{\centering \includegraphics[width=0.75\linewidth]{jamovi/Cyclones/Cyclones-RegressionCI} 

}

\caption{The software output for the cyclone data.}\label{fig:CyclonesRegressionjamoviCI}
\end{figure}

\subsection{Regression: making predictions}\label{RegressionForPrediction}

\index{Regression!equation!making predictions}

Regression equations can be used to make \emph{predictions} of the mean value of~\(y\) for a given value of~\(x\). For example, the regression equation for the sorghum data in Equation~\eqref{eq:BorersRegressionEqn} can be used to make \emph{predictions} of the mean yield for a given infestation percentage. For example, the equation can be used to predict the \emph{average} yield of crops with an infestation percentage of \(30\%\). Since~\(x\) represents the infestation percentage, use \(x = 30\) in the regression equation: \begin{align*}
  \hat{y} 
  &= 4\,814.1 - (101.4\times 30)\\
  &= 4\,814.1 - 3\,042 = 1\,772.1.
\end{align*} Crops with an infestation percentage of \(30\%\) are predicted to have a \emph{mean} yield of \(1\,772.1\,\text{kg}.\text{ha}^{-1}\) (though individual crops with an infestation percentage of~\(30\%\) may have smaller or greater yields). The model predicts that the \emph{mean} yield for crops with an infestation percentage of~\(30\%\) will be about \(1\,772.1\,\text{kg}.\text{ha}^{-1}\).

\begin{importantBox}{iconmonstr-warning-8-240.png}
The value of~\(\hat{y}\) is computed using the estimates~\(b_0\) and~\(b_1\), which are computed from sample data. Hence, the value of~\(\hat{y}\) also depends on which one of the countless possible samples is used. This means that~\(\hat{y}\) also has a sampling distribution and a standard error.

\end{importantBox}

Suppose we were interested in crops with an infestation percentage of \(50\%\); the mean yield is \[
  \hat{y} = 4\,814.1 - (101.4.39 \times 50) = -255.9,
\] or about~\(-256\,\text{kg}.\text{ha}^{-1}\), which is clearly silly (negative yields are impossible). In the data, the heaviest infestation is about~\(40\%\), so no data exists beyond a~\(40\%\) infestation percentage. As a result, the regression line does not even apply for infestations exceeding~\(40\%\). (This means that the relationship must be non-linear after \(40\%\).)

Making predictions outside the range of the available data is called \emph{extrapolation}, and \emph{extrapolation} beyond the data may lead to nonsense predictions.

\begin{definition}[Extrapolation]
\protect\hypertarget{def:Extrapolation}{}\label{def:Extrapolation}\emph{Extrapolation} refers to making predictions outside the range of the available data. Extrapolation beyond the data may lead to nonsense.\index{Extrapolation}
\end{definition}

\begin{example}[Extrapolation]
\protect\hypertarget{exm:GirlsHtExtrapolation}{}\label{exm:GirlsHtExtrapolation}The regression equation (Example~\ref{exm:RegressionGirlsHt}) used to predict the mean height of girls~\(\hat{y}\) from their age~\(x\) (for girls aged between four and seven) was given as \[
   \hat{y} = 73 + 7x.
\] For girls five years-of-age (i.e., \(x = 5\)), the predicted mean height is \[
   \hat{y} = 73 + (7\times 5) = 73 + 35 = 108.
\] The heights of girls will vary around a \emph{mean} height of~\(108\,\text{cm}\); some individual girls aged five will be taller than~\(108\,\text{cm}\), and some will be shorter than~\(108\,\text{cm}\).

Using the equation to estimate the height of girls aged~\(21\) would predict a mean height of~\(220\,\text{cm}\). However, this is extrapolation and the prediction is nonsense. Young children grow at a fast rate, but growth rate slows as children age.
\end{example}

\subsection{Regression: understanding relationships}\label{RegressionForUnderstanding}

\index{Regression!equation!for understanding}

The regression equation can be used to \emph{understand} the relationship between the two variables. Consider again the sorghum regression equation: \begin{equation}
   \hat{y}
   = 
   4\,814.1 
   - 
   101.4 x.
   \label{eq:BorersEquation}
\end{equation} What does this equation reveal about the relationship between~\(x\) and~\(y\)?

\(b_0\) is the \emph{predicted} value of~\(\hat{y}\) when \(x = 0\) (Sect.~\ref{Regression-Software}). Using \(x = 0\) in Equation~\eqref{eq:BorersEquation} predicts a mean yield of \[
  \hat{y} = 4\,814.1 - (101.4\times 0) = 4\,814.1
\] for crops with an infestation of zero; this is the value of \(b_1\). Sometimes, using \(x = 0\) is \emph{extrapolating},\index{Extrapolation} as no data exists near \(x = 0\), so sometimes this interpretation of \(b_0\) produces nonsense.

\begin{tipBox}{iconmonstr-info-6-240.png}
The value of the intercept~\(b_0\) is sometimes (but not always) meaningless. The value of the slope~\(b_1\) is usually of greater interest, as it explains the \emph{relationship} between the two variables.

\end{tipBox}

The slope~\(b_1\) quantifies how the value of \(\hat{y}\) changes (on average) when the value of~\(x\) \emph{increases} by one (Sect.~\ref{Regression-Software}). For the sorghum data, \(b_1\) is the change in predicted mean yield for each percentage point\footnote{A `percentage point' increase means a change from, say, \(10\)\% to~\(11\)\%, or \(35\)\% to \(36\)\%.} increase in borer infestation.

Specifically, each extra percentage point of borer infestation is associated with a mean change in yield of \(-101.4\,\text{kg}.\text{ha}^{-1}\) (from Equation~\eqref{eq:BorersEquation}); that is, a \emph{decrease} in yield by a mean of \(101.4\,\text{kg}.\text{ha}^{-1}\) for each extra percentage point of infestation.

To demonstrate, consider the case where \(x = 10\), when the regression equation predicts \(\hat{y} = 3\,800.1\,\text{kg}.\text{ha}^{-1}\). For infestations one percentage point greater than this (i.e., \(x = 11\)), the value of the prediction \(\hat{y}\) will increase by an average of \(-101.4\,\text{kg}.\text{ha}^{-1}\) (or, equivalently, \emph{decrease} by an average of \(101.4\,\text{kg}.\text{ha}^{-1}\)). That is, we would predict \(\hat{y} = 3\,800.1 - 101.4 = 3\,698.7\,\text{kg}.\text{ha}^{-1}\). This is the same prediction made by using \(x = 11\) in Equation~\eqref{eq:BorersEquation}.

\begin{tipBox}{iconmonstr-info-6-240.png}
If the value of~\(b_1\) is \emph{positive}, then the predicted mean values of~\(y\) \emph{increase} as the values of~\(x\) \emph{increase}. If the value of~\(b_1\) is \emph{negative}, then the predicted mean values of~\(y\) \emph{decrease} as the values of~\(x\) \emph{increase}.

\end{tipBox}

This interpretation of \(b_1\) explains the relationship: the predicted mean yield is, on average, about \(101.4\,\text{kg}.\text{ha}^{-1}\) less for each extra percentage point increase of infestation.

\begin{importantBox}{iconmonstr-warning-8-240.png}
In general, we say that a change in the value of~\(x\) is \emph{associated} with a change in the value of~\(\hat{y}\). Unless the study is experimental (Sect.~\ref{ExperimentalStudies}), we cannot say that the change in the value of~\(x\) \emph{causes} the change in the value of~\(\hat{y}\).

\end{importantBox}

\begin{importantBox}{iconmonstr-warning-8-240.png}
If the value of the slope is zero, there is \emph{no linear relationship} between~\(x\) and~\(\hat{y}\). A slope of zero means that a change in the value of~\(x\) is associated with a change of zero in the value of~\(\hat{y}\). In this case, the correlation coefficient is also zero.

\end{importantBox}

\section{\texorpdfstring{Regression: CIs and \(t\)-test for regression parameters}{Regression: CIs and t-test for regression parameters}}\label{RegressionCIHT}

\subsection{Introduction}\label{RegressionTests-Intro}

A regression equation exists in the \emph{population} that connects the values of~\(x\) and~\(\hat{y}\). This regression line is estimated from one of the countless possible samples, and is an estimate of the regression line in the population.

In the \emph{population}, the intercept is denoted by~\(\beta_0\) and the slope by~\(\beta_1\). The values of the parameters~\(\beta_0\) and~\(\beta_1\) are unknown, and are estimated by the statistics~\(b_0\) and~\(b_1\) respectively.

\begin{pronounceBox}{iconmonstr-microphone-7-240.png}

The symbol~\(\beta\) is the Greek letter `beta', pronounced `beater' (as in `egg beater'). So~\(\beta_0\) is pronounced as `beater-zero', and~\(\beta_1\) as `beater-one'.

\end{pronounceBox}

Every sample is likely to produce slightly different values for both~\(b_0\) and~\(b_1\) (sampling variation),\index{Sampling variation} so both~\(b_0\) and~\(b_1\) have a sampling distribution and a standard error. The formulas for computing the values of~\(b_0\) and~\(b_1\) (and their standard errors) are intimidating, so we will use software to perform the calculations. The sampling distributions for \(b_0\) and~\(b_1\) have approximate normal distributions under certain conditions (Sect.~\ref{ValidityCorrelationRegression}).

Usually the slope is of greater interest than the intercept, because the slope explains the \emph{relationship} between the two variables (Sect.~\ref{RegressionForUnderstanding}). For this reason, the sampling distribution for the slope only is given below, but the sampling distribution for the intercept is analogous.

\begin{definition}[Sampling distribution of a sample slope]
\protect\hypertarget{def:Beta1SamplingDistn}{}\label{def:Beta1SamplingDistn}The sampling distribution of the sample regression slope is (when certain conditions are met; Sect.~\ref{ValidityCorrelationRegression}) described by

\begin{itemize}
\tightlist
\item
  an approximate normal distribution,
\item
  with a mean of~\(\beta_1\), and
\item
  a standard deviation, called the \emph{standard error of the slope} and denoted \(\text{s.e.}(b_1)\).
\end{itemize}

A formula exists for finding \(\text{s.e.}(b_1)\), but is tedious to use, and we will not give it.
\end{definition}

\subsection{CIs for the regression parameters}\label{RegressionCI}

\index{Sampling distribution!regression parameters}\index{Confidence intervals!regression parameters|(}

The sampling distribution describes all possible values of the sample slope from all possible samples, through \emph{sampling variation}. For the sorghum data then, the values of the sample slope across all possible samples is described (Fig.~\ref{fig:BorersSlopeSampDistCI}) as, using Def.~\ref{def:Beta1SamplingDistn}:

\begin{itemize}
\tightlist
\item
  an approximate normal distribution,
\item
  with a sampling mean whose value is~\(\beta_1\), and
\item
  a standard deviation of \(\text{s.e.}(b_1) = 8.279\) (from software;\index{Software output!regression} Fig.~\ref{fig:BorersRegression}).
\end{itemize}

\begin{figure}[hbtp]

{\centering \includegraphics[width=0.85\linewidth]{33-CorrelationRegression_files/figure-latex/BorersSlopeSampDistCI-1} 

}

\caption{The distribution of sample slope for the sorghum data, around the population slope\ $\beta_1$.}\label{fig:BorersSlopeSampDistCI}
\end{figure}

Since the sampling distribution is an approximate normal distribution, CIs have the form \[
  \text{statistic} \pm ( \text{multiplier} \times \text{standard error}),
\] where the multiplier is~\(2\) for an \emph{approximate} \(95\)\%~CI (from the \(68\)--\(95\)--\(99.7\) rule). In this context, a CI for the slope is \[
  b_1 \pm \big( \text{multiplier} \times \text{s.e.}(b _1)\big).
\] Thus, an approximate \(95\)\%~CI for the slope is \[
  -101.4 \pm (2\times 8.279)\qquad\text{or}\qquad -101.4 \pm 16.558,
\] which is from~\(-118.0\) to~\(-84.8\,\text{kg}.\text{ha}^{-1}\) (after rounding).

Software can be used to produce \emph{exact} CIs too; the exact \(95\)\% CI is from~\(-118.6\) to~\(-84.3\,\text{kg}.\text{ha}^{-1}\) (Fig.~\ref{fig:BorersRegressionCI}).\index{Software output!regression} The \emph{approximate} and \emph{exact} \(95\)\%~CIs are very similar. We write:

\begin{quote}
For each increase of one percentage point in borer infestation, the mean yield \emph{increases} by~\(-101.4\,\text{kg}.\text{ha}^{-1}\) (\(95\)\%~CI: \(-118.6\) to~\(-84.3\); \(n = 24\)).
\end{quote}

Alternatively (and equivalently, but easier to understand):

\begin{quote}
For each increase of one percentage point in borer infestation, the mean yield \emph{decreases} by~\(101.4\,\text{kg}.\text{ha}^{-1}\) (\(95\)\%~CI: \(84.3\)~to \(118.6\); \(n = 24\)).
\end{quote}

\begin{figure}[hbtp]

{\centering \includegraphics[width=0.75\linewidth]{jamovi/Borers/Borers-Regression-CI} 

}

\caption{Output for the sorghum data, including the CIs for the regression parameters.}\label{fig:BorersRegressionCI}
\end{figure}

\begin{example}[Cyclones]
\protect\hypertarget{exm:CyclonesCI}{}\label{exm:CyclonesCI}Using the software output (Fig.~\ref{fig:CyclonesRegressionjamoviCI}) for the cyclone data, \(\text{s.e.}(b_1) = 0.404\), so the approximate \(95\)\%~CI for the regression slope \(\beta_1\) is \[
   -2.23 \pm (2 \times 0.404)\text{\qquad {or}\qquad} -2.23 \pm 0.808.
\] The interval from~\(-3.04\) to~\(-1.42\) is likely to straddle the population slope. This approximate CI is very similar to the exact CI shown in Fig.~\ref{fig:CyclonesRegressionjamoviCI}.
\end{example}

\index{Confidence intervals!regression parameters|)}

\subsection{\texorpdfstring{Regression: \(t\)-tests for regression parameters}{Regression: t-tests for regression parameters}}\label{RegressionHT}

\index{Hypothesis testing!regression parameters|(}

Since the regression line describing the relationship between~\(x\) and~\(\hat{y}\) is computed from one of countless possible samples, any relationship observed in the sample may be due to sampling variation; possibly, no relationship actually exists in the population (i.e., \(\beta_1 = 0\)). In other words, a hypothesis test can be conducted for the slope to determine if sampling variation can explain the discrepancy between \(\beta_1\) and~\(b_1\). (Similar hypothesis tests can be conducted for testing if the intercept is zero, but are usually of less interest.)

The null hypothesis for tests about the slope is the usual `no relationship' hypothesis. In this context, `no relationship' means that the slope is zero (Sect.~\ref{RegressionForUnderstanding}), so the null hypothesis (about the \emph{population}) is \(H_0\): \(\beta_1 = 0\). A slope of \(\beta_1 = 0\) is equivalent to \emph{no relationship} between the variables. (We would also find \(\rho = 0\).)

For the sorghum data, the RQ implies these hypotheses about the slope: \[
   \text{$H_0$: } \beta_1 = 0\quad\text{and}\quad\text{$H_1$: } \beta_1 < 0.
\] The parameter is~\(\beta_1\), the population slope for the regression equation predicting yield from infestation percentage. The alternative hypothesis is one-tailed, based on the RQ.

Assuming the null hypothesis is true (i.e., that \(\beta_1 = 0\)), the possible values of the sample slope~\(b_1\) can be described (Def.~\ref{def:Beta1SamplingDistn}).

For the sorghum data, the variation in the sample slope across all possible samples when \(\beta_1 = 0\) is described (Fig.~\ref{fig:BorersSlopeSampDist}) using:

\begin{itemize}
\tightlist
\item
  an approximate normal distribution,
\item
  with a sampling mean whose value is \(\beta_1 = 0\) (from \(H_0\)), and
\item
  a standard deviation of \(\text{s.e.}(b_1) = 8.279\) (from software; Fig.~\ref{fig:BorersRegressionCI}).
\end{itemize}

\begin{figure}[hbtp]

{\centering \includegraphics[width=0.85\linewidth]{33-CorrelationRegression_files/figure-latex/BorersSlopeSampDist-1} 

}

\caption{The distribution of sample slopes for the sorghum data, if the population slope is $\beta_1 = 0$.}\label{fig:BorersSlopeSampDist}
\end{figure}

The \emph{observed} sample slope for the sorghum data is \(b_1 = -101.4\). Locating this value on Fig.~\ref{fig:BorersSlopeSampDist} shows that it is \emph{very} unlikely that any of the many possible samples would produce such a slope, just through sampling variation, if the population slope really was \(\beta_1 = 0\). The \emph{test statistic} is found using the usual approach when the sampling distribution has an approximate normal distribution, using a \(t\)-score:\index{Test statistic!t@$t$-score} \begin{align*}
   t 
   &= \frac{\text{observed value} - \text{mean of the distribution of the statistic}}{\text{std deviation of the distribution of the statistic}}\\
   &= \frac{ b_1 - \beta_1}{\text{s.e.}(b_1)} 
    = \frac{-101.4 - 0}{8.279} = -12.25,
\end{align*} where the values of~\(b_1\) and~\(\text{s.e.}(b_1)\) are taken from the software output (Fig.~\ref{fig:BorersRegressionCI}). This \(t\)-score is the same value reported by the software.

To determine if the statistic is \emph{consistent} with the null hypothesis, the \(P\)-value can be approximated using the \(68\)--\(95\)--\(99.7\) rule, approximated using tables, or taken from software output (Fig.~\ref{fig:BorersRegressionCI}). Since \(t = -12.25\), the \(P\)-value will be very small; software shows the \emph{two}-tailed \(P\)-value is \(P < 0.001\) (so the one-tailed \(P\)-value is \(P < 0.0005\)).

We write:

\begin{quote}
The sample presents very strong evidence (\(t = -11.23\); one-tailed \(P < 0.0005\)) that, in the population, the yield of sorghum decreases as infestation percentage increases (slope: \(-101.4\); \(95\)\%~CI from~\(-84.3\) to~\(-118.6\); \(n = 24\)).
\end{quote}

Notice the three features of writing conclusions: an answer to the RQ; evidence to support the conclusion (a \(t\)-score and \(P\)-value); and sample summary information (including a CI).

\begin{importantBox}{iconmonstr-warning-8-240.png}
The \(P\)-value for a test of \(H_0\): \(\rho = 0\) will be the same as the \(P\)-value from a test of \(H_0\): \(\beta_1 = 0\). The test are effectively equivalent, both testing if the relationship observed in the sample can be explained by sampling variation.

\end{importantBox}

\begin{example}[Hypothesis testing]
\protect\hypertarget{exm:ONIRegression}{}\label{exm:ONIRegression}

For the cyclone data (Example~\ref{exm:RegressionCoefficients}), the RQ is:

\begin{quote}
In the Australian region, is there a relationship between ONI and the number of cyclones?
\end{quote}

This RQ implies these hypotheses: \[
   \text{$H_0$: } \beta_1 = 0\quad\text{and}\quad\text{$H_1$: } \beta_1 \ne 0.
\] From the output (Fig.~\ref{fig:CyclonesRegressionjamoviCI}), \(t = -5.52\) and the \(P\)-value is small: \(P < 0.0001\). We write:

\begin{quote}
The sample presents very strong evidence (\(t = -5.52\); two-tailed \(P < 0.0001\)) that, in the population, the number of cyclones is related to the ONI (slope: \(-2.23\); \(95\)\%~CI from~\(-3.04\) to~\(-1.42\); \(n = 37\)).
\end{quote}

\end{example}

\index{Hypothesis testing!regression parameters|)}

\section{Statistical validity conditions}\label{ValidityCorrelationRegression}

\index{Statistical validity (for inference)!correlation coefficient}\index{Statistical validity (for inference)!regression parameters}

As usual, these results hold under certain conditions. The conditions for which the CIs and tests are statistically valid are:

\begin{enumerate}
\def\labelenumi{\arabic{enumi}.}
\tightlist
\item
  The relationship is approximately linear (necessary for the (Pearson) correlation coefficient and regression line to be appropriate).
\item
  The variation in the response variable is approximately constant for all values of the explanatory variable.
\item
  The sample size is at least~\(25\).
\end{enumerate}

The sample size of~\(25\) is a rough figure; some books give other values. The units of analysis are also assumed to be \emph{independent} (e.g., from a simple random sample).

If the relationship non-linear but is increasing-only or decreasing-only, alternatives to the Pearson correlation coefficient include the Spearman\index{Correlation coefficient!Spearman} or Kendall correlation coefficients\index{Correlation coefficient!Kendall} \citep{conover2003practical}. Depending on which statistical validity conditions are not met, other regression-like options may be available. For example, generalised linear models \citep{mypapers:dunnsmyth:glms} may be appropriate for some non-linear relationships and/or relationships with non-constant variation in~\(y\).

\begin{example}[Statistical validity]
\protect\hypertarget{exm:StatisticalValidityBorers}{}\label{exm:StatisticalValidityBorers}For the sorghum data, the scatterplot (Fig.~\ref{fig:BorersScatterjamovi2}, left panel) shows the relationship is approximately linear, so using a (Pearson) correlation coefficient and a regression line is appropriate. For the hypothesis test, the variation in yield doesn't seem to be obviously getting consistently larger or smaller for heavier infestations, and the sample size is only just smaller than~\(25\) (with \(n = 24\)). The CIs and tests are very likely to be statistically valid.
\end{example}

\begin{example}[Cyclones]
\protect\hypertarget{exm:ONIValidity}{}\label{exm:ONIValidity}The scatterplot for the cyclone data (Fig.~\ref{fig:ONIcyclones}) shows the relationship is approximately linear, that the variation in the number of cyclones seems reasonably constant for different values of the ONI, and the sample size is larger than~\(25\) (\(n = 37\)). The CIs (Examples~\ref{exm:CycloneCorrelationCI} and~\ref{exm:CyclonesCI}) and the tests (Example~\ref{exm:CycloneCorrelationTest} and~\ref{exm:ONIRegression}) are statistically valid.
\end{example}

\section{Example: removal efficiency}\label{Removal-Efficiency}

In wastewater treatment facilities, air from biofiltration is passed through a membrane and dissolved in water, and is transformed into harmless by-products. The removal efficiency~\(y\) (in~\%) may depend on the inlet temperature~\(x\) (in~\(^\circ\)C). \citet{chitwood2001treatment} asked:

\begin{quote}
In treating biofiltration wastewater, is the removal efficiency linearly associated with the inlet temperature?
\end{quote}

The scatterplot of the \(n = 32\) observations was shown (and described) in Sect.~\ref{ScatterplotsRemoval-Efficiency}, and repeated here (Fig.~\ref{fig:CorrelationRemovalEfficiency}); the relationship is positive and approximately linear.

The output (Fig.~\ref{fig:OutputRemovalEfficiency}) shows that the sample correlation coefficient is \(r = 0.891\) (with a \(95\)\% CI from \(0.79\) to~\(0.95\)), and so \(R^2 = (0.891)^2 = 79.4\)\%.\index{R@$R^2$} This means that the unexplained variation in removal efficiency reduces by about~\(79.4\)\% by knowing the inlet temperature.

\begin{figure}[hbtp]

{\centering \includegraphics[width=0.8\linewidth]{33-CorrelationRegression_files/figure-latex/CorrelationRemovalEfficiency-1} 

}

\caption{The scatterplot showing the relationship between removal efficiency and inlet temperature.}\label{fig:CorrelationRemovalEfficiency}
\end{figure}

\begin{figure}[hbtp]

{\centering \includegraphics[width=0.45\linewidth]{jamovi/Removal/RemovalCorrelation} \includegraphics[width=0.54\linewidth]{jamovi/Removal/RemovalRegression} 

}

\caption{The software output exploring the relationship between removal efficiency and inlet temperature.}\label{fig:OutputRemovalEfficiency}
\end{figure}

As always, the RQ is about the parameter, the correlation between the removal efficiency and inlet temperature in the population \(\rho\). To test if a linear relationship exists in the population, write: \[
   \text{$H_0$: } \rho = 0\quad\text{and}\quad \text{$H_1$: } \rho \ne 0.
\] The alternative hypothesis is two-tailed (as implied by the RQ). The software output (Fig.~\ref{fig:CorrelationRemovalEfficiency}, right panel) shows that \(P < 0.001\).

The scatterplot of the data (Fig.~\ref{fig:CorrelationRemovalEfficiency}) shows the relationship is approximately linear, so a regression line is appropriate. From the software output (Fig.~\ref{fig:OutputRemovalEfficiency}), \(b_0 = 97.5\) and \(b_1 = 0.0757\); hence \[
  \hat{y} = 97.5 + 0.0757x
\] for~\(x\) and~\(y\) defined above. The slope quantifies the relationship, so we can test \[
   \text{$H_0$: } \beta_1 = 0 \qquad\text{and}\qquad \text{$H_1$: } \beta_1 \ne 0.
\] From the output, \(t = 10.7\) which is huge; the \(P\)-value is small as expected: \(P < 0.001\). The output does not include the CI for the slope, but since \(\text{s.e.}(b_1) = 0.0070\), the \emph{approximate} \(95\)\%~CI is \[
  0.0757 \pm (2 \times 0.0070), \quad\text{ or }\quad 0.0757 \pm 0.0140.
\] We write:

\begin{quote}
Very strong evidence exists (\(t = 10.7\); \(P < 0.001\)) that inlet temperature is linearly related to removal efficiency (slope:~\(0.0757\); approximate \(95\)\%~CI: \(0.0616\) to~\(0.0898\)).
\end{quote}

The CI and test are statistically valid: the relationship is approximately linear, the variation in \(y\) is approximately constant for all values of \(x\), and \(n = 32\).

\section{Chapter summary}\label{Chap34-Summary}

The CI for the correlation coefficient is found from software output. These steps are used to test a hypothesis about a correlation between two variables in the population,~\(\rho\).

\begin{itemize}
\tightlist
\item
  Write the null hypothesis~(\(H_0\): \(\rho = 0\)) and the alternative hypothesis~(\(H_1\)); initially \emph{assume} the value of \(\rho\) in the null hypothesis to be true (usually zero).
\item
  Find the \(P\)-value for the test from software.
\item
  Use the \(P\)-value to make a decision, and write a conclusion.
\item
  Check the statistical validity conditions.
\end{itemize}

\emph{Regression} mathematically describes the relationship between two \emph{quantitative} variables: the response variable~\(y\), and the explanatory variable~\(x\). The linear relationship between~\(x\) and~\(y\) (the \emph{regression equation}), in the sample, is \[
   \hat{y} = b_0 + b_1 x,
\] where~\(b_0\) is a number (the \emph{intercept}), \(b_1\) is a number (the \emph{slope}), and the `hat' above the~\(y\) indicates that the equation gives a \emph{predicted mean} value of~\(y\) for a given \(x\)-value. Software provides the values of \(b_0\) and \(b_1\).

The \emph{intercept} is the predicted mean value of~\(y\) when the value of~\(x\) is zero. The \emph{slope} is how much the predicted mean value of~\(y\) changes, on average, when the value of~\(x\) \emph{increases} by~\(1\).

The regression equation can be used to make \emph{predictions} or to \emph{understand} the relationship between the two variables. Predictions made with values of~\(x\) outside the values of~\(x\) used to create the regression equation (called \emph{extrapolation})\index{Extrapolation} may not be reliable.

To compute a CI for the population slope of a regression equation~\(\beta_1\), software provides the standard error of~\(b_1\); then, the CI is \[
   {b_1} \pm \big( \text{multiplier}\times\text{s.e.}(b_1) \big).
\] The \emph{margin of error} is (multiplier\({}\times{}\)standard error), where the multiplier is~\(2\) for an approximate \(95\)\%~CI (using the \(68\)--\(95\)--\(99.7\) rule).

These steps are used to test a hypothesis about a population slope \(\beta_1\):

\begin{itemize}
\tightlist
\item
  Write the null hypothesis (\(H_0\): \(\beta_1 = 0\)) and the alternative hypothesis~(\(H_1\)); initially \emph{assume} the value of~\(\beta_1\) in the null hypothesis to be true.
\item
  Describe the \emph{sampling distribution}, which describes what to \emph{expect} from the sample slope under this assumption: under certain statistical validity conditions, the sample slope varies with:

  \begin{itemize}
  \tightlist
  \item
    an approximate normal distribution,
  \item
    with sampling mean whose value is \(\beta_1 = 0\) (from~\(H_0\)), and
  \item
    having a standard deviation of \(\displaystyle \text{s.e.}(b_1)\).
  \end{itemize}
\item
  Compute the value of the \emph{test statistic}: \[
   t = \frac{b_1 - \beta_1}{\text{s.e.}(b_1)},
  \] where~\(b_1\) is sample slope.
\item
  The \(t\)-value is like a \(z\)-score, and so an approximate \emph{\(P\)-value} can be approximated using the \(68\)--\(95\)--\(99.7\) rule, or found using software. Use the \(P\)-value to make a decision, and write a conclusion.
\item
  Check the statistical validity conditions.
\end{itemize}

\section{Quick review questions}\label{Chap39-QuickReview}

\citet{data:Telford1991:sexsportsize} examined the relationship between the height and weight of \(n = 37\) rowers at the \emph{Australian Institute of Sport} (AIS; Fig.~\ref{fig:ScatterAISRowers}). The regression equation is \(\hat{y} = -138 + 1.2 x\), and \(P < 0.0001\) for the two-tailed \(P\)-value for a test of the correlation.

\begin{figure}[hbtp]

{\centering \includegraphics{33-CorrelationRegression_files/figure-latex/ScatterAISRowers-1} 

}

\caption{Scatterplot of weight against height for rowers at the AIS.}\label{fig:ScatterAISRowers}
\end{figure}

Are the following statements \emph{true} or \emph{false}?

\begin{enumerate}
\def\labelenumi{\arabic{enumi}.}
\item
  The \(x\)-variable is the height of the rower. \tightlist
\item
  Since the \(P\)-value is small, the correlation is quite strong.
\item
  The relationship is a \emph{positive} relationship.
\item
  Based on the scatterplot, `weight of the rower' is considered the \(y\)-variable. \tightlist
\item
  Using the rise-over-run idea, a very rough estimate of the value of the slope is \(1.2\).
\item
  The measurements units for the slope are kg.
\item
  The measurements units for the intercept are kg.
\item
  The standard error of the slope is \(0.112\), so the value of the \emph{test statistic} to test if the population slope is zero is \(t = 10.7\).
\item
  The \(P\)-value for this test will be \emph{very small}.
\item
  Predicting the mean weight of a \(220\,\text{cm}\)-tall rower would be \emph{extrapolation}.
\end{enumerate}

Select the correct answer:

\begin{enumerate}
\def\labelenumi{\arabic{enumi}.}
\setcounter{enumi}{10}
\item
  What does the `hat' above the \(y\) mean?

  \begin{enumerate}
  \def\labelenumii{\alph{enumii}.}
  \tightlist
  \item
    That the weights are not measured accurately.
  \item
    That the weights are population values.
  \item
    That the regression model gives \emph{poor} estimates.
  \item
    That the regression model gives \emph{good} estimates.
  \item
    That the regression model estimates the weight for a given height.
  \item
    That the regression model estimates the mean weight for a given height.
  \end{enumerate}
\item
  What mean weight is predicted for a rower who is \(180\,\text{cm}\) tall?\\
  \textbf{a.}~\(-24\,624\,\text{kg}\);~~\textbf{b.}~\(78\,\text{kg}\); ~\textbf{c.} \(138\,\text{kg}\).
\end{enumerate}

\section{Exercises}\label{CorrelationExercises}

\hyperref[Answers]{Answers to odd-numbered exercises} are given at the end of the book.

\captionsetup{font=small}

\begin{exercise}
\protect\hypertarget{exr:RegressionGuess}{}\label{exr:RegressionGuess}

For each of the plots in Fig.~\ref{fig:RegressionGuesstimate}, where appropriate:

\begin{enumerate}
\def\labelenumi{\arabic{enumi}.}
\tightlist
\item
  estimate the value of~\(r\) (this is hard!).
\item
  estimate the intercept of the regression line.
\item
  estimate the slope of the regression line, using the rise-over-run idea.
\item
  write down the estimated regression equation.
\end{enumerate}

\end{exercise}

\begin{figure}[hbtp]

{\centering \includegraphics[width=0.9\linewidth]{33-CorrelationRegression_files/figure-latex/RegressionGuesstimate-1} 

}

\caption{Four scatterplots.}\label{fig:RegressionGuesstimate}
\end{figure}

\begin{exercise}
\protect\hypertarget{exr:RegressionExerciseManifold}{}\label{exr:RegressionExerciseManifold}

{[}\emph{Dataset}: \texttt{Throttle}{]} \citet{amin2019robust} measured the throttle angle~(\(x\)) and the manifold air pressure~(\(y\)), as a fraction of the maximum value, in gas engines.

\begin{enumerate}
\def\labelenumi{\arabic{enumi}.}
\tightlist
\item
  The value of~\(r\) is given in the article as~\(0.972986604\). Comment on this, and what it means.
\item
  Comment on the use of a regression model, based on the scatterplot (Fig.~\ref{fig:ThrottleAngle}; reconstructed from \citet{amin2019robust}).
\item
  The authors fitted the following regression model: \(y = 0.009 + 0.458x\). Identify errors that the researchers have made when giving this regression equation.
\item
  Critique the researchers' approach.
\end{enumerate}

\end{exercise}

\begin{figure}[hbtp]

{\centering \includegraphics{33-CorrelationRegression_files/figure-latex/ThrottleAngle-1} 

}

\caption{Manifold air pressure and throttle angle for an internal-combustion gas engine.}\label{fig:ThrottleAngle}
\end{figure}

\begin{exercise}
\protect\hypertarget{exr:CorrelationConsistency1}{}\label{exr:CorrelationConsistency1}

In a correlation analysis, the researchers find that \(P = 0.0002\). Which (if any) of these statements are \emph{consistent} with this \(P\)-value?

\begin{cols}

\begin{col}{0.45\textwidth}

\begin{enumerate}
\def\labelenumi{\arabic{enumi}.}
\tightlist
\item
  \(r = 0.89\).
\item
  \(r = -0.891\).
\end{enumerate}

\end{col}

\begin{col}{0.025\textwidth}
~

\end{col}

\begin{col}{0.45\textwidth}

\begin{enumerate}
\def\labelenumi{\arabic{enumi}.}
\setcounter{enumi}{2}
\tightlist
\item
  \(r = 0.04\).
\item
  \(r = -0.06\).
\end{enumerate}

\end{col}

\end{cols}

\end{exercise}

\begin{exercise}
\protect\hypertarget{exr:CorrelationConsistency2}{}\label{exr:CorrelationConsistency2}

In a correlation analysis, the researchers find that \(r  = 0.36\). Which (if any) of these statements are \emph{consistent} with this value of the correlation coefficient?

\begin{cols}

\begin{col}{0.45\textwidth}

\begin{enumerate}
\def\labelenumi{\arabic{enumi}.}
\tightlist
\item
  The \(P\)-value is very small.
\item
  The \(P\)-value is very large.
\end{enumerate}

\end{col}

\begin{col}{0.025\textwidth}
~

\end{col}

\begin{col}{0.45\textwidth}

\begin{enumerate}
\def\labelenumi{\arabic{enumi}.}
\setcounter{enumi}{2}
\tightlist
\item
  The \(P\)-value is \(0.36\).
\item
  The \(P\)-value is \(0.36^2\), or \(13\)\%.
\end{enumerate}

\end{col}

\end{cols}

\end{exercise}

\begin{exercise}
\protect\hypertarget{exr:RegressionValues}{}\label{exr:RegressionValues}

For each regression equation below, identify the values of~\(b_0\) and~\(b_1\).

\begin{cols}

\begin{col}{0.45\textwidth}

\begin{enumerate}
\def\labelenumi{\arabic{enumi}.}
\tightlist
\item
  \(\hat{y} = 3.5 - 0.14x\).
\item
  \(\hat{y} = -0.0047x + 2.1\).
\end{enumerate}

\end{col}

\begin{col}{0.025\textwidth}
~

\end{col}

\begin{col}{0.45\textwidth}

\begin{enumerate}
\def\labelenumi{\arabic{enumi}.}
\setcounter{enumi}{2}
\tightlist
\item
  \(\hat{y} = -25.2 - 0.95x\).
\item
  \(\hat{y} = -0.22x + 0.15\).
\end{enumerate}

\end{col}

\end{cols}

\end{exercise}

\begin{exercise}
\protect\hypertarget{exr:RegressionValues2}{}\label{exr:RegressionValues2}

For each regression equation below, identify the values of~\(b_0\) and~\(b_1\).

\begin{cols}

\begin{col}{0.45\textwidth}

\begin{enumerate}
\def\labelenumi{\arabic{enumi}.}
\tightlist
\item
  \(\hat{y} = -1.03 +  7.2x\).
\item
  \(\hat{y} = -1.88x -  0.46\).
\end{enumerate}

\end{col}

\begin{col}{0.025\textwidth}
~

\end{col}

\begin{col}{0.45\textwidth}

\begin{enumerate}
\def\labelenumi{\arabic{enumi}.}
\setcounter{enumi}{2}
\tightlist
\item
  \(\hat{y} = 201x + 16\).
\item
  \(\hat{y} = 3.04x -  0.032\).
\end{enumerate}

\end{col}

\end{cols}

\end{exercise}

\begin{exercise}
\protect\hypertarget{exr:PlotAndPoints1}{}\label{exr:PlotAndPoints1}

Draw the regression line \(\hat{y} = 5 + 2x\) for values of~\(x\) between \(0\) and \(10\).

\begin{enumerate}
\def\labelenumi{\arabic{enumi}.}
\tightlist
\item
  Add some points to the scatterplot such that the correlation is approximately \(r = 0.9\).
\item
  Add some more points to the scatterplot such that the correlation is approximately \(r = 0.3\).
\end{enumerate}

\end{exercise}

\begin{exercise}
\protect\hypertarget{exr:PlotAndPoints2}{}\label{exr:PlotAndPoints2}

Draw the regression line \(\hat{y} = 20 - 3x\) for values of~\(x\) between \(0\) and \(5\).

\begin{enumerate}
\def\labelenumi{\arabic{enumi}.}
\tightlist
\item
  Add some points to the scatterplot such that the correlation is approximately \(r = -0.95\).
\item
  Add some more points to the scatterplot such that the correlation is approximately \(r = -0.2\).
\end{enumerate}

\end{exercise}

\begin{exercise}
\protect\hypertarget{exr:CorTestDrug}{}\label{exr:CorTestDrug}

\citet{leblanc2005paramedic} studied \(n = 30\) paramedicine students, using \emph{correlations} to study the relationship between the amount of stress experienced while performing drug-dose calculations (measured using the State--Trait Anxiety Inventory, \textsc{stai}), and length of work experience.

\begin{enumerate}
\def\labelenumi{\arabic{enumi}.}
\tightlist
\item
  Write the hypotheses for testing if a relationship exists between the \textsc{stai} score and the length of work experience.
\item
  The article gives the correlation coefficient as \(r = 0.346\) and \(P = 0.18\). What do you conclude?
\item
  What must be \emph{assumed} for the test to be statistically valid?
\end{enumerate}

\end{exercise}

\begin{exercise}
\protect\hypertarget{exr:CorTestPesticides}{}\label{exr:CorTestPesticides}

\citet{einsiedel2024investigating} used \emph{correlations} to study the relationship between amount of pesticide residue reported on a variety of fresh fruits and vegetables, and various weather measurements. One pesticide studied was perchlorate.

\begin{enumerate}
\def\labelenumi{\arabic{enumi}.}
\tightlist
\item
  Write the hypotheses for testing if a relationship exists between the perchlorate residue and \emph{maximum} temperature at the growing location.
\item
  The article gives the correlation coefficient as \(r = -0.059\) and \(P = 0.035\). What do you conclude?
\item
  Write the hypotheses for testing if a relationship exists between the perchlorate residue and \emph{minimum} temperature at the growing location.
\item
  The article gives the correlation coefficient as \(r = -0.025\) and \(P = 0.365\). What do you conclude?
\item
  What must be \emph{assumed} for the tests to be statistically valid?
\end{enumerate}

\end{exercise}

\begin{exercise}
\protect\hypertarget{exr:CorrelationSoftdrink}{}\label{exr:CorrelationSoftdrink}{[}\emph{Dataset}: \texttt{SDrink}{]} A study examined the time taken to deliver soft drinks to vending machines \citep{others:Montgomery:regressionanalysis} using a sample of size \(n = 25\) (Fig.~\ref{fig:MandibleGestationPlotHT}, left panel). To test if a linear relationship exists, are the statistical validity conditions met?
\end{exercise}

\begin{exercise}
\protect\hypertarget{exr:TwoQuantExercisesMandibleTEST}{}\label{exr:TwoQuantExercisesMandibleTEST}{[}\emph{Dataset}: \texttt{Mandible}{]} \citet{data:royston:mandible} examined the mandible length and gestational age for \(n = 167\) foetuses from the \(12\)th~week of gestation onward (Fig.~\ref{fig:MandibleGestationPlotHT}, right panel). To test if a linear relationship exists, are the statistical validity conditions met?
\end{exercise}

\begin{figure}[hbtp]

{\centering \includegraphics[width=1\linewidth]{33-CorrelationRegression_files/figure-latex/MandibleGestationPlotHT-1} 

}

\caption{Two scatterplots. Left: the time taken to deliver soft drinks to vending machines. Right: gestational age and mandible length. In both plots, the solid line displays the linear relationship.}\label{fig:MandibleGestationPlotHT}
\end{figure}

\begin{exercise}
\protect\hypertarget{exr:RegressionExerciseSunscreen}{}\label{exr:RegressionExerciseSunscreen}

\citet{data:Heerfordt2018:sunscreen} studied the relationship between the time (in minutes) spent on sunscreen application~\(x\), and the amount (in~g) of sunscreen applied~\(y\), using \(n = 31\) people. The fitted regression equation was \(\hat{y} = 0.27 + 2.21x\).

\begin{enumerate}
\def\labelenumi{\arabic{enumi}.}
\tightlist
\item
  Interpret the meaning of~\(b_0\) and~\(b_1\). Do they seem sensible?
\item
  What are the units of measurement for the slope and intercept?
\item
  According to the article, a hypothesis test for testing~\(H_0\): \(\beta_0 = 0\) produced a \(P\)-value \emph{much} larger than~\(0.05\). What does this mean?
\item
  For people who spend \(8\,\text{mins}\) applying sunscreen, how much sunscreen would they use, on average?
\item
  The article reports that \(R^2 = 0.64\). Interpret this value.
\item
  What is the value of the correlation coefficient?
\end{enumerate}

\end{exercise}

\begin{exercise}
\protect\hypertarget{exr:RegressionPredictBirthWeight}{}\label{exr:RegressionPredictBirthWeight}

\citet{bhargava1985mid} stated (p.~\(1\,617\)):

\begin{quote}
In developing countries {[}\ldots{]} logistic problems prevent the weighing of every newborn child. A study was performed to see whether other simpler measurements could be substituted for weight to identify neonates of low birth weight and those at risk.
\end{quote}

One relationship they studied was between infant chest circumference (in~cm)~\(x\) and birth weight (in~grams)~\(y\). The regression equation was given as: \[
   \hat{y} = -3440.2403 + 199.2987x.
\] The correlation coefficient was \(r = 0.8696\) with \(P < 0.001\).

\begin{enumerate}
\def\labelenumi{\arabic{enumi}.}
\tightlist
\item
  Critique the way in which the regression equation and correlation coefficient are reported.
\item
  Based on the \emph{correlation} information, could chest circumference be used as a useful predictor of birth weight? Explain.
\item
  Interpret the intercept and the slope of the regression equation.
\item
  What are the units of measurement for the intercept and slope?
\item
  Predict the mean birth weight of an infant with a chest circumference of~\(30\,\text{cm}\).
\end{enumerate}

\end{exercise}

\begin{exercise}
\protect\hypertarget{exr:CorTestDogs}{}\label{exr:CorTestDogs}

{[}\emph{Dataset}: \texttt{Dogs}{]} \citet{quan2017relation} studied Phu Quoc Ridgeback dogs (\emph{Canis familiaris}), and recorded many measurements of the dogs, including body length and body height. The scatterplot displaying this relationship and the software output are shown in Fig.~\ref{fig:DogsScatter}. In this example, it does not matter which variable is used as~\(x\) or~\(y\).

\begin{enumerate}
\def\labelenumi{\arabic{enumi}.}
\tightlist
\item
  Describe the relationship.
\item
  \emph{Taller} dogs might be expected to be \emph{longer}. To test this, write the hypotheses in terms of correlations.
\item
  Perform the test, using the output. Write a conclusion.
\item
  Is the test statistically valid?
\end{enumerate}

\end{exercise}

\begin{figure}[hbtp]

{\centering \includegraphics[width=0.5\linewidth]{33-CorrelationRegression_files/figure-latex/DogsScatter-1} \includegraphics[width=0.05\linewidth]{OtherImages/SPACER} \includegraphics[width=0.4\linewidth]{jamovi/Dogs/DogsCorrelationCI} 

}

\caption{Phu Quoc ridgeback dogs. Left: a scatterplot of the body height vs length. Right: software output.}\label{fig:DogsScatter}
\end{figure}

\begin{exercise}
\protect\hypertarget{exr:CorrelationExerciseSoil}{}\label{exr:CorrelationExerciseSoil}

{[}\emph{Dataset}: \texttt{Soils}{]} The \emph{California Bearing Ratio}~(CBR) value is used to describe soil sub-grade for flexible pavements (such as in the design of air field runways). \citet{talukdar2014study} examined the relationship between~CBR and other properties of soil, including the plasticity index (PI, a measure of the plasticity of the soil). The scatterplot and software output from \(16\)~different soil samples from Assam, India, are shown in Fig.~\ref{fig:SoilPlotCor}.

\begin{enumerate}
\def\labelenumi{\arabic{enumi}.}
\tightlist
\item
  Describe the plot in words
\item
  Find and interpret the value of~\(R^2\).
\item
  Write down the CI for the correlation coefficient.
\item
  Conduct a hypothesis test for \(\rho\).
\item
  Would the test be statistically valid?
\end{enumerate}

\end{exercise}

\begin{figure}[hbtp]

{\centering \includegraphics[width=0.5\linewidth]{33-CorrelationRegression_files/figure-latex/SoilPlotCor-1} \includegraphics[width=0.05\linewidth]{OtherImages/SPACER} \includegraphics[width=0.4\linewidth]{jamovi/Soils/SoilsCorrelationCI} 

}

\caption{The relationship between CBR and PI in $16$ soil samples. Left: scatterplot. Right: software output.}\label{fig:SoilPlotCor}
\end{figure}

\begin{exercise}
\protect\hypertarget{exr:Apnoea}{}\label{exr:Apnoea}

{[}\emph{Dataset}: \texttt{OSA}{]} \citet{carvalho2020stop} studied obstructive sleep apnoea~(OSA) in \(60\)~adults with Down Syndrome. The response variable \(y\) is OSA severity. The explanatory variable \(x\) is the average number of episodes of sleep disruption (according to specific criteria) per hour of sleep, the Respiratory Event Index (REI). One RQ is:

\begin{quote}
Among Down Syndrome adults, is there a linear relationship between REI and neck size?
\end{quote}

The data are plotted in Fig.~\ref{fig:OSAscatter} (left panel).

\begin{enumerate}
\def\labelenumi{\arabic{enumi}.}
\tightlist
\item
  Using the software output (Fig.~\ref{fig:OSAscatter}), determine the value of \(r\).
\item
  Interpret the value of~\(R^2\).
\item
  Write down the values of the intercept and the slope, and hence the regression equation.
\item
  Explain what the slope in the regression equation means.
\item
  Find an approximate \(95\)\%~CI for the slope.
\item
  Perform a hypothesis to test if a relationship exists between the variables.
\item
  Are the test and CI statistically valid?
\end{enumerate}

\end{exercise}

\begin{figure}[hbtp]

{\centering \includegraphics[width=0.45\linewidth]{33-CorrelationRegression_files/figure-latex/OSAscatter-1} \includegraphics[width=0.54\linewidth]{jamovi/OSA/OSARegression} 

}

\caption{Neck circumference vs REI for Down Syndrome adults. Left: scatterplot. Right: software output.}\label{fig:OSAscatter}
\end{figure}

\begin{exercise}
\protect\hypertarget{exr:EDpatientsCI}{}\label{exr:EDpatientsCI}

{[}\emph{Dataset}: \texttt{EDpatients}{]} \citet{brunette1991correlation} studied the relationship between the number of emergency department~(ED) patients and the number of days following the distribution of monthly welfare monies, from~1986 to~1988 in Minneapolis, MN (Fig.~\ref{fig:EDScatterjamovi}).

\begin{enumerate}
\def\labelenumi{\arabic{enumi}.}
\tightlist
\item
  Write down the estimated regression equation.
\item
  Interpret the slope in the regression equation.
\item
  Find an approximate \(95\)\%~CI for the slope.
\item
  Conduct a hypothesis test for the slope, and explain what the result means.
\item
  What is the value of the correlation coefficient?
\end{enumerate}

\end{exercise}

\begin{figure}[hbtp]

{\centering \includegraphics[width=0.38\linewidth]{33-CorrelationRegression_files/figure-latex/EDScatterjamovi-1} \includegraphics[width=0.61\linewidth]{jamovi/ED/EDRegression-crop} 

}

\caption{The number of emergency department patients, and the number of days since distribution of welfare. Left: scatterplot. Right: software output.}\label{fig:EDScatterjamovi}
\end{figure}

\begin{exercise}
\protect\hypertarget{exr:CorrelationRegressionExerciseBitumen}{}\label{exr:CorrelationRegressionExerciseBitumen}

{[}\emph{Dataset}: \texttt{Bitumen}{]} \citet{data:Panda2018:Bitumen} made \(n = 42\) observations of hot mix asphalt, and measured the volume of air voids and the bitumen content by weight (Fig.~\ref{fig:BitumenPlot}).

\begin{enumerate}
\def\labelenumi{\arabic{enumi}.}
\tightlist
\item
  Describe the plot in words.
\item
  For the data, \(R^2 = 99.29\)\%. Determine, and interpret, the value of~\(r\).
\item
  Write down the regression equation using the software output.
\item
  Interpret what the regression equation means.
\item
  Perform a test to determine if there is a relationship between the variables.
\item
  What is the \(P\)-value for testing~\(H_0\): \(\rho = 0\)?
\item
  Predict the mean percentage of air voids by volume when the percentage bitumen is~\(5.0\)\%. Do you expect this to be a good prediction? Why or why not?
\item
  Predict the mean percentage of air voids by volume when the percentage bitumen is~\(6.0\)\%. Do you expect this to be a good prediction?
\item
  Would the test be statistically valid?
\end{enumerate}

\end{exercise}

\begin{figure}[hbtp]

{\centering \includegraphics[width=0.45\linewidth]{33-CorrelationRegression_files/figure-latex/BitumenPlot-1} \includegraphics[width=0.54\linewidth]{jamovi/Bitumen/Bitumen-Regression} 

}

\caption{Air voids in bitumen. Left: scatterplot. Right: software output}\label{fig:BitumenPlot}
\end{figure}

\begin{exercise}
\protect\hypertarget{exr:CorrelationRegressionExercisePossums}{}\label{exr:CorrelationRegressionExercisePossums}

{[}\emph{Dataset}: \texttt{Possums}{]} \citet{data:Williams2022:Possums} studied Leadbeater's possums in the Victorian Central Highlands. They recorded, among other information, the body weight of the possums (in g) and their location, including the elevation (in m; \texttt{DEM}), as shown in Fig.~\ref{fig:PossumPlot}.

\begin{enumerate}
\def\labelenumi{\arabic{enumi}.}
\tightlist
\item
  The value of \(R^2\) is \(23.0\)\%. Determine, and interpret, the value of~\(r\).
\item
  Write down the regression equation.
\item
  Determine if there is a relationship between the possum weight and the elevation.
\item
  What is the \(P\)-value for a test of \(H_0\): \(\rho = 0\)?
\item
  Interpret the meaning of the slope.
\item
  Predict the mean weight of male possums at an elevation of~\(1\,000\,\text{m}\). Do you expect this to be a good prediction? Why or why not?
\item
  Predict the mean weight of male possums at an elevation of~\(200\,\text{m}\). Do you expect this to be a good prediction? Why or why not?
\end{enumerate}

\end{exercise}

\begin{figure}[hbtp]

{\centering \includegraphics[width=0.45\linewidth]{33-CorrelationRegression_files/figure-latex/PossumPlot-1} \includegraphics[width=0.54\linewidth]{jamovi/Possums/Possums-Regression-jamovi} 

}

\caption{The relationship between weight of possums and the elevation of their location. Left: scatterplot. Right: software output.}\label{fig:PossumPlot}
\end{figure}

\begin{exercise}
\protect\hypertarget{exr:CorrelationExercisesGorillas}{}\label{exr:CorrelationExercisesGorillas}

{[}\emph{Dataset}: \texttt{Gorillas}{]} \citet{wright2021chest} examined \(25\)~gorillas and recorded their chest-beating rates and size (the breadth of the gorillas' backs). The relationship is shown in Fig.~\ref{fig:GorillaPlotCorTest}. Use the software output (Fig.~\ref{fig:GorillajamoviHT}) to study the relationship.

\begin{enumerate}
\def\labelenumi{\arabic{enumi}.}
\tightlist
\item
  Determine the value of~\(r\) and~\(R^2\).
\item
  Perform a hypothesis test for the slope, and write a conclusion.
\item
  Find the regression equation.
\end{enumerate}

\end{exercise}

\begin{figure}[hbtp]

{\centering \includegraphics[width=0.8\linewidth]{33-CorrelationRegression_files/figure-latex/GorillaPlotCorTest-1} 

}

\caption{The scatterplot for the chest-beating data.}\label{fig:GorillaPlotCorTest}
\end{figure}

\begin{figure}[hbtp]

{\centering \includegraphics[width=0.49\linewidth]{jamovi/Gorillas/GorillasCorrelation-jamovi} \includegraphics[width=0.5\linewidth]{jamovi/Gorillas/GorillasRegression-jamovi} 

}

\caption{Software regression output for the gorilla data.}\label{fig:GorillajamoviHT}
\end{figure}

\begin{exercise}
\protect\hypertarget{exr:CorollaPrice}{}\label{exr:CorollaPrice}

{[}\emph{Dataset}: \texttt{Corollas}{]} On 25~June 2014, I searched \emph{Gum Tree} (an Australian online marketplace), for \texttt{Toyota\ Corolla} in the `Cars, Vans \& Utes' category. I recorded the age and the price of each (second-hand) car from the first two pages of results that were returned.

I restricted the data to cars less than~\(14\) years old at the time, removed one \(13\)-year-old Corolla advertised for sale for \$\(390\,000\), then produced the scatterplot in Fig.~\ref{fig:CorollasPriceAgeYear} (left panel).



\begin{figure}[hbtp]

{\centering \includegraphics[width=1\linewidth]{33-CorrelationRegression_files/figure-latex/CorollasPriceAgeYear-1} 

}

\caption{The price of second-hand Toyota Corollas (\(n = 38\)) as advertised on \emph{Gum Tree} on 25~June 2014, plotted against age (left) and year of manufacture (right).}\label{fig:CorollasPriceAgeYear}
\end{figure}

\begin{enumerate}
\def\labelenumi{\arabic{enumi}.}
\tightlist
\item
  Describe the relationship displayed in the graph, in words.
\item
  What else could influence the price of a second-hand Corolla besides the age?
\item
  Consider a seven-year-old Corolla selling for \$\(15\,000\). Would this be cheap or expensive? Explain.
\item
  As stated, I removed one observation: a \(13\)-year-old Corolla for sale at \$\(390\,000\). What do you think the price was meant to be listed as, by looking at the scatterplot? Explain.
\item
  With a ruler or another straight edge (such as a book), draw an estimate of the regression line on the scatterplot. Then, \emph{estimate} the value of~\(b_0\) (the intercept) from the line you drew. What does this mean? Do you think this value is meaningful?
\item
  \emph{Estimate} the value of~\(b_1\) (the slope) from the line you drew. What does this mean? Do you think this value is meaningful?
\item
  From the line you drew above, write down an \emph{estimate} of the regression equation.
\item
  What are the units of the intercept and the slope?
\item
  Use the software output (Fig.~\ref{fig:Corollasjamovi}) relating the price (in thousands of dollars) to age to write down the regression equation.
\item
  Using the software output, write down the value of~\(r\). Using this value of~\(r\), compute the value of~\(R^2\). What does this mean?
\end{enumerate}

\begin{figure}[hbtp]

{\centering \includegraphics[width=0.49\linewidth]{jamovi/Corollas/Corollas-Correlation-jamovi} \includegraphics[width=0.5\linewidth]{jamovi/Corollas/Corollas-Regression-jamovi} 

}

\caption{The jamovi output, analysing the Corolla data}\label{fig:Corollasjamovi}
\end{figure}

\begin{enumerate}
\def\labelenumi{\arabic{enumi}.}
\setcounter{enumi}{10}
\item
  Use the regression equation from the software output to estimate the sale price of a Corolla that is \(20\)-years-old, and explain your answer. \tightlist
\item
  Using the software output, perform a suitable hypothesis test to determine if there is evidence that lower prices are associated with older Corollas.
\item
  Compute an approximate \(95\)\% CI for the population slope (use the software output).
\item
  I could have drawn a scatterplot with Price on the vertical axis and Year of manufacture on the horizontal axis (Fig. \ref{fig:CorollasPriceAgeYear}, right panel). For this graph:

  \begin{enumerate}
  \def\labelenumii{\alph{enumii}.}
  \tightlist
  \item
    What is the value of the correlation coefficient?
  \item
    How would the value of \(R^2\) change (if at all)?
  \item
    How would the value of the slope change (if at all)?
  \item
    How would the value of the intercept change (if at all)?
  \end{enumerate}
\end{enumerate}

\end{exercise}

\begin{exercise}
\protect\hypertarget{exr:ElephantsCor}{}\label{exr:ElephantsCor}

{[}\emph{Dataset}: \texttt{Elephants}{]} Weighing elephants is not easy due to their size. Height (to the shoulder) is easier to measure, and may be a useful proxy for the mass \citep{lalande2022sex}. Two scatterplots of some relevant data \citep{lalande2022sexDATA} are shown in Fig.~\ref{fig:ElephantPlots}.

\begin{enumerate}
\def\labelenumi{\arabic{enumi}.}
\tightlist
\item
  Which graph do you think is for males and which for female elephants? Explain.
\item
  Which plot has a correlation coefficient closest to one? Explain.
\item
  Use software to find the correlation coefficients for each sex.
\item
  For which sex is the height likely to be better for estimating mass? Explain.
\item
  Use software to find the regression equations for predicting mass from height (one for each sex).
\item
  Test to confirm the relationship between mass and height, for each sex.
\item
  Use the regression lines to predict the mass of an elephant with a height of~\(225\,\text{m}\), for each sex.
\item
  Discuss the statistical validity conditions.
\end{enumerate}

\end{exercise}

\begin{figure}[hbtp]

{\centering \includegraphics[width=1\linewidth]{33-CorrelationRegression_files/figure-latex/ElephantPlots-1} 

}

\caption{Mass and height of elephants.}\label{fig:ElephantPlots}
\end{figure}

\begin{exercise}
\protect\hypertarget{exr:JeansCor}{}\label{exr:JeansCor}

{[}\emph{Dataset}: \texttt{Jeans}{]} \citet{PuddingJeans} recorded data on the size of front pockets in men's and women's jeans. This exercise considers the correlation between the maximum widths and maximum heights of front pockets (Fig.~\ref{fig:JeansPocketCorrelations}).

\begin{enumerate}
\def\labelenumi{\arabic{enumi}.}
\tightlist
\item
  The correlation for all jeans is \(r = 0.38\), with \(P = 0.00051\). What does this mean?
\item
  For men's jeans only, the correlation is \(r = -0.09\), with \(P = 0.59\). What does this mean?
\item
  For women's jeans only, the correlation is \(r = 0.14\), with \(P = 0.38\). What does this mean?
\item
  Compute the means for both variables for the combined data, for men's jeans only, and for women's jeans only.
\item
  From the last four questions, how would you describe the relationship between the maximum widths and maximum heights of the front pockets of jeans?\index{Simpson's paradox}\index{Confounding}
\end{enumerate}

\end{exercise}

\begin{figure}[hbtp]

{\centering \includegraphics[width=1\linewidth]{33-CorrelationRegression_files/figure-latex/JeansPocketCorrelations-1} 

}

\caption{The relationships between minimum and maximum heights of front pockets for all jeans (left), men's jeans only (centre) and women's jeans only (right).}\label{fig:JeansPocketCorrelations}
\end{figure}

\begin{exercise}
\protect\hypertarget{exr:RegressionDogsLife}{}\label{exr:RegressionDogsLife}{[}\emph{Dataset}: \texttt{DogsLife}{]} The \texttt{DogsLife} dataset gives the average breed weight and average breed lifetime for \(73\) breeds of dogs. Determine if a relationship exists between breed weight and breed lifetime.
\end{exercise}

\begin{exercise}
\protect\hypertarget{exr:GraphsTypingCor}{}\label{exr:GraphsTypingCor}{[}\emph{Dataset}: \texttt{Typing}{]} The \texttt{Typing} dataset contains four variables: typing speed (\texttt{mTS}), typing accuracy (\texttt{mAcc}), age (\texttt{Age}), and sex (\texttt{Sex}) for \(1\,301\)~students \citep{pinet2022typing}. Is there evidence of a linear relationship between a person's mean typing speed and mean accuracy? Explain.
\end{exercise}

\captionsetup{font=normalsize}

\begin{EOCanswerBox}{iconmonstr-check-mark-14-240.png}
\textbf{Answers to \emph{Quick review} questions:} \textbf{1.} True. \textbf{2.} Not necessarily true. \textbf{3.} True. \textbf{4.} True. \textbf{5.} True (\emph{very} roughly). \textbf{6.} False: kg/cm. \textbf{7.} True. \textbf{8.} True. \textbf{9.} True. \textbf{10.} True. \textbf{11.} f. \textbf{12.} b. \(78\,\text{kg}\).

\end{EOCanswerBox}

\chapter{Selecting an analysis}\label{SelectTest}

\index{Hypothesis testing!selecting}\index{Confidence intervals!selecting}

\begin{cols}
\begin{col}{0.52\textwidth}

\begin{objectivesBox}{iconmonstr-target-4-240.png}
So far, you have learnt about the research process, including analysing data using confidence intervals and conducting hypothesis tests.
\textbf{In this chapter}, you will learn to:
\begin{itemize}\tightlist
  \item
  select the correct analysis.
\end{itemize}
\end{objectivesBox}

\end{col}

\begin{col}{0.03\textwidth}
~
\end{col}

\begin{col}{0.45\textwidth}

\includegraphics[width=0.95\linewidth]{34-Testing-Selecting_files/figure-latex/unnamed-chunk-5-1} 
\end{col}
\end{cols}

\section{About selecting an appropriate analysis}\label{AboutSelectingAnalysis}

Selecting the correct CI or hypothesis test can be challenging, and this book only describes a few possible scenarios. For the situations studied in this book, identifying the \emph{type} of RQ (e.g., descriptive or correlational), and the \emph{number} and \emph{type} of variables (qualitative or quantitative) is important (Table~\ref{tab:InferenceTestCI2}). \emph{Appendix~\ref{StatisticsAndParameters}} may also prove useful.

\begin{table}
\centering
\caption{\label{tab:InferenceTestCI2}Analysis scenarios studied.}
\centering
\fontsize{8}{10}\selectfont
\begin{tabular}[t]{>{\raggedright\arraybackslash}p{28mm}>{\raggedright\arraybackslash}p{35mm}>{\raggedright\arraybackslash}p{29mm}>{\raggedright\arraybackslash}p{31mm}}
\toprule
\multicolumn{2}{c}{\textbf{Summarise}} & \multicolumn{2}{c}{\textbf{Analyse}} \\
\cmidrule(l{3pt}r{3pt}){1-2} \cmidrule(l{3pt}r{3pt}){3-4}
\textbf{Graphical display} & \textbf{Numerical summary} & \textbf{Confidence interval} & \textbf{Hypothesis test}\\
\midrule
\addlinespace[0.3em]
\multicolumn{4}{l}{\textit{Descriptive RQ: proportion in one sample (i.e., one qualitative variable)}}\\
\hspace{1em}\stackunder{Bar chart}{\stackunder{Pie chart}{Dot chart}} & \stackunder{Counts}{\stackunder{Percentages}{Odds}} & For one proportion & One-sample $z$-test for $p$\\
\addlinespace\hspace{1em}(Chap.\ \ref{TwoQuant}) & (Chap.\ \ref{TwoQuant}) & (Chap.\ \ref{CIOneProportion}) & (Chap.\ \ref{TestOneProportion})\\
\addlinespace\addlinespace[0.3em]
\hline
\multicolumn{4}{l}{\textit{Descriptive RQ: mean of one sample (i.e., one quantitative variable)}}\\
\hspace{1em}\stackunder{Histogram}{\stackunder{Stemplot}{Dot chart}} & \stackunder{Means, medians}{\stackunder{Std dev., IQR}{Outliers}} & For one mean & One-sample $t$-test for $\mu$\\
\addlinespace\hspace{1em}(Chap.\ \ref{SummariseQuantData}) & (Chap.\ \ref{SummariseQuantData}) & (Chap.\ \ref{OneMeanConfInterval}) & (Chap.\ \ref{TestOneMean})\\
\addlinespace\addlinespace[0.3em]
\hline
\multicolumn{4}{l}{\textit{Repeated-measures RQ: paired quantitative data}}\\
\hspace{1em}\stackunder{Histogram of}{\stackunder{\quad differences}{Case-profile}} & \stackunder{Mean, median of diffs.}{\stackunder{Std dev., IQR of diffs.}{Outliers, etc.}} & For mean difference & \stackunder{$t$-test for mean}{\quad differences $\mu_d$}\\
\addlinespace\hspace{1em}(Chap.\ \ref{SummariseWithin}) & (Chap.\ \ref{SummariseWithin}) & (Chap.\ \ref{AnalysisPaired}) & (Chap.\ \ref{AnalysisPaired})\\
\addlinespace\addlinespace[0.3em]
\hline
\multicolumn{4}{l}{\textit{Relational RQs: comparing quantitative variables}}\\
\hspace{1em}\stackunder{Boxplot}{\stackunder{Dot chart}{Error bar chart}} & \stackunder{Diff. between means}{\stackunder{Std error of difference}{Summary of both groups}} & \stackunder{For difference between}{\quad two means} & \stackunder{$t$-test for difference}{\stackunder{\quad between two means}{\quad$\mu_1 - \mu_2$}}\\
\addlinespace\hspace{1em}(Chap.\ \ref{BetweenQuantData}) & (Chap.\ \ref{BetweenQuantData}) & (Chap.\ \ref{AnalysisTwoMeans}) & (Chap.\ \ref{AnalysisTwoMeans})\\
\addlinespace\addlinespace[0.3em]
\hline
\multicolumn{4}{l}{\textit{Relational RQs: comparing qualitative variables}}\\
\hspace{1em}\stackunder{Side-by-side bar}{\stackunder{Stacked bar}{Dot chart}} & \stackunder{Odds}{\stackunder{Odds ratio (OR)}{\stackunder{Proportions}{Percentages}}} & \stackunder{For ORs}{\stackunder{For difference}{\stackunder{\quad between two}{\quad proportions}}} & \stackunder{$\chi^2$-test for OR}{\stackunder{$z$-test for difference}{\stackunder{\quad between two}{\quad proportions $p_1 - p_2$}}}\\
\addlinespace\hspace{1em}(Chap.\ \ref{CompareQualData}) & (Chap.\ \ref{CompareQualData}) & (Chap.\ \ref{AnalysisOddsRatio}) & (Chap.\ \ref{AnalysisOddsRatio})\\
\addlinespace\addlinespace[0.3em]
\hline
\multicolumn{4}{l}{\textit{Correlational RQs}}\\
\hspace{1em}Scatterplot & \stackunder{Correlation}{$R^2$} & \stackunder{For correlation}{\stackunder{For regression}{\quad parameters}} & \stackunder{Correlation test}{\stackunder{$t$-test for regression}{\quad parameters}}\\
\addlinespace\hspace{1em}(Chap.\ \ref{TwoQuant}) & (Chap.\ \ref{TwoQuant}) & (Chap.\ \ref{CorrelationRegression}) & (Chap.\ \ref{CorrelationRegression})\\
\bottomrule
\end{tabular}
\end{table}

\begin{example}[Selecting an analysis]
\citet{bjornsson2021effect} studied whether the `presence of a prehospital physician improves survival from cardiac arrest' (p.~227). They studied 471 cardiac arrests: \(200\) treated by prehospital physicians (2004 to~2007), and \(271\) treated by emergency medical technicians (2008 to~2014).

For each cardiac admission (the unit of analysis), two variables are recorded. \emph{Whether a prehospital physician is present} (the explanatory variable) is qualitative with two levels (Yes; No). \emph{Whether a patient survived} (the response variable) is qualitative with two levels (Yes; No). They compared the survival proportions for the two scenarios; this is a \emph{relational RQ}.

To study the \emph{proportion} of survivors for each scenario, a \(z\)-test for the difference between proportions (and corresponding CI) would be used. Alternatively, a \(\chi^2\)-test for comparing the odds of survival (and a CI for the OR) could also be used.
\end{example}

\begin{example}[Selecting an analysis]
\citet{lyons2023female} studied the relationship between the ball release speed (BRS) and the height of female cricket players, and BRS and arm length. Since they are exploring the relationship between two \emph{pairs of quantitative variables}, both RQs are correlational RQs.

If the relationships are approximately linear (determined by examining the two scatterplots), a test for correlations (and corresponding CI) would be used, for one each relationship to be studied. Alternatively, a linear regression model could be fitted (one for each relationship), and a test for the \emph{slope} of the fitted regression equation (and corresponding CI) could be conducted.
\end{example}

\begin{example}[Selecting an analysis]
\citet{hitt2023lead} studied the impact of soil lead levels in New Orleans (USA) neighbourhoods on northern mockingbirds (p.~2):

\begin{quote}
We tested the hypothesis that nestling mockingbird lead levels in blood and feathers differ with respect to neighborhood soil lead levels\ldots{}
\end{quote}

They compared the mean lead concentration in blood, for birds in neighbourhoods with \emph{low} and with \emph{high} lead levels. They also compared the mean lead concentration in feathers, for birds in neighbourhoods with \emph{low} and \emph{high} lead levels. These are both \emph{relational RQs}.

For each bird (the unit of analysis), three variables were recorded. The \emph{lead levels} (the explanatory variable) is qualitative with two levels: high lead-level neighbourhoods, and low lead-level neighbourhoods. The \emph{blood lead concentrations} (one response variable) is quantitative (continuous). The \emph{lead concentrations in feathers} (another response variable) is quantitative (continuous).

To study the difference between the mean lead concentrations in the two groups, a two-sample \(t\)-test for the difference between the means (and the corresponding CI) is needed (provided the statistical validity assumptions are met). One test is needed for comparing blood lead concentrations, and another for comparing concentrations in feathers.
\end{example}

\section{Exercises}\label{SelectTestExercises}

\hyperref[Answers]{Answers to odd-numbered exercises} are given at the end of the book.

\begin{exercise}
\protect\hypertarget{exr:SamplingDistributionA}{}\label{exr:SamplingDistributionA}

Identify which of these statistics \emph{do not} have a sampling distribution well-modelled by a normal distribution. Explain your answer.

\begin{enumerate}
\def\labelenumi{\arabic{enumi}.}
\tightlist
\item
  The difference between two sample means \(\bar{x}_1 - \bar{x}_2\), with samples of size \(n_1 = 55\) and \(n_2 = 61\), but slightly right-skewed distributions of the data for each sample.
\item
  The sample slope in a regression equation \(b_1\), with an approximate linear relationship between the variables, approximately constant variation in the values of~\(y\), and \(n = 24\).
\item
  The sample OR, with both samples of size \(n = 43\).
\end{enumerate}

\end{exercise}

\begin{exercise}
\protect\hypertarget{exr:SamplingDistributionB}{}\label{exr:SamplingDistributionB}

Identify which of these statistics \emph{do not} have a sampling distribution well-modelled by a normal distribution. Explain your answer.

\begin{enumerate}
\def\labelenumi{\arabic{enumi}.}
\tightlist
\item
  The sample mean of a set of differences \(\bar{d}\), with a sample of \(n = 32\) difference, but the distribution of the differences are slightly right-skewed.
\item
  The sample correlation coefficient~\(r\), with an approximate linear relationship between the variables, approximately constant variation in the values of~\(y\), and \(n = 29\).
\item
  The sample proportion~\(\hat{p}\), with \(n = 26\) and \(\hat{p} = 0.154\).
\end{enumerate}

\end{exercise}

\begin{exercise}
\protect\hypertarget{exr:Method1}{}\label{exr:Method1}Suppose researchers compare the mean number of hours of exercise per week for the same British office workers, both in summer and in winter, to study the mean change.

What methods would be a suitable for creating a summary and performing analyses?
\end{exercise}

\begin{exercise}
\protect\hypertarget{exr:Method2}{}\label{exr:Method2}\citet{castro2019sun} estimated the difference between the mean number of hours of sunlight exposure per day for physical education teachers and non-physical education teachers in Spain.

What methods would be a suitable for creating a summary and performing analyses?
\end{exercise}

\begin{exercise}
\protect\hypertarget{exr:Method3}{}\label{exr:Method3}Suppose researchers wanted to study the proportion of koalas that live in regions with tree canopies of different heights (classified as Class~1 (highest canopy height) to Class~4 (lowest canopy height)) in the `core' areas (areas of intensive use and feeding) and non-core areas (based on \citet{mitchell2023remote}).

What methods would be a suitable for creating a summary and performing analyses?
\end{exercise}

\begin{exercise}
\protect\hypertarget{exr:Method4}{}\label{exr:Method4}\citet{data:Chen2018:crocodiles} studied the relationship between the mass and the length of crocodile eggs.

What methods would be a suitable for creating a summary and performing analyses?
\end{exercise}

\begin{exercise}
\protect\hypertarget{exr:Method5}{}\label{exr:Method5}\citet{meadley2021comparison} studied the \emph{relationship} between maximal aerobic capacity (VO\textsubscript{2peak}) while swimming, and the maximal aerobic capacity while running, in helicopter rescue paramedics.

What methods would be a suitable for creating a summary and performing analyses?
\end{exercise}

\begin{exercise}
\protect\hypertarget{exr:Method6}{}\label{exr:Method6}Suppose researchers are wanting to \emph{estimate} the difference between the mean number of hours spent on social media for Indian people aged over~\(30\), to people aged~\(30\) and under.

What methods would be a suitable for creating a summary and performing analyses?
\end{exercise}

\part{Reporting and reading research}\label{part-reporting-and-reading-research}

\chapter{Reporting and writing research}\label{WritingResearch}

\begin{cols}
\begin{col}{0.52\textwidth}

\begin{objectivesBox}{iconmonstr-target-4-240.png}
So far, you have learnt the about the process of research: asking an RQ, designing a study, collecting data, describing and summarising the data, and analysing the data.
\textbf{In this chapter}, you will learn to:
\begin{itemize}\tightlist
  \item
  report research effectively and clearly.
  \item
  appropriately structure research writing.
\end{itemize}
\end{objectivesBox}

\end{col}

\begin{col}{0.03\textwidth}
~
\end{col}

\begin{col}{0.45\textwidth}

\includegraphics[width=0.95\linewidth]{35-Write_files/figure-latex/unnamed-chunk-4-1} 
\end{col}
\end{cols}

\section{Introduction}\label{Chap37-Intro}

Research needs to be effectively communicated and shared, so the results can be used, evaluated and built on by others. The purpose of writing about research is to effectively and clearly communicate.

Research may be shared using face-to-face or online presentations (Sect.~\ref{PreparingPresentations}) or written documents (Sect.~\ref{WritingDocuments}). The style and expectations vary widely between these two formats, between disciplines, and even between journals in the same discipline. Hence, this chapter gives general comments about writing, rather than specific requirements. Formal guidelines for writing about research exist, for both \emph{experimental}\index{Study types!experimental} \citep{hopewell2022update} and \emph{observational}\index{Study types!observational} studies \citep{von2007strengthening}, though we will not discuss these specifically. Since different disciplines and journals have their own styles, read articles from your discipline or target journal for examples of the required style and formatting.

\section{General writing advice}\label{WritingGeneralTips}

The purpose of writing about research is to effectively and clearly communicate the research. With this in mind, some general advice is given below.

\begin{itemize}
\item
  \emph{Write carefully and precisely}. Use simple, clear but technically-correct language. \tightlist  Use the \hyperref[Glossary]{Glossary} if necessary. Carefully choose every word you use to ensure it conveys the correct and intended meaning.\\
  \strut ~~\\
  \citet{oppenheimer2006consequences} concluded, from experiments\index{Study types!experimental}, that students often believe that using fancy words makes them appear smarter. However, he recommended students `write clearly and simply if you can, and you'll be more likely to be thought of as intelligent' (p.~153).
\item
  \emph{Use correct spelling, grammar, punctuation and formatting.} Use (but do not rely upon) a spell checker and grammar checker; use a dictionary. Specifically:

  \begin{itemize}
  \tightlist
  \item
    do \emph{not} confuse similar words (there/their/they're; your/you're; affect/effect; chose/choose; etc.).
  \item
    capitalise correctly.
  \item
    use apostrophes correctly. For example, \emph{it's} is only ever an abbreviation for \emph{it is}.
  \end{itemize}
\item
  \emph{Be inclusive}. Unless specifically referring to men or women, use inclusive language (e.g., `fire-fighter', not `fireman'; `nurse' rather than `male nurse').
\item
  \emph{Take care using comparative terms.} For example, writing `this treatment is \emph{better}' must be clarified. Better than \emph{what}? And `better' in what sense: cost? ease of use? patient outcomes?
\item
  \emph{Use terminology consistently.} Different words may be used for the same concept in research and statistics. Use the term that is common in your discipline; most of all, be consistent.
\item
  \emph{Be clear, concise and complete}. Place material in an Appendix (Sect.~\ref{WritingOther}) if it will interrupt the flow of the narrative. Often, this material can be made available online if too lengthy in printed form.
\item
  \emph{Ensure pronouns clearly identify the nouns they refer to.} For example, consider this sentence: `When the weeds and crops were sprayed, its growth rate reduced by \(80\)\%'. The word \emph{its} may refers to the growth rate of the weeds, the crops, or both.
\item
  \emph{Ensure verbs and nouns agree}. Both the nouns and verbs in a sentence should be singular or plural. For example, `the rats \emph{was} weighed' should be `the rats \emph{were} weighed'. Usually, `data' is considered plural (`datum' is the singular; `dataset' is also singular), so write `the data \emph{were} right skewed' rather than `the data \emph{was} right skewed' (but the latter use is becoming more common). In any case, be consistent.
\item
  \emph{Avoid leaps of logic, and reaching conclusions unsupported by the evidence}. Ensure your conclusions are consistent with the evidence in the study.\\
  \strut ~~\\
  For example, a student project found that the proportion of provisional drivers (those yet to get a full licence) was \emph{higher} in the free university car park, compared to paid car parks. They concluded that provisional drivers seek to `save money by parking in free car parks'. This \emph{may} be true, but is not supported by the evidence. The evidence simply shows a difference in proportions, but does not explain \emph{why}.
\item
  \emph{Present the facts in an unbiased manner, and avoid promoting personal opinions}. For example, do not describe results as `exciting'. Because academic writing generally shuns personal opinions, writing in third person (`the fertiliser was applied') is usually (but not always) preferred over writing in first person (`I applied the fertiliser').
\end{itemize}

Writing well is difficult; editing can be painful; revising is time-consuming. Revise your document carefully as many times as necessary; having someone else read and comment on your writing can be useful.

\begin{importantBox}{iconmonstr-warning-8-240.png}
Many authors have stated variations of this phrase:

\begin{quote}
Don't write so that you \emph{can} be understood; write so that you \emph{can't} be misunderstood.
\end{quote}

Be unambiguous: say what you mean, and mean what you say.

\end{importantBox}

\begin{example}[Write what you mean]
\protect\hypertarget{exm:WriteWell}{}\label{exm:WriteWell}A student project at my university asked:

\begin{quote}
Are dark-coloured car owners more likely to park undercover?
\end{quote}

They actually meant:

\begin{quote}
Are drivers of dark-coloured cars more likely to park undercover?
\end{quote}

Don't just be understood; avoid being \emph{mis}understood!
\end{example}

\section{Ethics when writing}\label{EthicsWhenWriting}

As always, ethical practice is important (Sect.~\ref{Ethics})\index{Ethics}, including when writing about research. Some relevant issues are given below.

\begin{itemize}
\item
  \emph{Producing reproducible research}. When possible, research should be \emph{reproducible} (Sect.~\ref{ReproducibleResearch}).\index{Research!reproducibility} \tightlist This includes describing the protocol,\index{Protocol} and making available any data (when possible; sometime this is not ethical or permitted) and any instructions or code used to analyse the data.
\item
  \emph{Authorship}. Ensure everyone who has made an intellectual contribution is listed as an author. \citet{brand2015beyond} suggests authorship be considered for those involved with:
\end{itemize}

\begin{cols}

\begin{col}{0.03\textwidth}
~

\end{col}

\begin{col}{0.27\textwidth}

\begin{itemize}
\tightlist
\item
  conceptualisation.
\item
  methodology.
\item
  software.
\item
  data analysis.
\item
  investigation.
\item
  resourcing.
\item
  data curation.
\end{itemize}

\end{col}

\begin{col}{0.02\textwidth}
~

\end{col}

\begin{col}{0.65\textwidth}

\begin{itemize}
\tightlist
\item
  creating images or taking photographs.
\item
  writing, including writing drafts, reviewing and editing.
\item
  visualization.
\item
  supervision.
\item
  project administration.
\item
  funding acquisition.
\end{itemize}

\end{col}

\end{cols}

\begin{itemize}
\tightlist
\item
  \emph{Acknowledgements}. An optional \emph{Acknowledgements} section is used to acknowledge research funding bodies, and people who supported the research. Avoid writing `The authors would like to thank\ldots{}'; instead, thank them: `We thank\ldots{}'. Reviewers of the article, when appropriate (who are almost always volunteers), are usually thanked also.
\item
  \emph{Use of artificial intelligence (AI)}.\index{Artificial intelligence} Any use of AI in the study should be disclosed. This includes using AI \emph{during} the research (e.g., generating figures or research design) or when writing \emph{about} the research. The description should indicate where AI was used, which AI systems (such as ChatGPT) were used, and how they were used. AI also may make mistakes, so any material generated using AI should be verified by the authors.
\item
  \emph{Plagiarism}.\index{Plagiarism} Writing about research almost always refers to, and builds on, others' work: to formulate the research question, to establish ideas and to explain the background of the research. However, \emph{plagiarism} (using other people's words and ideas without acknowledgement) \emph{must} be avoided. All sources used when writing research should be acknowledged.\\
  \strut ~~\\
  Plagiarism is a serious offence: it is theft of intellectual property. \emph{Do not plagiarise}; use quotes if necessary and cite the work of others as needed. Plagiarism applies to words, text, images, photographs, ideas, etc.
\end{itemize}

\begin{example}[Plagiarism]
\protect\hypertarget{exm:PlagiarismEG}{}\label{exm:PlagiarismEG}\citet{shamim2014development} published an article to discourage plagiarism. Later, the article was retracted because parts of the article were plagiarised.
\end{example}

\section{Preparing presentations}\label{PreparingPresentations}

\emph{Presentations} are often used to share progress reports of research, or give an overview of completed research. They are used at conferences, workshops, and progress meetings, and may be given to peers, stakeholders, funding bodies, small groups of other researchers, or work teams. Presentations should be adapted to suit the time allocated and the audience: a conference presentation to your research peers should be different from a presentation to a progress meeting.

Presentations are mostly a \emph{verbal} (speaking) and \emph{visual} (preparing slides) medium.

As a \emph{verbal} medium, speak slowly, clearly, loudly, and with expression. Use eye contact, and practice beforehand. Ensure you keep to your allocated time. Ensure technical or unusual words are pronounced correctly; aids to correct pronunciation of many unfamiliar terms have been given in this book.

As a \emph{visual} medium, presentations usually omit technical details and give the audience an overview of the major points and processes; sharing tedious technical details is unlikely to produce an engaging presentation. Presentations usually focus on the \emph{why} and the \emph{what} of the research. Presentations may encourage audience members to learn more by reading your written documents (Sect.~\ref{WritingDocuments}).

Presentations also tend to use graphs, images, short sentences, and minimal text. Presentation software encourages the use of fancy fonts, transitions and animations, but these are usually more distracting than informative; avoid. Ensure your fonts and colours are readable from a distance (especially in tables and graphs).

Using bullet points on slides, while common, is not necessary; short sentences are fine. Slides should \emph{not} contain information that you simply \emph{read} to the audience; a good presenter adds important details around the structure provided by information on the slides. The slides \emph{guide}, but do not have to \emph{tell}, the story of your research.

\section{Writing articles}\label{WritingDocuments}

Written documents are more likely to be formally written and prepared than presentations. Unlike presentations, written documents tend to provide details of \emph{how} the research was conducted. Written documents may be journal articles, progress reports, reports to stakeholders, or funding applications; these are all referred to as `articles' in what follows, for brevity.

\begin{importantBox}{iconmonstr-warning-8-240.png}
Journal articles, and most other written documents too, should contain sufficient details so that other professionals can repeat the study (Chap.~\ref{Protocols}); i.e., the research should, as far as possible, be reproducible (Sect.~\ref{ReproducibleResearch}).\index{Research!reproducibility}

\end{importantBox}

Articles usually have a more formal structure than presentations. Sometimes the acronym AIMRaD\index{AIMRaD} is used to remember these sections:

\begin{itemize}
\tightlist
\item
  \emph{\textbf{A}bstract}.
\item
  \emph{\textbf{I}ntroduction}.
\item
  \emph{\textbf{M}ethods}.
\item
  \emph{\textbf{R}esults}.
\item
  \emph{\textbf{D}iscussion} (or \emph{Summary}, or \emph{Conclusions}).
\end{itemize}

These components capture the six-step research process used in this book (Fig.~\ref{fig:ReportStructure}).



\begin{figure}[hbtp]

{\centering \includegraphics[width=0.6\linewidth]{35-Write_files/figure-latex/ReportStructure-1} 

}

\caption{The connection between the article and the steps studied. The \emph{Abstract} briefly covers all aspects of the study, and the \emph{Discussion} explains what has been learnt through the process, and discusses the results.}\label{fig:ReportStructure}
\end{figure}

\subsection{Article titles}\label{WritingTitles}

Titles are important: poor titles can discourage a reader from reading an article. A title should clearly describe the main purpose of the article. Sometimes this is achieved by posing questions in the title (`Do warning lights and sirens reduce ambulance response times?'; \citet{data:Brown2000:WarningLights}) or providing answers in the title (`No harm from five year ingestion of oats in coeliac disease'; \citet{data:Janatuinen2002:Coeliac}).

As far as possible, avoid overly-specific technical language and uncommon abbreviations in the title.

\subsection{Abstract}\label{WritingAbstract}

The \emph{Abstract} (or \emph{Summary}, or \emph{Overview}) is a short section at the start of an article summarising the \emph{whole} article, including the results; it is \emph{not} an introduction. The \emph{Abstract} is the most important part of any article: it is the only part that many people will read. Some journals require a \emph{structured abstract}, with specific sub-headings (for example, \emph{Introduction}, \emph{Methods}, \emph{Results}, and \emph{Conclusion} (or \emph{Discussion})).

\subsection{Introduction}\label{WritingIntroduction}

The purpose of the \emph{Introduction} is to:

\begin{itemize}
\tightlist
\item
  show how the research fills a gap in existing knowledge, by discussing existing literature (sometimes the \emph{Literature review} forms a separate section).
\item
  gain the interest of readers, and encourage them to read more of the article.
\item
  establish the context and background.
\item
  define the language, acronyms and definitions used in the study.
\item
  introduce the theoretical groundwork of the subject.
\item
  state the purpose of the article: why it was written, and what the authors hope to learn.
\item
  summarise the structure of what follows.
\end{itemize}

\subsection{Methods}\label{WritingMethods}

The \emph{Methods} section (sometimes called \emph{Materials and Methods} or similar) explains how the data were obtained. This includes:

\begin{itemize}
\tightlist
\item
  how the \emph{sample} was identified and located.
\item
  how the data were \emph{collected} from the individuals (the data collection \emph{protocol}).\index{Protocol}
\item
  how the study was designed to maximise external and internal validity, and manage confounding.
\item
  how the data were \emph{analysed}, including the software\index{Computers and software} (and version number) used, and the statistical methods used.
\item
  what specialised equipment was used (pencils, rulers, paper, etc. are not listed).
\end{itemize}

\subsection{Results}\label{WritingResults}

The \emph{Results} summarise the conclusions from the analysis, especially regarding the initial RQ.\spacex The \emph{Results} section:

\begin{itemize}
\tightlist
\item
  shows all the relevant findings from the research.
\item
  presents a summary of the data: the number of observations, the number of missing values, and a verbal description of important variables.
\item
  presents tabular, numerical and/or graphical summaries of the data and relationships of importance.
\item
  gives a brief verbal interpretation of these summaries.
\item
  gives the results from any hypothesis tests and CIs.
\item
  identifies trends, consistencies, anomalies, etc.
\item
  does \emph{not} provide interpretations or explanations of the results (that is the purpose of the \emph{Discussion}).
\end{itemize}

Unless the dataset is small, the data itself is usually not given (though may appear in an Appendix or online).

\begin{importantBox}{iconmonstr-warning-8-240.png}
Cutting-and-pasting software output into reports is rarely acceptable, except for carefully-prepared graphs (Chap.~\ref{SummariseComments}; Sect.~\ref{WritingGraphsTables}).

\end{importantBox}

\subsection{Discussion}\label{WritingDiscussion}

No new information should be presented in this section. This section:

\begin{itemize}
\tightlist
\item
  summarises the results.
\item
  gives a short evaluation of the results.
\item
  answers the stated RQ (i.e., makes a conclusion).
\item
  discusses limitations (Chap.~\ref{Interpretation}), strengths, weaknesses, problems, challenges.
\item
  tries to anticipate and respond to potential questions about the research.
\end{itemize}

Readers should reach the conclusions based on the \emph{evidence} presented. Sometimes, articles have separate \emph{Conclusion} and \emph{Discussion} sections; sometimes they are combined.

\subsection{Referencing}\label{Referencing}

\index{Plagiarism}

The \emph{References} (or \emph{Bibliography}) section gives the full citations of any work referenced, in the required format (such as APA, Harvard, etc.). Most journals have strict guidelines for how references should be listed and formatted (which must be followed).

\subsection{Appendices}\label{WritingOther}

Sometimes an \emph{Appendix} is included, which contains important material that would otherwise break the flow of the article's narrative. The \emph{Appendix} may include large tables, data, images, discussions of technical details, mathematical development, and so on. Sometimes, this material is placed online.

\section{Specific advice}\label{WriteSpecificAdvice}

\subsection{Constructing tables, graphs and images}\label{WritingGraphsTables}

\index{Tables!preparing}\index{Graphs!preparing}\index{Software output!graphs}\index{Graphs!using software}

Good figures and tables take time and care to prepare (Chap.~\ref{SummariseComments}). Their purpose should always be to \emph{display information in the simplest, clearest possible way}, and to highlight the important information. In general, tables, graphs and images:

\begin{itemize}
\tightlist
\item
  \emph{should} be discussed (not just presented) and referred to in the text.
\item
  \emph{should} be clear and uncluttered.
\item
  \emph{should} include units of measurement (such as kg or inches) where appropriate.
\item
  \emph{should} be able to be understood without reference to the article, as far as possible.
\item
  \emph{should} use easy-to-read fonts and colours: for example, ensure the font size is sufficiently large when placed in the article at the final size.
\item
  \emph{should} avoid using different colours, line types or fonts unless these have a purpose (i.e., to differentiate between groups in the study); if they are used, their purpose should be explained (e.g., using a figure legend or caption).
\item
  \emph{should not} include \emph{chart junk} (unnecessary additions), such as artificial third dimensions for graphs (Sect.~\ref{GraphsConstructing}) and unnecessary lines in tables.
\end{itemize}

Figures and images typically have captions \emph{below}, while tables typically have captions \emph{above}. The source of images (e.g., the photographer) should be acknowledged (as ethical practice),\index{Ethics} when appropriate. Tables should use very few horizontal lines, and no vertical lines.

\subsection{Presenting numbers}\label{presenting-numbers}

Any numbers presented should be rounded appropriately. Software may report more decimal places (or more significant figures) than necessary. When appropriate, ensure units of measurement\index{Units of measurement} are given.

Be consistent and careful with decimal numbers. Some journals require numbers to be written with a leading zero (e.g., \(P = 0.024\)), and some do not (e.g., \(P = .024\)). Counts are usually written in words when fewer than ten (or sometimes twelve), and otherwise presented using digits. However, usually numbers are written in full when starting a sentence (`Thirty-seven people volunteered').

\begin{importantBox}{iconmonstr-warning-8-240.png}
Numbers taken from software output may need to be sensibly rounded before being included in a report (including in tables and graphs), and units of measurement\index{Units of measurement} added.

\end{importantBox}

\subsection{Lexically ambiguous words}\label{LexicalAmbiguity}

Readers should not be able to misinterpret your meaning, so write \emph{carefully} and \emph{precisely}. One potential source of confusion is words with a different meaning in research compared to every-day use or in other disciplines (called \emph{lexical ambiguity}; \citet{dunn2016learning}). Some specific words where care is needed are given below.

\begin{itemize}
\tightlist
\item
  \emph{Average}:\index{Averages}\index{Mean!of a sample}\index{Median!of a sample} struct Use the specific word `mean' or `median' when that is what you intend.
\item
  \emph{Confidence}:\index{Confidence intervals} In research, `confidence' is usually used in the phrase `confidence interval' (Sect.~\ref{CIInterpretation}). Take care when using `confidence' in other contexts to avoid confusion.
\item
  \emph{Control}:\index{Control} In research, a `control' is usually used in the context of a control group (Def.~\ref{def:Control}), but may have other meanings in other disciplines.
\item
  \emph{Correlation}:\index{Correlation} In research, correlation describes the (often linear) relationship between two \emph{quantitative} variables (Sect.~\ref{CorrCoefficients}). In general usage, `correlation' may mean any `association' between any two variables.
\item
  \emph{Estimate}:\index{Estimate} In research, `estimating' means to \emph{calculate} an estimate for an unknown population parameter using sample information. In general usage, `estimate' often means to make an educated guess.
\item
  \emph{Experiment}:\index{Study types!experimental} In research, an experiment is a specific type of research study (Sect.~\ref{ExperimentalStudies}). The word `study' can be used to talk about research more generally.
\item
  \emph{Independent}:\index{Independence} This word has many different uses in statistics and research, in science, and in general usage. The word `independent' in this book refers to events that do not impact each other in a probabilistic sense (Sect.~\ref{Independence}).
\item
  \emph{Intervention}:\index{Intervention} In research, an `intervention' (Sect.~\ref{Intervention}) is specifically when the researchers can manipulate the comparison.
\item
  \emph{Normal}:\index{Normal distribution} In research, `normal' often refers to the `normal distribution' (Chap.~\ref{NormalDistribution}). If this is \emph{not} the meaning you intend to convey, consider using the word `usual' or similar.
\item
  \emph{Odds}:\index{Odds}\index{Probability} In research, `odds' has a specific meaning (Sect.~\ref{QualOdds}) and is not the same as probability. In general usage, `probability' and `odds' are often used interchangeably.
\item
  \emph{Population}:\index{Population} In research, the `population' refers to a larger group of interest (Def.~\ref{def:Population}). In general usage, `population' often refers to groups of people specifically.
\item
  \emph{Random}:\index{Random} In research, `random' has a specific meaning: based on impersonal chance. In general usage, it often means `haphazard' or `without structure'.
\item
  \emph{Regression}:\index{Regression} In research, `regression' refers to the mathematical (often linear) relationship between two quantitative variables (Chap.~\ref{CorrelationRegression}).
\item
  \emph{Sample}: \index{Sample} In research, we usually have `\emph{one} sample of \(30\) hyenas'; in some disciplines, this could be described as `taking \(30\) samples of hyenas'.
\item
  \emph{Significant}:\index{Statistical significance}\index{P@$P$-values!one-tailed} In research, `significance' is usually understood to refer to `statistical significance' (Sect.~\ref{AboutPvalues}). If this is \emph{not} the meaning you intend to convey, consider using the word `substantial' or similar.
\item
  \emph{Variable}:\index{Variables} In research, a `variable' is a characteristic that can vary from individual to individual (Def.~\ref{def:Variable}).
\end{itemize}

Some \emph{symbols} may also have different meanings in other disciplines. Ensure the meaning of symbols and notation is clearly defined.

\section{Chapter summary}\label{chapter-summary-2}

Communicating research is a vital step in the research process. Writing clearly is important.

Presentations are a verbal and visual medium, and usually focus on the major points and conclusions rather than the \emph{how}.

Written documents are usually formal, and include details of \emph{what} was done. They should be written carefully and precisely, using the appropriate technically-correct words. Use short sentences for easier reading and omit unnecessary words.

Remember: `Don't write so that you \emph{can} be understood; write so that you \emph{can't} be misunderstood'.

\section{Quick review questions}\label{Chap40-QuickReview}

Are these statements true or false?

\begin{enumerate}
\def\labelenumi{\arabic{enumi}.}
\item
  Using long, obscure words makes writing sound more scientific. \tightlist
\item
  Presentations generally focus on the details of how the study was done.
\item
  The \emph{Introduction} should explain why the study was done.
\item
  Numbers should be given to as many decimal places as possible, for the greatest accuracy.
\item
  The design of the study should be explained in detail in the \emph{Methods} section.
\end{enumerate}

\clearpage

\section{Exercises}\label{WriteExercises}

\hyperref[Answers]{Answers to odd-numbered exercises} are given at the end of the book.

\captionsetup{font=small}

\begin{exercise}
\protect\hypertarget{exr:WriteWordChoice}{}\label{exr:WriteWordChoice}\leavevmode

\begin{enumerate}
\def\labelenumi{\arabic{enumi}.}
\tightlist
\item
  \tightlist  Select the correct word to use to complete this sentence: \emph{to}, \emph{too} or \emph{two}?\\
  `Liquid fertiliser was applied {[}\_\_\_\_\_\_{]} pots each morning at 9am.'
\item
  Select the correct word to use to complete this sentence: \emph{its} or \emph{it's}?\\
  `Each kangaroo was observed for signs that {[}\_\_\_\_\_\_{]} tracking device caused discomfort.'
\item
  What are the problems with this sentence?\\
  `We took \(50\) samples of students; the average age of the \(50\) students was \(26.2\).'
\item
  What is the problem with this text?\\
  `The subjects are not blinded. Because the subjects would clearly know they were in a study.'
\end{enumerate}

\end{exercise}

\begin{exercise}
\protect\hypertarget{exr:WriteWordChoice2}{}\label{exr:WriteWordChoice2}\leavevmode

\begin{enumerate}
\def\labelenumi{\arabic{enumi}.}
\tightlist
\item
  \tightlist  Select the correct word to use to complete this sentence: \emph{there}, \emph{their} or \emph{they're}?\\
  `The subjects were told to eat {[}\_\_\_\_\_\_{]} snacks at about 8am.'
\item
  What is the problem with this text?\\
  `The sample of pedestrians were all taken on a Thursday.'
\item
  Select the correct word to use to complete this sentence: \emph{affect} or \emph{effect}?\\
  `The {[}\_\_\_\_\_\_{]} of the diet was to increase the blood pressure.'
\item
  What is the problem with this sentence?\\
  `The new formulation produces better concrete'.
\end{enumerate}

\end{exercise}

\begin{exercise}
\protect\hypertarget{exr:WriteAmbiguous}{}\label{exr:WriteAmbiguous}\leavevmode

\begin{enumerate}
\def\labelenumi{\arabic{enumi}.}
\tightlist
\item
  Explain how this sentence can be misinterpreted, and write an improved version:\\
  `There was one rat in the cage that was male.'
\item
  Explain how this sentence can be misinterpreted, and write an improved version:\\
  `The research assistant recorded the pH of the lake water in the beaker after removing weeds.'
\end{enumerate}

\end{exercise}

\begin{exercise}
\protect\hypertarget{exr:WriteAmbiguous2}{}\label{exr:WriteAmbiguous2}\leavevmode

\begin{enumerate}
\def\labelenumi{\arabic{enumi}.}
\tightlist
\item
  Explain how this sentence can be misinterpreted, and write an improved version:\\
  `Fertiliser was applied to one of the fields that was in liquid form.'
\item
  Explain how this sentence can be misinterpreted, and write an improved version:\\
  `The new diet lost more weight, on average, than the traditional diet.'
\end{enumerate}

\end{exercise}

\begin{exercise}
\protect\hypertarget{exr:WriteAmbiguous3}{}\label{exr:WriteAmbiguous3}\leavevmode

\begin{enumerate}
\def\labelenumi{\arabic{enumi}.}
\tightlist
\item
  Explain how this statement can be improved:\\
  `A significant change in the weight gain of the pigs is expected to be found'.
\item
  Explain how this statement can be improved:\\
  `The data is summarised in Table~2.'
\end{enumerate}

\end{exercise}

\begin{exercise}
\protect\hypertarget{exr:WriteAmbiguous4}{}\label{exr:WriteAmbiguous4}\leavevmode

\begin{enumerate}
\def\labelenumi{\arabic{enumi}.}
\tightlist
\item
  Explain how this statement can be improved:\\
  `There is a correlation between sex of the person and chance of contracting the disease'.
\item
  Explain how this statement can be improved:\\
  `The group were asked to sign a consent form.'
\end{enumerate}

\end{exercise}

\begin{exercise}
\protect\hypertarget{exr:WriteExercisesDecimals}{}\label{exr:WriteExercisesDecimals}\citet{oyerinde2019investigation} state (p.~1):

\begin{quote}
The regression correlation coefficients of \(0.999996066\) and \(0.999653453\) were obtained for the temperatures and speeds respectively {[}as associated with the time the engine had been running{]}.
\end{quote}

What is the problem with this statement?
\end{exercise}

\begin{exercise}
\protect\hypertarget{exr:WriteExercisesLikelyToDie}{}\label{exr:WriteExercisesLikelyToDie}\citet{david2007patients} published an article entitled `Are patients with self-inflicted injuries more likely to die?' What is the problem with this title?
\end{exercise}

\begin{exercise}
\protect\hypertarget{exr:WriteExercisesStudent1}{}\label{exr:WriteExercisesStudent1}

In a student project, students compared the mean reading speed for people when reading text displayed in one of two different fonts. Their RQ was:

\begin{quote}
Which font allows {[}\ldots{]} students to read a pangram the fastest, between a default and what is considered to be a `easy to read' font.
\end{quote}

(A pangram is a sentence that uses every letter of the alphabet at least once.) In their \emph{Abstract}, the conclusion was given as:

\begin{quote}
The Georgia font {[}\ldots{]} is therefore the faster of the two.
\end{quote}

\begin{enumerate}
\def\labelenumi{\arabic{enumi}.}
\tightlist
\item
  Explain why this is a poorly-worded RQ. Rewrite the RQ.
\item
  Explain what is wrong with the conclusion. Rewrite the statement.
\end{enumerate}

\end{exercise}

\begin{exercise}
\protect\hypertarget{exr:WriteExercisesStudent2}{}\label{exr:WriteExercisesStudent2}In a student project, the heights that students could jump vertically were compared, starting from a squat or standing position. Every student in the study performed both jumps. Critique the \emph{numerical summary} produced by the research team (Table~\ref{tab:WriteExercisesProject2}).
\end{exercise}

\begin{table}
\centering
\caption{\label{tab:WriteExercisesProject2}A numerical summary of the data, showing how much higher the standing jump height is compared to the squat jump.}
\centering
\fontsize{8}{10}\selectfont
\begin{tabular}[t]{l|c|c|c|c|c|c|c}
\hline
\multicolumn{1}{c|}{\textbf{ }} & \multicolumn{1}{c|}{\textbf{ }} & \multicolumn{1}{c|}{\textbf{ }} & \multicolumn{1}{c|}{\textbf{Standard}} & \multicolumn{1}{c|}{\textbf{Standard}} & \multicolumn{1}{c|}{\textbf{Confidence}} & \multicolumn{1}{c|}{\textbf{ }} & \multicolumn{1}{c}{\textbf{ }} \\
\textbf{ } & \textbf{$n$} & \textbf{Mean} & \textbf{deviation} & \textbf{error} & \textbf{interval $95$\%} & \textbf{$t$ value} & \textbf{$P$ value}\\
\hline
 & $50$ & $7.48$ & $4.674$ & $0.661$ & \text{$6.152$ to $8.808$} & $11.316$ & $0.000$\\
\hline
\end{tabular}
\end{table}

\begin{exercise}
\protect\hypertarget{exr:WriteExercisesStudent3}{}\label{exr:WriteExercisesStudent3}The aim of a student project was `to determine if the proportion of males and females that use disposable (coffee) cups on campus is the same'. The two variables observed on each person in the study were (a)~whether the person used a disposable cup, and (b)~the sex of the person. In reporting the results in their \emph{Abstract}, the students state:

\begin{quote}
Based on the sample results, the \(95\)\% confidence interval for the population mean number of disposable cups used by males and females is between \(0.690\) and \(1.625\). Meaning that the population mean is likely to fall between those two intervals.
\end{quote}

Critique this statement.
\end{exercise}

\begin{exercise}
\protect\hypertarget{exr:WriteExercisesStudent4}{}\label{exr:WriteExercisesStudent4}The aim of a student project was `to determine if the average hang time is different between two types of paper plane designs'. The two variables in the study were: design type (Basic Dart; Hunting Flight), and the hang time of the flight of the plane (in seconds). In reporting the results in their \emph{Abstract}, the students state:

\begin{quote}
Very strong evidence proving a difference (\(P = .000\)) between the Basic Dart mean hang time (\(881.84\pm 140.73\,\text{ms}\)) and the Hunting Flight mean hang time (\(1504.19\pm 699.86\,\text{ms}\)). \(95\)\%~CI for the means of The Basic Dart (\(829.29\) -- \(934.39\)) and the Hunting Flight (\(1242.86\) -- \(1765.52\)).
\end{quote}

Critique this statement.
\end{exercise}

\begin{EOCanswerBox}{iconmonstr-check-mark-14-240.png}
\textbf{Answers to \emph{Quick review} questions:} \textbf{1.} False. \textbf{2.} False. \textbf{3.} True. \textbf{4.} False. \textbf{5.} True.

\end{EOCanswerBox}

\chapter{Reading and critiquing research}\label{Reading}

\begin{cols}
\begin{col}{0.52\textwidth}

\begin{objectivesBox}{iconmonstr-target-4-240.png}
So far, you have learnt the about the process of research: asking an RQ, designing a study, collecting data, describing and summarising the data, and analysing the data.
You have learnt how to write about your research.
\textbf{In this chapter}, you will learn to:
\begin{itemize}\tightlist
  \item
  read and critique research.
\end{itemize}
\end{objectivesBox}

\end{col}

\begin{col}{0.03\textwidth}
~
\end{col}

\begin{col}{0.45\textwidth}

\includegraphics[width=0.95\linewidth]{36-Read_files/figure-latex/unnamed-chunk-5-1} 
\end{col}
\end{cols}

\section{Introduction}\label{Chap36-Intro}

All academic disciplines change and adapt. Staying current in your discipline requires reading, critiquing and understanding the research of others, as communicated in \emph{journal articles} or \emph{presentations} (Chap.~\ref{WritingResearch}). (\emph{Critiquing} means to evaluate: identifying what is good, and what can be improved.)

At some time during your studies or employment, you will need to read research articles:

\begin{itemize}
\tightlist
\item
  to understand current practices in your discipline.
\item
  to know \emph{why} your discipline uses the current procedures and practices.
\item
  to learn about new procedures and practices that may be adopted.
\item
  to critique the evidence for current or new practices.
\item
  to identify open or unresolved questions in your discipline.
\item
  to help answer these open or unresolved questions.
\end{itemize}

Familiarity with the language and concepts of research is important for understanding these articles, even if you will not be conducting your own research.

Reading research articles can be challenging. Rather than reading articles thoroughly from start to finish, first read the \emph{Abstract} to obtain an overview of the whole article, without becoming lost in the details. Then, read the \emph{Discussion}, which highlights the important findings. Next, skim the rest of the article (perhaps focusing on graphs and tables of results). Finally, if necessary, read the article for details.

\begin{importantBox}{iconmonstr-warning-8-240.png}
Terminology and notation varies widely in research (Sect.~\ref{LexicalAmbiguity}). When reading research, check the terminology and notation being used if you are unsure!

\end{importantBox}

The six steps of the research process (Sect.~\ref{SixStepsOfResearch}) can guide the research critique.\index{Research!six steps}

\begin{enumerate}
\def\labelenumi{\arabic{enumi}.}
\tightlist
\item
  \emph{Asking the RQ}.
\end{enumerate}

\begin{itemize}
\tightlist
\item
  What RQ is the research answering?
\item
  Why is this RQ important?\index{Research question}
\item
  To what population will the results apply?
\item
  What are the units of analysis\index{Units of analysis} and units of observation?\index{Units of observation}
\item
  How are important terms defined?
\end{itemize}

\begin{enumerate}
\def\labelenumi{\arabic{enumi}.}
\setcounter{enumi}{1}
\tightlist
\item
  \emph{Designing the study}.\index{Research design}
\end{enumerate}

\begin{itemize}
\tightlist
\item
  Is the study observational or experimental?
\item
  Is the study well-designed? What is not explained or clear?
\item
  What design features and used, and why?
\item
  How many individuals are in the study?
\item
  How was the sample obtained? What are the implications for external validity?\index{External validity}
\item
  Is the study designed to maximise internal validity?\index{Internal validity}
\item
  How is confounding managed?\index{Confounding}
\item
  What are the design limitations?
\item
  Are there ethical concerns?\index{Ethics}
\item
  What is the source of funding? Are there potential conflicts of interest?
\end{itemize}

\begin{enumerate}
\def\labelenumi{\arabic{enumi}.}
\setcounter{enumi}{2}
\tightlist
\item
  \emph{Collecting the data}.
\end{enumerate}

\begin{itemize}
\tightlist
\item
  How were the data collected?\index{Data collection}
\item
  Are the necessary details provided so the study can be approximately replicated?
\end{itemize}

\begin{enumerate}
\def\labelenumi{\arabic{enumi}.}
\setcounter{enumi}{3}
\tightlist
\item
  \emph{Classifying and summarising the data}.
\end{enumerate}

\begin{itemize}
\tightlist
\item
  Is the data summary appropriate, complete and clear?
\item
  What does the data summary reveal about the data?
\item
  What do the tables and graphs reveal about the data and relationships?
\end{itemize}

\begin{enumerate}
\def\labelenumi{\arabic{enumi}.}
\setcounter{enumi}{4}
\tightlist
\item
  \emph{Analysing the data}.
\end{enumerate}

\begin{itemize}
\tightlist
\item
  What types of confidence intervals (CIs) and/or hypothesis tests were used?
\item
  Is the analysis appropriate, accurate, valid and clear?
\item
  What do the results mean?
\item
  What software was used?
\item
  Are the results statistically valid?
\end{itemize}

\begin{enumerate}
\def\labelenumi{\arabic{enumi}.}
\setcounter{enumi}{5}
\tightlist
\item
  \emph{Reporting the results}.
\end{enumerate}

\begin{itemize}
\tightlist
\item
  What are the main conclusions, and how do they answer the RQ?
\item
  Are the conclusion consistent with the results?
\item
  Are the results accurate, appropriate and well-reported?
\item
  Are the results of practical importance?
\item
  Are the study limitations acknowledged, and their implications discussed?
\item
  What other questions have emerged?
\end{itemize}

\section{Example: walking while texting}\label{ReadWalkingTexting}

\citet{sajewicz2023texting} studied the impact of texting (on a smartphone) on how students walk (including walking speed). In this section, the article will be briefly discussed.

\subsection{The abstract}\label{ReadWalkingTextingAbstract}

Part of the unstructured abstract for the article reads:

\begin{quote}
The aim of this experiment was to investigate whether using a cell phone while walking affects walking velocity {[}\ldots{]} in young people. Forty-two subjects (\(20\)~males, \(22\)~females; mean age: \(20.74\pm 1.34\) years; mean height: \(173.21\pm 8.07\)~cm; mean weight: \(69.05\pm 14.07\)~kg) participated in the study. The subjects were asked to walk on an FDM--1.5 dynamometer platform four times at a constant comfortable velocity and a fast velocity of their choice. They were asked to continuously type one sentence on a cell phone while walking at the same velocity. The results showed that texting while walking led to a significant reduction in velocity compared to walking without the phone.
\end{quote}

As this is the \emph{abstract}, many details are absent (but explained in the article itself). Nonetheless, a lot can be learnt about the study from the abstract.

\begin{itemize}
\tightlist
\item
  \emph{Asking} the RQ:\index{Research question!repeated-measures}\tightlist

  \begin{itemize}
  \tightlist
  \item
    this is a repeated-measures RQ: data are collected from the same students `four times' (which are explained more fully in the article).
  \item
    the \emph{population} is `young people'.
  \item
    the numbers that follow the \(\pm\) are not explained: are they CI limits, standard deviations, IQRs, ranges, standard errors?
  \item
    the \emph{units of analysis} are the students in the study,\index{Units of analysis}\index{Units of observation} each with four measurements.
  \item
    the main outcome is the (average; presumably mean) `walking velocity'.
  \end{itemize}
\item
  \emph{Designing} the study:

  \begin{itemize}
  \tightlist
  \item
    the sampling method is not stated, but likely to be voluntary-response.
  \item
    the sample size is \(n = 42\) (\(20\)~males; \(22\)~females).
  \end{itemize}
\item
  \emph{Analyse} the data:

  \begin{itemize}
  \tightlist
  \item
    a quantitative variable (walking velocity) is being compared \emph{within} individuals, so paired \(t\)-tests are a likely method of analysis (Chap.~\ref{AnalysisPaired}).
  \end{itemize}
\item
  \emph{Report the results}:

  \begin{itemize}
  \tightlist
  \item
    details of the analysis are not given (e.g., \(P\)-values or CIs).
  \item
    nonetheless, the conclusion is that `texting led to a significant reduction in velocity compared to walking without the phone'.
  \end{itemize}
\end{itemize}

\subsection{Introduction}\label{ReadWalkingTextingIntroduction}

The \emph{Introduction} section introduces the context for the study, and establishes what is known about the topic. The \emph{aim} of the study is (p.~1):

\begin{quote}
\ldots{} to analyze how the use of a cell phone while walking at different velocities affects gait parameters, i.e., velocity, cadence, stride width, and stride length.
\end{quote}

(Cadence refers to the tempo or rhythm of the walking, measure in steps per minute.)

\subsection{Methods}\label{ReadWalkingTextingMethods}

The \emph{Methods} (or \emph{Materials and Methods}) section provides details of the study.

\begin{itemize}
\item
  The \emph{sample}\index{Sample} comprises students from the University School of Physical Education (Poland) studying a course in gait analysis. \tightlist This sample may not represent any general population, though the conclusions may possibly apply to non-students and not-Poles.\index{External validity}
\item
  Exclusion criteria\index{Exclusion criteria} (e.g., using lower limb prostheses) and inclusion criteria\index{Inclusion criteria} (e.g., daily use of a cell phone while walking) were given.
\item
  Extraneous variables\index{Variables!extraneous} collected included age, height, weight and sex.
\item
  The study was ethical,\index{Ethics} with permission sought from the students, and approval given by the Senate Committee on Research Ethics at the university.
\item
  Control variables\index{Variables!control} included the temperature and humidity: `the air temperature was constant at \(22\)~degrees Celsius, and the air humidity was \(47\)\%' (p.~3).
\item
  Details of the specialised equipment used was given: `The experiment was conducted using an FDM--1.5 Zebris dynamography platform'.
\item
  Further details of the protocol\index{Protocol} used were given.
\item
  Each subject participated in four tasks. In each, the subject (without footwear) made as many passes on the platform as possible in one minute:

  \begin{enumerate}
  \def\labelenumi{\arabic{enumi}.}
  \tightlist
  \item
    The subject walked at a constant \emph{comfortable} velocity (i.e., the velocity, chosen by the subject, that the person walks most naturally).
  \item
    The subject walked at a constant \emph{fast} velocity (i.e., as fast as the subject could comfortably walk, chosen by the subject).
  \item
    Task~1 was repeated, with subjects continuously typing a sentence on a cell phone.
  \item
    Task~2 was repeated, with subjects continuously typing a sentence on a cell phone.
  \end{enumerate}
\item
  Five response variables were used: left-side stride length, right-side stride length, cadence, stride width, and walking velocity. These appear to have been measured \emph{objectively}.\index{Data!objective}
\item
  The `sentences' to be typed were defined as `tongue twisters\ldots{} not used in everyday conversation', but were not provided.
\item
  The analysis was given as `paired Student's \(t\)-test'\index{Data!paired}\index{Hypothesis testing!mean difference} in most cases, or the Wilcoxon test\index{Wilcoxon signed ranks test} if the statistical validity conditions were not satisfied.
\item
  The software (and version) used was stated: TIBCO Statistica\textregistered~13.3.0 (StatSoft Poland).
\end{itemize}

\subsection{Results}\label{ReadWalkingTextingResults}

The \emph{Results} section provides the results of the analyses.

\begin{itemize}
\item
  The data are not immediately available (probably due to ethics concerns) but `are available upon request' (p.~7). Details of the analysis were not available.\tightlist
\item
  Case-profile plots\index{Graphs!case-profile plot} were produced for the five response variables, showing the means for each task rather than for all~\(42\) individuals (much like Fig.~\ref{fig:RunningRM}).
\item
  Correlations were computed between the five response variables. The correlations between stride width and the other variables were negative; all other correlations were positive. Since the relationships were non-linear, Spearman correlations\index{Correlation coefficient!Spearman} were used rather than the Pearson correlations studied in Chap.~\ref{CorrelationRegression}. All the corresponding \(P\)-values were less than \(0.05\) apart from the correlation between stride width and cadence.
\item
  The following comparisons were made for each response variable:

  \begin{enumerate}
  \def\labelenumi{\arabic{enumi}.}
  \tightlist
  \item
    Task~1 and Task~3 were compared: this explored the impact of texting on a smartphone when walking at a comfortable velocity.
  \item
    Task~2 and Task~4 were compared: this explored the impact of texting on a smartphone when walking at a fast velocity.
  \item
    Task~3 and Task~4 were compared: this explored the difference between texting and not texting when walking at a fast velocity.
  \end{enumerate}
\item
  The mean and standard deviation of the three response variables were provided for each task. However, the numerical summary for the \emph{differences} were not provided; a \(P\)-value only was provided.
\item
  Fifteen hypothesis tests were conducted: the three between-tasks comparison above, for the five response variables. This increases the chance of making a Type~I error.
\item
  `Statistically significant'\index{Statistical significance} was defined as a \(P\)-value smaller than \(0.017\), rather than the commonly-used \(P < 0.05\). The reason was to reduce the chance of making a Type~I error (Sect.~\ref{TypeErrors}),\index{Type\ I error} but the details are beyond the scope of this book.
\item
  The results are summarised as (p.~5):
\end{itemize}

\begin{quote}
\ldots{} right and left single step length and gait velocity, were found to be statistically significant in each comparison. The value of the change in step width was statistically significant only when comparing trials~1 and~3, and cadence showed statistical significance when comparing trials~2 and~4, as well as~3 and~4.
\end{quote}

\begin{itemize}
\tightlist
\item
  The researchers also made qualitative observations;\index{Research!qualitative} they observed `moments when the subject took their eyes off the phone in order to assess the direction of the path' (p.~5).
\end{itemize}

\subsection{Discussion}\label{ReadWalkingTextingDiscussion}

The \emph{Discussion} section makes the following observations.

\begin{itemize}
\tightlist
\item
  The researchers concluded that `the use of a cell phone while walking significantly affects gait parameters, causing a decrease in walking velocity and a reduction in stride length' (p.~6), thus providing answers to the RQs.
\item
  The researchers state (p.~6) that `This proves that texting on a cell phone has a major impact on gait'. However, the research does \emph{not} prove anything based on one of countless possible samples.
\item
  The study had poor ecological validity,\index{Ecological validity} as the study `was conducted in a measurement workshop room, the air temperature was constant at 22 degrees Celsius, and the air humidity was 47\%'.
\item
  The researchers listed limitations\index{Limitations!research design (effectiveness)} of the study, including:

  \begin{itemize}
  \tightlist
  \item
    that the sentence typed by the subjects was unchanged in both Trials~3 and~4; hence, Trial~4 may have been easier than the previous trial as the sentence was familiar (the carryover effect)\index{Carryover effect}.
  \item
    that the step width differed between comfortable walking speed and walking at the \emph{same} speed with the phone. The researchers attribute this to the carryover effect, as `walking with the phone at comfortable velocity and at fast velocity directly followed one after the other', and so `subjects were able to respond to the new conditions by adapting to them during the first transition'.
  \end{itemize}
\item
  The researchers made recommendations: `that the selection of the order of trials and the sentence to be typed on the cell phone be randomized' (p.~7).\index{Confounding!random allocation}
\end{itemize}

Overall, the study was conducted well and reported well.

\section{Chapter summary}\label{Read-ChapterSummary}

The six steps of research can be used as a scaffold for critiquing research articles. Starting by reading the \emph{Abstract} (or \emph{Summary}, or \emph{Overview}) for an overview, then the \emph{Discussion}, and then skim the rest of the article (perhaps focusing on graphs and tables of results). If necessary, read the article for details.

\section{Quick review questions}\label{Read-QuickReview}

Are these statements true or false?

\begin{enumerate}
\def\labelenumi{\arabic{enumi}.}
\item
  Reading an article thoroughly, from start to finish, is the best approach.\tightlist
\item
  The six steps of research are a useful scaffold for critiquing an article.
\item
  Critiquing an article means to focus on finding all the problems.
\end{enumerate}

\section{Exercises}\label{ReadExercises}

\hyperref[Answers]{Answers to odd-numbered exercises} are given at the end of the book.

\captionsetup{font=small}

\begin{exercise}
\protect\hypertarget{exr:ReadExerciseiPhoneStepCounts}{}\label{exr:ReadExerciseiPhoneStepCounts}

\citet{duncan2018walk} examined the accuracy of step counts, as recorded on iPhones. The article states that participants

\begin{quote}
\ldots{} were recruited through word of mouth and posters displayed around the {[}researcher's{]} university. Participants were eligible if they were ambulatory, \(\ge 18\) years of age, and owned an iPhone 6 {[}\ldots{]} or newer model.
\end{quote}

\begin{enumerate}
\def\labelenumi{\arabic{enumi}.}
\tightlist
\item
  How would you describe the sampling method? What is the implication?
\item
  How would you describe the information given about the subjects needing to be ambulatory and 18 years of age or over?
\end{enumerate}

Although \(33\) participants were selected, the authors note some parts of the study used a smaller sample size because one subject lost their phone, while others chose to withdraw from the study.

\begin{enumerate}
\def\labelenumi{\arabic{enumi}.}
\setcounter{enumi}{2}
\tightlist
\item
  Why did the authors discuss these changes in sample size for some parts of the study?
\end{enumerate}

The article notes that previous studies have been able to:

\begin{quote}
\ldots{} demonstrate the accuracy of the iPhone pedometer function in laboratory test conditions. However, no studies have attempted to evaluate evidence {[}\ldots{]} in the field.
\end{quote}

\begin{enumerate}
\def\labelenumi{\arabic{enumi}.}
\setcounter{enumi}{3}
\tightlist
\item
  Describe the issue that the authors raise with previous studies, using the language in this book.
\item
  Among many other comparisons, the researchers compared the \emph{mean difference} between the number of step counts recorded by manually counting steps (mean: \(92.6\)) and the iPhone-recorded number of steps (mean: \(85.4\)). What statistical test would be appropriate?
\item
  What hypotheses are being tested?
\item
  While walking at \(2.5\,\text{km}\).h\textsuperscript{\(-1\)}, the above statistical test resulted in \(t = 2.95\). What is the approximate \(P\)-value? Interpret the results.
\item
  The sample size for the analysis mentioned above was \(n = 32\). Is the test statistically valid?
\end{enumerate}

\end{exercise}

\begin{exercise}
\protect\hypertarget{exr:ReadExerciseHeadphones}{}\label{exr:ReadExerciseHeadphones}

\citet{mohammadpoorasl2019prevalence} studied the relationship between hearing loss, and headphone and earphone use in Iranian students, using a non-directional study.\index{Study types!directionality!non-directional} The article states:

\begin{quote}
\ldots{} \(890\) students were randomly selected from five schools at \textsc{qums} {[}\ldots{]} using a proportional cluster sampling method\ldots{}
\end{quote}

Only \(866\) of the \(890\) students agreed to participated in the study; of these, \(745\) used \emph{ear}phones. The participants completed a hearing test and a Hearing Loss Questionnaire (HLQ; values between \(17\) and \(34\); higher scores indicating more severe hearing loss).

\begin{enumerate}
\def\labelenumi{\arabic{enumi}.}
\tightlist
\item
  What is the population?
\item
  Is this an observational or experimental study?
\item
  Critique the sampling method. What is the implication for interpreting the results of the study?
\end{enumerate}

One question in the HLQ is:

\begin{quote}
Does a hearing problem cause you difficulty when listening to TV or radio?
\end{quote}

\begin{enumerate}
\def\labelenumi{\arabic{enumi}.}
\setcounter{enumi}{3}
\tightlist
\item
  What is a potential problem with this question?
\item
  Compute the \(95\)\% CI for the proportion of students who had used earphones.
\end{enumerate}

Some results are presented in Table~\ref{tab:HearingLossTable}.

\begin{enumerate}
\def\labelenumi{\arabic{enumi}.}
\setcounter{enumi}{5}
\tightlist
\item
  What statistical test was appropriate for comparing the mean scores for males and females?
\item
  What are the hypotheses being tested?
\item
  What is the standard error for the \emph{difference} between the means?
\item
  Perform the hypothesis tests; what do the results mean?
\item
  Compute the approximate \(95\)\% CI for the difference between the means.
\item
  Are the test and the CI statistically valid?
\end{enumerate}

Table~\ref{tab:HearingLossTable} also compares the HLQ scores for the frequency of \emph{earphone} use specifically.

\begin{enumerate}
\def\labelenumi{\arabic{enumi}.}
\setcounter{enumi}{11}
\tightlist
\item
  What are the hypotheses being tested?
\item
  Why is the sample size for this comparison only \(791\) and not \(845\)?
\item
  Interpret the \(P\)-value for this test; what do the results mean?
\end{enumerate}

Table~\ref{tab:HearingLossTable} also compares the HLQ scores for those who use and do not use \emph{earphones}.

\begin{enumerate}
\def\labelenumi{\arabic{enumi}.}
\setcounter{enumi}{14}
\tightlist
\item
  Form an approximate \(95\)\%~CI for the mean hearing loss score for students who use earphones.
\item
  Compute the standard error of the \emph{difference} between the mean hearing loss score for students who use and do earphones.
\item
  Perform a hypothesis test to compare the \emph{difference} between the mean hearing loss score for students who use and do not use earphones, and confirm that the \(P\)-value is indeed very small.
\end{enumerate}

\end{exercise}

\begin{table}
\centering
\caption{\label{tab:HearingLossTable}The Hearing Loss Questionnaire scores for various demographic variables.}
\centering
\fontsize{8}{10}\selectfont
\begin{tabular}[t]{rcccc}
\toprule
\multicolumn{2}{c}{\textbf{ }} & \multicolumn{2}{c}{\textbf{HLQ}} & \multicolumn{1}{c}{\textbf{ }} \\
\cmidrule(l{3pt}r{3pt}){3-4}
\textbf{Levels} & \textbf{Sample size} & \textbf{Mean} & \textbf{Std dev.} & \textbf{$P$-value}\\
\midrule
\addlinespace[0.3em]
\multicolumn{5}{l}{\textbf{Sex}}\\
\hspace{1em}Female & $543$ & $19.37$ & $2.91$ & $0.009$\\
\hspace{1em}Male & $302$ & $19.99$ & $3.51$ & \\
\addlinespace[0.3em]
\multicolumn{5}{l}{\textbf{Frequency of earphone use}}\\
\hspace{1em}$0$, $1$ times/day & $194$ & $19.20$ & $2.87$ & $0.001$\\
\hspace{1em}$2$ to $3$ times/day & $319$ & $19.60$ & $2.66$ & \\
\hspace{1em}More than $3$ times/day & $278$ & $20.20$ & $3.54$ & \\
\addlinespace[0.3em]
\multicolumn{5}{l}{\textbf{Earphone use}}\\
\hspace{1em}Yes & $745$ & $19.80$ & $3.08$ & $< 0.001$\\
\hspace{1em}No & $100$ & $19.00$ & $1.71$ & \\
\bottomrule
\end{tabular}
\end{table}

\begin{exercise}
\protect\hypertarget{exr:ReadExerciseQuakes}{}\label{exr:ReadExerciseQuakes}

\citet{mesrkanlou2023effect} studied the effect of an earthquake on pregnant mothers in Varzaghan, Iran (p.~2), using:

\begin{quote}
\ldots{} \(1000\) cases of pregnant women living in urban and rural areas of Varzaghan city that consisted of \(550\) pre-earthquake and \(450\) post-earthquake cases.
\end{quote}

The researchers compared the mothers in the two groups (pre- and post-earthquake) on various measurements. For example, the mean age of mothers in the \emph{pre}-group was \(25.82\)~y, and the mean age of the mothers in the \emph{post}-group was \(26.71\)~y; the difference has a \(P\)-value of \(0.084\).

\begin{enumerate}
\def\labelenumi{\arabic{enumi}.}
\tightlist
\item
  What does this result \emph{mean}?
\item
  \emph{Why} did the researchers make this comparison between the mothers' ages in two groups?
\item
  What type of hypothesis test was used to make this conclusion?
\end{enumerate}

The researchers also compared the mean birth weights of the babies born to the mothers in the two groups. In the \emph{pre}-group, the mean birth weight was \(3.25\,\text{kg}\) (\(s = 0.52\)) and in the \emph{post}-group the mean birth weight was \(3.18\,\text{kg}\) (\(s = 0.54\)).

\begin{enumerate}
\def\labelenumi{\arabic{enumi}.}
\setcounter{enumi}{3}
\tightlist
\item
  Compute the standard error for comparing the \emph{difference} between the two means.
\item
  Perform a hypothesis test to compare the mean birthweights. Interpret the results.
\item
  The two-tailed \(P\)-value for this test as given as \(0.001\). Is this consistent with your calculations?
\end{enumerate}

The researchers also compared the percentage of babies with a Low Birth Weight (LBW; less than \(2.5\,\text{kg}\)). For the \emph{pre}-group, the percentage was \(6.01\)\%; for the \emph{post}-group, the percentage was \(8.92\)\%.

\begin{enumerate}
\def\labelenumi{\arabic{enumi}.}
\setcounter{enumi}{6}
\tightlist
\item
  What \emph{type} of definition is given for LBW?
\item
  Construct the \(2\times 2\) table for displaying these data.
\item
  What type of test was probably used for this comparison?
\item
  For the test, \(\chi^2 = 3.052\). Deduce the equivalent \(z\)-score and the approximate \(P\)-value.
\item
  What limitations can you identify for this study?
\end{enumerate}

\end{exercise}

\begin{exercise}
\protect\hypertarget{exr:ReadExerciseSelinium}{}\label{exr:ReadExerciseSelinium}

\citet{tracy1990selenium} studied the selenium (Se) concentration in irrigation and stock water sources in California. For drinking water, the maximum recommended concentration was \(10\,\ensuremath{\mu}\text{g}.\text{L}^{-1}\); for irrigation water, the maximum recommended concentration was \(20\,\ensuremath{\mu}\text{g}.\text{L}^{-1}\).

Part of the study examined the area within \(5\,\text{km}\) of wells. When Pliocene rocks were within this radius, the relationship between the Se concentration \(y\) in the water and the electrical conductivity of the water \(x\) (in deciSiemens per meter, dS.m\textsuperscript{\(-1\)}) was \(\hat{y} = -3.1 + 7.0x\), where \(R^2 = 27\)\%.

\begin{enumerate}
\def\labelenumi{\arabic{enumi}.}
\tightlist
\item
  Interpret the meaning of \(R^2\).
\item
  What is the value of the correlation coefficient?
\item
  The \(P\)-value for testing the slope is \(P < 0.001\). Interpret what this means in this context.
\item
  What are the measurement units of the slope?
\end{enumerate}

For the \(n = 151\) wells in the study, Table~\ref{tab:seleniumTable} shows the Se concentration of the water and the geology within \(5\,\text{km}\) of the well.

\begin{enumerate}
\def\labelenumi{\arabic{enumi}.}
\setcounter{enumi}{4}
\tightlist
\item
  What hypotheses are being tested by the table?
\item
  The article states that \(\chi^2 = 31.5\). What is the equivalent \(z\)-score for the test?
\item
  What is the approximate \(P\)-value for the test? Interpret what this means.
\end{enumerate}

\end{exercise}

\begin{table}
\centering
\caption{\label{tab:seleniumTable}Number of wells with dissolved selenium (Se) concentration above $2\,\ensuremath{\mu}\text{g}.\text{L}^{-1}$, and the geology within $5\,\text{km}$.}
\centering
\fontsize{8}{10}\selectfont
\begin{tabular}[t]{>{}lcc}
\toprule
\textbf{ } & \textbf{Plioscene rocks present} & \textbf{Plioscene rocks not present}\\
\midrule
\textbf{Se concentration $\le 2$ $\,\ensuremath{\mu}\text{g}.\text{L}^{-1}$} & $78$ & $15$\\
\textbf{Se concentration $> 2\,\ensuremath{\mu}\text{g}.\text{L}^{-1}$} & $23$ & $35$\\
\bottomrule
\end{tabular}
\end{table}

\begin{exercise}
\protect\hypertarget{exr:ReadExerciseLarva}{}\label{exr:ReadExerciseLarva}

\citet{russell2023difference} compared the larvae of two types of mosquitoes: \emph{Ae. albopictus} (invasive) and \emph{Cx. pipiens} (native). One study compared the survival rates of the larvae at two temperatures, and in the presence or absence of predator (a small crustacean, called copepods).

At \(15\)\textsuperscript{o}C and \(25\)\textsuperscript{o}C, the survival rates were \(86.8\)\% and \(86.1\)\%, respectively, for the control group (i.e., no copepods). The papers quoted a \(P\)-value of \(P =  0.8076\).

\begin{enumerate}
\def\labelenumi{\arabic{enumi}.}
\tightlist
\item
  What type of test was probably used?
\item
  Interpret what the \(P\)-value means in this context.
\end{enumerate}

The researchers also compared the size of the surviving larvae in the control groups for both temperatures. In comparing \emph{Cx. pipiens} to \emph{Ae. albopictus} larvae, the paper gives this information:

\begin{quote}
\(\text{mean}\pm\text{SD}\): \emph{Cx. pipiens}\({} = 1.64 \pm 0.18\,\text{mm}\), \emph{Ae. albopictus}\({} = 1.36 \pm 0.13\,\text{mm}\); \(\text{$p$-value} =< .0001\).
\end{quote}

The two sample sizes are \(n = 410\) and \(n = 498\) respectively.

\begin{enumerate}
\def\labelenumi{\arabic{enumi}.}
\setcounter{enumi}{2}
\tightlist
\item
  How would these results be interpreted?
\item
  What type of test would probably have been used?
\item
  Compute the standard error for the difference between the two types of mosquitoes.
\item
  Compute the \(t\)-score and approximate \(P\)-value for the test. What does the mean?
\item
  Is the \(P\)-value in the article consistent with your calculations?
\item
  Is the test statistically valid?
\end{enumerate}

The length of the surviving larvae from both species were compared for the two temperatures also.. For surviving \emph{Cx. pipiens}, the paper reports:

\begin{quote}
\(\text{mean} \pm \text{SD}\): \(15\)\textsuperscript{o}C \({}= 1.66 \pm 0.01\,\text{mm}\), \(25\)\textsuperscript{o}C \({} = 1.60 \pm 0.02\,\text{mm}\); \(\text{$p$-value} = .0065\).
\end{quote}

For surviving \emph{Ae. albopictus}, the paper reports:

\begin{quote}
\(\text{mean} \pm \text{SD}\): \(15\)\textsuperscript{o}C \({}= 1.35 \pm 0.01\,\text{mm}\), \(25\)\textsuperscript{o}C \({} = 1.36 \pm 0.01\,\text{mm}\); \(\text{$p$-value} = .4343\).
\end{quote}

\begin{enumerate}
\def\labelenumi{\arabic{enumi}.}
\setcounter{enumi}{8}
\tightlist
\item
  How would these results be interpreted?
\item
  What type of test would probably have been used?
\end{enumerate}

\emph{Megacyclops viridis} (a copepod) preys on the larvae. The association between predation efficiency (\(y\); a percentage) and predator--prey size-ratio (\(x\); no units) was (using \(n = 45\)) \(\hat{y} = -19.56 + 31.64x\). The standard errors of the regression coefficients were \(17.92\) (intercept) and \(13.88\) (slope).

\begin{enumerate}
\def\labelenumi{\arabic{enumi}.}
\setcounter{enumi}{10}
\tightlist
\item
  Find an approximate \(95\)\% CI for each regression parameter.
\item
  Estimate the \(P\)-value for testing if the population slope is zero. Interpret what this means.
\item
  Is this test statistically valid?
\item
  Interpret the meaning of the slope.
\item
  The value of \(R^2\) was given as \(0.087\) (i.e., \(8.7\)\%). Interpret this value.
\item
  Find the value of the correlation coefficient, \(r\).
\end{enumerate}

\end{exercise}

\begin{exercise}
\protect\hypertarget{exr:ReadExerciseMouth}{}\label{exr:ReadExerciseMouth}

\citet{li2017normal} studied the maximum mouth opening (MMO; in mm) for \(452\) Chinese adults aged from \(20\) to~\(35\).

\begin{enumerate}
\def\labelenumi{\arabic{enumi}.}
\tightlist
\item
  Would the individuals in the study have been blinded? Explain. What are the implications?
\end{enumerate}

The correlation between height and MMO was given as \(r = 0.54\) with \(P < 0.001\).

\begin{enumerate}
\def\labelenumi{\arabic{enumi}.}
\setcounter{enumi}{1}
\tightlist
\item
  What does this \emph{mean}?
\item
  Compute and interpret the value of \(R^2\).
\end{enumerate}

The regression equation relating the height \(x\) (in cm) and MMO \(y\) was given as \(\hat{y} = 0.36x - 10.15\).

\begin{enumerate}
\def\labelenumi{\arabic{enumi}.}
\setcounter{enumi}{3}
\tightlist
\item
  Interpret the estimates of the regression parameters.
\item
  Use the regression equation to predict the MMO for a person \(179\,\text{cm}\) tall.
\end{enumerate}

The mean MMO of males was \(54.18\,\text{mm}\) (\(s = 5.21\)), and for females was \(49.62\,\text{mm}\) (\(s = 3.69\)).

\begin{enumerate}
\def\labelenumi{\arabic{enumi}.}
\setcounter{enumi}{5}
\tightlist
\item
  What \emph{type} of hypothesis tests was used to compare the mean MMO for males and females?
\item
  The \(t\)-score for comparing MMO for males and females is \(t = 10.63\). What is the \(P\)-value?
\item
  Is this result statistically valid?
\item
  What is the meaning of this comparison?
\item
  Is gender likely to be a confounding variable in this regression analysis? Explain carefully.
\end{enumerate}

The authors state one of the limitations as:

\begin{quote}
First, participants were recruited from a pool of people who were undergoing regular medical examinations in our hospital {[}\ldots{]}
\end{quote}

\begin{enumerate}
\def\labelenumi{\arabic{enumi}.}
\setcounter{enumi}{10}
\tightlist
\item
  What does this mean? What are the implications? Are there other limitations?
\end{enumerate}

\end{exercise}

\begin{exercise}
\protect\hypertarget{exr:ReadExerciseTomatoes}{}\label{exr:ReadExerciseTomatoes}

\citet{drinkwater1995fundamental} compared tomatoes growing on conventional (CNV; \(n = 14\)) and organic (ORG; \(n = 17\)) farms. Between~1989 and~1990, the researchers sampled tomato fields during April and September (p.~\(1\,100\)). An area between \(0.04\) and~\(0.1\,\text{ha}\) was set aside within each field for collecting data. Each area was divided into \(20\)~sections, then a \(1\,\text{m}\) row was selected at random within each section to be sampled.

\begin{enumerate}
\def\labelenumi{\arabic{enumi}.}
\tightlist
\item
  Explain what type of sampling is being used.
\end{enumerate}

One important measure of soil health is the number of actinomycetes. When comparing ORG and CNV, the researchers found that the (p.~\(1\,103\)):

\begin{quote}
\ldots{} total numbers of actinomycetes {[}\ldots{]} were significantly larger in the ORG soils {[}\ldots{]} (Student's \(t\) test, \(t = 5.4\), \(P = 0.006\))\ldots{}
\end{quote}

\begin{enumerate}
\def\labelenumi{\arabic{enumi}.}
\setcounter{enumi}{1}
\tightlist
\item
  Explain what the results mean.
\item
  Are the results statistically valid?
\end{enumerate}

The researchers also found that (p.~\(1\,103\)):

\begin{quote}
\ldots{} starch hydrolyzing actinomycetes were more numerous in CNV {[}\ldots{]} (Student's \(t\) test, \(t = 4.0\), \(P = 0.005\)).
\end{quote}

\begin{enumerate}
\def\labelenumi{\arabic{enumi}.}
\setcounter{enumi}{3}
\tightlist
\item
  Explain what these results mean.
\end{enumerate}

They also found that (p.~\(1\,103\)):

\begin{quote}
Total actinomycete abundance {[}was{]} negatively correlated with corky root {[}a disease{]} severity (\(r = -0.76\), \(P = 0.08\)\ldots).
\end{quote}

\begin{enumerate}
\def\labelenumi{\arabic{enumi}.}
\setcounter{enumi}{4}
\tightlist
\item
  Explain what these results mean.
\item
  Compute and interpret the value of \(R^2\).
\end{enumerate}

\end{exercise}

\begin{exercise}
\protect\hypertarget{exr:ReadExerciseEyeMasks}{}\label{exr:ReadExerciseEyeMasks}

\citet{teo2022eye} studied pregnant Malaysian women with sleeping disruptions in the last month of pregnancy. The \(56\) patients were (p.~1):

\begin{quote}
\ldots{} randomized to the use of eye-mask and earplugs or ``sham'' headbands during night sleep (both introduced as sleep aids).
\end{quote}

Thus, two groups were used: one using eye-masks and earplugs (treatment group, \(T\); \(n = 29\)) and one using sham headbands (control or placebo group, \(P\); \(n = 27\)).

\begin{enumerate}
\def\labelenumi{\arabic{enumi}.}
\tightlist
\item
  What was the purpose of using `sham' headbands if it was an ineffective intervention?
\item
  What type of study is this: experimental or observational? Explain.
\end{enumerate}

Sleep duration was measured in Week~1 (no intervention) and again in Week~2 (with the allocated intervention) for each subject, using a `wrist actigraphy monitor'.

\begin{enumerate}
\def\labelenumi{\arabic{enumi}.}
\setcounter{enumi}{2}
\tightlist
\item
  Why is using a `wrist actigraphy monitor' better than self-reported sleep duration?
\end{enumerate}

The women in the two groups were compared. For example, the mean age of the women was \(30.6\)~y (\(s = 3.6\)) (\(T\)) and \(30.1\)~y (\(s = 3.3\)) (\(P\)); the \(P\)-value for the comparison was given as \(P = 0.56\).

\begin{enumerate}
\def\labelenumi{\arabic{enumi}.}
\setcounter{enumi}{3}
\tightlist
\item
  \emph{Why} was this comparison made?
\item
  Compute the standard error for the difference between the two mean sleep durations.
\item
  Compute the \(t\)-score for the test.
\item
  Is the quoted \(P\)-value consistent with your calculations? What do these results mean?
\item
  Is the result statistically valid?
\end{enumerate}

Another comparison was the room `condition' where the women slept: in the treatment group, \(13\) had a room with a fan (\(16\) had air conditioning), while in the control group \(10\) women had a fan (and \(17\) air conditioning). The \(P\)-value for the comparison was given as \(P = 0.60\).

\begin{enumerate}
\def\labelenumi{\arabic{enumi}.}
\setcounter{enumi}{8}
\tightlist
\item
  \emph{Why} was this comparison made?
\item
  Construct the \(2\times 2\) table summarising the data.
\item
  The \(\chi^2\)-score for the test is \(0.35064\). Compute the equivalent \(z\)-score. Interpret the results.
\item
  Is the quoted \(P\)-value consistent with your calculations?
\item
  Is the result statistically valid?
\end{enumerate}

In the \emph{treatment} group, the mean sleep duration in Week~1 was \(279.0\,\text{mins}\) (\(s = 18.9\)) and in Week~2 was \(303.6\,\text{mins}\) (\(s = 18.8\)). The \emph{increase} was \(24.7\,\text{mins}\) (\(s = 14.9\)).

\begin{enumerate}
\def\labelenumi{\arabic{enumi}.}
\setcounter{enumi}{13}
\tightlist
\item
  Test if sleep duration \emph{increased} in the treatment group. Interpret the results mean.
\end{enumerate}

In the \emph{control} group, the mean sleep duration in Week~1 was \(286.3\,\text{mins}\) (\(s = 20.9\)) and in Week~2 was \(301.9\,\text{mins}\) (\(s = 21.8\); \(n = 26\)). The \emph{increase} was \(18.1\,\text{mins}\) (\(s = 17.3\)).

\begin{enumerate}
\def\labelenumi{\arabic{enumi}.}
\setcounter{enumi}{14}
\tightlist
\item
  Test if sleep duration \emph{increased} in the control group. Interpret the results.
\item
  Why would sleep duration \emph{increase}, if the control group used an ineffective intervention?
\end{enumerate}

The \emph{increase} in sleep duration can be compared for the two groups.

\begin{enumerate}
\def\labelenumi{\arabic{enumi}.}
\setcounter{enumi}{16}
\tightlist
\item
  Compute the standard error for difference between the mean increases \(\text{s.e.}( \bar{x}_T - \bar{x}_T)\).
\item
  Compare the increase in sleep duration for the two groups. Interpret the results.
\item
  Is the test statistically valid?
\end{enumerate}

\end{exercise}

\begin{EOCanswerBox}{iconmonstr-check-mark-14-240.png}
\textbf{Answers to \emph{Quick review} questions:} \textbf{1.} False. \textbf{2.} True. \textbf{3.} False.

\end{EOCanswerBox}

\appendix \addcontentsline{toc}{chapter}{\appendixname}


\chapter{Datasets}\label{AppendixDataSets}

Most datasets used in this book are available in the \textbf{R}~package \textbf{SRMData}, available free from \href{https://CRAN.R-project.org}{\texttt{CRAN}.} Most datasets used in this book can also be downloaded from the online version of this book (\texttt{https://bookdown.org/pkaldunn/SRM-Textbook/}).

In the list below (alphabetical within chapters), all datasets are from the \textbf{SRMData} package except when noted (in parentheses). Other packages listed are also available from \texttt{CRAN}.

\begin{small}

\begin{multicols}{3}\raggedcolumns
\begin{minipage}{\textwidth}
\textbf{Chapter 7}

\begin{itemize}\tightlist
\item
  \texttt{Placebos} 
\end{itemize} 
\end{minipage}

 \medskip\begin{minipage}{\textwidth}
\textbf{Chapter 10}

\begin{itemize}\tightlist
\item
  \texttt{Orthoses}  (Exercise) 
\end{itemize} 
\end{minipage}

 \medskip\begin{minipage}{\textwidth}
\textbf{Chapter 11}

\begin{itemize}\tightlist
\item
  \texttt{BabyBoom} 
\item
  \texttt{Cyclones} 
\item
  \texttt{faithful} (in \textbf{R}) 
\item
  \texttt{Gorillas} 
\item
  \texttt{MaryRiver} 
\item
  \texttt{Perm} 
\item
  \texttt{WaterAccess} 

\medskip
\item
  \texttt{CherryRipe}  (Exercise) 
\item
  \texttt{FriesWt}  (Exercise) 
\item
  \texttt{Jeans}  (Exercise) 
\item
  \texttt{Lime}  (Exercise) 
\item
  \texttt{NHANES}  (Exercise; in\\ \textbf{NHANES} package) 
\item
  \texttt{Orthoses}  (Exercise) 
\end{itemize} 
\end{minipage}

 \medskip\begin{minipage}{\textwidth}
\textbf{Chapter 12}

\begin{itemize}\tightlist
\item
  \texttt{WaterAccess} 

\medskip
\item
  \texttt{BabyBoom}  (Exercise) 
\item
  \texttt{LungCap}  (Exercise) 
\end{itemize} 
\end{minipage}

 \medskip\begin{minipage}{\textwidth}
\textbf{Chapter 13}

\begin{itemize}\tightlist
\item
  \texttt{IgE} 
\item
  \texttt{Running} 
\item
  \texttt{Tape} 

\medskip
\item
  \texttt{Captopril}  (Exercise) 
\item
  \texttt{Flowering}  (Exercise) 
\item
  \texttt{Insulation}  (Exercise) 
\item
  \texttt{Jumping}  (Exercise) 
\item
  \texttt{PainRelief}  (Exercise) 
\item
  \texttt{Running}  (Exercise) 
\item
  \texttt{Stress}  (Exercise) 
\item
  \texttt{WCTennis}  (Exercise) 
\end{itemize} 
\end{minipage}

 \medskip\begin{minipage}{\textwidth}
\textbf{Chapter 14}

\begin{itemize}\tightlist
\item
  \texttt{Gorillas} 
\item
  \texttt{Jellyfish} 
\item
  \texttt{WaterAccess} 

\medskip
\item
  \texttt{AISsub}  (Exercise) 
\item
  \texttt{Deceleration}  (Exercise) 
\item
  \texttt{Dental}  (Exercise) 
\item
  \texttt{ForwardFall}  (Exercise) 
\item
  \texttt{NHANES}  (Exercise; in\\ \textbf{NHANES} package) 
\item
  \texttt{Snakes}  (Exercise) 
\item
  \texttt{Speed}  (Exercise) 
\item
  \texttt{Typing}  (Exercise) 
\end{itemize} 
\end{minipage}

 \medskip\begin{minipage}{\textwidth}
\textbf{Chapter 15}

\begin{itemize}\tightlist
\item
  \texttt{KStones} 
\item
  \texttt{WaterAccess} 

\medskip
\item
  \texttt{EmeraldAug}  (Exercise) 
\item
  \texttt{PremierL}  (Exercise) 
\end{itemize} 
\end{minipage}

 \medskip\begin{minipage}{\textwidth}
\textbf{Chapter 16}

\begin{itemize}\tightlist
\item
  \texttt{LungCap} 
\item
  \texttt{RedDeer} 
\item
  \texttt{Removal} 
\item
  \texttt{Sanddollars} 
\item
  \texttt{YieldDen} 

\medskip
\item
  \texttt{BoneQuality}  (Exercise) 
\item
  \texttt{Cyclones}  (Exercise) 
\item
  \texttt{Gorillas}  (Exercise) 
\item
  \texttt{Lime}  (Exercise) 
\item
  \texttt{Mandible}  (Exercise) 
\item
  \texttt{Peas}  (Exercise) 
\item
  \texttt{SDrink}  (Exercise) 
\item
  \texttt{SoilCN}  (Exercise) 
\item
  \texttt{StudentWt}  (Exercise) 
\item
  \texttt{Windmill}  (Exercise) 
\end{itemize} 
\end{minipage}

 \medskip\begin{minipage}{\textwidth}
\textbf{Chapter 17}

\begin{itemize}\tightlist
\item
  \texttt{BodyTemp} 
\item
  \texttt{DanishLC} 
\item
  \texttt{NHANES} (in\\ \textbf{NHANES} package) 
\item
  \texttt{WaterAccess} 

\medskip
\item
  \texttt{AISsub}  (Exercise) 
\item
  \texttt{HCrabs}  (Exercise) 
\item
  \texttt{NHANES}  (Exercise; in\\ \textbf{NHANES} package) 
\item
  \texttt{NMiner}  (Exercise) 
\item
  \texttt{Typing}  (Exercise) 
\end{itemize} 
\end{minipage}

 \medskip\begin{minipage}{\textwidth}
\textbf{Chapter 18}

\begin{itemize}\tightlist
\item
  \texttt{HatSunglasses} 
\item
  \texttt{QSchools} 
\end{itemize} 
\end{minipage}

 \medskip\begin{minipage}{\textwidth}
\textbf{Chapter 20}

\begin{itemize}\tightlist
\item
  \texttt{Diabetes} 
\item
  \texttt{Possums} 
\end{itemize} 
\end{minipage}

 \medskip\begin{minipage}{\textwidth}
\textbf{Chapter 22}

\begin{itemize}\tightlist
\item
  \texttt{HatSunglasses}  (Exercise) 
\end{itemize} 
\end{minipage}

 \medskip\begin{minipage}{\textwidth}
\textbf{Chapter 23}

\begin{itemize}\tightlist
\item
  \texttt{Fluoro} 

\medskip
\item
  \texttt{LungCap}  (Exercise) 
\item
  \texttt{NHANES}  (Exercise; in\\ \textbf{NHANES} package) 
\item
  \texttt{PizzaSize}  (Exercise) 
\end{itemize} 
\end{minipage}

 \medskip\begin{minipage}{\textwidth}
\textbf{Chapter 26}

\begin{itemize}\tightlist
\item
  \texttt{PremierL}  (Exercise) 
\end{itemize} 
\end{minipage}

 \medskip\begin{minipage}{\textwidth}
\textbf{Chapter 27}

\begin{itemize}\tightlist
\item
  \texttt{BodyTemp} 

\medskip
\item
  \texttt{CherryRipe}  (Exercise) 
\item
  \texttt{LHconc}  (Exercise) 
\item
  \texttt{PizzaSize}  (Exercise) 
\end{itemize} 
\end{minipage}

 \medskip\begin{minipage}{\textwidth}
\textbf{Chapter 28}

\begin{itemize}\tightlist
\item
  \texttt{Battery}  (Exercise) 
\end{itemize} 
\end{minipage}

 \medskip\begin{minipage}{\textwidth}
\textbf{Chapter 29}

\begin{itemize}\tightlist
\item
  \texttt{Flowering} 
\item
  \texttt{SixMWT} 

\medskip
\item
  \texttt{Anorexia}  (Exercise) 
\item
  \texttt{Captopril}  (Exercise) 
\item
  \texttt{Ferritin}  (Exercise) 
\item
  \texttt{Fruit}  (Exercise) 
\item
  \texttt{Jumping}  (Exercise) 
\item
  \texttt{SoilCN}  (Exercise) 
\item
  \texttt{Stress}  (Exercise) 
\item
  \texttt{StudentWt}  (Exercise) 
\item
  \texttt{WCTennis}  (Exercise) 
\end{itemize} 
\end{minipage}

 \medskip\begin{minipage}{\textwidth}
\textbf{Chapter 30}

\begin{itemize}\tightlist
\item
  \texttt{Lime} 
\item
  \texttt{Snakes} 
\item
  \texttt{Speed} 

\medskip
\item
  \texttt{Anorexia}  (Exercise) 
\item
  \texttt{BMI}  (Exercise) 
\item
  \texttt{BodyTemp}  (Exercise) 
\item
  \texttt{Deceleration}  (Exercise) 
\item
  \texttt{Dental}  (Exercise) 
\item
  \texttt{ForwardFall}  (Exercise) 
\item
  \texttt{Lime}  (Exercise) 
\item
  \texttt{NHANES}  (Exercise; in\\ \textbf{NHANES} package) 
\item
  \texttt{ReactionTime}  (Exercise) 
\end{itemize} 
\end{minipage}

 \medskip\begin{minipage}{\textwidth}
\textbf{Chapter 31}

\begin{itemize}\tightlist
\item
  \texttt{Burros} 
\item
  \texttt{PetBirds} 
\item
  \texttt{RipsID} 
\item
  \texttt{StudentsEat} 

\medskip
\item
  \texttt{B12Diet}  (Exercise) 
\item
  \texttt{CarCrashes}  (Exercise) 
\item
  \texttt{CrabShells2}  (Exercise) 
\item
  \texttt{CrabShells3}  (Exercise) 
\item
  \texttt{DogWalks}  (Exercise) 
\item
  \texttt{EmeraldAug}  (Exercise) 
\item
  \texttt{EVpurchase}  (Exercise) 
\item
  \texttt{EVPurchase}  (Exercise) 
\item
  \texttt{HatSunglasses}  (Exercise) 
\item
  \texttt{Mumps}  (Exercise) 
\item
  \texttt{PetBirds}  (Exercise) 
\item
  \texttt{RipsID}  (Exercise) 
\item
  \texttt{ScarHeight}  (Exercise) 
\item
  \texttt{ShoppingBags}  (Exercise) 
\item
  \texttt{Turbines}  (Exercise) 
\end{itemize} 
\end{minipage}

 \medskip\begin{minipage}{\textwidth}
\textbf{Chapter 32}

\begin{itemize}\tightlist
\item
  \texttt{Diabetes} 
\end{itemize} 
\end{minipage}

 \medskip\begin{minipage}{\textwidth}
\textbf{Chapter 33}

\begin{itemize}\tightlist
\item
  \texttt{AISsub} 
\item
  \texttt{Borers} 
\item
  \texttt{Cyclones} 
\item
  \texttt{Removal} 

\medskip
\item
  \texttt{Bitumen}  (Exercise) 
\item
  \texttt{Corollas}  (Exercise) 
\item
  \texttt{Dogs}  (Exercise) 
\item
  \texttt{DogsLife}  (Exercise) 
\item
  \texttt{EDpatients}  (Exercise) 
\item
  \texttt{Elephants}  (Exercise) 
\item
  \texttt{Gorillas}  (Exercise) 
\item
  \texttt{Jeans}  (Exercise) 
\item
  \texttt{Mandible}  (Exercise) 
\item
  \texttt{OSA}  (Exercise) 
\item
  \texttt{Possums}  (Exercise) 
\item
  \texttt{SDrink}  (Exercise) 
\item
  \texttt{Soils}  (Exercise) 
\item
  \texttt{Throttle}  (Exercise) 
\item
  \texttt{Typing}  (Exercise) 
\end{itemize}
 
\end{minipage}

 \end{multicols}

\end{small}

\chapter{\texorpdfstring{\(z\)-score tables}{z-score tables}}\label{Tables}

\index{Normal distribution!tables|(}

This appendix contains \(z\)-score tables.

These tables provide the area \emph{to the left} of a given \(z\)-score associated with a normal distribution: Appendices~\ref{ZTablesNEG} (for negative values of \(z\)) and~\ref{ZTablesPOS} (for positive values of \(z\)).

The online version of this book (\url{https://bookdown.org/pkaldunn/SRM-Textbook}) has online tables, which are easier to use.

\subsection*{Using these tables}\label{using-these-tables}

Details for using these tables are provided in Sect.~\ref{ExactAreasUsingTables}. Here we give a short summary.

To find the probability that~\(z\) is less than \(-2.43\), follow these steps:

\begin{enumerate}
\def\labelenumi{\arabic{enumi}.}
\tightlist
\item
  Use Appendix~\ref{ZTablesNEG} to find \(-2.4\) in the \emph{left} margin of the table (see image below).
\item
  Then, find the second decimal place (in this case, \(3\)) in the \emph{top} margin of the table.
\item
  Where these intersect is the area (or probability) \emph{less than} the \(z\)-score of \(-2.43\).
\end{enumerate}

The probability of finding a \(z\)-score less than \(z = -2.43\) is \(0.0075\), or about \(0.75\)\%.

\begin{center}\includegraphics[width=0.5\linewidth]{TablesExampleLaTeX4} \end{center}

\pagebreak

\section{\texorpdfstring{Normal distribution: negative \(z\)-values probabilities}{Normal distribution: negative z-values probabilities}}\label{ZTablesNEG}

\begin{multicols}{2}


\begin{center}\includegraphics[width=1\linewidth]{51-App-Tables_files/figure-latex/unnamed-chunk-14-1} \end{center}

\columnbreak

\begin{normalsize}\raggedright
The table gives the probability (area) that a $z$-score is \emph{less} than a given value.
For example: the area \emph{less than} $z = -1.38$ is $0.0838$, or $8.38$\%.
\end{normalsize}

\end{multicols}

\begin{table}
\centering\begingroup\fontsize{9}{11}\selectfont

\begin{tabular}{lrrrrrrrrrr}
\toprule
  & $ 0.00 $ & $ 0.01 $ & $ 0.02 $ & $ 0.03 $ & $ 0.04 $ & $ 0.05 $ & $ 0.06 $ & $ 0.07 $ & $ 0.08 $ & $ 0.09 $\\
\midrule
$ -3.5 $ & $0.0002$ & $0.0002$ & $0.0002$ & $0.0002$ & $0.0002$ & $0.0002$ & $0.0002$ & $0.0002$ & $0.0002$ & $0.0002$\\
$ -3.4 $ & $0.0003$ & $0.0003$ & $0.0003$ & $0.0003$ & $0.0003$ & $0.0003$ & $0.0003$ & $0.0003$ & $0.0003$ & $0.0002$\\
$ -3.3 $ & $0.0005$ & $0.0005$ & $0.0005$ & $0.0004$ & $0.0004$ & $0.0004$ & $0.0004$ & $0.0004$ & $0.0004$ & $0.0003$\\
$ -3.2 $ & $0.0007$ & $0.0007$ & $0.0006$ & $0.0006$ & $0.0006$ & $0.0006$ & $0.0006$ & $0.0005$ & $0.0005$ & $0.0005$\\
$ -3.1 $ & $0.0010$ & $0.0009$ & $0.0009$ & $0.0009$ & $0.0008$ & $0.0008$ & $0.0008$ & $0.0008$ & $0.0007$ & $0.0007$\\
\vspace{3pt}
$ -3.0 $ & $0.0013$ & $0.0013$ & $0.0013$ & $0.0012$ & $0.0012$ & $0.0011$ & $0.0011$ & $0.0011$ & $0.0010$ & $0.0010$\\
$ -2.9 $ & $0.0019$ & $0.0018$ & $0.0018$ & $0.0017$ & $0.0016$ & $0.0016$ & $0.0015$ & $0.0015$ & $0.0014$ & $0.0014$\\
$ -2.8 $ & $0.0026$ & $0.0025$ & $0.0024$ & $0.0023$ & $0.0023$ & $0.0022$ & $0.0021$ & $0.0021$ & $0.0020$ & $0.0019$\\
$ -2.7 $ & $0.0035$ & $0.0034$ & $0.0033$ & $0.0032$ & $0.0031$ & $0.0030$ & $0.0029$ & $0.0028$ & $0.0027$ & $0.0026$\\
$ -2.6 $ & $0.0047$ & $0.0045$ & $0.0044$ & $0.0043$ & $0.0041$ & $0.0040$ & $0.0039$ & $0.0038$ & $0.0037$ & $0.0036$\\
\vspace{3pt}
$ -2.5 $ & $0.0062$ & $0.0060$ & $0.0059$ & $0.0057$ & $0.0055$ & $0.0054$ & $0.0052$ & $0.0051$ & $0.0049$ & $0.0048$\\
$ -2.4 $ & $0.0082$ & $0.0080$ & $0.0078$ & $0.0075$ & $0.0073$ & $0.0071$ & $0.0069$ & $0.0068$ & $0.0066$ & $0.0064$\\
$ -2.3 $ & $0.0107$ & $0.0104$ & $0.0102$ & $0.0099$ & $0.0096$ & $0.0094$ & $0.0091$ & $0.0089$ & $0.0087$ & $0.0084$\\
$ -2.2 $ & $0.0139$ & $0.0136$ & $0.0132$ & $0.0129$ & $0.0125$ & $0.0122$ & $0.0119$ & $0.0116$ & $0.0113$ & $0.0110$\\
$ -2.1 $ & $0.0179$ & $0.0174$ & $0.0170$ & $0.0166$ & $0.0162$ & $0.0158$ & $0.0154$ & $0.0150$ & $0.0146$ & $0.0143$\\
\vspace{3pt}
$ -2.0 $ & $0.0228$ & $0.0222$ & $0.0217$ & $0.0212$ & $0.0207$ & $0.0202$ & $0.0197$ & $0.0192$ & $0.0188$ & $0.0183$\\
$ -1.9 $ & $0.0287$ & $0.0281$ & $0.0274$ & $0.0268$ & $0.0262$ & $0.0256$ & $0.0250$ & $0.0244$ & $0.0239$ & $0.0233$\\
$ -1.8 $ & $0.0359$ & $0.0351$ & $0.0344$ & $0.0336$ & $0.0329$ & $0.0322$ & $0.0314$ & $0.0307$ & $0.0301$ & $0.0294$\\
$ -1.7 $ & $0.0446$ & $0.0436$ & $0.0427$ & $0.0418$ & $0.0409$ & $0.0401$ & $0.0392$ & $0.0384$ & $0.0375$ & $0.0367$\\
$ -1.6 $ & $0.0548$ & $0.0537$ & $0.0526$ & $0.0516$ & $0.0505$ & $0.0495$ & $0.0485$ & $0.0475$ & $0.0465$ & $0.0455$\\
\vspace{3pt}
$ -1.5 $ & $0.0668$ & $0.0655$ & $0.0643$ & $0.0630$ & $0.0618$ & $0.0606$ & $0.0594$ & $0.0582$ & $0.0571$ & $0.0559$\\
$ -1.4 $ & $0.0808$ & $0.0793$ & $0.0778$ & $0.0764$ & $0.0749$ & $0.0735$ & $0.0721$ & $0.0708$ & $0.0694$ & $0.0681$\\
$ -1.3 $ & $0.0968$ & $0.0951$ & $0.0934$ & $0.0918$ & $0.0901$ & $0.0885$ & $0.0869$ & $0.0853$ & $0.0838$ & $0.0823$\\
$ -1.2 $ & $0.1151$ & $0.1131$ & $0.1112$ & $0.1093$ & $0.1075$ & $0.1056$ & $0.1038$ & $0.1020$ & $0.1003$ & $0.0985$\\
$ -1.1 $ & $0.1357$ & $0.1335$ & $0.1314$ & $0.1292$ & $0.1271$ & $0.1251$ & $0.1230$ & $0.1210$ & $0.1190$ & $0.1170$\\
\vspace{3pt}
$ -1.0 $ & $0.1587$ & $0.1562$ & $0.1539$ & $0.1515$ & $0.1492$ & $0.1469$ & $0.1446$ & $0.1423$ & $0.1401$ & $0.1379$\\
$ -0.9 $ & $0.1841$ & $0.1814$ & $0.1788$ & $0.1762$ & $0.1736$ & $0.1711$ & $0.1685$ & $0.1660$ & $0.1635$ & $0.1611$\\
$ -0.8 $ & $0.2119$ & $0.2090$ & $0.2061$ & $0.2033$ & $0.2005$ & $0.1977$ & $0.1949$ & $0.1922$ & $0.1894$ & $0.1867$\\
$ -0.7 $ & $0.2420$ & $0.2389$ & $0.2358$ & $0.2327$ & $0.2296$ & $0.2266$ & $0.2236$ & $0.2206$ & $0.2177$ & $0.2148$\\
$ -0.6 $ & $0.2743$ & $0.2709$ & $0.2676$ & $0.2643$ & $0.2611$ & $0.2578$ & $0.2546$ & $0.2514$ & $0.2483$ & $0.2451$\\
\vspace{3pt}
$ -0.5 $ & $0.3085$ & $0.3050$ & $0.3015$ & $0.2981$ & $0.2946$ & $0.2912$ & $0.2877$ & $0.2843$ & $0.2810$ & $0.2776$\\
$ -0.4 $ & $0.3446$ & $0.3409$ & $0.3372$ & $0.3336$ & $0.3300$ & $0.3264$ & $0.3228$ & $0.3192$ & $0.3156$ & $0.3121$\\
$ -0.3 $ & $0.3821$ & $0.3783$ & $0.3745$ & $0.3707$ & $0.3669$ & $0.3632$ & $0.3594$ & $0.3557$ & $0.3520$ & $0.3483$\\
$ -0.2 $ & $0.4207$ & $0.4168$ & $0.4129$ & $0.4090$ & $0.4052$ & $0.4013$ & $0.3974$ & $0.3936$ & $0.3897$ & $0.3859$\\
$ -0.1 $ & $0.4602$ & $0.4562$ & $0.4522$ & $0.4483$ & $0.4443$ & $0.4404$ & $0.4364$ & $0.4325$ & $0.4286$ & $0.4247$\\
\vspace{3pt}
$-0.0$ & $0.5000$ & $0.4960$ & $0.4920$ & $0.4880$ & $0.4840$ & $0.4801$ & $0.4761$ & $0.4721$ & $0.4681$ & $0.4641$\\
\bottomrule
\end{tabular}
\endgroup{}
\end{table}

\begin{normalsize}
For $z = -4$, the probability is $0.00003$. For $z = -5$, the probability is $0.0000003$.
\end{normalsize}

\pagebreak

\section{\texorpdfstring{Normal distribution: positive \(z\)-values probabilities}{Normal distribution: positive z-values probabilities}}\label{ZTablesPOS}

\begin{multicols}{2}


\begin{center}\includegraphics[width=1\linewidth]{51-App-Tables_files/figure-latex/unnamed-chunk-16-1} \end{center}

\columnbreak

\begin{normalsize}\raggedright
The table gives the probability (area) that a $z$-score is \emph{less} than a given value.
For example: the area \textit{less than} $z = 1.87$ is $0.9693$, or $96.93$\%.
\end{normalsize}

\end{multicols}

\begin{table}
\centering\begingroup\fontsize{9}{11}\selectfont

\begin{tabular}{lrrrrrrrrrr}
\toprule
  & $ 0.00 $ & $ 0.01 $ & $ 0.02 $ & $ 0.03 $ & $ 0.04 $ & $ 0.05 $ & $ 0.06 $ & $ 0.07 $ & $ 0.08 $ & $ 0.09 $\\
\midrule
$0.0$ & $0.5000$ & $0.5040$ & $0.5080$ & $0.5120$ & $0.5160$ & $0.5199$ & $0.5239$ & $0.5279$ & $0.5319$ & $0.5359$\\
$ 0.1 $ & $0.5398$ & $0.5438$ & $0.5478$ & $0.5517$ & $0.5557$ & $0.5596$ & $0.5636$ & $0.5675$ & $0.5714$ & $0.5753$\\
$ 0.2 $ & $0.5793$ & $0.5832$ & $0.5871$ & $0.5910$ & $0.5948$ & $0.5987$ & $0.6026$ & $0.6064$ & $0.6103$ & $0.6141$\\
$ 0.3 $ & $0.6179$ & $0.6217$ & $0.6255$ & $0.6293$ & $0.6331$ & $0.6368$ & $0.6406$ & $0.6443$ & $0.6480$ & $0.6517$\\
$ 0.4 $ & $0.6554$ & $0.6591$ & $0.6628$ & $0.6664$ & $0.6700$ & $0.6736$ & $0.6772$ & $0.6808$ & $0.6844$ & $0.6879$\\
\vspace{3pt}
$ 0.5 $ & $0.6915$ & $0.6950$ & $0.6985$ & $0.7019$ & $0.7054$ & $0.7088$ & $0.7123$ & $0.7157$ & $0.7190$ & $0.7224$\\
$ 0.6 $ & $0.7257$ & $0.7291$ & $0.7324$ & $0.7357$ & $0.7389$ & $0.7422$ & $0.7454$ & $0.7486$ & $0.7517$ & $0.7549$\\
$ 0.7 $ & $0.7580$ & $0.7611$ & $0.7642$ & $0.7673$ & $0.7704$ & $0.7734$ & $0.7764$ & $0.7794$ & $0.7823$ & $0.7852$\\
$ 0.8 $ & $0.7881$ & $0.7910$ & $0.7939$ & $0.7967$ & $0.7995$ & $0.8023$ & $0.8051$ & $0.8078$ & $0.8106$ & $0.8133$\\
$ 0.9 $ & $0.8159$ & $0.8186$ & $0.8212$ & $0.8238$ & $0.8264$ & $0.8289$ & $0.8315$ & $0.8340$ & $0.8365$ & $0.8389$\\
\vspace{3pt}
$ 1.0 $ & $0.8413$ & $0.8438$ & $0.8461$ & $0.8485$ & $0.8508$ & $0.8531$ & $0.8554$ & $0.8577$ & $0.8599$ & $0.8621$\\
$ 1.1 $ & $0.8643$ & $0.8665$ & $0.8686$ & $0.8708$ & $0.8729$ & $0.8749$ & $0.8770$ & $0.8790$ & $0.8810$ & $0.8830$\\
$ 1.2 $ & $0.8849$ & $0.8869$ & $0.8888$ & $0.8907$ & $0.8925$ & $0.8944$ & $0.8962$ & $0.8980$ & $0.8997$ & $0.9015$\\
$ 1.3 $ & $0.9032$ & $0.9049$ & $0.9066$ & $0.9082$ & $0.9099$ & $0.9115$ & $0.9131$ & $0.9147$ & $0.9162$ & $0.9177$\\
$ 1.4 $ & $0.9192$ & $0.9207$ & $0.9222$ & $0.9236$ & $0.9251$ & $0.9265$ & $0.9279$ & $0.9292$ & $0.9306$ & $0.9319$\\
\vspace{3pt}
$ 1.5 $ & $0.9332$ & $0.9345$ & $0.9357$ & $0.9370$ & $0.9382$ & $0.9394$ & $0.9406$ & $0.9418$ & $0.9429$ & $0.9441$\\
$ 1.6 $ & $0.9452$ & $0.9463$ & $0.9474$ & $0.9484$ & $0.9495$ & $0.9505$ & $0.9515$ & $0.9525$ & $0.9535$ & $0.9545$\\
$ 1.7 $ & $0.9554$ & $0.9564$ & $0.9573$ & $0.9582$ & $0.9591$ & $0.9599$ & $0.9608$ & $0.9616$ & $0.9625$ & $0.9633$\\
$ 1.8 $ & $0.9641$ & $0.9649$ & $0.9656$ & $0.9664$ & $0.9671$ & $0.9678$ & $0.9686$ & $0.9693$ & $0.9699$ & $0.9706$\\
$ 1.9 $ & $0.9713$ & $0.9719$ & $0.9726$ & $0.9732$ & $0.9738$ & $0.9744$ & $0.9750$ & $0.9756$ & $0.9761$ & $0.9767$\\
\vspace{3pt}
$ 2.0 $ & $0.9772$ & $0.9778$ & $0.9783$ & $0.9788$ & $0.9793$ & $0.9798$ & $0.9803$ & $0.9808$ & $0.9812$ & $0.9817$\\
$ 2.1 $ & $0.9821$ & $0.9826$ & $0.9830$ & $0.9834$ & $0.9838$ & $0.9842$ & $0.9846$ & $0.9850$ & $0.9854$ & $0.9857$\\
$ 2.2 $ & $0.9861$ & $0.9864$ & $0.9868$ & $0.9871$ & $0.9875$ & $0.9878$ & $0.9881$ & $0.9884$ & $0.9887$ & $0.9890$\\
$ 2.3 $ & $0.9893$ & $0.9896$ & $0.9898$ & $0.9901$ & $0.9904$ & $0.9906$ & $0.9909$ & $0.9911$ & $0.9913$ & $0.9916$\\
$ 2.4 $ & $0.9918$ & $0.9920$ & $0.9922$ & $0.9925$ & $0.9927$ & $0.9929$ & $0.9931$ & $0.9932$ & $0.9934$ & $0.9936$\\
\vspace{3pt}
$ 2.5 $ & $0.9938$ & $0.9940$ & $0.9941$ & $0.9943$ & $0.9945$ & $0.9946$ & $0.9948$ & $0.9949$ & $0.9951$ & $0.9952$\\
$ 2.6 $ & $0.9953$ & $0.9955$ & $0.9956$ & $0.9957$ & $0.9959$ & $0.9960$ & $0.9961$ & $0.9962$ & $0.9963$ & $0.9964$\\
$ 2.7 $ & $0.9965$ & $0.9966$ & $0.9967$ & $0.9968$ & $0.9969$ & $0.9970$ & $0.9971$ & $0.9972$ & $0.9973$ & $0.9974$\\
$ 2.8 $ & $0.9974$ & $0.9975$ & $0.9976$ & $0.9977$ & $0.9977$ & $0.9978$ & $0.9979$ & $0.9979$ & $0.9980$ & $0.9981$\\
$ 2.9 $ & $0.9981$ & $0.9982$ & $0.9982$ & $0.9983$ & $0.9984$ & $0.9984$ & $0.9985$ & $0.9985$ & $0.9986$ & $0.9986$\\
\vspace{3pt}
$ 3.0 $ & $0.9987$ & $0.9987$ & $0.9987$ & $0.9988$ & $0.9988$ & $0.9989$ & $0.9989$ & $0.9989$ & $0.9990$ & $0.9990$\\
$ 3.1 $ & $0.9990$ & $0.9991$ & $0.9991$ & $0.9991$ & $0.9992$ & $0.9992$ & $0.9992$ & $0.9992$ & $0.9993$ & $0.9993$\\
$ 3.2 $ & $0.9993$ & $0.9993$ & $0.9994$ & $0.9994$ & $0.9994$ & $0.9994$ & $0.9994$ & $0.9995$ & $0.9995$ & $0.9995$\\
$ 3.3 $ & $0.9995$ & $0.9995$ & $0.9995$ & $0.9996$ & $0.9996$ & $0.9996$ & $0.9996$ & $0.9996$ & $0.9996$ & $0.9997$\\
$ 3.4 $ & $0.9997$ & $0.9997$ & $0.9997$ & $0.9997$ & $0.9997$ & $0.9997$ & $0.9997$ & $0.9997$ & $0.9997$ & $0.9998$\\
\vspace{3pt}
$ 3.5 $ & $0.9998$ & $0.9998$ & $0.9998$ & $0.9998$ & $0.9998$ & $0.9998$ & $0.9998$ & $0.9998$ & $0.9998$ & $0.9998$\\
\bottomrule
\end{tabular}
\endgroup{}
\end{table}

\begin{normalsize}
For $z = 4$, the probability is $0.99997$. 
For $z = 5$, the probability is $0.9999997$.
\end{normalsize}

\index{Normal distribution!tables|)}

\chapter{Symbols, formulas, statistics and parameters}\label{StatisticsAndParameters}

\section{Symbols and standard errors}\label{Symbols}

\index{Mean!of a population} \index{Mean!of a sample} \index{Mean!difference between} \index{Mean difference} \index{Proportions!of a population} \index{Proportions!of a sample} \index{Odds ratio} \index{Regression!coefficients} \index{Correlation coefficient (Pearson)} \index{Symbols used}

\begin{itemize}
\item
  The following table lists the statistics used to estimate unknown population parameters. \tightlist
\item
  When the sampling distribution is approximately normally distributed, under appropriate statistical validity conditions, this is indicated by~\ding{52}.\index{Sampling distribution}
\item
  The value of the mean of the sampling distribution (the \emph{sampling mean}) is\index{Sampling mean}:

  \begin{itemize}
  \tightlist
  \item
    unknown, for \emph{confidence intervals}.
  \item
    assumed to be the value given in the null hypothesis, for \emph{hypothesis tests}.
  \end{itemize}
\end{itemize}



























\begin{table}
\centering
\caption{\label{tab:ParametersStatistics2}Sample statistics used to estimate population parameters. Some statistics have appproximately normally-distributed sampling distributions under appropriate (statistical validity) conditions, as indicated using a \ding{52}.}
\centering
\fontsize{8}{10}\selectfont
\begin{tabular}[t]{>{\raggedright\arraybackslash}p{21mm}ccc>{\centering\arraybackslash}p{33mm}l}
\toprule
\multicolumn{2}{c}{\textbf{ }} & \multicolumn{3}{c}{\textbf{Sampling distribution}} & \multicolumn{1}{c}{\textbf{ }} \\
\cmidrule(l{3pt}r{3pt}){3-5}
\multicolumn{2}{c}{\textbf{ }} & \multicolumn{1}{c}{\textbf{Parameter, and}} & \multicolumn{1}{c}{\textbf{Normal}} & \multicolumn{1}{c}{\textbf{Standard}} & \multicolumn{1}{c}{\textbf{ }} \\
\textbf{ } & \textbf{Statistic} & \textbf{sampling mean} & \textbf{distn?} & \textbf{error} & \textbf{Ref.}\\
\midrule
 &  &  &  & CI: $\displaystyle \sqrt{\frac{ \hat{p} \times (1 - \hat{p})}{n}}$ & Ch.~\ref{CIOneProportion}\\
\cmidrule{5-6}
\multirow{-2}{21mm}[0.5\dimexpr\aboverulesep+\belowrulesep+\cmidrulewidth]{\raggedright\arraybackslash Proportion} & \multirow{-2}{*}[0.5\dimexpr\aboverulesep+\belowrulesep+\cmidrulewidth]{\centering\arraybackslash $\hat{p}$} & \multirow{-2}{*}[0.5\dimexpr\aboverulesep+\belowrulesep+\cmidrulewidth]{\centering\arraybackslash $p$} & \multirow{-2}{*}[0.5\dimexpr\aboverulesep+\belowrulesep+\cmidrulewidth]{\centering\arraybackslash \ding{52}} & HT: $\displaystyle \sqrt{\frac{ p \times (1 - p)}{n}}$ & Ch.~\ref{TestOneProportion}\\
\cmidrule{1-6}
Mean & $\bar{x}$ & $\mu$ & \ding{52} & $\displaystyle \frac{s}{\sqrt{n}}$ & Chs.~\ref{OneMeanConfInterval}, \ref{TestOneMean}\\
\cmidrule{1-6}
Mean difference & $\bar{d}$ & $\mu_d$ & \ding{52} & $\displaystyle \frac{s_d}{\sqrt{n}}$ & Ch.~\ref{AnalysisPaired}\\
\cmidrule{1-6}
Difference between means & $\bar{x}_1 - \bar{x}_2$ & $\mu_1 - \mu_2$ & \ding{52} & $\displaystyle \sqrt{\text{s.e.}(\bar{x}_1)^2 + \text{s.e.}(\bar{x}_2)^2}$ & Ch.~\ref{AnalysisTwoMeans}\\
\cmidrule{1-6}
 &  &  &  & CI: $\displaystyle \sqrt{\text{s.e.}(\hat{p}_1)^2 + \text{s.e.}(\hat{p}_2)^2}$ & \\
\cmidrule{5-5}
\multirow{-2}{21mm}[0.5\dimexpr\aboverulesep+\belowrulesep+\cmidrulewidth]{\raggedright\arraybackslash Difference between proportions} & \multirow{-2}{*}[0.5\dimexpr\aboverulesep+\belowrulesep+\cmidrulewidth]{\centering\arraybackslash $\hat{p}_1 - \hat{p}_2$} & \multirow{-2}{*}[0.5\dimexpr\aboverulesep+\belowrulesep+\cmidrulewidth]{\centering\arraybackslash $p_1 - p_2$} & \multirow{-2}{*}[0.5\dimexpr\aboverulesep+\belowrulesep+\cmidrulewidth]{\centering\arraybackslash \ding{52}} & \stackunder{HT: $\displaystyle \sqrt{\text{s.e.}(\hat{p}_1)^2 + \text{s.e.}(\hat{p}_2)^2}$}{using \emph{common} proportion $\hat{p}$} & \multirow{-2}{*}{\raggedright\arraybackslash Ch.~\ref{AnalysisOddsRatio}}\\
\cmidrule{1-6}
Odds ratio (OR) & Sample OR & Pop. OR & \ding{55} & (Not given) & Ch.~\ref{AnalysisOddsRatio}\\
\cmidrule{1-6}
Correlation & $r$ & $\rho$ & \ding{55} & (Not given) & Ch.~\ref{CorrelationRegression}\\
\cmidrule{1-6}
Regression: slope & $b_1$ & $\beta_1$ & \ding{52} & $\text{s.e.}(b_1)$ (value from software) & Ch.~\ref{CorrelationRegression}\\
\cmidrule{1-6}
Regression: intercept & $b_0$ & $\beta_0$ & \ding{52} & $\text{s.e.}(b_0)$ (value from software) & Ch.~\ref{CorrelationRegression}\\
\bottomrule
\end{tabular}
\end{table}

\section{Confidence intervals}\label{FormulasCI}

For statistics whose sampling distribution has an approximate normal distribution, \emph{confidence intervals (CIs)} have the form\index{Confidence intervals} \[ 
    \text{statistic} \pm \big( \text{multiplier} \times \text{s.e.}(\text{statistic})\big).
\]

\textbf{Notes:}

\begin{itemize}
\tightlist
\item
  The multiplier is \emph{approximately}~\(2\) to create an \emph{approximate} \(95\)\%~CI (based on the \(68\)--\(95\)--\(99.7\) rule).
\item
  The quantity `\(\text{multiplier} \times \text{s.e.}(\text{statistic})\)' is called the \emph{margin of error}.\index{Margin of error}
\item
  Software uses \emph{exact} multipliers to form \emph{exact} confidence intervals.
\item
  When the sampling distribution for the statistic does \emph{not} have an approximate normal distribution (e.g., for ORs and correlation coefficients), \emph{this formula does not apply} and the CIs are taken directly from software output when available.
\end{itemize}

\section{Hypothesis testing}\label{FormulasTest}

For statistics whose sampling distribution has an approximate normal distribution, the \emph{test statistic} has the form:\index{Hypothesis testing} \[
  \text{test statistic} = \frac{\text{statistic} - \text{parameter}}{\text{s.e.}(\text{statistic})},
\] where \(\text{s.e.}(\text{statistic})\) is the standard error of the statistic. The test-statistic is a \(t\)-score for most hypothesis tests in this book when the sampling distribution is described by a normal distribution, but is a \(z\)-score for a hypothesis test involving one or two \emph{proportions}.

\textbf{Notes:}

\begin{itemize}
\tightlist
\item
  If the test-statistic is a \(z\)-score,\index{Test statistic!z@$z$-score} the \(P\)-value can be found using tables (Appendices~\ref{ZTablesNEG} and~\ref{ZTablesPOS}), or \emph{approximated} using the \(68\)--\(95\)--\(99.7\) rule.
\item
  If the test-statistic is a \(t\)-score,\index{Test statistic!t@$t$-score} the \(P\)-value can be \emph{approximated} using tables (Appendices~\ref{ZTablesNEG} and~\ref{ZTablesPOS}), or \emph{approximated} using the \(68\)--\(95\)--\(99.7\) rule (since \(t\)-scores are similar to \(z\)-scores; Sect.~\ref{TestStatistic}).
\item
  When the sampling distribution for the statistic does not have an approximate normal distribution (e.g., for ORs and correlation coefficients), \emph{this formula does not apply} and \(P\)-values are taken from software when available.
\item
  A hypothesis test about ORs uses a \(\chi^2\) test statistic. For \(2\times 2\) tables only, the \(\chi^2\)-value is equivalent to a \(z\)-score with a value of \(\sqrt{\chi^2}\).\index{Test statistic!$\chi^2$-score}
\end{itemize}

\pagebreak

\section{Sample size estimation}\label{FormulasSampleSize}

The following formulas compute the \emph{approximate} minimum (i.e., conservative) sample size needed to produce a \(95\)\% CI with a specified margin of error\index{Sample size estimation} (i.e., the `give-or-take' amount).

\begin{itemize}
\item
  To estimate the sample size needed for \emph{estimating a proportion} (Sect.~\ref{SampleSizeProportions}), use: \[
   n = \frac{1}{(\text{Margin of error})^2}.
  \]
\item
  To estimate the sample size needed for \emph{estimating a mean} (Sect.~\ref{SampleSizeOneMean}) use: \[
   n = \left( \frac{2\times s}{\text{Margin of error}}\right)^2
  \] for some estimate~\(s\) of the standard deviation of the data.
\item
  To estimate the sample size needed for \emph{estimating a mean difference} (Sect.~\ref{SampleSizeMeanDifferences}) use: \[
   n = \left( \frac{2 \times s_d}{\text{Margin of error}}\right)^2
  \] for some estimate~\(s_d\) of the standard deviation of the differences.
\item
  To estimate the sample size needed for \emph{estimating the difference between two means} (Sect.~\ref{SampleSizeDifferenceTwoMeans}) use: \[
   n = 2\times \left( \frac{2 \times s}{\text{Margin of error}}\right)^2
  \] for \emph{each} group being compared, where \(s\) is an estimate of the common standard deviation in the population for both groups. This formula assumes:

  \begin{itemize}
  \tightlist
  \item
    the sample size for each group will be the same; and
  \item
    the standard deviation in each group is the same.
  \end{itemize}
\item
  To estimate the sample size needed for \emph{estimating the difference between two proportions} (Sect.~\ref{SampleSizeDifferenceTwoProportions}) use: \[
   n = \frac{2}{(\text{Margin of error})^2}
  \] for \emph{each} group being compared. This formula assumes the sample size in each group will be the same.
\end{itemize}

\textbf{Notes:}

\begin{itemize}
\tightlist
\item
  In \emph{sample size} calculations, \emph{round up} the sample size found from the above formulas.
\end{itemize}

\pagebreak

\section{Other formulas}\label{FormulasOther}

\begin{itemize}
\tightlist
\item
  To \emph{calculate \(z\)-scores} (Sect.~\ref{zScores}), use\index{z@$z$-score} \[
   z = \frac{\text{value of variable} - \text{mean of the distribution of the variable}}{\text{standard deviation of the distribution of the variable}}.
  \] \(t\)-scores are like \(z\)-scores.\index{Test statistic!t@$t$-score} When the `variable' is a sample estimate (such as \(\bar{x}\)), the `standard deviation of the distribution' is a standard error (such as \(\text{s.e.}(\bar{x})\)).
\item
  The \emph{unstandardising formula} (Sect.~\ref{Unstandardising}) is \(x = \mu + (z\times \sigma)\).\index{Unstandardising formula}
\item
  The \emph{interquartile range} (IQR) is \(Q_3 - Q_1\), where \(Q_1\) and \(Q_3\) are the first and third quartiles respectively (or, equivalently, the \(25\)th and \(75\)th percentiles).\index{Interquartile range (IQR)}
\item
  The smallest expected value (for assessing statistical validity when forming CIs and conducting hypothesis tests with proportions or ORs) is \[
  \frac{(\text{Smallest row total})\times(\text{Smallest column total})}{\text{Overall total}}.
  \]
\item
  The \emph{regression equation} in the \emph{sample} is \(\hat{y} = b_0 + b_1 x\), where \(b_0\) is the sample intercept and \(b_1\) is the sample slope.
\end{itemize}

\section{Other symbols and abbreviations used}\label{OtherSymbols}

\index{Standard deviation!of a population} \index{Standard deviation!of a sample} \index{Sample size}



















\begin{table}
\centering\begingroup\fontsize{9}{11}\selectfont

\begin{tabular}{>{\centering\arraybackslash}p{25mm}lc}
\toprule
\textbf{Symbol or abbreviation} & \textbf{Meaning} & \textbf{Reference}\\
\midrule
RQ & Research question & Chap.~\ref{RQs}\\
\addlinespace
$s$ & Sample standard deviation & Sect.~\ref{VariationStdDev}\\
$\sigma$ & Population standard deviation & Sect.~\ref{VariationStdDev}\\
\addlinespace
$s_d$ & Sample standard deviation of differences & Sect.~\ref{VariationStdDev}\\
$\sigma_d$ & Population standard deviation of differences & Sect.~\ref{VariationStdDev}\\
\addlinespace
$R^2$ & R-squared & Sect.~\ref{Rsquared}\\
\addlinespace
$H_0$ & Null hypothesis & Sect.~\ref{AboutHypotheses}\\
$H_1$ & Alternative hypothesis & Sect.~\ref{AboutHypotheses}\\
\addlinespace
CI & Confidence interval & Chap.~\ref{AboutCIs}\\
s.e. & Standard error & Def.~\ref{def:StandardError}\\
$n$ & Sample size & Def.~\ref{def:SampleSize}\\
\addlinespace
$\chi^2$ & The chi-squared test statistic & Sect.~\ref{TestStatObs}\\
$\pm$ & Plus-or-minus (give-or-take) & Sect.~\ref{CIpKnownp}\\
\bottomrule
\end{tabular}
\endgroup{}
\end{table}

\backmatter

\chapter*{Glossary}\label{Glossary}

\begin{small}

\begin{description}
\tightlist
\item[{\hyperref[def:EmpiricalRule]{\(68\)--\(95\)--\(99.7\) rule}\index{68@$68$--$95$--$99.7$ rule|textbf}}]
For \emph{any} bell-shaped distribution, \emph{approximately} \(68\)\% of values lie within one standard deviation of the mean, \(95\)\% of values lie within two standard deviations of the mean, and \(99.7\)\% of values lie within three standard deviations of the mean. Also called the \emph{empirical rule}. See also \emph{Normal distribution}.
\item[{\hyperref[PrecisionAccuracy]{Accuracy}\index{Accuracy|textbf}}]
\emph{Accuracy} refers to how close a \emph{sample} estimate is likely to be to the \emph{population} value, on average. See also \emph{Precision}.
\item[{\hyperref[def:AltHypothesis]{Alternative hypothesis}\index{Hypotheses!alternative|textbf}}]
The \emph{alternative hypothesis} \(H_1\) proposes that the discrepancy between the proposed value of the parameter and the observed value of the statistic cannot be explained by \emph{sampling variation}. It proposes that the value of the parameter is not the value claimed in the null hypothesis. See also \emph{Hypothesis test}, \emph{Null hypothesis}.
\item[{\hyperref[DistributionsExample]{Bell-shaped distributions}}]
See \emph{Normal distribution}.
\item[{\hyperref[def:ComparisonBetween]{Between-individual comparisons}\index{Comparison!between individuals|textbf}}]
See \emph{Comparison (between individuals)}, \emph{Comparison}.
\item[{\hyperref[def:BetweenWithinVariable]{\emph{Between}-individuals variables}\index{Variables!between-individuals|textbf}}]
\emph{Between}-individuals variables vary from one individual to another individual. See also \emph{Variables}, \emph{Within-individual variables}.
\item[{\hyperref[def:Bias]{Bias}\index{Bias|textbf}}]
\emph{Bias} refers to any systematic misrepresentation of the target population or a parameter caused by the sampling or the study design.
\item[{\hyperref[DescribingBlinding]{Blinding}\index{Blinding|textbf}}]
\emph{Blinding} occurs when those involved in the study do not know information about the study. A study can blind the \emph{researcher} to knowing what comparison group the individuals are in, the \emph{participants} to knowing what comparison group they are in, and/or the \emph{analysts} to knowing what comparison group the individuals are in during analysis.
\item[{\hyperref[def:Blocking]{Blocking}\index{Blocking|textbf}}]
\emph{Blocking} occurs when units of analysis are analysed as separate groups of similar units (called \emph{blocks}).
\item[{\hyperref[def:CarryoverEffect]{Carryover effect}\index{Carryover effect|textbf}}]
The \emph{carryover effect} occurs when the influence of one treatment or condition on the response variable influences the response variable for subsequent treatments or conditions (in a repeated-measures study).
\item[{\hyperref[def:UnitOfAnalysis]{Cases}\index{Cases|textbf}}]
\emph{Cases} are the individual units in the population; the \hyperref[def:UnitOfAnalysis]{\emph{units of analysis}}. Also called \emph{individuals}, or (when the individuals are people) \emph{subjects}.
\item[{\hyperref[def:QuantitativeData]{Categorical data}}]
See \emph{Qualitative data}.
\item[{\hyperref[NonRandomSamples]{Cherry-picking}\index{Sampling!non-random!cherry-picking|textbf}}]
\emph{Cherry-picking} is a non-random sampling method where individuals are specifically chosen to reach the conclusion that the researchers want.
\item[{\hyperref[TestStatObs]{Chi-square (\(\chi^2\)) score}}]
The \emph{chi-square (\(\chi^2\)) score} is the value of the test-statistic used to study the relationship between two qualitative variables.\index{Test statistic!$\chi^2$-score|textbf} The \(\chi^2\)-statistic measures the overall size of the differences between the expected counts and observed counts, over the entire two-way table.
\item[{\hyperref[def:ClassicalApproachToProbability]{Classical approach to probability}\index{Probability!classical approach|textbf}}]
In the \emph{classical approach to probability}, the probability of an event occurring is the number of elements of the sample space included in the event, divided by the total number of elements in the sample space, \emph{when all outcomes are equally likely} (i.e., no reason exists to expect one event to occur more often than the others). See also \emph{Relative-frequency approach to probability}, \emph{Sample space}, \emph{Subjective approach to probability}.
\item[{\hyperref[ClusterSampling]{Cluster sampling}\index{Sampling!random!cluster|textbf}}]
\emph{Cluster sampling} is a random sampling method where the population is split into a large number of small groups called \emph{clusters}, then a \emph{simple random sample} of clusters is selected and \emph{every} member of the chosen small groups is part of the sample. See also \emph{Simple random sampling}.
\item[Comparison]
In an RQ, a \emph{comparison} may be \emph{within} individuals, or \emph{between} groups of individuals. See also \emph{Comparison (between individuals)}, \emph{Comparison (within individuals)}.
\item[{\hyperref[def:ComparisonBetween]{Comparison (between individuals)}\index{Comparison!between individuals|textbf}}]
The \emph{between-individuals comparison} in the RQ identifies the small number of groups of different individuals for which the outcome is compared. See also \emph{Comparison}, \emph{Comparison (within individuals)}.
\item[{\hyperref[def:ComparisonWithin]{Comparison (within individuals)}}]
The \emph{within-individuals comparison} in the RQ identifies the small number of different, distinct situations for which the outcome is compared for each individual. See also \emph{Comparison}, \emph{Comparison (between individuals)}.
\item[{\hyperref[def:CompoundEvent]{Compound event}\index{Event!compound|textbf}}]
A \emph{compound event} is any combination of \emph{simple events}. See also \emph{Event}, \emph{Simple event}.
\item[{\hyperref[def:ConceptualDefinition]{Conceptual definition}\index{Definitions!conceptual|textbf}}]
A \emph{conceptual definition} articulates precisely \emph{what} words or phrases mean in a study. See also \emph{Operational definition}.
\item[{\hyperref[def:Conditions]{Conditions}\index{Conditions|textbf}}]
The \emph{conditions} are the values of the comparison that those in the \emph{observational} study have or experience, but are not manipulated or imposed by the researchers. See also \emph{Observational studies}, \emph{Treatments}.
\item[{\hyperref[def:ConfidenceInterval]{Confidence interval}\index{Confidence intervals|textbf}}]
A CI is an interval which contains the unknown value of the \hyperref[def:Parameter]{\emph{parameter}} a given percentage of the time (over repeated sampling). Informally: a \emph{confidence interval} (CI) is an interval likely to contain the unknown value of the \hyperref[def:Parameter]{\emph{parameter}}. We studied CIs in specific situations (see Sect.~\ref{Symbols}); there are hundreds more.
\item[{\hyperref[def:Confounding]{Confounding}\index{Confounding|textbf}}]
\emph{Confounding} is when a third variable influences the observed relationship between the response and explanatory variable.
\item[{\hyperref[def:ConfoundingVariable]{Confounding variable}\index{Variables!confounding|textbf}}]
A \emph{confounding variable} (or a \emph{confounder}) is an extraneous variable associated with the response \emph{and} explanatory variables. See also \emph{Confounding variable}, \emph{Extraneous variable}.
\item[{\hyperref[def:ContinuousData]{Continuous data}\index{Quantitative data!continuous|textbf}\index{Variables!continuous quantitative|textbf}}]
\emph{Continuous} quantitative data has (at least in theory) an infinite number of possible values between any two given values. See also \emph{Discrete data}, \emph{Quantitative data}.
\item[{\hyperref[def:Control]{Control}\index{Control|textbf}}]
A \emph{control} is a unit of analysis without the treatment or condition of interest, but as similar as possible in every other way to other units of analysis.
\item[{\hyperref[def:ControlVariables]{Control variable}\index{Control variable|textbf}}]
\emph{Control (or controlled) variables} are extraneous variables whose values are fixed for the study.
\item[{\hyperref[NonRandomSamples]{Convenience sampling}\index{Sampling!non-random!convenience|textbf}}]
\emph{Convenience sampling} is a non-random sampling method where individuals are selected because they are convenient for the researcher.
\item[{\hyperref[TwoQuant]{Correlation}\index{Correlation|textbf}}]
\emph{Correlation} refers to the association between two variables, measured by a correlation coefficient.
\item[{\hyperref[def:CorrelationCoefficient]{Correlation coefficient}\index{Correlation coefficient (Pearson)|textbf}}]
The (Pearson) \emph{correlation coefficient} (\(r\) for a sample; \(\rho\) for a population) measures the \emph{strength} and \emph{direction} of the \emph{linear} relationship between two quantitative variables. Its value is always between~\(-1\) and~\(1\). (Other types of correlation coefficients also exist.)
\item[{\hyperref[def:CorrelationalRQ]{Correlational research question}\index{Research question!correlational|textbf}}]
\emph{Correlational RQs} explore the relationship between two quantitative variables.
\item[{\hyperref[def:Data]{Data}\index{Data|textbf}}]
\emph{Data} refers to information (observations or measurements), such as numbers, labels, recordings, videos, text, etc. (such as height of seedlings, or the type of medication given).
\item[{\hyperref[def:Data]{Dataset}\index{Dataset|textbf}}]
A \emph{dataset} refers to an organised and structured collection of data.
\item[{\hyperref[def:DescriptiveRQ]{Descriptive research question}\index{Research question!descriptive|textbf}}]
\emph{Descriptive RQs} have a population and an outcome.
\item[{\hyperref[def:DescriptiveStudy]{Descriptive study}\index{Study types!descriptive|textbf}}]
\emph{Descriptive studies} answer descriptive research questions.
\item[{\hyperref[def:DiscreteData]{Discrete data}\index{Quantitative data!discrete|textbf}\index{Variables!discrete quantitative|textbf}}]
\emph{Discrete} quantitative data have a countable number of possible values between any two given values of the variable. See also \emph{Continuous data}, \emph{Quantitative data}.
\item[{\hyperref[def:Distribution]{Distribution}\index{Distribution|textbf}}]
The \emph{distribution} of a variable describes what values are present in the data, and how often those values appear. See also \emph{Normal distribution}.
\item[{\hyperref[def:EcologicalValidity]{Ecological validity}\index{Ecological validity|textbf}}]
A study is \emph{ecologically valid} if the study methods, materials and context closely approximate the real situation of interest.
\item[{\hyperref[SampleSpaceEvents]{Event}\index{Event|textbf}}]
An \emph{event} is any combination of the elements in the \hyperref[def:SampleSpace]{sample space}. See also \emph{Compound event}, \emph{Sample space}, \emph{Simple event}.
\item[{\hyperref[def:InclusionExclusionCriteria]{Exclusion criteria}\index{Exclusion criteria|textbf}}]
\emph{Exclusion criteria} are characteristics that disqualify potential individuals from being included in the study. See also \emph{Inclusion criteria}.
\item[{\hyperref[def:EmpiricalRule]{Empirical rule}}]
See the \emph{\(68\)--\(95\)--\(99.7\) rule}.
\item[{\hyperref[def:Experiment]{Experimental studies (or Experiments)}\index{Study types!experimental|textbf}}]
\emph{Experimental studies} (or \emph{experiments}) study relationships \emph{with} an intervention. See also \emph{Intervention}, \emph{Observational studies}.
\item[{\hyperref[def:ObserverEffect]{Experimenter effect}}]
See \emph{Observer effect}.
\item[{\hyperref[def:ExplanatoryVariable]{Explanatory variable}\index{Explanatory variable|textbf}}]
An \emph{explanatory variable} may (partially) explain or be associated with changes in another variable of interest (the response variable). In an experimental study, it is the variable that can be manipulated by the researchers. See also \emph{Response variable}.
\item[{\hyperref[def:ExternalValidity]{External validity}\index{External validity|textbf}}]
\emph{External validity} refers to the ability to generalise the results of the study to the rest of the population, beyond just those in the studied sample. For a study to be truly externally valid, the sample must be a random sample from the population. See also \emph{Internal validity}.
\item[{\hyperref[def:ExtraneousVariable]{Extraneous variable}\index{Variables!extraneous|textbf}}]
An \emph{extraneous} variable is any variable associated with the response variable, but is not the explanatory variable. See also \emph{Confounding variable}, \emph{Lurking variable}.
\item[{\hyperref[def:Extrapolation]{Extrapolation}\index{Extrapolation|textbf}}]
\emph{Extrapolation} refers to making a prediction outside the range of the available data. Extrapolation beyond the data may lead to nonsense.
\item[{\hyperref[def:HawthorneEffect]{Hawthorne effect}\index{Hawthorne effect|textbf}}]
The \emph{Hawthorne effect} is the tendency of individuals to change their behaviour if they know (or think) they are being observed.
\item[Hypothesis\index{Hypotheses|textbf}]
A \emph{hypothesis} is a possible answer to a (research) question. See also \emph{Alternative hypothesis}, \emph{Hypothesis test}, \emph{Null hypothesis}.
\item[Hypothesis test\index{Hypothesis testing|textbf}]
A \emph{hypothesis test} is a way to formally answer questions about a population, based on information obtained from a sample. In this book, we studied specific hypothesis tests (see Sect.~\ref{Symbols}); hundreds more exist.
\item[{\hyperref[def:InclusionExclusionCriteria]{Inclusion criteria}\index{Inclusion criteria|textbf}}]
\emph{Inclusion criteria} are characteristics that individuals must meet explicitly to be included in the study. See also \emph{Exclusion criteria}.
\item[{\hyperref[def:Independence]{Independence}\index{Independence|textbf}}]
Two events are \emph{independent} if the probability of one event doesn't change depending on whether or not other event has happened.
\item[{\hyperref[def:Population]{Individuals}\index{Individuals|textbf}}]
\emph{Individuals} are the units in the population from which the observations of interest could be made; the \hyperref[def:UnitOfAnalysis]{\emph{units of analysis}}. Also called \emph{Cases}, or \emph{Subjects} when the individuals are people. See also \emph{Units of analysis}.
\item[{\hyperref[def:InternalValidity]{Internal validity}\index{Internal validity|textbf}}]
\emph{Internal validity} refers to the extent to which a cause-and-effect relationship can be established in a study. A study with \emph{high} internal validity shows that the changes in the response variable can be (at least partially) attributed to changes in the explanatory variables; other explanations have been ruled out. See also \emph{External validity}.
\item[{\hyperref[def:Intervention]{Intervention}\index{Intervention|textbf}}]
An \emph{intervention} is present when researchers can manipulate (or impose) the values of the explanatory variable on the individuals to determine the impact on the response variable.
\item[{\hyperref[def:IQR]{IQR}\index{Interquartile range (IQR)|textbf}}]
The \emph{IQR} is a measure of variation. The \emph{IQR} is the range in which the middle~\(50\)\% of the data lie; the difference between the third and the first quartiles. See also \emph{Quartiles}.
\item[{\hyperref[def:IQRRuleForIdentifyingOutliers]{IQR rule for identifying outliers}\index{Outliers!IQR rule|textbf}}]
The \emph{IQR rule} is a way to identify outliers. The \emph{IQR rule} can identify outliers as either \emph{extreme} (observations \(3\times\text{IQR}\) more unusual than \(Q_1\) or \(Q_3\)) or \emph{mild} (observations \(1.5\times \text{IQR}\) more unusual than~\(Q_1\) or~\(Q_3\), that are not extreme outliers).
\item[{\hyperref[exm:DotsChartsQuant2]{Jittering}\index{Overplotting!jittering|textbf}}]
\emph{Jittering} is when a small amount of randomness is added in either the horizontal or vertical direction (or sometimes both) to separate points that would otherwise be overplotted. See also \emph{Overplotting}, \emph{Stacking}.
\item[{\hyperref[NonRandomSamples]{Judgement sampling}\index{Sampling!non-random!judgement|textbf}}]
\emph{Judgement sampling} is a non-random sampling method where individuals are selected, based on the researchers' judgement, depending on whether the researcher thinks they are likely to be agreeable or helpful.
\item[{\hyperref[def:Levels]{Levels} of a qualitative variable\index{Levels|textbf}\index{Qualitative data!levels|textbf}}]
The \emph{levels} (or the \emph{values}) of a qualitative variable refer to the names of the distinct categories of the variable.
\item[{\hyperref[def:LurkingVariable]{Lurking variable}\index{Variables!lurking|textbf}}]
A \emph{lurking variable} is an extraneous variable associated with the response \emph{and} explanatory variables (that is, a \emph{confounding} variable), but whose values \emph{are not} recorded in the study data. See also \emph{Confounding variable}, \emph{Extraneous variable}.
\item[{\hyperref[def:Mean]{Mean}\index{Mean!of a sample|textbf}}]
The \emph{mean} (\(\bar{x}\) for a sample; \(\mu\) for a population) is one way to measure the `average' value of quantitative data. The \emph{arithmetic mean} is the `balance point' of the data. The positive and negative distances from the mean add to zero. See also \emph{Median}.
\item[{\hyperref[def:Median]{Median}\index{Median!of a sample|textbf}}]
The \emph{median} is one way to measure the `average' value of some data. A \emph{median} is a value such that half the values are larger than the median, and half the values are smaller than the median. See also \emph{Mean}.
\item[{\hyperref[def:Mode]{Mode}\index{Mode|textbf}}]
A \emph{mode} is the level (or levels) of a qualitative variable with the most observations.
\item[{\hyperref[MultistageSampling]{Multi-stage sampling}\index{Sampling!random!multi-stage|textbf}}]
\emph{Multi-stage sampling} is a random sampling method where large groups are selected using a \emph{simple random sample}, then smaller groups within those large groups are selected using a \emph{simple random sample}. The simple randomly sampling can continue for as many levels as necessary. See also \emph{Simple random sampling}.
\item[{\hyperref[def:Nominal]{Nominal variable}\index{Variables!nominal qualitative|textbf}\index{Qualitative data!nominal|textbf}}]
A \emph{nominal} qualitative variable is a qualitative variable where the levels \emph{do not} have a natural order. See also \emph{Ordinal variable}, \emph{Qualitative data}.
\item[{\hyperref[def:SelectionBias]{Non-response bias}\index{Bias!non-response|textbf}}]
\emph{Non-response bias} occurs when chosen participants do not respond. See also \emph{Bias}.
\item[{\hyperref[DistributionsExample]{Normal distribution}\index{Normal distribution|textbf}}]
A \emph{normal distribution} is symmetrical distribution, with most values near the centre of the distribution (the mean). The normal distribution is described by its \emph{mean} and \emph{standard deviation}. A picture of a normal distribution is shown below. Normal distributions are also called \emph{bell-shaped} distributions. See also \(68\)--\(95\)--\(99.5\) rule.
\end{description}

\begin{center}\includegraphics[width=0.7\linewidth]{53-App-Glossary_files/figure-latex/DefNormalDistribution-1} \end{center}

\begin{description}
\tightlist
\item[{\hyperref[def:NullHypothesis]{Null hypothesis}\index{Hypotheses!null|textbf}}]
The \emph{null hypothesis} \(H_0\) proposes that \emph{sampling variation} explains the discrepancy between the proposed value of the parameter, and the observed value of the statistic. See also \emph{Alternative hypothesis}, \emph{Hypothesis test}.
\item[{\hyperref[def:SubjectiveObjective]{Objective data}\index{Data!objective|textbf}}]
\emph{Objective data} refers to facts and measurable evidence.
\item[{\hyperref[def:ObservationalStudy]{Observational studies}\index{Study types!observational|textbf}}]
\emph{Observational studies} study relationships \emph{without} an intervention. See also \emph{Experimental studies}.
\item[{\hyperref[def:ObserverEffect]{Observer effect}\index{Observer effect|textbf}}]
The \emph{observer effect} occurs when the researchers (unconsciously) change their behaviour to conform to expectations because they know what values of the explanatory variable apply to the individuals. This may then cause the \emph{individuals} to change their behaviour or reporting also.
\item[{\hyperref[def:Odds]{Odds}\index{Odds|textbf}}]
The \emph{odds} are the number (or proportion, or percentage) of results of interest, divided by the remaining number (or proportion, or percentage) of results. See also \emph{Probability}.
\item[{\hyperref[def:OddsRatio]{Odds ratio (OR)}\index{Odds ratio|textbf}}]
The \emph{odds ratio (OR)} is how many \emph{times} greater the odds of an event are in one group, compared to the odds of the \emph{same} event in a \emph{different} group. See also \emph{Odds}.
\item[{\hyperref[def:OperationalDefinition]{Operational definition}\index{Definitions!operational|textbf}}]
An \emph{operational definition} articulates exactly \emph{how} something will be identified, measured, observed or assessed. See also \emph{Conceptual definition}.
\item[{\hyperref[def:Ordinal]{Ordinal variable}\index{Variables!ordinal qualitative|textbf}\index{Qualitative data!ordinal|textbf}}]
An \emph{ordinal} qualitative variable is a qualitative variable where the levels \emph{do} have a natural order. See also \emph{Nominal variable}, \emph{Qualitative data}.
\item[Overplotting]
\emph{Overplotting} occurs when observations in a scatterplot or dot plot have the same, or nearly the same, values, and so are plotted at the same, or nearly the same, places on the graph. See also \emph{Jittering}, \emph{Stacking}.
\item[{\hyperref[def:Outcome]{Outcome}\index{Outcome|textbf}}]
The \emph{outcome} in an RQ is the result, output, consequence or effect of interest in a study, numerically summarised for a group of individuals.
\item[{\hyperref[def:Outliers]{Outliers}\index{Outliers|textbf}}]
An \emph{outlier} is an observation that is `unusual' (either larger or smaller) compared to the bulk of the data. Rules for identifying outliers are arbitrary. See also \emph{IQR rule for identifying outliers}, \emph{Standard deviation rule for identifying outliers}.
\item[{\hyperref[def:Pvalue]{\(P\)-value}\index{P@$P$-values|textbf}}]
A \emph{\(P\)-value} is the probability of observing the sample results (or something even more extreme) over repeated sampling, under the assumption that the null hypothesis about the population is true. \(P\)-values are used in decision-making. See also \emph{Hypothesis testing}.
\item[{\hyperref[def:Parameter]{Parameter}\index{Parameter|textbf}}]
A \emph{parameter} is a number, usually unknown, describing some feature of a population, and estimated by a \hyperref[def:Statistic]{statistic}. See also \emph{Statistic}.
\item[{\hyperref[def:PairedData]{Paired data}\index{Data!paired|textbf}}]
\emph{Paired data} occurs when the outcome in repeated-measures studies is compared for two different, distinct situations for each unit of analysis.\index{Study types!paired|textbf}
\item[{\hyperref[def:Percentage]{Percentage}\index{Percentages|textbf}}]
A \emph{percentage} is a \hyperref[def:Proportion]{proportion}, multiplied by~\(100\). In this context, percentages are numbers between~\(0\)\% and~\(100\)\%. See also \emph{Proportion}.
\item[{\hyperref[def:Percentiles]{Percentiles}\index{Percentiles|textbf}}]
The \(p\)th~\emph{percentile} of the data is a value separating the smallest~\(p\)\% of the data from the rest. See also \emph{Quartiles}.
\item[{\hyperref[def:PilotStudy]{Pilot study}\index{Pilot study|textbf}}]
A \emph{pilot study} is a small test run of the study used to check that the protocol is appropriate and practical, and to identify (and hence fix) possible problems with the research design or protocol.
\item[{\hyperref[def:Placebo]{Placebo}\index{Placebo|textbf}}]
A \emph{placebo} is a treatment with no intended effect or active ingredient, but appears to be the real treatment.
\item[{\hyperref[def:PlaceboEffect]{Placebo effect}\index{Placebo effect|textbf}}]
The \emph{placebo effect} occurs when individuals report perceived or actual effects despite not receiving an active treatment or condition, in experimental studies. See also \emph{Placebo}.
\item[{\hyperref[Referencing]{Plagiarism}\index{Plagiarism|textbf}}]
\emph{Plagiarism} is using other people's ideas and research to develop new conclusions, or confirm existing conclusions. All sources used when writing research should be acknowledged to avoid plagiarism.
\item[{\hyperref[def:Population]{Population}\index{Population|textbf}}]
A \emph{population} is a group of individuals (or cases, or subjects if the individuals are people) from which the total set of observations of interest could be made, and to which the results will generalise. See also \emph{Individuals}, \emph{Sample}, \emph{Units of analysis}.
\item[{\hyperref[PrecisionAccuracy]{Precision}\index{Precision|textbf}}]
\emph{Precision} refers to how similar the sample estimates from different samples are likely to be to each other (that is, how much variation is likely in the sample estimates). See also \emph{Accuracy}.
\item[{\hyperref[Probability]{Probability}\index{Probability|textbf}}]
A \emph{probability} is a number between zero and one inclusive (or between~\(0\)\% and~\(100\)\% inclusive) that quantifies the likelihood that a certain \emph{event} will occur. A probability of zero (or~\(0\)\%) means the event is `impossible' (will \emph{never} occur), and a probability of one (or~\(100\)\%) means that the event is \emph{certain} to happen (will always occur). Most events have a probability between the extremes of~\(0\)\% and~\(100\)\%. See also \emph{Odds}.
\item[{\hyperref[def:Proportion]{Proportion}\index{Proportions|textbf}}]
A \emph{proportion} is a fraction out of a total. Proportions (\(\hat{p}\) for a sample; \(p\) for a population) are numbers between~\(0\) and~\(1\). See also \emph{Percentage}.
\item[{\hyperref[def:Protocol]{Protocol}\index{Protocol|textbf}}]
A \emph{protocol} is a predefined procedure detailing the design and implementation of studies, and for data collection.
\item[{\hyperref[def:QualitativeData]{Qualitative data}\index{Qualitative data|textbf}}]
\emph{Qualitative data} are not \emph{mathematically} numerical data: they comprise mutually exclusive (and usually exhaustive) categories or labels (even if those labels are numbers). Also called \emph{Ordinal variable}, \emph{Qualitative data}, \emph{Quantitative data}.
\item[{\hyperref[def:QuantitativeData]{Quantitative data}\index{Quantitative data|textbf}}]
\emph{Quantitative data} are \emph{mathematically} numerical: the numbers have numerical meaning, and represent quantities or amounts. Quantitative data generally arise from counting or measuring. Also called \emph{Continuous data}, \emph{Discrete data}, \emph{Qualitative data}.
\item[{\hyperref[def:QualitativeResearch]{Quantitative research}\index{Research!quantitative|textbf}}]
\emph{Quantitative research} summarises and analyses data (quantitative or qualitative data) using numerical methods, such as producing averages and percentages.
\item[{\hyperref[def:Quartiles]{Quartiles}\index{Quartiles|textbf}}]
\emph{Quartiles} describe the variation and shape of data. The first quartile~\(Q_1\) is a value that separates the smallest~\(25\)\% of observations from the largest~\(75\)\%; it is like the median of the \emph{smaller} half of the data, halfway between the minimum value and the median.\\
\strut ~~\\
The second quartile~\(Q_2\) is a value that separates the smallest~\(50\)\% of observations from the largest~\(50\)\% (and is also the \emph{median}).\\
\strut ~~\\
The third quartile~\(Q_3\) is a value that separates the smallest~\(75\)\% of observations from the largest~\(25\)\%; it is like the median of the \emph{larger} half of the data, halfway between the median and the maximum value. See also \emph{Median}, \emph{Percentiles}.
\item[{\hyperref[def:QuasiExperiment]{Quasi-experiment}\index{Study types!experimental!quasi|textbf}}]
In a \emph{quasi-experiment}, the researchers (1)~allocate treatments to groups of individuals (i.e., allocate the values of the explanatory variable to the individuals, as it is an experiment), but (2)~do \emph{not} determine who or what is in those groups. See also \emph{True experiment}.
\item[{\hyperref[def:Questionnaire]{Questionnaire}\index{Questionnaire|textbf}}]
A \emph{questionnaire} is a set of questions for respondents to answer.
\item[{\hyperref[def:Random]{Random}\index{Random|textbf}}]
\emph{Random} means `determined completely by impersonal chance'. See also \emph{Simple random sampling}.
\item[{\hyperref[def:RandomProcedure]{Random procedure}\index{Random procedure|textbf}}]
A \emph{random procedure} is a sequence of well-defined steps that (a)~can be repeated, in theory, indefinitely under essentially identical conditions; (b)~has well-defined results; and (c)~has result that are unpredictable for any individual repetition.
\item[{\hyperref[def:RandomSampling]{Random sample}\index{Sampling!random|textbf}}]
In a \emph{random sample}, each individual in the population can be selected; and each individual is chosen on the basis of \emph{impersonal} chance. See also: \emph{Simple random sampling}, \emph{Representative sampling}.
\item[{\hyperref[def:Range]{Range}\index{Range|textbf}}]
The \emph{range} is a measure of variation. The \emph{range} is the maximum value \emph{minus} the minimum value.
\item[{\hyperref[def:RelationalRQ]{Relational research question}\index{Research question!relational|textbf}}]
\emph{Relational RQs} have a population, outcome, and a \emph{between}-individuals comparison.
\item[{\hyperref[def:RelativeFrequencyApproachToProbability]{Relative frequency approach to probability}\index{Probability!relative frequency approach|textbf}}]
In the \emph{relative frequency approach to probability}, the probability of an event is approximately the number of times the outcomes of interest has appeared in the past, divided by the number of `attempts' in the past. This produces an \emph{approximate} probability. See also \emph{Classical approach to probability}, \emph{Subjective approach to probability}.
\item[{\hyperref[def:RepeatedMeasuresRQ]{Repeated-measures research question}\index{Research question!repeated-measures|textbf}}]
\emph{Repeated-measures RQs} have a population, outcome and a \emph{within}-individuals comparison.
\item[{\hyperref[def:RepresentativeSample]{Representative sample}\index{Sampling!representative|textbf}}]
In a \emph{representative sample}, those \emph{in} the sample are not likely to be different from those \emph{not in} the sample, at least for the variables of interest. A representative sample is \emph{not} a random sample. See also: \emph{Random sample}.
\item[{\hyperref[def:StudyDesign]{Research design}\index{Research design|textbf}}]
\emph{Research design} refers to the decisions made by the researchers to maximise \emph{external validity} and \emph{internal validity}.
\item[{\hyperref[def:SelectionBias]{Response bias}\index{Bias!response|textbf}}]
\emph{Response bias} occurs when participants provide \emph{incorrect or false information}.
\item[{\hyperref[def:ResponseVariable]{Response variable}\index{Response variable|textbf}}]
A \emph{response variable} records the result, output, consequence or effect of interest from changes in another variable (the explanatory variable). See also \emph{Explanatory variable}.
\item[{\hyperref[def:Sample]{Sample}\index{Sample|textbf}}]
A \emph{sample} is a subset of individuals from the population. The data are collected from the sample. Usually, countless possible samples could be obtained from a population. See also \emph{Population}, \emph{Sample size}.
\item[{\hyperref[def:SampleSize]{Sample size}\index{Sample size|textbf}}]
The sample size~\(n\) is the number of units of analysis. See also \emph{Population}, \emph{Sample}, \emph{Random sample}.
\item[{\hyperref[def:SampleSpace]{Sample space}\index{Sample space|textbf}}]
The \emph{sample space} is a list of all possible and mutually exclusive (distinct) results after administering a random procedure once. See also \emph{Event}.
\item[{\hyperref[def:SamplingDistribution]{Sampling distribution}\index{Sampling distribution|textbf}}]
A \emph{sampling distribution} is the distribution of a statistic, showing how its value varies across all possible samples. See also \emph{Sampling mean}, \emph{Standard error}.
\item[{\hyperref[def:SamplingFrame]{Sampling frame}\index{Sampling frame|textbf}}]
The \emph{sampling frame} is a list of all the individuals in the population.
\item[{\hyperref[def:SamplingMean]{Sampling mean}\index{Sampling mean|textbf}}]
The \emph{sampling mean} is the mean of the sampling distribution of a statistic: the mean of the values of the statistic from all possible samples. See also \emph{Sampling distribution}, \emph{Sampling mean}, \emph{Sampling variation}, \emph{Standard error}.
\item[{\hyperref[def:SamplingVariation]{Sampling variation}\index{Sampling variation|textbf}}]
\emph{Sampling variation} refers to how the sample estimates (statistics) vary from sample to sample, because every possible sample is different. See also \emph{Sampling distribution}, \emph{Sampling mean}, \emph{Standard error}.
\item[{\hyperref[def:SelectionBias]{Selection bias}\index{Bias!selection|textbf}}]
\emph{Selection bias} is the tendency of a sample to over- or under-estimate a population quantity. See also \emph{Bias}.
\item[{\hyperref[def:QualitativeData]{Scale data}}]
See \emph{Quantitative data}.
\item[{\hyperref[def:SimpleEvent]{Simple event}\index{Event!simple|textbf}}]
A \emph{simple event} is a single element of the sample space. See also \emph{Compound event}, \emph{Event}, \emph{Sample space}.
\item[{\hyperref[def:SamplingSRS]{Simple random sampling}\index{Sampling!random!simple random|textbf}}]
\emph{Simple random sampling} is a random sampling method where \emph{every} possible sample of a given size has \emph{same} chance of being selected.
\item[{\hyperref[exm:DotsChartsQuant2]{Stacking}\index{Overplotting!stacking|textbf}}]
\emph{Stacking} is when points are plotted above other points with similar values, to separate points that would otherwise be overplotted. See also \emph{Jittering}, \emph{Stacking}.
\item[{\hyperref[def:StandardDeviation]{Standard deviation}\index{Standard deviation!of a sample|textbf}}]
The \emph{standard deviation} (\(s\) for a sample; \(\sigma\) for a population) is a measure of variation. The \emph{standard deviation} is, approximately, the mean distance of observations from the mean.
\item[{\hyperref[def:StandardDeviationRuleForIdentifyingOutliers]{Standard deviation rule for identifying outliers}}]
The \emph{standard deviation rule} is a way to identify outliers For approximately symmetric distributions, any observation more than three standard deviations from the mean can be considered an outlier.
\item[{\hyperref[def:StandardError]{Standard error}\index{Standard error|textbf}}]
A \emph{standard error} is the standard deviation of all possible values of the sample estimate (from samples of a certain size): the standard deviation of the values of the statistic from all possible samples. Any quantity estimated from a sample has a standard error. See also \emph{Sampling distribution}, \emph{Sampling mean}, \emph{Sampling variation}.
\item[{\hyperref[StratifiedSampling]{Stratified sampling}\index{Sampling!random!stratified|textbf}}]
\emph{Stratified sampling} is a random sampling method where the population is split into a small number of large (usually similar) groups called \emph{strata}, then cases are selected using a \emph{simple random sample} from \emph{each} stratum. See also \emph{Simple random sampling}.
\item[{\hyperref[def:Statistic]{Statistic}\index{Statistic|textbf}}]
A \emph{statistic} is a number describing some feature of a sample (to estimate the unknown value of the population \hyperref[def:Parameter]{parameter}). See also \emph{Parameter}.
\item[{\hyperref[def:StatisticalValidity]{Statistical validity}\index{Statistical validity (for inference)|textbf}}]
A result is \emph{statistically valid} if the conditions for the underlying mathematical calculations to be approximately correct are met, such as the sampling distribution having an approximate normal distribution. Every confidence interval and hypothesis test has statistical validity conditions.
\item[{\hyperref[def:SubjectiveApproachToProbability]{Subjective approach to probability}\index{Probability!subjective approach|textbf}}]
In the \emph{subjective approach to probability}, various factors are incorporated subjectively to determine the probability of an event occurring. See also \emph{Relative-frequency approach to probability}, \emph{Subjective approach to probability}.
\item[{\hyperref[def:SubjectiveObjective]{Subjective data}\index{Data!subjective|textbf}}]
\emph{Subjective data} refers to opinions, feelings, and interpretations (by the subjects or the researchers).
\item[{\hyperref[def:UnitOfAnalysis]{Subjects}\index{Subjects|textbf}}]
The individual units in the population when the units are people; the \hyperref[def:UnitOfAnalysis]{\emph{units of analysis}}. Also called \emph{individuals} or \emph{cases}; however, those two terms do not refer exclusively to people. See also \emph{Units of analysis}.
\item[{\hyperref[SystematicSampling]{Systematic sampling}\index{Sampling!random!systematic|textbf}}]
\emph{Systematic sampling} is a random sampling method where the first case is \emph{randomly} selected; then, every \(n\)th individual is selected thereafter.
\item[{\hyperref[TestStatistic]{\(t\)-score}\index{Test statistic!t@$t$-score|textbf}}]
A \emph{\(t\)-score} measures how many standard deviations a value is from the mean. A \(t\)-score is similar to a \(z\)-score. See also \emph{\(z\)-score}.
\item[{\hyperref[def:Treatments]{Treatments}\index{Treatments|textbf}}]
The \emph{treatments} are the values of the explanatory variable that the researchers can manipulate and impose upon the individuals in the \emph{experimental} study. See also \emph{Conditions}, \emph{Experiments}.
\item[{\hyperref[def:TrueExperiment]{True experiment}\index{Study types!experimental!true|textbf}}]
In a \emph{true experiment}, the researchers (1)~allocate treatments to groups of individuals (i.e., values of the explanatory variable to the individuals), \emph{and} (2)~determine who or what is in those groups. While the steps may not happen \emph{explicit}, they happen \emph{conceptually}. See also \emph{Quasi-experiment}.
\item[{\hyperref[def:UnitOfObservation]{Unit of observation}\index{Units of observation|textbf}}]
The \emph{unit of observation} is the entity that is observed, from or about which measurements are taken and data collected. See also \emph{Unit of analysis}.
\item[{\hyperref[def:UnitOfAnalysis]{Unit of analysis}\index{Units of analysis|textbf}}]
The \emph{unit of analysis} are the smallest collection of units of observations (and perhaps the units of observations themselves) about which conclusions are made; the smallest distinct, \emph{independent} elements of the population for which information is analysed. In an \emph{experimental study}, the unit of analysis is the smallest collection of units of observations that can be randomly allocated to separate treatments. See also \emph{Individuals}, \emph{Unit of observation}.
\item[{\hyperref[def:UnstandardisingFormula]{Unstandardising formula}\index{Unstandardising formula|textbf}}]
When the \(z\)-score is known, the \emph{unstandardising formula} determines the corresponding value of the observation \(x\): \(x = \mu + z(z\times\sigma)\). See also \emph{\(z\)-score}.
\item[{\hyperref[def:Levels]{Values} of a qualitative variable}]
See \emph{Levels}.
\item[{\hyperref[def:Variable]{Variables}\index{Variables|textbf}}]
A \emph{variable} is a single aspect or characteristic associated with the individuals, whose values can vary from individual to individual.
\item[{\hyperref[NonRandomSamples]{Voluntary response (self-selecting) sampling}\index{Sampling!non-random!voluntary|textbf}}]
\emph{Voluntary response (or self-selecting) sampling} is a non-random sampling method where individuals participate if they wish to.
\item[{\hyperref[def:ComparisonWithin]{Within-individuals comparison}\index{Comparison!within individuals|textbf}}]
See \emph{Comparison (within individuals)}, \emph{Comparison}.
\item[{\hyperref[def:BetweenWithinVariable]{\emph{Within}-individuals variables}\index{Variables!within-individuals|textbf}}]
\emph{Within}-individuals variables vary from one recording or measurement to another \emph{within} the same individuals. See also \emph{Between-individual variables}, \emph{Variables}.
\item[{\hyperref[def:zScore]{\(z\)-score}\index{z@$z$-score|textbf}\index{Test statistic!z@$z$-score|textbf}}]
A \emph{\(z\)-score} measures how many standard deviations a value is from the mean. In symbols: \[
   z 
   = \frac{\text{value} - \text{mean of the distribution}}{\text{standard deviation of the distribution}}
   = \frac{x - \mu}{\sigma} 
\] where~\(x\) is the value, \(\mu\) is the mean of the distribution, and~\(\sigma\) is the standard deviation of the distribution. See also \emph{\(t\)-score}.
\end{description}

\end{small}

\chapter*{Answers to odd-numbered exercises}\label{Answers}

\captionsetup{font=footnotesize}

\subsection*{Chap.~\ref{Intro}: Research an introduction}\label{chap.-refintro-research-an-introduction}

\begin{ChapAnswers}

\begin{answer}
\textbf{Ex.~\ref{exr:RQsTypeTourniquet}.} \textbf{1.} From many people: type of tourniquet; time to apply. \textbf{2.} \textbf{Quant}itative.

\end{answer}

\begin{answer}
\textbf{Ex.~\ref{exr:RQsTypeSideEffects}.} \textbf{1.} For many Egyptians: whether side effects experienced after medication. \textbf{2.} \textbf{Quant}itative.

\end{answer}

\begin{answer}
\textbf{Ex.~\ref{exr:RQsTypeGreenery}.} \textbf{Qual}itative

\end{answer}

\end{ChapAnswers}

\subsection*{Chap.~\ref{RQs}: Research questions}\label{chap.-refrqs-research-questions}

\begin{ChapAnswers}

\begin{answer}
\textbf{Ex.~\ref{exr:RQsOutcomeResponse1}.} \textbf{1.} \emph{Percentage} of vehicles that crash. \textbf{2.} \emph{Average} jump height. \textbf{3.} \emph{Average} number of tomatoes per plant.

\end{answer}

\begin{answer}
\textbf{Ex.~\ref{exr:RQsComparisonExplanatory1}.} \textbf{1.} Whether diet is vegan. \textbf{2.} Whether coffee is caffeinated \textbf{3.} Num. iron tablets/day.

\end{answer}

\begin{answer}
\textbf{Ex.~\ref{exr:RQsComparisonVsPaired1}.} \textbf{1.} \emph{Between}-individuals. Outcome: percentage wearing hats. \textbf{2.} \emph{Between}-intervals. Outcome: average yield (in kg/plant, tomatoes/plant, etc.).

\end{answer}

\begin{answer}
\textbf{Ex.~\ref{exr:RQsDogs}.} \textbf{1.} Correlational. \textbf{2.} Neither variable is explanatory, response.

\end{answer}

\begin{answer}
\textbf{Ex.~\ref{exr:RQsBloodPressure}.} \textbf{1.} P: Danish Uni students; O: \emph{average} resting diastolic blood pressure; C: between students who regularly drive, ride their bicycles to uni. \textbf{2.} No intervention. \textbf{3.} Relational. \textbf{4.} Decision-making. \textbf{5.} \emph{Conceptual}: `regularly'; `university student' (on-campus? undergraduate?). \emph{Operational}: how `resting diastolic blood pressure' measured. \textbf{6.} Resting diastolic blood pressure; whether they regularly drive, ride to uni. \textbf{7.} Both: Danish uni students.

\end{answer}

\begin{answer}
\textbf{Ex.~\ref{exr:RQsWalkingSpeed}.} \textbf{1.} Probably relational. \textbf{2.} Two-tailed. \textbf{3.} Probably not, but possible. \textbf{4.} How \emph{individual} people using phones (`Talking'; `texting'). \textbf{5.} Walking speed. \textbf{6.} \emph{Average} walking speed. \textbf{7.} Observation: person. Analysis: individuals not in a group, as people in group may not walk independently of others (i.e., they keep pace with each other).

\end{answer}

\begin{answer}
\textbf{Ex.~\ref{exr:RQsAnimals}.} \textbf{1.} Animal. \textbf{2.} \emph{Pen}: food allocated to pen. Animals in same pen \emph{not} independent: compete for same space, food, resources, have similar environments. \textbf{3.} Between diets.

\end{answer}

\begin{answer}
\textbf{Ex.~\ref{exr:ProjectRQ2}.} Ten adults is sample. Unclear how many (or which) fonts compared. Perhaps: `Among Australian adults, does average time taken to read passage of text differ when Arial font used compared to Times Roman font?'

\end{answer}

\begin{answer}
\textbf{Ex.~\ref{exr:RQsNoseHair}.} \textbf{1.} Analysis: person; Observation: individual nose hairs. Each unit of analysis has \(50\) units of observation. \textbf{2.} \(n = 2\).

\end{answer}

\begin{answer}
\textbf{Ex.~\ref{exr:POCIaccelerometer}.} \textbf{1.} P: American adults; individuals: American adults. \textbf{2.} O: average number recorded steps. \textbf{3.} Response: number steps recorded for individuals. Explanatory: location of accelerometer. \textbf{4.} \emph{Within} individuals.

\end{answer}

\begin{answer}
\textbf{Ex.~\ref{exr:RQsComparisonConnectionCaloric}.} \textbf{1.} Relational; decision-making. \textbf{2.} Correlational; estimation. Intervention unlikely.

\end{answer}

\begin{answer}
\textbf{Ex.~\ref{exr:WhatAreUnitsAnalysisA}.} \(n = 27\); unit of analysis: emu.

\end{answer}

\begin{answer}
\textbf{Ex.~\ref{exr:UnitsAnalysis}.} Unit of observation: tyre. Unit of analysis: car (brand allocated to car; each car gets one brand). Tyres on any car exposed to same day-to-day use, drivers, distances, etc. Each unit of analysis produces four units of observations. \emph{Sample size}: \(10\) cars (\(40\) observations).

\end{answer}

\begin{answer}
\textbf{Ex.~\ref{exr:UnitsAnalysisBamboo}.} \textbf{1.} Board. \textbf{2.} \(5\). \textbf{3.} \(10\). \textbf{4.} \(10\).. \textbf{5.} Within-board variation smaller (except first). \textbf{6.} \(1\).

\end{answer}

\end{ChapAnswers}

\subsection*{Chap.~\ref{ResearchDesignOverview}: Overview of research design}\label{chap.-refresearchdesignoverview-overview-of-research-design}

\begin{ChapAnswers}

\begin{answer}
\textbf{Ex.~\ref{exr:MineArsenic}.} \textbf{1.} Arsenic conc. \textbf{2.} Distance of lake from mine. \textbf{3.} No: recorded. \textbf{4.} Yes: may be related to response, explanatory variables. \textbf{5.} Probably confounding.

\end{answer}

\begin{answer}
\textbf{Ex.~\ref{exr:ResearchDesignOverviewStudy2}.} \emph{Response}: perhaps `whether woman develops cancer of digestive system'. \emph{Explanatory}: `whether participants drank green tea at least \(3\)~times per week'. \emph{Lurking}: `health consciousness of participants' (appears unrecorded).

\end{answer}

\begin{answer}
\textbf{Ex.~\ref{exr:ResearchDesignOverviewRTEC}.} \textbf{1.} Sex; number siblings. \textbf{2.} No. \textbf{3.} Yes.

\end{answer}

\begin{answer}
\textbf{Ex.~\ref{exr:DesignFindConfounderA}.} Age of person.

\end{answer}

\begin{answer}
\textbf{Ex.~\ref{exr:DesignConfoundingVarA}.} \textbf{1.} Soil quality; climate; size. Control var.: perhaps farm size (e.g., farms over certain size).

\end{answer}

\end{ChapAnswers}

\pagebreak

\subsection*{Chap.~\ref{ResearchDesign}: Types of research studies}\label{chap.-refresearchdesign-types-of-research-studies}

\begin{ChapAnswers}

\begin{answer}
\textbf{Ex.~\ref{exr:TypesOfDesignsAcuteOtitis}.} \textbf{1.} Between-individuals. \textbf{2.} Relational. \textbf{3.} Most likely. \textbf{4.} Estimation. \textbf{5.} Intervention: experiment. Likely true experiment.

\end{answer}

\begin{answer}
\textbf{Ex.~\ref{exr:ResearchDesignConcreteBeams}.} True experiment.

\end{answer}

\begin{answer}
\textbf{Ex.~\ref{exr:ResearchDesignMatresses}.} Quasi-experiment.

\end{answer}

\begin{answer}
\textbf{Ex.~\ref{exr:ResearchDesignPetsAndHealth}.} \textbf{1.} Answers vary. \textbf{2.} Researchers \emph{intervene}: \emph{give}, \emph{not give} subjects pet. \textbf{3.} Researchers \emph{do not intervene}: subjects already \emph{do}, \emph{do not} own pet.

\end{answer}

\end{ChapAnswers}

\subsection*{Chap.~\ref{Ethics}: Ethics in research}\label{chap.-refethics-ethics-in-research}

\begin{ChapAnswers}

\begin{answer}
\textbf{Ex.~\ref{exr:EthicsCougars}.} Answers vary.

\end{answer}

\begin{answer}
\textbf{Ex.~\ref{exr:EthicsSideEffects}.} Answers vary.

\end{answer}

\end{ChapAnswers}

\subsection*{Chap.~\ref{Sampling}: External validity: sampling}\label{chap.-refsampling-external-validity-sampling}

\begin{ChapAnswers}

\begin{answer}
\textbf{Ex.~\ref{exr:SamplingAdvantageRandom}.} c.~Externally-valid study more likely.

\end{answer}

\begin{answer}
\textbf{Ex.~\ref{exr:SamplingOverUnderA}.} \textbf{1.} Under: practicality; intentional. \textbf{2.} Under; deception; probably intentional. \textbf{3.} Over; deception; intentional (cherry-picking).

\end{answer}

\begin{answer}
\textbf{Ex.~\ref{exr:SamplingSystematicProblem}.} \textbf{1.} Every \(7\)th day is same day of week. \textbf{2.} Maybe select days at random over three-months.

\end{answer}

\begin{answer}
\textbf{Ex.~\ref{exr:SamplingApartments}.} \textbf{1.} Multi-stage. \textbf{2.} Stratified (floor), then convenience. \textbf{3.} Convenience. \textbf{4.} Part stratified (floors), then convenience. Use first.

\end{answer}

\begin{answer}
\textbf{Ex.~\ref{exr:SamplingSchools}.} Random sampling (schools), then, self-selecting.

\end{answer}

\begin{answer}
\textbf{Ex.~\ref{exr:SamplingForest}.} Stratified by zone; then convenience.

\end{answer}

\begin{answer}
\textbf{Ex.~\ref{exr:NotMultistageSampling}.} Stage~\(3\) selection \emph{not} random.

\end{answer}

\begin{answer}
\textbf{Ex.~\ref{exr:SolarSampling}.} \textbf{1.} Households in Santiago. \textbf{2.} (c): all households in Santiago. \textbf{3.} Voluntary response. \textbf{4.} Multi-stage.

\end{answer}

\end{ChapAnswers}

\subsection*{Chap.~\ref{DesignInternal}: Internal validity}\label{chap.-refdesigninternal-internal-validity}

\begin{ChapAnswers}

\begin{answer}
\textbf{Ex.~\ref{exr:DesignTrueFalse}.} All false.

\end{answer}

\begin{answer}
\textbf{Ex.~\ref{exr:DesignExpImproveIV}.} All but random samples (\emph{external}).

\end{answer}

\begin{answer}
\textbf{Ex.~\ref{exr:ResearchDesignHawthorneEffect}.} Also possible in observational studies.

\end{answer}

\begin{answer}
\textbf{Ex.~\ref{exr:ResearchDesignObsPollen}.} In case hive size a confounder.

\end{answer}

\begin{answer}
\textbf{Ex.~\ref{exr:DesignExpWeightLoss}.} Statements~1, 3, 4 and~8 true. `Sex', `Initial weight' \emph{possible} confounders.

\end{answer}

\begin{answer}
\textbf{Ex.~\ref{exr:ResearchDesignSunscreen}.} \textbf{1.} Observational. \textbf{2.} \emph{Response}: amount sunscreen used; \emph{explanatory}: time applying sunscreen. \textbf{3.} Potential extraneous, confounding variables. \textbf{4.} To determine if sex a \emph{confounder}. \textbf{5.} Participants, researchers blinded.

\end{answer}

\begin{answer}
\textbf{Ex.~\ref{exr:ResearchIcelandicGood}.} Random allocation; exclusion criteria; blinding; comparing two groups; ethical.

\end{answer}

\begin{answer}
\textbf{Ex.~\ref{exr:ResearchTanzaniaFarm}.} \textbf{1.} A: no blinding; B: double-blind. \textbf{2.} Hawthorne effect impacting internal validity. \textbf{3.} B; no Hawthorne effect.

\end{answer}

\begin{answer}
\textbf{Ex.~\ref{exr:ResearchTasteOfWater2}.} Randomly allocate type of water to subjects (or the \emph{order} that subjects taste \emph{each} drink.) Subjects blind to water type. Person providing water and receiving ratings blind to water type. Random or representative sampling is good (but hard). Use third-party if possible.

\end{answer}

\begin{answer}
\textbf{Ex.~\ref{exr:ResearchDesignFertilizer}.} Carryover effect; observer effect.

\end{answer}

\begin{answer}
\textbf{Ex.~\ref{exr:ResearchDesignFormwork}.} \textbf{1.} Floor area. \textbf{2.} Hours labour. \textbf{3.} Extraneous. \textbf{4.} Analysis \textbf{5.} F. \textbf{6.} T. \textbf{7.} F. \textbf{8.} F.

\end{answer}

\end{ChapAnswers}

\subsection*{Chap.~\ref{Interpretation}: Research design limitations}\label{chap.-refinterpretation-research-design-limitations}

\begin{ChapAnswers}

\begin{answer}
\textbf{Ex.~\ref{exr:ValidityLighting}.} External.

\end{answer}

\begin{answer}
\textbf{Ex.~\ref{exr:InterpretationExerciseExternalValidity}.} Population is students; external validity if applies to students at UniX not wider.

\end{answer}

\begin{answer}
\textbf{Ex.~\ref{exr:InterpretationSleep}.} Sample not random; researchers (rightly) state results may not \emph{generalise} to all hospitals. Data collected at night; not \emph{ecologically valid}?

\end{answer}

\begin{answer}
\textbf{Ex.~\ref{exr:InterpretationCoughDrops}.} Observational study: people with severe cough may take more cough drops.

\end{answer}

\begin{answer}
\textbf{Ex.~\ref{exr:ValidityBeer}.} Lacks \emph{ecological validity}.

\end{answer}

\end{ChapAnswers}

\subsection*{Chap.~\ref{CollectingDataProcedures}: Collecting data}\label{chap.-refcollectingdataprocedures-collecting-data}

\begin{ChapAnswers}

\begin{answer}
\textbf{Ex.~\ref{exr:CollectSurveyQuestions1}.} No place for \(18\)-year-olds.

\end{answer}

\begin{answer}
\textbf{Ex.~\ref{exr:CollectSurveyQuestions2}.} Best: second. First: \emph{leading} (\emph{concerned} cat owners\ldots) Third: \emph{leading} (Do you \emph{agree}\ldots)

\end{answer}

\begin{answer}
\textbf{Ex.~\ref{exr:SunscreenQuestions}.} First fine; `seldom' (for instance) may have different meanings to different people; possible recall bias. Second: overlapping options (both \(1\,\text{h}\) and \(2\,\text{h}\) in two categories).

\end{answer}

\end{ChapAnswers}

\pagebreak

\subsection*{Chap.~\ref{DescribingVars}: Classifying data and variables}\label{chap.-refdescribingvars-classifying-data-and-variables}

\begin{ChapAnswers}

\begin{answer}
\textbf{Ex.~\ref{exr:DescribeClassifying1}.} Quant.~continuous. Qual.~nominal. Quant.~continuous. Qual.~nominal.

\end{answer}

\begin{answer}
\textbf{Ex.~\ref{exr:DescribeClassifying2}.} F, T, F.

\end{answer}

\begin{answer}
\textbf{Ex.~\ref{exr:DescribeClassifying3}.} Nominal; qualitative.

\end{answer}

\begin{answer}
\textbf{Ex.~\ref{exr:VariablesLevelsA}.} Sex of person

\end{answer}

\begin{answer}
\textbf{Ex.~\ref{exr:DescribeClassifyingVariables1}.} \textbf{1.} Quant.~continuous. \textbf{2.} Qual.~nominal. \textbf{3.} Qual.~ordinal. \textbf{4.} Quant.~discrete.

\end{answer}

\begin{answer}
\textbf{Ex.~\ref{exr:BRFSS}.} \textbf{1.} Qual.~nominal. \textbf{2.} Quant.~discrete. \textbf{3.} Qual.~ordinal (perhaps quant.~discrete). \textbf{4.} Qual.~nominal. \textbf{5.} Quant.~continuous.

\end{answer}

\begin{answer}
\textbf{Ex.~\ref{exr:DescribeClassifyingOrthoses}.} \emph{Gender}: qual.~nominal. \emph{Age}: quant.~continuous. \emph{Height}: quant.~continuous. \emph{Weight}: quant.~continuous. \emph{\textsc{gmfcs}}: qual.~ordinal.

\end{answer}

\begin{answer}
\textbf{Ex.~\ref{exr:DescribeClassifyingKangaroos}.} \emph{Kangaroo response}: qual.~ordinal (perhaps nominal?). \emph{Drone height}: quant.; four values used; probably treated as qual.~ordinal. \emph{Mob size}: quant.~discrete. \emph{Sex}: qual.~nominal.

\end{answer}

\end{ChapAnswers}

\subsection*{Chap.~\ref{SummariseQuantData}: Summarising quantitative data}\label{chap.-refsummarisequantdata-summarising-quantitative-data}

\begin{ChapAnswers}

\begin{answer}
\textbf{Ex.~\ref{exr:GraphABSdeaths}.} Shape: skewed \emph{left}. Average: perhaps \(70\)--\(100\)? Variation: most between \(30\), \(80\). Outliers: none; `bump' at lower ages.

\end{answer}

\begin{answer}
\textbf{Ex.~\ref{exr:NumericalQuantNHANES}.} \textbf{1.} Probably median (but mean probably OK). \textbf{2.} Slightly right skewed; average near \(1.5\,\text{mmol}\)/L; most between~\(3\), \(4\,\text{mmol}\)/L; some large outliers.

\end{answer}

\begin{answer}
\textbf{Ex.~\ref{exr:NumericalQuantRides}.} \textbf{1.} \(3.7\). \textbf{2.} \(3.5\). \textbf{3.} \(1.888562\). \textbf{4.} \(5 - 2 = 3\).

\end{answer}

\begin{answer}
\textbf{Ex.~\ref{exr:NumericalQuantSOI}.} Plot not shown. \textbf{1.} \(-2.42\). \textbf{2.} \(0.8\). \textbf{3.} \(29.6\) (\(-19.8\) to~\(9.8\)). \textbf{4.} \(9.831172\); about \(9.83\). \textbf{5.} IQR: \(4.95 - (-11.4) = 16.35\) (\emph{not} including median in each half). (No units of measurement.)

\end{answer}

\begin{answer}
\textbf{Ex.~\ref{exr:SummaryQuantOrthoses}.} \textbf{1.} In cm: \(127.4\); \(129.0\); \(14.4\); \(24\) from software. Manually (\emph{without} median in each half): \(Q_1 = 113\), \(Q_3 = 138\), IQR is \(25\). \textbf{2.} Don't know. \textbf{3.--5.} Not shown. \textbf{6.} Hard to describe with standard language; approx. symmetric?.

\end{answer}

\begin{answer}
\textbf{Ex.~\ref{exr:NumericalQuantDescribeBrainFreezeHistogram}.} In seconds: Average about \(10\)?; variation \(0\) to~\(30\) perhaps; skewed right. Value between \(35\) and~\(40\) perhaps outlier.

\end{answer}

\begin{answer}
\textbf{Ex.~\ref{exr:JeansIQR1}.} \textbf{1.} Men's: about \(50\)\%; women's: about \(100\)\%. \textbf{2.} Men's: about \(0\)\%; women's: about \(50\)\%.

\end{answer}

\begin{answer}
\textbf{Ex.~\ref{exr:StatsHistograms}.} D; C; A; D.

\end{answer}

\end{ChapAnswers}

\subsection*{Chap.~\ref{SummariseQualData}: Summarising qualitative data}\label{chap.-refsummarisequaldata-summarising-qualitative-data}

\begin{ChapAnswers}

\begin{answer}
\textbf{Ex.~\ref{exr:SpiderMonkeys}.} \textbf{1.} Graph not shown; no commonly-observed social group include~M. \textbf{2.} Mode: many F plus offspring; median inappropriate.

\end{answer}

\begin{answer}
\textbf{Ex.~\ref{exr:GraphsCars}.} None \emph{bad}. I'd prefer bar chart; any OK.

\end{answer}

\begin{answer}
\textbf{Ex.~\ref{exr:OrdinalMedians}.} \textbf{1.} Gender: nominal; others ordinal. \textbf{2.}--\textbf{4.} Gender: modes are F, M; no median. Place: mode is city \(> 100\,000\) residents; median is city \(20\,000\) to~\(100\,000\) residents Response: mode is `Agree'; median is 'Neutral. \textbf{5.} \(5.12\): respondents about \(5\)~times more likely to come from city than rural. \textbf{6.} \(0.613\): respondents about \(0.61\)~times as likely to agree, strongly disagree than choose other option. \textbf{7.} \(1\): respondents as likely to be~M as~F.

\end{answer}

\begin{answer}
\textbf{Ex.~\ref{exr:StudentTransport}.} \textbf{1.} Walking; Bus \textbf{2.} Bus. \textbf{3.} No. \textbf{4.} \(0.102\), \(0.123\); \(0.246\), \(0.412\), \(0.116\). \textbf{5.} \(3.44\); i.e., students \(3.44\) times as likely to use motorised transport than active. \textbf{6.} \(0.141\); i.e, for every \(100\) students that \emph{do not walk}, \(100\times 0.141 = 14.1\) \emph{do walk}. \textbf{7.} Not shown.

\end{answer}

\begin{answer}
\textbf{Ex.~\ref{exr:FEVplots}.} \textbf{1.} Not shown. \textbf{2.} Gender: M; smoking: Non-smoking. \textbf{3.} Gender: percentage, odds M: \(51.4\)\%, \(1.06\); smoking: percentage, odds non-smoking: \(9.9\)\%, \(0.11\).

\end{answer}

\end{ChapAnswers}

\subsection*{Chap.~\ref{SummariseWithin}: Comparing quantitative data within individuals}\label{chap.-refsummarisewithin-comparing-quantitative-data-within-individuals}

\begin{ChapAnswers}

\begin{answer}
\textbf{Ex.~\ref{exr:CompareWithinInsulation}.} \textbf{1.} House. \textbf{2.} Each house has before, after. Graph, table not shown.

\end{answer}

\begin{answer}
\textbf{Ex.~\ref{exr:CompareWithinPainRelief}.} Graph, table not shown.

\end{answer}

\begin{answer}
\textbf{Ex.~\ref{exr:CompareWithinRunning}.} Not shown.

\end{answer}

\begin{answer}
\textbf{Ex.~\ref{exr:CompareWithinJumping}.} \textbf{1.} How much further people jump in shoes. Graph and table not shown.

\end{answer}

\end{ChapAnswers}

\subsection*{Chap.~\ref{BetweenQuantData}: Comparing quantitative data between individuals}\label{chap.-refbetweenquantdata-comparing-quantitative-data-between-individuals}

\begin{ChapAnswers}

\begin{answer}
\textbf{Ex.~\ref{exr:BoxplotsProjectCosts}.} \textbf{1.} In general, DB smaller cost over-runs. \textbf{2.} Tricky: DB: \(2\,\text{cm}\); DBB: \(3\,\text{cm}\). \textbf{3.} Tricky: DB: \(2\,\text{cm}\); DBB: \(3\,\text{cm}\).

\end{answer}

\begin{answer}
\textbf{Ex.~\ref{exr:NumericalQuantMatchingHistogramsAndBoxplots}.} I: B (mean; standard deviation). II: A (median; IQR). III: C (median; IQR).

\end{answer}

\begin{answer}
\textbf{Ex.~\ref{exr:NumericalQuantConstructionWorkerProductivity}.} \textbf{1.} \(0.61\); \(0.40\); \(0.42\) panels/min. \textbf{4.} Worker~2 faster, more consistent (using IQR); Worker~1 slower. Plots not shown.

\end{answer}

\begin{answer}
\textbf{Ex.~\ref{exr:CompareQuantExercisesNHANES}.} \textbf{1.} Error bar chart. \textbf{2.} Not shown.

\end{answer}

\begin{answer}
\textbf{Ex.~\ref{exr:QuantCompareSpeedSignage}.} \textbf{1.} Not shown. \textbf{2.} Not shown.

\end{answer}

\begin{answer}
\textbf{Ex.~\ref{exr:QuantCompareTyping}.} \textbf{1.} \texttt{mAcc}: highly \emph{left} skewed; \texttt{Age}: highly \emph{right} skewed; \texttt{mTS}: slightly right skewed. Perhaps medians, IQRs for summarising (mean, std dev.~probably OK for \texttt{mTS}). \textbf{2.} Not shown. \textbf{3.} Little diff between M, F in \emph{sample}.

\end{answer}

\begin{answer}
\textbf{Ex.~\ref{exr:QuantCompareSnakesConfounding}.} \textbf{1.} Very similar mean SVL. \textbf{2.} Crayfish regions: smaller mean SVL. \textbf{3.} Crayfish regions: larger mean SVL. \textbf{4.} Confounding.

\end{answer}

\end{ChapAnswers}

\subsection*{Chap.~\ref{CompareQualData}: Comparing qualitative data between individuals}\label{chap.-refcomparequaldata-comparing-qualitative-data-between-individuals}

\begin{ChapAnswers}

\begin{answer}
\textbf{Ex.~\ref{exr:NumericalQual1}.} Zero

\end{answer}

\begin{answer}
\textbf{Ex.~\ref{exr:NumericalQualHangovers}.} \textbf{1.} \emph{Vomited}: \(0.50\) beer, wine; \(0.50\) wine only. \emph{Didn't vomit}: \(0.738\) beer, wine, \(0.262\) wine only. Prop.~drank various drinks, among those who did, didn't vomit. \textbf{2.} \emph{Beer, wine}: \(8.8\)\% vomited, \(91.2\)\% didn't. \emph{Wine only}: \(21.4\)\% vomited, \(78.6\)\% didn't. Percentage that vomited, for each drinking type. \textbf{3.} \((6 + 6)/(6 + 6 + 62 + 22) = 0.125\). \textbf{4.} \(0.2727\). \textbf{5.} \(0.09677\). \textbf{6.} \(2.82\). \textbf{7.} \(0.354\). \textbf{8.} \(-0.176\). \textbf{9.} Higher percentage vomited after beer+wine, compared to beer only.

\end{answer}

\begin{answer}
\textbf{Ex.~\ref{exr:OddsAugustRainfall}.} \textbf{1.} About \(18.4\)\%. \textbf{2.} About \(25.9\)\%. \textbf{3.} About \(11.7\)\%. \textbf{4.} About \(0.226\). \textbf{5.} \(0.35\). \textbf{6.} About \(0.132\). \textbf{7.} About \(2.7\). \textbf{8.} Odds no August rainfall in Emerald \(2.7\) times higher in months with non-positive SOI.

\end{answer}

\begin{answer}
\textbf{Ex.~\ref{exr:AVquestions}.} Not shown.

\end{answer}

\begin{answer}
\textbf{Ex.~\ref{exr:SkippingBreakfast}.} \textbf{1.} \emph{Prop.}~F skipped: \(\hat{p}_F = 0.359\). \textbf{2.} \emph{Prop.}~M skipped: \(\hat{p}_M = 0.284\). \textbf{3.} Odds(Skips breakfast, F): \(0.5598\); \textbf{4.} Odds(Skips breakfast, M): \(0.3966\). \textbf{5.} OR: \(1.41\). \textbf{6.} Odds F skipping \(1.41\) \emph{times} odds M skipping. \textbf{7.} Not shown.

\end{answer}

\begin{answer}
\textbf{Ex.~\ref{exr:Dispatchers}.} \textbf{1.} Not shown. \textbf{2.} \(74.6%
\). \textbf{3.} \(60.9\)\%. \textbf{4.} \(2.487\). \textbf{5.} \(1.558\). \textbf{6.} \(1.596\). \textbf{7.} \(0.626\). \textbf{8.} Not shown.

\end{answer}

\begin{answer}
\textbf{Ex.~\ref{exr:PLHomeAway}.} \(\text{OR(W; home)} = 4/6  = 0.667\); \(\text{OR(W; away)} = 7/4 = 1.75\). \(\text{OR} = 0.6667/1.75 = 0.381\).

\end{answer}

\end{ChapAnswers}

\subsection*{Chap.~\ref{TwoQuant}: Correlations between quantitative variables}\label{chap.-reftwoquant-correlations-between-quantitative-variables}

\begin{ChapAnswers}

\begin{answer}
\textbf{Ex.~\ref{exr:CorrelationExerciseDrawNegR}.} Answers vary.

\end{answer}

\begin{answer}
\textbf{Ex.~\ref{exr:TwoQuantExercisesPeas}.} You cannot be precise. Software: \(r = 0.71\). Realistically: `reasonably high, positive \(r\)'.

\end{answer}

\begin{answer}
\textbf{Ex.~\ref{exr:TwoQuantExercisesLimeTrees}.} \textbf{1.} A tree. \textbf{2.} \emph{Form}: starts straight-ish, then hard to describe. \emph{Direction}: biomass increases as age increases (on average). \emph{Variation}: small-ish for small ages; large-ish for older trees (after \(60\)). \textbf{3.} No.

\end{answer}

\begin{answer}
\textbf{Ex.~\ref{exr:TwoQuantExercisesSoftdrink}.} \textbf{1.} Approximately linear; positive relationship; variation larger for more cases. \textbf{2.} A delivery. \textbf{3.} Non-constant variation: no.

\end{answer}

\begin{answer}
\textbf{Ex.~\ref{exr:TwoQuantExercisesGorillas}.} No relationship.

\end{answer}

\begin{answer}
\textbf{Ex.~\ref{exr:SoilCN}.} Approx. linear; positive; strong.

\end{answer}

\begin{answer}
\textbf{Ex.~\ref{exr:TwoQuantExercisesONI}.} \(R^2 = (-0.682)^2 = 0.465\): about \(46.5\)\% of the unknown variation in number cyclones explained by knowing value of ONI.

\end{answer}

\end{ChapAnswers}

\subsection*{Chap.~\ref{SummariseComments}: More details about tables and graphs data}\label{chap.-refsummarisecomments-more-details-about-tables-and-graphs-data}

\begin{ChapAnswers}

\begin{answer}
\textbf{Ex.~\ref{exr:Graphs123}.} Scatterplot; histogram of diffs; side-by-side bar.

\end{answer}

\begin{answer}
\textbf{Ex.~\ref{exr:GraphsLimeTrees}.} Individual variables: \emph{bar chart} for origin; \emph{histogram} for others. Between biomass, origin: \emph{boxplot}. Between biomass, other variables: \emph{scatterplot}. (On scatterplot, could encode origins with different colours, symbols.)

\end{answer}

\begin{answer}
\textbf{Ex.~\ref{exr:GraphNoisyMiners}.} Plotting symbols unexplained. Axis labels unhelpful. Vertical axis could stop at \(20\).

\end{answer}

\begin{answer}
\textbf{Ex.~\ref{exr:GraphsMADRS}.} \textbf{1.} Response: \emph{change} in \textsc{madrs} (quant.~cont.). \textbf{2.} Explanatory: treatment group (qual.~nominal, \(3\)~levels). \textbf{3.} Response: histogram. Explanatory: bar chart. Relationship: boxplot.

\end{answer}

\begin{answer}
\textbf{Ex.~\ref{exr:GraphsTyping}.} Plots not shown. \emph{Speed}: average: around \(60\)~wpm; variation: about \(30\) to~\(120\)~wpm. Slightly right skewed; no obvious outliers. \emph{Accuracy}: average: around \(85\)\%; variation: about \(65\)\% to~\(95\)\%. Left skewed; no obvious outliers. \emph{Age}: average: \(25\); variation: about \(15\) to~\(35\). \emph{Very} right skewed, perhaps unseen large outliers. \emph{Sex}: about twice as many F as M. \emph{Speed} and \emph{Sex}: not big difference between M, F.\spacex \emph{Accuracy} and \emph{Age}: hard to see relationship; no older people very slow.

Average speed, accuracy vary by age, not sex. How data collected (self-reported? Or measured how?). How students obtained: a random, somewhat representative or self-selecting sample?

\end{answer}

\end{ChapAnswers}

\subsection*{Chap.~\ref{Probability}: Probability}\label{chap.-refprobability-probability}

\begin{ChapAnswers}

\begin{answer}
\textbf{Ex.~\ref{exr:ProbabilityMethod}.} \textbf{1.} Subjective. \textbf{2.} Rel.~frequency.

\end{answer}

\begin{answer}
\textbf{Ex.~\ref{exr:ProbabilityAndOrNot}.} \textbf{1.} Just \textbf{Kings} and \textbf{Aces}. \textbf{2.} \(8/52 = 2/13\). \textbf{3.} Picture cards. \textbf{4.} \(16/52 = 4/13\). \textbf{5.} \textbf{Ace}, \textbf{King}, \textbf{Queen}, \textbf{Jack} of \(\spadesuit\). \textbf{6.} \(4/52 = 1/13\). \textbf{7.} Any \(\heartsuit\), \(\diamondsuit\) or \(\clubsuit\). \textbf{8.} \(39/52 = 3/4\). \textbf{9.} \(4/16 = 1/4\). \textbf{10.} \(4/13\).

\end{answer}

\begin{answer}
\textbf{Ex.~\ref{exr:ProbabilityThreeEvents}.} \textbf{1.} Yes. \textbf{2.} Yes. \textbf{3.} \(1/2\). \textbf{4.} \(1/2\). \textbf{5.} HH, HT, TH, TT (Coin~A listed first).

\end{answer}

\begin{answer}
\textbf{Ex.~\ref{exr:ProbabilityDie}.} \textbf{1.} \(4/6\). \textbf{2.} \(5\). \textbf{3.} Yes: die outcome won't change coin outcome. \textbf{4.} \(1/2\). \textbf{5.} \(1/6\). \textbf{6.} \(1/3\).

\end{answer}

\begin{answer}
\textbf{Ex.~\ref{exr:SampleSpaceCardsDiff}.} \textbf{1.} In order drawn: BB, BR, RB, RR. \textbf{2.} Equally-likely outcomes, so \(1/2\). \textbf{3.} \(1/2\). \textbf{4.} Yes.

\end{answer}

\begin{answer}
\textbf{Ex.~\ref{exr:FirstNationStudents}.} \textbf{1.} \(0.087\). \textbf{2.} \(0.708\). \textbf{3.} F: prob FN: \(0.107\); M: prob FN: \(0.108\); close to independent. \textbf{4.} F: prob FN: \(0.040\); M: prob FN: \(0.035\); close to independent. \textbf{5.} Gov: prob FN: \(0.107\); NGov: prob FN: \(0.040\); not independent. \textbf{6.} Gov: prob FN: \(0.108\); NGov: prob FN: \(0.035\); not independent. \textbf{7.} Regardless of sex, First Nations children more likely to be at government school.

\end{answer}

\begin{answer}
\textbf{Ex.~\ref{exr:IndependentEvents}.} \textbf{1.} \emph{Not independent}: less likely to walk in rain. \textbf{2.} \emph{Not independent}: smoker far more likely to suffer from lung cancer than non-smoker. \textbf{3.} \emph{Not independent}: if it rains, I won't water garden.

\end{answer}

\begin{answer}
\textbf{Ex.~\ref{exr:CoinOutcomes}.} Reasoning assumes three \emph{equally likely} outcomes (HH, TT, HT); untrue. Consider tossing \(20\)-c coin (lower-case, normal) and \(1\)-coin (capitals, \textbf{bold}). \emph{Four} outcomes: h\textbf{H}, h\textbf{T}, t\textbf{H} t\textbf{T}.

\end{answer}

\end{ChapAnswers}

\subsection*{Chap.~\ref{SamplingVariation}: Sampling variation}\label{chap.-refsamplingvariation-sampling-variation}

\begin{ChapAnswers}

\begin{answer}
\textbf{Ex.~\ref{exr:StdErrorOrStdDeviationA}.} \textbf{1.} Std~dev. \textbf{2.} Std~error (of mean).

\end{answer}

\begin{answer}
\textbf{Ex.~\ref{exr:HasStandardErrorA}.} \textbf{1.} No. \textbf{2.} Yes. \textbf{3.} Yes.

\end{answer}

\begin{answer}
\textbf{Ex.~\ref{exr:RouletteWheelA}.} \textbf{1.} Reasonable, if fair. \textbf{2.} Almost impossible, if fair. \textbf{3.} Unlikely (not impossible), if fair. \textbf{4.} Highly unlikely, if fair.

\end{answer}

\begin{answer}
\textbf{Ex.~\ref{exr:QuoteStdError}.} \emph{Std~error of the mean} describes how sample mean varies from sample to sample.

\end{answer}

\end{ChapAnswers}

\subsection*{Chap.~\ref{SamplingDistributions}: Models and normal distributions}\label{chap.-refsamplingdistributions-models-and-normal-distributions}

\begin{ChapAnswers}

\begin{answer}
\textbf{Ex.~\ref{exr:Statements}.} Only \textbf{1.} and \textbf{2.} false.

\end{answer}

\begin{answer}
\textbf{Ex.~\ref{exr:BasiczA}.} \textbf{1.} \(0.9671\). \textbf{2.} \(0.0183\). \textbf{3.} Close to zero. \textbf{4.} Close to one.

\end{answer}

\begin{answer}
\textbf{Ex.~\ref{exr:SamplingDistributionsGrowthChartA}.} About \(2.5\)\% of girls under \(100\,\text{cm}\) tall.

\end{answer}

\begin{answer}
\textbf{Ex.~\ref{exr:SamplingDistributionsIQForwards}.} \textbf{1:} C; \textbf{2:} A; \textbf{3:} B; \textbf{4:} D.

\end{answer}

\begin{answer}
\textbf{Ex.~\ref{exr:SamplingDistributionsEmpiricalA}.} \(68.26\)\%; very close to \(68\)\%.

\end{answer}

\begin{answer}
\textbf{Ex.~\ref{exr:SamplingDistributionsTrees}.} \textbf{1.} \(z = -0.30\); about \(38.2\)\%. \textbf{2.} \(z = 0.07\); about \(47.2\)\%. \textbf{3.} \(z = -0.67\) and \(z = 0.44\); about \(41.9\)\%. \textbf{4.} About \(z = 1.04\); diameter about \(11.6\)~inches. \textbf{5.} About \(z = -0.67\); diameter about \(7.0\)~inches.

\end{answer}

\begin{answer}
\textbf{Ex.~\ref{exr:SamplingDistributionsGestationLength}.} \textbf{1.} \(z = -0.61\); \(72.9\)\%. \textbf{2.} \(z = -1.83\); \(3.4\)\%. \textbf{3.} \(z = -4.878\) and \(z = -1.83\); \(3.4\)\%. \textbf{4.} \(z = 1.64\) (or \(1.65\)); \(5\)\% \emph{longer} than \(42.7\) weeks. \textbf{5.} \(z\)-score: \(-1.28\); \(10\)\% \emph{shorter} than \(37.9\) weeks.

\end{answer}

\begin{answer}
\textbf{Ex.~\ref{exr:SamplingDistributionsIQs}.} \(z = 2.05\). IQ: \(130.75\). \(\text{IQ} > 130\).

\end{answer}

\begin{answer}
\textbf{Ex.~\ref{exr:SamplingDistributionsChargingEVs}.} Use \emph{number minutes from (say) 5:30pm}. Std~dev.: \(120\,\text{mins}\), plus \(0.28\times 60 = 16.8\,\text{mins}\) = \(136.8\,\text{mins}\). \textbf{1.} \(9\)pm; \(210\)mins\$ from \(5\):\(30\)pm; \(z = 1.54\); \(6.2\)\%. \textbf{2.} \(z = -0.22\); \(41.3\)\%. \textbf{3.} \(z_1 = -0.22\) and \(z_2 = 0.22\); \(0.5871 - 0.4129\); \(17.4\)\%. \textbf{4.} \(z\)-score: \(0.52\); \(x  = 71.136\) mins after \(5\)pm; about one hour and \(11\)~mins after \(5\):\(30\)pm, or \(6\):\(41\)pm. \textbf{5.} \(z\)-score: \(-1.04\); \(x = -141.272\), or \(141.272\,\text{mins}\) \emph{before} \(5\):\(30\)pm; about two hours and \(21\)\,\text{mins} before \(5\):\(30\)pm, or \(3\):\(09\)pm.

\end{answer}

\end{ChapAnswers}

\subsection*{Chap.~\ref{CIOneProportion}: Confidence intervals: one proportion}\label{chap.-refcioneproportion-confidence-intervals-one-proportion}

\begin{ChapAnswers}

\begin{answer}
\textbf{Ex.~\ref{exr:CIOneProportionHiccups}.} \(\hat{p} = 0.81944\), \(n = 864\). \(\text{s.e.}(\hat{p}) = 0.01309\); approx. \(95\)\%~CI: \(0.819 \pm (2\times 0.0131)\) or \(0.793\) to~\(0.845\). Stat.~valid. (Many decimal places used for working; final answers rounded.)

\end{answer}

\begin{answer}
\textbf{Ex.~\ref{exr:CIzA}.} \(z = 1.96\).

\end{answer}

\begin{answer}
\textbf{Ex.~\ref{exr:CIOneProportionSnacking}.} \(\hat{p} = 0.051948\); \(\text{s.e.}(\hat{p}) = 0.00178\); approx. \(95\)\%~CI: \(0.0519\pm 0.0358\). Stat.~valid.

\end{answer}

\begin{answer}
\textbf{Ex.~\ref{exr:CIOneProportionSaltIntake}.} \(\hat{p} = 0.317059\); \(n = 6882\). \(\text{s.e.}(\hat{p}) =  0.0056092\). CI: \(0.317\pm 0.011\). Stat.~valid.

\end{answer}

\begin{answer}
\textbf{Ex.~\ref{exr:CanadianEnergyDrinks}.} \(\hat{p} = 0.241\). \(\text{s.e.}(\hat{p}) =  0.010984\). Approx. \(95\)\%~CI: \(0.219\) to~\(0.263\).

\end{answer}

\begin{answer}
\textbf{Ex.~\ref{exr:OnePropCIAI}.} \(\hat{p} = 0.3182\). \(\text{s.e.}(\hat{p}) =  0.0702175\). Approx. \(95\)\%~CI: \(0.178\) to~\(0.459\).

\end{answer}

\begin{answer}
\textbf{Ex.~\ref{exr:OnePropCIWearHats}.} \(\hat{p} = 0.13431\). \(\text{s.e.}(\hat{p}) = 0.012434\). Approx. \(95\)\%~CI: \(0.109\) to~\(0.159\).

\end{answer}

\end{ChapAnswers}

\subsection*{Chap.~\ref{OneMeanConfInterval}: Confidence intervals: one mean}\label{chap.-refonemeanconfinterval-confidence-intervals-one-mean}

\begin{ChapAnswers}

\begin{answer}
\textbf{Ex.~\ref{exr:OneMeanCIBears}.} \textbf{1.} \emph{Parameter}: pop. mean weight of American black bear, \(\mu\). \textbf{2.} \(\text{s.e.}(\bar{x}) = 3.756947\). \textbf{3.} Normal; mean \(\mu\); std dev: \(3.757\), \textbf{4.} \(77.4\) to~\(92.4\,\text{kg}\). \textbf{5.} Approx. \(95\)\% confident \emph{population mean} weight of male American black bears between \(77.4\) and \(92.4\,\text{kg}\). \textbf{6.} Stat.~valid: \(n \ge 25\).

\end{answer}

\begin{answer}
\textbf{Ex.~\ref{exr:CIOneMeanLungCapacityInChildren}.} \(\text{s.e.} = 0.06410\). \emph{Approx.} \(95\)\%~CI: \(2.72\,\text{L}\) to~\(2.98\,\text{L}\).

\end{answer}

\begin{answer}
\textbf{Ex.~\ref{exr:CIOneMeanToothbrushing}.} Approx. \(95\)\%~CI: \(29.9\) to~\(36.1\,\text{s}\).

\end{answer}

\begin{answer}
\textbf{Ex.~\ref{exr:OneMeanCINHANESInterpret}.} \emph{None} acceptable. \textbf{1.} CIs not about observations, but \emph{statistics}. \textbf{2.} CIs not about observations, but \emph{statistics}. \textbf{3.} \emph{Samples} don't vary between values; \emph{statistics} do. (CIs about populations anyway.) \textbf{4.} \emph{Populations} don't vary between values. \textbf{5.} Parameters do not vary. \textbf{6.} \emph{Know} \(\bar{x} = 1.3649\,\text{mmol}\)/L.\spacex \textbf{7.} \emph{Know} \(\bar{x} = 1.3649\,\text{mmol}\)/L.

\end{answer}

\begin{answer}
\textbf{Ex.~\ref{exr:ChewingTime}.} \(\text{s.e.}(\bar{x}) = 5.36768\); approx. \(95\)\%~CI: \(50.56\) to~\(72.04\,\text{s}\). Stat.~valid.

\end{answer}

\begin{answer}
\textbf{Ex.~\ref{exr:CIMatchesPerBox}.} \textbf{1.} \emph{One} observation \(x = 44\); claimed \emph{population mean} is \(\mu = 45\). \textbf{2.} OK to have decimal value as \emph{mean}. \textbf{3.} \(\bar{x} = 44.9\); \(\mu = 45\): different things; why should they be same? \textbf{4.} CI allows for sampling variation. \textbf{5.} \(44.850\) to~\(44.950\). \textbf{6.} Possibly lying; not certain. \textbf{7.} Sample mean; \(\bar{x} = 44.9\). \textbf{8.} Population mean; value unknown, but \emph{claimed} to be \(\mu = 45\).

\end{answer}

\end{ChapAnswers}

\subsection*{Chap.~\ref{AboutCIs}: More details about CIs}\label{chap.-refaboutcis-more-details-about-cis}

\begin{ChapAnswers}

\begin{answer}
\textbf{Ex.~\ref{exr:AboutCIsInterpretationP}.} \textbf{1.} CIs are intervals for unknown \emph{parameters}, not known \emph{statistics}. \textbf{2.} CIs for proportion (or percentage), not \emph{number} of trees. (The CI is \(68\)\% anyway, not \(95\)\%.)

\end{answer}

\begin{answer}
\textbf{Ex.~\ref{exr:AboutCIsInterpretationMean}.} \textbf{1.} CIs \emph{not} about individuals. \textbf{2.} CIs \emph{not} about \emph{sample} means.

\end{answer}

\begin{answer}
\textbf{Ex.~\ref{exr:CIPossums}.} Intervals for different things. First: \(95\)\%~CI for \emph{mean} weight. Second: \emph{not} CI; for weights of \emph{individuals} possums.

\end{answer}

\end{ChapAnswers}

\pagebreak

\subsection*{Chap.~\ref{MakingDecisions}: Making decisions}\label{chap.-refmakingdecisions-making-decisions}

\begin{ChapAnswers}

\begin{answer}
\textbf{Ex.~\ref{exr:MakingDecisionsDice}}. \textbf{1.} Yes; very likely (can't be sure). \textbf{2.} Assuming fair die, \emph{not} expect \largedice{6} ten times in row.

\end{answer}

\begin{answer}
\textbf{Ex.~\ref{exr:MakingDecisionsJuryService}}. Seems unlikely.

\end{answer}

\begin{answer}
\textbf{Ex.~\ref{exr:MakingDecisionsOlder}}. \textbf{1.} \(p = 0.36\). \textbf{2.} \(\hat{p} = 0.433\); probably not. \textbf{3.} \(\hat{p} = 0.1\); probably.

\end{answer}

\end{ChapAnswers}

\subsection*{Chap.~\ref{TestOneProportion}: Hypothesis tests: one proportion}\label{chap.-reftestoneproportion-hypothesis-tests-one-proportion}

\begin{ChapAnswers}

\begin{answer}
\textbf{Ex.~\ref{exr:sepPWhy1}.} In tests, \(p\) assumed known. In CI, have no value for \(p\) to use.

\end{answer}

\begin{answer}
\textbf{Ex.~\ref{exr:OneProportionTestExplainA}.} \(0.38\) is \emph{sample} proportion; RQ asks about \emph{pop.} proportion of \(1/6\).

\end{answer}

\begin{answer}
\textbf{Ex.~\ref{exr:OneProportionTestExercisesDodgyA}.} Tests \emph{not} about sample value (we \emph{know} value of \(\hat{p}\)), but about unknown pop. value (i.e., \(p\)).

\end{answer}

\begin{answer}
\textbf{Ex.~\ref{exr:OneProportionTestExercisesPlacebos}.} \textbf{1.} One-in-five: \(0.2\). \textbf{2.} \(H_0\): \(p = 0.2\); \(H_1\): \(p > 0.2\). \textbf{3.} One-tailed. \textbf{4.} Normal; mean \(0.2\), std~deviation \(\text{s.e.}(\hat{p}) = 0.0444\). \textbf{5.} \(\hat{p} = 0.6173\); \(z = 9.39\): \(P\) \emph{very small}: Very strong evidence people do better-than-guessing at identifying placebo.

\end{answer}

\begin{answer}
\textbf{Ex.~\ref{exr:OneProportionTestTurtleSex}.} \(H_0\): \(p = 0.5\); \(H_1\): \(p \ne 0.5\). \(\hat{p} = 0.39726\); \(\text{s.e.}(\hat{p}) = 0.05727\); \(z = -1.794\). \(P\) not that small. No evidence of difference.

\end{answer}

\begin{answer}
\textbf{Ex.~\ref{exr:OneProportionTestExercisesPedalMachines}.} \(H_0\): \(p = 0.0602\); \(H_1\): \(p < 0.602\) (\emph{one}-tailed). \(\hat{p} = 0.5008489\); \(n = 589\): \(\text{s.e.}(\hat{p}) = 0.0201689\), so \(z = -5.015\). \(P\) very small. Strong evidence prop. F using machines lower than prop. F in uni pop.

\end{answer}

\begin{answer}
\textbf{Ex.~\ref{exr:OneProportionBreadfruitPasta}.} \(H_0\): \(p = 0.5\); \(H_1\): \(p > 0.5\) (one-tailed). \(\hat{p} = 0.802817\); \(n = 71\): \(\text{s.e.}(\hat{p}) = 0.0593391\), so \(z = 5.10\). \(P\) very small. Strong evidence majority like breadfruit pasta (for pop. represented by sample anyway).

\end{answer}

\begin{answer}
\textbf{Ex.~\ref{exr:OneProportionTestExercisesBorers}.} \(H_0\): \(p = 1/16 = 0.625\); \(H_1\): \(p \ne 0.0625\). \(\hat{p} = 0.139535\) and \(\text{s.e.}(\hat{p}) = 0.018457\); \(z = 26.3\): \emph{massive}; \(P\) very small. Very strong evidence pop.~proportion not \(1/16\); borers \emph{not} resistant.

\end{answer}

\begin{answer}
\textbf{Ex.~\ref{exr:OneProportionTestExercisesCoinSpin}.} \(\hat{p} = 0.56\); \(z = 1.91\). Slight evidence of bias.

\end{answer}

\end{ChapAnswers}

\subsection*{Chap.~\ref{TestOneMean}: Hypothesis tests: one mean}\label{chap.-reftestonemean-hypothesis-tests-one-mean}

\begin{ChapAnswers}

\begin{answer}
\textbf{Ex.~\ref{exr:OneTSpeed}.} \textbf{1.} \(\mu\), pop. mean speed (km.h\textsuperscript{\(-1\)}). \textbf{2.} \(H_0\): \(\mu = 90\); \(H_1\): \(\mu > 90\) (one-tailed). \textbf{3.} \(\text{s.e.}(\bar{x}) = 0.6937\). \textbf{5.} \(t = 9.46\). \textbf{6.} \(t\)-score \emph{huge}; (one-tailed) \(P\) very small: very strong evidence mean speed of vehicles on road greater than \(90\,\text{km}.\text{h}^{-1}\). \textbf{7.} Stat.~valid.

\end{answer}

\begin{answer}
\textbf{Ex.~\ref{exr:TestOneMeanExercisesAutomatedVehicles}.} \(H_0\): \(\mu = 50\); \(H_1\): \(\mu > 50\) (one-tailed). \(\text{s.e.}(\bar{x}) = 4.701076\). \(t = 7.23\): \(P\) very small. \emph{Very} strong evidence (\(P < 0.001\)) mean mental demand \emph{greater} than \(50\).

\end{answer}

\begin{answer}
\textbf{Ex.~\ref{exr:TestOneMeanExercisesCherryRipes}.} \(H_0\): \(\mu = 14\); \(H_1\): \(\mu \ne 14\) (two-tailed). \(\text{s.e.}(\bar{x}) = 0.092493\). \(t\)-score: \(10.35\): \emph{huge}; \(P\) very small. \emph{Very} strong evidence (\(P < 0.001\)) mean weight of Fun Size \emph{Cherry Ripe} bar not \(14\,\text{g}\). SD: the variation in weight of individual bars. SE: the variation in sample means for \(n = 67\).

\end{answer}

\begin{answer}
\textbf{Ex.~\ref{exr:TestOneMeanExercisesSleep}.} \(H_0\): \(\mu = 10\) (or \(\mu \ge 10\)) and \(H_1\): \(\mu < 10\). \emph{F}: \(\text{s.e.}(\bar{x}) = 0.05924742\); \(t = -25.32\). \emph{M}: \(\text{s.e.}(\bar{x}) = 0.0700152\); \(t = -19.42\). Both \(P\) extremely small. For both boys and girls, very strong evidence mean sleep time on weekend less than \(10\,\text{h}\).

\end{answer}

\begin{answer}
\textbf{Ex.~\ref{exr:PizzasHT}.} \textbf{1.} \(\mu\): population mean pizza diameter. \textbf{2.} \(\bar{x} = 11.486\); \(s = 0.247\). \textbf{3.} \(0.02205479\). \textbf{4.} \(H_0\): \(\mu = 12\); \(H_1\): \(\mu\ne 12\). \textbf{5.} Two-tailed; RQ asks if diameter is \(12\) inches, or not. \textbf{6.} Normal distribution, mean \(12\) and std dev.~of \(\text{s.e.}(\bar{x}) = 0.02205\). \textbf{7.} \(t = -23.3\). \textbf{8.} \(P\) \emph{really} small. \textbf{9.} Very strong evidence mean diameter not \(12\)~inches. \textbf{10.} \(n\) much larger than \(25\); stat.~valid.

\end{answer}

\end{ChapAnswers}

\subsection*{Chap.~\ref{MoreAboutTests}: More details about hypothesis tests}\label{chap.-refmoreabouttests-more-details-about-hypothesis-tests}

\begin{ChapAnswers}

\begin{answer}
\textbf{Ex.~\ref{exr:MoreAboutExercisesApproximatingPValues}.} Use \(68\)--\(95\)--\(99.7\) rule and diagram: \textbf{1.} Very small; certainly less than \(0.003\) (\(99.7\)\% between \(-3\) and~\(3\)). \textbf{2.} Very small; bit bigger than \(0.003\) (\(99.7\)\% between \(-3\) and~\(3\)). \textbf{3.} Bit smaller than \(0.05\) (\(95\)\% between \(-2\) and~\(2\)). \textbf{4.} \emph{Very} small; \emph{much} smaller than \(0.003\).

\end{answer}

\begin{answer}
\textbf{Ex.~\ref{exr:MoreAboutExercisesApproximatingPValuesOneTailed}.} \emph{Half} values in Ex.~\ref{exr:MoreAboutExercisesApproximatingPValues}. \textbf{1.} Very small; certainly less than \(0.0015\). \textbf{2.} Very small; bit bigger than \(0.0015\). \textbf{3.} Bit smaller than \(0.025\). \textbf{4.} \emph{Very} small; \emph{much} smaller than \(0.0015\).

\end{answer}

\begin{answer}
\textbf{Ex.~\ref{exr:MoreAboutTestsInterpretingResults}.} \(P\) \emph{just larger} than \(0.05\); `slight evidence' to support \(H_1\). \(P\) \emph{just smaller} than \(0.05\); `moderate evidence' to support \(H_1\). The difference between \(0.0501\) and~\(0.0499\) trivial.

\end{answer}

\begin{answer}
\textbf{Ex.~\ref{exr:MoreAboutTestsInterpretingHypotheses}.} \textbf{1.} Hypotheses about \emph{parameters}. \textbf{2.} RQ two-tailed. \textbf{3.} \(36.8052\) is sample mean; hypothesis written \emph{before} data collected. \textbf{4.} Hypotheses about parameters; \(36.8052\) is sample mean. These test if \emph{sample} mean is \(36.8052\); we \emph{know} it is. \textbf{5.} Hypothesis written \emph{before} data collected. \textbf{6.} Hypotheses about parameters.

\end{answer}

\begin{answer}
\textbf{Ex.~\ref{exr:MoreAboutTestsConclusions}.} \textbf{1.} Conclusion about pop.~\textbf{mean} energy intake. \textbf{2.} Conclusions \emph{never} about statistics. \textbf{3.} Conclusion about pop.~\textbf{mean} energy intake. \textbf{4.} \(P\) \emph{is} \(0.018\), not \emph{less than} \(0.018\).

\end{answer}

\begin{answer}
\textbf{Ex.~\ref{exr:MoreAboutTestsConsistency}.} Statements~1 and~3 consistent.

\end{answer}

\end{ChapAnswers}

\pagebreak

\subsection*{Chap.~\ref{AnalysisPaired}: CIs and tests: mean differences (paired data)}\label{chap.-refanalysispaired-cis-and-tests-mean-differences-paired-data}

\begin{ChapAnswers}

\begin{answer}
\textbf{Ex.~\ref{exr:MeanDiffWhichPaired}.} \textbf{1.} Paired. \textbf{2.} Paired.

\end{answer}

\begin{answer}
\textbf{Ex.~\ref{exr:MeanDiffGDiffsA}.} How much longer task takes on the PC.

\end{answer}

\begin{answer}
\textbf{Ex.~\ref{exr:MeanDiffFlowering}.} CI calculated as: \(-3.24\) to~\(0.52\) days. Meaning, interpretation same as in Sect.~\ref{PairedInvasivePlants}.

\end{answer}

\begin{answer}
\textbf{Ex.~\ref{exr:MeanDiffGrowingSquash}.} \textbf{1.} \emph{Analysis}: farm. \emph{Observation}: individual fruits. \textbf{2.} Pairs have same farm management, soil, etc. \textbf{3.} Not shown. \textbf{4.} Not shown. \textbf{5.} Mean increase in fruit weight from normal (2015) to dry (i.e., normal minus dry). \textbf{6.} \(H_0\): \(\mu_d = 0\) and \(H_1\): \(\mu_d \ne 0\). \textbf{7.} Not shown. \textbf{8.} \(t = 0.205\). \textbf{9.} \(P\) large; from software, \(P = 0.839\). \textbf{10.} \(19.53\,\text{g}\) to \(23.99\,\text{g}\) heavier in normal year (2015). \textbf{11.} Probably stat.~valid; \(n\) just less than \(25\). \textbf{12.} No evidence (\(t = 0.205\); two-tailed \(P = 0.839\)) of mean increase in weight of squash from dry to normal years (mean change: \(2.23\,\text{g}\) (\(95\)\%~CI \(-19.53\) to \(23.99\,\text{g}\)), heavier in normal year).

\end{answer}

\begin{answer}
\textbf{Ex.~\ref{exr:TestPairedMeansTasteOfBroccoli}.} \textbf{1.} Not shown. \textbf{2.} How much tastier broccoli is with dip. \textbf{3.} \(t = 1.699\); \emph{approx.} one-tailed \(P\) between \(16\)\% and~\(2.5\)\%; not sure if \(P\) larger than \(0.05\), but likely (\(t\)-score quite a distance from \(z = 1\)). Evidence \emph{probably} doesn't support \(H_1\). \textbf{4.} Approx. \(95\)\%~CI: \(-0.92\) to~\(11.32\). \textbf{5.} Stat.~valid.

\end{answer}

\begin{answer}
\textbf{Ex.~\ref{exr:TestPairedMeansFerritin}.} \(\bar{d} = -424.25\); \(\text{s.e.}(\bar{d}) = 467.9404\); \(n = 20\): \(t = -0.907\). \(P = 0.376\): evidence doesn't support \(H_1\). \(95\)\% CI: \(-1403.7\) to~\(555.2\). Test perhaps not stat.~valid \(n < 25\)); but histogram of data suggests population \emph{might} have normal distribution); \(P\) so large probably makes little difference.

\end{answer}

\begin{answer}
\textbf{Ex.~\ref{exr:MeanDiffCOVIDCI}.} \textbf{1.} \emph{Diffs} are \emph{during} minus \emph{before}: \emph{positive} diffs means \emph{during} value higher. \textbf{2.} \(\text{s.e.}(\bar{d}) = 3.5150\). \textbf{3.} \(t = 0.762\); \(P\) is large; no evidence of change; \(-4.35\) to \(9.71\,\text{mins}\).

\end{answer}

\begin{answer}
\textbf{Ex.~\ref{exr:PairedCIExercisesAnorexia}.} \(0.89\) to~\(4.64\) pounds. Possibly not practically important.

\end{answer}

\begin{answer}
\textbf{Ex.~\ref{exr:PairedCIJumping}.} Exact CI: \(1.204\) to~\(0.069\,\text{cm}\) greater barefoot.

\end{answer}

\end{ChapAnswers}

\subsection*{Chap.~\ref{AnalysisTwoMeans}: CIs and tests: comparing two means}\label{chap.-refanalysistwomeans-cis-and-tests-comparing-two-means}

\begin{ChapAnswers}

\begin{answer}
\textbf{Ex.~\ref{exr:TwoSampleDiffsA}.} How much greater the mean lymphocytes cell diameter is compared to tumour cells.

\end{answer}

\begin{answer}
\textbf{Ex.~\ref{exr:TwoMeansCISamplingDistSignage}.} Normal; mean \(\mu_B - \mu_A\); std~dev.: \(2.965\,\text{km}\).h\textsuperscript{\(-1\)}.

\end{answer}

\begin{answer}
\textbf{Ex.~\ref{exr:TwoMeansWhales}.} \textbf{1.} Parameter: \(\mu_F - \mu_M\). Estimate: \(\bar{x}_F - \bar{x}_M = -0.06\,\text{m}\). \textbf{2.} Not shown. \textbf{3.} \(0.092487\) \textbf{4.} Not shown. \textbf{5.} \(-1.94\)~to \(0.24\,\text{m}\). \textbf{6.} \(H_0: \mu_F - \mu_M = 0\); \(H_1: \mu_F - \mu_M \ne 0\). \textbf{7.} \(t = 0.65\); \(P\) very large. \textbf{8.} No evidence (\(t = 0.65\); two-tailed \(P > 0.10\)) in sample that mean length of adult gray whales is diff in pop. for F (mean: \(4.66\,\text{m}\); std dev.: \(0..38\,\text{m}\)) and M (mean: \(4.60\,\text{m}\); std dev.: \(0.30\,\text{m}\); \(95\)\%~CI for the diff: \(-1.94\,\text{m}\) to \(0.24\,\text{m}\)). \textbf{9.} Yes.

\end{answer}

\begin{answer}
\textbf{Ex.~\ref{exr:MeansIndSamplesExercisesEchinacea}.} \textbf{1.} \(\mu_P - \mu_E\), reduction in mean duration for those using echinacea. \textbf{2.} \(0.2728678\) days; echinacea: \(0.2446822\) days \textbf{3.} \(0.3665054\). \textbf{4.} Not shown. \textbf{5.} \(-0.203\) to~\(1.263\) days. \textbf{6.} \(5.85\) to \(6.83\) days. \textbf{7.} \(H_0\): \(\mu_P - \mu_E = 0\); \(H_1\): \(\mu_P - \mu_E > 0\) (one-tailed). \textbf{8.} \(0.3665054\). \textbf{9.} \(t = 1.45\); one-tailed \(P\) between \(0.025\) and \(0.32\); using \(z\)-tables, \(P\) approx. \(0.074\). \textbf{10.} Slight evidence of diff. \textbf{11.} Not given. \textbf{12.} Yes. \textbf{13.} Probably not practically important (diff \(0.53\) days).

\end{answer}

\begin{answer}
\textbf{Ex.~\ref{exr:TwoMeansDental}.} \textbf{1.} Amount of DMFT greater in non-industrialised countries, as upper table has negative mean. \textbf{2.} \(H_0\): \(\mu_I - \mu_{NI} = 0\); \(H_1\): \(\mu_I - \mu_{NI} \ne 0\). \textbf{3.} \(11.9\) to \(22.5\), greater for industrialised. \textbf{4.} Very strong evidence in sample (\(P < 0.001\)) mean annual sugar consumption per person diff for industrialised (mean: \(41.8\,\text{kg}\)/person/y) and non-industrialised (mean: \(24.6\,\text{kg}\)/person/y) countries (\(95\)\%~CI for the diff \(11.9\) to~\(22.5\)). \textbf{5.} Yes.

\end{answer}

\begin{answer}
\textbf{Ex.~\ref{exr:FacePlant}.} \textbf{1.} \(\mu_Y - \mu_O\): mean amount younger women can lean further forward than older. \textbf{2.} Boxplot. \textbf{3.} Not shown. \textbf{4.} Approx.: \(10.2\) to~\(18.8\)\textsuperscript{o}C.\spacex Exact CI (Row 2): \(9.1\) to~\(19.9\)\textsuperscript{o}C.\spacex Different: sample sizes small. \textbf{5.} One. \textbf{6.} \(H_0\): \(\mu_Y - \mu_O = 0\); \(H_1\): \(\mu_Y - \mu_O > 0\). \textbf{7.} \(t = 6.69\) (\emph{second row}); \(P < 0.001/2\) as one-tailed; i.e., \(P < 0.0005\). \textbf{8.} Very strong evidence in sample (\(t = 6.69\); one-tailed \(P < 0.0005\)) that pop. mean one-step fall recovery angle for healthy women \emph{greater} for young women (mean: \(30.7\)\textsuperscript{o}C; std~dev.: \(2.58\)\textsuperscript{o}C; \(n = 10\)) compared to older women (mean: \(16.20\)\textsuperscript{o}C; std~dev.: \(4.44\)\textsuperscript{o}C; \(n = 5\); \(95\)\%~CI for the diff: \(9.1\)\textsuperscript{o}C to \(19.9\)\textsuperscript{o}C). \textbf{9.} Probably not stat.~valid.

\end{answer}

\begin{answer}
\textbf{Ex.~\ref{exr:TestTwoMeansBodyTemperature}.} \(H_0\): \(\mu_M - \mu_{F} = 0\); \(H_1\): \(\mu_M - \mu_{F} \ne 0\). From output, \(t = -2.29\); (two-tailed) \(P = 0.024\). Moderate evidence (\(P = 0.024\)) mean internal body temperature diff for F (mean: \(36.9\)\textsuperscript{o}C) and M (mean: \(36.7\)\textsuperscript{o}C). Diff between the means (\(0.2\) of degree) of little \emph{practical} importance.

\end{answer}

\begin{answer}
\textbf{Ex.~\ref{exr:TwoMeansCIExercisesAnorexia}.} \textbf{1.} \(2.76\,\text{kg}\). \textbf{2.} CB: \(0.227\) to \(5.79\,\text{kg}\); Control: \(-3.68\) to \(2.78\,\text{kg}\). \textbf{3.} \(-0.68\) to \(7.59\,\text{kg}\), greater for CB. \textbf{4.} \(t = 1.67\); Two-tailed \(P = 0.095\). Slight evidence of diff.

\end{answer}

\begin{answer}
\textbf{Ex.~\ref{exr:TwoMeansReactionTimes}.} \(t = 2.631\); one-tailed \(P = 0.0055\); evidence of diff.

\end{answer}

\end{ChapAnswers}

\subsection*{Chap.~\ref{AnalysisOddsRatio}: CIs and tests: comparing two odds or proportions}\label{chap.-refanalysisoddsratio-cis-and-tests-comparing-two-odds-or-proportions}

\begin{ChapAnswers}

\begin{answer}
\textbf{Ex.~\ref{exr:OddsSame}.} Both odds: \(6.04\).

\end{answer}

\begin{answer}
\textbf{Ex.~\ref{exr:OddsRatioCISamplingDistA}.} Normal; mean \(p_P - p_N\) and std~dev. \(0.0428\). For OR: not normal distribution.

\end{answer}

\begin{answer}
\textbf{Ex.~\ref{exr:Chi2z}.} \textbf{1.} \(z = 3.26\). \textbf{2.} Very small.

\end{answer}

\begin{answer}
\textbf{Ex.~\ref{exr:EVAdoption}.} \textbf{1.} No-PG proportion (Row~1), minus PG (Row~2). \textbf{2.} \(H_0\): \(p_{\text{With}} - p_{\text{W'out}} = 0\); \(H_1\): \(p_{\text{With}} - p_{\text{W'out}} \ne 0\). \textbf{3.} \(z = 1.14\); \(P = 0.253\); no evidence of diff. \textbf{4.} \(-0.071\) to~\(0.296\), larger intent to purchase for those without PG study. \textbf{5.} Odds yes (Col~1), comparing no-PG (Row~1) to PG (Row~2). \textbf{6.} One option: \(H_0\): \(\text{OR} = 1\); \(H_1\): \(\text{OR}\ne 1\). \textbf{7.} \(\chi^2 = 1.31\); \(P = 0.253\); no evidence of diff. \textbf{8.} OR between \(0.68\), \(4.28\). \textbf{9.} Smallest expected: \(10.6\); yes.

\end{answer}

\begin{answer}
\textbf{Ex.~\ref{exr:ORcrashes}.} \textbf{1.} Not shown. \textbf{2.} \(0.3000\); \(0.3033\). \textbf{3.} \(-0.0033\). \textbf{4.} \(\text{s.e.}(\hat{p}_1) = 0.0648074\); \(\text{s.e.}(\hat{p}_2) = 0.0416170\); s.e. for diff.: \(0.077019\). \textbf{5.} \(-0.0033\pm 0.154\): \(-0.157\) to~\(0.151\). \textbf{6.} \(-0.154\) to~\(0.148\). \textbf{7.} Not given. \textbf{8.} \(0.429\); \(0.436\). \textbf{9.} \(0.429/0.436 = 0.985\). \textbf{10.} \(0.480\) to~\(2.02\). \textbf{11.} Not shown \textbf{12.} \(p = 0.4333\); s.e. for diff: \(0.0832097\). \textbf{13.} \(z = (-0.0033 - 0)/0.0832097 = -0.040\); very small \(P\); no evidence of diff. \textbf{14.} \(\chi^2 = 0.002\) (output); \(P = 0.966\); no evidence of diff.

\end{answer}

\begin{answer}
\textbf{Ex.~\ref{exr:PetBirdsTest}.} \textbf{1.} \(-0.1918\). \textbf{2.} \(0.04643\). \textbf{3.} \(0.04202\). \textbf{4.} Second uses common \(p\): test assumes \(p\) same in both groups. \textbf{5.} \(-0.1918 \pm (2\times 0.04643)\): \(-0.285\) to~\(-0.099\). \textbf{6.} \(-0.273\) to~\(-0.111\). \textbf{7.} Not given. \textbf{8.} \(z = (-0.1918 - 0)/0.04202 = -4.56\); small \(P\); evidence of diff. \textbf{9.} \(0.443\). \textbf{10.} \(0.315\) to~\(0.623\). \textbf{11.} \(P < 0.0001\); small \(P\); evidence of diff. \textbf{12.} Yes. \textbf{13.} Observational.

\end{answer}

\begin{answer}
\textbf{Ex.~\ref{exr:TestOddsRatioSunglasses}.} \textbf{1.} Not shown. \textbf{2.} F: \(0.060\), M: \(0.205\); diff: \(0.145\). \textbf{3.} \(0.0640\), \(0.256\); \(0.250\). \textbf{4.} s.e. for diff: \(0.0243\); \(0.096\) to~\(0.193\). \textbf{5.} \(0.098\) to~\(0.192\). \textbf{6.} Not given. \textbf{7.} \(2.45\) to~\(6.61\). \textbf{8.} \(P < 0.0001\). \textbf{9.} Evidence of a diff. \textbf{10.} Yes.

\end{answer}

\begin{answer}
\textbf{Ex.~\ref{exr:TestOddsRatioBearTree}.} \textbf{1.} OR: \(0.3478261\). \textbf{2.} Diff: \(-0.12\). \textbf{3.} Proportions equal; not equal. \textbf{4.} Odds equal; not equal. \textbf{5.} \(z = 2.12\); \(P\) `small'. \textbf{6.} Some evidence of diff. \textbf{7.} Yes.

\end{answer}

\begin{answer}
\textbf{Ex.~\ref{exr:DogsHT}.} \textbf{1.} \(H_0\): No association; \(H_1\): Association. \textbf{2.} \(23.0522\); \(P = 0.00004\). \textbf{3.} Very strong evidence of association. \textbf{4.} Yes.

\end{answer}

\end{ChapAnswers}

\subsection*{Chap.~\ref{EstimatingSampleSize}: Finding sample sizes for CIs}\label{chap.-refestimatingsamplesize-finding-sample-sizes-for-cis}

\begin{ChapAnswers}

\begin{answer}
\textbf{Ex.~\ref{exr:SampleSizeNarrow1}.} Larger.

\end{answer}

\begin{answer}
\textbf{Ex.~\ref{exr:SampleSizeMean1}.} \textbf{1.} At least \(25\). \textbf{2.} At least \(100\) (\(4\) times as many). \textbf{3.} At least \(400\) (\(16\) times as many). \textbf{4.} Halve width: \(4\) times as many. \textbf{5.} Quarter width: \(16\) times as many. \textbf{6.} More needed for greater precision.

\end{answer}

\begin{answer}
\textbf{Ex.~\ref{exr:SampleSizePropEating}.} \textbf{1.} At least \(10\,000\). \textbf{2.} At least \(2\,500\). \textbf{3.} At least \(1\,000\). \textbf{4.} Expensive (time \emph{and} money): \(10\,000\) and \(2\,500\) probably unrealistic.

\end{answer}

\begin{answer}
\textbf{Ex.~\ref{exr:SampleSizeMeanLungCapacity}.} Use \(s = 0.43\). \textbf{1.} At least \(1\,849\). \textbf{2.} At least \(296\). \textbf{3.} At least \(74\). \textbf{4.} Expensive (time \emph{and} money); \(74\) more realistic.

\end{answer}

\begin{answer}
\textbf{Ex.~\ref{exr:SampleSizeInvasivePlants}.} Use, say, \(s = 13\). \textbf{1.} At least \(81\) pairs. \textbf{2.} At least \(76\) pairs.

\end{answer}

\begin{answer}
\textbf{Ex.~\ref{exr:SampleSizeWhales}.} Use, say, \(s = 0.35\). \textbf{1.} At least \(44\) in each group. \textbf{2.} At least \(98\) in each group. \textbf{3.} Info not relevant to goldfish.

\end{answer}

\begin{answer}
\textbf{Ex.~\ref{exr:SampleSizeHAttack}.} \(2\,223\).

\end{answer}

\end{ChapAnswers}

\subsection*{Chap.~\ref{CorrelationRegression}: Correlation and regression}\label{chap.-refcorrelationregression-correlation-and-regression}

\begin{ChapAnswers}

\begin{answer}
\textbf{Ex.~\ref{exr:RegressionGuess}.} Answers \emph{very} approximate. \textbf{1.} \(r\) moderate, positive; \(\hat{y} = 4 + 1.5x\) \textbf{2.} \(r\) reasonably strong, positive; \(\hat{y} = 6 + 2.3x\). \textbf{3.} \(r\) not apt: variation in \(y\) increases as \(x\) increases. \textbf{4.} \(r\) reasonably strong, negative; \(\hat{y} = 8 - 1.5x\).

\end{answer}

\begin{answer}
\textbf{Ex.~\ref{exr:CorrelationConsistency1}.} Any could be; can't tell.

\end{answer}

\begin{answer}
\textbf{Ex.~\ref{exr:RegressionValues}.} \textbf{1.} \(b_0 = 3.5\); \(b_1 = -0.14\). \textbf{2.} \(b_0 = 2.1\); \(b_1 = -0.0047\). \textbf{3.} \(b_0 = -25.2\); \(b_1 = -0.95\). \textbf{4.} \(b_0 = 0.15\); \(b_1 = -0.22\).

\end{answer}

\begin{answer}
\textbf{Ex.~\ref{exr:PlotAndPoints1}.} Not shown.

\end{answer}

\begin{answer}
\textbf{Ex.~\ref{exr:CorTestDrug}.} \textbf{1.} \(H_0\): \(\rho = 0\); \(H_1\): \(\rho \ne 0\). \textbf{2.} No evidence of relationship. \textbf{3.} Approx. linear; variation in STAI same for diff levels of experience. \(n \ge 25\).

\end{answer}

\begin{answer}
\textbf{Ex.~\ref{exr:CorrelationSoftdrink}.} No: variation increasing.

\end{answer}

\begin{answer}
\textbf{Ex.~\ref{exr:RegressionExerciseSunscreen}.} \textbf{1.} \(b_0\): \emph{no} time spent on application, mean \(0.27\,\text{g}\) applied; nonsense. \(b_1\): each extra minute adds average of \(2.21\,\text{g}\) sunscreen. \textbf{2.} Slope: g/min; intercept: g. \textbf{3.} \(\beta_0\) could be zero; makes sense. \textbf{4.} \(\hat{y} = 18\,\text{g}\). \textbf{5.} \(64\)\% reduction in unexplained variation using time. \textbf{6.} \(r = 0.8\); strong positive correlation.

\end{answer}

\begin{answer}
\textbf{Ex.~\ref{exr:CorTestDogs}.} \textbf{1.} Probably linear; increasing; approx. constant variance in \(y\) as \(x\) increases. \textbf{2.} \(H_0\): \(\rho = 0\); \(H_0\): \(\rho > 0\). \textbf{3.} \(r = 0.837\); \(P < 0.00005\). Very strong evidence of positive relationship. \textbf{4.} Yes.

\end{answer}

\begin{answer}
\textbf{Ex.~\ref{exr:Apnoea}.} \textbf{1.} \(r = 0.264\). \textbf{2.} \(R^2 = 6.99\)\%; using neck circumference reduces unknown variation by about \(7\)\%. \textbf{3.} \(\hat{y} = -24.47 + 1.36x\): \(y\) is REI; \(x\) is neck circum. (in cm). \textbf{4.} Each \(1\,\text{cm}\) increase in neck circum. increases REI by average of \(1.36\). \textbf{5.} Approx. CI: \(0.0575\) to~\(2.675\). \textbf{6.} \(t = 2.09\), \(P = 0.041\): slight evidence of relationship. \textbf{7.} Stat.~valid.

\end{answer}

\begin{answer}
\textbf{Ex.~\ref{exr:CorrelationRegressionExerciseBitumen}.} \textbf{1.} Very strong, negative linear relationship. \textbf{2.} \(r = -\sqrt{0.9929} = -0.9964\); must be negative. \textbf{3.} \(\hat{y} = 17.47 - 2.59x\): \(x\) is percentage bitumen by wt; \(y\) is percentage air voids by volume. \textbf{4.} \emph{Slope}: increase in bitumen wt by one percentage point \emph{decreases} average percentage air voids by volume by \(2.59\) percentage points. \emph{Intercept}: extrapolation: \(0\)\% bitumen content by wt, percentage air voids by volume \(17.47\)\%. \textbf{5.} \(t = -74.9\): massive; extremely strong evidence (\(P < 0.001\)) of relationship. \textbf{6.} \(P < 0.001\), as for slope. \textbf{7.} \(\hat{y} = 4.5027\), or \(4.5\)\%; good prediction, as relationship strong. \textbf{8.} \(\hat{y} = 1.909\), or \(1.9\)\%; perhaps poor: extrapolation. \textbf{9.} Yes.

\end{answer}

\begin{answer}
\textbf{Ex.~\ref{exr:CorrelationExercisesGorillas}.} \textbf{1.} \(r = 0.271\); \(R^2 = 7.3\)\%. \textbf{2.} \(t = 1.35\); \(P = 0.190\); no evidence of relationship. \textbf{3.} \(\hat{y} = -5.647 + 0.123x\).

\end{answer}

\begin{answer}
\textbf{Ex.~\ref{exr:ElephantsCor}.} \textbf{1.} Left probably F. \textbf{2.} Sex~B. \textbf{3.} A: \(r = 0.600\); B: \(r = 0.815\). \textbf{4.} B (M). \textbf{5.} A: \(\hat{y} = -2289 + 21.31x\); B: \(\hat{y} = -3621 + 27.63x\). \textbf{6.} Both: \(P < 0 .0001\) (probably use one-tailed test). \textbf{7.} A: \(2\,503\,\text{kg}\); B: \(2\,596\,\text{kg}\). \textbf{8.} Sample sizes, linearity OK; for Sex~B, perhaps increasing variation.

\end{answer}

\begin{answer}
\textbf{Ex.~\ref{exr:RegressionDogsLife}.} \(r = -0.712\) and \(P < 0.001\). Regression: \(\hat{y} = 13.39 - 0.093x\).

\end{answer}

\end{ChapAnswers}

\subsection*{Chap.~\ref{SelectTest}: Selecting an analysis}\label{chap.-refselecttest-selecting-an-analysis}

\begin{ChapAnswers}

\begin{answer}
\textbf{Ex.~\ref{exr:SamplingDistributionA}.} Only \textbf{3.} \textbf{2.} Almost certainly; \(n = 24\) very close to \(n = 25\).

\end{answer}

\begin{answer}
\textbf{Ex.~\ref{exr:Method1}.} Summary of mean \emph{diffs}; histogram of diffs. Paired samples \(t\)-test; CI for mean diff.

\end{answer}

\begin{answer}
\textbf{Ex.~\ref{exr:Method3}.} Comparing two proportions (or ORs); stacked, side-by-side bar chart. CI for diff in proportions (or CI for OR).

\end{answer}

\begin{answer}
\textbf{Ex.~\ref{exr:Method5}.} Correlation or regression, if linear.

\end{answer}

\end{ChapAnswers}

\subsection*{Chap.~\ref{WritingResearch}: Writing and reporting research}\label{chap.-refwritingresearch-writing-and-reporting-research}

\begin{ChapAnswers}

\begin{answer}
\textbf{Ex.~\ref{exr:WriteWordChoice}.} \textbf{1.} \emph{to}. \textbf{2.} \emph{its}. \textbf{3.} One sample of \(50\) individuals; use `mean' or `median', not `average'; units of age is `years'. \textbf{4.} Should all be one sentence.

\end{answer}

\begin{answer}
\textbf{Ex.~\ref{exr:WriteAmbiguous}.} \textbf{1.} Ambiguous; sound like cage is M; passive. `The cage contained one male rat.' \textbf{2.} Seaweed removed from beaker, or from lake water? `The research assistant recorded the pH of the lake water (after removing weeds) in the beaker.'

\end{answer}

\begin{answer}
\textbf{Ex.~\ref{exr:WriteAmbiguous3}.} \textbf{1.} `Substantial' if a \emph{large} change is expected (quote statistics (e.g., \(P\)-value) if \emph{statistically} significant intended). \textbf{2.} `The data \emph{are}\ldots{}'

\end{answer}

\begin{answer}
\textbf{Ex.~\ref{exr:WriteExercisesDecimals}.} Number decimal places ridiculous.

\end{answer}

\begin{answer}
\textbf{Ex.~\ref{exr:WriteExercisesStudent1}.} RQ: P, O, C and I unclear; fonts should be identified. Perhaps better: for students, is mean reading speed for text in Georgia font same as for Calibri font? \textbf{Abstract} statement poor (\emph{fonts} are not fast or slow). Perhaps: sample provided evidence mean reading speeds different (\(P = ?\)), when comparing text in Georgia font (mean: ?) and Calibri font (mean: ?; \(95\)\%~CI for diff: ? to~?).

\end{answer}

\begin{answer}
\textbf{Ex.~\ref{exr:WriteExercisesStudent3}.} Variables \emph{qualitative}: means inappropriate; use OR; values almost certainly refer to CI for OR.\spacex Without more information, we can't be sure what the OR means.

\end{answer}

\end{ChapAnswers}

\subsection*{Chap.~\ref{Reading}: Reading and critiquing research}\label{chap.-refreading-reading-and-critiquing-research}

\begin{ChapAnswers}

\begin{answer}
\textbf{Ex.~\ref{exr:ReadExerciseiPhoneStepCounts}.} \textbf{1.} Convenience; self-selected. Those in study \emph{may} be different than those not in. \textbf{2.} Inclusion criteria. \textbf{3.} Ethical (drop-outs happen); accurate description of study. \textbf{4.} Not ecologically valid. \textbf{5.} Paired \(t\)-test. \textbf{6.} Null: no mean diff between counts on phone, manually counted; alternative: a diff. \textbf{7.} \(P\) small; evidence mean diff in step-count between methods cannot be explained by chance: likely a diff. \textbf{8.} Valid.

\end{answer}

\begin{answer}
\textbf{Ex.~\ref{exr:ReadExerciseQuakes}.} \textbf{1.} Only some evidence of diff in mean age. \textbf{2.} Comparing the two groups; age possible confounder. \textbf{3.} Two-sample \(t\)-test. \textbf{4.} \(0.03376\). \textbf{5.} \(t = 2.07\); small \(P\); evidence of diff. \textbf{6.} Probably, given std errors rounded. \textbf{7.} Conceptual. \textbf{8.} Not shown. \textbf{9.} \(\chi^2\). \textbf{10.} \(z = 1.75\); \(P\) between \(5\)\% and~\(32\)\%: not helpful. \textbf{11.} Observational: not cause-and-effect; no confounders noted; very restricted population.

\end{answer}

\begin{answer}
\textbf{Ex.~\ref{exr:ReadExerciseLarva}.} \textbf{1.} \(\chi^2\)-test to compare proportions. \textbf{2.} No evidence of diff in survival rates at temps. \textbf{3.} Evidence surviving~\emph{Cx.} had larger mean size compared to surviving~\emph{Ae.}. \textbf{4.} Two-sample \(t\)-test. \textbf{5.} \(0.010628\). \textbf{6.} \(t = 26.3\); very small \(P\); very strong evidence of diff in mean lengths. \textbf{7.} Yes. \textbf{8.} Yes. \textbf{9.} \emph{Cx.}: evidence mean sizes at temps diff; \emph{Ae.}: no evidence mean sizes at temps diff. \textbf{10.} Two-sample \(t\)-tests. \textbf{11.} Intercept: \(-55.40\) to~\(16.28\); slope: \(3.88\) to~\(59.40\). \textbf{12.} \(t = 2.28\); expect small \(P\); evidence of linear association. \textbf{13.} Need scatterplot to be sure, but \(n \ge 25\). \textbf{14.} When predator--size ratio increases by one, predation efficiency increases by \(31.64\) percentage points. \textbf{15.} Unknown variation reduces by \(8.7\)\% using predator--size ratio. \textbf{16.} \(r = 0.294\).

\end{answer}

\begin{answer}
\textbf{Ex.~\ref{exr:ReadExerciseTomatoes}.} \textbf{1.} Stratified? \textbf{2.} Strong evidence the mean number of actinomycetes diff. \textbf{3.} Possibly not; sample sizes small. \textbf{4.} Very strong evidence mean number higher in CNV. \textbf{5.} Larger actinomycetes numbers linearly associated with lower corky root severity. \textbf{6.} \(R^2 = 57.8\)\%; unknown variation decreases by \(57.8\)\% using actinomycete abundance.

\end{answer}

\end{ChapAnswers}

\captionsetup{font=normalsize}

\def\bibfont{\footnotesize} % Change fontsize of bibliography

\bibliography{packages.bib,ReferenceList.bib}



\backmatter
\indexprologue{\noindent Page numbers in \textbf{bold} refer to the Glossary entry.}
\printindex



\end{document}
